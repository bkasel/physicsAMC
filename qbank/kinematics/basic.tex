
%% this section contains 23 problems


%% Basic Problems
%%------------------------------
\element{basic}{
\begin{question}{basic-Q01}
    What is displacement?
    \begin{choices}
        \wrongchoice{The length of the total path traveled.}
        \wrongchoice{How far an object travels in a certain amount of time.}
      \correctchoice{How far the final position of an object is from the starting position.}
        \wrongchoice{How much velocity changes as time changes.}
    \end{choices}
\end{question}
}

\element{basic}{
\begin{question}{basic-Q02}
    What is distance?
    \begin{choices}
      \correctchoice{The length of the total path traveled.}
        \wrongchoice{How far an object travels in a certain amount of time.}
        \wrongchoice{How far the final position of an object is from the starting position.}
        \wrongchoice{How much velocity changes as time changes.}
    \end{choices}
\end{question}
}

\element{basic}{
\begin{question}{basic-Q03}
    Which of the following scenarios describes an object that
        does \emph{not} have zero displacement?
    \begin{choices}
        \wrongchoice{A runner runs 8 laps around the Mervo track.}
        \wrongchoice{A baseball player hits a home run and runs around all of the bases and back to home plate.}
      \correctchoice{A rocket is launched into outer space and travels to the moon.}
        \wrongchoice{A car leaves the garage, drives to the grocery store, then drives back to the garage.}
    \end{choices}
\end{question}
}

\element{basic}{
\begin{question}{basic-Q04}
    How are velocity and speed different?
    \begin{choices}
        \wrongchoice{Velocity measures position, and speed measured a change in position over time}
        \wrongchoice{Speed measures position, and velocity measures a change in speed over time.}
        \wrongchoice{Speed tells the direction of the motion and velocity does not.}
      \correctchoice{Speed does not tell the direction of the motion and velocity does.}
    \end{choices}
\end{question}
}

\element{basic}{
\begin{question}{basic-Q05}
    How is a scalar quantity different from a vector quantity?
    \begin{choices}
        \wrongchoice{A scalar has a number and a unit, while a vector has only a unit}
        \wrongchoice{A scalar has a number, while a vector has a number and a unit.}
      \correctchoice{A scalar has a number and a unit, while a vector has a number, unit and direction}
        \wrongchoice{A scalar has a number, unit and direction, while a vector has a number and a unit.}
    \end{choices}
\end{question}
}


\element{basic}{
\begin{question}{basic-Q06}
    Which of the following is a scalar?
    \begin{multicols}{2}
    \begin{choices}
        \wrongchoice{Displacement}
      \correctchoice{Speed}
        \wrongchoice{Velocity}
        \wrongchoice{Acceleration}
        \wrongchoice{Force}
    \end{choices}
    \end{multicols}
\end{question}
}

\element{basic}{
\begin{question}{basic-Q07}
    Which of the following scenarios describes the acceleration
        of a moving object?
    \begin{choices}
      \correctchoice{A car speeds up at a rate of \SI{2}{\meter\per\second\squared} to pass another car.}
        \wrongchoice{The waves in the harbor move North at \SI{2}{\meter\per\second}.}
        \wrongchoice{A flower falls \SI{2}{\meter} off of a tree to the ground.}
        \wrongchoice{A runner jogs around the lake at a speed of \SI{2}{\meter\per\second}.}
    \end{choices}
\end{question}
}

%% Graphing Howto
%%--------------------
\element{basic}{
\begin{question}{basic-Q11}
    How do you find the velocity of an object from a displacement-time graph?
    \begin{choices}
        \wrongchoice{Read the numbers on the x-axis}
        \wrongchoice{Read the numbers on the y-axis}
        \wrongchoice{Sum the area under the line}
      \correctchoice{Calculate the slope of the line}
        \wrongchoice{You cannot find velocity from a displacement-time graph}
    \end{choices}
\end{question}
}

\element{basic}{
\begin{question}{basic-Q12}
    How do you find the velocity of an object from an acceleration-time graph?
    \begin{choices}
        \wrongchoice{Read the numbers on the x-axis}
        \wrongchoice{Read the numbers on the y-axis}
      \correctchoice{Sum the area under the line}
        \wrongchoice{Calculate the slope of the line}
        \wrongchoice{You cannot find velocity from an acceleration-time graph}
    \end{choices}
\end{question}
}

\element{basic}{
\begin{question}{basic-Q13}
    How do you find the displacement of an object from a velocity graph?
    \begin{choices}
        \wrongchoice{Read the numbers on the x-axis}
        \wrongchoice{Read the numbers on the y-axis}
      \correctchoice{Sum the area under the line}
        \wrongchoice{Calculate the slope of the line}
        \wrongchoice{You cannot find velocity from a velocity-time graph}
    \end{choices}
\end{question}
}

\element{basic}{
\begin{question}{basic-Q14}
    How do you find the acceleration of an object from a velocity-time graph?
    \begin{choices}
        \wrongchoice{Read the numbers on the x-axis}
        \wrongchoice{Read the numbers on the y-axis}
        \wrongchoice{Sum the area under the line}
      \correctchoice{Calculate the slope of the line}
        \wrongchoice{You cannot find acceleration from a velocity-time graph}
    \end{choices}
\end{question}
}

\element{basic}{
\begin{question}{basic-Q15}
    The area under the curve in a velocity-time graph yields
    \begin{choices}
      \correctchoice{displacement}
        \wrongchoice{velocity}
        \wrongchoice{time}
        \wrongchoice{acceleration}
        \wrongchoice{no kinematic significance}
    \end{choices}
\end{question}
}

\element{basic}{
\begin{question}{basic-Q16}
    The area under the curve in a displacement-time graph yields
    \begin{choices}
        \wrongchoice{displacement}
        \wrongchoice{velocity}
        \wrongchoice{time}
        \wrongchoice{acceleration}
      \correctchoice{no kinematic significance}
    \end{choices}
\end{question}
}

\element{basic}{
\begin{question}{basic-Q17}
    The area under the curve in a displacement-time graph yields
    \begin{choices}
        \wrongchoice{displacement}
        \wrongchoice{velocity}
        \wrongchoice{time}
        \wrongchoice{acceleration}
      \correctchoice{no kinematic significance}
    \end{choices}
\end{question}
}

\element{basic}{
\begin{question}{basic-Q18}
    The slope of the curve in a displacement-time graph yields
    \begin{choices}
        \wrongchoice{displacement}
      \correctchoice{velocity}
        \wrongchoice{time}
        \wrongchoice{acceleration}
        \wrongchoice{no kinematic significance}
    \end{choices}
\end{question}
}

\element{basic}{
\begin{question}{basic-Q19}
    The slope of the curve in a velocity-time graph yields
    \begin{choices}
        \wrongchoice{displacement}
        \wrongchoice{velocity}
        \wrongchoice{time}
      \correctchoice{acceleration}
        \wrongchoice{no kinematic significance}
    \end{choices}
\end{question}
}


%% Misc
%%----------
\element{basic}{
\begin{question}{basic-Q21}
    The acceleration of any falling object is \SI{9.8}{\meter\per\second\squared}.
    What does this mean for the object's motion?
    \begin{choices}
        \wrongchoice{The object has a constant velocity of \SI{9.8}{\meter\per\second}.}
      \correctchoice{The object's velocity will increase by \SI{9.8}{\meter\per\second} each second.}
        \wrongchoice{The object will speed up to \SI{9.8}{\meter\per\second}, but will never travel faster than \SI{9.8}{\meter\per\second}.}
        \wrongchoice{The object travels \SI{9.8}{\meter} each second.}
    \end{choices}
\end{question}
}

\element{basic}{
\begin{question}{basic-Q22}
    Distance is a 
    \begin{multicols}{2}
    \begin{choices}
      \correctchoice{scalar quantity}
        \wrongchoice{vector quantity}
        \wrongchoice{derived unit}
        \wrongchoice{defined unit}
    \end{choices}
    \end{multicols}
\end{question}
}

\element{basic}{
\begin{question}{basic-Q23}
    Speed is a 
    \begin{multicols}{2}
    \begin{choices}
      \correctchoice{scalar quantity}
        \wrongchoice{vector quantity}
        \wrongchoice{derived unit}
        \wrongchoice{defined unit}
    \end{choices}
    \end{multicols}
\end{question}
}

\element{basic}{
\begin{question}{basic-Q24}
    Displacement is a 
    \begin{multicols}{2}
    \begin{choices}
        \wrongchoice{scalar quantity}
      \correctchoice{vector quantity}
        \wrongchoice{derived unit}
        \wrongchoice{defined unit}
    \end{choices}
    \end{multicols}
\end{question}
}

\element{basic}{
\begin{question}{basic-Q25}
    Velocity is a
    \begin{multicols}{2}
    \begin{choices}
        \wrongchoice{scalar quantity}
      \correctchoice{vector quantity}
        \wrongchoice{derived unit}
        \wrongchoice{defined unit}
    \end{choices}
    \end{multicols}
\end{question}
}

\element{basic}{
\begin{question}{basic-Q26}
    Acceleration is a
    \begin{multicols}{2}
    \begin{choices}
        \wrongchoice{scalar quantity}
      \correctchoice{vector quantity}
        \wrongchoice{derived unit}
        \wrongchoice{defined unit}
    \end{choices}
    \end{multicols}
\end{question}
}

\element{basic}{
\begin{question}{basic-Q27}
    A meter (\si{\meter}) is a
    \begin{multicols}{2}
    \begin{choices}
        \wrongchoice{scalar quantity}
        \wrongchoice{vector quantity}
        \wrongchoice{derived unit}
      \correctchoice{defined unit}
    \end{choices}
    \end{multicols}
\end{question}
}

\endinput


