
%%--------------------------------------------------
%% Schaum's Outline of Applied Physics
%%--------------------------------------------------


%% Chapter 17: Fluids in Motion
%%--------------------------------------------------


%% Schaum's Multiple Choice Questions
%%--------------------------------------------------
\element{schaums-mc}{
\begin{question}{ch17-Q01}
    When a fast train goes through a station,
        someone standing at the edge of the platform is:
    \begin{choices}
      \correctchoice{pulled toward the train}
        \wrongchoice{pushed away from the train}
        \wrongchoice{either pulled or pushed, depending on the ratio between the velocity of the train and the velocity of sound}
        \wrongchoice{not affected by the train's passage}
    \end{choices}
\end{question}
}

\element{schaums-mc}{
\begin{questionmult}{ch17-Q02}
    Water leaks out of a hole in the bottom of a water tank. 
    The rate at which water flows out depends on which one or more of the following?
    \begin{choices}
        \wrongchoice{the density of water}
      \correctchoice{the acceleration of gravity}
      \correctchoice{the area of the hole}
      \correctchoice{the height of the water}
    \end{choices}
\end{questionmult}
}

\element{schaums-mc}{
\begin{question}{ch17-Q03}
    The Reynolds number for fluid flow in a pipe is independent of:
    \begin{choices}
        \wrongchoice{the viscosity of the fluid}
        \wrongchoice{the velocity of the fluid}
      \correctchoice{the length of the pipe}
        \wrongchoice{the diameter of the pipe}
    \end{choices}
\end{question}
}

\element{schaums-mc}{
\begin{question}{ch17-Q04}
    Blood flows at \SI{0.10}{\liter\per\second} through the circulatory system of a person whose capillaries have a total cross-sectional area of \SI{0.25}{\meter\squared}. 
    The average velocity of blood in the capillaries is:
    \begin{multicols}{2}
    \begin{choices}
      \correctchoice{\SI{0.4}{\milli\meter\per\second}}
        \wrongchoice{\SI{4}{\milli\meter\per\second}}
        \wrongchoice{\SI{25}{\milli\meter\per\second}}
        \wrongchoice{\SI{40}{\centi\meter\per\second}}
    \end{choices}
    \end{multicols}
\end{question}
}

\element{schaums-mc}{
\begin{question}{ch17-Q05}
    Blood flows at \SI{0.10}{\liter\per\second} through the circulatory system of a person whose capillaries have a total cross-sectional area of \SI{0.25}{\meter\squared}. 
    The average velocity of blood in the capillaries is:
    %% NOTE: reword
    The average blood pressure of the person in Question 17.4 is \SI{14}{\kilo\pascal}. 
    The power developed by the person’s heart,
        which beats 1.2 times per second,
        in pumping blood against this pressure is:
    \begin{multicols}{2}
    \begin{choices}
      \correctchoice{\SI{1.2}{\watt}}
        \wrongchoice{\SI{1.4}{\watt}}
        \wrongchoice{\SI{1.7}{\watt}}
        \wrongchoice{\SI{12}{\watt}}
    \end{choices}
    \end{multicols}
\end{question}
}

\element{schaums-mc}{
\begin{question}{ch17-Q06}
    A boiler in which the gauge pressure is \SI{4.0}{\bar} develops a leak. 
    The velocity at which water comes out is:
    \begin{multicols}{2}
    \begin{choices}
        \wrongchoice{\SI{89}{\milli\meter\per\second}}
        \wrongchoice{\SI{20}{\meter\per\second}}
      \correctchoice{\SI{28}{\meter\per\second}}
        \wrongchoice{\SI{0.80}{\kilo\meter\per\second}}
    \end{choices}
    \end{multicols}
\end{question}
}

\element{schaums-mc}{
\begin{question}{ch17-Q07}
    Water comes out of a valve at the bottom of a tank at \SI{20}{\foot\per\second}. 
    The height of water in the tank is:
    \begin{multicols}{2}
    \begin{choices}
      \correctchoice{\SI{6.25}{\foot}}
        \wrongchoice{\SI{12.5}{\foot}}
        \wrongchoice{\SI{20}{\foot}}
        \wrongchoice{\SI{25}{\foot}}
    \end{choices}
    \end{multicols}
\end{question}
}

\element{schaums-mc}{
\begin{question}{ch17-Q08}
    Water flows through a pipe whose cross-sectional area is \SI{50}{\centi\meter\squared} into another pipe whose cross-sectional area is \SI{10}{\centi\meter\squared}. 
    Both pipes are horizontal. 
    If the velocity of the water in the large pipe is \SI{1.2}{\meter\per\second},
        the velocity in the small pipe is:
    \begin{multicols}{2}
    \begin{choices}
        \wrongchoice{\SI{0.24}{\meter\per\second}}
        \wrongchoice{\SI{1.2}{\meter\per\second}}
      \correctchoice{\SI{6.0}{\meter\per\second}}
        \wrongchoice{\SI{30}{\meter\per\second}}
    \end{choices}
    \end{multicols}
\end{question}
}

\element{schaums-mc}{
\begin{question}{ch17-Q09}
    Water flows through a pipe whose cross-sectional area is \SI{50}{\centi\meter\squared} into another pipe whose cross-sectional area is \SI{10}{\centi\meter\squared}. 
    Both pipes are horizontal. 
    If the velocity of the water in the large pipe is \SI{1.2}{\meter\per\second},
        the velocity in the small pipe is:
    %% NOTE: reword
    Water flows through the small pipe of Question 17.8 at
    \begin{multicols}{2}
    \begin{choices}
        \wrongchoice{\SI{0.6}{\liter\per\second}}
        \wrongchoice{\SI{1.2}{\liter\per\second}}
        \wrongchoice{\SI{3.0}{\liter\per\second}}
      \correctchoice{\SI{6.0}{\liter\per\second}}
    \end{choices}
    \end{multicols}
\end{question}
}

\element{schaums-mc}{
\begin{question}{ch17-Q10}
    Water flows through a pipe whose cross-sectional area is \SI{50}{\centi\meter\squared} into another pipe whose cross-sectional area is \SI{10}{\centi\meter\squared}. 
    Both pipes are horizontal. 
    If the velocity of the water in the large pipe is \SI{1.2}{\meter\per\second},
        the velocity in the small pipe is:
    %% NOTE: reword
    If the gauge pressure in the large pipe of Question 17.8 is \SI{200}{\kilo\pascal},
        the gauge pressure in the small pipe is:
    \begin{multicols}{2}
    \begin{choices}
        \wrongchoice{\SI{181}{\kilo\pascal}}
        \wrongchoice{\SI{182}{\kilo\pascal}}
      \correctchoice{\SI{183}{\kilo\pascal}}
        \wrongchoice{\SI{217}{\kilo\pascal}}
    \end{choices}
    \end{multicols}
\end{question}
}


\endinput

