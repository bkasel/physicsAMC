
%%--------------------------------------------------
%% Schaum's Outline of Applied Physics
%%--------------------------------------------------


%% Chapter 18: Heat
%%--------------------------------------------------


%% Schaum's Multiple Choice Questions
%%--------------------------------------------------
\element{schaums-mc}{
\begin{question}{ch18-Q01}
    Two thermometers,
        one calibrated in \si{\degree\Fahrenheit} and the other in \si{\degreeCelsius},
        are used to measure the same temperature. 
    The reading in \si{\degreeCelsius}:
    \begin{choices}
        \wrongchoice{is proportional to that in \si{\degree\Fahrenheit}}
        \wrongchoice{is less than that in \si{\degree\Fahrenheit}}
        \wrongchoice{is greater than that in \si{\degree\Fahrenheit}}
      \correctchoice{may be less or greater than that in \si{\degree\Fahrenheit}}
    \end{choices}
\end{question}
}

\element{schaums-mc}{
\begin{question}{ch18-Q02}
    Nitrogen boils at \SI{-196}{\degreeCelsius}.
    The nheit equivalent of this temperature is:
    \begin{multicols}{2}
    \begin{choices}
        \wrongchoice{\SI{-228}{\degree\Fahrenheit}}
        \wrongchoice{\SI{-295}{\degree\Fahrenheit}}
      \correctchoice{\SI{-321}{\degree\Fahrenheit}}
        \wrongchoice{\SI{-385}{\degree\Fahrenheit}}
    \end{choices}
    \end{multicols}
\end{question}
}

\element{schaums-mc}{
\begin{question}{ch18-Q03}
    Lead melts at \SI{626}{\degree\Fahrenheit}.
    The Celsius equivalent of this temperature is:
    \begin{multicols}{2}
    \begin{choices}
        \wrongchoice{\SI{316}{\degreeCelsius}}
      \correctchoice{\SI{330}{\degreeCelsius}}
        \wrongchoice{\SI{366}{\degreeCelsius}}
        \wrongchoice{\SI{1069}{\degreeCelsius}}
    \end{choices}
    \end{multicols}
\end{question}
}

\element{schaums-mc}{
\begin{question}{ch18-Q04}
    Steam at \SI{100}{\degreeCelsius} is more dangerous than the same mass of water at \SI{100}{\degreeCelsius} because the steam:
    \begin{choices}
        \wrongchoice{moves faster}
        \wrongchoice{is less dense}
      \correctchoice{contains more internal energy}
        \wrongchoice{has a higher specific heat capacity}
    \end{choices}
\end{question}
}

\element{schaums-mc}{
\begin{question}{ch18-Q05}
    When a liquid evaporates:
    \begin{choices}
        \wrongchoice{it gives off heat}
      \correctchoice{it absorbs heat}
        \wrongchoice{its temperature drops}
        \wrongchoice{its temperature rises}
    \end{choices}
\end{question}
}

\element{schaums-mc}{
\begin{question}{ch18-Q06}
    Which of the following substances requires the most heat per kilogram for a temperature rise of \SI{1}{\degreeCelsius}?
    \begin{multicols}{2}
    \begin{choices}
      \correctchoice{water}
        \wrongchoice{ice}
        \wrongchoice{steam}
        \wrongchoice{copper}
    \end{choices}
    \end{multicols}
\end{question}
}

\element{schaums-mc}{
\begin{question}{ch18-Q07}
    The freezing point of water is \SI{0}{\degreeCelsius}.
    Its melting point is:
    \begin{choices}
      \correctchoice{slightly lower than \SI{0}{\degreeCelsius}}
        \wrongchoice{\SI{0}{\degreeCelsius}}
        \wrongchoice{slightly higher than \SI{0}{\degreeCelsius}}
        \wrongchoice{\SI{32}{\degreeCelsius}}
    \end{choices}
\end{question}
}

\element{schaums-mc}{
\begin{question}{ch18-Q08}
    Food cooks faster in a pressure cooker than in an ordinary pot with a loose lid because the higher pressure:
    \begin{choices}
        \wrongchoice{forces heat into the food more rapidly}
        \wrongchoice{lowers the boiling point of water}
      \correctchoice{raises the boiling point of water}
        \wrongchoice{increases the specific heat capacity of water}
    \end{choices}
\end{question}
}

\element{schaums-mc}{
\begin{question}{ch18-Q09}
    In an hour a \SI{1000}{\watt} electric heater produces:
    \begin{multicols}{2}
    \begin{choices}
        \wrongchoice{\SI{3.4}{Btu}}
        \wrongchoice{\SI{1054}{Btu}}
      \correctchoice{\SI{3416}{Btu}}
        \wrongchoice{\SI{3600}{Btu}}
    \end{choices}
    \end{multicols}
\end{question}
}

\element{schaums-mc}{
\begin{question}{ch18-Q10}
    When \SI{20}{\kilo\joule} of heat is removed from \SI{1.2}{\kilo\gram} of ice originally at \SI{-15}{\degreeCelsius},
        its new temperature is:
    \begin{multicols}{2}
    \begin{choices}
        \wrongchoice{\SI{-18}{\degreeCelsius}}
      \correctchoice{\SI{-23}{\degreeCelsius}}
        \wrongchoice{\SI{-26}{\degreeCelsius}}
        \wrongchoice{\SI{-35}{\degreeCelsius}}
    \end{choices}
    \end{multicols}
\end{question}
}

\element{schaums-mc}{
\begin{question}{ch18-Q11}
    A hot liquid at \SI{80}{\degreeCelsius} is added to \SI{600}{\gram} of the same liquid originally at \SI{10}{\degreeCelsius}. 
    When the mixture reaches \SI{30}{\degreeCelsius},
        the total mass of liquid is:
    \begin{choices}
        \wrongchoice{\SI{825}{\gram}}
      \correctchoice{\SI{840}{\gram}}
        \wrongchoice{\SI{857}{\gram}}
        \wrongchoice{Impossible to calculate without knowing the specific heat capacity of the liquid.}
    \end{choices}
\end{question}
}

\element{schaums-mc}{
\begin{question}{ch18-Q12}
    If \SI{400}{\gram} of water at \SI{10}{\degreeCelsius} is poured into a \SI{600}{\gram} pitcher [$c=\SI{0.80}{\kilo\joule\per\kilo\gram\per\degreeCelsius}$] at \SI{20}{\degreeCelsius},
        the final temperature of the water is:
    \begin{multicols}{2}
    \begin{choices}
        \wrongchoice{\SI{11}{\degreeCelsius}}
      \correctchoice{\SI{12}{\degreeCelsius}}
        \wrongchoice{\SI{14}{\degreeCelsius}}
        \wrongchoice{\SI{17}{\degreeCelsius}}
    \end{choices}
    \end{multicols}
\end{question}
}

\element{schaums-mc}{
\begin{question}{ch18-Q13}
    A \SI{1.0}{\kilo\gram} iron bar [$c=\SI{0.11}{\kilo\calorie\per\kilo\gram\per\degreeCelsius}$] at \SI{100}{\degreeCelsius} is placed in \SI{3.0}{\kilo\gram} of water at \SI{15}{\degreeCelsius}.
    The temperature of the water increases by:
    \begin{multicols}{2}
    \begin{choices}
        \wrongchoice{\SI{0.7}{\degreeCelsius}}
      \correctchoice{\SI{3}{\degreeCelsius}}
        \wrongchoice{\SI{5}{\degreeCelsius}}
        \wrongchoice{\SI{18}{\degreeCelsius}}
    \end{choices}
    \end{multicols}
\end{question}
}

\element{schaums-mc}{
\begin{question}{ch18-Q14}
    When \SI{10}{\pound} of water at \SI{50}{\degree\Fahrenheit} is poured over \SI{1.0}{\pound} of ice at \SI{0}{\degree\Fahrenheit},
        the resulting mixture is at:
    \begin{multicols}{2}
    \begin{choices}
        \wrongchoice{\SI{19}{\degree\Fahrenheit}}
        \wrongchoice{\SI{31}{\degree\Fahrenheit}}
        \wrongchoice{\SI{32}{\degree\Fahrenheit}}
      \correctchoice{\SI{34}{\degree\Fahrenheit}}
    \end{choices}
    \end{multicols}
\end{question}
}

\element{schaums-mc}{
\begin{question}{ch18-Q15}
    Which one or more of the following combinations will result in water at \SI{50}{\degreeCelsius}?
    \begin{choices}
        \wrongchoice{\SI{1}{\kilo\gram} each of ice at \SI{0}{\degreeCelsius} and steam at \SI{100}{\degreeCelsius}}
        \wrongchoice{\SI{1}{\kilo\gram} each of ice at \SI{0}{\degreeCelsius} and water at \SI{100}{\degreeCelsius}}
        \wrongchoice{\SI{1}{\kilo\gram} each of water at \SI{0}{\degreeCelsius} and steam at \SI{100}{\degreeCelsius}}
      \correctchoice{\SI{1}{\kilo\gram} each of water at \SI{0}{\degreeCelsius} and water at \SI{100}{\degreeCelsius}}
    \end{choices}
\end{question}
}

\element{schaums-mc}{
\begin{question}{ch18-Q16}
    If \SI{3}{\mega\joule} of heat is removed from \SI{1}{\kilo\gram} of steam at \SI{200}{\degreeCelsius},
        the result is:
    \begin{choices}
        \wrongchoice{ice}
      \correctchoice{water and ice}
        \wrongchoice{water}
        \wrongchoice{water and steam}
    \end{choices}
\end{question}
}


\endinput


