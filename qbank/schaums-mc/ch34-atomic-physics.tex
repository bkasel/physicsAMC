
%%--------------------------------------------------
%% Schaum's Outline of Applied Physics
%%--------------------------------------------------


%% Chapter 34: Atomic Physics
%%--------------------------------------------------


%% Schaum's Multiple Choice Questions
%%--------------------------------------------------
\element{schaums-mc}{
\begin{question}{ch34-Q01}
    De Broglie waves can be regarded as waves of:
    \begin{multicols}{2}
    \begin{choices}
        \wrongchoice{energy}
        \wrongchoice{momentum}
        \wrongchoice{electric charge}
      \correctchoice{probability}
    \end{choices}
    \end{multicols}
\end{question}
}

\element{schaums-mc}{
\begin{question}{ch34-Q02}
    Wave behavior is exhibited by:
    \begin{choices}
        \wrongchoice{only particles at rest}
      \correctchoice{only moving particles}
        \wrongchoice{only charged particles}
        \wrongchoice{all particles}
    \end{choices}
\end{question}
}

\element{schaums-mc}{
\begin{question}{ch34-Q03}
    The velocity of the wave packet that corresponds to a moving particle is:
    \begin{choices}
        \wrongchoice{lower than the particle's velocity}
      \correctchoice{equal to the particle's velocity}
        \wrongchoice{higher than the particle's velocity}
        \wrongchoice{may be any of the provided}
    \end{choices}
\end{question}
}

\element{schaums-mc}{
\begin{question}{ch34-Q04}
    According to the uncertainty principle,
        it is impossible to precisely determine at the same time a particle's:
    \begin{choices}
        \wrongchoice{position and charge}
      \correctchoice{position and momentum}
        \wrongchoice{momentum and energy}
        \wrongchoice{charge and mass}
    \end{choices}
\end{question}
}

\element{schaums-mc}{
\begin{question}{ch34-Q05}
    A hydrogen atom is in its ground state when its electron is:
    \begin{choices}
        \wrongchoice{at rest}
        \wrongchoice{inside the nucleus}
      \correctchoice{in its lowest energy level}
        \wrongchoice{in its highest energy level}
    \end{choices}
\end{question}
}

\element{schaums-mc}{
\begin{question}{ch34-Q06}
    A photon is emitted by an atom when one of the atom's electrons:
    \begin{choices}
        \wrongchoice{leaves the atom}
        \wrongchoice{collides with another of its electrons}
      \correctchoice{shifts to a lower energy level}
        \wrongchoice{shifts to a higher energy level}
    \end{choices}
\end{question}
}

\element{schaums-mc}{
\begin{question}{ch34-Q07}
    The wavelengths in the bright-line emission spectrum of an element are:
    \begin{choices}
      \correctchoice{characteristic of the particular element}
        \wrongchoice{the same for all elements}
        \wrongchoice{evenly distributed throughout the visible spectrum}
        \wrongchoice{different from the wavelengths in its dark-line absorption spectrum}
    \end{choices}
\end{question}
}

\element{schaums-mc}{
\begin{questionmult}{ch34-Q08}
    When an atom absorbs a photon of light,
        which one or more of the following can happen?
    \begin{choices}
         \wrongchoice{An electron shifts to a state of smaller principal quantum number.}
       \correctchoice{An electron shifts to a state of higher principal quantum number.}
       \correctchoice{An electron leaves the atom.}
         \wrongchoice{An X-ray photon is emitted.}
    \end{choices}
\end{questionmult}
}

\element{schaums-mc}{
\begin{question}{ch34-Q09}
    The sun's spectrum consists of a bright background crossed by dark lines. 
    This suggests that the sun has a:
    \begin{choices}
        \wrongchoice{hot interior surrounded by a hot atmosphere}
      \correctchoice{hot interior surrounded by a cool atmosphere}
        \wrongchoice{cool interior surrounded by a hot atmosphere}
        \wrongchoice{cool interior surrounded by a cool atmosphere}
    \end{choices}
\end{question}
}

\element{schaums-mc}{
\begin{question}{ch34-Q10}
    A quantum number is not associated with an atomic electron's:
    \begin{choices}
      \correctchoice{mass}
        \wrongchoice{energy}
        \wrongchoice{spin}
        \wrongchoice{orbital angular momentum}
    \end{choices}
\end{question}
}

\element{schaums-mc}{
\begin{question}{ch34-Q11}
    The exclusion principle states that no two electrons in an atom can have the same:
    \begin{choices}
        \wrongchoice{velocity}
        \wrongchoice{orbit}
        \wrongchoice{spin}
      \correctchoice{set of quantum numbers}
    \end{choices}
\end{question}
}

\element{schaums-mc}{
\begin{questionmult}{ch34-Q12}
    Which one or more of the following events cannot raise an atom from its ground state to an excited state?
    \begin{choices}
      \correctchoice{spontaneous emission of a photon}
      \correctchoice{induced emission of a photon}
        \wrongchoice{absorption of a photon}
        \wrongchoice{a collision with another atom}
    \end{choices}
\end{questionmult}
}

\element{schaums-mc}{
\begin{question}{ch34-Q13}
    The operation of the laser is based on which one or more of the following?
    \begin{choices}
        \wrongchoice{the uncertainty principle}
        \wrongchoice{the exclusion principle}
      \correctchoice{induced emission of radiation}
        \wrongchoice{interference of matter waves}
    \end{choices}
\end{question}
}

\element{schaums-mc}{
\begin{question}{ch34-Q14}
    The waves emitted by a laser do not:
    \begin{choices}
        \wrongchoice{all have the same wavelength}
        \wrongchoice{emerge in step with one another}
        \wrongchoice{form a narrow beam}
      \correctchoice{have higher photon energies than waves of the same frequency from an ordinary source}
    \end{choices}
\end{question}
}

\element{schaums-mc}{
\begin{question}{ch34-Q15}
    An electron with a velocity of \SI{1.5e7}{\meter\per\second} has a de Broglie wavelength of:
    \begin{multicols}{2}
    \begin{choices}
        \wrongchoice{\SI{9.1e-57}{\meter}}
        \wrongchoice{\SI{6.5e-18}{\meter}}
      \correctchoice{\SI{4.9e-11}{\meter}}
        \wrongchoice{\SI{4.9e-10}{\meter}}
    \end{choices}
    \end{multicols}
\end{question}
}

\element{schaums-mc}{
\begin{question}{ch34-Q16}
    The kinetic energy of a neutron (mass \SI{1.675e-22}{\kilo\gram}) whose de Broglie wavelength is \SI{2.0e-14}{\meter} is:
    \begin{multicols}{2}
    \begin{choices}
        \wrongchoice{\SI{3.3e-13}{\eV}}
        \wrongchoice{\SI{0.21}{\eV}}
      \correctchoice{\SI{2.0}{\mega\eV}}
        \wrongchoice{\SI{0.37}{\giga\eV}}
    \end{choices}
    \end{multicols}
\end{question}
}


\endinput


