
%%--------------------------------------------------
%% Schaum's Outline of Applied Physics
%%--------------------------------------------------


%% Chapter 9: Circular Motion and Gravitation
%%--------------------------------------------------


%% Schaum's Multiple Choice Questions
%%--------------------------------------------------
\element{schaums-mc}{
\begin{question}{ch09-Q01}
    A \SI{500}{\gram} ball moves in a horizontal circle \SI{40}{\centi\meter} in radius at \SI{4}{\meter\per\second}. 
    Its centripetal acceleration is:
    \begin{multicols}{2}
    \begin{choices}
        \wrongchoice{\SI{10}{\meter\per\second\squared}}
        \wrongchoice{\SI{20}{\meter\per\second\squared}}
      \correctchoice{\SI{40}{\meter\per\second\squared}}
        \wrongchoice{\SI{80}{\meter\per\second\squared}}
    \end{choices}
    \end{multicols}
\end{question}
}

\element{schaums-mc}{
\begin{question}{ch09-Q02}
    An object in uniform circular motion is being acted upon by a centripetal force $F$. 
    If the radius of the object's path is to be doubled while its velocity remains the same,
        the new centripetal force must be:
    \begin{multicols}{2}
    \begin{choices}
      \correctchoice{$F/2$}
        \wrongchoice{$F$}
        \wrongchoice{$2F$}
        \wrongchoice{$4F$}
    \end{choices}
    \end{multicols}
\end{question}
}

\element{schaums-mc}{
\begin{question}{ch09-Q03}
    A \SI{1200}{\kilo\gram} car whose velocity is \SI{6}{\meter\per\second} rounds a turn whose radius is \SI{30}{\meter}. 
    The centripetal force on the car is:
    \begin{multicols}{2}
    \begin{choices}
        \wrongchoice{\SI{48}{\newton}}
        \wrongchoice{\SI{147}{\newton}}
        \wrongchoice{\SI{240}{\newton}}
      \correctchoice{\SI{1440}{\newton}}
    \end{choices}
    \end{multicols}
\end{question}
}

\element{schaums-mc}{
\begin{question}{ch09-Q04}
    A \SI{1200}{\kilo\gram} car whose velocity is \SI{6}{\meter\per\second} rounds a turn whose radius is \SI{30}{\meter}. 
    The centripetal force on the car is:
    %% NOTE: change wording
    If the car of Question 9.3 rounds the same turn at \SI{12}{\meter\per\second},
        the required centripetal force is:
    \begin{multicols}{2}
    \begin{choices}
        \wrongchoice{halved}
        \wrongchoice{the same}
        \wrongchoice{doubled}
      \correctchoice{quadrupled}
    \end{choices}
    \end{multicols}
\end{question}
}

\element{schaums-mc}{
\begin{question}{ch09-Q05}
    A \SI{1200}{\kilo\gram} car whose velocity is \SI{6}{\meter\per\second} rounds a turn whose radius is \SI{30}{\meter}. 
    The centripetal force on the car is:
    %% NOTE: change wording
    The maximum centripetal force that friction can provide the car of Question 9.3 on a rainy day is \SI{8000}{\newton}. 
    The highest velocity at which the car can round the turn is:
    \begin{multicols}{2}
    \begin{choices}
      \correctchoice{\SI{14}{\meter\per\second}}
        \wrongchoice{\SI{77}{\meter\per\second}}
        \wrongchoice{\SI{200}{\meter\per\second}}
        \wrongchoice{\SI{40}{\kilo\meter\per\second}}
    \end{choices}
    \end{multicols}
\end{question}
}

\element{schaums-mc}{
\begin{question}{ch09-Q06}
    A \SI{1200}{\kilo\gram} car whose velocity is \SI{6}{\meter\per\second} rounds a turn whose radius is \SI{30}{\meter}. 
    The centripetal force on the car is:
    %% NOTE: change wording
    The rain stops, the road dries, and the coefficient of friction between the tires of the car of Question 9.3 and the road increases to 1.40 times its value in Question 9.5. The highest velocity at which the car can now round the turn is:
    \begin{choices}
        \wrongchoice{unchanged}
      \correctchoice{1.18 its former value}
        \wrongchoice{1.40 its former value}
        \wrongchoice{1.96 its former value}
    \end{choices}
\end{question}
}

\element{schaums-mc}{
\begin{question}{ch09-Q07}
    The coefficient of static friction between the tires of a truck whose velocity is \SI{60}{\kilo\meter\per\hour} and a road is \num{0.65}. 
    The smallest turning radius of the truck is:
    \begin{multicols}{2}
    \begin{choices}
        \wrongchoice{\SI{9.4}{\meter}}
        \wrongchoice{\SI{22}{\meter}}
      \correctchoice{\SI{44}{\meter}}
        \wrongchoice{\SI{565}{\meter}}
    \end{choices}
    \end{multicols}
\end{question}
}

\element{schaums-mc}{
\begin{question}{ch09-Q08}
    A \SI{500}{\gram} ball is whirled in a vertical circle at the end of a string \SI{60}{\centi\meter} long. 
    If the velocity of the ball at the bottom of the circle is \SI{4.0}{\meter\per\second},
        the tension in the string there is:
    \begin{multicols}{2}
    \begin{choices}
        \wrongchoice{\SI{4.9}{\newton}}
        \wrongchoice{\SI{8.4}{\newton}}
        \wrongchoice{\SI{13.3}{\newton}}
      \correctchoice{\SI{18.2}{\newton}}
    \end{choices}
    \end{multicols}
\end{question}
}

\element{schaums-mc}{
\begin{question}{ch09-Q09}
    A car just leaves the road when it passes over a hump whose radius is \SI{40}{\meter}. 
    The car's velocity is:
    \begin{choices}
      \correctchoice{\SI{20}{\meter\per\second}}
        \wrongchoice{\SI{62}{\meter\per\second}}
        \wrongchoice{\SI{392}{\meter\per\second}}
        \wrongchoice{impossible to find without knowing the car's mass}
    \end{choices}
\end{question}
}

\element{schaums-mc}{
\begin{question}{ch09-Q10}
    A woman whose mass is \SI{60}{\kilo\gram} on the earth's surface is in a spacecraft at an altitude of two earth radii. 
    Her mass there is:
    \begin{multicols}{2}
    \begin{choices}
      \correctchoice{\SI{6.7}{\kilo\gram}}
        \wrongchoice{\SI{15}{\kilo\gram}}
        \wrongchoice{\SI{20}{\kilo\gram}}
        \wrongchoice{\SI{60}{\kilo\gram}}
    \end{choices}
    \end{multicols}
\end{question}
}

\element{schaums-mc}{
\begin{question}{ch09-Q11}
    A woman whose mass is \SI{60}{\kilo\gram} on the earth's surface is in a spacecraft at an altitude of two earth radii. 
    Her mass there is:
    %% NOTE: change wording
    A man whose mass is \SI{80}{\kilo\gram} on the earth's surface is also in the spacecraft of Question 9.10. 
    His weight there is:
    \begin{multicols}{2}
    \begin{choices}
      \correctchoice{\SI{87}{\newton}}
        \wrongchoice{\SI{196}{\newton}}
        \wrongchoice{\SI{261}{\newton}}
        \wrongchoice{\SI{784}{\newton}}
    \end{choices}
    \end{multicols}
\end{question}
}

\element{schaums-mc}{
\begin{question}{ch09-Q12}
    An astronaut weighs \SI{3200}{\newton} on a planet whose mass is the same as that of the earth but whose radius is half that of the earth. 
    The astronaut's weight on the earth is:
    \begin{multicols}{2}
    \begin{choices}
        \wrongchoice{\SI{400}{\newton}}
      \correctchoice{\SI{800}{\newton}}
        \wrongchoice{\SI{1600}{\newton}}
        \wrongchoice{\SI{3200}{\newton}}
    \end{choices}
    \end{multicols}
\end{question}
}

\element{schaums-mc}{
\begin{question}{ch09-Q13}
    A satellite orbits the earth in a circle of radius \SI{8000}{\kilo\meter}. 
    At that distance from the earth $g=\SI{6.2}{\meter\per\second\squared}$. 
    The velocity of the satellite is:
    \begin{choices}
        \wrongchoice{\SI{0.90}{\kilo\meter\per\second}}
      \correctchoice{\SI{7.0}{\kilo\meter\per\second}}
        \wrongchoice{\SI{8.9}{\kilo\meter\per\second}}
        \wrongchoice{impossible to find without knowing the satellite's mass}
    \end{choices}
\end{question}
}


\endinput

