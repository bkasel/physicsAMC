
%%--------------------------------------------------
%% Schaum's Outline of Applied Physics
%%--------------------------------------------------


%% Chapter 13: Elasticity
%%--------------------------------------------------


%% Schaum's Multiple Choice Questions
%%--------------------------------------------------
\element{schaums-mc}{
\begin{questionmult}{ch13-Q01}
    The stress on a wire that supports a load depends on which one or more of the following?
    \begin{choices}
        \wrongchoice{the wire's length}
      \correctchoice{the wire's diameter}
      \correctchoice{acceleration of gravity}
      \correctchoice{the mass of the load}
    \end{choices}
\end{questionmult}
}

\element{schaums-mc}{
\begin{question}{ch13-Q02}
    A load causes a wire to stretch by the amount $s$.
    If the same load is applied to another wire of the same material but twice as long and with twice the diameter,
        the second wire will stretch by the amount:
    \begin{multicols}{2}
    \begin{choices}
        \wrongchoice{$s/4$}
      \correctchoice{$s/2$}
        \wrongchoice{$s$}
        \wrongchoice{$2s$}
    \end{choices}
    \end{multicols}
\end{question}
}

\element{schaums-mc}{
\begin{question}{ch13-Q03}
    A load causes a wire to stretch by the amount $s$. 
    If the same load is applied to another wire of the same material but twice as long and with twice the diameter,
        the second wire will stretch by the amount:
    %% NOTE: reword
    At its elastic limit,
        the first wire of Question 13.2 could support a load of $F$. 
    At its elastic limit,
        the second wire could support a load of:
    \begin{multicols}{2}
    \begin{choices}
        \wrongchoice{$F/2$}
        \wrongchoice{$F$}
        \wrongchoice{$2F$}
      \correctchoice{$4F$}
    \end{choices}
    \end{multicols}
\end{question}
}

\element{schaums-mc}{
\begin{question}{ch13-Q04}
    Young's modulus for iron is \SI{1.9e11}{\pascal}. 
    When an iron wire \SI{1.0}{\meter} long with a cross-sectional area of \SI{4.0}{\milli\meter\squared} supports a \SI{100}{\kilo\gram} load,
        the wire stretches by:
    \begin{multicols}{2}
    \begin{choices}
        \wrongchoice{\SI{0.0027}{\milli\meter}}
        \wrongchoice{\SI{0.27}{\milli\meter}}
      \correctchoice{\SI{1.3}{\milli\meter}}
        \wrongchoice{\SI{3.7}{\milli\meter}}
    \end{choices}
    \end{multicols}
\end{question}
}

\element{schaums-mc}{
\begin{question}{ch13-Q05}
    Young's modulus for brass is \SI{1.3e7}{\pound\per\inch\squared}.
    When a brass rod \SI{2}{\foot} and \SI{4}{\inch} long with a cross-sectional area of \SI{0.50}{\inch\squared} supports a \SI{400}{\pound}b load,
        the rod stretches by:
    \begin{multicols}{2}
    \begin{choices}
        \wrongchoice{\SI{0.00015}{\inch}}
      \correctchoice{\SI{0.0017}{\inch}}
        \wrongchoice{\SI{0.0034}{\inch}}
        \wrongchoice{\SI{0.029}{\inch}}
    \end{choices}
    \end{multicols}
\end{question}
}

\element{schaums-mc}{
\begin{question}{ch13-Q06}
    Young's modulus for aluminum is \SI{7.0e10}{\pascal}. 
    When an aluminum wire \SI{1.5}{\milli\meter} in diameter and \SI{50}{\centi\meter} long is stretched by \SI{1.0}{\milli\meter},
        the force applied to the wire is:
    \begin{multicols}{2}
    \begin{choices}
      \correctchoice{\SI{247}{\newton}}
        \wrongchoice{\SI{315}{\newton}}
        \wrongchoice{\SI{990}{\newton}}
        \wrongchoice{\SI{2.42}{\kilo\newton}}
    \end{choices}
    \end{multicols}
\end{question}
}

\element{schaums-mc}{
\begin{question}{ch13-Q07}
    Young's modulus for aluminum is \SI{7.0e10}{\pascal}. 
    When an aluminum wire \SI{1.5}{\milli\meter} in diameter and \SI{50}{\centi\meter} long is stretched by \SI{1.0}{\milli\meter},
        the force applied to the wire is:
    %% NOTE: reword
    The elastic limit of aluminum is \SI{1.8e8}{\pascal}. 
    The maximum mass the wire in Question 13.6 can support without going beyond its elastic limit is:
    \begin{multicols}{2}
    \begin{choices}
      \correctchoice{\SI{32}{\kilo\gram}}
        \wrongchoice{\SI{65}{\kilo\gram}}
        \wrongchoice{\SI{623}{\kilo\gram}}
        \wrongchoice{\SI{636}{\kilo\gram}}
    \end{choices}
    \end{multicols}
\end{question}
}

\element{schaums-mc}{
\begin{question}{ch13-Q08}
    The ultimate strength of concrete is \SI{2.0e7}{\pascal} in compression. 
    A concrete cube \SI{80}{\centi\meter} on each edge can support a maximum mass of:
    \begin{multicols}{2}
    \begin{choices}
        \wrongchoice{\SI{1.3e5}{\kilo\gram}}
      \correctchoice{\SI{1.3e6}{\kilo\gram}}
        \wrongchoice{\SI{1.6e6}{\kilo\gram}}
        \wrongchoice{\SI{1.3e7}{\kilo\gram}}
    \end{choices}
    \end{multicols}
\end{question}
}

\element{schaums-mc}{
\begin{question}{ch13-Q09}
    The shear strength of a steel sheet \SI{2.5}{\milli\meter} thick is \SI{3.0e8}{\pascal}.
    The force needed to punch a hole \SI{6.0}{\milli\meter} square in this sheet is:
    \begin{multicols}{2}
    \begin{choices}
        \wrongchoice{\SI{7.2}{\kilo\newton}}
      \correctchoice{\SI{18}{\kilo\newton}}
        \wrongchoice{\SI{27}{\kilo\newton}}
        \wrongchoice{\SI{18}{\mega\newton}}
    \end{choices}
    \end{multicols}
\end{question}
}

\element{schaums-mc}{
\begin{question}{ch13-Q10}
    The bulk modulus of kerosene is \SI{1.3}{\giga\pascal}. 
    When a pressure of \SI{1.8}{\mega\pascal} is applied to a liter of kerosene,
        its volume decreases by:
    \begin{multicols}{2}
    \begin{choices}
      \correctchoice{\SI{1.4}{\milli\liter}}
        \wrongchoice{\SI{2.3}{\milli\liter}}
        \wrongchoice{\SI{7.2}{\milli\liter}}
        \wrongchoice{\SI{14}{\milli\liter}}
    \end{choices}
    \end{multicols}
\end{question}
}

\endinput

