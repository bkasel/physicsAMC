

%% St. Louis Area Physics Teachers
%%----------------------------------------

%% SLAPT 1991
%%----------------------------------------
\element{slapt}{
\begin{question}{slapt-1991-q01}
    \begin{center}
    \begin{tikzpicture}
    \end{tikzpicture}
    \end{center}
    The velocity at $t^{\prime}$ of the particle moving on a straight line,
        whose position $x$ is plotted against time $t$ below, is the
    \begin{choices}
        \wrongchoice{slope of line $a$}
        \wrongchoice{intercept of line $b$}
        \wrongchoice{slope of line $b$}
        \wrongchoice{length of the curve from $t=0$ to $t=t^{\prime}$}
        \wrongchoice{shaded area of the graph}
    \end{choices}
\end{question}
}

\element{slapt}{
\begin{question}{slapt-1991-q02}
    \begin{center}
    \begin{tikzpicture}
    \end{tikzpicture}
    \end{center}
    If the adjacent graph represents the velocity vs time plot for a particle in straight-line motion,
        then its acceleration is positive for
    \begin{choices}
        \wrongchoice{$\SI{-2.5}{\second} < t < \SI{5.0}{\second}$}
        \wrongchoice{$\SI{0.0}{\second} < t < \SI{5.0}{\second}$}
        \wrongchoice{$\SI{5.0}{\second} < t < \SI{10.0}{\second}$}
        \wrongchoice{$\SI{0.0}{\second} < t < \SI{10.0}{\second}$}
        \wrongchoice{all $t$}
    \end{choices}
\end{question}
}

\element{slapt}{
\begin{question}{slapt-1991-q03}
    A particle in straight line motion starts at time $t=0$ with an initial velocity of \SI{+12.0}{\meter\per\second} and it has a constant acceleration of \SI{-3.00}{\meter\per\second}.
    %% Start question
    How far does it travel before coming to rest momentarily?
    \begin{multicols}{3}
    \begin{choices}
        \wrongchoice{\SI{12.0}{\meter}}
        \wrongchoice{\SI{24.0}{\meter}}
        \wrongchoice{\SI{36.0}{\meter}}
        \wrongchoice{\SI{48.0}{\meter}}
        \wrongchoice{\SI{72.0}{\meter}}
    \end{choices}
    \end{multicols}
\end{question}
}

\element{slapt}{
\begin{question}{slapt-1991-q04}
    A particle in straight line motion starts at time $t=0$ with an initial velocity of \SI{+12.0}{\meter\per\second} and it has a constant acceleration of \SI{-3.00}{\meter\per\second}.
    %% Start question
    At what time $t$ will it be back at its original position?
    \begin{multicols}{3}
    \begin{choices}
        \wrongchoice{\SI{2.00}{\second}}
        \wrongchoice{\SI{4.00}{\second}}
        \wrongchoice{\SI{6.00}{\second}}
        \wrongchoice{\SI{8.00}{\second}}
        \wrongchoice{\SI{12.0}{\second}}
    \end{choices}
    \end{multicols}
\end{question}
}

\element{slapt}{
\begin{question}{slapt-1991-q05}
    Suppose that a particle moving on a straight line (the $x$-axis) with constant acceleration started with nonzero initial velocity.
    Then for this particle the following plot description results in a straight line:
    \begin{multicols}{3}
    \begin{choices}
        \wrongchoice{$x$ vs $\sqrt{t}$}
        \wrongchoice{$x$ vs $t$}
        \wrongchoice{$x$ vs $t^2$}
        \wrongchoice{$x$ vs $v$}
        \wrongchoice{$x$ vs $v^2$}
    \end{choices}
    \end{multicols}
\end{question}
}

\element{slapt}{
\begin{question}{slapt-1991-q06}
    A car starts from rest and moves on a straight line with a constant acceleration of \SI{6.00}{\meter\per\second\squared}.
    How far does it travel in \SI{10.0}{\second}?
    \begin{multicols}{3}
    \begin{choices}
        \wrongchoice{\SI{16.0}{\meter}}
        \wrongchoice{\SI{30.0}{\meter}}
        \wrongchoice{\SI{60.0}{\meter}}
        \wrongchoice{\SI{300}{\meter}}
        \wrongchoice{\SI{600}{\meter}}
    \end{choices}
    \end{multicols}
\end{question}
}

\element{slapt}{
\begin{question}{slapt-1991-q07}
    For an object in free fall, the acceleration of its center of mass depends on:
    \begin{multicols}{2}
    \begin{choices}
        %% questionmult ??
        \wrongchoice{its shape}
        \wrongchoice{its size only}
        \wrongchoice{its mass only}
        \wrongchoice{its density}
        \wrongchoice{none of the previous}
    \end{choices}
    \end{multicols}
\end{question}
}

\element{slapt}{
\begin{question}{slapt-1991-q08}
    A force $F$ produces an acceleration $a$ on an object of mass $m$.
    A force of $3F$ is exerted on a second object and an acceleration of $8a$ results.
    What is the mass of the second object?
    \begin{multicols}{2}
    \begin{choices}
        \wrongchoice{$\dfrac{3m}{8}$}
        \wrongchoice{$\dfrac{8m}{3}$}
        \wrongchoice{$3m$}
        \wrongchoice{$8m$}
        \wrongchoice{$24m$}
    \end{choices}
    \end{multicols}
\end{question}
}

\element{slapt}{
\begin{question}{slapt-1991-q09}
    During uniform 
    \begin{choices}
        \wrongchoice{its $\vec{v}$ does not change with time and its $\vec{a}$ vanishes.}
    \end{choices}
\end{question}
}


\endinput


