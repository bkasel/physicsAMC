
%% XXXXXX Questions used on the
%% NYSED Physics Regents Examination
%%--------------------------------------------------

%% this section contains XX problems


%% Section June2015
%%--------------------
\element{nysed}{
\begin{question}{June2015-Q29}
    What is the minimum energy required to ionize a hydrogen atom in the $n=3$ state?
    \begin{multicols}{2}
    \begin{choices}
        \wrongchoice{\SI{0.00}{\eV}}
        \wrongchoice{\SI{0.66}{\eV}}
      \correctchoice{\SI{1.51}{\eV}}
        \wrongchoice{\SI{12.09}{\eV}}
    \end{choices}
    \end{multicols}
\end{question}
}


%% Section June2014
%%--------------------
\element{nysed}{
\begin{question}{June2014-Q43}
    Which electron transition between the energy levels of hydrogen causes the emission of a photon of visible light?
    \begin{multicols}{2}
    \begin{choices}
        \wrongchoice{$n = 6$ to $n = 5$}
      \correctchoice{$n = 5$ to $n = 2$}
        \wrongchoice{$n = 5$ to $n = 6$}
        \wrongchoice{$n = 2$ to $n = 5$}
    \end{choices}
    \end{multicols}
\end{question}
}


%% Section June2013
%%--------------------
\element{nysed}{
\begin{question}{June2013-Q34}
    A photon is emitted as the electron in a hydrogen atom drops from the $n=5$ energy level directly to the $n=3$ energy level.
    What is the energy of the emitted photon?
    \begin{multicols}{2}
    \begin{choices}
        \wrongchoice{\SI{0.85}{\eV}}
        \wrongchoice{\SI{1.51}{\eV}}
      \correctchoice{\SI{0.97}{\eV}}
        \wrongchoice{\SI{2.05}{\eV}}
    \end{choices}
    \end{multicols}
\end{question}
}


%% Section June2012
%%--------------------
\element{nysed}{
\begin{question}{June2012-Q44}
    Electrons in excited hydrogen atoms are in the $n=3$ energy level.
    How many different photon frequencies could be emitted as the atoms return to the ground state?
    \begin{multicols}{4}
    \begin{choices}
        \wrongchoice{\num{1}}
        \wrongchoice{\num{2}}
      \correctchoice{\num{3}}
        \wrongchoice{\num{4}}
    \end{choices}
    \end{multicols}
\end{question}
}


%% Section June2011
%%--------------------


%% Section June2010
%%--------------------


%% Section June2009
%%--------------------


%% Section Jan2009
%%--------------------


%% Section June2008
%%--------------------


%% Section Jan2008
%%--------------------


%% Section June2007
%%--------------------


%% Section Jan2007
%%--------------------


%% Section June2006
%%--------------------


%% Section Jan2006
%%--------------------


%% Section June2005
%%--------------------


%% Section Jan2005
%%--------------------


%% Section June2004
%%--------------------


%% Section Jan2004
%%--------------------


%% Section June2003
%%--------------------


%% Section Jan2003
%%--------------------


%% Section Aug2002
%%--------------------


%% Section June2002
%%--------------------


%% Section Jan2002
%%--------------------


%% Section June2001
%%--------------------


%% Section Jan2001
%%--------------------


%% Section June2000
%%--------------------


%% Section June1999
%%--------------------


%% Section June1998
%%--------------------


%% Section June1997
%%--------------------


%% Section June1996
%%--------------------


%% Section June1995
%%--------------------
\element{nysed}{
\begin{question}{June1995-Q48}
    Which graph below best represents the relationship between the frequency of a light source causing photoemission and the maximum kinetic energy (KE\textsubscript{max}) of the photoelectrons produced? 
    \begin{multicols}{2}
    \begin{choices}
        \AMCboxDimensions{down=-2.5em}
        \wrongchoice{
            \begin{tikzpicture}
                \begin{axis}[
                    axis y line=left,
                    axis x line=bottom,
                    axis line style={->},
                    xlabel={frequency},
                    xtick=\empty,
                    ylabel={KE\textsubscript{max}},
                    ytick=\empty,
                    xmin=0,xmax=11,
                    ymin=0,ymax=11,
                    width=\columnwidth,
                    very thin,
                ]
                \addplot[line width=1pt,domain=0:10]{8};
                \end{axis}
            \end{tikzpicture}
        }
        %% ANS is 2
        \correctchoice{
            \begin{tikzpicture}
                \begin{axis}[
                    axis y line=left,
                    axis x line=bottom,
                    axis line style={->},
                    xlabel={frequency},
                    xtick=\empty,
                    ylabel={KE\textsubscript{max}},
                    ytick=\empty,
                    xmin=0,xmax=11,
                    ymin=0,ymax=11,
                    width=\columnwidth,
                    very thin,
                ]
                \addplot[line width=1pt,mark=\empty] plot coordinates {(2,0) (10,10)};
                \end{axis}
            \end{tikzpicture}
        }
        \wrongchoice{
            \begin{tikzpicture}
                \begin{axis}[
                    axis y line=left,
                    axis x line=bottom,
                    axis line style={->},
                    xlabel={frequency},
                    xtick=\empty,
                    ylabel={KE\textsubscript{max}},
                    ytick=\empty,
                    xmin=0,xmax=11,
                    ymin=0,ymax=11,
                    width=\columnwidth,
                    very thin,
                ]
                \addplot[line width=1pt,mark=\empty] plot coordinates {(0,10) (10,0)};
                \end{axis}
            \end{tikzpicture}
        }
        \wrongchoice{
            \begin{tikzpicture}
                \begin{axis}[
                    axis y line=left,
                    axis x line=bottom,
                    axis line style={->},
                    xlabel={frequency},
                    xtick=\empty,
                    ylabel={KE\textsubscript{max}},
                    ytick=\empty,
                    xmin=0,xmax=11,
                    ymin=0,ymax=11,
                    width=\columnwidth,
                    very thin,
                ]
                \addplot[line width=1pt,domain=0:10] {10/x};
                \end{axis}
            \end{tikzpicture}
        }
    \end{choices}
    \end{multicols}
\end{question}
}

\element{nysed}{
\begin{question}{June1995-Q50}
    Which electron transition in the hydrogen atom results in the emission of a photon of greatest energy?
    \begin{multicols}{2}
    \begin{choices}
      \correctchoice{$n=2$ to $n=1$}
        \wrongchoice{$n=3$ to $n=2$}
        \wrongchoice{$n=4$ to $n=2$}
        \wrongchoice{$n=5$ to $n=3$}
    \end{choices}
    \end{multicols}
\end{question}
}

\element{nysed}{
\begin{question}{June1995-Q51}
    The term ``electron cloud'' refers to the:
    \begin{choices}
        \wrongchoice{electron plasma surrounding a hot wire}
        \wrongchoice{cathode rays in a gas discharge tube}
      \correctchoice{high probability region for an electron in an atom}
        \wrongchoice{negatively charged cloud that can produce a lightning strike}
    \end{choices}
\end{question}
}


%% Section June1994
%%--------------------
\element{nysed}{
\begin{question}{June1994-Q48}
    A metal surface emits photoelectrons when illuminated by green light.
    This surface must also emit photoelectrons when illuminated by:
    \begin{multicols}{2}
    \begin{choices}
      \correctchoice{blue light}
        \wrongchoice{yellow light}
        \wrongchoice{orange light}
        \wrongchoice{red light}
    \end{choices}
    \end{multicols}
\end{question}
}

\element{nysed}{
\begin{question}{June1994-Q52}
    The threshold frequency for a photoemissive surface is \SI{1.0e14}{\hertz}.
    What is the work function of this surface?
    \begin{multicols}{2}
    \begin{choices}
        \wrongchoice{\SI{1.0e-14}{\joule}}
      \correctchoice{\SI{6.6e-20}{\joule}}
        \wrongchoice{\SI{6.6e-48}{\joule}}
        \wrongchoice{\SI{2.2e-28}{\joule}}
    \end{choices}
    \end{multicols}
\end{question}
}

\element{nysed}{
\begin{question}{June1994-Q53}
    What is the \emph{minimum} amount of energy needed to ionize a mercury electron in the $c$ energy level?
    \begin{multicols}{2}
    \begin{choices}
        \wrongchoice{\SI{0.57}{\eV}}
        \wrongchoice{\SI{4.86}{\eV}}
      \correctchoice{\SI{5.52}{\eV}}
        \wrongchoice{\SI{10.38}{\eV}}
    \end{choices}
    \end{multicols}
\end{question}
}


%% Section June1986
%%--------------------
\element{nysed}{
\begin{question}{June1986-Q51}
    The absorption of a photon causes a hydrogen atom to change from the $n=2$ to the $n=3$ energy state.
    What is the energy of the absorbed photon?
    \begin{multicols}{2}
    \begin{choices}
      \correctchoice{\SI{1.9}{\eV}}
        \wrongchoice{\SI{3.4}{\eV}}
        \wrongchoice{\SI{4.9}{\eV}}
        \wrongchoice{\SI{10.2}{\eV}}
    \end{choices}
    \end{multicols}
\end{question}
}



\endinput


