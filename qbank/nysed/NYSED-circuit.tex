
%% DC Circuit and Kirchoff's Laws Questions for the
%% NYSED Physics Regents Examination
%%--------------------------------------------------

%% this section contains 84 problems


%% Section June2015
%%--------------------
\element{nysed}{
\begin{question}{June2015-Q19}
    If several resistors are connected in series in an electric circuit,
        the potential difference across each resistor:
    \begin{choices}
      \correctchoice{varies directly with its resistance.}
        \wrongchoice{varies inversely with its resistance.}
        \wrongchoice{varies inversely with the square of its resistance.}
        \wrongchoice{is independent of its resistance.}
    \end{choices}
\end{question}
}


%% Section June2014
%%--------------------


%% Section June2013
%%--------------------
\element{nysed}{
\begin{question}{June2013-Q31}
    The diagram below shows currents in a segment of an electric circuit.
    \begin{center}
    \ctikzset{bipoles/length=0.75cm}
    \begin{circuitikz}[scale=0.8]
        %% NOTE: this is a good reference
        \draw[thick] (-3,0) to [short,i^>=\SI{5}{\ampere}] (0,0)
                            to [short,i^>=\SI{3}{\ampere}] (3,0);
        \draw[thick] (0,2)  to [short,i^>=\SI{7}{\ampere}] (0,0)
                            to [ammeter] (0,-2);
    \end{circuitikz}
    \end{center}
    What is the reading of the ammeter?
    \begin{multicols}{2}
    \begin{choices}
        \wrongchoice{\SI{1}{\ampere}}
        \wrongchoice{\SI{5}{\ampere}}
      \correctchoice{\SI{9}{\ampere}}
        \wrongchoice{\SI{15}{\ampere}}
    \end{choices}
    \end{multicols}
\end{question}
}


%% Section June2012
%%--------------------
\element{nysed}{
\begin{question}{June2012-Q25}
    A \SI{3.0}{\ohm} resistor and a \SI{6.0}{\ohm} resistor are connected in parallel across a \SI{9}{\volt} battery.
    Which statement best compares the potential difference across each resistor?
    \begin{choices}
      \correctchoice{The potential difference across the \SI{6.0}{\ohm} resistor is the same as the potential difference across the \SI{3.0}{\ohm} resistor}
        \wrongchoice{The potential difference across the \SI{6.0}{\ohm} resistor is twice as great as the potential difference across the \SI{3.0}{\ohm} resistor}
        \wrongchoice{The potential difference across the \SI{6.0}{\ohm} resistor is half as great as the potential difference across the \SI{3.0}{\ohm} resistor}
        \wrongchoice{The potential difference across the \SI{6.0}{\ohm} resistor is four times as great as the potential difference across the \SI{3.0}{\ohm} resistor}
    \end{choices}
\end{question}
}



%% Section June2011
%%--------------------
\element{nysed}{
\begin{question}{June2011-Q22}
    Circuit $A$ has four \SI{3.0}{\ohm} resistors connected in series with a \SI{24}{\volt} battery,
        and circuit $B$ has two \SI{3.0}{\ohm} resistors connected in series with a \SI{24}{\volt} battery.
    Compared to the total potential drop across circuit $A$,
        the total potential drop across circuit $B$ is:
    \begin{choices}
        \wrongchoice{one-half as great}
        \wrongchoice{twice as great}
      \correctchoice{the same}
        \wrongchoice{four times as great}
    \end{choices}
\end{question}
}

\element{nysed}{
\begin{question}{June2011-Q44}
    The diagram below represents a circuit consisting of two resistors connected to a source of potential difference.
    \begin{center}
    \ctikzset{bipoles/length=0.75cm}
    \begin{circuitikz}[xscale=1.25]
        \draw (0,0) to [battery,l=$\SI{120}{\volt}$] (0,2)
                    to [R,l=$\SI{10}{\ohm}$] (2,2)
                    to [R,l=$\SI{20}{\ohm}$] (2,0)
                    to (0,0);
    \end{circuitikz}
    \end{center}
    What is the current through the \SI{20}{\ohm} resistor?
    \begin{multicols}{2}
    \begin{choices}
        \wrongchoice{\SI{0.25}{\ampere}}
        \wrongchoice{\SI{6.0}{\ampere}}
        \wrongchoice{\SI{12}{\ampere}}
      \correctchoice{\SI{4.0}{\ampere}}
    \end{choices}
    \end{multicols}
\end{question}
}


%% Section June2010
%%--------------------
\element{nysed}{
\begin{question}{June2010-Q23}
    Which circuit has the \emph{smallest} equivalent resistance?
    \begin{choices}
        \AMCboxDimensions{down=-0.6cm}
        \wrongchoice{
            \ctikzset{bipoles/length=0.75cm}
            \begin{circuitikz}[xscale=1.33]
                \draw[white] (-0.3cm,-0.3cm) rectangle (4.3cm,1.3cm);
                %% Parallel Two
                \draw (0,0) to [battery] (0,1)
                            to (4,1)
                            to [R,l=\SI{2}{\ohm}] (4,0)
                            to (0,0);
                \draw (2,1) to [R,l=\SI{2}{\ohm}] (2,0);
            \end{circuitikz}
        }
        \wrongchoice{
            \ctikzset{bipoles/length=0.75cm}
            \begin{circuitikz}[xscale=1.33]
                \draw[white] (-0.3cm,-0.3cm) rectangle (4.3cm,1.3cm);
                %% Series Two
                \draw (0,0) to [battery] (0,1)
                            to [R,l_=\SI{2}{\ohm}] (2,1) to (4,1) to (4,0)
                            to [R,l_=\SI{2}{\ohm}] (2,0) to (0,0);
            \end{circuitikz}
        }
        \correctchoice{
            \ctikzset{bipoles/length=0.75cm}
            \begin{circuitikz}[xscale=1.33]
                %% Parallel Four
                \draw[white] (-0.3cm,-0.3cm) rectangle (4.3cm,1.3cm);
                \draw (0,0) to [battery] (0,1)
                            to (4,1)
                            to [R,l=\SI{2}{\ohm}] (4,0)
                            to (0,0);
                \draw (1,1) to [R,l=\SI{2}{\ohm}] (1,0);
                \draw (2,1) to [R,l=\SI{2}{\ohm}] (2,0);
                \draw (3,1) to [R,l=\SI{2}{\ohm}] (3,0);
            \end{circuitikz}
        }
        \wrongchoice{
            \ctikzset{bipoles/length=0.75cm}
            \begin{circuitikz}[xscale=1.33]
                \draw[white] (-0.3cm,-0.3cm) rectangle (4.3cm,1.3cm);
                %% Series Four
                \draw (0,0) to [battery] (0,1)
                            to [R,l_=\SI{2}{\ohm}] (1,1) to (2,1)
                            to [R,l_=\SI{2}{\ohm}] (3,1) to (4,1) to (4,0)
                            to [R,l_=\SI{2}{\ohm}] (3,0) to (2,0)
                            to [R,l_=\SI{2}{\ohm}] (1,0) to (0,0);
            \end{circuitikz}
        }
    \end{choices}
\end{question}
}

\element{nysed}{
\begin{question}{June2010-Q45}
    The circuit diagram below represents four resistors connected to a \SI{12}{\volt} source.
    \begin{center}
    \ctikzset{bipoles/length=0.75cm}
    \begin{circuitikz}[scale=1.0]
        \draw (0,0) to [battery,l=$\SI{12}{\volt}$] (0,2)
                    to [R,l=$\SI{4.0}{\ohm}$] (2,2)
                    to [R,l=$\SI{6.0}{\ohm}$] (4,2)
                    to [R,l=$\SI{8.0}{\ohm}$] (4,0)
                    to [R,l_=$\SI{6.0}{\ohm}$] (0,0);
    \end{circuitikz}
    \end{center}
    What is the total current in the circuit?
    \begin{multicols}{2}
    \begin{choices}
      \correctchoice{\SI{0.50}{\ampere}}
        \wrongchoice{\SI{2.0}{\ampere}}
        \wrongchoice{\SI{8.6}{\ampere}}
        \wrongchoice{\SI{24}{\ampere}}
    \end{choices}
    \end{multicols}
\end{question}
}


%% Section June2009
%%--------------------
\element{nysed}{
\begin{question}{June2009-Q46}
    A \SI{3.0}{\ohm} resistor and a \SI{6.0}{\ohm} resistor are connected in series in an operating electric circuit.
    If the current through the \SI{3.0}{\ohm} resistor is \SI{4.0}{\ampere},
        what is the potential difference across the \SI{6.0}{\ohm} resistor?
    \begin{multicols}{2}
    \begin{choices}
        \wrongchoice{\SI{8.0}{\volt}}
        \wrongchoice{\SI{2.0}{\volt}}
        \wrongchoice{\SI{12}{\volt}}
      \correctchoice{\SI{24}{\volt}}
    \end{choices}
    \end{multicols}
\end{question}
}

\element{nysed}{
\begin{question}{June2009-Q47}
    Which combination of resistors has the \emph{smallest} equivalent resistance?
    \begin{multicols}{2}
    \begin{choices}
        \AMCboxDimensions{down=-0.80cm}
        \ctikzset{bipoles/length=0.75cm}
        \correctchoice{
            \begin{circuitikz}[scale=0.75]
                \draw[draw=white] (-2,-1.5) rectangle (2,1.5);
                \draw node[circ] (A) at (-2,0) {}
                      node[circ] (B) at ( 2,0) {}
                      (A) to (-1,0)
                          to (-1,0.5)
                          to [R,l=\SI{1}{\ohm}] (1,0.5)
                          to (1,0)
                      (B) to (1,0)
                          to (1,-0.5)
                          to [R,l=\SI{1}{\ohm}] (-1,-0.5)
                          to (-1,0);
            \end{circuitikz}
        }
        \wrongchoice{
            \begin{circuitikz}[scale=0.75]
                \draw[draw=white] (-2,-1.5) rectangle (2,1.5);
                \draw node[circ] (A) at (-2,0) {}
                      node[circ] (B) at ( 2,0) {}
                      (A) to (-1,0)
                          to (-1,0.5)
                          to [R,l=\SI{2}{\ohm}] (1,0.5)
                          to (1,0)
                      (B) to (1,0)
                          to (1,-0.5)
                          to [R,l=\SI{2}{\ohm}] (-1,-0.5)
                          to (-1,0);
            \end{circuitikz}
        }
        \wrongchoice{
            \begin{circuitikz}[scale=0.75]
                \draw[draw=white] (-2,-1.5) rectangle (2,1.5);
                \draw node[circ] (A) at (-2,0) {}
                      node[circ] (B) at ( 2,0) {}
                      (A) to [R,l=\SI{2}{\ohm}] (0,0)
                          to [R,l=\SI{2}{\ohm}] (B);
            \end{circuitikz}
        }
        \wrongchoice{
            \begin{circuitikz}[scale=0.75]
                \draw[draw=white] (-2,-1.5) rectangle (2,1.5);
                \draw node[circ] (A) at (-2,0) {}
                      node[circ] (B) at ( 2,0) {}
                      (A) to [R,l=\SI{1}{\ohm}] (0,0)
                          to [R,l=\SI{1}{\ohm}] (B);
            \end{circuitikz}
        }
    \end{choices}
    \end{multicols}
\end{question}
}


%% Section Jan2009
%%--------------------
\element{nysed}{
\begin{question}{Jan2009-Q17}
    Three identical lamps are connected in parallel with each other.
    If the resistance of each lamp is $X$ ohms,
        what is the equivalent resistance of this parallel combination?
    \begin{multicols}{2}
    \begin{choices}
        \wrongchoice{$X\si{\ohm}$}
      \correctchoice{$\dfrac{X}{3}\si{\ohm}$}
        \wrongchoice{$3X\si{\ohm}$}
        \wrongchoice{$\dfrac{3}{X}\si{\ohm}$}
    \end{choices}
    \end{multicols}
\end{question}
}

\element{nysed}{
\begin{question}{Jan2009-Q18}
    A \SI{2.0}{\ohm} resistor and a \SI{4.0}{\ohm} resistor are connected in series with a \SI{12}{\volt} battery.
    If the current through the \SI{2.0}{ohm} resistor is \SI{2.0}{\ampere},
        the current through the \SI{4.0}{\ohm} resistor is:
    \begin{multicols}{2}
    \begin{choices}
        \wrongchoice{\SI{1.0}{\ampere}}
      \correctchoice{\SI{2.0}{\ampere}}
        \wrongchoice{\SI{3.0}{\ampere}}
        \wrongchoice{\SI{4.0}{\ampere}}
    \end{choices}
    \end{multicols}
\end{question}
}

\element{nysed}{
\begin{question}{Jan2009-Q20}
    In which circuit would current flow through resistor $R_1$,
        but \emph{not} through $R_2$ while switch $S$ is open?
    \begin{multicols}{2}
    \begin{choices}
        \AMCboxDimensions{down=-2.5em}
        \ctikzset{bipoles/length=0.75cm}
        \correctchoice{
            \begin{circuitikz}[scale=1.0]
                \draw (0,0) to [battery] (0,2)
                            to (2,2)
                            to [R,l_=$R_2$] (2,0)
                            to [ospst=$S$,mirror] (1,0)
                            to (0,0);
                \draw (1,2) to [R,l_=$R_1$] (1,0);
            \end{circuitikz}
        }
        \wrongchoice{
            \begin{circuitikz}[scale=1.0]
                \draw (0,0) to [battery] (0,2)
                            to [R,l_=$R_1$] (2,2)
                            to [ospst=$S$,mirror] (2,0)
                            to [R,l_=$R_2$] (0,0);
            \end{circuitikz}
        }
        \wrongchoice{
            \begin{circuitikz}[scale=1.0]
                \draw (0,0) to [battery] (0,2)
                            to (2,2)
                            to [R,l_=$R_2$] (2,0)
                            to (1,0)
                            to [ospst=$S$,mirror] (0,0);
                \draw (1,2) to [R,l_=$R_1$] (1,0);
            \end{circuitikz}
        }
        \wrongchoice{
            \begin{circuitikz}[scale=1.0]
                \draw (0,0) to [battery] (0,2)
                            to (2,2)
                            to [R,l_=$R_2$] (2,0)
                            to (0,0);
                \draw (1,2) to [R,l_=$R_1$] (1,1)
                            to [ospst=$S$,mirror] (1,0);
            \end{circuitikz}
        }
    \end{choices}
    \end{multicols}
\end{question}
}

\element{nysed}{
\begin{question}{Jan2009-Q42}
    In the electric circuit diagram below,
        possible locations of an ammeter and a voltmeter are indicated by circles 1, 2, 3, and 4.
    \begin{center}
    \ctikzset{bipoles/length=1.00cm}
    \begin{circuitikz}[font=\small]
        %% jphafner
        \draw (0,0) to [battery] (0,3) to (2,3) to [R] (2,1.5) to (2,0);
        \draw (2,3) to (4,3) to [R] (4,0) to (0,0);
        %% 1,2,3,4
        \node[draw,circle,fill=white,anchor=center] at (1,3) {$1$};
        \node[draw,circle,fill=white,anchor=center] at (2,0.75) {$2$};
        \draw (0,0.5) -- (-1,0.5) -- (-1,2.5) node[pos=0.5,anchor=center,draw,circle,fill=white] {$4$} -- (0,2.5) ;
        \draw (4,0.5) -- (5,0.5) -- (5,2.5) node[pos=0.5,anchor=center,draw,circle,fill=white] {$3$} -- (4,2.5);
    \end{circuitikz}
    \end{center}
    Where should an ammeter be located to correctly measure the total current and where should a voltmeter be located to correctly measure the total voltage?
    \begin{choices}
      \correctchoice{ammeter at 1 and voltmeter at 4}
        \wrongchoice{ammeter at 2 and voltmeter at 3}
        \wrongchoice{ammeter at 3 and voltmeter at 4}
        \wrongchoice{ammeter at 1 and voltmeter at 2}
    \end{choices}
\end{question}
}

\element{nysed}{
\begin{question}{Jan2009-Q44}
    A \SI{150}{\watt} lightbulb is brighter than a \SI{60}{\watt} lightbulb when both are operating at a potential difference of \SI{110}{\volt}.
    Compared to the resistance of and the current drawn by the \SI{150}{\watt} lightbulb,
        the \SI{60}{\watt} lightbulb has:
    \begin{choices}
        \wrongchoice{less resistance and draws more current.}
        \wrongchoice{less resistance and draws less current.}
        \wrongchoice{more resistance and draws more current.}
      \correctchoice{more resistance and draws less current.}
    \end{choices}
\end{question}
}

\element{nysed}{
\begin{question}{Jan2009-Q45}
    What is the minimum equipment needed to determine the power dissipated in a resistor of unknown value?
    \begin{choices}
        \wrongchoice{a voltmeter, only}
        \wrongchoice{an ammeter, only}
      \correctchoice{a voltmeter and an ammeter, only}
        \wrongchoice{a voltmeter, an ammeter, and a stopwatch}
    \end{choices}
\end{question}
}


%% Section June2008
%%--------------------
\element{nysed}{
\begin{question}{June2008-Q20}
    Three resistors, \SI{4}{\ohm}, \SI{6}{\ohm}, and \SI{8}{\ohm},
        are connected in parallel in an electric circuit.
    The equivalent resistance of the circuit is:
    \begin{choices}
      \correctchoice{less than \SI{4}{\ohm}}
        \wrongchoice{between \SI{4}{\ohm} and \SI{8}{\ohm}}
        \wrongchoice{between \SI{10}{\ohm} and \SI{18}{\ohm}}
        \wrongchoice{\SI{18}{\ohm}}
    \end{choices}
\end{question}
}



%% Section Jan2008
%%--------------------
\element{nysed}{
\begin{question}{Jan2008-Q20}
    A circuit consists of a \SI{10.0}{\ohm} resistor,
        a \SI{15.0}{\ohm} resistor, and a \SI{20.0}{\ohm} resistor connected in parallel across a \SI{9.00}{\volt} battery.
    What is the equivalent resistance of this circuit?
    \begin{multicols}{2}
    \begin{choices}
        \wrongchoice{\SI{0.200}{\ohm}}
        \wrongchoice{\SI{1.95}{\ohm}}
      \correctchoice{\SI{4.62}{\ohm}}
        \wrongchoice{\SI{45.0}{\ohm}}
    \end{choices}
    \end{multicols}
\end{question}
}

\element{nysed}{
\begin{question}{Jan2008-Q22}
    In the circuit diagram below, two \SI{4.0}{\ohm} resistors are connected to a \SI{16}{\volt} battery as shown.
    \begin{center}
    \ctikzset{bipoles/length=0.75cm}
    \begin{circuitikz}[xscale=1.25]
        \draw (0,0) to [battery,l=$\SI{16}{\volt}$] (0,2)
                    to [R,l_=$\SI{4.0}{\ohm}$] (2,2)
                    to (2,0)
                    to [R,l_=$\SI{4.0}{\ohm}$] (0,0);
    \end{circuitikz}
    \end{center}
    The rate at which electrical energy is expended in this circuit is:
    \begin{multicols}{2}
    \begin{choices}
        \wrongchoice{\SI{8.0}{\watt}}
        \wrongchoice{\SI{16}{\watt}}
      \correctchoice{\SI{32}{\watt}}
        \wrongchoice{\SI{64}{\watt}}
    \end{choices}
    \end{multicols}
\end{question}
}



%% Section June2007
%%--------------------
\element{nysed}{
\begin{question}{June2007-Q18}
    A \SI{6.0}{\ohm} lamp requires \SI{0.25}{\ampere} of current to operate.
    In which circuit below would the lamp operate correctly when switch $S$ is closed?
    \begin{choices}
        \AMCboxDimensions{down=-0.75cm}
        \ctikzset{bipoles/length=0.75cm}
        \correctchoice{
            \begin{circuitikz}
                \draw[draw=white] (-0.5,0) rectangle (5.0,2);
                \draw (0,0) to [battery,l_=$\SI{1.5}{\volt}$] (0,2)
                            to (4,2)
                            to [ospst,mirror,l=$S$] (4,0)
                            to [R,l_=$\SI{6.0}{\ohm}$] (0,0);
            \end{circuitikz}
        }
        \wrongchoice{
            \begin{circuitikz}
                \draw[draw=white] (-0.5,0) rectangle (5.0,2);
                \draw (0,0) to (0,2)
                            to (2,2)
                            to [ospst,l=$S$] (2,1);
                \draw (0,0) to (2,0)
                            to [battery,l=$\SI{1.5}{\volt}$] (2,1);
                \draw (2,2) to (4,2)
                            to [R,l^=$\SI{6.0}{\ohm}$] (4,0)
                            to (2,0);
            \end{circuitikz}
        }
        \wrongchoice{
            \begin{circuitikz}
                \draw[draw=white] (-0.5,0) rectangle (5.0,2);
                \draw (0,0) to [ospst,mirror,l=$S$] (0,2)
                            to (2,2)
                            to [battery,l=$\SI{1.5}{\volt}$] (2,0)
                            to (0,0);
                \draw (2,2) to (4,2)
                            to [R,l=$\SI{6.0}{\ohm}$] (4,0)
                            to (2,0);
            \end{circuitikz}
        }
        \wrongchoice{
            \begin{circuitikz}
                \draw[draw=white] (-0.5,0) rectangle (5.0,2);
                \draw (0,0) to [battery,l_=$\SI{1.5}{\volt}$] (0,2)
                            to (4,2)
                            to [ospst,l=$S$] (4,0)
                            to (0,0);
                \draw (2,2) to [R,l=$\SI{6.0}{\ohm}$] (2,0);
            \end{circuitikz}
        }
    \end{choices}
\end{question}
}

\element{nysed}{
\begin{question}{June2007-Q19}
    What is the total current in a circuit consisting of six operating \SI{100}{\watt} lamps connected in parallel to a \SI{120}{\volt} source?
    \begin{multicols}{2}
    \begin{choices}
      \correctchoice{\SI{5}{\ampere}}
        \wrongchoice{\SI{600}{\ampere}}
        \wrongchoice{\SI{12000}{\ampere}}
        \wrongchoice{\SI{20}{\ampere}}
    \end{choices}
    \end{multicols}
\end{question}
}



%% Section Jan2007
%%--------------------
\element{nysed}{
\begin{question}{Jan2007-Q17}
    The diagram below represents a simple circuit consisting of a variable resistor, a battery, an ammeter, and a voltmeter.
    \begin{center}
    \ctikzset{bipoles/length=0.75cm}
    \begin{circuitikz}[yscale=-0.80,xscale=1.5]
        \draw (0,0) to [battery] (2,0)
                    to [ammeter](2,2)
                    to [vR,l_=$R_2$] (0,2)
                    to (0,0);
        \draw (0.5,2) to (0.5,3.0)
                    to [voltmeter] (1.5,3.0)
                    to (1.5,2);
    \end{circuitikz}
    \end{center}
    What is the effect of increasing the resistance of the variable resistor from \SI{1000}{\ohm} to \SI{10000}{\ohm}?
    Assume constant temperature.
    \begin{choices}
      \correctchoice{The ammeter reading decreases.}
        \wrongchoice{The ammeter reading increases.}
        \wrongchoice{The voltmeter reading decreases.}
        \wrongchoice{The voltmeter reading increases.}
    \end{choices}
\end{question}
}

\newcommand{\myJuneZeroSevenQfortyOneTikz}{
    \ctikzset{bipoles/length=0.75cm}
    \begin{circuitikz}[xscale=1.40,font=\small]
        \draw (0,0) to [battery,l_=\SI{12}{\volt}] (0,2)
                    to (1,2)
                    to [R=\SI{6.0}{\ohm}](1,1)
                    to [ammeter](1,0)
                    to (0,0);
        \draw (1,2) to (2,2)
                    to [R,l=$\SI{12}{\ohm}$] (2,0)
                    to (1,0);
        \draw (2,2) to (3,2)
                    to [R,l=$\SI{36}{\ohm}$] (3,0)
                    to (2,0);
        \draw (3,2) to (4,2)
                    to [R,l=$\SI{18}{\ohm}$] (4,0)
                    to (3,0);
    \end{circuitikz}
    %\begin{circuitikz}[xscale=0.75,yscale=0.60]
    %    %\draw (0,0) to [battery,l=$\SI{12}{\volt}$] (0,4)
    %    \draw (-1,4) to [battery,l=$\SI{12}{\volt}$] (-1,0);
    %    \draw (-1,4)to (1,4)
    %                to [R,l=$\SI{6.0}{\ohm}$] (1,2)
    %                to [ammeter,l=$A$] (1,0)
    %                to (-1,0);
    %    \draw (1,4) to (3,4)
    %                to [R,l=$\SI{12}{\ohm}$] (3,0)
    %                to (1,0);
    %    \draw (3,4) to (5,4)
    %                to [R,l=$\SI{36}{\ohm}$] (5,0)
    %                to (3,0);
    %    \draw (5,4) to (7,4)
    %                to [R,l=$\SI{18}{\ohm}$] (7,0)
    %                to (5,0);
    %\end{circuitikz}
}

\element{nysed}{
\begin{question}{Jan2007-Q41}
    The diagram below represents an electric circuit consisting of four resistors and a \SI{12}{\volt} battery.
    \begin{center}
        \myJuneZeroSevenQfortyOneTikz
    \end{center}
    What is the current measured by ammeter $A$?
    \begin{multicols}{2}
    \begin{choices}
      \correctchoice{\SI{2.0}{\ampere}}
        \wrongchoice{\SI{72}{\ampere}}
        \wrongchoice{\SI{4.0}{\ampere}}
        \wrongchoice{\SI{0.50}{\ampere}}
    \end{choices}
    \end{multicols}
\end{question}
}

\element{nysed}{
\begin{question}{Jan2007-Q42}
    The diagram below represents an electric circuit consisting of four resistors and a \SI{12}{\volt} battery.
    \begin{center}
        \myJuneZeroSevenQfortyOneTikz
    \end{center}
    What is the equivalent resistance of this circuit?
    \begin{multicols}{2}
    \begin{choices}
      \correctchoice{\SI{3.0}{\ohm}}
        \wrongchoice{\SI{0.33}{\ohm}}
        \wrongchoice{\SI{72}{\ohm}}
        \wrongchoice{\SI{18}{\ohm}}
    \end{choices}
    \end{multicols}
\end{question}
}

\element{nysed}{
\begin{question}{Jan2007-Q43}
    The diagram below represents an electric circuit consisting of four resistors and a \SI{12}{\volt} battery.
    \begin{center}
    \ctikzset{bipoles/length=0.75cm}
    \begin{circuitikz}[xscale=0.75,yscale=0.60]
        %\draw (0,0) to [battery,l=$\SI{12}{\volt}$] (0,4)
        \draw (-1,4) to [battery,l=$\SI{12}{\volt}$] (-1,0);
        \draw (-1,4)to (1,4)
                    to [R,l=$\SI{6.0}{\ohm}$] (1,2)
                    to [ammeter,l=$A$] (1,0)
                    to (-1,0);
        \draw (1,4) to (3,4)
                    to [R,l=$\SI{12}{\ohm}$] (3,0)
                    to (1,0);
        \draw (3,4) to (5,4)
                    to [R,l=$\SI{36}{\ohm}$] (5,0)
                    to (3,0);
        \draw (5,4) to (7,4)
                    to [R,l=$\SI{18}{\ohm}$] (7,0)
                    to (5,0);
    \end{circuitikz}
    \end{center}
    How much power is dissipated in the \SI{36}{\ohm} resistor?
    \begin{multicols}{2}
    \begin{choices}
      \correctchoice{\SI{4.0}{\watt}}
        \wrongchoice{\SI{3.0}{\watt}}
        \wrongchoice{\SI{110}{\watt}}
        \wrongchoice{\SI{48}{\watt}}
    \end{choices}
    \end{multicols}
\end{question}
}


%% Section June2006
%%--------------------
\element{nysed}{
\begin{question}{June2006-Q20}
    In which circuit represented below are meters properly connected to measure the current through resistor $R_1$ and the potential difference across resistor $R_2$?
    \begin{multicols}{2}
    \begin{choices}
        \AMCboxDimensions{down=-1.5cm}
        \ctikzset{bipoles/length=0.75cm}
        \wrongchoice{
            \begin{circuitikz}
                \draw (0,0) to [R,l=$R_1$] (1,0) to [ammeter] (2,0) to [R,l=$R_2$] (3,0)
                            to [voltmeter] (3,3) to [battery] (0,3) to (0,0);
            \end{circuitikz}
        }
        \wrongchoice{
            \begin{circuitikz}
                \draw (0,0) to [R,l=$R_1$] (1.5,0) to [R,l=$R_2$] (3,0)
                            to (3,2) to [battery] (0,2) to (0,0);
                \draw (0.25,0) to (0.25,-1) to [ammeter] (1.25,-1) to (1.25,0);
                \draw (1.75,0) to (1.75,-1) to [voltmeter] (2.75,-1) to (2.75,0);
            \end{circuitikz}
        }
        \wrongchoice{
            \begin{circuitikz}
                \draw (0,0) to [R,l=$R_2$] (1.5,0) to (3,0)
                            to (3,2) to [battery] (0,2) to (0,0);
                \draw (0,1.25) to (1.5,1.25) to [R,l=$R_1$] (3,1.25);
                \draw (1.75,1.25) to (1.75,0.5) to [ammeter] (2.75,0.5) to (2.74,1.25);
                \draw (0.25,0) to (0.25,-1) to [voltmeter] (1.25,-1) to (1.25,0);
            \end{circuitikz}
        }
        \correctchoice{
            \begin{circuitikz}
                \draw (0,0) to [R,l=$R_2$] (3,0)
                            to (3,2) to [battery] (0,2) to (0,0);
                \draw (0,1) to [R,l=$R_1$] (1.5,1) to [ammeter] (3,1);
                \draw (1,0) to (1,-1) to [voltmeter] (2,-1) to (2,0);
            \end{circuitikz}
        }
    \end{choices}
    \end{multicols}
\end{question}
}

\element{nysed}{
\begin{question}{June2006-Q21}
    Two identical resistors connected in series have an equivalent resistance of \SI{4}{\ohm}.
    The same two resistors, when connected in parallel,
        have an equivalent resistance of:
    \begin{multicols}{2}
    \begin{choices}
      \correctchoice{\SI{1}{\ohm}}
        \wrongchoice{\SI{2}{\ohm}}
        \wrongchoice{\SI{8}{\ohm}}
        \wrongchoice{\SI{4}{\ohm}}
    \end{choices}
    \end{multicols}
\end{question}
}

\element{nysed}{
\begin{question}{June2006-Q24}
    As the number of resistors in a parallel circuit is increased,
        what happens to the equivalent resistance of the circuit and total current in the circuit?
    \begin{choices}
      \correctchoice{Equivalent resistance decreases and total current increases.}
        \wrongchoice{Equivalent resistance decreases and total current decreases.}
        \wrongchoice{Both equivalent resistance and total current increases.}
        \wrongchoice{Both equivalent resistance and total current decreases.}
    \end{choices}
\end{question}
}



%% Section Jan2006
%%--------------------
\element{nysed}{
\begin{question}{Jan2006-Q25}
    What must be inserted between points $A$ and $B$ to establish a steady electrical current in the incomplete circuit represented in the diagram below?
    \begin{center}
    \ctikzset{bipoles/length=1.00cm}
    \begin{circuitikz}[scale=1.5]
        \draw node[circ,label=above:{$A$}] at (-.5,1) {};
        \draw node[circ,label=above:{$B$}] at (.5,1) {} ;
        \draw (-.5,1) to (-1,1)
                    to (-1,0)
                    to [R,l=$R$] (1,0)
                    to (1,1)
                    to (0.5,1);
    \end{circuitikz}
    \end{center}
    \begin{choices}
      \correctchoice{source of potential difference}
        \wrongchoice{switch}
        \wrongchoice{voltmeter}
        \wrongchoice{magnetic field source}
    \end{choices}
\end{question}
}

\element{nysed}{
\begin{question}{Jan2006-Q26}
    In a series of circuit containing two lamps,
        the battery supplies a potential difference of \SI{1.5}{\volt}.
    If the current in the circuit is \SI{0.10}{\ampere},
        at what rate does the circuit use energy?
    \begin{multicols}{2}
    \begin{choices}
      \correctchoice{\SI{0.15}{\watt}}
        \wrongchoice{\SI{0.015}{\watt}}
        \wrongchoice{\SI{1.5}{\watt}}
        \wrongchoice{\SI{15}{\watt}}
    \end{choices}
    \end{multicols}
\end{question}
}



%% Section June2005
%%--------------------
\element{nysed}{
\begin{question}{June2005-Q22}
    The diagram below represents part of an electric circuit containing three resistors.
    \begin{center}
    \ctikzset{bipoles/length=0.75cm}
    \begin{circuitikz}[scale=1.25]
        \draw (0,0) to [R,l=$\SI{4}{\ohm}$,*-*] (4,0);
        \draw (1,0) to (1,1)
                    to [R,l=$\SI{3}{\ohm}$] (3,1)
                    to (3,0);
        \draw (1,0) to (1,-1)
                    to [R,l=$\SI{12}{\ohm}$] (3,-1)
                    to (3,0);
    \end{circuitikz}
    \end{center}
    What is the equivalent resistance of this part of the circuit?
    \begin{multicols}{2}
    \begin{choices}
      \correctchoice{\SI{1.5}{\ohm}}
        \wrongchoice{\SI{0.67}{\ohm}}
        \wrongchoice{\SI{6.3}{\ohm}}
        \wrongchoice{\SI{19}{\ohm}}
    \end{choices}
    \end{multicols}
\end{question}
}

\element{nysed}{
\begin{question}{June2005-Q23}
    %In the circuit represented by the diagram below, what is the reading of voltmeter $V$?
    In the circuit diagram below,
        two resistors are connected in series to a \SI{60}{\volt} battery.
    \begin{center}
    \ctikzset{bipoles/length=0.75cm}
    \begin{circuitikz}[scale=1.25]
        \draw (0,0) to [battery,l=$\SI{60}{\volt}$] (0,2)
                    to [R,l_=$\SI{20}{\ohm}$] (2,2)
                    to [R,l=$\SI{10}{\ohm}$] (2,0)
                    to (0,0);
        \draw (0.5,2) to (0.5,2.5)
                    to [voltmeter] (1.5,2.5)
                    to (1.5,2);
    \end{circuitikz}
    \end{center}
    What is the reading of the voltmeter?
    \begin{multicols}{2}
    \begin{choices}
      \correctchoice{\SI{40}{\volt}}
        \wrongchoice{\SI{30}{\volt}}
        \wrongchoice{\SI{20}{\volt}}
        \wrongchoice{\SI{2.0}{\volt}}
    \end{choices}
    \end{multicols}
\end{question}
}


%% Section Jan2005
%%--------------------
\element{nysed}{
\begin{question}{Jan2005-Q29}
    A \SI{9.0}{\volt} battery is connected to a \SI{4.0}{\ohm} resistor and a \SI{5.0}{\ohm} resistor as shown in the diagram below.
    \begin{center}
    \ctikzset{bipoles/length=0.75cm}
    \begin{circuitikz}[scale=1.0]
        \draw (0,0) to [battery,l=$\SI{9}{\volt}$] (0,2)
                    to [R,l=$\SI{4.0}{\ohm}$] (2,2)
                    to [R,l=$\SI{5.0}{\ohm}$] (2,0)
                    to (0,0);
    \end{circuitikz}
    \end{center}
    What is the current in the \SI{5.0}{\ohm} resistor?
    \begin{multicols}{2}
    \begin{choices}
      \correctchoice{\SI{1.0}{\ampere}}
        \wrongchoice{\SI{1.8}{\ampere}}
        \wrongchoice{\SI{2.3}{\ampere}}
        \wrongchoice{\SI{4.0}{\ampere}}
    \end{choices}
    \end{multicols}
\end{question}
}

\element{nysed}{
\begin{question}{Jan2005-Q30}
    A \SI{100}{\ohm} resistor and an unknown resistor are connected in series to a \SI{10.0}{\volt} battery.
    If the potential drop across the \SI{100}{\ohm} resistor is \SI{4.00}{\volt},
        the resistance of the unknown resistor is:
    \begin{multicols}{2}
    \begin{choices}
      \correctchoice{\SI{150}{\ohm}}
        \wrongchoice{\SI{100}{\ohm}}
        \wrongchoice{\SI{200}{\ohm}}
        \wrongchoice{\SI{50.0}{\ohm}}
    \end{choices}
    \end{multicols}
\end{question}
}

\element{nysed}{
\begin{question}{Jan2005-Q42}
    In the circuit diagram shown below,
        ammeter $A_1$ reads \SI{10.}{\ampere}.
    \begin{center}
    \ctikzset{bipoles/length=0.75cm}
    \begin{circuitikz}[scale=1.25]
        \draw (0,0) to [battery,l=$\SI{12}{\volt}$] (0,2)
                    to [ammeter,i=$\SI{10}{\ampere}$,l=$A_1$] (2,2)
                    to [ammeter,l=$A_{2}$](2,1)
                    to [R,l=$\SI{20}{\ohm}$](2,0)
                    to (0,0);
        \draw (2,2) to (4,2)
                    to [ammeter,l=$A_{3}$](4,1)
                    to [R,l=$\SI{30}{\ohm}$] (4,0)
                    to (2,0);
    \end{circuitikz}
    \end{center}
    What is the reading of ammeter $A_2$?
    \begin{multicols}{2}
    \begin{choices}
      \correctchoice{\SI{6.0}{\ampere}}
        \wrongchoice{\SI{10.}{\ampere}}
        \wrongchoice{\SI{20}{\ampere}}
        \wrongchoice{\SI{4.0}{\ampere}}
    \end{choices}
    \end{multicols}
\end{question}
}


%% Section June2004
%%--------------------
\element{nysed}{
\begin{question}{June2004-Q23}
    The diagram below represents an electric circuit consisting of a \SI{12}{\volt} battery, a \SI{3.0}{\ohm} resistor,
        $R_1$, and a variable resistor, $R_2$.
    \begin{center}
    \ctikzset{bipoles/length=1.00cm}
    \begin{circuitikz}
        \draw (0,0) to [battery,l=$\SI{12}{\volt}$] (0,2)
                    to [R=$R_1$] (3,2)
                    to [vR=$R_2$] (3,0)
                    to (0,0);
        \node[anchor=north] at (1.5,1.8) {\SI{3.0}{\ohm}};
    \end{circuitikz}
    \end{center}
    At what value must the variable resistor be set to produce a current of \SI{1.0}{\ampere} through $R_1$?
    \begin{multicols}{2}
    \begin{choices}
      \correctchoice{\SI{9.0}{\ohm}}
        \wrongchoice{\SI{6.0}{\ohm}}
        \wrongchoice{\SI{3.0}{\ohm}}
        \wrongchoice{\SI{12}{\ohm}}
    \end{choices}
    \end{multicols}
\end{question}
}

\element{nysed}{
\begin{question}{June2004-Q42}
    A \SI{20}{\ohm} resistor and a \SI{30}{\ohm} resistor are connected in parallel to a \SI{12}{\volt} battery as shown.
    An ammeter is connected as shown.
    \begin{center}
    \ctikzset{bipoles/length=0.75cm}
    \begin{circuitikz}[scale=1.0]
        \draw (0,0) to [battery,l=$\SI{12}{\volt}$] (0,2)
                    to (2,2)
                    to [ammeter](2,1)
                    to [R,l=$\SI{20}{\ohm}$](2,0)
                    to (0,0);
        \draw (2,2) to (4,2)
                    to [R,l=$\SI{30}{\ohm}$] (4,0)
                    to (2,0);
    \end{circuitikz}
    \end{center}
    What is the equivalent resistance of the circuit?
    \begin{multicols}{2}
    \begin{choices}
      \correctchoice{\SI{12}{\ohm}}
        \wrongchoice{\SI{10}{\ohm}}
        \wrongchoice{\SI{25}{\ohm}}
        \wrongchoice{\SI{50}{\ohm}}
    \end{choices}
    \end{multicols}
\end{question}
}

\element{nysed}{
\begin{question}{June2004-Q43}
    A \SI{20}{\ohm} resistor and a \SI{30}{\ohm} resistor are connected in parallel to a \SI{12}{\volt} battery as shown.
    An ammeter is connected as shown.
    \begin{center}
    \ctikzset{bipoles/length=0.75cm}
    \begin{circuitikz}[scale=1.0]
        \draw (0,0) to [battery,l=$\SI{12}{\volt}$] (0,2)
                    to (2,2)
                    to [ammeter](2,1)
                    to [R,l=$\SI{20}{\ohm}$](2,0)
                    to (0,0);
        \draw (2,2) to (4,2)
                    to [R,l=$\SI{30}{\ohm}$] (4,0)
                    to (2,0);
    \end{circuitikz}
    \end{center}
    What is the current reading of the ammeter?
    \begin{multicols}{2}
    \begin{choices}
      \correctchoice{\SI{0.60}{\ampere}}
        \wrongchoice{\SI{1.0}{\ampere}}
        \wrongchoice{\SI{0.40}{\ampere}}
        \wrongchoice{\SI{0.20}{\ampere}}
    \end{choices}
    \end{multicols}
\end{question}
}

\element{nysed}{
\begin{question}{June2004-Q44}
    A \SI{20}{\ohm} resistor and a \SI{30}{\ohm} resistor are connected in parallel to a \SI{12}{\volt} battery as shown.
    An ammeter is connected as shown.
    \begin{center}
    \ctikzset{bipoles/length=0.75cm}
    \begin{circuitikz}[scale=1.0]
        \draw (0,0) to [battery,l=$\SI{12}{\volt}$] (0,2)
                    to (2,2)
                    to [ammeter](2,1)
                    to [R,l=$\SI{20}{\ohm}$](2,0)
                    to (0,0);
        \draw (2,2) to (4,2)
                    to [R,l=$\SI{30}{\ohm}$] (4,0)
                    to (2,0);
    \end{circuitikz}
    \end{center}
    What is the power of the \SI{30}{\ohm} resistor?
    \begin{multicols}{2}
    \begin{choices}
      \correctchoice{\SI{4.8}{\watt}}
        \wrongchoice{\SI{30}{\watt}}
        \wrongchoice{\SI{12}{\watt}}
        \wrongchoice{\SI{75}{\watt}}
    \end{choices}
    \end{multicols}
\end{question}
}


%% Section Jan2004
%%--------------------
\element{nysed}{
\begin{question}{Jan2004-Q15}
    %In which circuit would ammeter $A$ show the greatest current?
    In which circuit would the ammeter show the greatest current?
    \begin{choices}
        \AMCboxDimensions{down=-0.75cm}
        \ctikzset{bipoles/length=0.75cm}
        \correctchoice{
            \begin{circuitikz}[xscale=1.25]
                \draw (0,0) to [ammeter] (0,1)
                            to [battery,l=$\SI{1.5}{\volt}$] (0,2)
                            to (1,2)
                            to [R,l=$\SI{5}{\ohm}$] (1,0)
                            to (0,0);
                \draw (1,2) to (2,2)
                            to [R,l=$\SI{5}{\ohm}$] (2,0)
                            to (1,0);
                \draw (2,2) to (3,2)
                            to [R,l=$\SI{5}{\ohm}$] (3,0)
                            to (2,0);
            \end{circuitikz}
        }
        \wrongchoice{
            \begin{circuitikz}[xscale=1.25]
                \draw (0,0) to [ammeter] (0,1)
                            to [battery,l=$\SI{1.5}{\volt}$] (0,2)
                            to [R,l=$\SI{5}{\ohm}$] (2,2)
                            to [R,l=$\SI{5}{\ohm}$] (2,0)
                            to (0,0);
            \end{circuitikz}
        }
        \wrongchoice{
            \begin{circuitikz}[xscale=1.25]
                \draw (0,0) to [ammeter] (0,1)
                            to [battery,l=$\SI{1.5}{\volt}$] (0,2)
                            to (1,2)
                            to [R,l=$\SI{5}{\ohm}$] (1,0)
                            to (0,0);
                \draw (1,2) to (2,2)
                            to [R,l=$\SI{5}{\ohm}$] (2,0)
                            to (1,0);
            \end{circuitikz}
        }
        \wrongchoice{
            \begin{circuitikz}[xscale=1.25]
                \draw (0,0) to [ammeter] (0,1)
                            to [battery,l=$\SI{1.5}{\volt}$] (0,2)
                            to (2,2)
                            to [R,l=$\SI{5}{\ohm}$] (2,0)
                            to (0,0);
            \end{circuitikz}
        }
    \end{choices}
\end{question}
}

\element{nysed}{
\begin{question}{Jan2004-Q44}
    The diagram below represents a lamp, a \SI{10}{\volt} battery,
        and a length of nichrome wire connected in series.
    \begin{center}
    \ctikzset{bipoles/length=0.75cm}
    \begin{circuitikz}[scale=1.0]
        \draw (0,0) to [battery,l=$\SI{10}{\volt}$] (0,2)
                    to [R,l={Lamp}] (2,2)
                    to [european resistor,l={Nichrome}] (2,0)
                    to (0,0);
    \end{circuitikz}
    \end{center}
    As the temperature of the nichrome is decreased,
        the brightness of lamp will:
    \begin{choices}
      \correctchoice{increase}
        \wrongchoice{decrease}
        \wrongchoice{remain the same}
    \end{choices}
\end{question}
}

\newcommand{\myJanZeroFourQfortyFiveTikz}{
    \ctikzset{bipoles/length=0.75cm}
    \begin{circuitikz}%[xscale=1.40,font=\small]
        \draw (0,0) to [battery,l_=\SI{24}{\volt}] (0,2)
                    to [ammeter] (2,2)
                    to [R,l=\SI{4.0}{\ohm}] (2,1)
                    to [cspst,l=$S$] (2,0)
                    to (0,0);
        \draw (2,2) to (4,2)
                    to [R,l=$\SI{12}{\ohm}$] (4,0)
                    to (2,0);
    \end{circuitikz}
}

\element{nysed}{
\begin{question}{Jan2004-Q45}
    The diagram below shows two resistors connected in parallel to a \SI{24}{\volt} battery.
    \begin{center}
        \myJanZeroFourQfortyFiveTikz
    \end{center}
    If switch $S$ is open, the reading of ammeter $A$ is:
    %If the switch is open, the ammeter reading is
    \begin{multicols}{2}
    \begin{choices}
      \correctchoice{\SI{2.0}{\ampere}}
        \wrongchoice{\SI{0.50}{\ampere}}
        \wrongchoice{\SI{1.5}{\ampere}}
        \wrongchoice{\SI{6.0}{\ampere}}
    \end{choices}
    \end{multicols}
\end{question}
}

\element{nysed}{
\begin{question}{Jan2004-Q46}
    The diagram below shows two resistors connected in parallel to a \SI{24}{\volt} battery.
    \begin{center}
        \myJanZeroFourQfortyFiveTikz
    \end{center}
    %If the switch is closed, the equivalent resistance of the circuit is
    If switch $S$ is closed, the equivalent resistance of the circuit is:
    \begin{multicols}{2}
    \begin{choices}
      \correctchoice{\SI{3.0}{\ohm}}
        \wrongchoice{\SI{16}{\ohm}}
        \wrongchoice{\SI{8.0}{\ohm}}
        \wrongchoice{\SI{2.0}{\ohm}}
    \end{choices}
    \end{multicols}
\end{question}
}


%% Section June2003
%%--------------------
\element{nysed}{
\begin{question}{June2003-Q24}
    Two identical resistors connected in parallel have an equivalent resistance of \SI{40}{\ohm}.
    What is the resistance of each resistor?
    \begin{multicols}{2}
    \begin{choices}
      \correctchoice{\SI{20}{\ohm}}
        \wrongchoice{\SI{40}{\ohm}}
        \wrongchoice{\SI{80}{\ohm}}
        \wrongchoice{\SI{160}{\ohm}}
    \end{choices}
    \end{multicols}
\end{question}
}

\element{nysed}{
\begin{question}{June2003-Q43}
    %Which circuit diagram correctly shows the connection of ammeter $A$ and voltmeter $V$ to measure the current through and potential differences across resistor $R$?
    Which circuit diagram correctly shows the connection of an ammeter and a voltmeter to measure the current through and potential differences across resistor $R$?
    \begin{choices}
        \AMCboxDimensions{down=-1.0cm}
        \ctikzset{bipoles/length=0.75cm}
        \correctchoice{
            \begin{circuitikz}[scale=1.0]
                \draw[draw=white] (-0.5,0) rectangle (5.0,2);
                \draw (0,0) to [battery] (0,2)
                            to [ammeter,l_=$A$] (2,2)
                            to [R,l=$R$] (2,0)
                            to (0,0);
                \draw (2,2) to (4,2)
                            to [voltmeter,l=$V$] (4,0)
                            to (2,0);
            \end{circuitikz}
        }
        \wrongchoice{
            \begin{circuitikz}[scale=1.0]
                \draw[draw=white] (-0.5,0) rectangle (5.0,2);
                \draw (0,0) to [battery] (0,2)
                            to [ammeter,l_=$A$] (4,2)
                            to [voltmeter,l=$V$] (4,0)
                            to [R,l_=$R$] (0,0);
            \end{circuitikz}
        }
        \wrongchoice{
            \begin{circuitikz}[scale=1.0]
                \draw[draw=white] (-0.5,0) rectangle (5.0,2);
                \draw (0,0) to [battery] (0,2)
                            to (1.33,2)
                            to [R,l=$R$] (1.33,0)
                            to (0,0);
                \draw (1.33,2) to (2.66,2)
                            to [ammeter,l=$A$] (2.66,0)
                            to (1.33,0);
                \draw (2.66,2) to (4,2)
                            to [voltmeter,l=$V$] (4,0)
                            to (2.66,0);
            \end{circuitikz}
        }
        \wrongchoice{
            \begin{circuitikz}[scale=1.0]
                \draw[draw=white] (-0.5,0) rectangle (5.0,2);
                \draw (0,0) to [battery] (0,2)
                            to (2,2)
                            to [R,l=$R$] (2,0)
                            to (0,0);
                \draw (2,2) to [ammeter,l_=$A$] (4,2)
                            to [voltmeter,l=$V$] (4,0)
                            to (2,0);
            \end{circuitikz}
        }
    \end{choices}
\end{question}
}

\element{nysed}{
\begin{question}{June2003-Q44}
    Identical resistors ($R$) are connected across the same \SI{12}{\volt} battery.
    Which circuit uses the greatest power?
    \begin{choices}
        \AMCboxDimensions{down=-0.5em}
        \ctikzset{bipoles/length=0.75cm}
        \correctchoice{
            \begin{circuitikz}[scale=1.0]
                \draw (0,0) to [battery,l=$\SI{12}{\volt}$] (0,1)
                            to (4,1)
                            to [R,l=$R$] (4,0)
                            to (0,0);
                \draw (1,1) to [R,l=$R$] (1,0);
                \draw (2,1) to [R,l=$R$] (2,0);
                \draw (3,1) to [R,l=$R$] (3,0);
            \end{circuitikz}
        }
        \wrongchoice{
            \begin{circuitikz}[scale=1.0]
                \draw (0,0) to [battery,l=$\SI{12}{\volt}$] (0,1)
                            to (4,1)
                            to [R,l=$R$] (4,0)
                            to (0,0);
            \end{circuitikz}
        }
        \wrongchoice{
            \begin{circuitikz}[scale=1.0]
                \draw (0,0) to [battery,l=$\SI{12}{\volt}$] (0,1)
                            to [R,l_=$R$] (2,1)
                            to (4,1)
                            to [R,l=$R$] (4,0)
                            to [R,l_=$R$] (2,0)
                            to (0,0);
            \end{circuitikz}
        }
        \wrongchoice{
            \begin{circuitikz}[scale=1.0]
                \draw (0,0) to [battery,l=$\SI{12}{\volt}$] (0,1)
                            to (4,1)
                            to [R,l=$R$] (4,0)
                            to (0,0);
                \draw (2,1) to [R,l=$R$] (2,0);
            \end{circuitikz}
        }
    \end{choices}
\end{question}
}


%% Section Jan2003
%%--------------------
\element{nysed}{
\begin{question}{Jan2003-Q43}
    The diagram below shows a circuit with two resistors.
    \begin{center}
    \ctikzset{bipoles/length=0.75cm}
    \begin{circuitikz}[scale=1.2]
        \draw (0,0) to [battery,l=\SI{12}{\volt}] (2,0)
                    to [ammeter] (4,0)
                    to (4,2)
                    to [R,l_=\SI{8.0}{\ohm}](2,2)
                    to [R,l_=\SI{8.0}{\ohm}](0,2)
                    to (0,0);
    \end{circuitikz}
    \end{center}
    What is the reading of the ammeter?
    \begin{multicols}{2}
        \begin{choices}
            \wrongchoice{\SI{1.3}{\ampere}}
            \wrongchoice{\SI{1.5}{\ampere}}
            \wrongchoice{\SI{3.0}{\ampere}}
          \correctchoice{\SI{0.75}{\ampere}}
        \end{choices}
    \end{multicols}
\end{question}
}

\element{nysed}{
\begin{question}{Jan2003-Q47}
    The diagram below represents currents in a segment of an electric circuit.
    \begin{center}
    \ctikzset{bipoles/length=0.75cm}
    \begin{circuitikz}[scale=0.8]
        \draw[thick] (-3,0) to [short,i=\SI{2}{\ampere}] (0,0);
        \draw[thick] (0,0) to [short,i=\SI{1}{\ampere}] (3,0);
        \draw[thick] (120:3) to [short,i=\SI{3}{\ampere}] (120:1) to (0,0);
        \draw[thick] (0,0) to [short,i=\SI{4}{\ampere}] (60:2) to (60:3);
        \draw[thick] (0,0) to [short,i=\SI{2}{\ampere}] (-60:3);
        \draw[thick] (240:3) to [ammeter] (0,0);
    \end{circuitikz}
    \end{center}
    %% removed $A$
    What is the reading of the ammeter?
    \begin{multicols}{2}
    \begin{choices}
      \correctchoice{\SI{1.0}{\ampere}}
        \wrongchoice{\SI{2.0}{\ampere}}
        \wrongchoice{\SI{3.0}{\ampere}}
        \wrongchoice{\SI{4.0}{\ampere}}
    \end{choices}
    \end{multicols}
\end{question}
}


%% Section Aug2002
%%--------------------
\element{nysed}{
\begin{question}{Aug2002-Q13}
    A \SI{10}{\ohm} resistor and a \SI{20}{\ohm} resistor are connected in series to a voltage source.
    When the current through the \SI{10}{\ohm} resistor is \SI{2.0}{\ampere},
        what is the current through the \SI{20}{\ohm} resistor?
    \begin{multicols}{2}
    \begin{choices}
        \wrongchoice{\SI{1.0}{\ampere}}
      \correctchoice{\SI{2.0}{\ampere}}
        \wrongchoice{\SI{0.50}{\ampere}}
        \wrongchoice{\SI{4.0}{\ampere}}
    \end{choices}
    \end{multicols}
\end{question}
}

\element{nysed}{
\begin{question}{Aug2002-Q14}
    In the circuit diagram below, what are the correct readings of voltmeters $V_1$ and $V_2$?
    \begin{center}
    \ctikzset{bipoles/length=0.75cm}
    \begin{circuitikz}
        \draw (0,0) to [battery,l=\SI{6}{\volt}] (0,2) to (2,2) to [R,l_=\SI{10}{\ohm}] (2,0);
        \draw (2,2) to (5,2) to [R,l_=\SI{5}{\ohm}] (5,0) to (0,0);
        \draw (2,0.25) to (3,0.25) to [voltmeter,l_=$V_1$] (3,1.75) to (2,1.75);
        \draw (5,0.25) to (6,0.25) to [voltmeter,l_=$V_2$] (6,1.75) to (5,1.75);
    \end{circuitikz}
    \end{center}
    \begin{choices}
        \wrongchoice{$V_1$ reads \SI{2.0}{\volt} and $V_2$ reads \SI{4.0}{\volt}}
        \wrongchoice{$V_1$ reads \SI{4.0}{\volt} and $V_2$ reads \SI{2.0}{\volt}}
        \wrongchoice{$V_1$ reads \SI{3.0}{\volt} and $V_2$ reads \SI{3.0}{\volt}}
      \correctchoice{$V_1$ reads \SI{6.0}{\volt} and $V_2$ reads \SI{6.0}{\volt}}
    \end{choices}
\end{question}
}

\element{nysed}{
\begin{question}{Aug2002-Q34}
    What is the total resistance of the circuit segment shown in the diagram below?
    \begin{center}
    \ctikzset{bipoles/length=0.75cm}
    \begin{circuitikz}[yscale=0.50]
        \draw (0,0) to [R,l=$\SI{3.0}{\ohm}$,*-*] (4,0);
        \draw (1,0) to (1,2) to [R,l=$\SI{3.0}{\ohm}$] (3,2) to (3,0);
        \draw (1,0) to (1,-2) to [R,l=$\SI{3.0}{\ohm}$] (3,-2) to (3,0);
    \end{circuitikz}
    \end{center}
    \begin{multicols}{2}
    \begin{choices}
      \correctchoice{\SI{1.0}{\ohm}}
        \wrongchoice{\SI{9.0}{\ohm}}
        \wrongchoice{\SI{3.0}{\ohm}}
        \wrongchoice{\SI{27}{\ohm}}
    \end{choices}
    \end{multicols}
\end{question}
}

\element{nysed}{
\begin{question}{Aug2002-Q41}
    %% removed $V$ and $A$
    Which circuit diagram shows a voltmeter and ammeter correctly positioned to measure the total potential difference of the circuit and the current through each resistor?
    \begin{multicols}{2}
    \begin{choices}
        \AMCboxDimensions{down=-1.5em}
        \ctikzset{bipoles/length=0.75cm}
        \correctchoice{
            \hspace{-1ex}
            \begin{circuitikz}
                \draw[white] (1,-0.5) -- (1,2.5);
                \draw (0,0) to [battery] (0,2)
                            to [R] (1.5,2) to [R] (3,2)
                            to (3,1) to [ammeter] (3,0) to (0,0);
                \draw (0.25,2) to (0.25,1.25) to [voltmeter] (2.75,1.25) to (2.75,2);
            \end{circuitikz}
        }
        \wrongchoice{
            \hspace{-1ex}
            \begin{circuitikz}
                \draw[white] (1,-0.5) -- (1,2.5);
                \draw (0,0) to [battery] (0,2)
                            to [R] (1.5,2) to [R] (3,2)
                            to [voltmeter] (3,0) to (0,0);
                \draw (0.25,2) to (0.25,1.25) to [ammeter] (1.5,1.25) to (1.5,2);
            \end{circuitikz}
        }
        \wrongchoice{
            \hspace{-1ex}
            \begin{circuitikz}
                \draw[white] (1,-0.5) -- (1,2.5);
                \draw (0,0) to [battery] (0,2)
                            to [voltmeter] (1,2) to [R] (1,0);
                \draw (1,2) to (2,2) to [R] (2,0);
                \draw (2,2) to (3,2) to [ammeter] (3,0) to (0,0);
            \end{circuitikz}
        }
        \wrongchoice{
            \hspace{-1ex}
            \begin{circuitikz}
                \draw[white] (1,-0.5) -- (1,2.5);
                \draw (0,0) to [battery] (0,2)
                            to [ammeter] (1.5,2) to [R] (1.5,0);
                \draw (1.5,2) to [voltmeter] (3,2) to [R] (3,0) to (0,0);
            \end{circuitikz}
        }
    \end{choices}
    \end{multicols}
\end{question}
}


%% Section June2002
%%--------------------
\element{nysed}{
\begin{question}{June2002-Q23}
    A \SI{30}{\ohm} resistor and a \SI{60}{\ohm} resistor are connected in an electric circuit as shown below.
    \begin{center}
    \ctikzset{bipoles/length=0.75cm}
    \begin{circuitikz}[scale=0.8]
        \draw (2,0) to [battery,l_=$\SI{12}{\volt}$] (-2,0)
                    to (-2,2)
                    to [R,l=$\SI{30}{\ohm}$] (0,2)
                    to [R,l=$\SI{60}{\ohm}$] (2,2)
                    to (2,0);
    \end{circuitikz}
    \end{center}
    Compared to the electric current through the \SI{30}{\ohm} resistor,
        the electric current through the \SI{60}{\ohm} resistor is:
    \begin{choices}
        \wrongchoice{smaller}
        \wrongchoice{larger}
      \correctchoice{the same}
    \end{choices}
\end{question}
}

\element{nysed}{
\begin{question}{June2002-Q38}
    In the diagram below, lamps $L_1$ and $L_2$ are connected to a constant voltage power supply.
    \begin{center}
    \ctikzset{bipoles/length=0.75cm}
    \begin{circuitikz}[scale=1.0]
            \draw (0,0) to [battery] (0,2)
                        to (4,2)
                        to [R,l=$L_2$] (4,0)
                        to (0,0);
            \draw (2,2) to [R,l=$L_2$] (2,0);
    %\begin{circuitikz}[scale=1.0]
    %    \draw (2,0) to [battery,l=$\SI{12}{\volt}$] (-2,0)
    %                to (-2,1)
    %                to [R,l=$\SI{60}{\ohm}$] (0,1)
    %                to [R,l=$\SI{30}{\ohm}$](2,1)
    %                to (2,0);
    \end{circuitikz}
    \end{center}
    If lamp $L_1$ burns out, the brightness of $L_2$ will
    \begin{choices}
        \wrongchoice{decrease}
        \wrongchoice{increase}
      \correctchoice{remain the same}
    \end{choices}
\end{question}
}


%% Section Jan2002
%%--------------------
\element{nysed}{
\begin{question}{Jan2002-Q34}
    Which diagram shows correct current direction in a segment of an electric circuit?
    \begin{multicols}{2}
    \begin{choices}
        \AMCboxDimensions{down=-1cm}
        \ctikzset{bipoles/length=0.75cm}
        \wrongchoice{
            \begin{circuitikz}[yscale=0.8]
                \draw[white] (0,-1.5) -- (0,1.5);
                \draw (-1.5,0) to [short,i=\SI{2}{\ampere}] (0,0)
                               to (0,1) to [short,i=\SI{5}{\ampere}] (1.5,1);
                \draw (0,1) to (0,-1) to [short,i=\SI{3}{\ampere}] (1.5,-1);
            \end{circuitikz}
        }
        \correctchoice{
            \begin{circuitikz}[yscale=0.8]
                \draw[white] (0,-1.5) -- (0,1.5);
                \draw (-1.5,0) to [short,i=\SI{2}{\ampere}] (0,0)
                               to (0,1) to [short,i=\SI{5}{\ampere}] (1.5,1);
                \draw (1.5,-1) to [short,i=\SI{3}{\ampere}] (0,-1) to (0,0);
            \end{circuitikz}
        }
        \wrongchoice{
            \begin{circuitikz}[yscale=0.8]
                \draw[white] (0,-1.5) -- (0,1.5);
                \draw (-1.5,0) to [short,i=\SI{2}{\ampere}] (0,0);
                \draw (+1.5,+1) to [short,i=\SI{5}{\ampere}] (0,+1) to (0,0);
                \draw (+1.5,-1) to [short,i=\SI{3}{\ampere}] (0,-1) to (0,0);
            \end{circuitikz}
        }
        \wrongchoice{
            \begin{circuitikz}[yscale=0.8]
                \draw[white] (0,-1.5) -- (0,1.5);
                \draw (-1.5,0) to [short,i=\SI{2}{\ampere}] (0,0);
                \draw (+1.5,+1) to [short,i=\SI{5}{\ampere}] (0,+1) to (0,0);
                \draw (0,0) to (0,-1) to [short,i=\SI{3}{\ampere}] (1.5,-1);
            \end{circuitikz}
        }
    \end{choices}
    \end{multicols}
\end{question}
}

\element{nysed}{
\begin{question}{Jan2002-Q36}
    In the circuit shown below, voltmeter $V_2$ reads \SI{80}{\volt}.
    \begin{center}
    \ctikzset{bipoles/length=0.75cm}
    \begin{circuitikz}
        \draw (0,0) to [battery] (4,0) to (4,2) to [R,l=\SI{20}{\ohm}] (2,2) to [R,l=\SI{40}{\ohm}] (0,2) to (0,0);
        \draw (0.25,2) to (0.25,3) to [voltmeter,l=$V_1$] (1.75,3) to (1.75,2);
        \draw (2.25,2) to (2.25,3) to [voltmeter,l=$V_2$] (3.75,3) to (3.75,2);
    \end{circuitikz}
    \end{center}
    What is the reading of voltmeter $V_1$?
    \begin{multicols}{2}
    \begin{choices}
      \correctchoice{\SI{160}{\volt}}
        \wrongchoice{\SI{80}{\volt}}
        \wrongchoice{\SI{40}{\volt}}
        \wrongchoice{\SI{20}{\volt}}
    \end{choices}
    \end{multicols}
\end{question}
}

\element{nysed}{
\begin{question}{Jan2002-Q37}
    A physics student is given three \SI{12}{\ohm} resistors with instructions to create the circuit that would have the lowest possible resistance.
    The correct circuit would be a:
    \begin{choices}
        \wrongchoice{series circuit with an equivalent resistance of \SI{36}{\ohm}}
        \wrongchoice{series circuit with an equivalent resistance of \SI{4}{\ohm}}
        \wrongchoice{parallel circuit with an equivalent resistance of \SI{36}{\ohm}}
      \correctchoice{parallel circuit with an equivalent resistance of \SI{4}{\ohm}}
    \end{choices}
\end{question}
}

\element{nysed}{
\begin{question}{Jan2002-Q38}
    Two resistors are connected to a source of voltage as shown in the diagram below.
    \begin{center}
    \ctikzset{bipoles/length=0.75cm}
    \begin{circuitikz}[scale=1.0]
        \draw (0,0) to [battery,l=$V$] (0,2)
                    to [R,l=$R_1$] (2,2)
                    to [ammeter,l=$A_3$] (2,0)
                    to [ammeter,l_=$A_4$] (0,0);
        \draw (2,2) to [ammeter,l=$A_2$] (4,2)
                    to [R,l=$R_2$] (4,0)
                    to (2,0);
        \draw (4,2) to (6,2)
                    to [ammeter,l=$A_1$] (6,0)
                    to (4,0);
    \end{circuitikz}
    \end{center}
    At which position should an ammeter be placed to measure the current passing only through resistor $R_1$?
    \begin{multicols}{2}
    \begin{choices}
        \wrongchoice{$A_1$}
        \wrongchoice{$A_2$}
      \correctchoice{$A_3$}
        \wrongchoice{$A_4$}
    \end{choices}
    \end{multicols}
\end{question}
}


%% Section June2001
%%--------------------
\element{nysed}{
\begin{question}{June2001-Q29}
    Which diagram below correctly shows currents traveling near junction $P$ in an electric circuit?
    \begin{multicols}{2}
    \begin{choices}
        \AMCboxDimensions{down=-1.2cm}
        \ctikzset{bipoles/length=0.75cm}
        \wrongchoice{
            \begin{circuitikz}[yscale=0.8]
                \draw[white] (0,-1.5) -- (0,1.5);
                \draw[fill] (0,0) circle (1.5pt) node[anchor=west] {$P$};
                \draw (-2,0) to [short,i=\SI{3}{\ampere}] (0,0);
                \draw (0,0) to [short,i=\SI{4}{\ampere}] (0,2);
                \draw (0,-2) to [short,i=\SI{7}{\ampere}] (0,0);
            \end{circuitikz}
        }
        \wrongchoice{
            \begin{circuitikz}[yscale=0.8]
                \draw[white] (0,-1.5) -- (0,1.5);
                \draw[fill] (0,0) circle (1.5pt) node[anchor=west] {$P$};
                \draw (-2,0) to [short,i=\SI{3}{\ampere}] (0,0);
                \draw (0,2) to [short,i=\SI{4}{\ampere}] (0,0);
                \draw (0,-2) to [short,i=\SI{7}{\ampere}] (0,0);
            \end{circuitikz}
        }
        \wrongchoice{
            \begin{circuitikz}[yscale=0.8]
                \draw[white] (0,-1.5) -- (0,1.5);
                \draw[fill] (0,0) circle (1.5pt) node[anchor=west] {$P$};
                \draw (0,0) to [short,i=\SI{3}{\ampere}] (-2,0);
                \draw (0,2) to [short,i=\SI{4}{\ampere}] (0,0);
                \draw (0,-2) to [short,i=\SI{7}{\ampere}] (0,0);
            \end{circuitikz}
        }
        \correctchoice{
            \begin{circuitikz}[yscale=0.8]
                \draw[white] (0,-1.5) -- (0,1.5);
                \draw[fill] (0,0) circle (1.5pt) node[anchor=west] {$P$};
                \draw (0,0) to [short,i=\SI{3}{\ampere}] (-2,0);
                \draw (0,0) to [short,i=\SI{4}{\ampere}] (0,2);
                \draw (0,-2) to [short,i=\SI{7}{\ampere}] (0,0);
            \end{circuitikz}
        }
    \end{choices}
    \end{multicols}
\end{question}
}

\element{nysed}{
\begin{question}{June2001-Q30}
    The diagram below shows three resistors, $R_1$, $R_2$,
        and $R_3$, connected to a \SI{12}{\volt} battery.
    \begin{center}
    \ctikzset{bipoles/length=0.75cm}
    \begin{circuitikz}[xscale=1.5]
        \draw (0,0) to [battery,l=\SI{12}{\volt}] (0,2)
                    to [R,l_=$R_1$](2,2)
                    to [R,l_=$R_2$](2,0)
                    to [R,l=$R_3$](0,0);
        \draw (0.5,2) to (0.5,2.5) to [voltmeter,l=$V_1$] (1.5,2.5) to (1.5,2);
        \draw (2,1.5) to (2.5,1.5) to [voltmeter,l=$V_2$] (2.5,0.5) to (2,0.5);
    \end{circuitikz}
    \end{center}
    If voltmeter $V_1$ reads \SI{3}{\volt} and voltmeter $V_2$ reads \SI{4}{\volt},
        what is the potential drop across resistor $R_3$?
    \begin{multicols}{2}
    \begin{choices}
        \wrongchoice{\SI{12}{\volt}}
      \correctchoice{\SI{5}{\volt}}
        \wrongchoice{\SI{0}{\volt}}
        \wrongchoice{\SI{4}{\volt}}
    \end{choices}
    \end{multicols}
\end{question}
}

\newcommand{\myJuneZeroOneQthirtyTwoTikz}{
    \ctikzset{bipoles/length=0.75cm}
    \begin{circuitikz}[yscale=0.66]
        \draw (2,0)to [battery,l_=$\SI{6.0}{\volt}$] (-2,0)
                    to (-2,2)
                    to [R,l=$\SI{3.0}{\ohm}$] (2,2)
                    to (2,0);
        \draw (-2,2)to [ammeter,l=$A$] (-2,4)
                    to [R,l=$\SI{1.0}{\ohm}$] (2,4)
                    to (2,2);
    \end{circuitikz}
}

\element{nysed}{
\begin{question}{June2001-Q32}
    The diagram below shows two resistors connected in parallel across a \SI{6}{\volt} source.
    \begin{center}
        \myJuneZeroOneQthirtyTwoTikz
    \end{center}
    The equivalent resistance of the two resistors is:
    \begin{multicols}{2}
    \begin{choices}
      \correctchoice{\SI{0.75}{\ohm}}
        \wrongchoice{\SI{2.0}{\ohm}}
        \wrongchoice{\SI{1.3}{\ohm}}
        \wrongchoice{\SI{4.0}{\ohm}}
    \end{choices}
    \end{multicols}
\end{question}
}

\element{nysed}{
\begin{question}{June2001-Q33}
    The diagram below shows two resistors connected in parallel across a \SI{6}{\volt} source.
    \begin{center}
        \myJuneZeroOneQthirtyTwoTikz
    \end{center}
    Compared to the power dissipated in the \SI{1.0}{\ohm} resistor,
        the power dissipated in the \SI{3.0}{\ohm} resistor is:
    \begin{multicols}{2}
    \begin{choices}
      \correctchoice{less}
        \wrongchoice{greater}
        \wrongchoice{the same}
    \end{choices}
    \end{multicols}
\end{question}
}


%% Section Jan2001
%%--------------------
\element{nysed}{
\begin{question}{Jan2001-Q27}
    If a \SI{15}{\ohm} resistor is connected in parallel with a \SI{30}{\ohm} resistor,
        the equivalent resistance is:
    \begin{multicols}{2}
    \begin{choices}
        \wrongchoice{\SI{15}{\ohm}}
        \wrongchoice{\SI{2.0}{\ohm}}
      \correctchoice{\SI{10}{\ohm}}
        \wrongchoice{\SI{45}{\ohm}}
    \end{choices}
    \end{multicols}
\end{question}
}

\element{nysed}{
\begin{question}{Jan2001-Q30}
    The diagram below shows electric currents in conductors that meet at junction $P$.
    \begin{center}
    \begin{circuitikz}
        \draw[fill] (0,0) circle (2pt) node[anchor=north west] {$P$};
        \draw[fill] (2,0) circle (2pt) node[anchor=west] {$Q$};
        \draw (0,0) -- (2,0);
        \draw (-2,0) to [short,i=\SI{3}{\ampere}] (0,0);
        \draw (0,0) to [short,i=\SI{2}{\ampere}] (0,2);
        \draw (0,-2) to [short,i=\SI{4}{\ampere}] (0,0);
    \end{circuitikz}
    \end{center}
    What are the magnitude and direction of the current in conductor $PQ$?
    \begin{multicols}{2}
    \begin{choices}
        \wrongchoice{\SI{9}{\ampere} toward $P$}
        \wrongchoice{\SI{9}{\ampere} toward $Q$}
        \wrongchoice{\SI{5}{\ampere} toward $P$}
      \correctchoice{\SI{5}{\ampere} toward $Q$}
    \end{choices}
    \end{multicols}
\end{question}
}

%% completely reworded these questions
\newcommand{\myJanZeroOneQthirtyTwoTikz}{
    \ctikzset{bipoles/length=0.75cm}
    \begin{circuitikz}[xscale=1.40,font=\small]
        \draw (0,0) to [battery] (0,2)
                    to [ammeter,i^>=\SI{10}{\ampere}] (2,2)
                    to [R,l=\SI{6.0}{\ohm}](2,1)
                    to [ammeter,l=$A_1$] (2,0)
                    to (0,0);
        \draw (2,2) to (4,2)
                    to [R,l=$\SI{30}{\ohm}$] (4,1)
                    to [ammeter,i^>=\SI{4}{\ampere}] (4,0)
                    to (2,0);
    \end{circuitikz}
}

\element{nysed}{
\begin{question}{Jan2001-Q32}
    The diagram below shows two resistors and three ammeters connected to a voltage source.
    \begin{center}
        \myJanZeroOneQthirtyTwoTikz
    \end{center}
    What is the potential difference across the source?
    \begin{multicols}{2}
    \begin{choices}
        \wrongchoice{\SI{440}{\volt}}
        \wrongchoice{\SI{240}{\volt}}
      \correctchoice{\SI{120}{\volt}}
        \wrongchoice{\SI{60}{\volt}}
    \end{choices}
    \end{multicols}
\end{question}
}

\element{nysed}{
\begin{question}{Jan2001-Q33}
    The diagram below shows two resistors and three ammeters connected to a voltage source.
    \begin{center}
        \myJanZeroOneQthirtyTwoTikz
    \end{center}
    What is the current reading of ammeter $A_1$?
    \begin{multicols}{2}
    \begin{choices}
        \wrongchoice{\SI{10}{\ampere}}
      \correctchoice{\SI{6.0}{\ampere}}
        \wrongchoice{\SI{3.0}{\ampere}}
        \wrongchoice{\SI{4.0}{\ampere}}
    \end{choices}
    \end{multicols}
\end{question}
}

\element{nysed}{
\begin{question}{Jan2001-Q80}
    In order to measure the current through an electrical device,
        an ammeter is placed in series with the device.
    Compared to the electrical device,
        the ammeter should have a much:
    \begin{choices}
        \wrongchoice{lower permeability}
        \wrongchoice{higher permeability}
      \correctchoice{lower resistance}
        \wrongchoice{higher resistance}
    \end{choices}
\end{question}
}


%% Section June2000
%%--------------------
\element{nysed}{
\begin{question}{June2000-Q35}
    The diagram below shows two resistors connected in series to a \SI{20}{\volt} battery.
    \begin{center}
    \ctikzset{bipoles/length=0.75cm}
    \begin{circuitikz}
        \draw (0,0) to [battery,l=\SI{20}{\volt}] (0,2)
                    to [R,l=\SI{5.0}{\ohm}] (2,2)
                    to [R,l=\SI{15.0}{\ohm}] (2,0)
                    to (0,0);
    \end{circuitikz}
    \end{center}
    If the current through the \SI{5.0}{\ohm} resistor is \SI{1.0}{\ampere},
        then current through the \SI{15}{\ohm} resistor is:
    \begin{multicols}{2}
    \begin{choices}
      \correctchoice{\SI{1.0}{\ampere}}
        \wrongchoice{\SI{0.33}{\ampere}}
        \wrongchoice{\SI{3.0}{\ampere}}
        \wrongchoice{\SI{1.3}{\ampere}}
    \end{choices}
    \end{multicols}
\end{question}
}

\element{nysed}{
\begin{question}{June2000-Q36}
    Resistors $R_1$ and $R_2$ have an equivalent resistance of \SI{6}{\ohm} when connected in the circuit shown below.
    \begin{center}
    \ctikzset{bipoles/length=0.75cm}
    \begin{circuitikz}
        \draw (0,0) to [battery] (0,2)
                    to (2,2)
                    to [R,l=$R_1$] (2,0)
                    to (0,0);
        \draw (2,2) to (4,2)
                    to [R,l=$R_2$] (4,0)
                    to (2,0);
    \end{circuitikz}
    \end{center}
    The resistance of $R_1$ could be:
    \begin{multicols}{2}
    \begin{choices}
        \wrongchoice{\SI{1}{\ohm}}
        \wrongchoice{\SI{5}{\ohm}}
      \correctchoice{\SI{8}{\ohm}}
        \wrongchoice{\SI{4}{\ohm}}
    \end{choices}
    \end{multicols}
\end{question}
}

\element{nysed}{
\begin{question}{June2000-Q37}
    The diagram below represents an electric circuit.
    \begin{center}
    \ctikzset{bipoles/length=0.75cm}
    \begin{circuitikz}
        \draw (0,0) to [battery] (0,2)
                    to (2,2)
                    to [R,l_=\SI{8}{\ohm}] (2,0)
                    to (0,0);
        \draw (2,1.5) to (3,1.5)
                    to [voltmeter,l=\SI{4}{\volt}] (3,0.5)
                    to (2,0.5);
    \end{circuitikz}
    \end{center}
    The total amount of energy delivered to the resistor in \SI{10}{\second} is:
    \begin{multicols}{2}
    \begin{choices}
        \wrongchoice{\SI{3.2}{\joule}}
        \wrongchoice{\SI{5.0}{\joule}}
      \correctchoice{\SI{20}{\joule}}
        \wrongchoice{\SI{320}{\joule}}
    \end{choices}
    \end{multicols}
\end{question}
}

\element{nysed}{
\begin{question}{June2000-Q39}
    A copper wire is part of a complete circuit through which current flows.
    Which graph best represents the relationship between the wire's length and its resistance?
    \begin{multicols}{2}
    \begin{choices}
        \AMCboxDimensions{down=-2.5em}
        \correctchoice{
            \begin{tikzpicture}
                \begin{axis}[
                    axis y line=left,
                    axis x line=bottom,
                    axis line style={->},
                    ylabel={length},
                    ytick=\empty,
                    xlabel={resistance},
                    xtick=\empty,
                    xmin=0,xmax=11,
                    ymin=0,ymax=11,
                    width=\columnwidth,
                    very thin,
                ]
                \addplot[line width=1pt,domain=0:10]{x};
                \end{axis}
            \end{tikzpicture}
        }
        \wrongchoice{
            \begin{tikzpicture}
                \begin{axis}[
                    axis y line=left,
                    axis x line=bottom,
                    axis line style={->},
                    ylabel={length},
                    ytick=\empty,
                    xlabel={resistance},
                    xtick=\empty,
                    xmin=0,xmax=11,
                    ymin=0,ymax=11,
                    width=\columnwidth,
                    very thin,
                ]
                \addplot[line width=1pt,domain=0:10]{5};
                \end{axis}
            \end{tikzpicture}
        }
        \wrongchoice{
            \begin{tikzpicture}
                \begin{axis}[
                    axis y line=left,
                    axis x line=bottom,
                    axis line style={->},
                    ylabel={length},
                    ytick=\empty,
                    xlabel={resistance},
                    xtick=\empty,
                    xmin=0,xmax=11,
                    ymin=0,ymax=11,
                    width=\columnwidth,
                    very thin,
                ]
                \addplot[line width=1pt,domain=0:10]{10-x};
                \end{axis}
            \end{tikzpicture}
        }
        \wrongchoice{
            \begin{tikzpicture}
                \begin{axis}[
                    axis y line=left,
                    axis x line=bottom,
                    axis line style={->},
                    ylabel={length},
                    ytick=\empty,
                    xlabel={resistance},
                    xtick=\empty,
                    xmin=0,xmax=11,
                    ymin=0,ymax=11,
                    width=\columnwidth,
                    very thin,
                ]
                \addplot[line width=1pt,domain=0:10]{10/x};
                \end{axis}
            \end{tikzpicture}
        }
    \end{choices}
    \end{multicols}
\end{question}
}


%% Section June1999
%%--------------------
\element{nysed}{
\begin{question}{June1999-Q33}
    In the diagram below of a parallel circuit,
        ammeter $A$ measures the current supplied by the \SI{110}{\volt} source.
    \begin{center}
    \ctikzset{bipoles/length=0.75cm}
    \begin{circuitikz}[xscale=1.75]
        \draw (0,0) to [battery,l=\SI{110}{\volt}] (0,2)
                    to (3,2) 
                    to (3,1) to [R,l_=\SI{60}{\ohm}] (3,0)
                    to (1,0) to [ammeter,l_=$A$] (0,0);
        \draw (2,2) to [R,l_=\SI{30}{\ohm}] (2,0);
        \draw (1,2) to [R,l_=\SI{20}{\ohm}] (1,1) to (1,0);
    \end{circuitikz}
    \end{center}
    The current measured by ammeter $A$ is:
    \begin{multicols}{2}
    \begin{choices}
        \wrongchoice{\SI{1.0}{\ampere}}
        \wrongchoice{\SI{0.10}{\ampere}}
        \wrongchoice{\SI{5.5}{\ampere}}
      \correctchoice{\SI{11}{\ampere}}
    \end{choices}
    \end{multicols}
\end{question}
}

\element{nysed}{
\begin{question}{June1999-Q35}
    The diagram below represents a simple electric circuit.
    \begin{center}
    \ctikzset{bipoles/length=0.75cm}
    \begin{circuitikz}[xscale=1.75,yscale=0.9]
        \draw (0,0) to [battery,l=\SI{12}{\volt}] (0,2)
                    to [R,l=\SI{3.0}{\ohm}] (2,2)
                    to (2,0)
                    to [ammeter,i>=\SI{4.0}{\ampere}] (0,0);
    \end{circuitikz}
    \end{center}
    How much charge passes through the resistor in \SI{2.0}{\second}?
    \begin{multicols}{2}
    \begin{choices}
        \wrongchoice{\SI{6.0}{\coulomb}}
        \wrongchoice{\SI{2.0}{\coulomb}}
      \correctchoice{\SI{8.0}{\coulomb}}
        \wrongchoice{\SI{4.0}{\coulomb}}
    \end{choices}
    \end{multicols}
\end{question}
}


%% Section June1998
%%--------------------
\element{nysed}{
\begin{question}{June1998-Q28}
    The diagram below shows a circuit with three resistors.
    \begin{center}
    \ctikzset{bipoles/length=0.75cm}
    \begin{circuitikz}[xscale=1.50]
        \draw (0,0) to [battery,l=\SI{24}{\volt}] (1,0)
                    to [ammeter,i>=\SI{2.0}{\ampere}] (3,0)
                    to (3,2)
                    to [R,l=$R$] (2,2)
                    to [R,l=\SI{6.0}{\ohm}] (1,2)
                    to [R,l=\SI{4.0}{\ohm}] (0,2)
                    to (0,0);
    \end{circuitikz}
    \end{center}
    What is the resistance of resistor $R$?
    \begin{multicols}{2}
    \begin{choices}
        \wrongchoice{\SI{6.0}{\ohm}}
      \correctchoice{\SI{2.0}{\ohm}}
        \wrongchoice{\SI{12}{\ohm}}
        \wrongchoice{\SI{4.0}{\ohm}}
    \end{choices}
    \end{multicols}
\end{question}
}

\element{nysed}{
\begin{question}{June1998-Q29}
    Three ammeters are placed in a circuit as shown below.
    \begin{center}
    \ctikzset{bipoles/length=0.75cm}
    \begin{circuitikz}[xscale=1.33]
        \draw (0,0) to [battery] (0,2)
                    to [ammeter,l_=$A_1$] (2,2)
                    to [ammeter,l=$A_2$] (2,1)
                    to [R] (2,0)
                    to (0,0);
        \draw (2,2) to (4,2)
                    to [ammeter,l=$A_3$] (4,1)
                    to [R] (4,0)
                    to (2,0);
    \end{circuitikz}
    \end{center}
    If $A_1$ reads \SI{5}{\ampere} and $A_2$ reads \SI{2.0}{\ampere},
        what does $A_3$ read?
    \begin{multicols}{2}
    \begin{choices}
        \wrongchoice{\SI{1.0}{\ampere}}
        \wrongchoice{\SI{2.0}{\ampere}}
      \correctchoice{\SI{3.0}{\ampere}}
        \wrongchoice{\SI{7.0}{\ampere}}
    \end{choices}
    \end{multicols}
\end{question}
}

\element{nysed}{
\begin{questionmult}{June1998-Q32}
    In which pair of circuits shown below could the readings
        of the two voltmeters and one ammeter be correct?
        %of voltmeters $V_1$ and $V_2$ and ammeter $A$ be correct?
    \begin{choices}
        \ctikzset{bipoles/length=0.75cm}
        \AMCboxDimensions{down=-2.5em}
        \correctchoice{
            \begin{circuitikz}
                %% series A
                \draw (0,0) to [battery,l_=\SI{100}{\volt}] (0,2) to [ammeter,i=\SI{10}{\ampere}] (2,2) to [R] (4,2) to [R] (4,0) to (0,0);
                \draw (4,2) to (4,3) to [voltmeter,v=\SI{50}{\volt}] (2,3) to (2,2);
                \draw (4,0) to (5,0) to [voltmeter,v=\SI{50}{\volt}] (5,2) to (4,2);
            \end{circuitikz}
        }
        \wrongchoice{
            \begin{circuitikz}
                %% series B
                \draw (0,0) to [battery,l_=\SI{50}{\volt}] (0,2) to [ammeter,i=\SI{5}{\ampere}] (2,2) to [R] (4,2) to [R] (4,0) to (0,0);
                \draw (4,2) to (4,3) to [voltmeter,v=\SI{50}{\volt}] (2,3) to (2,2);
                \draw (4,0) to (5,0) to [voltmeter,v=\SI{50}{\volt}] (5,2) to (4,2);
            \end{circuitikz}
        }
        \wrongchoice{
            \begin{circuitikz}
                %% parallel C
                \draw (0,0) to [battery,l_=\SI{100}{\volt}] (0,3) to [ammeter,i=\SI{10}{\ampere}] (2,3) to [R] (2,0) to (0,0);
                \draw (2,3) to (5,3) to [R] (5,0) to (2,0);
                \draw (2,0.5) to (3,0.5) to [voltmeter,v=\SI{50}{\volt}] (3,2.5) to (2,2.5);
                \draw (5,0.5) to (6,0.5) to [voltmeter,v=\SI{50}{\volt}] (6,2.5) to (5,2.5);
            \end{circuitikz}
        }
        \correctchoice{
            \begin{circuitikz}
                %% parallel D
                \draw (0,0) to [battery,l_=\SI{50}{\volt}] (0,3) to [ammeter,i=\SI{5}{\ampere}] (2,3) to [R] (2,0) to (0,0);
                \draw (2,3) to (5,3) to [R] (5,0) to (2,0);
                \draw (2,0.5) to (3,0.5) to [voltmeter,v=\SI{50}{\volt}] (3,2.5) to (2,2.5);
                \draw (5,0.5) to (6,0.5) to [voltmeter,v=\SI{50}{\volt}] (6,2.5) to (5,2.5);
            \end{circuitikz}
        }
    \end{choices}
\end{questionmult}
}

\element{nysed}{
\begin{question}{June1998-Q53}
    When an incandescent light bulb is turned on,
        its thin wire filament heats up quickly.
    As the temperature of this wire filament increases,
        its electrical resistance:
    \begin{choices}
        \wrongchoice{decreases}
      \correctchoice{increases}
        \wrongchoice{remains the same}
    \end{choices}
\end{question}
}

\element{nysed}{
\begin{question}{June1998-Q79}
    Which device consists of a galvanometer with a low-resistance shunt placed in parallel across its terminals?
    \begin{choices}
        \wrongchoice{mass spectrometer}
        \wrongchoice{transformer}
      \correctchoice{voltmeter}
        \wrongchoice{ammeter}
    \end{choices}
\end{question}
}


%% Section June1997
%%--------------------
\element{nysed}{
\begin{question}{June1997-Q27}
    The diagram below shows currents in a segment of an electric circuit.
    \begin{center}
    \ctikzset{bipoles/length=0.75cm}
    \begin{circuitikz}[scale=0.8]
        \draw[thick] (3,0) to [short,i^>=\SI{2}{\ampere}] (0,0)
                           to [short,i^>=\SI{5}{\ampere}] (-3,0);
        \draw[thick] (0,2) to [short,i^>=\SI{6}{\ampere}] (0,0)
                           to [ammeter,l=$A$] (0,-2);
    \end{circuitikz}
    \end{center}
    What is the reading on ammeter $A$?
    \begin{multicols}{4}
    \begin{choices}
        \wrongchoice{\SI{8}{\ampere}}
        \wrongchoice{\SI{2}{\ampere}}
      \correctchoice{\SI{3}{\ampere}}
        \wrongchoice{\SI{13}{\ampere}}
    \end{choices}
    \end{multicols}
\end{question}
}


%% Section June1996
%%--------------------
\element{nysed}{
\begin{question}{June1996-Q30}
    In the circuit shown below, voltmeter $V_2$ reads \SI{80}{\volt}.
    \begin{center}
    \ctikzset{bipoles/length=0.75cm}
    \begin{circuitikz}
        \draw (0,0) to [battery] (5,0) to (5,2) to [R,l=\SI{20}{\ohm}] (3,2) to (2,2)  to [R,l=\SI{40}{\ohm}] (0,2) to (0,0);
        \draw (0,2) to (0,3) to [voltmeter,l_=$V_1$] (2,3) to (2,2);
        \draw (3,2) to (3,3) to [voltmeter,l_=$V_2$,v^=\SI{80}{\volt}] (5,3) to (5,2);
    \end{circuitikz}
    \end{center}
    What is the reading of voltmeter $V_1$?
    \begin{multicols}{2}
    \begin{choices}
      \correctchoice{\SI{160}{\volt}}
        \wrongchoice{\SI{80}{\volt}}
        \wrongchoice{\SI{40}{\volt}}
        \wrongchoice{\SI{20}{\volt}}
    \end{choices}
    \end{multicols}
\end{question}
}

\element{nysed}{
\begin{question}{June1996-Q32}
    The diagram below shows the current in a segment of a direct current circuit.
    \begin{center}
    \ctikzset{bipoles/length=0.75cm}
    \begin{circuitikz}[scale=0.8]
        \draw[thick] (-3,0) to [short,i^>=\SI{4}{\ampere}] (0,0)
                            to [short,i^>=\SI{8}{\ampere}] (3,0);
        \draw[thick] (0,2)  to [short,i^>=\SI{3}{\ampere}] (0,0)
                            to [ammeter,l=$A$] (0,-2);
    \end{circuitikz}
    \end{center}
    What is the reading of ammeter $A$?
    \begin{multicols}{4}
    \begin{choices}
        \wrongchoice{\SI{1}{\ampere}}
      \correctchoice{\SI{5}{\ampere}}
        \wrongchoice{\SI{7}{\ampere}}
        \wrongchoice{\SI{8}{\ampere}}
    \end{choices}
    \end{multicols}
\end{question}
}


%% Section June1995
%%--------------------
\element{nysed}{
\begin{question}{June1995-Q28}
    In the circuit diagram below, the ammeter measured the current supplied by the \SI{10}{\volt} battery.
    \begin{center}
    \ctikzset{bipoles/length=1.00cm}
    \begin{circuitikz}
        \draw (0,0) to [battery,l=\SI{10}{\volt}] (0,1.5) to [ammeter] (0,3) to (2,3) to (2,1.5) to [R,l=\SI{40}{\ohm}] (2,0) to (0,0);
        \draw (2,3) to (4,3) to (4,1.5) to [R,l=\SI{40}{\ohm}] (4,0) to (2,0);
    \end{circuitikz}
    \end{center}
    The current measured by the ammeter is:
    \begin{multicols}{2}
    \begin{choices}
        \wrongchoice{\SI{0.13}{\ampere}}
        \wrongchoice{\SI{2.0}{\ampere}}
      \correctchoice{\SI{0.50}{\ampere}}
        \wrongchoice{\SI{4.0}{\ampere}}
    \end{choices}
    \end{multicols}
\end{question}
}


%% Section June1994
%%--------------------
\element{nysed}{
\begin{question}{June1994-Q28}
    A series circuit has a total resistance of \SI{1.00e2}{\ohm} and an applied potential difference of \SI{2.00e2}{\volt}.
    The amount of charge passing any point in the circuit in \SI{2.00}{\second} is:
    \begin{multicols}{2}
    \begin{choices}
        \wrongchoice{\SI{1.26e19}{\coulomb}}
        \wrongchoice{\SI{2.00}{\coulomb}}
        \wrongchoice{\SI{2.52e19}{\coulomb}}
      \correctchoice{\SI{4.00}{\coulomb}}
    \end{choices}
    \end{multicols}
\end{question}
}

\element{nysed}{
\begin{questionmult}{June1994-Q30}
    Which two of the resistor arrangements shown below have equivalent resistance?
    \begin{multicols}{2}
    \begin{choices}
        \AMCboxDimensions{down=-1.2cm}
        \ctikzset{bipoles/length=0.75cm}
        \wrongchoice{
            \begin{circuitikz}
                \draw[dashed,white!60!black] (-1.5,-1.5) rectangle (1.5,1.5);
                \draw (-1.5,0) to [R,l=\SI{1}{\ohm}] (0,0) to [R,l=\SI{1}{\ohm}] (1.5,0);
            \end{circuitikz}
        }
        %% ANS is 2: B and C
        \correctchoice{
            \begin{circuitikz}
                \draw[dashed,white!60!black] (-1.5,-1.5) rectangle (1.5,1.5);
                \draw (-1.5,0) to (-1,0) to (-1,1) to [R,l_=\SI{8}{\ohm}] (1,1) to (1,0) to (1.5,0);
                \draw (-1,0) to (-1,-1) to [R,l=\SI{8}{\ohm}] (1,-1) to (1,0);
            \end{circuitikz}
        }
        \correctchoice{
            \begin{circuitikz}
                \draw[dashed,white!60!black] (-1.5,-1.5) rectangle (1.5,1.5);
                \draw (-1.5,0) to [R,l=\SI{2}{\ohm}] (0,0) to [R,l=\SI{2}{\ohm}] (1.5,0);
            \end{circuitikz}
        }
        \wrongchoice{
           \begin{circuitikz}
                \draw[dashed,white!60!black] (-1.5,-1.5) rectangle (1.5,1.5);
                \draw (-1.5,0) to (-1,0) to (-1,1) to [R,l_=\SI{2}{\ohm}] (1,1) to (1,0) to (1.5,0);
                \draw (-1,0) to (-1,-1) to [R,l=\SI{2}{\ohm}] (1,-1) to (1,0);
           \end{circuitikz}
        }
    \end{choices}
    \end{multicols}
\end{questionmult}
}

\element{nysed}{
\begin{question}{June1994-Q33}
    The diagram below shows a circuit with two resistors.
    \begin{center}
    \ctikzset{bipoles/length=0.75cm}
    \begin{circuitikz}
        \draw (0,0) to [R,l=\SI{4}{\ohm}] (2,0) to [R,l=\SI{8}{\ohm}] (4,0) to (4,2) to [battery,l=\SI{6}{\volt}] (0,2) to (0,0);
    \end{circuitikz}
    \end{center}
    Compared to the potential drop across the \SI{8}{\ohm} resistor,
        the potential drop across the \SI{4}{\ohm} resistor is:
    \begin{multicols}{2}
    \begin{choices}
        \wrongchoice{the same}
        \wrongchoice{twice as great}
      \correctchoice{one-half as great}
        \wrongchoice{four times as great}
    \end{choices}
    \end{multicols}
\end{question}
}




%% Section June1990
%%--------------------
\element{nysed}{
\begin{question}{June1990-Q35}
    In the circuit shown below, what is the potential difference of the source?
    \begin{center}
    \ctikzset{bipoles/length=0.75cm}
    \begin{circuitikz}[american voltages]
        \draw (6,0) to (6,2) to [R,l=$R_1$] (4,2) to [R,l=$R_2$] (2,2) to [R,l=$R_3$] (0,2) to (0,0);
        \draw (6,0) to [battery] (0,0);
        \draw (0.25,2) to (0.25,3) to [voltmeter,v^=\SI{10}{\volt}] (1.75,3) to (1.75,2);
        \draw (2.25,2) to (2.25,3) to [voltmeter,v^=\SI{10}{\volt}] (3.75,3) to (3.75,2);
        \draw (4.25,2) to (4.25,3) to [voltmeter,v^=\SI{10}{\volt}] (5.75,3) to (5.75,2);
    \end{circuitikz}
    \end{center}
    \begin{multicols}{2}
    \begin{choices}
        \wrongchoice{\SI{3.3}{\volt}}
        \wrongchoice{\SI{10}{\volt}}
      \correctchoice{\SI{30}{\volt}}
        \wrongchoice{\SI{1 000}{\volt}}
    \end{choices}
    \end{multicols}
\end{question}
}

\element{nysed}{
\begin{question}{June1990-Q36}
    The diagram below shows the current in three of the branches of a direct current electric circuit.
    \begin{center}
    \ctikzset{bipoles/length=0.75cm}
    \begin{circuitikz}
        %% node labels
        \fill (0,-2) circle (2.0pt) node[anchor=north] {$W$};
        \fill (+3,0) circle (2.0pt) node[anchor=north] {$X$};
        \fill (0,+2) circle (2.0pt) node[anchor=south] {$Y$};
        \fill (-3,0) circle (2.0pt) node[anchor=north] {$Z$};
        \fill (0,0) circle (2.0pt) node[anchor=north west] {$P$};
        %% wires
        \draw[thick] (-3,0) to [short,i^>=\SI{3}{\ampere}] (0,0)
                            to [short,i^>=\SI{6}{\ampere}] (3,0);
        \draw[thick] (0,2)  to [short,i^>=\SI{4}{\ampere}] (0,0) to (0,-2);
    \end{circuitikz}
    \end{center}
    The current in the fourth branch,
        between junction $P$ and point $W$, must be:
    \begin{multicols}{2}
    \begin{choices}
      \correctchoice{\SI{1}{\ampere} toward point $W$}
        \wrongchoice{\SI{1}{\ampere} toward point $P$}
        \wrongchoice{\SI{7}{\ampere} toward point $W$}
        \wrongchoice{\SI{7}{\ampere} toward point $P$}
    \end{choices}
    \end{multicols}
\end{question}
}

\element{nysed}{
\begin{question}{June1990-Q37}
    A \SI{5}{\ohm} and a \SI{10}{\ohm} resistor are connected in series.
    The current in the \SI{5}{\ohm} resistor is \SI{2}{\ampere}.
    The current in the \SI{10}{\ohm} resistor is:
    \begin{multicols}{2}
    \begin{choices}
        \wrongchoice{\SI{1}{\ampere}}
      \correctchoice{\SI{2}{\ampere}}
        \wrongchoice{\SI{0.5}{\ampere}}
        \wrongchoice{\SI{8}{\ampere}}
    \end{choices}
    \end{multicols}
\end{question}
}

\element{nysed}{
\begin{question}{June1990-Q38}
    A lamp and an ammeter are connected to a source as shown.
    \begin{center}
    \ctikzset{bipoles/length=1.00cm}
    \begin{circuitikz}
        \draw (0,0) to [battery,l=\SI{10}{\volt}] (0,2)  to [ammeter,i=\SI{5.0}{\ampere}] (3,2) to [R,l={lamp}] (3,0) to (0,0);
    \end{circuitikz}
    \end{center}
    What is the electrical energy expended in the lamp in 3.0 seconds?
    \begin{multicols}{2}
    \begin{choices}
        \wrongchoice{\SI{50}{\joule}}
      \correctchoice{\SI{150}{\joule}}
        \wrongchoice{\SI{50}{\joule}}
        \wrongchoice{\SI{150}{\joule}}
    \end{choices}
    \end{multicols}
\end{question}
}


%% Section June1989
%%--------------------
\element{nysed}{
\begin{question}{June1989-Q34}
    Which diagram below shows correct current direction in a circuit segment?
    \begin{multicols}{2}
    \begin{choices}
        \ctikzset{bipoles/length=0.75cm}
        \AMCboxDimensions{down=-1.2cm}
        \wrongchoice{
            \begin{circuitikz}
                \draw[dashed,white!60!black] (-1.5,-1.5) rectangle (1.5,1.5);
                \draw[thick] (-1.5,0) to [short,i=\SI{2}{\ampere}] (0,0);
                \draw (0,0) to (0,+1) to [short,i=\SI{5}{\ampere}] (1.5,+1);
                \draw (0,0) to (0,-1) to [short,i=\SI{3}{\ampere}] (1.5,-1);
            \end{circuitikz}
        }
        %% ANS is 2
        \correctchoice{
            \begin{circuitikz}
                \draw[dashed,white!60!black] (-1.5,-1.5) rectangle (1.5,1.5);
                \draw[thick] (-1.5,0) to [short,i=\SI{2}{\ampere}] (0,0);
                \draw (0,0) to (0,+1) to [short,i=\SI{5}{\ampere}] (1.5,+1);
                \draw (1.5,-1) to [short,i=\SI{3}{\ampere}] (0,-1) to (0,0);
            \end{circuitikz}
        }
        \wrongchoice{
            \begin{circuitikz}
                \draw[dashed,white!60!black] (-1.5,-1.5) rectangle (1.5,1.5);
                \draw[thick] (-1.5,0) to [short,i=\SI{2}{\ampere}] (0,0);
                \draw (1.5,+1) to [short,i=\SI{5}{\ampere}] (0,+1) to (0,0);
                \draw (1.5,-1) to [short,i=\SI{3}{\ampere}] (0,-1) to (0,0);
            \end{circuitikz}
        }
        \wrongchoice{
            \begin{circuitikz}
                \draw[dashed,white!60!black] (-1.5,-1.5) rectangle (1.5,1.5);
                \draw[thick] (-1.5,0) to [short,i=\SI{2}{\ampere}] (0,0);
                \draw (1.5,+1) to [short,i=\SI{5}{\ampere}] (0,+1) to (0,0);
                \draw (0,0) to (0,-1) to [short,i=\SI{3}{\ampere}] (1.5,-1);
            \end{circuitikz}
        }
    \end{choices}
    \end{multicols}
\end{question}
}


%% Section June1986
%%--------------------
\newcommand{\nysedJuneNineteenEightySixQOneHundredOne}{
\ctikzset{bipoles/length=1.00cm}
\begin{circuitikz}
    \draw (0,0) to [battery,l=\SI{30}{\volt}] (0,3) to (3,3) to [R,l=$R_1$] (3,0) to [ammeter] (0,0);
    \node[anchor=east,xshift=-1ex] at (3,1.5) {\SI{10}{\ohm}};
    \draw (3,3) to (5,3) to [R,l=$R_2$] (5,0) to (3,0);
\end{circuitikz}
}

\element{nysed}{
\begin{question}{June1986-Q101}
    The circuit diagram below shows to resistors ($R_1$ and $R_2$) and an ammeter connected to a constant \SI{30}{\volt} source.
    The combined resistance of the circuit is \SI{6.0}{\ohm}.
    \begin{center}
        \nysedJuneNineteenEightySixQOneHundredOne
    \end{center}
    The resistance of $R_2$ is equal to:
    \begin{multicols}{2}
    \begin{choices}
        \wrongchoice{\SI{6.0}{\ohm}}
        \wrongchoice{\SI{2.0}{\ohm}}
      \correctchoice{\SI{15}{\ohm}}
        \wrongchoice{\SI{4.0}{\ohm}}
    \end{choices}
    \end{multicols}
\end{question}
}

\element{nysed}{
\begin{question}{June1986-Q102}
    The circuit diagram below shows to resistors ($R_1$ and $R_2$) and an ammeter connected to a constant \SI{30}{\volt} source.
    The combined resistance of the circuit is \SI{6.0}{\ohm}.
    \begin{center}
        \nysedJuneNineteenEightySixQOneHundredOne
    \end{center}
    The ammeter read:
    \begin{multicols}{2}
    \begin{choices}
        \wrongchoice{\SI{7.5}{\ampere}}
      \correctchoice{\SI{5.0}{\ampere}}
        \wrongchoice{\SI{3.0}{\ampere}}
        \wrongchoice{\SI{1.2}{\ampere}}
    \end{choices}
    \end{multicols}
\end{question}
}

\element{nysed}{
\begin{question}{June1986-Q103}
    The circuit diagram below shows to resistors ($R_1$ and $R_2$) and an ammeter connected to a constant \SI{30}{\volt} source.
    The combined resistance of the circuit is \SI{6.0}{\ohm}.
    \begin{center}
        \nysedJuneNineteenEightySixQOneHundredOne
    \end{center}
    What power is developed in resistor $R_1$ along?
    \begin{multicols}{2}
    \begin{choices}
        \wrongchoice{\SI{60}{\watt}}
      \correctchoice{\SI{90}{\watt}}
        \wrongchoice{\SI{150}{\watt}}
        \wrongchoice{\SI{250}{\watt}}
    \end{choices}
    \end{multicols}
\end{question}
}

\element{nysed}{
\begin{question}{June1986-Q104}
    The circuit diagram below shows to resistors ($R_1$ and $R_2$) and an ammeter connected to a constant \SI{30}{\volt} source.
    The combined resistance of the circuit is \SI{6.0}{\ohm}.
    \begin{center}
        \nysedJuneNineteenEightySixQOneHundredOne
    \end{center}
    Compared to the potential difference across the source,
        the potential difference across $R_2$ is:
    \begin{multicols}{3}
    \begin{choices}
        \wrongchoice{less}
        \wrongchoice{greater}
      \correctchoice{the same}
    \end{choices}
    \end{multicols}
\end{question}
}

\element{nysed}{
\begin{question}{June1986-Q105}
    The circuit diagram below shows to resistors ($R_1$ and $R_2$) and an ammeter connected to a constant \SI{30}{\volt} source.
    The combined resistance of the circuit is \SI{6.0}{\ohm}.
    \begin{center}
        \nysedJuneNineteenEightySixQOneHundredOne
    \end{center}
    If the resistance of $R_2$ were increased,
        the current through $R_2$ would:
    \begin{choices}
      \correctchoice{decrease}
        \wrongchoice{increase}
        \wrongchoice{remain the same}
    \end{choices}
\end{question}
}



\endinput


