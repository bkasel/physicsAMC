
%% Power Questions used on the
%% NYSED Physics Regents Examination
%%--------------------------------------------------

%% this section contains 42 problems


%% Section June2016
%%--------------------
\element{nysed}{
\begin{question}{June2016-Q01}
	Which quantity is a vector?
    \begin{multicols}{2}
    \begin{choices}
        \wrongchoice{power}
        \wrongchoice{kinetic energy}
        \wrongchoice{speed}
      \correctchoice{weight}
    \end{choices}
    \end{multicols}
\end{question}
}

\element{nysed}{
\begin{question}{June2016-Q17}
    A motor does a total of \SI{480}{\joule} of work in \SI{5.0}{\second} to lift a \SI{12}{\kilo\gram} block to the top of a ramp. 
    The average power developed by the motor is:
    \begin{multicols}{2}
    \begin{choices}
        \wrongchoice{\SI{8.0}{\watt}}
        \wrongchoice{\SI{40.}{\watt}}
      \correctchoice{\SI{96}{\watt}}
        \wrongchoice{\SI{2400}{\watt}}
    \end{choices}
    \end{multicols}
\end{question}
}


%% Section June2015
%%--------------------
\element{nysed}{
\begin{question}{June2015-Q44}
    An electric motor has a rating of \SI{4.0e2}{\watt}.
    How much time will it take for this motor to lift a \SI{50}{\kilo\gram} mass a vertical distance of \SI{8.0}{\meter}?
    [Assume \SI{100}{\percent} efficiency.]
    \begin{multicols}{2}
    \begin{choices}
        \wrongchoice{\SI{0.98}{\second}}
        \wrongchoice{\SI{9.8}{\second}}
      \correctchoice{\SI{98}{\second}}
        \wrongchoice{\SI{980}{\second}}
    \end{choices}
    \end{multicols}
\end{question}
}


%% Section June2014
%%--------------------
\element{nysed}{
\begin{question}{June2014-Q50}
    The graph below represents the work done against gravity by a student as she walks up a flight of stairs at constant speed.
    \begin{center}
    \begin{tikzpicture}
        \begin{axis}[
            axis y line=left,
            axis x line=bottom,
            axis line style={->},
            xlabel={time},
            x unit=\si{\second},
            xtick={0,1,2,3,4,5,6,7},
            ylabel={work},
            y unit=\si{\joule},
            ytick={0,400,800,1200,1600},
            xmin=0,xmax=7.2,
            ymin=0,ymax=1650,
            grid=major,
            width=0.8\columnwidth,
            height=0.5\columnwidth,
            very thin,
        ]
        \addplot[line width=1pt,domain=0:7]{800*x/3};
        \end{axis}
    \end{tikzpicture}
    \end{center}
    Compared to the power generated by the student after \SI{2.0}{\second} the power generated by the student after \SI{4.0}{\second} is:
    \begin{choices}
      \correctchoice{twice as great}
        \wrongchoice{the same}
        \wrongchoice{half as great}
        \wrongchoice{four times as great}
    \end{choices}
\end{question}
}


%% Section June2013
%%--------------------
\element{nysed}{
\begin{question}{June2013-Q39}
    If a motor lifts a \SI{400}{\kilo\gram} mass a vertical distance of \SI{10}{\meter} in \SI{8.0}{\second},
        the \emph{minimum} power generated by the motor is:
    \begin{multicols}{2}
    \begin{choices}
        \wrongchoice{\SI{3.2e2}{\watt}}
        \wrongchoice{\SI{5.0e2}{\watt}}
      \correctchoice{\SI{4.9e3}{\watt}}
        \wrongchoice{\SI{3.2e4}{\watt}}
    \end{choices}
    \end{multicols}
\end{question}
}


%% Section June2012
%%--------------------
\element{nysed}{
\begin{question}{June2012-Q22}
    The Watt second (\si{\watt\second}) is a unit of:
    \begin{choices}
        \wrongchoice{power}
      \correctchoice{energy}
        \wrongchoice{potential difference}
        \wrongchoice{electric field strength}
    \end{choices}
\end{question}
}

\element{nysed}{
\begin{question}{June2012-Q23}
    Which quantity has both a magnitude and a direction?
    \begin{multicols}{2}
    \begin{choices}
        \wrongchoice{energy}
      \correctchoice{impulse}
        \wrongchoice{power}
        \wrongchoice{work}
    \end{choices}
    \end{multicols}
\end{question}
}

\element{nysed}{
\begin{question}{June2012-Q39}
    Two elevators, $A$ and $B$, move at constant speed.
    Elevator $B$ moves with twice the speed of elevator $A$.
    Elevator $B$ weighs twice as much as elevator $A$.
    Compared to the power needed to lift elevator $A$,
        the power needed to lift elevator $B$ is:
    \begin{choices}
        \wrongchoice{the same}
        \wrongchoice{twice as great}
        \wrongchoice{half as great}
      \correctchoice{four times as great}
    \end{choices}
\end{question}
}

\element{nysed}{
\begin{question}{June2012-Q40}
    What is the maximum height to which a motor having a power rating of \SI{20.4}{\watt} can lift a \SI{5.00}{\kilo\gram} stone vertically in \SI{10.0}{\second}?
    \begin{multicols}{2}
    \begin{choices}
        \wrongchoice{\SI{0.0416}{\meter}}
        \wrongchoice{\SI{0.408}{\meter}}
      \correctchoice{\SI{4.16}{\meter}}
        \wrongchoice{\SI{40.8}{\meter}}
    \end{choices}
    \end{multicols}
\end{question}
}


%% Section June2011
%%--------------------
\element{nysed}{
\begin{question}{June2011-Q10}
    What is the power output of an electric motor that lifts a \SI{2.0}{\kilo\gram} block \SI{15}{\meter} vertically in \SI{6.0}{\second}?
    \begin{multicols}{2}
    \begin{choices}
        \wrongchoice{\SI{5.0}{\joule}}
        \wrongchoice{\SI{5.0}{\watt}}
        \wrongchoice{\SI{49}{\joule}}
      \correctchoice{\SI{49}{\watt}}
    \end{choices}
    \end{multicols}
\end{question}
}


%% Section June2010
%%--------------------
\element{nysed}{
\begin{question}{June2010-Q38}
    A small electric motor is used to lift a \SI{0.50}{\kilo\gram} mass at constant speed.
    If the mass is lifted a vertical distance of \SI{1.5}{\meter} in \SI{5.0}{\second},
        the average power developed by the motor is:
    \begin{multicols}{2}
    \begin{choices}
        \wrongchoice{\SI{0.15}{\watt}}
      \correctchoice{\SI{1.5}{\watt}}
        \wrongchoice{\SI{3.8}{\watt}}
        \wrongchoice{\SI{7.5}{\watt}}
    \end{choices}
    \end{multicols}
\end{question}
}


%% Section June2009
%%--------------------
\element{nysed}{
\begin{question}{June2009-Q13}
    A \SI{70}{\kilo\gram} cyclist develops \SI{210}{\watt} of power while pedaling at a constant velocity of \SI{7.0}{\meter\per\second} east.
    What average force is exerted eastward on the bicycle to maintain this constant speed?
    \begin{multicols}{2}
    \begin{choices}
        \wrongchoice{\SI{490}{\newton}}
      \correctchoice{\SI{30}{\newton}}
        \wrongchoice{\SI{3.0}{\newton}}
        \wrongchoice{\SI{0}{\newton}}
    \end{choices}
    \end{multicols}
\end{question}
}


%% Section Jan2009
%%--------------------
\element{nysed}{
\begin{question}{Jan2009-Q09}
    What is the average power required to raise a \SI{1.81e4}{\newton} elevator \SI{12.0}{\meter} in \SI{22.5}{\second}?
    \begin{multicols}{2}
    \begin{choices}
        \wrongchoice{\SI{8.04e2}{\watt}}
      \correctchoice{\SI{9.65e3}{\watt}}
        \wrongchoice{\SI{2.17e5}{\watt}}
        \wrongchoice{\SI{4.89e6}{\watt}}
    \end{choices}
    \end{multicols}
\end{question}
}


%% Section June2008
%%--------------------
\element{nysed}{
\begin{question}{June2008-Q17}
    A \SI{60}{\kilo\gram} student climbs a ladder a vertical distance of \SI{4.0}{\meter} in \SI{8.0}{\second}.
    Approximately how much total work is done against gravity by the student during the climb?
    \begin{multicols}{2}
    \begin{choices}
      \correctchoice{\SI{2.4e3}{\joule}}
        \wrongchoice{\SI{2.9e2}{\joule}}
        \wrongchoice{\SI{2.4e2}{\joule}}
        \wrongchoice{\SI{3.0e1}{\joule}}
    \end{choices}
    \end{multicols}
\end{question}
}

\element{nysed}{
\begin{question}{June2008-Q19}
    What is the maximum amount of work that a \SI{6000}{\watt} motor can do in \SI{10}{\second}?
    \begin{multicols}{2}
    \begin{choices}
        \wrongchoice{\SI{6.0e1}{\joule}}
        \wrongchoice{\SI{6.0e2}{\joule}}
        \wrongchoice{\SI{6.0e3}{\joule}}
      \correctchoice{\SI{6.0e4}{\joule}}
    \end{choices}
    \end{multicols}
\end{question}
}


%% Section Jan2008
%%--------------------
\element{nysed}{
\begin{question}{Jan2008-Q14}
    Student $A$ lifts a \SI{50}{\newton} box from the floor to a height of \SI{0.40}{\meter} in \SI{2.0}{\second}.
    Student $B$ lifts a \SI{40}{\newton} box from the floor to a height of \SI{0.5}{\meter} in \SI{1.0}{\second}.
    Compared to student $A$, student $B$ does:
    \begin{choices}
        \wrongchoice{the same work but develops more power}
      \correctchoice{the same work but develops less power}
        \wrongchoice{more work but develops less power}
        \wrongchoice{less work but develops more power}
    \end{choices}
\end{question}
}


%% Section June2007
%%--------------------
\element{nysed}{
\begin{question}{June2007-Q13}
    Which quantity is a measure of the rate at which work is done?
    \begin{multicols}{2}
    \begin{choices}
      \correctchoice{power}
        \wrongchoice{momentum}
        \wrongchoice{velocity}
        \wrongchoice{energy}
    \end{choices}
    \end{multicols}
\end{question}
}

\element{nysed}{
\begin{question}{June2007-Q44}
    Which graph best represents the relationship between the power required to raise an elevator and the speed at which the elevator rises?
    \begin{multicols}{2}
    \begin{choices}
        \AMCboxDimensions{down=-2.5em}
        \correctchoice{
            \begin{tikzpicture}
                \begin{axis}[
                    axis y line=left,
                    axis x line=bottom,
                    axis line style={->},
                    xlabel={speed},
                    xtick=\empty,
                    ylabel={power},
                    ytick=\empty,
                    xmin=0,xmax=11,
                    ymin=0,ymax=11,
                    width=0.95\columnwidth,
                    height=\columnwidth,
                    very thin,
                ]
                \addplot[line width=1pt,domain=0:10]{x};
                \end{axis}
            \end{tikzpicture}
        }
        \wrongchoice{
            \begin{tikzpicture}
                \begin{axis}[
                    axis y line=left,
                    axis x line=bottom,
                    axis line style={->},
                    xlabel={speed},
                    xtick=\empty,
                    ylabel={power},
                    ytick=\empty,
                    xmin=0,xmax=11,
                    ymin=0,ymax=11,
                    width=0.95\columnwidth,
                    height=\columnwidth,
                    very thin,
                ]
                \addplot[line width=1pt,domain=0:10]{10/x};
                \end{axis}
            \end{tikzpicture}
        }
        \wrongchoice{
            \begin{tikzpicture}
                \begin{axis}[
                    axis y line=left,
                    axis x line=bottom,
                    axis line style={->},
                    xlabel={speed},
                    xtick=\empty,
                    ylabel={power},
                    ytick=\empty,
                    xmin=0,xmax=11,
                    ymin=0,ymax=11,
                    width=0.95\columnwidth,
                    height=\columnwidth,
                    very thin,
                ]
                \addplot[line width=1pt,domain=0:10]{8};
                \end{axis}
            \end{tikzpicture}
        }
        \wrongchoice{
            \begin{tikzpicture}
                \begin{axis}[
                    axis y line=left,
                    axis x line=bottom,
                    axis line style={->},
                    xlabel={speed},
                    xtick=\empty,
                    ylabel={power},
                    ytick=\empty,
                    xmin=0,xmax=11,
                    ymin=0,ymax=11,
                    width=0.95\columnwidth,
                    height=\columnwidth,
                    very thin,
                ]
                \addplot[line width=1pt,domain=0:10]{10-x};
                \end{axis}
            \end{tikzpicture}
        }
    \end{choices}
    \end{multicols}
\end{question}
}


%% Section Jan2007
%%--------------------
\element{nysed}{
\begin{question}{Jan2007-Q46}
    A \SI{110}{\kilo\gram} bodybuilder and his \SI{55}{\kilo\gram} friend run up identical flights of stairs.
    The body building reaches the top in \SI{4.0}{\second} while his friend takes \SI{2.0}{\second}.
    Compared to the power developed by the bodybuilder while running up the stairs,
        the power developed by his friend is:
    \begin{choices}
      \correctchoice{the same}
        \wrongchoice{twice as much}
        \wrongchoice{four times as much}
        \wrongchoice{half as much}
    \end{choices}
\end{question}
}


%% Section June2006
%%--------------------


%% Section Jan2006
%%--------------------
\element{nysed}{
\begin{question}{Jan2006-Q21}
    A truck weighing \SI{3.0e4}{\newton} was driven up a hill that is \SI{1.6e3}{\meter} long to a level area that is \SI{8.0e2}{\meter} above the starting point.
    If the trip took \SI{480}{\second},
        what was the \emph{minimum} power required?
    \begin{multicols}{2}
    \begin{choices}
      \correctchoice{\SI{5.0e4}{\watt}}
        \wrongchoice{\SI{1.0e5}{\watt}}
        \wrongchoice{\SI{1.2e10}{\watt}}
        \wrongchoice{\SI{2.3e10}{\watt}}
    \end{choices}
    \end{multicols}
\end{question}
}

\element{nysed}{
\begin{question}{Jan2006-Q37}
    Which pair of quantities can be expressed using the same units:
    \begin{choices}
      \correctchoice{work and kinetic energy}
        \wrongchoice{power and momentum}
        \wrongchoice{impulse and potential energy}
        \wrongchoice{acceleration and weight}
    \end{choices}
\end{question}
}


%% Section June2005
%%--------------------
\element{nysed}{
\begin{question}{June2005-Q18}
    A \SI{95}{\kilo\gram} student climbs \SI{4.0}{\meter} up a rope in \SI{3.0}{\second}.
    What is the power output of the student?
    \begin{multicols}{2}
    \begin{choices}
      \correctchoice{\SI{1.2e3}{\watt}}
        \wrongchoice{\SI{3.7e3}{\watt}}
        \wrongchoice{\SI{1.3e2}{\watt}}
        \wrongchoice{\SI{3.8e2}{\watt}}
    \end{choices}
    \end{multicols}
\end{question}
}


%% Section Jan2005
%%--------------------
\element{nysed}{
\begin{question}{Jan2005-Q24}
    A motor used \SI{120}{\watt} of power to raise a \SI{15}{\newton} object in \SI{4.0}{\second}.
    Through what vertical distance was the object raised?
    \begin{multicols}{2}
    \begin{choices}
      \correctchoice{\SI{40}{\meter}}
        \wrongchoice{\SI{1.6}{\meter}}
        \wrongchoice{\SI{8.0}{\meter}}
        \wrongchoice{\SI{360}{\meter}}
    \end{choices}
    \end{multicols}
\end{question}
}

\element{nysed}{
\begin{question}{Jan2005-Q39}
    Which unit is equivalent to a newton per kilogram (\si{\newton\per\kilo\gram})?
    \begin{choices}
      \correctchoice{meter per second squared (\si{\meter\per\second\squared})}
        \wrongchoice{watt per meter (\si{\watt\per\meter})}
        \wrongchoice{joule second (\si{\joule\second})}
        \wrongchoice{kilogram meter per second (\si{\kilo\gram\meter\per\second})}
    \end{choices}
\end{question}
}


%% Section June2004
%%--------------------
\element{nysed}{
\begin{question}{June2004-Q15}
    A \SI{40}{\newton} student runs up a staircase to a floor that is \SI{5.0}{\meter} higher than her starting point in \SI{7.0}{\second}.
    The students power output is:
    \begin{multicols}{2}
    \begin{choices}
      \correctchoice{\SI{280}{\watt}}
        \wrongchoice{\SI{1.4e3}{\watt}}
        \wrongchoice{\SI{1.4e4}{\watt}}
        \wrongchoice{\SI{29}{\watt}}
    \end{choices}
    \end{multicols}
\end{question}
}


%% Section Jan2004
%%--------------------
\element{nysed}{
\begin{question}{Jan2004-Q19}
    A boat weighing \SI{9.0e2}{\newton} requires a horizontal force of \SI{6.0e2}{\newton} to move it across the water at \SI{1.5e1}{\meter\per\second}.
    The boat's engine must provide energy at the rate of:
    \begin{multicols}{2}
    \begin{choices}
      \correctchoice{\SI{9.0e3}{\watt}}
        \wrongchoice{\SI{4.0e1}{\watt}}
        \wrongchoice{\SI{2.5e-2}{\joule}}
        \wrongchoice{\SI{7.5e3}{\joule}}
    \end{choices}
    \end{multicols}
\end{question}
}

\element{nysed}{
\begin{question}{Jan2004-Q40}
    The graph below represents the relationship between work done by a student running up a flight and the time of ascent.
    \begin{center}
    \begin{tikzpicture}
        \begin{axis}[
            axis y line=left,
            axis x line=bottom,
            axis line style={->},
            xlabel={time},
            x unit=\si{\second},
            xtick=\empty,
            ylabel={work},
            y unit=\si{\joule},
            ytick=\empty,
            xmin=0,xmax=10,
            ymin=0,ymax=10,
            grid=major,
            width=0.8\columnwidth,
            height=0.5\columnwidth,
            very thin,
        ]
        \addplot[line width=1pt,domain=0:10]{x};
        \end{axis}
    \end{tikzpicture}
    \end{center}
    What does the slope of this graph represent?
    \begin{multicols}{2}
    \begin{choices}
      \correctchoice{power}
        \wrongchoice{impulse}
        \wrongchoice{speed}
        \wrongchoice{momentum}
    \end{choices}
    \end{multicols}
\end{question}
}


%% Section June2003
%%--------------------
\element{nysed}{
\begin{question}{June2003-Q20}
    What is the average power developed by a motor as it lifts a \SI{400}{\kilo\gram} mass at constant speed through a vertical distance of \SI{10.0}{\meter} in \SI{8.0}{\second}?
    \begin{multicols}{2}
    \begin{choices}
      \correctchoice{\SI{4900}{\watt}}
        \wrongchoice{\SI{32000}{\watt}}
        \wrongchoice{\SI{320}{\watt}}
        \wrongchoice{\SI{500}{\watt}}
    \end{choices}
    \end{multicols}
\end{question}
}

\element{nysed}{
\begin{question}{June2003-Q40}
    The graph below shows the relationship between the work done by a student and the time of ascent as the student runs up a flight of stairs.
    \begin{center}
    \begin{tikzpicture}
        \begin{axis}[
            axis y line=left,
            axis x line=bottom,
            axis line style={->},
            xlabel={time},
            x unit=\si{\second},
            xtick=\empty,
            ylabel={work},
            y unit=\si{\joule},
            ytick=\empty,
            xmin=0,xmax=10,
            ymin=0,ymax=10,
            grid=major,
            width=0.8\columnwidth,
            height=0.5\columnwidth,
            very thin,
        ]
        \addplot[line width=1pt,domain=0:10]{x};
        \end{axis}
    \end{tikzpicture}
    \end{center}
    The slope of the graph would have units of:
    \begin{multicols}{2}
    \begin{choices}
      \correctchoice{watts}
        \wrongchoice{joules}
        \wrongchoice{seconds}
        \wrongchoice{newtons}
    \end{choices}
    \end{multicols}
\end{question}
}


%% Section Jan2003
%%--------------------
\element{nysed}{
\begin{question}{Jan2003-Q18}
    A \SI{3.0}{\kilo\gram} block is initially at rest on a frictionless, horizontal surface.
    The block is moved \SI{8.0}{\meter} in \SI{2.0}{\second} by the application of a \SI{12}{\newton} horizontal force,
        as shown in the diagram below.
    \begin{center}
    \begin{tikzpicture}[font=\small]
        %% Frictionless surface
        \draw (-4,0) -- (4,0);
        \node[anchor=north,fill,pattern=north east lines,minimum width=8cm, minimum height=0.01cm] at (0,0) {};
        \node[text centered,anchor=north] at (0,-0.25) {Frictionless Surface}; 
        %% 3 kg block
        \node[draw,anchor=south,fill=white!90!black,minimum size=1cm] (A) at (-3.25,0) {\SI{3.0}{\kilo\gram}};
        \draw[thick,->] (A.east) -- ++(0:2) node[pos=0.5,anchor=south] {\SI{12}{\newton}};
        %% distance
        \draw[thick,<->] (-2.75,1.5) -- (4,1.5) node[pos=0.5,anchor=center,fill=white] {\SI{8.0}{\meter}};
        \draw (-2.75,1.25) -- (-2.75,1.75);
        \draw (4,1.25) -- (4,1.75);
    \end{tikzpicture}
    \end{center}
    What is the average power developed while moving the block?
    \begin{multicols}{2}
    \begin{choices}
      \correctchoice{\SI{48}{\watt}}
        \wrongchoice{\SI{96}{\watt}}
        \wrongchoice{\SI{24}{\watt}}
        \wrongchoice{\SI{32}{\watt}}
    \end{choices}
    \end{multicols}
\end{question}
}

\element{nysed}{
\begin{question}{Jan2003-Q35}
    One watt (\si{\watt}) is equivalent to one:
    \begin{choices}
      \correctchoice{joule per second (\si{\joule\per\second})}
        \wrongchoice{joule second (\si{\joule\second})}
        \wrongchoice{newton per meter (\si{\newton\per\meter})}
        \wrongchoice{newton meter (\si{\newton\meter})}
    \end{choices}
\end{question}
}


%% Section Aug2002
%%--------------------
\element{nysed}{
\begin{question}{Aug2002-Q29}
    In raising an object vertically at a constant speed of \SI{2.0}{\meter\per\second},
        \SI{10}{\watt} of power is developed.
    %% altered wording
    The weight of the object is one watt is equivalent to:
    \begin{multicols}{2}
    \begin{choices}
      \correctchoice{\SI{5.0}{\newton}}
        \wrongchoice{\SI{20}{\newton}}
        \wrongchoice{\SI{40}{\newton}}
        \wrongchoice{\SI{50}{\newton}}
    \end{choices}
    \end{multicols}
\end{question}
}


%% Section June2002
%%--------------------
\element{nysed}{
\begin{question}{June2002-Q01}
    Which is a vector quantity?
    \begin{multicols}{2}
    \begin{choices}
        \wrongchoice{distance}
        \wrongchoice{speed}
        \wrongchoice{power}
      \correctchoice{force}
    \end{choices}
    \end{multicols}
\end{question}
}

\element{nysed}{
\begin{question}{June2002-Q32}
    What is the maximum height to which a \SI{1200}{\watt} motor could lift an object weighing \SI{200}{\newton} in \SI{4.0}{\second}?
    \begin{multicols}{2}
    \begin{choices}
        \wrongchoice{\SI{0.67}{\meter}}
        \wrongchoice{\SI{1.5}{\meter}}
        \wrongchoice{\SI{6.0}{\meter}}
      \correctchoice{\SI{24}{\meter}}
    \end{choices}
    \end{multicols}
\end{question}
}


%% Section Jan2002
%%--------------------
\element{nysed}{
\begin{question}{Jan2002-Q23}
    A \SI{10}{\newton} force is required to move a \SI{3.0}{\kilo\gram} box at constant speed.
    How much power is required to move the box \SI{8.0}{\meter} in \SI{2.0}{\second}?
    \begin{multicols}{2}
    \begin{choices}
      \correctchoice{\SI{40}{\watt}}
        \wrongchoice{\SI{20}{\watt}}
        \wrongchoice{\SI{15}{\watt}}
        \wrongchoice{\SI{12}{\watt}}
    \end{choices}
    \end{multicols}
\end{question}
}


%% Section June2001
%%--------------------
\element{nysed}{
\begin{question}{June2001-Q17}
    A \SI{2000}{\watt} motor working at full capacity can vertically lift a \SI{400}{\newton} weight at a constant speed of:
    \begin{multicols}{2}
    \begin{choices}
        \wrongchoice{\SI{2e3}{\meter\per\second}}
        \wrongchoice{\SI{50}{\meter\per\second}}
      \correctchoice{\SI{5}{\meter\per\second}}
        \wrongchoice{\SI{0.2}{\meter\per\second}}
    \end{choices}
    \end{multicols}
\end{question}
}


%% Section Jan2001
%%--------------------
\element{nysed}{
\begin{question}{Jan2001-Q19}
    A girl weighing \SI{500}{\newton} takes \SI{50}{\second} to climb a flight of stairs \SI{18}{\meter} high.
    Her power output vertically is:
    \begin{multicols}{2}
    \begin{choices}
        \wrongchoice{\SI{9000}{\watt}}
        \wrongchoice{\SI{4000}{\watt}}
        \wrongchoice{\SI{1400}{\watt}}
      \correctchoice{\SI{180}{\watt}}
    \end{choices}
    \end{multicols}
\end{question}
}

\element{nysed}{
\begin{question}{Jan2001-Q36}
    How much time is required for an operating \SI{100}{\watt} light bulb to dissipate \SI{10}{\joule} of electrical energy?
    \begin{multicols}{2}
    \begin{choices}
        \wrongchoice{\SI{1}{\second}}
      \correctchoice{\SI{0.1}{\second}}
        \wrongchoice{\SI{10}{\second}}
        \wrongchoice{\SI{1000}{\second}}
    \end{choices}
    \end{multicols}
\end{question}
}


%% Section June2000
%%--------------------
\element{nysed}{
\begin{question}{June2000-Q20}
    A \SI{5.0e2}{\newton} girl takes \SI{10}{\second} to run up two flights of stairs to a landing,
        a total of \SI{5.0}{\meter} vertically above her starting point.
    What power does the girl develop during her run?
    \begin{multicols}{2}
    \begin{choices}
        \wrongchoice{\SI{25}{\watt}}
        \wrongchoice{\SI{50}{\watt}}
      \correctchoice{\SI{250}{\watt}}
        \wrongchoice{\SI{2500}{\watt}}
    \end{choices}
    \end{multicols}
\end{question}
}


%% Section June1999
%%--------------------
\element{nysed}{
\begin{question}{June1999-Q23}
    If the time required for a student to swim \SI{500}{\meter} is doubled,
        the power developed by the student will be:
    \begin{multicols}{2}
    \begin{choices}
      \correctchoice{halved}
        \wrongchoice{doubled}
        \wrongchoice{quartered}
        \wrongchoice{quadrupled}
    \end{choices}
    \end{multicols}
\end{question}
}


%% Section June1998
%%--------------------
\element{nysed}{
\begin{question}{June1998-Q17}
    A \SI{45}{\kilo\gram} bicyclist climbs a hill at a constant speed of \SI{2.5}{\meter\per\second} by applying an average force of \SI{85}{\newton}.
    Approximately how much power does the bicyclist develop?
    \begin{multicols}{2}
    \begin{choices}
        \wrongchoice{\SI{110}{\watt}}
      \correctchoice{\SI{210}{\watt}}
        \wrongchoice{\SI{1100}{\watt}}
        \wrongchoice{\SI{1400}{\watt}}
    \end{choices}
    \end{multicols}
\end{question}
}


%% Section June1997
%%--------------------
\element{nysed}{
\begin{question}{June1997-Q17}
    %% this question was altered
    Which combination of base SI units is correctly paired with its corresponding derived SI unit?
    \begin{choices}
        \wrongchoice{\si{\kilo\gram\meter\per\second} and watt}
        \wrongchoice{\si{\kilo\gram\meter\squared\per\second} and watt}
      \correctchoice{\si{\kilo\gram\meter\squared\per\second\squared} and joule}
        \wrongchoice{\si{\kilo\gram\meter\per\second\cubed} and joule}
    \end{choices}
\end{question}
}

\element{nysed}{
\begin{question}{June1997-Q20}
    A motor having a maximum power rating of \SI{8.1e4}{\watt} is used to operate an elevator with a weight of \SI{1.8e4}{\newton}.
    What is the maximum weight this motor can lift at an average speed of \SI{3.0}{\meter\per\second}?
    \begin{multicols}{2}
    \begin{choices}
        \wrongchoice{\SI{6.0e3}{\newton}}
        \wrongchoice{\SI{1.8e4}{\newton}}
        \wrongchoice{\SI{2.4e4}{\newton}}
      \correctchoice{\SI{2.7e4}{\newton}}
    \end{choices}
    \end{multicols}
\end{question}
}


%% Section June1996
%%--------------------
\element{nysed}{
\begin{question}{June1996-Q19}
    A \SI{10}{\newton} force is required to move a \SI{3.0}{\kilo\gram} box at constant speed.
    How much power is required to move the box \SI{8.0}{\meter} in \SI{2.0}{\second}?
    \begin{multicols}{2}
    \begin{choices}
      \correctchoice{\SI{40}{\watt}}
        \wrongchoice{\SI{20}{\watt}}
        \wrongchoice{\SI{15}{\watt}}
        \wrongchoice{\SI{12}{\watt}}
    \end{choices}
    \end{multicols}
\end{question}
}


%% Section June1995
%%--------------------
\element{nysed}{
\begin{question}{June1995-Q05}
    Which term represents a vector quantity?
    \begin{multicols}{2}
    \begin{choices}
        \wrongchoice{work}
        \wrongchoice{power}
      \correctchoice{force}
        \wrongchoice{distance}
    \end{choices}
    \end{multicols}
\end{question}
}

\element{nysed}{
\begin{question}{June1995-Q16}
    A \SI{4.0e3}{\watt} motor applies a force of \SI{8.0e2}{\newton} to move a boat at constant speed.
    How far does the boat move in \SI{16}{\second}?
    \begin{multicols}{2}
    \begin{choices}
        \wrongchoice{\SI{3.2}{\meter}}
        \wrongchoice{\SI{5.0}{\meter}}
        \wrongchoice{\SI{32}{\meter}}
      \correctchoice{\SI{80}{\meter}}
    \end{choices}
    \end{multicols}
\end{question}
}


%% Section June1994
%%--------------------


%% Section June1990
%%--------------------
\element{nysed}{
\begin{question}{June1990-Q19}
    A motor has an output of \SI{1 000}{\watt}.
    When the motor is working at full capacity,
        how much time will it require to lift a \SI{50}{\newton} weight \SI{100}{\meter}?
    \begin{multicols}{2}
    \begin{choices}
      \correctchoice{\SI{5}{\second}}
        \wrongchoice{\SI{10}{\second}}
        \wrongchoice{\SI{50}{\second}}
        \wrongchoice{\SI{100}{\second}}
    \end{choices}
    \end{multicols}
\end{question}
}


%% Section June1989
%%--------------------
\element{nysed}{
\begin{question}{June1989-Q21}
    Which terms represent scalar quantities?
    \begin{choices}
        \wrongchoice{power and force}
        \wrongchoice{work and displacement}
      \correctchoice{time and energy}
        \wrongchoice{distance and velocity}
    \end{choices}
\end{question}
}

\element{nysed}{
\begin{question}{June1989-Q22}
    What is the maximum distance that a \SI{60}{\watt} motor may vertically lift a \SI{90}{\newton} weight in \SI{7.5}{\second}?
    \begin{multicols}{2}
    \begin{choices}
        \wrongchoice{\SI{2.3}{\meter}}
      \correctchoice{\SI{5.0}{\meter}}
        \wrongchoice{\SI{140}{\meter}}
        \wrongchoice{\SI{1100}{\meter}}
    \end{choices}
    \end{multicols}
\end{question}
}


%% Section June1986
%%--------------------
\element{nysed}{
\begin{question}{June1986-Q15}
    A weightlifter lifts a \SI{2 000}{\newton} weight a vertical distance of \SI{0.5}{\meter} in \SI{0.1}{\second}.
    What is the power output?
    \begin{multicols}{2}
    \begin{choices}
        \wrongchoice{\SI{1e-4}{\watt}}
        \wrongchoice{\SI{4e-4}{\watt}}
      \correctchoice{\SI{1e4}{\watt}}
        \wrongchoice{\SI{4e4}{\watt}}
    \end{choices}
    \end{multicols}
\end{question}
}


%% Section June1985
%%--------------------
\element{nysed}{
\begin{question}{June1985-Q15}
    If an engine rated at \SI{5.0e4}{\watt} exerts a constant force of \SI{2.5e3}{\newton} on a vehicle,
        the velocity of the vehicle is:
    \begin{multicols}{2}
    \begin{choices}
        \wrongchoice{\SI{0.050}{\meter\per\second}}
        \wrongchoice{\SI[parse-numbers=false]{2\sqrt{10}}{\meter\per\second}}
      \correctchoice{\SI{20}{\meter\per\second}}
        \wrongchoice{\SI{1.25e8}{\meter\per\second}}
    \end{choices}
    \end{multicols}
\end{question}
}


\endinput


