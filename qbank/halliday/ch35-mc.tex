
%%--------------------------------------------------
%% Halliday: Fundamentals of Physics
%%--------------------------------------------------


%% Chapter 35: Interference
%%--------------------------------------------------


%% Learning Objectives
%%--------------------------------------------------

%% 35.01: Using a sketch, explain Huygens' principle.
%% 35.02: With a few simple sketches, explain refraction in terms of the gradual change in the speed of a wavefront as it passes through an interface at an angle to the normal.
%% 35.03: Apply the relationship between the speed of light in vacuum $c$, the speed of light in a material $v$, and the index of refraction of the material $n$.
%% 35.04: Apply the relationship between a distance $L$ in a material, the speed of light in that material, and the time required for a pulse of the light to travel through $L$.
%% 35.05: Apply Snell's law of refraction.
%% 35.06: When light refracts through an interface, identify that the frequency does not change but the wavelength and effective speed do.
%% 35.07: Apply the relationship between the wavelength in vacuum $\lambda$, the wavelength $\lambda_n$ in a material (the internal wavelength), and the index of refraction $n$ of the material.
%% 35.08: For light in a certain length of a material, calculate the number of internal wavelengths that fit into the length.
%% 35.09: If two light waves travel through different materials with different indexes of refraction and then reach a common point, determine their phase difference and interpret the resulting interference in terms of maximum brightness, intermediate brightness, and darkness.
%% 35.10: Apply the learning objectives of Module 17-3 (sound waves there, light waves here) to find the phase difference and interference of two waves that reach a common point after traveling paths of different lengths.
%% 35.11: Given the initial phase difference between two waves with the same wavelength, determine their phase difference after they travel through different path lengths and through different indexes of refraction.
%% 35.12: Identify that rainbows are examples of optical interference.


%% Halliday Multiple Choice Questions
%%--------------------------------------------------
\element{halliday-mc}{
\begin{question}{halliday-ch35-q01}
    A ``wave front'' is a surface of constant:
    \begin{choices}
      \correctchoice{phase}
        \wrongchoice{frequency}
        \wrongchoice{wavelength}
        \wrongchoice{amplitude}
        \wrongchoice{speed}
    \end{choices}
\end{question}
}

\element{halliday-mc}{
\begin{question}{halliday-ch35-q02}
    Huygens' construction can be used only:
    \begin{choices}
        \wrongchoice{for light}
        \wrongchoice{for an electromagnetic wave}
        \wrongchoice{if one of the media is vacuum (or air)}
        \wrongchoice{for transverse waves}
      \correctchoice{for all of the provided and other situations}
    \end{choices}
\end{question}
}

\element{halliday-mc}{
\begin{question}{halliday-ch35-q03}
    Consider (I) the law of reflection and (II) the law of refraction. 
    Huygens’ principle can be used to derive:
    \begin{choices}
        \wrongchoice{only I}
        \wrongchoice{only II}
      \correctchoice{both I and II}
        \wrongchoice{neither I nor II}
        \wrongchoice{the question is meaningless because Huygens’ principle is for wave fronts whereas both I and II concern rays}
    \end{choices}
\end{question}
}

\element{halliday-mc}{
\begin{question}{halliday-ch35-q04}
    Units of ``optical path length'' are:
    \begin{choices}
        \wrongchoice{per meter (\si{\per\meter})}
      \correctchoice{meter (\si{\meter})}
        \wrongchoice{meter per second (\si{\meter\per\second})}
        \wrongchoice{hertz per meter (\si{\hertz\per\meter})}
        \wrongchoice{meter per hertz (\si{\meter\per\hertz})}
    \end{choices}
\end{question}
}

\element{halliday-mc}{
\begin{question}{halliday-ch35-q05}
    The light waves represented by the three rays shown in the diagram all have the same frequency.
    4.7 wavelengths fit into layer 1, 3.2 wavelengths fit into layer 2, and 5.3 wavelengths fit into layer 3. 
    \begin{center}
    \begin{tikzpicture}
        %% NOTE: tikz
    \end{tikzpicture}
    \end{center}
    Rank the layers according to the speeds of the waves,
        least to greatest.
    \begin{multicols}{2}
    \begin{choices}
        \wrongchoice{1, 2, 3}
        \wrongchoice{2, 1, 3}
        \wrongchoice{3, 1, 2}
      \correctchoice{3, 1, 2}
        \wrongchoice{1, 3, 2}
    \end{choices}
    \end{multicols}
\end{question}
}

\element{halliday-mc}{
\begin{question}{halliday-ch35-q06}
    Interference of light is evidence that:
    \begin{choices}
        \wrongchoice{the speed of light is very large}
        \wrongchoice{light is a transverse wave}
        \wrongchoice{light is electromagnetic in character}
        \wrongchoice{light is a wave phenomenon}
        \wrongchoice{light does not obey conservation of energy}
    \end{choices}
\end{question}
}

\element{halliday-mc}{
\begin{question}{halliday-ch35-q07}
    The reason there are two slits, rather than one,
        in a Young's experiment is:
    \begin{choices}
        \wrongchoice{to increase the intensity}
        \wrongchoice{one slit is for frequency, the other for wavelength}
      \correctchoice{to create a path length difference}
        \wrongchoice{one slit is for E fields, the other is for B fields}
        \wrongchoice{two slits in parallel offer less resistance}
    \end{choices}
\end{question}
}

\element{halliday-mc}{
\begin{question}{halliday-ch35-q08}
    In a Young's double-slit experiment the center of a bright fringe occurs wherever waves from the slits differ in the distance they travel by a multiple of:
    \begin{choices}
        \wrongchoice{a fourth of a wavelength}
        \wrongchoice{a half a wavelength}
        \wrongchoice{a wavelength}
        \wrongchoice{three-fourths of a wavelength}
      \correctchoice{none of the provided}
    \end{choices}
\end{question}
}

\element{halliday-mc}{
\begin{question}{halliday-ch35-q09}
    In a Young's double-slit experiment the center of a bright fringe occurs wherever waves from the slits differ in phase by a multiple of:
    \begin{multicols}{3}
    \begin{choices}
        \wrongchoice{$\dfrac{\pi}{4}$}
        \wrongchoice{$\dfrac{\pi}{2}$}
        \wrongchoice{$\pi$}
        \wrongchoice{$\dfrac{3\pi}{4}$}
      \correctchoice{$2\pi$}
    \end{choices}
    \end{multicols}
\end{question}
}

\element{halliday-mc}{
\begin{question}{halliday-ch35-q10}
    Waves from two slits are in phase at the slits and travel to a distant screen to produce the third side maximum of the interference pattern. 
    The difference in the distance traveled by the waves is:
    \begin{choices}
        \wrongchoice{half a wavelength}
        \wrongchoice{a wavelength}
        \wrongchoice{three halves of a wavelength}
        \wrongchoice{two wavelengths}
      \correctchoice{three wavelengths}
    \end{choices}
\end{question}
}

\element{halliday-mc}{
\begin{question}{halliday-ch35-q11}
    Waves from two slits are in phase at the slits and travel to a distant screen to produce the second minimum of the interference pattern. 
    The difference in the distance traveled by the waves is:
    \begin{choices}
        \wrongchoice{half a wavelength}
        \wrongchoice{a wavelength}
      \correctchoice{three halves of a wavelength}
        \wrongchoice{two wavelengths}
        \wrongchoice{five halves of a wavelength}
    \end{choices}
\end{question}
}

\element{halliday-mc}{
\begin{question}{halliday-ch35-q12}
    A monochromatic light source illuminates a double slit and the resulting interference pattern is observed on a distant screen. 
    Let $d=$ center-to-center slit spacing, $a=$ individual slit width,
        $D=$ screen-to-slit distance, and $l=$ adjacent dark line spacing in the interference pattern. 
    The wavelength of the light is then:
    \begin{multicols}{3}
    \begin{choices}
      \correctchoice{$\dfrac{d}{D}$}
        \wrongchoice{$\dfrac{Ld}{a}$}
        \wrongchoice{$\dfrac{da}{D}$}
        \wrongchoice{$\dfrac{D}{a}$}
        \wrongchoice{$\dfrac{Dd}{l}$}
    \end{choices}
    \end{multicols}
\end{question}
}

\element{halliday-mc}{
\begin{question}{halliday-ch35-q13}
    Light from a small region of an ordinary incandescent bulb is passed through a yellow filter and then serves as the source for a Young's double-slit experiment. 
    Which of the following changes would cause the interference pattern to be more closely spaced?
    \begin{choices}
        \wrongchoice{Use slits that are closer together}
        \wrongchoice{Use a light source of lower intensity}
        \wrongchoice{Use a light source of higher intensity}
      \correctchoice{Use a blue filter instead of a yellow filter}
        \wrongchoice{Move the light source further away from the slits.}
    \end{choices}
\end{question}
}

\element{halliday-mc}{
\begin{question}{halliday-ch35-q14}
    In a Young's double-slit experiment,
        the slit separation is doubled. 
    To maintain the same fringe spacing on the screen,
        the screen-to-slit distance $D$ must be changed to:
    \begin{multicols}{3}
    \begin{choices}
        \wrongchoice{$\dfrac{D}{2}$}
        \wrongchoice{$\dfrac{D}{\sqrt{2}}$}
        \wrongchoice{$D\sqrt{2}$}
      \correctchoice{$2D$}
        \wrongchoice{$4D$}
    \end{choices}
    \end{multicols}
\end{question}
}

\element{halliday-mc}{
\begin{question}{halliday-ch35-q15}
    In a Young's double-slit experiment,
        light of wavelength \SI{500}{\nano\meter} illuminates two slits that are separated by \SI{1}{\milli\meter}.
    The separation between adjacent bright fringes on a screen \SI{5}{\meter} from the slits is:
    \begin{multicols}{2}
    \begin{choices}
        \wrongchoice{\SI{0.10}{\centi\meter}}
      \correctchoice{\SI{0.25}{\centi\meter}}
        \wrongchoice{\SI{0.50}{\centi\meter}}
        \wrongchoice{\SI{1.0}{\centi\meter}}
        \wrongchoice{none of the provided}
    \end{choices}
    \end{multicols}
\end{question}
}

\element{halliday-mc}{
\begin{question}{halliday-ch35-q16}
    In a Young's double-slit experiment,
        the separation between slits is $d$ and the screen is a distance $D$ from the slits. 
    $D$ is much greater than $d$ and $\lambda$ is the wavelength of the light. 
    The number of bright fringes per unit width on the screen is:
    \begin{multicols}{3}
    \begin{choices}
        \wrongchoice{$\dfrac{Dd}{\lambda}$}
        \wrongchoice{$\dfrac{D\lambda}{d}$}
        \wrongchoice{$\dfrac{D}{d\lambda}$}
        \wrongchoice{$\dfrac{\lambda}{Dd}$}
      \correctchoice{$\dfrac{d}{D\lambda}$}
    \end{choices}
    \end{multicols}
\end{question}
}

\element{halliday-mc}{
\begin{question}{halliday-ch35-q17}
    In a Young’s double-slit experiment,
        the slit separation is doubled. 
    This results in:
    \begin{choices}
        \wrongchoice{an increase in fringe intensity}
        \wrongchoice{a decrease in fringe intensity}
        \wrongchoice{a halving of the wavelength}
      \correctchoice{a halving of the fringe spacing}
        \wrongchoice{a doubling of the fringe spacing}
    \end{choices}
\end{question}
}

\element{halliday-mc}{
\begin{question}{halliday-ch35-q18}
    In an experiment to measure the wavelength of light using a double slit,
        it is found that the fringes are too close together to easily count them. 
    To spread out the fringe pattern, one could:
    \begin{choices}
      \correctchoice{decrease the slit separation}
        \wrongchoice{increase the slit separation}
        \wrongchoice{increase the width of each slit}
        \wrongchoice{decrease the width of each slit}
        \wrongchoice{none of the provided}
    \end{choices}
\end{question}
}

\element{halliday-mc}{
\begin{question}{halliday-ch35-q19}
    The phase difference between the two waves that give rise to a dark spot in a Young's double-slit experiment is
        (where $m=$ integer):
    \begin{multicols}{2}
    \begin{choices}
        \wrongchoice{zero}
        \wrongchoice{$2\pi m + \dfrac{\pi}{8}$}
        \wrongchoice{$2\pi m + \dfrac{\pi}{4}$}
        \wrongchoice{$2\pi m + \dfrac{\pi}{2}$}
      \correctchoice{$2\pi m + \pi$}
    \end{choices}
    \end{multicols}
\end{question}
}

\element{halliday-mc}{
\begin{question}{halliday-ch35-q20}
    In a Young's experiment,
        it is essential that the two beams:
    \begin{choices}
        \wrongchoice{have exactly equal intensity}
        \wrongchoice{be exactly parallel}
        \wrongchoice{travel equal distances}
      \correctchoice{come originally from the same source}
        \wrongchoice{be composed of a broad band of frequencies}
    \end{choices}
\end{question}
}

\element{halliday-mc}{
\begin{question}{halliday-ch35-q21}
    A light wave with an electric field amplitude of $E_0$ and a phase constant of zero is to be combined with one of the following waves:
    Which of these combinations produces the greatest intensity?
    \begin{choices}
        %% NOTE: an amplitude of ... a a phase const ...
        \wrongchoice{has an amplitude of $E_0$ and a phase constant of zero}
        \wrongchoice{has an amplitude of $E_0$ and a phase constant of $\pi$}
      \correctchoice{has an amplitude of $2E_0$ and a phase constant of zero}
        \wrongchoice{has an amplitude of $2E_0$ and a phase constant of $\pi$}
        \wrongchoice{has an amplitude of $3E_0$ and a phase constant of $\pi$}
    \end{choices}
\end{question}
}

\element{halliday-mc}{
\begin{question}{halliday-ch35-q22}
    A light wave with an electric field amplitude of $2E_0$ and a phase constant of zero is to be combined with one of the following waves:
    Which of these combinations produces the least intensity?
    \begin{choices}
        %% NOTE: an amplitude of ... a a phase const ...
        \wrongchoice{has an amplitude of $E_0$ and a phase constant of zero}
        \wrongchoice{has an amplitude of $E_0$ and a phase constant of $\pi$}
        \wrongchoice{has an amplitude of $2E_0$ and a phase constant of zero}
      \correctchoice{has an amplitude of $2E_0$ and a phase constant of $\pi$}
        \wrongchoice{has an amplitude of $3E_0$ and a phase constant of $\pi$}
    \end{choices}
\end{question}
}

\element{halliday-mc}{
\begin{question}{halliday-ch35-q23}
    One of the two slits in a Young's experiment is painted over so that it transmits only one-half the intensity of the other slit. 
    As a result:
    \begin{choices}
        \wrongchoice{the fringe system disappears}
        \wrongchoice{the bright fringes get brighter and the dark ones get darker}
        \wrongchoice{the fringes just get dimmer}
        \wrongchoice{the dark fringes just get brighter}
      \correctchoice{the dark fringes get brighter and the bright ones get darker}
    \end{choices}
\end{question}
}

\element{halliday-mc}{
\begin{question}{halliday-ch35-q24}
    In a Young's double-slit experiment,
        a thin sheet of mica is placed over one of the two slits. 
    As a result,
        the center of the fringe pattern (on the screen) shifts by an amount corresponding to 30 dark bands. 
    The wavelength of the light in this experiment is \SI{480}{\nano\meter} and the index of the mica is 1.60. 
    The mica thickness is:
    \begin{multicols}{2}
    \begin{choices}
        \wrongchoice{\SI{0.090}{\milli\meter}}
        \wrongchoice{\SI{0.012}{\milli\meter}}
        \wrongchoice{\SI{0.014}{\milli\meter}}
      \correctchoice{\SI{0.024}{\milli\meter}}
        \wrongchoice{\SI{0.062}{\milli\meter}}
    \end{choices}
    \end{multicols}
\end{question}
}

\element{halliday-mc}{
\begin{question}{halliday-ch35-q25}
    Light from a point source $X$ contains only blue and red components. 
    After passing through a mysterious box,
        the light falls on a screen. 
    \begin{center}
    \begin{tikzpicture}
        %% NOTE: tikz
    \end{tikzpicture}
    \end{center}
    Bright red and blue hands are observed as shown.
    The box must contain:
    \begin{multicols}{2}
    \begin{choices}
        \wrongchoice{a lens}
        \wrongchoice{a mirror}
        \wrongchoice{a prism}
      \correctchoice{a double slit}
        \wrongchoice{a blue and red filter}
    \end{choices}
    \end{multicols}
\end{question}
}

\element{halliday-mc}{
\begin{question}{halliday-ch35-q26}
    Binoculars and microscopes are frequently made with coated optics by adding a thin layer of transparent material to the lens surface as shown. 
    \begin{center}
    \begin{tikzpicture}
        %% NOTE: tikz
    \end{tikzpicture}
    \end{center}
    One wants:
    \begin{choices}
        \wrongchoice{constructive interference between waves 1 and 2}
        \wrongchoice{destructive interference between waves 3 and 4}
      \correctchoice{constructive interference between 3 and 4}
        \wrongchoice{the coating to be more transparent than the lens}
        \wrongchoice{the speed of light in the coating to be less than that in the lens}
    \end{choices}
\end{question}
}

\element{halliday-mc}{
\begin{question}{halliday-ch35-q27}
    Monochromatic light, at normal incidence, strikes a thin film in air. 
    If $\lambda$ denotes the wavelength in the film,
        what is the thinnest film in which the reflected light will be a maximum?
    \begin{multicols}{2}
    \begin{choices}
        \wrongchoice{Much less than $\lambda$}
        \wrongchoice{$\dfrac{\lambda}{4}$}
        \wrongchoice{$\dfrac{\lambda}{2}$}
        \wrongchoice{$\dfrac{3\lambda}{4}$}
        \wrongchoice{$\lambda$}
    \end{choices}
    \end{multicols}
\end{question}
}

\element{halliday-mc}{
\begin{question}{halliday-ch35-q28}
    A soap film is illuminated by white light normal to its surface. 
    The index of refraction of the film is \num{1.50}.
    Wavelengths of \SI{480}{\nano\meter} and \SI{800}{\nano\meter} and no wavelengths between are be intensified in the reflected beam. 
    The thickness of the film is:
    \begin{multicols}{2}
    \begin{choices}
        \wrongchoice{\SI{1.5e-5}{\centi\meter}}
        \wrongchoice{\SI{2.4e-5}{\centi\meter}}
        \wrongchoice{\SI{3.6e-5}{\centi\meter}}
      \correctchoice{\SI{4.0e-5}{\centi\meter}}
        \wrongchoice{\SI{6.0e-5}{\centi\meter}}
    \end{choices}
    \end{multicols}
\end{question}
}

\element{halliday-mc}{
\begin{question}{halliday-ch35-q29}
    Red light is viewed through a thin vertical soap film. 
    \begin{center}
    \begin{tikzpicture}
        %% NOTE: tikz
    \end{tikzpicture}
    \end{center}
    At the third dark area shown,
        the thickness of the film,
        in terms of the wavelength within the film, is:
    \begin{multicols}{3}
    \begin{choices}
        \wrongchoice{$\lambda$}
        \wrongchoice{$\dfrac{3\lambda}{4}$}
        \wrongchoice{$\dfrac{\lambda}{2}$}
        \wrongchoice{$\dfrac{\lambda}{4}$}
      \correctchoice{$\dfrac{5\lambda}{4}$}
    \end{choices}
    \end{multicols}
\end{question}
}

\element{halliday-mc}{
\begin{question}{halliday-ch35-q30}
    Yellow light is viewed by reflection from a thin vertical soap film. 
    \begin{center}
    \begin{tikzpicture}
        %% NOTE: tikz
    \end{tikzpicture}
    \end{center}
    Let $\lambda$ be the wavelength of the light within the film. 
    Why is there a large dark space at the top of the film?
    \begin{choices}
        \wrongchoice{no light is transmitted through this part of the film}
        \wrongchoice{the film thickness there is $\dfrac{\lambda}{4}$}
      \correctchoice{the film thickness there is much less than $\lambda$}
        \wrongchoice{the film is too thick in this region for thin film formulas to apply}
        \wrongchoice{the reflected light is in the infrared}
    \end{choices}
\end{question}
}

\element{halliday-mc}{
\begin{question}{halliday-ch35-q31}
    Three experiments involving a thin film (in air) are shown. 
    If $t$ denotes the film thickness and $\lambda$ denotes the wavelength of the light in the film, which experiments will produce constructive interference as seen by the observer? 
    \begin{center}
    \begin{tikzpicture}
        %% NOTE: tikz
    \end{tikzpicture}
    \end{center}
    The incident light is nearly normal to the surface.
    \begin{multicols}{2}
    \begin{choices}
        \wrongchoice{I only}
        \wrongchoice{II only}
        \wrongchoice{III only}
      \correctchoice{I and III only}
        \wrongchoice{II and III only}
    \end{choices}
    \end{multicols}
\end{question}
}

\element{halliday-mc}{
\begin{question}{halliday-ch35-q32}
    A liquid of refractive index $n=4/3$ replaces the air between a fixed wedge formed from two glass plates as shown. 
    As a result, the spacing between adjacent dark bands in the interference pattern:
    \begin{center}
    \begin{tikzpicture}
        %% NOTE: tikz
    \end{tikzpicture}
    \end{center}
    \begin{choices}
        \wrongchoice{increases by a factor of $4/3$}
        \wrongchoice{increases by a factor of $3$}
        \wrongchoice{remains the same}
      \correctchoice{decreases to $3/4$ of its original value}
        \wrongchoice{decreases to $1/3$ of its original value}
    \end{choices}
\end{question}
}

\element{halliday-mc}{
\begin{question}{halliday-ch35-q33}
    A lens with a refractive index of \num{1.5} is coated with a material of refractive index \num{1.2} in order to minimize reflection.
    \begin{center}
    \begin{tikzpicture}
        %% NOTE: tikz
    \end{tikzpicture}
    \end{center}
    If $\lambda$ denotes the wavelength of the incident light in air,
        what is the thinnest possible such coating?
    \begin{multicols}{3}
    \begin{choices}
        \wrongchoice{$0.5\lambda$}
        \wrongchoice{$0.416\lambda$}
        \wrongchoice{$0.3\lambda$}
      \correctchoice{$0.208\lambda$}
        \wrongchoice{$0.25\lambda$}
    \end{choices}
    \end{multicols}
\end{question}
}

\element{halliday-mc}{
\begin{question}{halliday-ch35-q34}
    A lens with a refractive index of \num{1.5} is coated with a material of refractive index \num{1.2} in order to minimize reflection.
    \begin{center}
    \begin{tikzpicture}
        %% NOTE: tikz
    \end{tikzpicture}
    \end{center}
    In a thin film experiment, a wedge of air is used between two glass plates. 
    If the wavelength of the incident light in air is \SI{480}{\nano\meter},
        how much thicker is the air wedge at the 16$^{th}$ dark fringe than it is at the 6$^{th}$?
    \begin{multicols}{2}
    \begin{choices}
      \correctchoice{\SI{2400}{\nano\meter}}
        \wrongchoice{\SI{4800}{\nano\meter}}
        \wrongchoice{\SI{240}{\nano\meter}}
        \wrongchoice{\SI{480}{\nano\meter}}
        \wrongchoice{None of the provided}
    \end{choices}
    \end{multicols}
\end{question}
}

\element{halliday-mc}{
\begin{question}{halliday-ch35-q35}
    An air wedge is formed from two glass plates that are in contact at their left edges. 
    There are ten dark bands when viewed by reflection using monochromatic light. 
    The left edge of the top plate is now slowly lifted until the plates are parallel. 
    During this process:
    \begin{choices}
        \wrongchoice{the dark bands crowd toward the right edge}
        \wrongchoice{the dark bands remain stationary}
        \wrongchoice{the dark bands crowd toward the left edge}
        \wrongchoice{the dark bands spread out, disappearing off the right edge}
      \correctchoice{the dark bands spread out, disappearing off the left edge}
    \end{choices}
\end{question}
}

\element{halliday-mc}{
\begin{question}{halliday-ch35-q36}
    An air wedge is formed using two glass plates that are in contact along their left edge. 
    When viewed by highly monochromatic light,
        there are exactly \num{4001} dark bands in the reflected light.
    The air is now evacuated (with the glass plates remaining rigidly fixed) and the number of dark bands decreases to exactly \num{4000}.
    The index of refraction of the air is:
    \begin{choices}
        \wrongchoice{0.00025}
        \wrongchoice{0.00050}
      \correctchoice{1.00025}
        \wrongchoice{1.00050}
        \wrongchoice{1.00000, by definition}
    \end{choices}
\end{question}
}

\element{halliday-mc}{
\begin{question}{halliday-ch35-q37}
    A glass ($n=1.6$) lens is coated with a thin film ($n=1.3$) to reduce reflection of certain incident light. 
    If $\lambda$ is the wavelength of the light in the film,
        the least film thickness is:
    \begin{multicols}{2}
    \begin{choices}
        \wrongchoice{less than $\dfrac{\lambda}{4}$}
      \correctchoice{$\dfrac{\lambda}{4}$}
        \wrongchoice{$\dfrac{\lambda}{2}$}
        \wrongchoice{$\lambda$}
        \wrongchoice{more than $\lambda$}
    \end{choices}
    \end{multicols}
\end{question}
}

\element{halliday-mc}{
\begin{question}{halliday-ch35-q38}
    Two point sources, vibrating in phase,
        produce an interference pattern in a ripple tank. 
    If the frequency is increased by \SI{20}{\percent},
        the number of nodal lines:
    \begin{choices}
      \correctchoice{is increased by \SI{20}{\percent}}
        \wrongchoice{is increased by \SI{40}{\percent}}
        \wrongchoice{remains the same}
        \wrongchoice{is decreased by \SI{20}{\percent}}
        \wrongchoice{is decreased by \SI{40}{\percent}}
    \end{choices}
\end{question}
}

\element{halliday-mc}{
\begin{question}{halliday-ch35-q39}
    If two light waves are coherent:
    \begin{choices}
        \wrongchoice{their amplitudes are the same}
        \wrongchoice{their frequencies are the same}
        \wrongchoice{their wavelengths are the same}
      \correctchoice{their phase difference is constant}
        \wrongchoice{the difference in their frequencies is constant}
    \end{choices}
\end{question}
}

\element{halliday-mc}{
\begin{question}{halliday-ch35-q40}
    To obtain an observable double-slit interference pattern:
    \begin{choices}
        \wrongchoice{the light must be incident normally on the slits}
        \wrongchoice{the light must be monochromatic}
        \wrongchoice{the light must consist of plane waves}
      \correctchoice{the light must be coherent}
        \wrongchoice{the screen must be far away from the slits}
    \end{choices}
\end{question}
}


\endinput


