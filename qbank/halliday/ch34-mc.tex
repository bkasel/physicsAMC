
%%--------------------------------------------------
%% Halliday: Fundamentals of Physics
%%--------------------------------------------------


%% Chapter 34: Images
%%--------------------------------------------------


%% Learning Objectives
%%--------------------------------------------------

%% 34.01: Distinguish virtual images from real images.
%% 34.02: Explain the common roadway mirage.
%% 34.03: Sketch a ray diagram for the reflection of a point source of light by a plane mirror, indicating the object distance and image distance.
%% 34.04: Using the proper algebraic sign, relate the object distance $p$ to the image distance $i$.
%% 34.05: Give an example of the apparent hallway that you can see in a mirror maze based on equilateral triangles.


%% Halliday Multiple Choice Questions
%%--------------------------------------------------
\element{halliday-mc}{
\begin{question}{halliday-ch34-q01}
    A virtual image is one:
    \begin{choices}
        \wrongchoice{toward which light rays converge but do not pass through}
      \correctchoice{from which light rays diverge but do not pass through}
        \wrongchoice{from which light rays diverge as they pass through}
        \wrongchoice{toward which light rays converge and pass through}
        \wrongchoice{with a ray normal to a mirror passing through it}
    \end{choices}
\end{question}
}

\element{halliday-mc}{
\begin{question}{halliday-ch34-q02}
    Which of the following is true of all virtual images?
    \begin{choices}
        \wrongchoice{They can be seen but not photographed}
        \wrongchoice{They are ephemeral}
        \wrongchoice{They are smaller than the objects}
        \wrongchoice{They are larger than the objects}
      \correctchoice{None of the provided}
    \end{choices}
\end{question}
}

\element{halliday-mc}{
\begin{question}{halliday-ch34-q03}
    When you stand in front of a plane mirror,
        your image is:
    \begin{choices}
        \wrongchoice{real, erect, and smaller than you}
        \wrongchoice{real, erect, and the same size as you}
        \wrongchoice{virtual, erect, and smaller than you}
      \correctchoice{virtual, erect, and the same size as you}
        \wrongchoice{real, inverted, and the same size as you}
    \end{choices}
\end{question}
}

\element{halliday-mc}{
\begin{question}{halliday-ch34-q04}
    An object is \SI{2}{\meter} in front of a plane mirror.
    Its image is:
    \begin{choices}
        \wrongchoice{virtual, inverted, and \SI{2}{\meter} behind the mirror}
        \wrongchoice{virtual, inverted, and \SI{2}{\meter} in front of the mirror}
        \wrongchoice{virtual, erect, and \SI{2}{\meter} in front of the mirror}
        \wrongchoice{real, erect, and \SI{2}{\meter} behind the mirror}
      \correctchoice{none of the provided}
    \end{choices}
\end{question}
}

\element{halliday-mc}{
\begin{question}{halliday-ch34-q05}
    A ball is held \SI{50}{\centi\meter} in front of a plane mirror.
    The distance between the ball and its image is:
    \begin{multicols}{3}
    \begin{choices}
      \correctchoice{\SI{100}{\centi\meter}}
        \wrongchoice{\SI{150}{\centi\meter}}
        \wrongchoice{\SI{200}{\centi\meter}}
        \wrongchoice{zero}
        \wrongchoice{\SI{50}{\centi\meter}}
    \end{choices}
    \end{multicols}
\end{question}
}

\element{halliday-mc}{
\begin{question}{halliday-ch34-q06}
    A card marked IAHIO8 is standing upright in front of a plane mirror.
    Which of the following is \emph{not} true?
    \begin{choices}
        \wrongchoice{The image is virtual}
      \correctchoice{The image shifts its position as the observer shifts his position}
        \wrongchoice{The image appears as 8OIHAI to a person looking in the mirror}
        \wrongchoice{The image is caused mostly by specular rather than diffuse reflection}
        \wrongchoice{The image is the same size as the object}
    \end{choices}
\end{question}
}

\element{halliday-mc}{
\begin{question}{halliday-ch34-q07}
    The angle between a horizontal ruler and a vertical plane mirror is \ang{30}.
    The angle between the ruler and its image is:
    \begin{multicols}{3}
    \begin{choices}
        \wrongchoice{\ang{15}}
        \wrongchoice{\ang{30}}
      \correctchoice{\ang{60}}
        \wrongchoice{\ang{90}}
        \wrongchoice{\ang{180}}
    \end{choices}
    \end{multicols}
\end{question}
}

\element{halliday-mc}{
\begin{question}{halliday-ch34-q08}
    A \SI{5.0}{\foot} woman wishes to see a full length image of herself in a plane mirror.
    The minimum length mirror required is:
    %\begin{multicols}{2}
    \begin{choices}
        \wrongchoice{\SI{5}{\foot}}
        \wrongchoice{\SI{10}{\foot}}
      \correctchoice{\SI{2.5}{\foot}}
        \wrongchoice{\SI{3.54}{\foot}}
        \wrongchoice{variable: the farther away she stands the smaller the required mirror length}
    \end{choices}
    %\end{multicols}
\end{question}
}

\element{halliday-mc}{
\begin{question}{halliday-ch34-q09}
    A man holds a rectangular card in front of and parallel to a plane mirror.
    In order for him to see the entire image of the card,
        the least mirror area needed is:
    \begin{choices}
        \wrongchoice{that of the whole mirror, regardless of its size}
        \wrongchoice{that of the pupil of his eye}
        \wrongchoice{one-half that of the card}
      \correctchoice{one-fourth that of the card}
        \wrongchoice{an amount which decreases with his distance from the mirror}
    \end{choices}
\end{question}
}

\element{halliday-mc}{
\begin{question}{halliday-ch34-q10}
    A light bulb burns in front of the center of a \SI{40}{\centi\meter} wide plane mirror that is hung vertically on a wall.
    A man walks in front of the mirror along a line that is parallel to the mirror and twice as far from it as the bulb.
    The greatest distance he can walk and still see the image of the bulb is:
    \begin{multicols}{3}
    \begin{choices}
        \wrongchoice{\SI{20}{\centi\meter}}
        \wrongchoice{\SI{40}{\centi\meter}}
        \wrongchoice{\SI{60}{\centi\meter}}
        \wrongchoice{\SI{80}{\centi\meter}}
      \correctchoice{\SI{120}{\centi\meter}}
    \end{choices}
    \end{multicols}
\end{question}
}

\element{halliday-mc}{
\begin{question}{halliday-ch34-q11}
    A plane mirror is in a vertical plane and is rotating about a vertical axis at \SI{100}{\rotation\per\minute}. 
    A horizontal beam of light is incident on the mirror. 
    The reflected beam will rotate at:
    \begin{multicols}{2}
    \begin{choices}
        \wrongchoice{\SI{100}{\rotation\per\minute}}
        \wrongchoice{\SI{141}{\rotation\per\minute}}
        \wrongchoice{\SI{0}{\rotation\per\minute}}
      \correctchoice{\SI{200}{\rotation\per\minute}}
        \wrongchoice{\SI{10 000}{\rotation\per\minute}}
    \end{choices}
    \end{multicols}
\end{question}
}

\element{halliday-mc}{
\begin{question}{halliday-ch34-q12}
    A candle C sits between two parallel mirrors,
        a distance $0.2d$ from mirror 1. 
    \begin{center}
    \begin{tikzpicture}
        %% NOTE: tikz
    \end{tikzpicture}
    \end{center}
    Here $d$ is the distance between the mirrors. 
    Multiple images of the candle appear in both mirrors. 
    How far behind mirror 1 are the nearest three images of the candle in that mirror?
    \begin{multicols}{2}
    \begin{choices}
      \correctchoice{$0.2d$, $1.8d$, $2.2d$}
        \wrongchoice{$0.2d$, $2.2d$, $4.2d$}
        \wrongchoice{$0.2d$, $1.8d$, $3.8d$}
        \wrongchoice{$0.2d$, $0.8d$, $1.4d$}
        \wrongchoice{$0.2d$, $1.8d$, $3.4d$}
    \end{choices}
    \end{multicols}
\end{question}
}

\element{halliday-mc}{
\begin{question}{halliday-ch34-q13}
    Two plane mirrors make an angle of \ang{120} with each other.  
    The maximum number of images of an object placed between them is:
    \begin{multicols}{2}
    \begin{choices}
        \wrongchoice{one}
      \correctchoice{two}
        \wrongchoice{three}
        \wrongchoice{four}
        \wrongchoice{more than four}
    \end{choices}
    \end{multicols}
\end{question}
}

\element{halliday-mc}{
\begin{question}{halliday-ch34-q14}
    A parallel beam of monochromatic light in air is incident on a plane glass surface. 
    In the glass, the beam:
    \begin{choices}
      \correctchoice{remains parallel}
        \wrongchoice{undergoes dispersion}
        \wrongchoice{becomes diverging}
        \wrongchoice{follows a parabolic path}
        \wrongchoice{becomes converging}
    \end{choices}
\end{question}
}

\element{halliday-mc}{
\begin{question}{halliday-ch34-q15}
    The focal length of a spherical mirror is $N$ times its radius of curvature where $N$ is:
    \begin{multicols}{3}
    \begin{choices}
        \wrongchoice{\num{1/4}}
      \correctchoice{\num{1/2}}
        \wrongchoice{1}
        \wrongchoice{2}
        \wrongchoice{4}
    \end{choices}
    \end{multicols}
\end{question}
}

\element{halliday-mc}{
\begin{question}{halliday-ch34-q16}
    Real images formed by a spherical mirror are always:
    \begin{choices}
        \wrongchoice{on the side of the mirror opposite the source}
        \wrongchoice{on the same side of the mirror as the source but closer to the mirror than the source}
        \wrongchoice{on the same side of the mirror as the source but closer to the mirror than the focal point}
        \wrongchoice{on the same side of the mirror as the source but further from the mirror than the focal point}
      \correctchoice{none of the provided}
    \end{choices}
\end{question}
}

\element{halliday-mc}{
\begin{question}{halliday-ch34-q17}
    The image produced by a convex mirror of an erect object in front of the mirror is always:
    \begin{choices}
        \wrongchoice{virtual, erect, and larger than the object}
      \correctchoice{virtual, erect, and smaller than the object}
        \wrongchoice{real, erect, and larger than the object}
        \wrongchoice{real, erect, and smaller than the object}
        \wrongchoice{none of the provided}
    \end{choices}
\end{question}
}

\element{halliday-mc}{
\begin{question}{halliday-ch34-q18}
    An erect object is located between a concave mirror and its focal point. 
    Its image is:
    \begin{choices}
        \wrongchoice{real, erect, and larger than the object}
        \wrongchoice{real, inverted, and larger than the object}
      \correctchoice{virtual, erect, and larger than the object}
        \wrongchoice{virtual, inverted, and larger than the object}
        \wrongchoice{virtual, erect, and smaller than the object}
    \end{choices}
\end{question}
}

\element{halliday-mc}{
\begin{question}{halliday-ch34-q19}
    An erect object is in front of a convex mirror a distance greater than the focal length. 
    The image is:
    \begin{choices}
        \wrongchoice{real, inverted, and smaller than the object}
        \wrongchoice{virtual, inverted, and larger than the object}
        \wrongchoice{real, inverted, and larger than the object}
      \correctchoice{virtual, erect, and smaller than the object}
        \wrongchoice{real, erect, and larger than the object}
    \end{choices}
\end{question}
}

\element{halliday-mc}{
\begin{question}{halliday-ch34-q20}
    As an object is moved from the center of curvature of a concave mirror toward its focal point its image:
    \begin{choices}
        \wrongchoice{remains virtual and becomes larger}
        \wrongchoice{remains virtual and becomes smaller}
      \correctchoice{remains real and becomes larger}
        \wrongchoice{remains real and becomes smaller}
        \wrongchoice{remains real and approaches the same size as the object}
    \end{choices}
\end{question}
}

\element{halliday-mc}{
\begin{question}{halliday-ch34-q21}
    As an object is moved from a distant location toward the center of curvature of a concave mirror its image:
    \begin{choices}
        \wrongchoice{remains virtual and becomes smaller}
        \wrongchoice{remains virtual and becomes larger}
        \wrongchoice{remains real and becomes smaller}
      \correctchoice{remains real and becomes larger}
        \wrongchoice{changes from real to virtual}
    \end{choices}
\end{question}
}

\element{halliday-mc}{
\begin{question}{halliday-ch34-q22}
    The image of an erect candle,
        formed using a convex mirror,
        is always:
    \begin{choices}
        \wrongchoice{virtual, inverted, and smaller than the candle}
        \wrongchoice{virtual, inverted, and larger than the candle}
        \wrongchoice{virtual, erect, and larger than the candle}
      \correctchoice{virtual, erect, and smaller than the candle}
        \wrongchoice{real, erect, and smaller than the candle}
    \end{choices}
\end{question}
}

\element{halliday-mc}{
\begin{question}{halliday-ch34-q23}
    At what distance in front of a concave mirror must an object be placed so that the image and object are the same size?
    \begin{choices}
        \wrongchoice{a focal length}
      \correctchoice{half a focal length}
        \wrongchoice{twice a focal length}
        \wrongchoice{less than half focal length}
        \wrongchoice{more than twice a focal length}
    \end{choices}
\end{question}
}

\element{halliday-mc}{
\begin{question}{halliday-ch34-q24}
    A point source is to be used with a concave mirror to produce a beam of parallel light. 
    The source should be placed:
    \begin{choices}
        \wrongchoice{as close to the mirror as possible}
        \wrongchoice{at the center of curvature}
        \wrongchoice{midway between the center of curvature and the focal point}
      \correctchoice{midway between the center of curvature and the mirror}
        \wrongchoice{midway between the focal point and the mirror}
    \end{choices}
\end{question}
}

\element{halliday-mc}{
\begin{question}{halliday-ch34-q25}
    A concave mirror forms a real image that is twice the size of the object. 
    If the object is \SI{20}{\centi\meter} from the mirror,
        the radius of curvature of the mirror must be about:
    \begin{multicols}{3}
    \begin{choices}
        \wrongchoice{\SI{13}{\centi\meter}}
        \wrongchoice{\SI{20}{\centi\meter}}
      \correctchoice{\SI{27}{\centi\meter}}
        \wrongchoice{\SI{40}{\centi\meter}}
        \wrongchoice{\SI{80}{\centi\meter}}
    \end{choices}
    \end{multicols}
\end{question}
}

\element{halliday-mc}{
\begin{question}{halliday-ch34-q26}
    A man stands with his nose \SI{8}{\centi\meter} from a concave shaving mirror of radius \SI{32}{\centi\meter}.
    The distance from the mirror to the image of his nose is:
    \begin{multicols}{3}
    \begin{choices}
        \wrongchoice{\SI{8}{\centi\meter}}
        \wrongchoice{\SI{12}{\centi\meter}}
      \correctchoice{\SI{16}{\centi\meter}}
        \wrongchoice{\SI{24}{\centi\meter}}
        \wrongchoice{\SI{32}{\centi\meter}}
    \end{choices}
    \end{multicols}
\end{question}
}

\element{halliday-mc}{
\begin{question}{halliday-ch34-q27}
    The figure shows a concave mirror with a small object located at the point marked $6$. 
    \begin{center}
    \begin{tikzpicture}
        %% NOTE: tikz
    \end{tikzpicture}
    \end{center}
    If the image is also at this point,
        then the center of curvature of the mirror is at the point marked:
    \begin{multicols}{3}
    \begin{choices}
        \wrongchoice{\num{3}}
        \wrongchoice{\num{4}}
      \correctchoice{\num{6}}
        \wrongchoice{\num{9}}
        \wrongchoice{\num{12}}
    \end{choices}
    \end{multicols}
\end{question}
}

\element{halliday-mc}{
\begin{question}{halliday-ch34-q28}
    A concave spherical mirror has a focal length of \SI{12}{\centi\meter}.
    If an object is placed \SI{6}{\centi\meter} in front of it the image position is:
    \begin{choices}
        \wrongchoice{\SI{4}{\centi\meter} behind the mirror}
        \wrongchoice{\SI{4}{\centi\meter} in front of the mirror}
      \correctchoice{\SI{12}{\centi\meter} behind the mirror}
        \wrongchoice{\SI{12}{\centi\meter} in front of the mirror}
        \wrongchoice{at infinity}
    \end{choices}
\end{question}
}

\element{halliday-mc}{
\begin{question}{halliday-ch34-q29}
    A concave spherical mirror has a focal length of \SI{12}{\centi\meter}.
    If an object is placed \SI{18}{\centi\meter} in front of it the image position is:
    \begin{choices}
        \wrongchoice{\SI{7.2}{\centi\meter} behind the mirror}
        \wrongchoice{\SI{7.2}{\centi\meter} in front of the mirror}
        \wrongchoice{\SI{36}{\centi\meter} behind the mirror}
      \correctchoice{\SI{36}{\centi\meter} in front of the mirror}
        \wrongchoice{at infinity}
    \end{choices}
\end{question}
}

\element{halliday-mc}{
\begin{question}{halliday-ch34-q30}
    A convex spherical mirror has a focal length of \SI{12}{\centi\meter}.
    If an object is placed \SI{6}{\centi\meter} in front of it the image position is:
    \begin{choices}
      \correctchoice{\SI{4}{\centi\meter} behind the mirror}
        \wrongchoice{\SI{4}{\centi\meter} in front of the mirror}
        \wrongchoice{\SI{12}{\centi\meter} behind the mirror}
        \wrongchoice{\SI{12}{\centi\meter} in front of the mirror}
        \wrongchoice{at infinity}
    \end{choices}
\end{question}
}

\element{halliday-mc}{
\begin{question}{halliday-ch34-q31}
    A concave spherical mirror has a focal length of \SI{12}{\centi\meter}.
    If an erect object is placed \SI{6}{\centi\meter} in front of it:
    \begin{choices}
      \correctchoice{the magnification is 2 and the image is erect}
        \wrongchoice{the magnification is 2 and the image is inverted}
        \wrongchoice{the magnification is 0.67 and the image is erect}
        \wrongchoice{the magnification is 0.67 and the image is inverted}
        \wrongchoice{the magnification is 0.5 and the image is erect}
    \end{choices}
\end{question}
}

\element{halliday-mc}{
\begin{question}{halliday-ch34-q32}
    An erect object is located on the central axis of a spherical mirror. 
    The magnification is $-3$.
    This means:
    \begin{choices}
      \correctchoice{its image is real, inverted, and on the same side of the mirror}
        \wrongchoice{its image is virtual, erect, and on the opposite side of the mirror}
        \wrongchoice{its image is real, erect, and on the same side of the mirror}
        \wrongchoice{its image is real, inverted, and on the opposite side of the mirror}
        \wrongchoice{its image is virtual, inverted, and on the opposite side of the mirror}
    \end{choices}
\end{question}
}

\element{halliday-mc}{
\begin{question}{halliday-ch34-q33}
    An object $O$, in air, is in front of the concave spherical refracting surface of a piece of glass.
    Which of the general situations depicted below is like this situation?
    \begin{multicols}{2}
    \begin{choices}
        %% NOTE: ANS is C
        \wrongchoice{
            \begin{tikzpicture}
                %% NOTE:
            \end{tikzpicture}
        }
    \end{choices}
    \end{multicols}
\end{question}
}

\element{halliday-mc}{
\begin{question}{halliday-ch34-q34}
    A concave refracting surface is one with a center of curvature:
    \begin{choices}
        \wrongchoice{to the left of the surface}
        \wrongchoice{to the right of the surface}
      \correctchoice{on the side of the incident light}
        \wrongchoice{on the side of the refracted light}
        \wrongchoice{on the side with the higher index of refraction}
    \end{choices}
\end{question}
}

\element{halliday-mc}{
\begin{question}{halliday-ch34-q35}
    A convex refracting surface has a radius of \SI{12}{\centi\meter}.
    Light is incident in air ($n=1$) and is refracted into a medium with an index of refraction of 2. 
    Light incident parallel to the central axis is focused at a point:
    \begin{choices}
        \wrongchoice{\SI{3}{\centi\meter} from the surface}
        \wrongchoice{\SI{6}{\centi\meter} from the surface}
        \wrongchoice{\SI{12}{\centi\meter} from the surface}
        \wrongchoice{\SI{18}{\centi\meter} from the surface}
      \correctchoice{\SI{24}{\centi\meter} from the surface}
    \end{choices}
\end{question}
}

\element{halliday-mc}{
\begin{question}{halliday-ch34-q36}
    A convex refracting surface has a radius of \SI{12}{\centi\meter}.
    Light is incident in air ($n=1$) and refracted into a medium with an index of refraction of 2. 
    To obtain light with rays parallel to the central axis after refraction a point source should be placed on the axis:
    \begin{choices}
        \wrongchoice{\SI{3}{\centi\meter} from the surface}
        \wrongchoice{\SI{6}{\centi\meter} from the surface}
      \correctchoice{\SI{12}{\centi\meter} from the surface}
        \wrongchoice{\SI{18}{\centi\meter} from the surface}
        \wrongchoice{\SI{24}{\centi\meter} from the surface}
    \end{choices}
\end{question}
}

\element{halliday-mc}{
\begin{question}{halliday-ch34-q37}
    A concave refracting surface of a medium with index of refraction $n$ produces a real image no matter where an object is placed outside:
    \begin{choices}
        \wrongchoice{always}
        \wrongchoice{only if the index of refraction of the surrounding medium is less than $n$}
        \wrongchoice{only if the index of refraction of the surrounding medium is greater than $n$}
        \wrongchoice{never}
      \correctchoice{none of the provided}
    \end{choices}
\end{question}
}

\element{halliday-mc}{
\begin{question}{halliday-ch34-q38}
    A convex spherical refracting surface separates a medium with index of refraction 2 from air.
    The image of an object outside the surface is real:
    \begin{choices}
        \wrongchoice{always}
        \wrongchoice{never}
        \wrongchoice{only if it is close to the surface}
      \correctchoice{only if it is far from the surface}
        \wrongchoice{only if the radius of curvature is small}
    \end{choices}
\end{question}
}

\element{halliday-mc}{
\begin{question}{halliday-ch34-q39}
    A convex spherical surface with radius $r$ separates a medium with index of refraction 2 from air. 
    As an object is moved toward the surface from far away along the central axis,
        its image:
    \begin{choices}
        \wrongchoice{changes from virtual to real when it is $r/2$ from the surface}
        \wrongchoice{changes from virtual to real when it is $r$ from the surface}
        \wrongchoice{changes from real to virtual when it is $r/2$ from the surface}
      \correctchoice{changes from real to virtual when it is $r$ from the surface}
        \wrongchoice{remains real}
    \end{choices}
\end{question}
}

\element{halliday-mc}{
\begin{question}{halliday-ch34-q40}
    A concave spherical surface with radius $r$ separates a medium with index of refraction 2 from air. 
    As an object is moved toward the surface from far away along the central axis,
        its image:
    \begin{choices}
        \wrongchoice{changes from virtual to real when it is $r/2$ from the surface}
        \wrongchoice{changes from virtual to real when it is $2r$ from the surface}
        \wrongchoice{changes from real to virtual when it is $r/2$ from the surface}
        \wrongchoice{changes from real to virtual when it is $2r$ from the surface}
      \correctchoice{remains virtual}
    \end{choices}
\end{question}
}

\element{halliday-mc}{
\begin{question}{halliday-ch34-q41}
    An erect object is placed on the central axis of a thin lens,
        further from the lens than the magnitude of its focal length. 
    The magnification is $+0.4$. 
    This means:
    \begin{choices}
        \wrongchoice{the image is real and erect and the lens is a converging lens}
        \wrongchoice{the image is real and inverted and the lens is a converging lens}
      \correctchoice{the image is virtual and erect, and the lens is a diverging lens}
        \wrongchoice{the image is virtual and erect, and the lens is a converging lens}
        \wrongchoice{the image is virtual and inverted and the lens is a diverging lens}
    \end{choices}
\end{question}
}

\element{halliday-mc}{
\begin{question}{halliday-ch34-q42}
    Where must an object be placed in front of a converging lens in order to obtain a virtual image?
    \begin{choices}
        \wrongchoice{At the focal point}
        \wrongchoice{At twice the focal length}
        \wrongchoice{Greater than the focal length}
      \correctchoice{Between the focal point and the lens}
        \wrongchoice{Between the focal length and twice the focal length}
    \end{choices}
\end{question}
}

\element{halliday-mc}{
\begin{question}{halliday-ch34-q43}
    An erect object placed outside the focal point of a converging lens will produce an image that is:
    \begin{choices}
        \wrongchoice{erect and virtual}
        \wrongchoice{inverted and virtual}
        \wrongchoice{erect and real}
      \correctchoice{inverted and real}
        \wrongchoice{impossible to locate}
    \end{choices}
\end{question}
}

\element{halliday-mc}{
\begin{question}{halliday-ch34-q44}
    An object is \SI{30}{\centi\meter} in front of a converging lens of focal length \SI{10}{\centi\meter}.
    The image is:
    \begin{choices}
        \wrongchoice{real and larger than the object}
        \wrongchoice{real and the same size than the object}
      \correctchoice{real and smaller than the object}
        \wrongchoice{virtual and the same size than the object}
        \wrongchoice{virtual and smaller than the object}
    \end{choices}
\end{question}
}

\element{halliday-mc}{
\begin{question}{halliday-ch34-q45}
    Let $p$ denote the object-lens distance and $i$ the image-lens distance. 
    The image produced by a lens of focal length $f$ has a height that can be obtained from the object height by multiplying it by:
    \begin{multicols}{3}
    \begin{choices}
        \wrongchoice{$\dfrac{p}{i}$}
      \correctchoice{$\dfrac{i}{p}$}
        \wrongchoice{$\dfrac{f}{p}$}
        \wrongchoice{$\dfrac{f}{i}$}
        \wrongchoice{$\dfrac{i}{f}$}
    \end{choices}
    \end{multicols}
\end{question}
}

\element{halliday-mc}{
\begin{question}{halliday-ch34-q46}
    A camera with a lens of focal length \SI{6.0}{\centi\meter} takes a picture of a \SI{1.4}{\meter} tall man standing \SI{11}{\meter} away. 
    The height of the image is about:
    \begin{multicols}{3}
    \begin{choices}
        \wrongchoice{\SI{0.39}{\centi\meter}}
      \correctchoice{\SI{0.77}{\centi\meter}}
        \wrongchoice{\SI{1.5}{\centi\meter}}
        \wrongchoice{\SI{3.0}{\centi\meter}}
        \wrongchoice{\SI{6.0}{\centi\meter}}
    \end{choices}
    \end{multicols}
\end{question}
}

\element{halliday-mc}{
\begin{question}{halliday-ch34-q47}
    A hollow lens is made of thin glass, as shown. 
    It can be filled with air, water ($n=1.3$) or \ce{CS2} ($n=1.6$). 
    \begin{center}
    \begin{tikzpicture}
        %% NOTE: tikz
    \end{tikzpicture}
    \end{center}
    The lens will diverge a beam of parallel light if it is filled with:
    \begin{choices}
        \wrongchoice{air and immersed in air}
        \wrongchoice{air and immersed in water}
        \wrongchoice{water and immersed in \ce{CS2}}
      \correctchoice{\ce{CS2} and immersed in water}
        \wrongchoice{\ce{CS2} and immersed in \ce{CS2}}
    \end{choices}
\end{question}
}

\element{halliday-mc}{
\begin{question}{halliday-ch34-q48}
    The object-lens distance for a certain converging lens is \SI{400}{\milli\meter}.
    The image is three times the size of the object. 
    To make the image five times the size of the object-lens distance must be changed to:
    \begin{multicols}{3}
    \begin{choices}
      \correctchoice{\SI{360}{\milli\meter}}
        \wrongchoice{\SI{540}{\milli\meter}}
        \wrongchoice{\SI{600}{\milli\meter}}
        \wrongchoice{\SI{720}{\milli\meter}}
        \wrongchoice{\SI{960}{\milli\meter}}
    \end{choices}
    \end{multicols}
\end{question}
}

\element{halliday-mc}{
\begin{question}{halliday-ch34-q49}
    An erect object is $2f$ in front of a converging lens of focal length $f$. 
    The image is:
    \begin{choices}
        \wrongchoice{real, inverted, magnified}
        \wrongchoice{real, erect, same size}
      \correctchoice{real, inverted, same size}
        \wrongchoice{virtual, inverted, reduced}
        \wrongchoice{real, inverted, reduced}
    \end{choices}
\end{question}
}

\element{halliday-mc}{
\begin{question}{halliday-ch34-q50}
    An ordinary magnifying glass in front of an erect object produces an image that is:
    \begin{choices}
        \wrongchoice{real and erect}
        \wrongchoice{real and inverted}
        \wrongchoice{virtual and inverted}
      \correctchoice{virtual and erect}
        \wrongchoice{none of these}
    \end{choices}
\end{question}
}

\element{halliday-mc}{
\begin{question}{halliday-ch34-q51}
    The Sun subtends \ang{0.5} as seen from Earth. 
    The diameter of its image,
        using a \SI{1.0}{\meter} focal length lens, is about:
    \begin{multicols}{3}
    \begin{choices}
        \wrongchoice{\SI{10}{\centi\meter}}
        \wrongchoice{\SI{2}{\centi\meter}}
      \correctchoice{\SI{1}{\centi\meter}}
        \wrongchoice{\SI{5}{\milli\meter}}
        \wrongchoice{\SI{1}{\milli\meter}}
    \end{choices}
    \end{multicols}
\end{question}
}

\element{halliday-mc}{
\begin{question}{halliday-ch34-q52}
    An object is in front of a converging lens,
        at a distance less than the focal length from the lens.
    Its image is:
    \begin{choices}
      \correctchoice{virtual and larger than the object}
        \wrongchoice{real and smaller than the object}
        \wrongchoice{virtual and smaller than the object}
        \wrongchoice{real and larger than the object}
        \wrongchoice{virtual and the same size as the object}
    \end{choices}
\end{question}
}

\element{halliday-mc}{
\begin{question}{halliday-ch34-q53}
    A plano-convex glass ($n=1.5$) lens has a curved side whose radius is \SI{50}{\centi\meter}.
    If the image size is to be the same as the object size,
        the object should be placed at a distance from the lens of:
    \begin{multicols}{3}
    \begin{choices}
        \wrongchoice{\SI{50}{\centi\meter}}
        \wrongchoice{\SI{100}{\centi\meter}}
      \correctchoice{\SI{200}{\centi\meter}}
        \wrongchoice{\SI{400}{\centi\meter}}
        \wrongchoice{\SI{340}{\centi\meter}}
    \end{choices}
    \end{multicols}
\end{question}
}

\element{halliday-mc}{
\begin{question}{halliday-ch34-q54}
    Which of the following five glass lenses is a diverging lens?
    \begin{multicols}{2}
    \begin{choices}
        %% NOTE: ANS is A
        \wrongchoice{
            \begin{tikzpicture}
                %% NOTE:
            \end{tikzpicture}
        }
    \end{choices}
    \end{multicols}
\end{question}
}

\element{halliday-mc}{
\begin{question}{halliday-ch34-q55}
    The bellows of an adjustable camera can be extended so that the largest film to lens distance is one and one-half times the focal length. 
    If the focal length is \SI{12}{\centi\meter},
        the nearest object that can be sharply focused on the film must be what distance from the lens?
    \begin{multicols}{3}
    \begin{choices}
        \wrongchoice{\SI{12}{\centi\meter}}
        \wrongchoice{\SI{24}{\centi\meter}}
      \correctchoice{\SI{36}{\centi\meter}}
        \wrongchoice{\SI{48}{\centi\meter}}
        \wrongchoice{\SI{72}{\centi\meter}}
    \end{choices}
    \end{multicols}
\end{question}
}

\element{halliday-mc}{
\begin{question}{halliday-ch34-q56}
    A \SI{3}{\centi\meter} high object is in front of a thin lens. 
    The object distance is \SI{4}{\centi\meter} and the image distance is \SI{-8}{\centi\meter}.
    The image height is:
    \begin{multicols}{3}
    \begin{choices}
        \wrongchoice{\SI{0.5}{\centi\meter}}
        \wrongchoice{\SI{1}{\centi\meter}}
        \wrongchoice{\SI{1.5}{\centi\meter}}
      \correctchoice{\SI{6}{\centi\meter}}
        \wrongchoice{\SI{24}{\centi\meter}}
    \end{choices}
    \end{multicols}
\end{question}
}

\element{halliday-mc}{
\begin{question}{halliday-ch34-q57}
    When a single-lens camera is focused on a distant object,
        the lens-to-film distance is found to be \SI{40.0}{\milli\meter}. 
    To focus on an object \SI{0.54}{\meter} in front of the lens,
        the film-to-lens distance should be:
    \begin{multicols}{3}
    \begin{choices}
        \wrongchoice{\SI{40.0}{\milli\meter}}
        \wrongchoice{\SI{37.3}{\milli\meter}}
        \wrongchoice{\SI{36.8}{\milli\meter}}
        \wrongchoice{\SI{42.7}{\milli\meter}}
      \correctchoice{\SI{43.2}{\milli\meter}}
    \end{choices}
    \end{multicols}
\end{question}
}

\element{halliday-mc}{
\begin{question}{halliday-ch34-q58}
    In a cinema,
        a picture \SI{2.5}{\centi\meter} wide on the film is projected to an image \SI{3.0}{\meter} wide on a screen that is \SI{18}{\meter} away. 
    The focal length of the lens is about:
    \begin{multicols}{3}
    \begin{choices}
        \wrongchoice{\SI{7.5}{\centi\meter}}
        \wrongchoice{\SI{10}{\centi\meter}}
        \wrongchoice{\SI{12.5}{\centi\meter}}
      \correctchoice{\SI{15}{\centi\meter}}
        \wrongchoice{\SI{20}{\centi\meter}}
    \end{choices}
    \end{multicols}
\end{question}
}

\element{halliday-mc}{
\begin{question}{halliday-ch34-q59}
    The term ``virtual'' as applied to an image made by a mirror means that the image:
    \begin{choices}
        \wrongchoice{is on the mirror surface}
        \wrongchoice{cannot be photographed by a camera}
        \wrongchoice{is in front of the mirror}
        \wrongchoice{is the same size as the object}
      \correctchoice{cannot be shown directly on a screen}
    \end{choices}
\end{question}
}

\element{halliday-mc}{
\begin{question}{halliday-ch34-q60}
    Which instrument uses a single converging lens with the object placed just inside the focal point?
    \begin{choices}
        \wrongchoice{Camera}
        \wrongchoice{Compound microscope}
      \correctchoice{Magnifying glass}
        \wrongchoice{Overhead projector}
        \wrongchoice{Telescope}
    \end{choices}
\end{question}
}

\element{halliday-mc}{
\begin{question}{halliday-ch34-q61}
    Let $f_o$ and $f_e$ be the focal lengths of the objective and eyepiece of a compound microscope. 
    In ordinary use, the object:
    \begin{choices}
        \wrongchoice{is less than $f_o$ from the objective lens}
      \correctchoice{is more that $f_o$ from the objective}
        \wrongchoice{produces an intermediate image that is slightly more than $f_e$ from the eyepiece}
        \wrongchoice{produces an intermediate image that is $2f_e$ away from the eyepiece}
        \wrongchoice{produces an intermediate image that is less than $f_o$ from the objective lens}
    \end{choices}
\end{question}
}

\element{halliday-mc}{
\begin{question}{halliday-ch34-q62}
    Consider the following four statements concerning a compound microscope:
    \begin{enumerate}
        \item Each lens produces an image that is virtual and inverted.
        \item The objective lens has a very short focal length.
        \item The eyepiece is used as a simple magnifying glass.
        \item The objective lens is convex and the eyepiece is concave.
    \end{enumerate}
    Which two of the four statements are correct?
    \begin{multicols}{3}
    \begin{choices}
        %% NOTE: questionmult??
        \wrongchoice{1, 2}
        \wrongchoice{1, 3}
        \wrongchoice{1, 4}
      \correctchoice{2, 3}
        \wrongchoice{2, 4}
    \end{choices}
    \end{multicols}
\end{question}
}

\element{halliday-mc}{
\begin{question}{halliday-ch34-q63}
    What type of eyeglasses should a nearsighted person wear?
    \begin{choices}
      \correctchoice{diverging lenses}
        \wrongchoice{bifocal lenses}
        \wrongchoice{converging lenses}
        \wrongchoice{plano-convex lenses}
        \wrongchoice{double convex lenses}
    \end{choices}
\end{question}
}

\element{halliday-mc}{
\begin{question}{halliday-ch34-q64}
    Which of the following is \emph{not} correct for a simple magnifying glass?
    \begin{choices}
        \wrongchoice{The image is virtual}
        \wrongchoice{The image is erect}
        \wrongchoice{The image is larger than the object}
        \wrongchoice{The object is inside the focal point}
      \correctchoice{The lens is diverging}
    \end{choices}
\end{question}
}

\element{halliday-mc}{
\begin{question}{halliday-ch34-q65}
    A nearsighted person can see clearly only objects within \SI{1.4}{\meter} of her eye. 
    To see distant objects,
        she should wear eyeglasses of what type and focal length?
    \begin{choices}
        \wrongchoice{diverging, \SI{2.8}{\meter}}
      \correctchoice{diverging, \SI{1.4}{\meter}}
        \wrongchoice{converging, \SI{2.8}{\meter}}
        \wrongchoice{converging, \SI{1.4}{\meter}}
        \wrongchoice{diverging, \SI{0.72}{\meter}}
    \end{choices}
\end{question}
}

\element{halliday-mc}{
\begin{question}{halliday-ch34-q66}
    A magnifying glass has a focal length of \SI{15}{\centi\meter}.
    If the near point of the eye is \SI{25}{\centi\meter} from the eye the angular magnification of the glass is about:
    \begin{multicols}{3}
    \begin{choices}
        \wrongchoice{\num{0.067}}
        \wrongchoice{\num{0.33}}
        \wrongchoice{\num{0.67}}
      \correctchoice{\num{1.7}}
        \wrongchoice{\num{15}}
    \end{choices}
    \end{multicols}
\end{question}
}

\element{halliday-mc}{
\begin{question}{halliday-ch34-q67}
    An object is \SI{20}{\centi\meter} to the left of a lens of focal length \SI{+10}{\centi\meter}.
    A second lens,
        of focal length \SI{+12.5}{\centi\meter},
        is \SI{30}{\centi\meter} to the right of the first lens. 
    The distance between the original object and the final image is:
    \begin{multicols}{3}
    \begin{choices}
        \wrongchoice{\SI{28}{\centi\meter}}
        \wrongchoice{\SI{50}{\centi\meter}}
        \wrongchoice{\SI{100}{\centi\meter}}
      \correctchoice{zero}
        \wrongchoice{infinity}
    \end{choices}
    \end{multicols}
\end{question}
}

\element{halliday-mc}{
\begin{question}{halliday-ch34-q68}
    A converging lens of focal length \SI{20}{\centi\meter} is placed in contact with a converging lens of focal length \SI{30}{\centi\meter}. 
    The focal length of this combination is:
    \begin{multicols}{2}
    \begin{choices}
      \correctchoice{\SI{+10}{\centi\meter}}
        \wrongchoice{\SI{-10}{\centi\meter}}
        \wrongchoice{\SI{+60}{\centi\meter}}
        \wrongchoice{\SI{-60}{\centi\meter}}
        \wrongchoice{\SI{+25}{\centi\meter}}
    \end{choices}
    \end{multicols}
\end{question}
}

\element{halliday-mc}{
\begin{question}{halliday-ch34-q69}
    A student sets the cross-hairs of an eyepiece in line with an image that he is measuring. 
    He then notes that when he moves his head slightly to the right,
        the image moves slightly to the left (with respect to the cross-hairs). 
    Therefore the image is:
    \begin{choices}
        \wrongchoice{infinitely far away}
        \wrongchoice{farther away from him that the cross-hairs}
      \correctchoice{nearer to him than the cross-hairs}
        \wrongchoice{in the focal plane of the eyepiece}
        \wrongchoice{in the plane of the cross-hairs}
    \end{choices}
\end{question}
}

\element{halliday-mc}{
\begin{question}{halliday-ch34-q70}
    In a two lens microscope,
        the intermediate image is:
    \begin{choices}
        \wrongchoice{virtual, erect, and magnified}
        \wrongchoice{real, erect, and magnified}
      \correctchoice{real, inverted, and magnified}
        \wrongchoice{virtual, inverted, and reduced}
        \wrongchoice{virtual, inverted, and magnified}
    \end{choices}
\end{question}
}

\element{halliday-mc}{
\begin{question}{halliday-ch34-q71}
    Two thin lenses (focal lengths $f_1$ and $f_2$) are in contact. 
    Their equivalent focal length is:
    \begin{multicols}{2}
    \begin{choices}
        \wrongchoice{$f_1 + f_2$}
      \correctchoice{$\dfrac{f_1 f_2}{f_1+f_2}$}
        \wrongchoice{$\dfrac{1}{f_1} + \dfrac{1}{f_2}$}
        \wrongchoice{$f_1 - f_2$}
        \wrongchoice{$\dfrac{f_1\left(f_1-f_2\right)}{f_2}$}
    \end{choices}
    \end{multicols}
\end{question}
}

\element{halliday-mc}{
\begin{question}{halliday-ch34-q72}
    The two lenses shown are illuminated by a beam of parallel light from the left. 
    \begin{center}
    \begin{tikzpicture}
        %% NOTE: tikz
    \end{tikzpicture}
    \end{center}
    Lens $B$ is then moved slowly toward lens $A$. 
    The beam emerging from lens $B$ is:
    \begin{choices}
      \correctchoice{initially parallel and then diverging}
        \wrongchoice{always diverging}
        \wrongchoice{initially converging and finally parallel}
        \wrongchoice{always parallel}
        \wrongchoice{initially converging and finally diverging}
    \end{choices}
\end{question}
}


\endinput


