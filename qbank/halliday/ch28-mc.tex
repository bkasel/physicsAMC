
%%--------------------------------------------------
%% Halliday: Fundamentals of Physics
%%--------------------------------------------------


%% Chapter 28: Magnetic Fields
%%--------------------------------------------------


%% Learning Objectives
%%--------------------------------------------------

%% 28.01: Distinguish an electromagnet from a permanent magnet.
%% 28.02: Identify that a magnetic field is a vector quantity and thus has both magnitude and direction.
%% 28.03: Explain how a magnetic field can be defined in terms of what happens to a charged particle moving through the field.
%% 28.04: For a charged particle moving through a uniform magnetic field, apply the relationship between force magnitude $F_B$, charge $q$, speed $v$, field magnitude $B$, and the angle $\phi$ between the directions of the velocity vector $\vec{v}$ and the magnetic field vector $\vec{B}$ .
%% 28.05: For a charged particle sent through a uniform magnetic field, find the direction of the magnetic force $\vec{F}_B$ by (1) applying the right-hand rule to find the direction of the cross product $\vec{v}\times\vec{B}$ and (2) determining what effect the charge q has on the direction.
%% 28.06: Find the magnetic force $\vec{F}_B$ acting on a moving charged particle by evaluating the cross product $q\left(\vec{v}\times\vec{B}\right)$ in unit-vector notation and magnitude-angle notation.
%% 28.07: Identify that the magnetic force vector $\vec{F}_B$ must always be perpendicular to both the velocity vector $\vec{v}$ and the magnetic field vector $\vec{B}$.
%% 28.08: Identify the effect of the magnetic force on the particle's speed and kinetic energy.
%% 28.09: Identify a magnet as being a magnetic dipole.
%% 28.10: Identify that opposite magnetic poles attract each other and like magnetic poles repel each other.
%% 28.11: Explain magnetic field lines, including where they originate and terminate and what their spacing represents.


%% Halliday Multiple Choice Questions
%%--------------------------------------------------
\element{halliday-mc}{
\begin{question}{halliday-ch28-q01}
    Units of a magnetic field might be:
    \begin{choices}
        \wrongchoice{coulomb meter per second (\si{\coulomb\meter\per\second})}
        \wrongchoice{coulomb second per meter (\si{\coulomb\second\per\meter})}
        \wrongchoice{coulomb per kilogram (\si{\coulomb\per\kilo\gram})}
      \correctchoice{kilogram per couloumb per second (\si{\kilo\gram\per\coulomb\per\second})}
        \wrongchoice{newton per coulomb per meter (\si{\newton\per\coulomb\per\meter})}
    \end{choices}
\end{question}
}

\element{halliday-mc}{
\begin{question}{halliday-ch28-q02}
    In the formula $\vec{F} = q\vec{v}\times\vec{B}$:
    %% NOTE: bold vec
    \begin{choices}
        \wrongchoice{$\vec{F}$ must be perpendicular to $\vec{v}$ but not necessarily to $\vec{B}$}
        \wrongchoice{$\vec{F}$ must be perpendicular to $\vec{B}$ but not necessarily to $\vec{v}$}
        \wrongchoice{$\vec{v}$ must be perpendicular to $\vec{B}$ but not necessarily to $\vec{F}$}
        \wrongchoice{all three vectors must be mutually perpendicular}
      \correctchoice{$\vec{F}$ must be perpendicular to both $\vec{v}$ and $\vec{B}$}
    \end{choices}
\end{question}
}

\element{halliday-mc}{
\begin{question}{halliday-ch28-q03}
    An electron moves in the negative $x$ direction,
        through a uniform magnetic field in the negative $y$ direction.
    \begin{center}
    \begin{tikzpicture}
        %% NOTE: tikz
    \end{tikzpicture}
    \end{center}
    The  magnetic force on the electron is:
    \begin{choices}
        \wrongchoice{in the negative $x$ direction}
        \wrongchoice{in the positive $y$ direction}
        \wrongchoice{in the negative $y$ direction}
        \wrongchoice{in the positive $z$ direction}
      \correctchoice{in the negative $z$ direction}
    \end{choices}
\end{question}
}

\element{halliday-mc}{
\begin{question}{halliday-ch28-q04}
    At any point the magnetic field lines are in the direction of:
    \begin{choices}
        \wrongchoice{the magnetic force on a moving positive charge}
        \wrongchoice{the magnetic force on a moving negative charge}
        \wrongchoice{the velocity of a moving positive charge}
        \wrongchoice{the velocity of a moving negative charge}
      \correctchoice{none of the provided}
    \end{choices}
\end{question}
}

\element{halliday-mc}{
\begin{question}{halliday-ch28-q05}
    The magnetic force on a charged particle is in the direction of its velocity if:
    \begin{choices}
        \wrongchoice{it is moving in the direction of the field}
        \wrongchoice{it is moving opposite to the direction of the field}
        \wrongchoice{it is moving perpendicular to the field}
        \wrongchoice{it is moving in some other direction}
      \correctchoice{never}
    \end{choices}
\end{question}
}

\element{halliday-mc}{
\begin{question}{halliday-ch28-q06}
    A magnetic field exerts a force on a charged particle:
    \begin{choices}
        \wrongchoice{always}
        \wrongchoice{never}
      \correctchoice{if the particle is moving across the field lines}
        \wrongchoice{if the particle is moving along the field lines}
        \wrongchoice{if the particle is at rest}
    \end{choices}
\end{question}
}

\element{halliday-mc}{
\begin{question}{halliday-ch28-q07}
    The direction of the magnetic field in a certain region of space is determined by firing a test charge into the region with its velocity in various directions in different trials.
    The field direction is:
    \begin{choices}
      \correctchoice{one of the directions of the velocity when the magnetic force is zero}
        \wrongchoice{the direction of the velocity when the magnetic force is a maximum}
        \wrongchoice{the direction of the magnetic force}
        \wrongchoice{perpendicular to the velocity when the magnetic force is zero}
        \wrongchoice{none of the provided}
    \end{choices}
\end{question}
}

\element{halliday-mc}{
\begin{question}{halliday-ch28-q08}
    An electron is moving north in a region where the magnetic field is south.
    The magnetic force exerted on the electron is:
    \begin{multicols}{3}
    \begin{choices}
      \correctchoice{zero}
        \wrongchoice{up}
        \wrongchoice{down}
        \wrongchoice{east}
        \wrongchoice{west}
    \end{choices}
    \end{multicols}
\end{question}
}

\element{halliday-mc}{
\begin{question}{halliday-ch28-q09}
    A magnetic field \emph{cannot}:
    \begin{choices}
        \wrongchoice{exert a force on a charged particle}
        \wrongchoice{change the velocity of a charged particle}
        \wrongchoice{change the momentum of a charged particle}
      \correctchoice{change the kinetic energy of a charged particle}
        \wrongchoice{change the trajectory of a charged particle}
    \end{choices}
\end{question}
}

\element{halliday-mc}{
\begin{question}{halliday-ch28-q10}
    A proton (charge $e$), traveling perpendicular to a magnetic field,
        experiences the same force as an alpha particle (charge $2e$) which is also traveling perpendicular to the same field.
    The ratio of their speeds, $v_{\text{proton}}/v_{\text{alpha}}$, is:
    \begin{multicols}{3}
    \begin{choices}
        \wrongchoice{\num{0.5}}
        \wrongchoice{\num{1}}
      \correctchoice{\num{2}}
        \wrongchoice{\num{4}}
        \wrongchoice{\num{8}}
    \end{choices}
    \end{multicols}
\end{question}
}

\element{halliday-mc}{
\begin{question}{halliday-ch28-q11}
    A hydrogen atom that has lost its electron is moving east in a region where the magnetic field is directed from south to north.
    It will be deflected:
    \begin{multicols}{3}
    \begin{choices}
      \correctchoice{up}
        \wrongchoice{down}
        \wrongchoice{north}
        \wrongchoice{south}
        \wrongchoice{not at all}
    \end{choices}
    \end{multicols}
\end{question}
}

\element{halliday-mc}{
\begin{question}{halliday-ch28-q12}
    A beam of electrons is sent horizontally down the axis of a tube to strike a fluorescent screen at the end of the tube.
    On the way,
        the electrons encounter a magnetic field directed vertically downward.
    The spot on the screen will therefore be deflected:
    \begin{choices}
        \wrongchoice{upward}
        \wrongchoice{downward}
      \correctchoice{to the right as seen from the electron source}
        \wrongchoice{to the left as seen from the electron source}
        \wrongchoice{not at all}
    \end{choices}
\end{question}
}

\element{halliday-mc}{
\begin{question}{halliday-ch28-q13}
    An electron (charge = \SI{-1.6e-19}{\coulomb}) is moving at \SI{3e5}{\meter\per\second} in the positive $x$ direction.
    A magnetic field of \SI{0.8}{\tesla} is in the positive $z$ direction.
    The magnetic force on the electron is:
    \begin{choices}
        \wrongchoice{zero}
        \wrongchoice{\SI{4e-14}{\newton}, in the positive $z$ direction}
        \wrongchoice{\SI{4e-14}{\newton}, in the negative $z$ direction}
      \correctchoice{\SI{4e-14}{\newton}, in the positive $y$ direction}
        \wrongchoice{\SI{4e-14}{\newton}, in the negative $y$ direction}
    \end{choices}
\end{question}
}

\element{halliday-mc}{
\begin{question}{halliday-ch28-q14}
    At one instant an electron (charge = \SI{-1.6e-19}{\coulomb}) is moving in the $xy$ plane,
        the components of its velocity being $v_x=\SI{5e5}{\meter\per\second}$ and $v_y=\SI{3e5}{\meter\per\second}$.
    A magnetic field of \SI{0.8}{\tesla} is in the positive $x$ direction.
    At that instant the magnitude of the magnetic force on the electron is:
    \begin{multicols}{2}
    \begin{choices}
        \wrongchoice{zero}
        \wrongchoice{\SI{2.6e-14}{\newton}}
      \correctchoice{\SI{3.8e-14}{\newton}}
        \wrongchoice{\SI{6.4e-14}{\newton}}
        \wrongchoice{\SI{1.0e-3}{\newton}}
    \end{choices}
    \end{multicols}
\end{question}
}

\element{halliday-mc}{
\begin{question}{halliday-ch28-q15}
    At one instant an electron (charge = \SI{-1.6e-19}{\coulomb}) is moving in the $xy$ plane,
    the components of its velocity being $v_x=\SI{5e5}{\meter\per\second}$ and $v_y=\SI{3e5}{\meter\per\second}$.
    A magnetic field of \SI{0.8}{\tesla} is in the positive $x$ direction.
    At that instant the magnitude of the magnetic force on the electron is:
    \begin{multicols}{2}
    \begin{choices}
        \wrongchoice{zero}
      \correctchoice{\SI{3.8e-14}{\newton}}
        \wrongchoice{\SI{5.1e-14}{\newton}}
        \wrongchoice{\SI{6.4e-14}{\newton}}
        \wrongchoice{\SI{7.5e-14}{\newton}}
    \end{choices}
    \end{multicols}
\end{question}
}

\element{halliday-mc}{
\begin{question}{halliday-ch28-q16}
    An electron travels due north through a vacuum in a region of uniform magnetic field $\vec{B}$ that is also directed due north.
    It will:
    \begin{choices}
      \correctchoice{be unaffected by the field}
        \wrongchoice{speed up}
        \wrongchoice{slow down}
        \wrongchoice{follow a right-handed corkscrew path}
        \wrongchoice{follow a left-handed corkscrew path}
    \end{choices}
\end{question}
}

\element{halliday-mc}{
\begin{question}{halliday-ch28-q17}
    At one instant an electron is moving in the positive $x$ direction along the $x$ axis in a region where there is a uniform magnetic field in the positive $z$ direction.
    When viewed from a point on the positive $z$ axis,
        it subsequent motion is:
    \begin{choices}
        \wrongchoice{straight ahead}
      \correctchoice{counterclockwise around a circle in the $xy$ plane}
        \wrongchoice{clockwise around a circle in the $xy$ plane}
        \wrongchoice{in the positive $z$ direction}
        \wrongchoice{in the negative $z$ direction}
    \end{choices}
\end{question}
}

\element{halliday-mc}{
\begin{question}{halliday-ch28-q18}
    A uniform magnetic field is directed into the page.
    A charged particle, moving in the plane of the page,
        follows a clockwise spiral of decreasing radius as shown.
    \begin{center}
    \begin{tikzpicture}
        %% NOTE: tikz
    \end{tikzpicture}
    \end{center}
    A reasonable explanation is:
    \begin{choices}
        \wrongchoice{the charge is positive and slowing down}
      \correctchoice{the charge is negative and slowing down}
        \wrongchoice{the charge is positive and speeding up}
        \wrongchoice{the charge is negative and speeding up}
        \wrongchoice{none of the provided}
    \end{choices}
\end{question}
}

\element{halliday-mc}{
\begin{question}{halliday-ch28-q19}
    An electron and a proton each travel with equal speeds around circular orbits in the same uniform magnetic field,
        as shown in the diagram (not to scale).
    The field is into the page on the diagram.
    \begin{center}
    \begin{tikzpicture}
        %% NOTE: tikz
    \end{tikzpicture}
    \end{center}
    Because the electron is less massive than the proton and because the electron is negatively charged and the proton is positively charged:
    \begin{choices}
      \correctchoice{the electron travels clockwise around the smaller circle and the proton travels counterclockwise around the larger circle}
        \wrongchoice{the electron travels counterclockwise around the smaller circle and the proton travels clockwise around the larger circle}
        \wrongchoice{the electron travels clockwise around the larger circle and the proton travels counterclockwise around the smaller circle}
        \wrongchoice{the electron travels counterclockwise around the larger circle and the proton travels clockwise around the smaller circle}
        \wrongchoice{the electron travels counterclockwise around the smaller circle and the proton travels counterclockwise around the larger circle}
    \end{choices}
\end{question}
}

\element{halliday-mc}{
\begin{question}{halliday-ch28-q20}
    An electron is launched with velocity $v$ in a uniform magnetic field $B$.
    The angle $\theta$ between $v$ and $B$ is between \ang{0} and \ang{90}.
    As a result, the electron follows a helix,
        its velocity vector $v$ returning to its initial value in a time interval of:
    \begin{multicols}{2}
    \begin{choices}
      \correctchoice{$\dfrac{2\pi m}{eB}$}
        \wrongchoice{$\dfrac{2\pi mv}{eB}$}
        \wrongchoice{$\dfrac{2\pi mv\sin\theta}{eB}$}
        \wrongchoice{$\dfrac{2\pi πmv\cos\theta}{eB}$}
        \wrongchoice{none of the provided}
    \end{choices}
    \end{multicols}
\end{question}
}

\element{halliday-mc}{
\begin{question}{halliday-ch28-q21}
    An electron and a proton are both initially moving with the same speed and in the same direction at \ang{90} to the same uniform magnetic field.
    They experience magnetic forces, which are initially:
    \begin{choices}
        \wrongchoice{identical}
      \correctchoice{equal in magnitude but opposite in direction}
        \wrongchoice{in the same direction and differing in magnitude by a factor of 1840}
        \wrongchoice{in opposite directions and differing in magnitude by a factor of 1840}
        \wrongchoice{equal in magnitude but perpendicular to each other.}
    \end{choices}
\end{question}
}

\element{halliday-mc}{
\begin{question}{halliday-ch28-q22}
    An electron enters a region of uniform perpendicular $\vec{E}$ and $\vec{B}$ fields.
    It is observed that the velocity $v$ of the electron is unaffected.
    A possible explanation is:
    \begin{choices}
        \wrongchoice{$v$ is parallel to $\vec{E}$ and has magnitude $E/B$}
        \wrongchoice{$v$ is parallel to $\vec{B}$}
        \wrongchoice{$v$ is perpendicular to both $\vec{E}$ and $\vec{B}$ and has magnitude $B/E$}
      \correctchoice{$v$ is perpendicular to both $\vec{E}$ and $\vec{B}$ and has magnitude $E/B$}
        \wrongchoice{the given situation is impossible}
    \end{choices}
\end{question}
}

\element{halliday-mc}{
\begin{question}{halliday-ch28-q23}
    A charged particle is projected into a region of uniform,
        parallel, $\vec{E}$ and $\vec{B}$ fields.
    The force on the particle is:
    \begin{choices}
        \wrongchoice{zero}
      \correctchoice{at some angle $<\ang{90}$ with the field lines}
        \wrongchoice{along the field lines}
        \wrongchoice{perpendicular to the field lines}
        \wrongchoice{unknown (need to know the sign of the charge)}
    \end{choices}
\end{question}
}

\element{halliday-mc}{
\begin{question}{halliday-ch28-q24}
    A uniform magnetic field is in the positive $z$ direction.
    A positively charged particle is moving in the positive $x$ direction through the field.
    The net force on the particle can be made zero by applying an electric field in what direction?
    \begin{multicols}{2}
    \begin{choices}
        \wrongchoice{Positive $y$}
      \correctchoice{Negative $y$}
        \wrongchoice{Positive $x$}
        \wrongchoice{Negative $x$}
        \wrongchoice{Positive $z$}
    \end{choices}
    \end{multicols}
\end{question}
}

\element{halliday-mc}{
\begin{question}{halliday-ch28-q25}
    An electron is traveling in the positive $x$ direction.
    A uniform electric field $\vec{E}$ is in the negative $y$ direction.
    \begin{center}
    \begin{tikzpicture}
        %% NOTE: tikz
    \end{tikzpicture}
    \end{center}
    If a uniform magnetic field with the appropriate magnitude and direction also exists in the region, the total force on the electron will be zero.
    The appropriate direction for the magnetic field is:
    \begin{choices}
        \wrongchoice{the positive $y$ direction}
        \wrongchoice{the negative $y$ direction}
      \correctchoice{into the page}
        \wrongchoice{out of the page}
        \wrongchoice{the negative $x$ direction}
    \end{choices}
\end{question}
}

\element{halliday-mc}{
\begin{question}{halliday-ch28-q26}
    An ion with a charge of \SI{+3.2e-19}{\coulomb} is in a region where a uniform electric field of \SI{5e4}{\volt\per\meter} is perpendicular to a uniform magnetic field of \SI{0.8}{\tesla}.
    If its acceleration is zero then its speed must be:
    \begin{multicols}{2}
    \begin{choices}
        \wrongchoice{zero}
        \wrongchoice{\SI{1.6e4}{\meter\per\second}}
        \wrongchoice{\SI{4.0e4}{\meter\per\second}}
      \correctchoice{\SI{6.3e4}{\meter\per\second}}
        \wrongchoice{any value but zero}
    \end{choices}
    \end{multicols}
\end{question}
}

\element{halliday-mc}{
\begin{question}{halliday-ch28-q27}
    The current is from left to right in the conductor shown.
    \begin{center}
    \begin{tikzpicture}
        %% NOTE: tikz
    \end{tikzpicture}
    \end{center}
    The magnetic field is into the page and point $S$ is at a higher potential than point $T$.
    The charge carriers are:
    \begin{multicols}{2}
    \begin{choices}
      \correctchoice{positive}
        \wrongchoice{negative}
        \wrongchoice{neutral}
        \wrongchoice{absent}
        \wrongchoice{moving near the speed of light}
    \end{choices}
    \end{multicols}
\end{question}
}

\element{halliday-mc}{
\begin{question}{halliday-ch28-q28}
    Electrons (mass $m$, charge $-e$) are accelerated from rest through a potential difference $V$ and are then deflected by a magnetic field $\vec{B}$ that is perpendicular to their velocity. 
    The radius of the resulting electron trajectory is:
    \begin{multicols}{2}
    \begin{choices}
        \wrongchoice{$\dfrac{1}{B}\sqrt{\dfrac{2eV}{m}}$}
        \wrongchoice{$\dfrac{B\sqrt{2eV}}{m}$}
      \correctchoice{$\dfrac{1}{b}\sqrt{\dfrac{2mV}{e}}$}
        \wrongchoice{$\dfrac{B\sqrt{2mV}}{e}$}
        \wrongchoice{none of the provided}
    \end{choices}
    \end{multicols}
\end{question}
}

\element{halliday-mc}{
\begin{question}{halliday-ch28-q29}
    In a certain mass spectrometer, an ion beam passes through a velocity filter consisting of mutually perpendicular fields $\vec{E}$ and $\vec{B}$. 
    The beam then enters a region of another magnetic field $\vec{B}^{\prime}$ perpendicular to the beam.
    The radius of curvature of the resulting ion beam is proportional to:
    \begin{multicols}{3}
    \begin{choices}
        \wrongchoice{$\dfrac{EB^{\prime}}{B}$}
        \wrongchoice{$\dfrac{EB}{B^{\prime}}$}
        \wrongchoice{$\dfrac{BB^{\prime}}{E}$}
        \wrongchoice{$\dfrac{B}{EB^{\prime}}$}
      \correctchoice{$\dfrac{E}{BB^{\prime}}$}
    \end{choices}
    \end{multicols}
\end{question}
}

\element{halliday-mc}{
\begin{question}{halliday-ch28-q30}
    A cyclotron operates with a given magnetic field and at a given frequency. 
    If $R$ denotes the radius of the final orbit,
        the final particle energy is proportional to:
    \begin{multicols}{3}
    \begin{choices}
        \wrongchoice{$\dfrac{1}{R}$}
        \wrongchoice{$R$}
      \correctchoice{$R^2$}
        \wrongchoice{$R^3$}
        \wrongchoice{$R^4$}
    \end{choices}
    \end{multicols}
\end{question}
}

\element{halliday-mc}{
\begin{question}{halliday-ch28-q31}
    J. J. Thomson’s experiment,
        involving the motion of an electron beam in mutually perpendicular $\vec{E}$ and $\vec{B}$ fields,
        gave the value of:
    \begin{choices}
        \wrongchoice{mass of an electron}
        \wrongchoice{charge of an electron}
        \wrongchoice{Earth's magnetic field}
      \correctchoice{charge/mass ratio for electrons}
        \wrongchoice{Avogadro's number}
    \end{choices}
\end{question}
}

\element{halliday-mc}{
\begin{question}{halliday-ch28-q32}
    The diagram shows a straight wire carrying a flow of electrons into the page. 
    \begin{center}
    \begin{tikzpicture}
        %% NOTE: tikz
    \end{tikzpicture}
    \end{center}
    The wire is between the poles of a permanent magnet. 
    The direction of the magnetic force exerted on the wire is:
    \begin{multicols}{2}
    \begin{choices}
        %% NOTE: tikz options, like projectphysics
      \correctchoice{↑}
        \wrongchoice{↓}
        \wrongchoice{←}
        \wrongchoice{→}
        \wrongchoice{into the page}
    \end{choices}
    \end{multicols}
\end{question}
}

\element{halliday-mc}{
\begin{question}{halliday-ch28-q33}
    The figure shows the motion of electrons in a wire that is near the $N$ pole of a magnet. 
    \begin{center}
    \begin{tikzpicture}
        %% NOTE: tikz
    \end{tikzpicture}
    \end{center}
    The wire will be pushed:
    \begin{choices}
        \wrongchoice{toward the magnet}
        \wrongchoice{away from the magnet}
        \wrongchoice{downwards}
      \correctchoice{upwards}
        \wrongchoice{along its length}
    \end{choices}
\end{question}
}

\element{halliday-mc}{
\begin{question}{halliday-ch28-q34}
    The diagram shows a straight wire carrying current $i$ in a uniform magnetic field. 
    The magnetic force on the wire is indicated by an arrow but the magnetic field is not shown.
    \begin{center}
    \begin{tikzpicture}
        %% NOTE: tikz
    \end{tikzpicture}
    \end{center}
    Of the following possibilities,
        the direction of the magnetic field is:
    \begin{choices}
        \wrongchoice{opposite the direction of the current}
        \wrongchoice{opposite the direction of $F$}
        \wrongchoice{in the direction of $F$}
        \wrongchoice{into the page}
      \correctchoice{out of the page}
    \end{choices}
\end{question}
}

\element{halliday-mc}{
\begin{question}{halliday-ch28-q35}
    The figure shows a uniform magnetic field $\vec{B}$ directed to the left and a wire carrying a current into the page.
    \begin{center}
    \begin{tikzpicture}
        %% NOTE: tikz
    \end{tikzpicture}
    \end{center}
    The magnetic force acting on the wire is:
    \begin{choices}
      \correctchoice{toward the top of the page}
        \wrongchoice{toward the bottom of the page}
        \wrongchoice{toward the left}
        \wrongchoice{toward the right}
        \wrongchoice{zero}
    \end{choices}
\end{question}
}

\element{halliday-mc}{
\begin{question}{halliday-ch28-q36}
    A loop of wire carrying a current of \SI{2.0}{\ampere} is in the shape of a right triangle with two equal sides,
        each \SI{15}{\centi\meter} long. 
    A \SI{0.7}{\tesla} uniform magnetic field is parallel to the hypotenuse. 
    The resultant magnetic force on the two equal sides has a magnitude of:
    \begin{multicols}{2}
    \begin{choices}
      \correctchoice{zero}
        \wrongchoice{\SI{0.21}{\newton}}
        \wrongchoice{\SI{0.30}{\newton}}
        \wrongchoice{\SI{0.41}{\newton}}
        \wrongchoice{\SI{0.51}{\newton}}
    \end{choices}
    \end{multicols}
\end{question}
}

\element{halliday-mc}{
\begin{question}{halliday-ch28-q37}
    A loop of wire carrying a current of \SI{2.0}{\ampere} is in the shape of a right triangle with two equal sides,
        each \SI{15}{\centi\meter} long. 
    A \SI{0.7}{\tesla} uniform magnetic field is in the plane of the triangle and is perpendicular to the hypotenuse. 
    The magnetic force on either of the two equal sides has a magnitude of:
    \begin{multicols}{2}
    \begin{choices}
        \wrongchoice{zero}
        \wrongchoice{\SI{0.105}{\newton}}
      \correctchoice{\SI{0.15}{\newton}}
        \wrongchoice{\SI{0.21}{\newton}}
        \wrongchoice{\SI{0.25}{\newton}}
    \end{choices}
    \end{multicols}
\end{question}
}

\element{halliday-mc}{
\begin{question}{halliday-ch28-q38}
    A current is clockwise around the outside edge of this page and a uniform magnetic field is directed parallel to the page,
        from left to right. 
    If the magnetic force is the only force acting on the page,
        the page will turn so the right edge:
    \begin{choices}
      \correctchoice{moves toward you}
        \wrongchoice{moves away from you}
        \wrongchoice{moves to your right}
        \wrongchoice{moves to your left}
        \wrongchoice{does not move}
    \end{choices}
\end{question}
}

\element{halliday-mc}{
\begin{question}{halliday-ch28-q39}
    A square loop of wire lies in the plane of the page and carries a current $I$ as shown.
    \begin{center}
    \begin{tikzpicture}
        %% NOTE: tikz
    \end{tikzpicture}
    \end{center}
    There is a uniform magnetic field $\vec{B}$ parallel to the side $MK$ as indicated. 
    The loop will tend to rotate:
    \begin{choices}
      \correctchoice{about $PQ$ with $KL$ coming out of the page}
        \wrongchoice{about $PQ$ with $KL$ going into the page}
        \wrongchoice{about $RS$ with $MK$ coming out of the page}
        \wrongchoice{about $RS$ with $MK$ going into the page}
        \wrongchoice{about an axis perpendicular to the page.}
    \end{choices}
\end{question}
}

\element{halliday-mc}{
\begin{question}{halliday-ch28-q40}
    The units of magnetic dipole moment are:
    \begin{choices}
        \wrongchoice{ampere (\si{\ampere})}
        \wrongchoice{ampere meter (\si{\ampere\meter})}
      \correctchoice{ampere meter squared (\si{\ampere\meter\squared})}
        \wrongchoice{ampere per meter (\si{\ampere\per\meter})}
        \wrongchoice{ampere per meter squared (\si{\ampere\per\meter\squared})}
    \end{choices}
\end{question}
}

\element{halliday-mc}{
\begin{question}{halliday-ch28-q41}
    You are facing a loop of wire which carries a clockwise current of \SI{3.0}{\ampere} and which surrounds an area of \SI{5.8e-2}{\meter\squared}. 
    The magnetic dipole moment of the loop is:
    \begin{choices}
        \wrongchoice{\SI{3.0}{\ampere\meter\squared}, away from you}
        \wrongchoice{\SI{3.0}{\ampere\meter\squared}, toward you}
      \correctchoice{\SI{0.17}{\ampere\meter\squared}, away from you}
        \wrongchoice{\SI{0.17}{\ampere\meter\squared}, toward you}
        \wrongchoice{\SI{0.17}{\ampere\meter\squared}, left to right}
    \end{choices}
\end{question}
}

\element{halliday-mc}{
\begin{question}{halliday-ch28-q42}
    The magnetic torque exerted on a flat current-carrying loop of wire by a uniform magnetic field B is:
    \begin{choices}
        \wrongchoice{maximum when the plane of the loop is perpendicular to $\vec{B}$}
      \correctchoice{maximum when the plane of the loop is parallel to $\vec{B}$}
        \wrongchoice{dependent on the shape of the loop for a fixed loop area}
        \wrongchoice{independent of the orientation of the loop}
        \wrongchoice{such as to rotate the loop around the magnetic field lines}
    \end{choices}
\end{question}
}

\element{halliday-mc}{
\begin{question}{halliday-ch28-q43}
    A circular loop of wire with a radius of \SI{20}{\centi\meter} lies in the $xy$ plane and carries a current of \SI{2}{\ampere},
        counterclockwise when viewed from a point on the positive $z$ axis.
    Its magnetic dipole moment is:
    \begin{choices}
      \correctchoice{\SI{0.25}{\ampere\meter\squared}, in the positive $z$ direction}
        \wrongchoice{\SI{0.25}{\ampere\meter\squared}, in the negative $z$ direction}
        \wrongchoice{\SI{2.5}{\ampere\meter\squared}, in the positive $z$ direction}
        \wrongchoice{\SI{2.5}{\ampere\meter\squared}, in the negative $z$ direction}
        \wrongchoice{\SI{0.25}{\ampere\meter\squared}, in the $xy$ plane}
    \end{choices}
\end{question}
}

\element{halliday-mc}{
\begin{question}{halliday-ch28-q44}
    The diagrams show five possible orientations of a magnetic dipole $\vec{\mu}$ in a uniform magnetic field $\vec{B}$.
    For which of these does the magnetic torque on the dipole have the greatest magnitude?
    \begin{multicols}{2}
    \begin{choices}
        %% NOTE: ANS is A
        \wrongchoice{
            \begin{tikzpicture}
                %% NOTE:
            \end{tikzpicture}
        }
    \end{choices}
    \end{multicols}
\end{question}
}

\element{halliday-mc}{
\begin{question}{halliday-ch28-q45}
    The magnetic dipole moment of a current-carrying loop of wire is in the positive $z$ direction.
    If a uniform magnetic field is in the positive $x$ direction the magnetic torque on the loop is:
    \begin{multicols}{2}
    \begin{choices}
        \wrongchoice{zero}
        \wrongchoice{in the positive $y$ direction}
        \wrongchoice{in the negative $y$ direction}
        \wrongchoice{in the positive $z$ direction}
        \wrongchoice{in the negative $z$ direction}
    \end{choices}
    \end{multicols}
\end{question}
}

\element{halliday-mc}{
\begin{question}{halliday-ch28-q46}
    For a loop of current-carrying wire in a uniform magnetic field the potential energy is a minimum if the magnetic dipole moment of the loop is:
    \begin{choices}
      \correctchoice{in the same direction as the field}
        \wrongchoice{in the direction opposite to that of the field}
        \wrongchoice{perpendicular to the field}
        \wrongchoice{at an angle of \ang{45} to the field}
        \wrongchoice{none of the provided}
    \end{choices}
\end{question}
}

\element{halliday-mc}{
\begin{question}{halliday-ch28-q47}
    The diagrams show five possible orientations of a magnetic dipole $\vec{\mu}$ in a uniform magnetic field $\vec{B}$. 
    For which of these is the potential energy the greatest?
    \begin{multicols}{2}
    \begin{choices}
        %% NOTE: ANS is E
        \wrongchoice{
            \begin{tikzpicture}
                %% NOTE:
            \end{tikzpicture}
        }
    \end{choices}
    \end{multicols}
\end{question}
}

\element{halliday-mc}{
\begin{question}{halliday-ch28-q48}
    A loop of current-carrying wire has a magnetic dipole moment of \SI{5e-4}{\ampere\meter\squared}.
    The moment initially is aligned with a \SI{0.5}{\tesla} magnetic field. 
    To rotate the loop so its dipole moment is perpendicular to the field and hold it in that orientation,
        you must do work of:
    \begin{multicols}{2}
    \begin{choices}
        \wrongchoice{zero}
      \correctchoice{\SI{2.5e-4}{\joule}}
        \wrongchoice{\SI{-2.5e-4}{\joule}}
        \wrongchoice{\SI{1.0e-3}{\joule}}
        \wrongchoice{\SI{-1.0e-3}{\joule}}
    \end{choices}
    \end{multicols}
\end{question}
}


\endinput


