
%%--------------------------------------------------
%% Halliday: Fundamentals of Physics
%%--------------------------------------------------


%% Chapter 38: Photons and Matter Waves
%%--------------------------------------------------


%% Learning Objectives
%%--------------------------------------------------

%% 38.01: Explain the absorption and emission of light in terms of quantized energy and photons.
%% 38.02: For photon absorption and emission, apply the relationships between energy, power, intensity, rate of photons, the Planck constant, the associated frequency, and the associated wavelength.


%% Halliday Multiple Choice Questions
%%--------------------------------------------------
\element{halliday-mc}{
\begin{question}{halliday-ch38-q01}
    The units of the Planck constant $h$ are those of:
    \begin{choices}
        \wrongchoice{energy}
        \wrongchoice{power}
        \wrongchoice{momentum}
      \correctchoice{angular momentum}
        \wrongchoice{frequency}
    \end{choices}
\end{question}
}

\element{halliday-mc}{
\begin{question}{halliday-ch38-q02}
    If $h$ is the Planck constant,
        then $\hbar$ is:
    \begin{multicols}{3}
    \begin{choices}
        \wrongchoice{$2πh$}
        \wrongchoice{$2h$}
        \wrongchoice{$\dfrac{h}{2}$}
      \correctchoice{$\dfrac{h}{2\pi}$}
        \wrongchoice{$\dfrac{2h}{\pi}$}
    \end{choices}
    \end{multicols}
\end{question}
}

\element{halliday-mc}{
\begin{question}{halliday-ch38-q03}
    The quantization of energy, $E = nhf$,
        is not important for an ordinary pendulum because:
    \begin{choices}
        \wrongchoice{the formula applies only to mass-spring oscillators}
      \correctchoice{the allowed energy levels are too closely spaced}
        \wrongchoice{the allowed energy levels are too widely spaced}
        \wrongchoice{the formula applies only to atoms}
        \wrongchoice{the value of h for a pendulum is too large}
    \end{choices}
\end{question}
}

\element{halliday-mc}{
\begin{question}{halliday-ch38-q04}
    The frequency of light beam $A$ is twice that of light beam $B$. 
    The ratio $E_A/E_B$ of photon energies is:
    \begin{multicols}{3}
    \begin{choices}
        \wrongchoice{\num{1/2}}
        \wrongchoice{\num{1/4}}
        \wrongchoice{\num{1}}
      \correctchoice{\num{2}}
        \wrongchoice{\num{4}}
    \end{choices}
    \end{multicols}
\end{question}
}

\element{halliday-mc}{
\begin{question}{halliday-ch38-q05}
    The wavelength of light beam $A$ is twice the wavelength of light beam $B$.
    The energy of a photon in beam $A$ is:
    \begin{choices}
      \correctchoice{half the energy of a photon in beam $B$}
        \wrongchoice{one-fourth the energy of a photon in beam $B$}
        \wrongchoice{equal to the energy of a photon in beam $B$}
        \wrongchoice{twice the energy of a photon in beam $B$}
        \wrongchoice{four times the energy of a photon in beam $B$}
    \end{choices}
\end{question}
}

\element{halliday-mc}{
\begin{question}{halliday-ch38-q06}
    A photon in light beam A has twice the energy of a photon in light beam $B$.
    The ratio $\dfrac{p_A}{p_B}$ of their momenta is:
    \begin{multicols}{3}
    \begin{choices}
        \wrongchoice{\num{1/2}}
        \wrongchoice{\num{1/4}}
        \wrongchoice{\num{1}}
      \correctchoice{\num{2}}
        \wrongchoice{\num{4}}
    \end{choices}
    \end{multicols}
\end{question}
}

\element{halliday-mc}{
\begin{question}{halliday-ch38-q07}
    Which of the following electromagnetic radiations has photons with the greatest energy?
    \begin{multicols}{2}
    \begin{choices}
        \wrongchoice{blue light}
        \wrongchoice{yellow light}
      \correctchoice{x rays}
        \wrongchoice{radio waves}
        \wrongchoice{microwaves}
    \end{choices}
    \end{multicols}
\end{question}
}

\element{halliday-mc}{
\begin{question}{halliday-ch38-q08}
    Which of the following electromagnetic radiations has photons with the greatest momentum?
    \begin{multicols}{2}
    \begin{choices}
        \wrongchoice{blue light}
        \wrongchoice{yellow light}
      \correctchoice{x rays}
        \wrongchoice{radio waves}
        \wrongchoice{microwaves}
    \end{choices}
    \end{multicols}
\end{question}
}

\element{halliday-mc}{
\begin{question}{halliday-ch38-q09}
    Rank following electromagnetic radiations according to the energies of their photons,
        from least to greatest.
    \begin{enumerate}
        \item blue light
        \item yellow light
        \item x rays
        \item radio waves
    \end{enumerate}
    \begin{multicols}{2}
    \begin{choices}
        \wrongchoice{1, 2, 3, 4}
      \correctchoice{4, 2, 1, 3}
        \wrongchoice{4, 1, 2, 3}
        \wrongchoice{3, 2, 1, 4}
        \wrongchoice{3, 1, 2, 4}
    \end{choices}
    \end{multicols}
\end{question}
}

\element{halliday-mc}{
\begin{question}{halliday-ch38-q10}
    The intensity of a uniform light beam with a wavelength of \SI{500}{\nano\meter} is \SI{2000}{\watt\per\meter\squared}.
    The photon flux (in number/\si{\meter\squared\second}) is about:
    \begin{multicols}{2}
    \begin{choices}
        \wrongchoice{\num{5e17}}
        \wrongchoice{\num{5e19}}
      \correctchoice{\num{5e21}}
        \wrongchoice{\num{5e23}}
        \wrongchoice{\num{5e25}}
    \end{choices}
    \end{multicols}
\end{question}
}

\element{halliday-mc}{
\begin{question}{halliday-ch38-q11}
    The concentration of photons in a uniform light beam with a wavelength of \SI{500}{\nano\meter} is \SI{1.7e13}{\per\meter\cubed}. 
    The intensity of the beam is:
    \begin{multicols}{2}
    \begin{choices}
        \wrongchoice{\SI{6.7e-6}{\watt\per\meter\squared}}
        \wrongchoice{\SI{1.0e3}{\watt\per\meter\squared}}
      \correctchoice{\SI{2.0e3}{\watt\per\meter\squared}}
        \wrongchoice{\SI{4.0e3}{\watt\per\meter\squared}}
        \wrongchoice{\SI{3.2e2}{\watt\per\meter\squared}}
    \end{choices}
    \end{multicols}
\end{question}
}

\element{halliday-mc}{
\begin{question}{halliday-ch38-q12}
    Light beams $A$ and $B$ have the same intensity but the wavelength associated with beam $A$ is longer than that associated with beam $B$. 
    The photon flux (number crossing a unit area per unit time) is:
    \begin{choices}
      \correctchoice{greater for $A$ than for $B$}
        \wrongchoice{greater for $B$ than for $A$}
        \wrongchoice{the same for $A$ and $B$}
        \wrongchoice{greater for $A$ than for $B$ only if both have short wavelengths}
        \wrongchoice{greater for $B$ than for $A$ only if both have short wavelengths}
    \end{choices}
\end{question}
}

\element{halliday-mc}{
\begin{question}{halliday-ch38-q13}
    In a photoelectric effect experiment the stopping potential is:
    \begin{choices}
        \wrongchoice{the energy required to remove an electron from the sample}
        \wrongchoice{the kinetic energy of the most energetic electron ejected}
        \wrongchoice{the potential energy of the most energetic electron ejected}
        \wrongchoice{the photon energy}
      \correctchoice{the electric potential that causes the electron current to vanish}
    \end{choices}
\end{question}
}

\element{halliday-mc}{
\begin{question}{halliday-ch38-q14}
    In a photoelectric effect experiment at a frequency above cut off,
        the stopping potential is proportional to:
    \begin{choices}
        \wrongchoice{the energy of the least energetic electron before it is ejected}
        \wrongchoice{the energy of the least energetic electron after it is ejected}
        \wrongchoice{the energy of the most energetic electron before it is ejected}
      \correctchoice{the energy of the most energetic electron after it is ejected}
        \wrongchoice{the electron potential energy at the surface of the sample}
    \end{choices}
\end{question}
}

\element{halliday-mc}{
\begin{question}{halliday-ch38-q15}
    In a photoelectric effect experiment at a frequency above cut off,
        the number of electrons ejected is proportional to:
    \begin{choices}
        \wrongchoice{their kinetic energy}
        \wrongchoice{their potential energy}
        \wrongchoice{the work function}
        \wrongchoice{the frequency of the incident light}
      \correctchoice{the number of photons that hit the sample}
    \end{choices}
\end{question}
}

\element{halliday-mc}{
\begin{question}{halliday-ch38-q16}
    In a photoelectric effect experiment no electrons are ejected if the frequency of the incident light is less than $\dfrac{A}{h}$,
        where $h$ is the Planck constant and $A$ is:
    \begin{choices}
        \wrongchoice{the maximum energy needed to eject the least energetic electron}
        \wrongchoice{the minimum energy needed to eject the least energetic electron}
        \wrongchoice{the maximum energy needed to eject the most energetic electron}
      \correctchoice{the minimum energy needed to eject the most energetic electron}
        \wrongchoice{the intensity of the incident light}
    \end{choices}
\end{question}
}

\element{halliday-mc}{
\begin{question}{halliday-ch38-q17}
    The diagram shows the graphs of the stopping potential as a function of the frequency of the incident light for photoelectric experiments performed on three different materials. 
    \begin{center}
    \begin{tikzpicture}
        %% NOTE: tikz
    \end{tikzpicture}
    \end{center}
    Rank the materials according to the values of their work functions,
        from least to greatest.
    \begin{multicols}{3}
    \begin{choices}
      \correctchoice{1, 2, 3}
        \wrongchoice{3, 2, 1}
        \wrongchoice{2, 3, 1}
        \wrongchoice{2, 1, 3}
        \wrongchoice{1, 3, 2}
    \end{choices}
    \end{multicols}
\end{question}
}

\element{halliday-mc}{
\begin{question}{halliday-ch38-q18}
    The work function for a certain sample is \SI{2.3}{\eV}. 
    The stopping potential for electrons ejected from the sample by \SI{7.0e14}{\hertz} electromagnetic radiation is:
    \begin{multicols}{3}
    \begin{choices}
        \wrongchoice{zero}
      \correctchoice{\SI{0.60}{\volt}}
        \wrongchoice{\SI{2.3}{\volt}}
        \wrongchoice{\SI{2.9}{\volt}}
        \wrongchoice{\SI{5.2}{\volt}}
    \end{choices}
    \end{multicols}
\end{question}
}

\element{halliday-mc}{
\begin{question}{halliday-ch38-q19}
    The stopping potential for electrons ejected by \SI{6.8e14}{\hertz} electromagnetic radiation incident on a certain sample is \SI{1.8}{\volt}. 
    The kinetic energy of the most energetic electrons ejected and the work function of the sample,
        respectively, are:
    \begin{multicols}{2}
    \begin{choices}
        \wrongchoice{\SI{1.8}{\eV}, \SI{2.8}{\eV}}
      \correctchoice{\SI{1.8}{\eV}, \SI{1.0}{\eV}}
        \wrongchoice{\SI{1.8}{\eV}, \SI{4.6}{\eV}}
        \wrongchoice{\SI{2.8}{\eV}, \SI{1.0}{\eV}}
        \wrongchoice{\SI{1.0}{\eV}, \SI{4.6}{\eV}}
    \end{choices}
    \end{multicols}
\end{question}
}

\element{halliday-mc}{
\begin{question}{halliday-ch38-q20}
    Separate Compton effect experiments are carried out using visible light and x rays. 
    The scattered radiation is observed at the same scattering angle. 
    For these experiments:
    \begin{choices}
        \wrongchoice{the x rays have the greater shift in wavelength and the greater change in photon energy}
      \correctchoice{the two radiations have the same shift in wavelength and the x rays have the greater change in photon energy}
        \wrongchoice{the two radiations have the same shift in wavelength and the visible light has the greater change in photon energy}
        \wrongchoice{the two radiations have the same shift in wavelength and the same change in photon energy}
        \wrongchoice{the visible light has the greater shift in wavelength and the greater shift in photon energy}
    \end{choices}
\end{question}
}

\element{halliday-mc}{
\begin{question}{halliday-ch38-q21}
    In Compton scattering from stationary particles the maximum change in wavelength can be made smaller by using:
    \begin{choices}
        \wrongchoice{higher frequency radiation}
        \wrongchoice{lower frequency radiation}
      \correctchoice{more massive particles}
        \wrongchoice{less massive particles}
        \wrongchoice{particles with greater charge}
    \end{choices}
\end{question}
}

\element{halliday-mc}{
\begin{question}{halliday-ch38-q22}
    Of the following, Compton scattering from electrons is most easily observed for:
    \begin{choices}
        \wrongchoice{microwaves}
        \wrongchoice{infrared light}
        \wrongchoice{visible light}
        \wrongchoice{ultraviolet light}
      \correctchoice{x rays}
    \end{choices}
\end{question}
}

\element{halliday-mc}{
\begin{question}{halliday-ch38-q23}
    In Compton scattering from stationary electrons the largest change in wavelength occurs when the photon is scattered through:
    \begin{multicols}{3}
    \begin{choices}
        \wrongchoice{\ang{0}}
        \wrongchoice{\ang{22.5}}
        \wrongchoice{\ang{45}}
        \wrongchoice{\ang{90}}
      \correctchoice{\ang{180}}
    \end{choices}
    \end{multicols}
\end{question}
}

\element{halliday-mc}{
\begin{question}{halliday-ch38-q24}
    In Compton scattering from stationary electrons the frequency of the emitted light is independent of:
    \begin{choices}
        \wrongchoice{the frequency of the incident light}
        \wrongchoice{the speed of the electron}
        \wrongchoice{the scattering angle}
        \wrongchoice{the electron recoil energy}
      \correctchoice{none of the provided}
    \end{choices}
\end{question}
}

\element{halliday-mc}{
\begin{question}{halliday-ch38-q25}
    In Compton scattering from stationary electrons the largest change in wavelength that can occur is:
    \begin{choices}
        \wrongchoice{\SI{2.43e-15}{\meter}}
      \correctchoice{\SI{2.43e-12}{\meter}}
        \wrongchoice{\SI{2.43e-9}{\meter}}
        \wrongchoice{dependent on the frequency of the incident light}
        \wrongchoice{dependent on the work function}
    \end{choices}
\end{question}
}

\element{halliday-mc}{
\begin{question}{halliday-ch38-q26}
    Electromagnetic radiation with a wavelength of \SI{5.7e-12}{\meter} is incident on stationary electrons.
    Radiation that has a wavelength of \SI{6.57e-12}{\meter} is detected at a scattering angle of:
    \begin{multicols}{3}
    \begin{choices}
        \wrongchoice{\ang{10}}
        \wrongchoice{\ang{121}}
        \wrongchoice{\ang{40}}
      \correctchoice{\ang{50}}
        \wrongchoice{\ang{69}}
    \end{choices}
    \end{multicols}
\end{question}
}

\element{halliday-mc}{
\begin{question}{halliday-ch38-q27}
    Electromagnetic radiation with a wavelength of \SI{3.5e-12}{\meter} is scattered from stationary electrons and photons that have been scattered through \ang{50} are detected. 
    An electron from which one of these photons was scattered receives an energy of:
    \begin{multicols}{2}
    \begin{choices}
        \wrongchoice{zero}
      \correctchoice{\SI{1.1e-14}{\joule}}
        \wrongchoice{\SI{1.9e-14}{\joule}}
        \wrongchoice{\SI{2.3e-14}{\joule}}
        \wrongchoice{\SI{1.3e-13}{\joule}}
    \end{choices}
    \end{multicols}
\end{question}
}

\element{halliday-mc}{
\begin{question}{halliday-ch38-q28}
    Electromagnetic radiation with a wavelength of \SI{3.5e-12}{\meter} is scattered from stationary electrons and photons that have been scattered through \ang{50} are detected. 
    After a scattering event the magnitude of the electron’s momentum is:
    \begin{multicols}{2}
    \begin{choices}
        \wrongchoice{zero}
      \correctchoice{\SI{1.5e-22}{\kilo\gram\meter\per\second}}
        \wrongchoice{\SI{2.0e-22}{\kilo\gram\meter\per\second}}
        \wrongchoice{\SI{2.2e-22}{\kilo\gram\meter\per\second}}
        \wrongchoice{\SI{8.7e-23}{\kilo\gram\meter\per\second}}
    \end{choices}
    \end{multicols}
\end{question}
}

\element{halliday-mc}{
\begin{question}{halliday-ch38-q29}
    Consider the following:
    \begin{enumerate}
        \item a photoelectric process in which some emitted electrons have kinetic energy greater than hf , where f is the frequency of the incident light.
        \item a photoelectric process in which all emitted electrons have energy less than hf .
        \item Compton scattering from stationary electrons for which the emitted light has a wavelength that is greater than that of the incident light.
        \item Compton scattering from stationary electrons for which the emitted light has a wavelength that is less than that of the incident light.
    \end{enumerate}
    The only possible processes are:
    \begin{multicols}{3}
    \begin{choices}
        \wrongchoice{1}
        \wrongchoice{3}
        \wrongchoice{1 and 3}
      \correctchoice{2 and 3}
        \wrongchoice{2 and 4}
    \end{choices}
    \end{multicols}
\end{question}
}

\element{halliday-mc}{
\begin{question}{halliday-ch38-q30}
    J. J. Thompson’s measurement of $\dfrac{e}{m}$ for electrons provides evidence of the:
    \begin{choices}
        \wrongchoice{wave nature of matter}
      \correctchoice{particle nature of matter}
        \wrongchoice{wave nature of radiation}
        \wrongchoice{particle nature of radiation}
        \wrongchoice{transverse wave nature of light}
    \end{choices}
\end{question}
}

\element{halliday-mc}{
\begin{question}{halliday-ch38-q31}
    Evidence for the wave nature of matter is:
    \begin{choices}
      \correctchoice{electron diffraction experiments of Davisson and Germer}
        \wrongchoice{Thompson's measurement of $\dfrac{e}{m}$}
        \wrongchoice{Young's double slit experiment}
        \wrongchoice{the Compton effect}
        \wrongchoice{Lenz's law}
    \end{choices}
\end{question}
}

\element{halliday-mc}{
\begin{question}{halliday-ch38-q32}
    Which of the following is \emph{not} evidence for the wave nature of matter?
    \begin{choices}
      \correctchoice{The photoelectric effect}
        \wrongchoice{The diffraction pattern obtained when electrons pass through a slit}
        \wrongchoice{Electron tunneling}
        \wrongchoice{The validity of the Heisenberg uncertainty principle}
        \wrongchoice{The interference pattern obtained when electrons pass through a two-slit system}
    \end{choices}
\end{question}
}

\element{halliday-mc}{
\begin{question}{halliday-ch38-q33}
    Of the following which is the best evidence for the wave nature of matter?
    \begin{choices}
        \wrongchoice{The photoelectric effect}
        \wrongchoice{The Compton effect}
        \wrongchoice{The spectral radiancy of cavity radiation}
        \wrongchoice{The relationship between momentum and energy for an electron}
      \correctchoice{The reflection of electrons by crystals}
    \end{choices}
\end{question}
}

\element{halliday-mc}{
\begin{question}{halliday-ch38-q34}
    Monoenergetic electrons are incident on a single slit barrier. 
    If the energy of each incident electron is increased the central maximum of the diffraction pattern:
    \begin{choices}
        \wrongchoice{widens}
        \wrongchoice{narrows}
        \wrongchoice{stays the same width}
        \wrongchoice{widens for slow electrons and narrows for fast electrons}
        \wrongchoice{narrows for slow electrons and widens for fast electrons}
    \end{choices}
\end{question}
}

\element{halliday-mc}{
\begin{question}{halliday-ch38-q35}
    A free electron and a free proton have the same kinetic energy. 
    This means that, compared to the matter wave associated with the proton,
        the matter wave associated with the electron has:
    \begin{choices}
        \wrongchoice{a shorter wavelength and a greater frequency}
        \wrongchoice{a longer wavelength and a greater frequency}
        \wrongchoice{a shorter wavelength and the same frequency}
      \correctchoice{a longer wavelength and the same frequency}
        \wrongchoice{a shorter wavelength and a smaller frequency}
    \end{choices}
\end{question}
}

\element{halliday-mc}{
\begin{question}{halliday-ch38-q36}
    A free electron and a free proton have the same momentum. 
    This means that, compared to the matter wave associated with the proton,
        the matter wave associated with the electron:
    \begin{choices}
        \wrongchoice{has a shorter wavelength and a greater frequency}
        \wrongchoice{has a longer wavelength and a greater frequency}
        \wrongchoice{has the same wavelength and the same frequency}
      \correctchoice{has the same wavelength and a greater frequency}
        \wrongchoice{has the same wavelength and a smaller frequency}
    \end{choices}
\end{question}
}

\element{halliday-mc}{
\begin{question}{halliday-ch38-q37}
    A free electron and a free proton have the same speed. 
    This means that,
        compared to the matter wave associated with the proton,
        the matter wave associated with the electron:
    \begin{choices}
        \wrongchoice{has a shorter wavelength and a greater frequency}
        \wrongchoice{has a longer wavelength and a greater frequency}
        \wrongchoice{has the same wavelength and the same frequency}
        \wrongchoice{has the same wavelength and a greater frequency}
      \correctchoice{has a longer wavelength and a smaller frequency}
    \end{choices}
\end{question}
}

\element{halliday-mc}{
\begin{question}{halliday-ch38-q38}
    Consider the following three particles:
    \begin{enumerate}
        \item a free electron with speed $v_0$
        \item a free proton with speed $v_0$
        \item a free proton with speed $2v_0$
    \end{enumerate}
    Rank them according to the wavelengths of their matter waves,
        least to greatest.
    \begin{multicols}{2}
    \begin{choices}
        \wrongchoice{1, 2, 3}
      \correctchoice{3, 2, 1}
        \wrongchoice{2, 3, 1}
        \wrongchoice{1, 3, 2}
        \wrongchoice{1, then 2 and 3 tied}
    \end{choices}
    \end{multicols}
\end{question}
}

\element{halliday-mc}{
\begin{question}{halliday-ch38-q39}
    Consider the following three particles:
    \begin{enumerate}
        \item a free electron with kinetic energy $K_0$
        \item a free proton with kinetic energy $K_0$
        \item a free proton with kinetic energy $2K_0$
    \end{enumerate}
    Rank them according to the wavelengths of their matter waves,
        least to greatest.
    \begin{choices}
        \wrongchoice{1, 2, 3}
      \correctchoice{3, 2, 1}
        \wrongchoice{2, 3, 1}
        \wrongchoice{1, 3, 2}
        \wrongchoice{1, then 2 and 3 tied}
    \end{choices}
\end{question}
}

\element{halliday-mc}{
\begin{question}{halliday-ch38-q40}
    A free electron has a momentum of \SI{5.0e-24}{\kilo\gram\meter\per\second}.
    The wavelength of its wave function is:
    \begin{multicols}{2}
    \begin{choices}
        \wrongchoice{\SI{1.3e-8}{\meter}}
      \correctchoice{\SI{1.3e-10}{\meter}}
        \wrongchoice{\SI{2.1e-11}{\meter}}
        \wrongchoice{\SI{2.1e-13}{\meter}}
        \wrongchoice{none of the provided}
    \end{choices}
    \end{multicols}
\end{question}
}

\element{halliday-mc}{
\begin{question}{halliday-ch38-q41}
    The frequency and wavelength of the matter wave associated with a \SI{10}{\eV} free electron are:
    \begin{choices}
        \wrongchoice{\SI{1.5e34}{\hertz}, \SI{3.9e-10}{\meter}}
        \wrongchoice{\SI{1.5e34}{\hertz}, \SI{1.3e-34}{\meter}}
        \wrongchoice{\SI{2.4e15}{\hertz}, \SI{1.2e-9}{\meter}}
      \correctchoice{\SI{2.4e15}{\hertz}, \SI{3.9e-10}{\meter}}
        \wrongchoice{\SI{4.8e15}{\hertz}, \SI{1.9e-10}{\meter}}
    \end{choices}
\end{question}
}

\element{halliday-mc}{
\begin{question}{halliday-ch38-q42}
    If the kinetic energy of a non-relativistic free electron doubles,
        the frequency of its wave function changes by the factor:
    \begin{multicols}{3}
    \begin{choices}
        \wrongchoice{$\dfrac{1}{\sqrt{2}}$}
        \wrongchoice{$\dfrac{1}{2}$}
        \wrongchoice{$\dfrac{1}{4}$}
        \wrongchoice{$\sqrt{2}$}
      \correctchoice{$2$}
    \end{choices}
    \end{multicols}
\end{question}
}

\element{halliday-mc}{
\begin{question}{halliday-ch38-q43}
    A non-relativistic free electron has kinetic energy $K$. 
    If its wavelength doubles,
        its kinetic energy is:
    \begin{multicols}{3}
    \begin{choices}
        \wrongchoice{$4K$}
        \wrongchoice{$K$}
        \wrongchoice{still $K$}
        \wrongchoice{$\dfrac{K}{2}$}
      \correctchoice{$\dfrac{K}{4}$}
    \end{choices}
    \end{multicols}
\end{question}
}

\element{halliday-mc}{
\begin{question}{halliday-ch38-q44}
    The probability that a particle is in a given small region of space is proportional to:
    \begin{choices}
        \wrongchoice{its energy}
        \wrongchoice{its momentum}
        \wrongchoice{the frequency of its wave function}
        \wrongchoice{the wavelength of its wave function}
      \correctchoice{the square of the magnitude of its wave function}
    \end{choices}
\end{question}
}

\element{halliday-mc}{
\begin{question}{halliday-ch38-q45}
    $\Psi(x)$ is the wave function for a particle moving along the $x$ axis. 
    The probability that the particle is in the interval from $x=a$ to $x=b$ is given by:
    \begin{multicols}{2}
    \begin{choices}
        \wrongchoice{$\Psi(b) - \Psi(a)$}
        \wrongchoice{$\dfrac{|\Psi(b)|}{|\Psi(a)|}$}
        \wrongchoice{$\dfrac{|\Psi(b)|^2}{|\Psi(a)|^2}$}
        \wrongchoice{$\int^b_a\,\Psi(x)\,\mathrm{d}x$}
      \correctchoice{$\int^b_a\,|\Psi(x)|^2\,\mathrm{d}x$}
    \end{choices}
    \end{multicols}
\end{question}
}

\element{halliday-mc}{
\begin{question}{halliday-ch38-q46}
    The significance of $|\Psi|^2$ is:
    \begin{multicols}{2}
    \begin{choices}
        \wrongchoice{probability}
        \wrongchoice{energy}
      \correctchoice{probability density}
        \wrongchoice{energy density}
        \wrongchoice{wavelength}
    \end{choices}
    \end{multicols}
\end{question}
}

\element{halliday-mc}{
\begin{question}{halliday-ch38-q47}
    Maxwell's equations are to electric and magnetic fields as \rule[-0.1pt]{4em}{0.1pt} equation is to the wave function for a particle.
    \begin{multicols}{2}
    \begin{choices}
        \wrongchoice{Einstein's}
        \wrongchoice{Fermi's}
        \wrongchoice{Newton's}
      \correctchoice{Schr\"{o}dinger's}
        \wrongchoice{Bohr's}
    \end{choices}
    \end{multicols}
\end{question}
}

\element{halliday-mc}{
\begin{question}{halliday-ch38-q48}
    A free electron in motion along the $x$ axis has a localized wave function. 
    The uncertainty in its momentum is decreased if:
    \begin{choices}
        \wrongchoice{the wave function is made more narrow}
      \correctchoice{the wave function is made less narrow}
        \wrongchoice{the wave function remains the same but the energy of the electron is increased}
        \wrongchoice{the wave function remains the same but the energy of the electron is decreased}
        \wrongchoice{none of the provided}
    \end{choices}
\end{question}
}

\element{halliday-mc}{
\begin{questionmult}{halliday-ch38-q49}
    The uncertainty in position of an electron in a certain state is \SI{5e-10}{\meter}.
    The uncertainty in its momentum might be:
    \begin{choices}
      \correctchoice{\SI{5.0e-24}{\kilo\gram\meter\per\second}}
      \correctchoice{\SI{4.0e-24}{\kilo\gram\meter\per\second}}
      \correctchoice{\SI{3.0e-24}{\kilo\gram\meter\per\second}}
    \end{choices}
\end{questionmult}
}

\element{halliday-mc}{
\begin{question}{halliday-ch38-q50}
    The reflection coefficient $R$ for a certain barrier tunneling problem is \num{0.80}. 
    The corresponding transmission coefficient $T$ is:
    \begin{multicols}{3}
    \begin{choices}
        \wrongchoice{\num{0.80}}
        \wrongchoice{\num{0.60}}
        \wrongchoice{\num{0.50}}
      \correctchoice{\num{0.20}}
        \wrongchoice{zero}
    \end{choices}
    \end{multicols}
\end{question}
}

\element{halliday-mc}{
\begin{question}{halliday-ch38-q51}
    An electron with energy $E$ is incident upon a potential energy barrier of height $E_{\text{pot}}>E$ and thickness $L$.
    The transmission coefficient $T$:
    \begin{choices}
        \wrongchoice{is zero}
      \correctchoice{decreases exponentially with $L$}
        \wrongchoice{is proportional to $\dfrac{1}{L}$}
        \wrongchoice{is proportional to $\dfrac{1}{L^2}$}
        \wrongchoice{is non-zero and independent of $L$}
    \end{choices}
\end{question}
}

\element{halliday-mc}{
\begin{question}{halliday-ch38-q52}
    In order to tunnel through a potential barrier a particle must:
    \begin{choices}
        \wrongchoice{have energy greater than the barrier height}
        \wrongchoice{have spin}
        \wrongchoice{be massive}
        \wrongchoice{have a wavelength longer than the barrier width}
      \correctchoice{none of the provided}
    \end{choices}
\end{question}
}

\element{halliday-mc}{
\begin{question}{halliday-ch38-q53}
    An electron with energy $E$ is incident on a potential energy barrier of height $E_{\text{pot}}$ and thickness $L$. 
    The probability of tunneling increases if:
    \begin{choices}
        \wrongchoice{$E$ decreases without any other changes}
        \wrongchoice{$E_{\text{pot}}$ increases without any other changes}
      \correctchoice{$L$ decreases without any other changes}
        \wrongchoice{$E$ and $E_{\text{pot}}$ increase by the same amount}
        \wrongchoice{$E$ and $E_{\text{pot}}$ decrease by the same amount}
    \end{choices}
\end{question}
}

\element{halliday-mc}{
\begin{question}{halliday-ch38-q54}
    Identical particles, each with energy $E$,
        are incident on the following four potential energy barriers:
    \begin{enumerate}
        \item barrier height = $5E$, barrier width = $2L$
        \item barrier height = $10E$, barrier width = $L$
        \item barrier height = $17E$, barrier width = $\dfrac{L}{2}$
        \item barrier height = $26E$, barrier width = $\dfrac{L}{3}$
    \end{enumerate}
    Rank the barriers in terms of the probability that the particles tunnel through them,
        from least probability to greatest probability.
    \begin{choices}
      \correctchoice{1, 2, 3, 4}
        \wrongchoice{4, 3, 2, 1}
        \wrongchoice{1 and 2 tied, then 3, then 4}
        \wrongchoice{2, then 1 and 3 tied, then 4}
        \wrongchoice{3, 2, 1, 4}
    \end{choices}
\end{question}
}

\endinput


