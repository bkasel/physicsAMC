
%%--------------------------------------------------
%% Halliday: Fundamentals of Physics
%%--------------------------------------------------


%% Chapter 15: Oscillations
%%--------------------------------------------------


%% Learning Objectives
%%--------------------------------------------------

%% 15.01: Distinguish simple harmonic motion from other types of periodic motion.
%% 15.02: For a simple harmonic oscillator, apply the relationship between position $x$ and time $t$ to calculate either if given a value for the other.
%% 15.03: Relate period $T$, frequency $f$, and angular frequency $\omega$.
%% 15.04: Identify (displacement) amplitude $x_m$, phase constant (or phase angle) $\phi$, and phase $\omega t + \phi$.
%% 15.05: Sketch a graph of the oscillator's position $x$ versus time $t$, identifying amplitude $x_m$ and period $T$.
%% 15.06: From a graph of position versus time, velocity versus time, or acceleration versus time, determine the amplitude of the plot and the value of the phase constant $\phi$.
%% 15.07: On a graph of position $x$ versus time $t$ describe the effects of changing period $T$, frequency $f$, amplitude $x_m$, or phase constant $\phi$.
%% 15.08: Identify the phase constant f that corresponds to the starting time ($t=0$) being set when a particle in SHM is at an extreme point or passing through the center point.
%% 15.09: Given an oscillator's position $x(t)$ as a function of time, find its velocity $v(t)$ as a function of time, identify the velocity amplitude $v_m$ in the result, and calculate the velocity at any given time.
%% 15.10: Sketch a graph of an oscillator's velocity $v$ versus time $t$, identifying the velocity amplitude v m .
%% 15.11: Apply the relationship between velocity amplitude $v_m$, angular frequency $v$, and (displacement) amplitude $x_m$.
%% 15.12: Given an oscillator's velocity $v(t)$ as a function of time, calculate its acceleration $a(t)$ as a function of time, identify the acceleration amplitude $a_m$ in the result, and calculate the acceleration at any given time.
%% 15.13: Sketch a graph of an oscillator's acceleration a versus time $t$, identifying the acceleration amplitude $a_m$.
%% 15.14: Identify that for a simple harmonic oscillator the acceleration a at any instant is always given by the product of a negative constant and the displacement $x$ just then.
%% 15.15: For any given instant in an oscillation, apply the relationship between acceleration $a$, angular frequency $\omega$, and displacement $x$.
%% 15.16: Given data about the position $x$ and velocity $v$ at one instant, determine the phase $\omega t + \phi$ and phase constant $\phi$.
%% 15.17: For a spring--block oscillator, apply the relationships between spring constant $k$ and mass $m$ and either period $T$ or angular frequency $\omega$.
%% 15.18: Apply Hooke's law to relate the force $F$ on a simple harmonic oscillator at any instant to the displacement $x$ of the oscillator at that instant.


%% Halliday Multiple Choice Questions
%%--------------------------------------------------
\element{halliday-mc}{
\begin{question}{halliday-ch15-q01}
    In simple harmonic motion, the restoring force must be proportional to the:
    \begin{choices}
        \wrongchoice{amplitude}
        \wrongchoice{frequency}
        \wrongchoice{velocity}
      \correctchoice{displacement}
        \wrongchoice{displacement squared}
    \end{choices}
\end{question}
}

\element{halliday-mc}{
\begin{question}{halliday-ch15-q02}
    An oscillatory motion must be simple harmonic if:
    \begin{choices}
        \wrongchoice{the amplitude is small}
        \wrongchoice{the potential energy is equal to the kinetic energy}
        \wrongchoice{the motion is along the arc of a circle}
      \correctchoice{the acceleration varies sinusoidally with time}
        \wrongchoice{the derivative, $\mathrm{d}U/\mathrm{d}x$, of the potential energy is negative}
    \end{choices}
\end{question}
}

\element{halliday-mc}{
\begin{question}{halliday-ch15-q03}
    In simple harmonic motion,
        the magnitude of the acceleration is:
    \begin{choices}
        \wrongchoice{constant}
      \correctchoice{proportional to the displacement}
        \wrongchoice{inversely proportional to the displacement}
        \wrongchoice{greatest when the velocity is greatest}
        \wrongchoice{never greater than $g$}
    \end{choices}
\end{question}
}

\element{halliday-mc}{
\begin{question}{halliday-ch15-q04}
    A particle is in simple harmonic motion with period $T$. 
    At time $t=0$ it is at the equilibrium point. 
    Of the following times,
        at which time is it furthest from the equilibrium point?
    \begin{multicols}{3}
    \begin{choices}
        \wrongchoice{$0.5 T$}
      \correctchoice{$0.7 T$}
        \wrongchoice{$T$}
        \wrongchoice{$1.4 T$}
        \wrongchoice{$1.5 T$}
    \end{choices}
    \end{multicols}
\end{question}
}

\element{halliday-mc}{
\begin{question}{halliday-ch15-q05}
    A particle moves back and forth along the $x$ axis from $x=-x_m$ to $x=+x_m$,
        in simple harmonic motion with period $T$. 
    At time $t=0$ it is at $x=+x_m$.
    When $t=0.75 T$:
    \begin{choices}
      \correctchoice{it is at $x=0$ and is traveling toward $x=+x_m$}
        \wrongchoice{it is at $x=0$ and is traveling toward $x=-x_m$}
        \wrongchoice{it at $x=+x_m$ and is at rest}
        \wrongchoice{it is between $x=0$ and $x=+x_m$ and is traveling toward $x=-x_m$}
        \wrongchoice{it is between $x=0$ and $x=-x_m$ and is traveling toward $x=-x_m$}
    \end{choices}
\end{question}
}

\element{halliday-mc}{
\begin{question}{halliday-ch15-q06}
    A particle oscillating in simple harmonic motion is:
    \begin{choices}
        \wrongchoice{never in equilibrium because it is in motion}
        \wrongchoice{never in equilibrium because there is always a force}
        \wrongchoice{in equilibrium at the ends of its path because its velocity is zero there}
      \correctchoice{in equilibrium at the center of its path because the acceleration is zero there}
        \wrongchoice{in equilibrium at the ends of its path because the acceleration is zero there}
    \end{choices}
\end{question}
}

\element{halliday-mc}{
\begin{question}{halliday-ch15-q07}
    An object is undergoing simple harmonic motion. 
    Throughout a complete cycle it:
    \begin{choices}
        \wrongchoice{has constant speed}
        \wrongchoice{has varying amplitude}
        \wrongchoice{has varying period}
      \correctchoice{has varying acceleration}
        \wrongchoice{has varying mass}
    \end{choices}
\end{question}
}

\element{halliday-mc}{
\begin{question}{halliday-ch15-q08}
    When a body executes simple harmonic motion,
        its acceleration at the ends of its path must be:
    \begin{choices}
        \wrongchoice{zero}
        \wrongchoice{less than $g$}
        \wrongchoice{more than $g$}
        \wrongchoice{suddenly changing in sign}
      \correctchoice{none of the provided}
    \end{choices}
\end{question}
}

\element{halliday-mc}{
\begin{question}{halliday-ch15-q09}
    A particle is in simple harmonic motion with period $T$. 
    At time $t=0$ it is halfway between the equilibrium point and an end point of its motion,
        traveling toward the end point. 
    The next time it is at the same place is:
    \begin{multicols}{2}
    \begin{choices}
        \wrongchoice{$t = T$}
        \wrongchoice{$t = \dfrac{T}{2}$}
        \wrongchoice{$t = \dfrac{T}{4}$}
        \wrongchoice{$t = \dfrac{T}{8}$}
      \correctchoice{none of the provided}
    \end{choices}
    \end{multicols}
\end{question}
}

\element{halliday-mc}{
\begin{question}{halliday-ch15-q10}
    An object attached to one end of a spring makes \num{20} complete oscillations in \SI{10}{\second}. 
    Its period is:
    \begin{multicols}{3}
    \begin{choices}
        \wrongchoice{\SI{2}{\hertz}}
        \wrongchoice{\SI{10}{\second}}
        \wrongchoice{\SI{0.5}{\hertz}}
        \wrongchoice{\SI{2}{\second}}
      \correctchoice{\SI{0.50}{\second}}
    \end{choices}
    \end{multicols}
\end{question}
}

\element{halliday-mc}{
\begin{question}{halliday-ch15-q11}
    An object attached to one end of a spring makes \num{20} complete oscillations in \SI{10}{\second}. 
    Its frequency is:
    \begin{multicols}{3}
    \begin{choices}
      \correctchoice{\SI{2}{\hertz}}
        \wrongchoice{\SI{10}{\second}}
        \wrongchoice{\SI{0.05}{\hertz}}
        \wrongchoice{\SI{2}{\second}}
        \wrongchoice{\SI{0.50}{\second}}
    \end{choices}
    \end{multicols}
\end{question}
}

\element{halliday-mc}{
\begin{question}{halliday-ch15-q12}
    An object attached to one end of a spring makes \num{20} vibrations in \SI{10}{\second}. 
    Its angular frequency is:
    \begin{multicols}{3}
    \begin{choices}
        \wrongchoice{\SI{0.79}{\radian\per\second}}
        \wrongchoice{\SI{1.57}{\radian\per\second}}
        \wrongchoice{\SI{2.0}{\radian\per\second}}
        \wrongchoice{\SI{6.3}{\radian\per\second}}
      \correctchoice{\SI{12.6}{\radian\per\second}}
    \end{choices}
    \end{multicols}
\end{question}
}

\element{halliday-mc}{
\begin{question}{halliday-ch15-q13}
    Frequency $f$ and angular frequency $\omega$ are related by:
    \begin{multicols}{3}
    \begin{choices}
        \wrongchoice{$f = \pi\omega$}
        \wrongchoice{$f = 2\pi\omega$}
        \wrongchoice{$f = \dfrac{\omega}{\pi}$}
      \correctchoice{$f = \dfrac{\omega}{2\pi}$}
        \wrongchoice{$f = \dfrac{2\omega}{\pi}$}
    \end{choices}
    \end{multicols}
\end{question}
}

\element{halliday-mc}{
\begin{question}{halliday-ch15-q14}
    A block attached to a spring oscillates in simple harmonic motion along the $x$ axis. 
    The limits of its motion are $x=\SI{10}{\centi\meter}$ and $x=\SI{50}{\centi\meter}$ and it goes from one of these extremes to the other in \SI{0.25}{\second}. 
    Its amplitude and frequency are:
    \begin{multicols}{2}
    \begin{choices}
        \wrongchoice{\SI{40}{\centi\meter}, \SI{2}{\hertz}}
      \correctchoice{\SI{20}{\centi\meter}, \SI{4}{\hertz}}
        \wrongchoice{\SI{40}{\centi\meter}, \SI{2}{\hertz}}
        \wrongchoice{\SI{25}{\centi\meter}, \SI{4}{\hertz}}
        \wrongchoice{\SI{20}{\centi\meter}, \SI{2}{\hertz}}
    \end{choices}
    \end{multicols}
\end{question}
}

\element{halliday-mc}{
\begin{question}{halliday-ch15-q15}
    A weight suspended from an ideal spring oscillates up and down with a period $T$. 
    If the amplitude of the oscillation is doubled,
        the period will be:
    \begin{multicols}{3}
    \begin{choices}
      \correctchoice{$T$}
        \wrongchoice{$\dfrac{3T}{2}$}
        \wrongchoice{$2T$}
        \wrongchoice{$\dfrac{T}{2}$}
        \wrongchoice{$4T$}
    \end{choices}
    \end{multicols}
\end{question}
}

\element{halliday-mc}{
\begin{question}{halliday-ch15-q16}
    In simple harmonic motion,
        the magnitude of the acceleration is greatest when:
    \begin{choices}
        \wrongchoice{the displacement is zero}
      \correctchoice{the displacement is maximum}
        \wrongchoice{the speed is maximum}
        \wrongchoice{the force is zero}
        \wrongchoice{the speed is between zero and its maximum}
    \end{choices}
\end{question}
}

\element{halliday-mc}{
\begin{question}{halliday-ch15-q17}
    In simple harmonic motion,
        the displacement is maximum when the:
    \begin{choices}
        \wrongchoice{acceleration is zero}
        \wrongchoice{velocity is maximum}
      \correctchoice{velocity is zero}
        \wrongchoice{kinetic energy is maximum}
        \wrongchoice{momentum is maximum}
    \end{choices}
\end{question}
}

\element{halliday-mc}{
\begin{question}{halliday-ch15-q18}
    In simple harmonic motion:
    \begin{choices}
      \correctchoice{the acceleration is greatest at the maximum displacement}
        \wrongchoice{the velocity is greatest at the maximum displacement}
        \wrongchoice{the period depends on the amplitude}
        \wrongchoice{the acceleration is constant}
        \wrongchoice{the acceleration is greatest at zero displacement}
    \end{choices}
\end{question}
}

\element{halliday-mc}{
\begin{question}{halliday-ch15-q19}
    The amplitude and phase constant of an oscillator are determined by:
    \begin{choices}
        \wrongchoice{the frequency}
        \wrongchoice{the angular frequency}
        \wrongchoice{the initial displacement alone}
        \wrongchoice{the initial velocity alone}
      \correctchoice{both the initial displacement and velocity}
    \end{choices}
\end{question}
}

\element{halliday-mc}{
\begin{question}{halliday-ch15-q20}
    Two identical undamped oscillators have the same amplitude of oscillation only if:
    \begin{choices}
        \wrongchoice{they are started with the same displacement $x_0$}
        \wrongchoice{they are started with the same velocity $v_0$}
        \wrongchoice{they are started with the same phase}
      \correctchoice{they are started so the combination $\omega^2 x_0^2 + v_0^2$ is the same}
        \wrongchoice{they are started so the combination $x_0^2 + \omega^2 v_0^2$ is the same}
    \end{choices}
\end{question}
}

\element{halliday-mc}{
\begin{question}{halliday-ch15-q21}
    The amplitude of any oscillator can be doubled by:
    \begin{choices}
        \wrongchoice{doubling only the initial displacement}
        \wrongchoice{doubling only the initial speed}
        \wrongchoice{doubling the initial displacement and halving the initial speed}
        \wrongchoice{doubling the initial speed and halving the initial displacement}
      \correctchoice{doubling both the initial displacement and the initial speed}
    \end{choices}
\end{question}
}

\element{halliday-mc}{
\begin{question}{halliday-ch15-q22}
    It is impossible for two particles,
        each executing simple harmonic motion,
        to remain in phase with each other if they have different:
    \begin{choices}
        \wrongchoice{masses}
      \correctchoice{periods}
        \wrongchoice{amplitudes}
        \wrongchoice{spring constants}
        \wrongchoice{kinetic energies}
    \end{choices}
\end{question}
}

\element{halliday-mc}{
\begin{question}{halliday-ch15-q23}
    The acceleration of a body executing simple harmonic motion leads the velocity by what phase?
    \begin{multicols}{3}
    \begin{choices}
        \wrongchoice{zero}
        \wrongchoice{\SI{\pi/8}{\radian}}
        \wrongchoice{\SI{\pi/4}{\radian}}
      \correctchoice{\SI{\pi/2}{\radian}}
        \wrongchoice{\SI{\pi}{\radian}}
    \end{choices}
    \end{multicols}
\end{question}
}

\element{halliday-mc}{
\begin{question}{halliday-ch15-q24}
    The displacement of an object oscillating on a spring
        is given by $x\left(t\right) = x_m\cos\left(\omega t + \phi\right)$.
    If the initial displacement is zero and the initial velocity is in the negative $x$ direction,
        then the phase constant $\phi$ is:
    \begin{multicols}{3}
    \begin{choices}
        \wrongchoice{zero}
      \correctchoice{\SI{\pi/2}{\radian}}
        \wrongchoice{\SI{\pi}{\radian}}
        \wrongchoice{\SI{3\pi/2}{\radian}}
        \wrongchoice{\SI{2\pi}{\radian}}
    \end{choices}
    \end{multicols}
\end{question}
}

\element{halliday-mc}{
\begin{question}{halliday-ch15-q25}
    The displacement of an object oscillating on a spring is given by $x(t) = x_m\cos\left(\omega t + \phi\right)$.
    If the object is initially displaced in the negative $x$ direction and given a negative initial velocity,
        then the phase constant $\phi$ is between:
    \begin{choices}
        \wrongchoice{zero and \SI{\pi/2}{\radian}}
      \correctchoice{\SI{\pi/2}{\radian} and \SI{\pi}{\radian}}
        \wrongchoice{\SI{\pi}{\radian} and \SI{3\pi/2}{\radian}}
        \wrongchoice{\SI{3\pi/2}{\radian} and \SI{2\pi}{\radian}}
        \wrongchoice{none of the provided ($\phi$ is exactly zero, \num{\pi/2}, \num{\pi}, or \num{3\pi/2} radians)}
    \end{choices}
\end{question}
}

\element{halliday-mc}{
\begin{question}{halliday-ch15-q26}
    A certain spring elongates \SI{9.0}{\milli\meter} when it is suspended vertically and a block of mass $M$ is hung on it. 
    The natural angular frequency of this block-spring system:
    \begin{choices}
        \wrongchoice{is \SI{0.088}{\radian\per\second}}
        \wrongchoice{is \SI{33}{\radian\per\second}}
        \wrongchoice{is \SI{200}{\radian\per\second}}
        \wrongchoice{is \SI{1140}{\radian\per\second}}
        \wrongchoice{cannot be computed unless the value of $M$ is given}
    \end{choices}
\end{question}
}

\element{halliday-mc}{
\begin{question}{halliday-ch15-q27}
    An object of mass $m$,
        oscillating on the end of a spring with spring constant $k$,
        has amplitude $A$.
    Its maximum speed is:
    \begin{multicols}{2}
    \begin{choices}
      \correctchoice{$A\sqrt{\dfrac{k}{m}}$}
        \wrongchoice{$\dfrac{A^2 k}{m}$}
        \wrongchoice{$A \sqrt{\dfrac{m}{k}}$}
        \wrongchoice{$\dfrac{Am}{k}$}
        \wrongchoice{$\dfrac{A^2 m}{k}$}
    \end{choices}
    \end{multicols}
\end{question}
}

\element{halliday-mc}{
\begin{question}{halliday-ch15-q28}
    A \SI{0.20}{\kilo\gram} object attached to a spring whose spring constant is \SI{500}{\newton\per\meter} executes simple harmonic motion.
    If its maximum speed is \SI{5.0}{\meter\per\second},
        the amplitude of its oscillation is:
    \begin{multicols}{3}
    \begin{choices}
        \wrongchoice{\SI{0.0020}{\meter}}
      \correctchoice{\SI{0.10}{\meter}}
        \wrongchoice{\SI{0.20}{\meter}}
        \wrongchoice{\SI{25}{\meter}}
        \wrongchoice{\SI{250}{\meter}}
    \end{choices}
    \end{multicols}
\end{question}
}

\element{halliday-mc}{
\begin{question}{halliday-ch15-q29}
    A simple harmonic oscillator consists of an particle of mass $m$ and an ideal spring with spring constant $k$.
    \begin{center}
    \begin{tikzpicture}
        \begin{scope}[xshift=-1.5cm]
            %% Ceiling
            \draw (-1.25,0) --  (1.25,0);
            \node[anchor=south,fill,pattern=north east lines,minimum width=2.5cm, minimum height=0.05cm] at (0,0) {};
            %% Spring
            \draw[thick,decoration={aspect=0.2,segment length=3mm,amplitude=4mm,coil},decorate] (0,0) -- (0,-3)
                node[pos=0.25,anchor=east,xshift=-4mm] {(i)};
                %node[pos=0.25,anchor=west,xshift=4mm] {$k$};
            %% Weight
            \node[anchor=north,draw,minimum size=0.75cm,fill=white!90!black] (M) at (0,-3) {$m$};
        \end{scope}
        \begin{scope}[xshift=+1.5cm]
            %% Ceiling
            \draw (-1.25,0) --  (1.25,0);
            \node[anchor=south,fill,pattern=north east lines,minimum width=2.5cm, minimum height=0.05cm] at (0,0) {};
            %% Spring
            \draw[thick,decoration={aspect=0.2,segment length=3mm,amplitude=4mm,coil},decorate] (0,0) -- (0,-1.5)
                node[pos=0.5,anchor=east,xshift=-4mm] {(ii)};
                %node[pos=0.5,anchor=west,xshift=4mm] {$k$};
            %% Weight
            \node[anchor=north,draw,minimum size=0.75cm,fill=white!90!black] (M) at (0,-1.5) {$m$};
        \end{scope}
    \end{tikzpicture}
    \end{center}
    Particle oscillates as shown in (i) with period $T$.
    If the spring is cut in half and used with the same particle,
        as shown in (ii), the period will be:
    \begin{multicols}{3}
    \begin{choices}
        \wrongchoice{$2T$}
        \wrongchoice{$\sqrt{2T}$}
      \correctchoice{$\dfrac{T}{\sqrt{2}}$}
        \wrongchoice{$T$}
        \wrongchoice{$\dfrac{T}{2}$}
    \end{choices}
    \end{multicols}
\end{question}
}

\element{halliday-mc}{
\begin{question}{halliday-ch15-q30}
    A particle moves in simple harmonic motion according to $x=2\cos\left(50t\right)$,
        where $x$ is in meters and $t$ is in seconds. 
    Its maximum velocity is:
    \begin{multicols}{2}
    \begin{choices}
        \wrongchoice{$100\sin\left(50t\right)\,\si{\meter\per\second}$}
        \wrongchoice{$100\cos\left(50t\right)\,\si{\meter\per\second}$}
      \correctchoice{$100\,\si{\meter\per\second}$}
        \wrongchoice{$200\,\si{\meter\per\second}$}
        \wrongchoice{none of the provided}
    \end{choices}
    \end{multicols}
\end{question}
}

\element{halliday-mc}{
\begin{question}{halliday-ch15-q31}
    A \SI{3}{\kilo\gram} block, attached to a spring,
    executes simple harmonic motion according to $x=2\cos\left(50t\right)$ where $x$ is in meters and $t$ is in seconds. 
        The spring constant of the spring is:
    \begin{multicols}{2}
    \begin{choices}
        \wrongchoice{\SI{1}{\newton\per\meter}}
        \wrongchoice{\SI{100}{\newton\per\meter}}
        \wrongchoice{\SI{150}{\newton\per\meter}}
      \correctchoice{\SI{7500}{\newton\per\meter}}
        \wrongchoice{none of the provided}
    \end{choices}
    \end{multicols}
\end{question}
}

\element{halliday-mc}{
\begin{question}{halliday-ch15-q32}
    Let U be the potential energy (with the zero at zero displacement) and $K$ be the kinetic energy of a simple harmonic oscillator.
    $U_{avg}$ and $K_{avg}$ are the average values over a cycle. 
    Then:
    \begin{multicols}{2}
    \begin{choices}
        \wrongchoice{$K_{avg} > U avg$}
        \wrongchoice{$K_{avg} < U avg$}
        \wrongchoice{$K avg = U avg$}
        \wrongchoice{$K = 0$, when $U=0$}
        \wrongchoice{$K + U = 0$}
    \end{choices}
    \end{multicols}
\end{question}
}

\element{halliday-mc}{
\begin{question}{halliday-ch15-q33}
    A particle is in simple harmonic motion along the $x$ axis. 
    The amplitude of the motion is $x_m$.
    At one point in its motion its kinetic energy is $K=\SI{5}{\joule}$ and its potential energy (measured with $U=0$ at $x=0$) is $U=\SI{3}{\joule}$. 
    When it is at $x=x_m$,
        the kinetic and potential energies are:
    \begin{choices}
        \wrongchoice{$K = \SI{5}{\joule}$ and $U = \SI{3}{\joule}$}
        \wrongchoice{$K = \SI{5}{\joule}$ and $U = \SI{3}{\joule}$}
        \wrongchoice{$K = \SI{8}{\joule}$ and $U =$ zero}
      \correctchoice{$K = $zero and $U = \SI{8}{\joule}$}
        \wrongchoice{$K = $zero and $U = \SI{-8}{\joule}$}
    \end{choices}
\end{question}
}

\element{halliday-mc}{
\begin{question}{halliday-ch15-q34}
    A particle is in simple harmonic motion along the $x$ axis. 
    The amplitude of the motion is $x_m$.
    When it is at $x=x_1$,
        its kinetic energy is $K=\SI{5}{\joule}$ and its potential energy
        (measured with $U=0$ at $x=0$) is $U=\SI{3}{\joule}$. 
    When it is at $x=-\frac{1}{2} x_1$,
        the kinetic and potential energies are:
    \begin{choices}
      \correctchoice{$K=\SI{5}{\joule}$ and $U=\SI{3}{\joule}$}
        \wrongchoice{$K=\SI{5}{\joule}$ and $U=\SI{-3}{\joule}$}
        \wrongchoice{$K=\SI{8}{\joule}$ and $U=$zero}
        \wrongchoice{$K=$zero and $U=\SI{8}{\joule}$}
        \wrongchoice{$K=$zero and $U=\SI{-8}{\joule}$}
    \end{choices}
\end{question}
}

\element{halliday-mc}{
\begin{question}{halliday-ch15-q35}
    A \SI{0.25}{\kilo\gram} block oscillates on the end of the spring with a spring constant of \SI{200}{\newton\per\meter}. 
    If the system has an energy of \SI{6.0}{\joule},
        then the amplitude of the oscillation is:
    \begin{multicols}{3}
    \begin{choices}
        \wrongchoice{\SI{0.06}{\meter}}
        \wrongchoice{\SI{0.17}{\meter}}
      \correctchoice{\SI{0.24}{\meter}}
        \wrongchoice{\SI{4.9}{\meter}}
        \wrongchoice{\SI{6.9}{\meter}}
    \end{choices}
    \end{multicols}
\end{question}
}

\element{halliday-mc}{
\begin{question}{halliday-ch15-q36}
    A \SI{0.25}{\kilo\gram} block oscillates on the end of the spring with a spring constant of \SI{200}{\newton\per\meter}.
    If the system has an energy of \SI{6.0}{\joule},
        then the maximum speed of the block is:
    \begin{multicols}{3}
    \begin{choices}
        \wrongchoice{\SI{0.06}{\meter\per\second}}
        \wrongchoice{\SI{0.17}{\meter\per\second}}
        \wrongchoice{\SI{0.24}{\meter\per\second}}
        \wrongchoice{\SI{4.9}{\meter\per\second}}
      \correctchoice{\SI{6.9}{\meter\per\second}}
    \end{choices}
    \end{multicols}
\end{question}
}

\element{halliday-mc}{
\begin{question}{halliday-ch15-q37}
    A \SI{0.25}{\kilo\gram} block oscillates on the end of the spring with a spring constant of \SI{200}{\newton\per\meter}. 
    If the oscillation is started by elongating the spring \SI{0.15}{\meter} and giving the block a speed of \SI{3.0}{\meter\per\second},
        then the maximum speed of the block is:
    \begin{multicols}{3}
    \begin{choices}
        \wrongchoice{\SI{0.13}{\meter\per\second}}
        \wrongchoice{\SI{0.18}{\meter\per\second}}
        \wrongchoice{\SI{3.7}{\meter\per\second}}
      \correctchoice{\SI{5.2}{\meter\per\second}}
        \wrongchoice{\SI{13}{\meter\per\second}}
    \end{choices}
    \end{multicols}
\end{question}
}

\element{halliday-mc}{
\begin{question}{halliday-ch15-q38}
    A \SI{0.25}{\kilo\gram} block oscillates on the end of the spring with a spring constant of \SI{200}{\newton\per\meter}. 
    If the oscillation is started by elongating the spring \SI{0.15}{\meter} and giving the block a speed of \SI{3.0}{\meter\per\second},
        then the amplitude of the oscillation is:
    \begin{multicols}{3}
    \begin{choices}
        \wrongchoice{\SI{0.13}{\meter}}
      \correctchoice{\SI{0.18}{\meter}}
        \wrongchoice{\SI{3.7}{\meter}}
        \wrongchoice{\SI{5.2}{\meter}}
        \wrongchoice{\SI{13}{\meter}}
    \end{choices}
    \end{multicols}
\end{question}
}

\element{halliday-mc}{
\begin{question}{halliday-ch15-q39}
    An object on the end of a spring is set into oscillation by giving it an initial velocity while it is at its equilibrium position. 
    In the first trial the initial velocity is $v_0$ and in the second it is $4v_0$.
    In the second trial:
    \begin{choices}
        \wrongchoice{the amplitude is half as great and the maximum acceleration is twice as great}
        \wrongchoice{the amplitude is twice as great and the maximum acceleration is half as great}
      \correctchoice{both the amplitude and the maximum acceleration are twice as great}
        \wrongchoice{both the amplitude and the maximum acceleration are four times as great}
        \wrongchoice{the amplitude is four times as great and the maximum acceleration is twice as great}
    \end{choices}
\end{question}
}

\element{halliday-mc}{
\begin{question}{halliday-ch15-q40}
    A block attached to a spring undergoes simple harmonic motion on a horizontal frictionless surface.
    Its total energy is \SI{50}{\joule}.
    When the displacement is half the amplitude,
        the kinetic energy is:
    \begin{multicols}{3}
    \begin{choices}
        \wrongchoice{zero}
        \wrongchoice{\SI{12.5}{\joule}}
        \wrongchoice{\SI{25}{\joule}}
      \correctchoice{\SI{37.5}{\joule}}
        \wrongchoice{\SI{50}{\joule}}
    \end{choices}
    \end{multicols}
\end{question}
}

\element{halliday-mc}{
\begin{question}{halliday-ch15-q41}
    A mass-spring system is oscillating with amplitude $A$. 
    The kinetic energy will equal the potential energy only when the displacement is:
    \begin{choices}
        \wrongchoice{zero}
        \wrongchoice{$\pm \dfrac{A}{4}$}
      \correctchoice{$\pm \dfrac{A}{\sqrt{2}}$}
        \wrongchoice{$\pm \dfrac{A}{2}$}
        \wrongchoice{anywhere between $-A$ and $+A$}
    \end{choices}
\end{question}
}

\element{halliday-mc}{
\begin{question}{halliday-ch15-q42}
    If the length of a simple pendulum is doubled,
        its period will:
    \begin{choices}
        \wrongchoice{halve}
      \correctchoice{be greater by a factor $\sqrt{2}$}
        \wrongchoice{be less by a factor of $2$}
        \wrongchoice{double}
        \wrongchoice{remain the same}
    \end{choices}
\end{question}
}

\element{halliday-mc}{
\begin{question}{halliday-ch15-q43}
    The period of a simple pendulum is \SI{1}{\second} on Earth. 
    When brought to a planet where $g$ is one-tenth that on Earth,
        its period becomes:
    \begin{multicols}{3}
    \begin{choices}
        \wrongchoice{\SI{1}{\second}}
        \wrongchoice{\SI[parse-numbers=false]{\dfrac{1}{\sqrt{10}}}{\second}}
        \wrongchoice{\SI{1/10}{\second}}
      \correctchoice{\SI[parse-numbers=false]{\sqrt{10}}{\second}}
        \wrongchoice{\SI{10}{\second}}
    \end{choices}
    \end{multicols}
\end{question}
}

\element{halliday-mc}{
\begin{question}{halliday-ch15-q44}
    The amplitude of oscillation of a simple pendulum is increased from \ang{1} to \ang{4}.
    Its maximum acceleration changes by a factor of:
    \begin{multicols}{3}
    \begin{choices}
        \wrongchoice{\num{1/4}}
        \wrongchoice{\num{1/2}}
        \wrongchoice{\num{2}}
      \correctchoice{\num{4}}
        \wrongchoice{\num{16}}
    \end{choices}
    \end{multicols}
\end{question}
}

\element{halliday-mc}{
\begin{question}{halliday-ch15-q45}
    A simple pendulum of length $L$ and mass $M$ has frequency $f$.
    To increase its frequency to $2f$:
    \begin{choices}
        \wrongchoice{increase its length to $4L$}
        \wrongchoice{increase its length to $2L$}
        \wrongchoice{decrease its length to $\dfrac{L}{2}$}
      \correctchoice{decrease its length to $\dfrac{L}{4}$}
        \wrongchoice{decrease its mass to $<\dfrac{M}{4}$}
    \end{choices}
\end{question}
}

\element{halliday-mc}{
\begin{question}{halliday-ch15-q46}
    A simple pendulum consists of a small ball tied to a string and set in oscillation. 
    As the pendulum swings the tension force of the string is:
    \begin{choices}
        \wrongchoice{constant.}
        \wrongchoice{a sinusoidal function of time.}
        \wrongchoice{the square of a sinusoidal function of time.}
        \wrongchoice{the reciprocal of a sinusoidal function of time.}
      \correctchoice{none of the provided.}
    \end{choices}
\end{question}
}

\element{halliday-mc}{
\begin{question}{halliday-ch15-q47}
    A simple pendulum has length $L$ and period $T$. 
    As it passes through its equilibrium position,
        the string is suddenly clamped at its midpoint. 
    The period then becomes:
    \begin{multicols}{2}
    \begin{choices}
        \wrongchoice{$2T$}
        \wrongchoice{$T$}
        \wrongchoice{$\dfrac{T}{2}$}
        \wrongchoice{$\dfrac{T}{4}$}
      \correctchoice{none of the provided}
    \end{choices}
    \end{multicols}
\end{question}
}

\element{halliday-mc}{
\begin{question}{halliday-ch15-q48}
    A simple pendulum is suspended from the ceiling of an elevator. 
    The elevator is accelerating upwards with acceleration $a$. 
    The period of this pendulum,
        in terms of its length $L$, $g$, and $a$ is:
    \begin{multicols}{2}
    \begin{choices}
        \wrongchoice{$2\pi \sqrt{\dfrac{L}{g}}$}
      \correctchoice{$2\pi \sqrt{\dfrac{L}{g + a}}$}
        \wrongchoice{$2\pi \sqrt{\dfrac{L}{g - a}}$}
        \wrongchoice{$2\pi \sqrt{\dfrac{L}{a}}$}
        \wrongchoice{$\dfrac{1}{2\pi} \sqrt{\dfrac{g}{L}}$}
    \end{choices}
    \end{multicols}
\end{question}
}

\element{halliday-mc}{
\begin{question}{halliday-ch15-q49}
    Three physical pendulums, with masses $m_1$, $m_2 = 2m_1$, and $m_3 = 3m_1$,
        have the same shape and size and are suspended at the same point. 
    Rank them according to their periods,
        from shortest to longest.
    \begin{multicols}{2}
    \begin{choices}
        \wrongchoice{1, 2, 3}
        \wrongchoice{3, 2, 1}
        \wrongchoice{2, 3, 1}
        \wrongchoice{2, 1, 3}
      \correctchoice{All the same}
    \end{choices}
    \end{multicols}
\end{question}
}

\element{halliday-mc}{
\begin{question}{halliday-ch15-q50}
    Five hoops are each pivoted at a point on the rim and allowed to swing as physical pendulums.
    The masses and radii are:
    \begin{description}
        \itemsep=0pt
        \item[hoop 1:] $M=\SI{150}{\gram}$ and $R=\SI{50}{\centi\meter}$
        \item[hoop 2:] $M=\SI{200}{\gram}$ and $R=\SI{40}{\centi\meter}$
        \item[hoop 3:] $M=\SI{250}{\gram}$ and $R=\SI{30}{\centi\meter}$
        \item[hoop 4:] $M=\SI{300}{\gram}$ and $R=\SI{20}{\centi\meter}$
        \item[hoop 5:] $M=\SI{350}{\gram}$ and $R=\SI{10}{\centi\meter}$
    \end{description}
    Order the hoops according to the periods of their motions,
        smallest to largest.
    \begin{multicols}{2}
    \begin{choices}
        \wrongchoice{1, 2, 3, 4, 5}
      \correctchoice{5, 4, 3, 2, 1}
        \wrongchoice{1, 2, 3, 5, 4}
        \wrongchoice{1, 2, 5, 4, 3}
        \wrongchoice{5, 4, 1, 2, 3}
    \end{choices}
    \end{multicols}
\end{question}
}

\element{halliday-mc}{
\begin{question}{halliday-ch15-q51}
    A meter stick is pivoted at a point a distance $a$ from its center and swings as a physical pendulum.
    Of the following values for $a$,
        which results in the shortest period of oscillation?
    \begin{multicols}{2}
    \begin{choices}
        \wrongchoice{$a = \SI{0.1}{\meter}$}
        \wrongchoice{$a = \SI{0.2}{\meter}$}
      \correctchoice{$a = \SI{0.3}{\meter}$}
        \wrongchoice{$a = \SI{0.4}{\meter}$}
        \wrongchoice{$a = \SI{0.5}{\meter}$}
    \end{choices}
    \end{multicols}
\end{question}
}

\element{halliday-mc}{
\begin{question}{halliday-ch15-q52}
    The rotational inertia of a uniform thin rod about its end is $M L^2/3$,
    where $M$ is the mass and $L$ is the length. 
    Such a rod is hung vertically from one end and set into small amplitude oscillation. 
    If $L=\SI{1.0}{\meter}$ this rod will have the same period as a simple pendulum of length:
    \begin{multicols}{3}
    \begin{choices}
        \wrongchoice{\SI{33}{\centi\meter}}
        \wrongchoice{\SI{50}{\centi\meter}}
      \correctchoice{\SI{67}{\centi\meter}}
        \wrongchoice{\SI{100}{\centi\meter}}
        \wrongchoice{\SI{150}{\centi\meter}}
    \end{choices}
    \end{multicols}
\end{question}
}

\element{halliday-mc}{
\begin{question}{halliday-ch15-q53}
    Two uniform spheres are pivoted on horizontal axes that are tangent to their surfaces. 
    The one with the longer period of oscillation is the one with:
    \begin{choices}
        \wrongchoice{the larger mass.}
        \wrongchoice{the smaller mass.}
        \wrongchoice{the larger rotational inertia.}
        \wrongchoice{the smaller rotational inertia.}
      \correctchoice{the larger radius.}
    \end{choices}
\end{question}
}

\element{halliday-mc}{
\begin{question}{halliday-ch15-q54}
    The $x$ and $y$ coordinates of a point each execute simple harmonic motion. 
    The result might be a circular orbit if:
    \begin{choices}
        \wrongchoice{the amplitudes are the same but the frequencies are different.}
      \correctchoice{the amplitudes and frequencies are both the same.}
        \wrongchoice{the amplitudes and frequencies are both different.}
        \wrongchoice{the phase constants are the same but the amplitudes are different.}
        \wrongchoice{the amplitudes and the phase constants are both different.}
    \end{choices}
\end{question}
}

\element{halliday-mc}{
\begin{question}{halliday-ch15-q55}
    The $x$ and $y$ coordinates of a point each execute simple harmonic motion. 
    The frequencies are the same but the amplitudes are different. 
    The resulting orbit might be:
    \begin{multicols}{2}
    \begin{choices}
      \correctchoice{an ellipse}
        \wrongchoice{a circle}
        \wrongchoice{a parabola}
        \wrongchoice{a hyperbola}
        \wrongchoice{a square}
    \end{choices}
    \end{multicols}
\end{question}
}

\element{halliday-mc}{
\begin{question}{halliday-ch15-q56}
    For an oscillator subjected to a damping force proportional to its velocity:
    \begin{choices}
        \wrongchoice{the displacement is a sinusoidal function of time.}
        \wrongchoice{the velocity is a sinusoidal function of time.}
        \wrongchoice{the frequency is a decreasing function of time.}
        \wrongchoice{the mechanical energy is constant.}
      \correctchoice{none of the provided are true.}
    \end{choices}
\end{question}
}

\element{halliday-mc}{
\begin{question}{halliday-ch15-q57}
    Five particles undergo damped harmonic motion. 
    Values for the spring constant $k$, the damping constant $b$,
        and the mass $m$ are given below. 
    Which leads to the smallest rate of loss of mechanical energy?
    \begin{choices}
        \wrongchoice{\makebox[7em][l]{$k=\SI{100}{\newton\per\meter}$,} \makebox[5em][l]{$m=\SI{50}{\gram}$,} \makebox[5em][l]{$b=\SI{8}{\gram\per\second}$}}
      \correctchoice{\makebox[7em][l]{$k=\SI{150}{\newton\per\meter}$,} \makebox[5em][l]{$m=\SI{50}{\gram}$,} \makebox[5em][l]{$b=\SI{5}{\gram\per\second}$}}
        \wrongchoice{\makebox[7em][l]{$k=\SI{150}{\newton\per\meter}$,} \makebox[5em][l]{$m=\SI{10}{\gram}$,} \makebox[5em][l]{$b=\SI{8}{\gram\per\second}$}}
        \wrongchoice{\makebox[7em][l]{$k=\SI{200}{\newton\per\meter}$,} \makebox[5em][l]{$m=\SI{8}{\gram}$,}  \makebox[5em][l]{$b=\SI{6}{\gram\per\second}$}}
        \wrongchoice{\makebox[7em][l]{$k=\SI{100}{\newton\per\meter}$,} \makebox[5em][l]{$m=\SI{2}{\gram}$,}  \makebox[5em][l]{$b=\SI{4}{\gram\per\second}$}}
    \end{choices}
\end{question}
}

\element{halliday-mc}{
\begin{question}{halliday-ch15-q58}
    A sinusoidal force with a given amplitude is applied to an oscillator. 
    To maintain the largest amplitude oscillation the frequency of the applied force should be:
    \begin{choices}
        \wrongchoice{half the natural frequency of the oscillator.}
      \correctchoice{the same as the natural frequency of the oscillator.}
        \wrongchoice{twice the natural frequency of the oscillator.}
        \wrongchoice{unrelated to the natural frequency of the oscillator.}
        \wrongchoice{determined from the maximum speed desired.}
    \end{choices}
\end{question}
}

\element{halliday-mc}{
\begin{question}{halliday-ch15-q59}
    A sinusoidal force with a given amplitude is applied to an oscillator. 
    At resonance the amplitude of the oscillation is limited by:
    \begin{choices}
      \correctchoice{the damping force.}
        \wrongchoice{the initial amplitude.}
        \wrongchoice{the initial velocity.}
        \wrongchoice{the force of gravity.}
        \wrongchoice{none of the provided.}
    \end{choices}
\end{question}
}

\element{halliday-mc}{
\begin{question}{halliday-ch15-q60}
    An oscillator is subjected to a damping force that is proportional to its velocity. 
    A sinusoidal force is applied to it. 
    After a long time:
    \begin{choices}
        \wrongchoice{its amplitude is an increasing function of time.}
        \wrongchoice{its amplitude is a decreasing function of time.}
      \correctchoice{its amplitude is constant.}
        \wrongchoice{its amplitude is a decreasing function of time only if the damping constant is large.}
        \wrongchoice{its amplitude increases over some portions of a cycle and decreases over other portions.}
    \end{choices}
\end{question}
}

\element{halliday-mc}{
\begin{question}{halliday-ch15-q61}
    A block on a spring is subjected to a damping force that is proportional to its velocity and to an applied sinusoidal force.
    The energy dissipated by damping is supplied by:
    \begin{choices}
        \wrongchoice{the potential energy of the spring}
        \wrongchoice{the kinetic energy of the mass}
        \wrongchoice{gravity}
        \wrongchoice{friction}
      \correctchoice{the applied force}
    \end{choices}
\end{question}
}

\element{halliday-mc}{
\begin{question}{halliday-ch15-q62}
    The table below gives the values of the spring constant $k$,
        damping constant $b$, and mass $m$ for a particle in damped harmonic motion. 
    Which of these takes the longest time for its mechanical energy to decrease to one-fourth of its initial value?
    \begin{choices}
        \wrongchoice{\makebox[5em][l]{$k=k_0$,}     \makebox[5em][l]{$b=b_0$,}          \makebox[5em][l]{$m=m_0$}}
        \wrongchoice{\makebox[5em][l]{$k=3k_0$,}    \makebox[5em][l]{$b=2b_0$,}         \makebox[5em][l]{$m=m_0$}}
        \wrongchoice{\makebox[5em][l]{$k=\dfrac{k_0}{2}$,} \makebox[5em][l]{$b=6b_0$,}  \makebox[5em][l]{$m=2m_0$}}
        \wrongchoice{\makebox[5em][l]{$k=4k_0$,}    \makebox[5em][l]{$b=b_0$,}          \makebox[5em][l]{$m=2m_0$}}
      \correctchoice{\makebox[5em][l]{$k=k_0$,}     \makebox[5em][l]{$b=b_0$,}          \makebox[5em][l]{$m=10m_0$}}
    \end{choices}
\end{question}
}


\endinput


