
%%--------------------------------------------------
%% Halliday: Fundamentals of Physics
%%--------------------------------------------------


%% Chapter 18: Temperature, Heat,
%%    and the First Law of Thermodynamics
%%--------------------------------------------------


%% Learning Objectives
%%--------------------------------------------------

%% 18.01: Identify the lowest temperature as 0 on the Kelvin scale (absolute zero).
%% 18.02: Explain the zeroth law of thermodynamics.
%% 18.03: Explain the conditions for the triple-point temperature.
%% 18.04: Explain the conditions for measuring a temperature with a constant-volume gas thermometer.
%% 18.05: For a constant-volume gas thermometer, relate the pressure and temperature of the gas in some given state to the pressure and temperature at the triple point.


%% Halliday Multiple Choice Questions
%%--------------------------------------------------
\element{halliday-mc}{
\begin{question}{halliday-ch18-q01}
    If two objects are in thermal equilibrium with each other:
    \begin{choices}
        \wrongchoice{they cannot be moving}
        \wrongchoice{they cannot be undergoing an elastic collision}
        \wrongchoice{they cannot have different pressures}
      \correctchoice{they cannot be at different temperatures}
        \wrongchoice{they cannot be falling in Earth’s gravitational field}
    \end{choices}
\end{question}
}

\element{halliday-mc}{
\begin{question}{halliday-ch18-q02}
    When two gases separated by a diathermal wall are in thermal equilibrium with each other:
    \begin{choices}
        \wrongchoice{only their pressures must be the same}
        \wrongchoice{only their volumes must be the same}
        \wrongchoice{they must have the same number of particles}
        \wrongchoice{they must have the same pressure and the same volume}
      \correctchoice{only their temperatures must be the same}
    \end{choices}
\end{question}
}

\element{halliday-mc}{
\begin{question}{halliday-ch18-q03}
    A balloon is filled with cold air and placed in a warm room.
    It is \emph{not} in thermal equilibrium with the air of the room until:
    \begin{choices}
        \wrongchoice{it rises to the ceiling}
        \wrongchoice{it sinks to the floor}
      \correctchoice{it stops expanding}
        \wrongchoice{it starts to contract}
        \wrongchoice{none of the provided}
    \end{choices}
\end{question}
}

\element{halliday-mc}{
\begin{question}{halliday-ch18-q04}
    Suppose object $C$ is in thermal equilibrium with object $A$ and with object $B$.
    The zeroth law of thermodynamics states:
    \begin{choices}
        \wrongchoice{that $C$ will always be in thermal equilibrium with both $A$ and $B$}
        \wrongchoice{that $C$ must transfer energy to both $A$ and $B$}
      \correctchoice{that $A$ is in thermal equilibrium with $B$}
        \wrongchoice{that $A$ cannot be in thermal equilibrium with $B$}
        \wrongchoice{nothing about the relationship between $A$ and $B$}
    \end{choices}
\end{question}
}

\element{halliday-mc}{
\begin{question}{halliday-ch18-q05}
    The zeroth law of thermodynamics allows us to define:
    \begin{choices}
        \wrongchoice{work}
        \wrongchoice{pressure}
      \correctchoice{temperature}
        \wrongchoice{thermal equilibrium}
        \wrongchoice{internal energy}
    \end{choices}
\end{question}
}

\element{halliday-mc}{
\begin{question}{halliday-ch18-q06}
    If the zeroth law of thermodynamics were not valid,
        which of the following could not be considered a property of an object?
    \begin{choices}
        \wrongchoice{Pressure}
        \wrongchoice{Center of mass energy}
        \wrongchoice{Internal energy}
        \wrongchoice{Momentum}
      \correctchoice{Temperature}
    \end{choices}
\end{question}
}

\element{halliday-mc}{
\begin{question}{halliday-ch18-q07}
    The international standard thermometer is kept:
    \begin{choices}
        \wrongchoice{near Washington, D.C.}
        \wrongchoice{near Paris, France}
        \wrongchoice{near the north pole}
        \wrongchoice{near Rome, Italy}
      \correctchoice{nowhere (there is none)}
    \end{choices}
\end{question}
}

\element{halliday-mc}{
\begin{question}{halliday-ch18-q08}
    In constructing a thermometer it is \emph{necessary} to use a substance that:
    \begin{choices}
        \wrongchoice{expands with rising temperature}
        \wrongchoice{expands linearly with rising temperature}
        \wrongchoice{will not freeze}
        \wrongchoice{will not boil}
        \wrongchoice{undergoes some change when heated or cooled}
    \end{choices}
\end{question}
}

\element{halliday-mc}{
\begin{question}{halliday-ch18-q09}
    The ``triple point'' of a substance is that point for which the temperature and pressure are such that:
    \begin{choices}
        \wrongchoice{only solid and liquid are in equilibrium}
        \wrongchoice{only liquid and vapor are in equilibrium}
        \wrongchoice{only solid and vapor are in equilibrium}
      \correctchoice{solid, liquid, and vapor are all in equilibrium}
        \wrongchoice{the temperature, pressure and density are all numerically equal}
    \end{choices}
\end{question}
}

\element{halliday-mc}{
\begin{question}{halliday-ch18-q10}
    Constant-volume gas thermometers using different gases all indicate nearly the same temperature when in contact with the same object if:
    \begin{choices}
        \wrongchoice{the volumes are all extremely large}
        \wrongchoice{the volumes are all the same}
        \wrongchoice{the pressures are all extremely large}
        \wrongchoice{the pressures are the same}
      \correctchoice{the particle concentrations are all extremely small}
    \end{choices}
\end{question}
}

\element{halliday-mc}{
\begin{question}{halliday-ch18-q11}
    A constant-volume gas thermometer is used to measure the temperature of an object.
    When the thermometer is in contact with water at its triple point (\SI{273.16}{\kelvin}) the pressure in the thermometer is \SI{8.500e4}{\pascal}.
    When it is in contact with the object the pressure is \SI{9.650e4}{\pascal}.
    The temperature of the object is:
    \begin{multicols}{3}
    \begin{choices}
        \wrongchoice{\SI{37.0}{\kelvin}}
        \wrongchoice{\SI{241}{\kelvin}}
      \correctchoice{\SI{310}{\kelvin}}
        \wrongchoice{\SI{314}{\kelvin}}
        \wrongchoice{\SI{2020}{\kelvin}}
    \end{choices}
    \end{multicols}
\end{question}
}

\element{halliday-mc}{
\begin{question}{halliday-ch18-q12}
    When a certain constant-volume gas thermometer is in thermal contact with water at its triple point (\SI{273.16}{\kelvin}) the pressure is \SI{6.30e4}{\pascal}.
    For this thermometer a kelvin corresponds to a change in pressure of about:
    \begin{multicols}{2}
    \begin{choices}
        \wrongchoice{\SI{4.34e2}{\pascal}}
      \correctchoice{\SI{2.31e2}{\pascal}}
        \wrongchoice{\SI{1.72e3}{\pascal}}
        \wrongchoice{\SI{2.31e3}{\pascal}}
        \wrongchoice{\SI{1.72e7}{\pascal}}
    \end{choices}
    \end{multicols}
\end{question}
}

\element{halliday-mc}{
\begin{question}{halliday-ch18-q13}
    The diagram shows four thermometers, labeled $W$, $X$, $Y$, and $Z$.
    The freezing and boiling points of water are indicated.
    \begin{center}
    \begin{tikzpicture}
        %% NOTE:
    \end{tikzpicture}
    \end{center}
    Rank the thermometers according to the size of a degree on their scales,
        smallest to largest.
    \begin{multicols}{2}
    \begin{choices}
        \wrongchoice{$W$, $X$, $Y$, $Z$}
        \wrongchoice{$Z$, $Y$, $X$, $W$}
        \wrongchoice{$Z$, $Y$, $W$, $X$}
      \correctchoice{$Z$, $X$, $W$, $Y$}
        \wrongchoice{$W$, $Y$, $Z$, $X$}
    \end{choices}
    \end{multicols}
\end{question}
}

\element{halliday-mc}{
\begin{question}{halliday-ch18-q14}
    There is a temperature at which the reading on the Kelvin scale is numerically:
    \begin{choices}
        \wrongchoice{equal to that on the Celsius scale}
        \wrongchoice{lower than that on the Celsius scale}
      \correctchoice{equal to that on the Fahrenheit scale}
        \wrongchoice{less than zero}
        %% NOTE: questionmult??
        \wrongchoice{none of the provided}
    \end{choices}
\end{question}
}

\element{halliday-mc}{
\begin{question}{halliday-ch18-q15}
    Fahrenheit and Kelvin scales agree numerically at a reading of:
    \begin{multicols}{3}
    \begin{choices}
        \wrongchoice{\num{-40}}
        \wrongchoice{\num{0}}
        \wrongchoice{\num{273}}
        \wrongchoice{\num{301}}
      \correctchoice{\num{574}}
    \end{choices}
    \end{multicols}
\end{question}
}

\element{halliday-mc}{
\begin{question}{halliday-ch18-q16}
    Which one of the following statements is true?
    \begin{choices}
        \wrongchoice{Temperatures differing by \ang{25} on the Fahrenheit scale must differ by \ang{45} on the Celsius scale}
        \wrongchoice{\SI{40}{\kelvin} corresponds to \SI{-40}{\degreeCelsius}}
      \correctchoice{Temperatures which differ by \ang{10} on the Celsius scale must differ by \ang{18} on the Fahrenheit scale}
        \wrongchoice{Water at \SI{90}{\degreeCelsius} is warmer than water at \SI{202}{\degree\Fahrenheit}}
        \wrongchoice{\SI{0}{\degree\Fahrenheit} corresponds to \SI{-32}{\degreeCelsius}}
    \end{choices}
\end{question}
}

\element{halliday-mc}{
\begin{question}{halliday-ch18-q17}
    A Kelvin thermometer and a Fahrenheit thermometer both give the same reading for a certain sample.
    The corresponding Celsius temperature is:
    \begin{multicols}{3}
    \begin{choices}
        \wrongchoice{\SI{574}{\degreeCelsius}}
        \wrongchoice{\SI{232}{\degreeCelsius}}
      \correctchoice{\SI{301}{\degreeCelsius}}
        \wrongchoice{\SI{614}{\degreeCelsius}}
        \wrongchoice{\SI{276}{\degreeCelsius}}
    \end{choices}
    \end{multicols}
\end{question}
}

\element{halliday-mc}{
\begin{question}{halliday-ch18-q18}
    Room temperature is about 20 degrees on the:
    \begin{choices}
        \wrongchoice{Kelvin scale}
      \correctchoice{Celsius scale}
        \wrongchoice{Fahrenheit scale}
        \wrongchoice{absolute scale}
        \wrongchoice{C major scale}
    \end{choices}
\end{question}
}

\element{halliday-mc}{
\begin{question}{halliday-ch18-q19}
    A thermometer indicates \SI{98.6}{\degreeCelsius}.
    It may be:
    \begin{choices}
        \wrongchoice{outdoors on a cold day}
        \wrongchoice{in a comfortable room}
      \correctchoice{in a cup of hot tea}
        \wrongchoice{in a normal person's mouth}
        \wrongchoice{in liquid air}
    \end{choices}
\end{question}
}

\element{halliday-mc}{
\begin{question}{halliday-ch18-q20}
    The air temperature on a summer day might be about:
    \begin{multicols}{3}
    \begin{choices}
        \wrongchoice{\SI{0}{\degreeCelsius}}
        \wrongchoice{\SI{10}{\degreeCelsius}}
      \correctchoice{\SI{25}{\degreeCelsius}}
        \wrongchoice{\SI{80}{\degreeCelsius}}
        \wrongchoice{\SI{125}{\degreeCelsius}}
    \end{choices}
    \end{multicols}
\end{question}
}

\element{halliday-mc}{
\begin{question}{halliday-ch18-q21}
    The two metallic strips that constitute some thermostats must differ in:
    \begin{choices}
        \wrongchoice{length}
        \wrongchoice{thickness}
        \wrongchoice{mass}
        \wrongchoice{rate at which they conduct heat}
      \correctchoice{coefficient of linear expansion}
    \end{choices}
\end{question}
}

\element{halliday-mc}{
\begin{question}{halliday-ch18-q22}
    Thin strips of iron and zinc are riveted together to form a bimetallic strip that bends when heated.
    The iron is on the inside of the bend because:
    \begin{choices}
        \wrongchoice{it has a higher coefficient of linear expansion}
      \correctchoice{it has a lower coefficient of linear expansion}
        \wrongchoice{it has a higher specific heat}
        \wrongchoice{it has a lower specific heat}
        \wrongchoice{it conducts heat better}
    \end{choices}
\end{question}
}

\element{halliday-mc}{
\begin{question}{halliday-ch18-q23}
    It is more difficult to measure the coefficient of volume expansion of a liquid than that of a solid because:
    \begin{choices}
        \wrongchoice{no relation exists between linear and volume expansion coefficients}
        \wrongchoice{a liquid tends to evaporate}
        \wrongchoice{a liquid expands too much when heated}
        \wrongchoice{a liquid expands too little when heated}
      \correctchoice{the containing vessel also expands}
    \end{choices}
\end{question}
}

\element{halliday-mc}{
\begin{question}{halliday-ch18-q24}
    A surveyor's \SI{30}{\meter} steel tape is correct at \SI{68}{\degree\Fahrenheit}.
    On a hot day the tape has expanded to \SI{30.01}{\meter}.
    On that day, the tape indicates a distance of \SI{15.52}{\meter} between two points.
    The true distance between these points is:
    \begin{multicols}{3}
    \begin{choices}
        \wrongchoice{\SI{15.50}{\meter}}
      \correctchoice{\SI{15.51}{\meter}}
        \wrongchoice{\SI{15.52}{\meter}}
        \wrongchoice{\SI{15.53}{\meter}}
        \wrongchoice{\SI{15.54}{\meter}}
    \end{choices}
    \end{multicols}
\end{question}
}

\element{halliday-mc}{
\begin{question}{halliday-ch18-q25}
    The figure shows a rectangular brass plate at \SI{0}{\degreeCelsius} in which there is cut a rectangular hole of dimensions indicated.
    \begin{center}
    \begin{tikzpicture}
        %% NOTE:
    \end{tikzpicture}
    \end{center}
    If the temperature of the plate is raised to \SI{150}{\degreeCelsius}.
    \begin{choices}
        \wrongchoice{$x$ will increase and $y$ will decrease}
        \wrongchoice{both $x$ and $y$ will decrease}
        \wrongchoice{$x$ will decrease and $y$ will increase}
      \correctchoice{both $x$ and $y$ will increase}
        \wrongchoice{the changes in $x$ and $y$ depend on the dimension $z$}
    \end{choices}
\end{question}
}

\element{halliday-mc}{
\begin{question}{halliday-ch18-q26}
    The Stanford linear accelerator contains hundreds of brass disks tightly fitted into a steel tube (see figure).
    The coefficient of linear expansion of the brass is \SI{2.00e-5}{\per\degreeCelsius}.
    The system was assembled by cooling the disks in dry ice (\SI{-57}{\degreeCelsius}) to enable them to just slide into the close-fitting tube.
    \begin{center}
    \begin{tikzpicture}
        %% NOTE:
    \end{tikzpicture}
    \end{center}
    If the diameter of a disk is \SI{80.00}{\milli\meter} at \SI{43}{\degreeCelsius},
        what is its diameter in the dry ice?
    \begin{multicols}{2}
    \begin{choices}
        \wrongchoice{\SI{78.40}{\milli\meter}}
        \wrongchoice{\SI{79.68}{\milli\meter}}
        \wrongchoice{\SI{80.16}{\milli\meter}}
      \correctchoice{\SI{79.84}{\milli\meter}}
        \wrongchoice{None of the provided}
    \end{choices}
    \end{multicols}
\end{question}
}

\element{halliday-mc}{
\begin{question}{halliday-ch18-q27}
    When the temperature of a copper penny is increased by \SI{100}{\degreeCelsius},
        its diameter increases by \SI{0.17}{\percent}.
    The area of one of its faces increases by:
    \begin{multicols}{3}
    \begin{choices}
        \wrongchoice{\SI{0.17}{\percent}}
      \correctchoice{\SI{0.34}{\percent}}
        \wrongchoice{\SI{0.51}{\percent}}
        \wrongchoice{\SI{0.13}{\percent}}
        \wrongchoice{\SI{0.27}{\percent}}
    \end{choices}
    \end{multicols}
\end{question}
}

\element{halliday-mc}{
\begin{question}{halliday-ch18-q28}
    An annular ring of aluminum is cut from an aluminum sheet as shown.
        %% NOTE:
    When this ring is heated:
    \begin{choices}
        \wrongchoice{the aluminum expands outward and the hole remains the same in size}
        \wrongchoice{the hole decreases in diameter}
        \wrongchoice{the area of the hole expands the same percent as any area of the aluminum}
        \wrongchoice{the area of the hole expands a greater percent than any area of the aluminum}
        \wrongchoice{linear expansion forces the shape of the hole to be slightly elliptical}
    \end{choices}
\end{question}
}

\element{halliday-mc}{
\begin{question}{halliday-ch18-q29}
    Possible units for the coefficient of volume expansion are:
    \begin{choices}
        \wrongchoice{millimeter per degree Celsius (\si{\milli\meter\per\degreeCelsius})}
        \wrongchoice{millimeter cubed per degree Celsius (\si{\milli\meter\cubed\per\degreeCelsius})}
        \wrongchoice{degree Celsius cubed (\si{\degreeCelsius\cubed})}
        \wrongchoice{per degree Celsius cubed (\si{\per\degreeCelsius\cubed})}
      \correctchoice{degree Celsius (\si{\per\degreeCelsius})}
    \end{choices}
\end{question}
}

\element{halliday-mc}{
\begin{question}{halliday-ch18-q30}
    The mercury column in an ordinary medical thermometer doubles in length when its temperature changes from \SI{95}{\degree\Fahrenheit} to \SI{105}{\degree\Fahrenheit}.
    Choose the correct statement:
    \begin{choices}
        \wrongchoice{the coefficient of volume expansion of mercury is \SI{0.1}{\per\degree\Fahrenheit}}
        \wrongchoice{the coefficient of volume expansion of mercury is \SI{0.3}{\per\degree\Fahrenheit}}
        \wrongchoice{the coefficient of volume expansion of mercury is \SI{1/30}{\per\degree\Fahrenheit}}
        \wrongchoice{the vacuum above the column helps to ``pull up'' the mercury this large amount}
      \correctchoice{none of the provided is true}
    \end{choices}
\end{question}
}

\element{halliday-mc}{
\begin{question}{halliday-ch18-q31}
    The coefficient of linear expansion of iron is \SI{1.0e-5}{\per\degreeCelsius}.
    The surface area of an iron cube,
        with an edge length of \SI{5.0}{\centi\meter},
        will increase by what amount if it is heated from \SI{10}{\degreeCelsius} to \SI{60}{\degreeCelsius}?
    \begin{multicols}{2}
    \begin{choices}
        \wrongchoice{\SI{0.0125}{\centi\meter\squared}}
        \wrongchoice{\SI{0.025}{\centi\meter\squared}}
        \wrongchoice{\SI{0.075}{\centi\meter\squared}}
      \correctchoice{\SI{0.15}{\centi\meter\squared}}
        \wrongchoice{\SI{0.30}{\centi\meter\squared}}
    \end{choices}
    \end{multicols}
\end{question}
}

\element{halliday-mc}{
\begin{question}{halliday-ch18-q32}
    The diagram shows four rectangular plates and their dimensions.
    All are made of the same material.
    \begin{center}
    \begin{tikzpicture}
        %% NOTE:
    \end{tikzpicture}
    \end{center}
    The temperature now increases.
    Of these plates:
    \begin{choices}
        \wrongchoice{the vertical dimension of plate 1 increases the most and the area of plate 1 increases the most}
        \wrongchoice{the vertical dimension of plate 2 increases the most and the area of plate 4 increases the most}
        \wrongchoice{the vertical dimension of plate 3 increases the most and the area of plate 1 increases the most}
      \correctchoice{the vertical dimension of plate 4 increases the most and the area of plate 3 increases the most}
        \wrongchoice{the vertical dimension of plate 4 increases the most and the area of plate 4 increases the most}
    \end{choices}
\end{question}
}

\element{halliday-mc}{
\begin{question}{halliday-ch18-q33}
    The coefficient of linear expansion of steel is \SI{11e-6}{\per\degreeCelsius}.
    A steel ball has a volume of exactly \SI{100}{\centi\meter\cubed} at \SI{0}{\degreeCelsius}.
    When heated to \SI{100}{\degreeCelsius},
        its volume becomes:
    \begin{multicols}{2}
    \begin{choices}
      \correctchoice{\SI{100.33}{\centi\meter\cubed}}
        \wrongchoice{\SI{100.0011}{\centi\meter\cubed}}
        \wrongchoice{\SI{100.0033}{\centi\meter\cubed}}
        \wrongchoice{\SI{100.000011}{\centi\meter\cubed}}
        \wrongchoice{none of the provided}
    \end{choices}
    \end{multicols}
\end{question}
}

\element{halliday-mc}{
\begin{question}{halliday-ch18-q34}
    The coefficient of linear expansion of a certain steel is \SI{0.000012}{\per\degreeCelsius}.
    The coefficient of volume expansion, in \si{\per\degreeCelsius}, is:
    \begin{choices}
        \wrongchoice{$\left(0.000\,012\right)^3$}
        \wrongchoice{$\dfrac{4\pi}{3}\left(0.000\,012\right)^3$}
      \correctchoice{$3\times 0.000\,012$}
        \wrongchoice{$0.000\,012$}
        \wrongchoice{depends on the shape of the volume to which it will be applied}
    \end{choices}
\end{question}
}

\element{halliday-mc}{
\begin{question}{halliday-ch18-q35}
    Metal pipes, used to carry water,
        sometimes burst in the winter because:
    \begin{choices}
        \wrongchoice{metal contracts more than water}
        \wrongchoice{outside of the pipe contracts more than the inside}
        \wrongchoice{metal becomes brittle when cold}
        \wrongchoice{ice expands when it melts}
      \correctchoice{water expands when it freezes}
    \end{choices}
\end{question}
}

\element{halliday-mc}{
\begin{question}{halliday-ch18-q36}
    A gram of distilled water at \SI{4}{\degreeCelsius}:
    \begin{choices}
        \wrongchoice{will increase slightly in weight when heated to \SI{6}{\degreeCelsius}}
        \wrongchoice{will decrease slightly in weight when heated to \SI{6}{\degreeCelsius}}
        \wrongchoice{will increase slightly in volume when heated to \SI{6}{\degreeCelsius}}
      \correctchoice{will decrease slightly in volume when heated to \SI{6}{\degreeCelsius}}
        \wrongchoice{will not change in either volume or weight}
    \end{choices}
\end{question}
}

\element{halliday-mc}{
\begin{question}{halliday-ch18-q37}
    Heat is:
    \begin{choices}
      \correctchoice{energy transferred by virtue of a temperature difference}
        \wrongchoice{energy transferred by macroscopic work}
        \wrongchoice{energy content of an object}
        \wrongchoice{a temperature difference}
        \wrongchoice{a property objects have by virtue of their temperatures}
    \end{choices}
\end{question}
}

\element{halliday-mc}{
\begin{question}{halliday-ch18-q38}
    Heat has the same units as:
    \begin{choices}
        \wrongchoice{temperature}
      \correctchoice{work}
        \wrongchoice{energy/time}
        \wrongchoice{heat capacity}
        \wrongchoice{energy/volume}
    \end{choices}
\end{question}
}

\element{halliday-mc}{
\begin{question}{halliday-ch18-q39}
    A calorie is about:
    \begin{multicols}{3}
    \begin{choices}
        \wrongchoice{\SI{0.24}{\joule}}
        \wrongchoice{\SI{8.3}{\joule}}
        \wrongchoice{\SI{250}{\joule}}
      \correctchoice{\SI{4.2}{\joule}}
        \wrongchoice{\SI{4200}{\joule}}
    \end{choices}
    \end{multicols}
\end{question}
}

\element{halliday-mc}{
\begin{question}{halliday-ch18-q40}
    The heat capacity of an object is:
    \begin{choices}
      \correctchoice{the amount of heat energy that raises its temperature by \SI{1}{\degreeCelsius}}
        \wrongchoice{the amount of heat energy that changes its state without changing its temperature}
        \wrongchoice{the amount of heat energy per kilogram that raises its temperature by \SI{1}{\degreeCelsius}}
        \wrongchoice{the ratio of its specific heat to that of water}
        \wrongchoice{the change in its temperature caused by adding \SI{1}{\joule} of heat}
    \end{choices}
\end{question}
}

\element{halliday-mc}{
\begin{question}{halliday-ch18-q41}
    The specific heat of a substance is:
    \begin{choices}
        \wrongchoice{the amount of heat energy to change the state of one gram of the substance}
        \wrongchoice{the amount of heat energy per unit mass emitted by oxidizing the substance}
        \wrongchoice{the amount of heat energy per unit mass to raise the substance from its freezing to its boiling point}
      \correctchoice{the amount of heat energy per unit mass to raise the temperature of the substance by \SI{1}{\degreeCelsius}}
        \wrongchoice{the temperature of the object divided by its mass}
    \end{choices}
\end{question}
}

\element{halliday-mc}{
\begin{question}{halliday-ch18-q42}
    Two different samples have the same mass and temperature.
    Equal quantities of energy are absorbed as heat by each.
    Their final temperatures may be different because the samples have different:
    \begin{choices}
        \wrongchoice{thermal conductivities}
        \wrongchoice{coefficients of expansion}
        \wrongchoice{densities}
        \wrongchoice{volumes}
        \wrongchoice{heat capacities}
    \end{choices}
\end{question}
}

\element{halliday-mc}{
\begin{question}{halliday-ch18-q43}
    The same energy $Q$ enters five different substances as heat.
    Which substance has the greatest specific heat?
    \begin{choices}
        \wrongchoice{The temperature of \SI{3}{\gram} of substance increases by \SI{10}{\kelvin}}
      \correctchoice{The temperature of \SI{4}{\gram} of substance increases by \SI{4}{\kelvin}}
        \wrongchoice{The temperature of \SI{6}{\gram} of substance increases by \SI{15}{\kelvin}}
        \wrongchoice{The temperature of \SI{8}{\gram} of substance increases by \SI{6}{\kelvin}}
        \wrongchoice{The temperature of \SI{10}{\gram} of substance increases by \SI{10}{\kelvin}}
    \end{choices}
\end{question}
}

\element{halliday-mc}{
\begin{question}{halliday-ch18-q44}
    For constant-volume processes the heat capacity of gas $A$ is greater than the heat capacity of gas $B$.
    We conclude that when they both absorb the same energy as heat at constant volume:
    \begin{choices}
        \wrongchoice{the temperature of $A$ increases more than the temperature of $B$}
      \correctchoice{the temperature of $B$ increases more than the temperature of $A$}
        \wrongchoice{the internal energy of $A$ increases more than the internal energy of $B$}
        \wrongchoice{the internal energy of $B$ increases more than the internal energy of $A$}
        \wrongchoice{$A$ does more positive work than $B$}
    \end{choices}
\end{question}
}

\element{halliday-mc}{
\begin{question}{halliday-ch18-q45}
    The heat capacity at constant volume and the heat capacity at constant pressure have different values because:
    \begin{choices}
        \wrongchoice{heat increases the temperature at constant volume but not at constant pressure}
        \wrongchoice{heat increases the temperature at constant pressure but not at constant volume}
        \wrongchoice{the system does work at constant volume but not at constant pressure}
      \correctchoice{the system does work at constant pressure but not at constant volume}
        \wrongchoice{the system does more work at constant volume than at constant pressure}
    \end{choices}
\end{question}
}

\element{halliday-mc}{
\begin{question}{halliday-ch18-q46}
    A cube of aluminum has an edge length of \SI{20}{\centi\meter}.
    Aluminum has a density \num{2.7} times that of water (\SI{1}{\gram\per\centi\meter\cubed}) and a specific heat \num{0.217} times that of water (\SI{1}{\calorie\per\gram\degreeCelsius}).
    When the internal energy of the cube increases by \SI{47000}{\calorie} its temperature increases by:
    \begin{multicols}{3}
    \begin{choices}
        \wrongchoice{\SI{5}{\degreeCelsius}}
      \correctchoice{\SI{10}{\degreeCelsius}}
        \wrongchoice{\SI{20}{\degreeCelsius}}
        \wrongchoice{\SI{100}{\degreeCelsius}}
        \wrongchoice{\SI{200}{\degreeCelsius}}
    \end{choices}
    \end{multicols}
\end{question}
}

\element{halliday-mc}{
\begin{question}{halliday-ch18-q47}
    An insulated container, filled with water,
        contains a thermometer and a paddle wheel.
    The paddle wheel can be rotated by an external source.
    This apparatus can be used to determine:
    \begin{choices}
        \wrongchoice{specific heat of water}
        \wrongchoice{relation between kinetic energy and absolute temperature}
        \wrongchoice{thermal conductivity of water}
        \wrongchoice{efficiency of changing work into heat}
      \correctchoice{mechanical equivalent of heat}
    \end{choices}
\end{question}
}

\element{halliday-mc}{
\begin{question}{halliday-ch18-q48}
    Take the mechanical equivalent of heat as \SI{4}{\joule\per\calorie}.
    A 10-g bullet moving at \SI{2000}{\meter\per\second} plunges into \SI{1}{\kilo\gram} of paraffin wax (specific heat \SI{0.7}{\calorie\per\gram\per\degreeCelsius}).
    The wax was initially at \SI{20}{\degreeCelsius}.
    Assuming that all the bullet's energy heats the wax,
        its final temperature is:
    \begin{multicols}{3}
    \begin{choices}
        \wrongchoice{\SI{20.14}{\degreeCelsius}}
        \wrongchoice{\SI{23.5}{\degreeCelsius}}
        \wrongchoice{\SI{20.006}{\degreeCelsius}}
      \correctchoice{\SI{27.1}{\degreeCelsius}}
        \wrongchoice{\SI{30.23}{\degreeCelsius}}
    \end{choices}
    \end{multicols}
\end{question}
}

\element{halliday-mc}{
\begin{question}{halliday-ch18-q49}
    The energy given off as heat by \SI{300}{\gram} of an alloy as it cools through \SI{50}{\degreeCelsius} raises the temperature of \SI{300}{\gram} of water from \SI{30}{\degreeCelsius} to \SI{40}{\degreeCelsius}.
    The specific heat of the alloy is:
    \begin{multicols}{2}
    \begin{choices}
        \wrongchoice{\SI{0.015}{\calorie\per\gram\per\degreeCelsius}}
        \wrongchoice{\SI{0.10}{\calorie\per\gram\per\degreeCelsius}}
        \wrongchoice{\SI{0.15}{\calorie\per\gram\per\degreeCelsius}}
      \correctchoice{\SI{0.20}{\calorie\per\gram\per\degreeCelsius}}
        \wrongchoice{\SI{0.50}{\calorie\per\gram\per\degreeCelsius}}
    \end{choices}
    \end{multicols}
\end{question}
}

\element{halliday-mc}{
\begin{question}{halliday-ch18-q50}
    The specific heat of lead is \SI{0.030}{\calorie\per\gram\per\degreeCelsius}.
    \SI{300}{\gram} of lead shot at \SI{100}{\degreeCelsius} is mixed with \SI{100}{\gram} of water at \SI{70}{\degreeCelsius} in an insulated container.
    The final temperature of the mixture is:
    \begin{multicols}{3}
    \begin{choices}
        \wrongchoice{\SI{100}{\degreeCelsius}}
        \wrongchoice{\SI{85.5}{\degreeCelsius}}
        \wrongchoice{\SI{79.5}{\degreeCelsius}}
        \wrongchoice{\SI{74.5}{\degreeCelsius}}
      \correctchoice{\SI{72.5}{\degreeCelsius}}
    \end{choices}
    \end{multicols}
\end{question}
}

\element{halliday-mc}{
\begin{question}{halliday-ch18-q51}
    Object $A$, with heat capacity $C_A$ and initially at temperature $T_A$,
        is placed in thermal contact with object $B$,
        with heat capacity $C_B$ and initially at temperature $T_B$.
    The combination is thermally isolated.
    If the heat capacities are independent of the temperature and no phase changes occur,
        the final temperature of both objects is:
    \begin{multicols}{2}
    \begin{choices}
        \wrongchoice{$\dfrac{C_A T_A - C_B T_B}{C_A + C_B}$}
      \correctchoice{$\dfrac{C_A T_A + C_B T_B}{C_A + C_B}$}
        \wrongchoice{$\dfrac{C_A T_A - C_B T_B}{C_A - C_B}$}
        \wrongchoice{$\left(C_A - C_B\right) |T_A - T_B|$}
        \wrongchoice{$\left(C_A + C_B\right) |T_A - T_B|$}
    \end{choices}
    \end{multicols}
\end{question}
}

\element{halliday-mc}{
\begin{question}{halliday-ch18-q52}
    The heat capacity of object $B$ is twice that of object $A$.
    Initially $A$ is at \SI{300}{\kelvin} and $B$ is at \SI{450}{\kelvin}.
    They are placed in thermal contact and the combination is isolated.
    The final temperature of both objects is:
    \begin{multicols}{3}
    \begin{choices}
        \wrongchoice{\SI{200}{\kelvin}}
        \wrongchoice{\SI{300}{\kelvin}}
      \correctchoice{\SI{400}{\kelvin}}
        \wrongchoice{\SI{450}{\kelvin}}
        \wrongchoice{\SI{600}{\kelvin}}
    \end{choices}
    \end{multicols}
\end{question}
}

\element{halliday-mc}{
\begin{question}{halliday-ch18-q53}
    A heat of transformation of a substance is:
    \begin{choices}
        \wrongchoice{the energy absorbed as heat during a phase transformation}
      \correctchoice{the energy per unit mass absorbed as heat during a phase transformation}
        \wrongchoice{the same as the heat capacity}
        \wrongchoice{the same as the specific heat}
        \wrongchoice{the same as the molar specific heat}
    \end{choices}
\end{question}
}

\element{halliday-mc}{
\begin{question}{halliday-ch18-q54}
    The heat of fusion of water is \SI{80}{\calorie\per\gram}.
    This means \SI{80}{\calorie} of energy are required to:
    \begin{choices}
        \wrongchoice{raise the temperature of \SI{1}{\gram} of water by \SI{1}{\kelvin}}
        \wrongchoice{turn \SI{1}{\gram} of water to steam}
        \wrongchoice{raise the temperature of \SI{1}{\gram} of ice by \SI{1}{\kelvin}}
      \correctchoice{melt \SI{1}{\gram} of ice}
        \wrongchoice{increase the internal energy of \SI{80}{\gram} of water by \SI{1}{\calorie}}
    \end{choices}
\end{question}
}

\element{halliday-mc}{
\begin{question}{halliday-ch18-q55}
    Solid $A$, with mass $M$, is at its melting point $T_A$.
    It is placed in thermal contact with solid $B$,
    with heat capacity $C_B$ and initially at temperature $T_B$ ($T_B > T_A$).
    The combination is thermally isolated.
    $A$ has latent heat of fusion $L$ and when it has melted has heat capacity $C_A$.
    If $A$ completely melts the final temperature of both $A$ and $B$ is:
    \begin{choices}
      \correctchoice{$\dfrac{C_A T_A + C_B T B - M L}{C_A + C_B}$}
        \wrongchoice{$\dfrac{C_A T_A - C_B T B + M L}{C_A + C_B}$}
        \wrongchoice{$\dfrac{C_A T_A - C_B T B - M L}{C_A + C_B}$}
        \wrongchoice{$\dfrac{C_A T_A + C_B T B + M L}{C_A - C_B}$}
        \wrongchoice{$\dfrac{C_A T_A + C_B T B + M L}{C_A - C_B}$}
    \end{choices}
\end{question}
}

\element{halliday-mc}{
\begin{question}{halliday-ch18-q56}
    During the time that latent heat is involved in a change of state:
    \begin{choices}
        \wrongchoice{the temperature does not change}
        \wrongchoice{the substance always expands}
        \wrongchoice{a chemical reaction takes place}
        \wrongchoice{molecular activity remains constant}
        \wrongchoice{kinetic energy changes into potential energy}
    \end{choices}
\end{question}
}

\element{halliday-mc}{
\begin{question}{halliday-ch18-q57}
    The formation of ice from water is accompanied by:
    \begin{choices}
      \correctchoice{absorption of energy as heat}
        \wrongchoice{temperature increase}
        \wrongchoice{decrease in volume}
        \wrongchoice{an evolution of heat}
        \wrongchoice{temperature decrease}
    \end{choices}
\end{question}
}

\element{halliday-mc}{
\begin{question}{halliday-ch18-q58}
    How many calories are required to change one gram of \SI{0}{\degreeCelsius} ice to \SI{100}{\degreeCelsius} steam?
    The latent heat of fusion is \SI{80}{\calorie\per\gram} and the latent heat of vaporization is \SI{540}{\calorie\per\gram}.
    The specific heat of water is \SI{1.00}{\calorie\per\gram\per\kelvin}.
    \begin{multicols}{3}
    \begin{choices}
        \wrongchoice{\SI{100}{\calorie}}
        \wrongchoice{\SI{540}{\calorie}}
        \wrongchoice{\SI{620}{\calorie}}
      \correctchoice{\SI{720}{\calorie}}
        \wrongchoice{\SI{900}{\calorie}}
    \end{choices}
    \end{multicols}
\end{question}
}

\element{halliday-mc}{
\begin{question}{halliday-ch18-q59}
    Ten grams of ice at \SI{-20}{\degreeCelsius} is to be changed to steam at \SI{130}{\degreeCelsius}.
    The specific heat of both ice and steam is \SI{0.5}{\calorie\per\gram\per\degreeCelsius}.
    The heat of fusion is \SI{80}{\calorie\per\gram} and the heat of vaporization is \SI{540}{\calorie\per\gram}.
    The entire process requires:
    \begin{multicols}{3}
    \begin{choices}
        \wrongchoice{\SI{750}{\calorie}}
        \wrongchoice{\SI{1250}{\calorie}}
        \wrongchoice{\SI{6950}{\calorie}}
      \correctchoice{\SI{7450}{\calorie}}
        \wrongchoice{\SI{7700}{\calorie}}
    \end{choices}
    \end{multicols}
\end{question}
}

\element{halliday-mc}{
\begin{question}{halliday-ch18-q60}
    Steam at \SI{1}{\atm} and \SI{100}{\degreeCelsius} enters a radiator and leaves as water at \SI{1}{\atm} and \SI{80}{\degreeCelsius}.
    Take the heat of vaporization to be \SI{540}{\calorie\per\gram}.
    Of the total energy given off as heat,
        what percent arises from the cooling of the water?
    \begin{multicols}{3}
    \begin{choices}
        \wrongchoice{\SI{100}{\percent}}
        \wrongchoice{\SI{54}{\percent}}
        \wrongchoice{\SI{26}{\percent}}
        \wrongchoice{\SI{14}{\percent}}
      \correctchoice{\SI{3.6}{\percent}}
    \end{choices}
    \end{multicols}
\end{question}
}

\element{halliday-mc}{
\begin{question}{halliday-ch18-q61}
    A certain humidifier operates by raising water to the boiling point and then evaporating it.
    Every minute \SI{30}{\gram} of water at \SI{20}{\degreeCelsius} are added to replace the \SI{30}{\gram} that are evaporated.
    The heat of fusion of water is \SI{333}{\kilo\joule\per\kilo\gram},
        the heat of vaporization is \SI{2256}{\kilo\joule\per\kilo\gram},
        and the specific heat is \SI{4190}{\joule\per\kilo\gram\per\kelvin}.
    %% NOTE: what is the power consumption of this humidifier? in watts
    How many joules of energy per minute does this humidifier require?
    \begin{multicols}{2}
    \begin{choices}
        \wrongchoice{\SI{3.0e4}{\joule\per\minute}}
      \correctchoice{\SI{8.8e4}{\joule\per\minute}}
        \wrongchoice{\SI{7.8e4}{\joule\per\minute}}
        \wrongchoice{\SI{1.1e5}{\joule\per\minute}}
        \wrongchoice{\SI{2.0e4}{\joule\per\minute}}
    \end{choices}
    \end{multicols}
\end{question}
}

\element{halliday-mc}{
\begin{question}{halliday-ch18-q62}
    A metal sample of mass $M$ requires a power input $P$ to just remain molten.
    When the heater is turned off,
        the metal solidifies in a time $T$.
    The specific latent heat of fusion of this metal is:
    \begin{multicols}{3}
    \begin{choices}
        \wrongchoice{$\dfrac{P}{M T}$}
        \wrongchoice{$\dfrac{T}{P M}$}
        \wrongchoice{$\dfrac{P M}{T}$}
        \wrongchoice{$P M T$}
      \correctchoice{$\dfrac{P T}{M}$}
    \end{choices}
    \end{multicols}
\end{question}
}

\element{halliday-mc}{
\begin{question}{halliday-ch18-q63}
    Fifty grams of ice at \SI{0}{\degreeCelsius} is placed in a thermos bottle containing one hundred grams of water at \SI{6}{\degreeCelsius}.
    How many grams of ice will melt?
    The heat of fusion of water is \SI{333}{\kilo\joule\per\kilo\gram}
        and the specific heat is \SI{4190}{\joule\per\kilo\gram\per\kelvin}.
    \begin{multicols}{3}
    \begin{choices}
      \correctchoice{\SI{7.5}{\gram}}
        \wrongchoice{\SI{2.0}{\gram}}
        \wrongchoice{\SI{8.3}{\gram}}
        \wrongchoice{\SI{17}{\gram}}
        \wrongchoice{\SI{50}{\gram}}
    \end{choices}
    \end{multicols}
\end{question}
}

\element{halliday-mc}{
\begin{question}{halliday-ch18-q64}
    According to the first law of thermodynamics,
        applied to a gas, the increase in the internal energy during any process:
    \begin{choices}
        \wrongchoice{equals the heat input minus the work done on the gas}
        \wrongchoice{equals the heat input plus the work done on the gas}
        \wrongchoice{equals the work done on the gas minus the heat input}
        \wrongchoice{is independent of the heat input}
        \wrongchoice{is independent of the work done on the gas}
    \end{choices}
\end{question}
}

\element{halliday-mc}{
\begin{question}{halliday-ch18-q65}
    Pressure versus volume graphs for a certain gas undergoing five different cyclic processes are shown below.
    During which cycle does the gas do the greatest positive work?
    \begin{multicols}{2}
    \begin{choices}
        \wrongchoice{
            \begin{tikzpicture}
                %% NOTE: options
            \end{tikzpicture}
        }
    \end{choices}
    \end{multicols}
\end{question}
}

\element{halliday-mc}{
\begin{question}{halliday-ch18-q66}
    During an adiabatic process an object does \SI{100}{\joule} of work and its temperature decreases by \SI{5}{\kelvin}.
    During another process it does \SI{25}{\joule} of work and its temperature decreases by \SI{5}{\kelvin}.
    Its heat capacity for the second process is:
    \begin{multicols}{3}
    \begin{choices}
        \wrongchoice{\SI{20}{\joule\per\kelvin}}
        \wrongchoice{\SI{24}{\joule\per\kelvin}}
        \wrongchoice{\SI{5}{\joule\per\kelvin}}
      \correctchoice{\SI{15}{\joule\per\kelvin}}
        \wrongchoice{\SI{100}{\joule\per\kelvin}}
    \end{choices}
    \end{multicols}
\end{question}
}

\element{halliday-mc}{
\begin{question}{halliday-ch18-q67}
    A system undergoes an adiabatic process in which its internal energy increases by \SI{20}{\joule}.
    Which of the following statements is true?
    \begin{choices}
      \correctchoice{\SI{20}{\joule} of work was done on the system}
        \wrongchoice{\SI{20}{\joule} of work was done by the system}
        \wrongchoice{the system received \SI{20}{\joule} of energy as heat}
        \wrongchoice{the system lost \SI{20}{\joule} of energy as heat}
        \wrongchoice{none of the provided are true}
    \end{choices}
\end{question}
}

\element{halliday-mc}{
\begin{question}{halliday-ch18-q68}
    In an adiabatic process:
    \begin{choices}
        \wrongchoice{the energy absorbed as heat equals the work done by the system on its environment}
        \wrongchoice{the energy absorbed as heat equals the work done by the environment on the system}
        \wrongchoice{the absorbed as heat equals the change in internal energy}
      \correctchoice{the work done by the environment on the system equals the change in internal energy}
        \wrongchoice{the work done by the system on its environment equals to the change in internal energy}
    \end{choices}
\end{question}
}

\element{halliday-mc}{
\begin{question}{halliday-ch18-q69}
    In a certain process a gas ends in its original thermodynamic state.
    Of the following,
        which is possible as the net result of the process?
    \begin{choices}
        \wrongchoice{It is adiabatic and the gas does \SI{50}{\joule} of work}
        \wrongchoice{The gas does no work but absorbs \SI{50}{\joule} of energy as heat}
        \wrongchoice{The gas does no work but loses \SI{50}{\joule} of energy as heat}
        \wrongchoice{The gas loses \SI{50}{\joule} of energy as heat and does \SI{50}{\joule} of work}
      \correctchoice{The gas absorbs \SI{50}{\joule} of energy as heat and does \SI{50}{\joule} of work}
    \end{choices}
\end{question}
}

\element{halliday-mc}{
\begin{question}{halliday-ch18-q70}
    Of the following which might \emph{not} vanish over one cycle of a cyclic process?
    \begin{choices}
        \wrongchoice{the change in the internal energy of the substance}
        \wrongchoice{the change in pressure of the substance}
      \correctchoice{the work done by the substance}
        \wrongchoice{the change in the volume of the substance}
        \wrongchoice{the change in the temperature of the substance}
    \end{choices}
\end{question}
}

\element{halliday-mc}{
\begin{question}{halliday-ch18-q71}
    Of the following which might \emph{not} vanish over one cycle of a cyclic process?
    \begin{choices}
        \wrongchoice{the work done by the substance minus the energy absorbed by the substance as heat}
        \wrongchoice{the change in the pressure of the substance}
        \wrongchoice{the energy absorbed by the substance as heat}
        \wrongchoice{the change in the volume of the substance}
        \wrongchoice{the change in the temperature of the substance}
    \end{choices}
\end{question}
}

\element{halliday-mc}{
\begin{question}{halliday-ch18-q72}
    The unit of thermal conductivity might be:
    \begin{choices}
        \wrongchoice{calorie centimeter per second per degreeCelsius (\si{\calorie\centi\meter\per\second\per\degreeCelsius})}
      \correctchoice{calorie per centimeter per second per degree Celsius (\si{\calorie\per\centi\meter\per\second\per\degreeCelsius})}
        \wrongchoice{calorie second per centimeter per degree Celsius (\si{\calorie\second\per\centi\meter\per\degreeCelsius})}
        \wrongchoice{centimeter second degree Celsius per calorie (\si{\centi\meter\second\degreeCelsius\per\calorie})}
        \wrongchoice{degree Celsius per calorie per centimeter per second (\si{\degreeCelsius\per\calorie\per\centi\meter\per\second})}
    \end{choices}
\end{question}
}

\element{halliday-mc}{
\begin{question}{halliday-ch18-q73}
    A slab of material has area $A$, thickness $L$, and thermal conductivity $k$.
    One of its surfaces ($P$) is maintained at temperature $T_1$ and the other surface ($Q$) is maintained at a lower temperature $T_2$.
    The rate of heat flow by conduction from $P$ to $Q$ is:
    \begin{multicols}{2}
    \begin{choices}
        \wrongchoice{$kA\dfrac{T_1-T_2}{L^2}$}
        \wrongchoice{$kL\dfrac{T_1-T_2}{A}$}
      \correctchoice{$kA\dfrac{T_1-T_2}{L}$}
        \wrongchoice{$k \dfrac{T_1-T_2}{LA}$}
        \wrongchoice{$LA\dfrac{T_1-T_2}{k}$}
    \end{choices}
    \end{multicols}
\end{question}
}

\element{halliday-mc}{
\begin{question}{halliday-ch18-q74}
    The rate of heat flow by conduction through a slab does \emph{not} depend upon the:
    \begin{choices}
        \wrongchoice{temperature difference between opposite faces of the slab}
        \wrongchoice{thermal conductivity of the slab}
        \wrongchoice{slab thickness}
        \wrongchoice{cross-sectional area of the slab}
      \correctchoice{specific heat of the slab}
    \end{choices}
\end{question}
}

\element{halliday-mc}{
\begin{question}{halliday-ch18-q75}
    The rate of heat flow by conduction through a slab is $P_{cond}$.
    If the slab thickness is doubled,
        its cross-sectional area is halved,
        and the temperature difference across it is doubled,
        then the rate of heat flow becomes:
    \begin{multicols}{3}
    \begin{choices}
        \wrongchoice{$2P_{cond}$}
      \correctchoice{$\dfrac{P_{cond}}{2}$}
        \wrongchoice{$P_{cond}$}
        \wrongchoice{$\dfrac{P_{cond}}{8}$}
        \wrongchoice{$8P_{cond}$}
    \end{choices}
    \end{multicols}
\end{question}
}

\element{halliday-mc}{
\begin{question}{halliday-ch18-q76}
    The diagram shows four slabs of different materials with equal thickness,
        placed side by side.
    Heat flows from left to right and the steady-state temperatures of the interfaces are given.
    \begin{center}
    \begin{tikzpicture}
        %% NOTE:
    \end{tikzpicture}
    \end{center}
    Rank the materials according to their thermal conductivities,
        smallest to largest.
    \begin{multicols}{2}
    \begin{choices}
        \wrongchoice{1, 2, 3, 4}
        \wrongchoice{2, 1, 3, 4}
        \wrongchoice{3, 4, 1, 2}
      \correctchoice{3, 4, 2, 1}
        \wrongchoice{4, 3, 2, 1}
    \end{choices}
    \end{multicols}
\end{question}
}

\element{halliday-mc}{
\begin{question}{halliday-ch18-q77}
    Inside a room at a uniform comfortable temperature,
        metallic objects generally feel cooler to the touch than wooden objects do. 
    This is because:
    \begin{choices}
        \wrongchoice{a given mass of wood contains more heat than the same mass of metal}
      \correctchoice{metal conducts heat better than wood}
        \wrongchoice{heat tends to flow from metal to wood}
        \wrongchoice{the equilibrium temperature of metal in the room is lower than that of wood}
        \wrongchoice{the human body, being organic, resembles wood more closely than it resembles metal}
    \end{choices}
\end{question}
}

\element{halliday-mc}{
\begin{question}{halliday-ch18-q78}
    On a very cold day, a child puts his tongue against a fence post. 
    It is much more likely that his tongue will stick to a steel post than to a wooden post.
    This is because:
    \begin{choices}
        \wrongchoice{steel has a higher specific heat}
        \wrongchoice{steel is a better radiator of heat}
        \wrongchoice{steel has a higher specific gravity}
      \correctchoice{steel is a better heat conductor}
        \wrongchoice{steel is a highly magnetic material}
    \end{choices}
\end{question}
}

\element{halliday-mc}{
\begin{question}{halliday-ch18-q79}
    An iron stove, used for heating a room by radiation,
        is more efficient if:
    \begin{choices}
        \wrongchoice{its inner surface is highly polished}
        \wrongchoice{its inner surface is covered with aluminum paint}
        \wrongchoice{its outer surface is covered with aluminum paint}
      \correctchoice{its outer surface is rough and black}
        \wrongchoice{its outer surface is highly polished}
    \end{choices}
\end{question}
}

\element{halliday-mc}{
\begin{question}{halliday-ch18-q80}
    To help keep buildings cool in the summer,
        dark colored window shades have been replaced by light colored shades. 
    This is because light colored shades:
    \begin{choices}
        \wrongchoice{are more pleasing to the eye}
        \wrongchoice{absorb more sunlight}
      \correctchoice{reflect more sunlight}
        \wrongchoice{transmit more sunlight}
        \wrongchoice{have a lower thermal conductivity}
    \end{choices}
\end{question}
}

\element{halliday-mc}{
\begin{question}{halliday-ch18-q81}
    Which of the following statements pertaining to a vacuum flask (thermos) is \emph{not} correct?
    \begin{choices}
        \wrongchoice{Silvering reduces radiation loss}
        \wrongchoice{Vacuum reduces conduction loss}
        \wrongchoice{Vacuum reduces convection loss}
      \correctchoice{Vacuum reduces radiation loss}
        \wrongchoice{Glass walls reduce conduction loss}
    \end{choices}
\end{question}
}

\element{halliday-mc}{
\begin{question}{halliday-ch18-q82}
    A thermos bottle works well because:
    \begin{choices}
        \wrongchoice{its glass walls are thin}
        \wrongchoice{silvering reduces convection}
        \wrongchoice{vacuum reduces heat radiation}
        \wrongchoice{silver coating is a poor heat conductor}
      \correctchoice{none of the provided}
    \end{choices}
\end{question}
}


\endinput


