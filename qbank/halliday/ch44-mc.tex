
%%--------------------------------------------------
%% Halliday: Fundamentals of Physics
%%--------------------------------------------------


%% Chapter 44: Quarks, Leptons, and the Big Bang
%%--------------------------------------------------


%% Learning Objectives
%%--------------------------------------------------

%% 44.01: Identify that a great many different elementary particles exist or can be created and that nearly all of them are unstable.
%% 44.02: For the decay of an unstable particle, apply the same decay equations as used for the radioactive decay of nuclei.
%% 44.03: Identify spin as the intrinsic angular momentum of a particle.
%% 44.04: Distinguish fermions from bosons, and identify which are required to obey the Pauli exclusion principle.
%% 44.05: Distinguish leptons and hadrons, and then identify the two types of hadrons.
%% 44.06: Distinguish particle from antiparticle, and identify that if they meet, they undergo annihilation and are transformed into photons or into other elementary particles.
%% 44.07: Distinguish the strong force and the weak force.
%% 44.08: To see if a given process for elementary particles is physically possible, apply the conservation laws for charge, linear momentum, spin angular momentum, and energy (including mass energy).


%% Halliday Multiple Choice Questions
%%--------------------------------------------------
\element{halliday-mc}{
\begin{question}{halliday-ch44-q01}
    Which of the following particles is stable?
    \begin{multicols}{3}
    \begin{choices}
        \wrongchoice{Neutron}
      \correctchoice{Proton}
        \wrongchoice{Pion}
        \wrongchoice{Muon}
        \wrongchoice{Kaon}
    \end{choices}
    \end{multicols}
\end{question}
}

\element{halliday-mc}{
\begin{question}{halliday-ch44-q02}
    The stability of the proton is predicted by the laws of conservation of energy and conservation of:
    \begin{choices}
        \wrongchoice{momentum}
        \wrongchoice{angular momentum}
      \correctchoice{baryon number}
        \wrongchoice{lepton number}
        \wrongchoice{strangeness}
    \end{choices}
\end{question}
}

\element{halliday-mc}{
\begin{question}{halliday-ch44-q03}
    When a kaon decays via the strong interaction the products must include a:
    \begin{multicols}{2}
    \begin{choices}
        \wrongchoice{baryon}
        \wrongchoice{lepton}
      \correctchoice{strange particle}
        \wrongchoice{electron}
        \wrongchoice{neutrino}
    \end{choices}
    \end{multicols}
\end{question}
}

\element{halliday-mc}{
\begin{question}{halliday-ch44-q04}
    A particle with spin angular momentum $\hbar/2$ is called a:
    \begin{multicols}{2}
    \begin{choices}
        \wrongchoice{lepton}
        \wrongchoice{hadron}
      \correctchoice{fermion}
        \wrongchoice{boson}
        \wrongchoice{electron}
    \end{choices}
    \end{multicols}
\end{question}
}

\element{halliday-mc}{
\begin{question}{halliday-ch44-q05}
    A particle with spin angular momentum $\hbar$ is called a:
    \begin{multicols}{2}
    \begin{choices}
        \wrongchoice{lepton}
        \wrongchoice{hadron}
        \wrongchoice{fermion}
      \correctchoice{boson}
        \wrongchoice{electron}
    \end{choices}
    \end{multicols}
\end{question}
}

\element{halliday-mc}{
\begin{question}{halliday-ch44-q06}
    An example of a fermion is a:
    \begin{choices}
        \wrongchoice{photon}
        \wrongchoice{pion}
      \correctchoice{neutrino}
        \wrongchoice{kaon}
        \wrongchoice{none of the provided}
    \end{choices}
\end{question}
}

\element{halliday-mc}{
\begin{question}{halliday-ch44-q07}
    An example of a boson is a:
    \begin{multicols}{2}
    \begin{choices}
      \correctchoice{photon}
        \wrongchoice{electron}
        \wrongchoice{neutrino}
        \wrongchoice{proton}
        \wrongchoice{neutron}
    \end{choices}
    \end{multicols}
\end{question}
}

\element{halliday-mc}{
\begin{question}{halliday-ch44-q08}
    All particles with spin angular momentum $\hbar/2$:
    \begin{choices}
        \wrongchoice{interact via the strong force}
        \wrongchoice{travel at the speed of light}
      \correctchoice{obey the Pauli exclusion principle}
        \wrongchoice{have non-zero rest mass}
        \wrongchoice{are charged}
    \end{choices}
\end{question}
}

\element{halliday-mc}{
\begin{question}{halliday-ch44-q09}
    All leptons interact with each other via the:
    \begin{choices}
        \wrongchoice{strong force}
      \correctchoice{weak force}
        \wrongchoice{electromagnetic force}
        \wrongchoice{strange force}
        \wrongchoice{none of the provided}
    \end{choices}
\end{question}
}

\element{halliday-mc}{
\begin{question}{halliday-ch44-q10}
    An electron participates in:
    \begin{choices}
        \wrongchoice{the strong force only}
        \wrongchoice{the strong and weak forces only}
        \wrongchoice{the electromagnetic and gravitational forces only}
      \correctchoice{the electromagnetic, gravitational, and weak forces only}
        \wrongchoice{the electromagnetic, gravitational, and strong forces only}
    \end{choices}
\end{question}
}

\element{halliday-mc}{
\begin{question}{halliday-ch44-q11}
    Which of the following particles has a lepton number of zero?
    \begin{multicols}{3}
    \begin{choices}
        \wrongchoice{$e^+$}
        \wrongchoice{$\mu^+$}
        \wrongchoice{$\nu_e$}
        \wrongchoice{$\nu_{\mu}$}
      \correctchoice{$p$}
    \end{choices}
    \end{multicols}
\end{question}
}

\element{halliday-mc}{
\begin{question}{halliday-ch44-q12}
    Which of the following particles has a lepton number of $+1$?
    \begin{multicols}{3}
    \begin{choices}
        \wrongchoice{$e^+$}
        \wrongchoice{$\mu^+$}
      \correctchoice{$\mu^-$}
        \wrongchoice{$\nu_e$}
        \wrongchoice{$p$}
    \end{choices}
    \end{multicols}
\end{question}
}

\element{halliday-mc}{
\begin{question}{halliday-ch44-q13}
    $\pi^+$ represents a pion (a meson), $\mu^-$ represents a muon (a lepton),
        $\nu_e$ represents an electron neutrino (a lepton),
        $\nu_{\mu}$ and $p$ represents a proton represents a muon neutrino (a lepton). 
    Which of the following decays might occur?
    \begin{multicols}{2}
    \begin{choices}
        \wrongchoice{\ce{\pi^{+} -> \mu^- + \nu_{\mu}}}
        \wrongchoice{\ce{\pi^{+} -> p +  \nu_{e}}}
        \wrongchoice{\ce{\pi^{+} -> \mu^{+} + \nu_{e}}}
        \wrongchoice{\ce{\pi^{+} -> p + \nu_{\mu}}}
      \correctchoice{\ce{\pi^{+} -> \mu^{+} + \nu_{\mu}}}
    \end{choices}
    \end{multicols}
\end{question}
}

\element{halliday-mc}{
\begin{question}{halliday-ch44-q14}
    A particle can decay to particles with greater total rest mass:
    \begin{choices}
        \wrongchoice{only if antiparticles are produced}
        \wrongchoice{only if photons are also produced}
        \wrongchoice{only if neutrinos are also produced}
        \wrongchoice{only if the original particle has kinetic energy}
      \correctchoice{never}
    \end{choices}
\end{question}
}

\element{halliday-mc}{
\begin{question}{halliday-ch44-q15}
    The interaction \ce{\pi^- + p -> \pi^- + \Sigma^+} violates the principle of conservation of:
    \begin{choices}
        \wrongchoice{baryon number}
        \wrongchoice{lepton number}
      \correctchoice{strangeness}
        \wrongchoice{angular momentum}
        \wrongchoice{none of these}
    \end{choices}
\end{question}
}

\element{halliday-mc}{
\begin{question}{halliday-ch44-q16}
    The interaction \ce{\pi^- + p -> K^{-} + \Sigma^+} violates the principle of conservation of:
    \begin{choices}
        \wrongchoice{baryon number}
        \wrongchoice{lepton number}
        \wrongchoice{strangeness}
        \wrongchoice{angular momentum}
        \correctchoice{none of the provided}
    \end{choices}
\end{question}
}

\element{halliday-mc}{
\begin{question}{halliday-ch44-q17}
    A neutral muon cannot decay into two neutrinos. 
    Of the following conservation laws,
        which would be violated if it did?
    \begin{choices}
        \wrongchoice{Energy}
        \wrongchoice{Baryon number}
        \wrongchoice{Charge}
      \correctchoice{Angular momentum}
        \wrongchoice{None of the provided}
    \end{choices}
\end{question}
}

\element{halliday-mc}{
\begin{question}{halliday-ch44-q18}
    A positron cannot decay into three neutrinos. 
    Of the following conservation laws,
        which would be violated if it did?
    \begin{choices}
        \wrongchoice{Energy}
        \wrongchoice{Baryon number}
      \correctchoice{Lepton number}
        \wrongchoice{Linear momentum}
        \wrongchoice{Angular momentum}
    \end{choices}
\end{question}
}

\element{halliday-mc}{
\begin{question}{halliday-ch44-q19}
    Two particles interact to produce only photons,
        with the original particles disappearing. 
    The particles must have been:
    \begin{choices}
        \wrongchoice{mesons}
        \wrongchoice{strange particles}
        \wrongchoice{strongly interacting}
        \wrongchoice{leptons}
      \correctchoice{a particle, antiparticle pair}
    \end{choices}
\end{question}
}

\element{halliday-mc}{
\begin{question}{halliday-ch44-q20}
    Two baryons interact to produce pions only,
        the original baryons disappearing. 
    One of the baryons must have been:
    \begin{choices}
        \wrongchoice{a proton}
        \wrongchoice{an omega minus}
        \wrongchoice{a sigma}
      \correctchoice{an antiparticle}
        \wrongchoice{none of these}
    \end{choices}
\end{question}
}

\element{halliday-mc}{
\begin{question}{halliday-ch44-q21}
    A baryon with strangeness $-1$ decays via the strong interaction into two particles,
        one of which is a baryon with strangeness 0. 
    The other might be:
    \begin{choices}
        \wrongchoice{a baryon with strangeness $0$}
        \wrongchoice{a baryon with strangeness $+1$}
      \correctchoice{a meson with strangeness $-1$}
        \wrongchoice{a meson with strangeness $+1$}
        \wrongchoice{a meson with strangeness $0$}
    \end{choices}
\end{question}
}

\element{halliday-mc}{
\begin{question}{halliday-ch44-q22}
    A baryon with strangeness $0$ decays via the strong interaction into two particles,
        one of which is a baryon with strangeness $+1$.
    The other might be:
    \begin{choices}
        \wrongchoice{a baryon with strangeness $0$}
        \wrongchoice{a baryon with strangeness $+1$}
        \wrongchoice{a baryon with strangeness $-1$}
        \wrongchoice{a meson with strangeness $+1$}
      \correctchoice{a meson with strangeness $-1$}
    \end{choices}
\end{question}
}

\element{halliday-mc}{
\begin{question}{halliday-ch44-q23}
    In order of increasing strength the four basic interactions are:
    \begin{choices}
      \correctchoice{gravitational, weak, electromagnetic, and strong}
        \wrongchoice{gravitational, electromagnetic, weak, and strong}
        \wrongchoice{weak, gravitational, electromagnetic, and strong}
        \wrongchoice{weak, electromagnetic, gravitational, and strong}
        \wrongchoice{weak, electromagnetic, strong, and gravitational}
    \end{choices}
\end{question}
}

\element{halliday-mc}{
\begin{question}{halliday-ch44-q24}
    The two basic interactions that have finite ranges are:
    \begin{choices}
        \wrongchoice{electromagnetic and gravitational}
        \wrongchoice{electromagnetic and strong}
        \wrongchoice{electromagnetic and weak}
        \wrongchoice{gravitational and weak}
      \correctchoice{weak and strong}
    \end{choices}
\end{question}
}

\element{halliday-mc}{
\begin{question}{halliday-ch44-q25}
    A certain process produces baryons that decay with a lifetime of \SI{4e-24}{\second}. 
    The decay is a result of:
    \begin{choices}
        \wrongchoice{the gravitational interaction}
        \wrongchoice{the weak interaction}
        \wrongchoice{the electromagnetic interaction}
      \correctchoice{the strong interaction}
        \wrongchoice{some combination of the above}
    \end{choices}
\end{question}
}

\element{halliday-mc}{
\begin{question}{halliday-ch44-q26}
    A certain process produces mesons that decay with a lifetime of \SI{6e-10}{\second}. 
    The decay is a result of:
    \begin{choices}
        \wrongchoice{the gravitational interaction}
      \correctchoice{the weak interaction}
        \wrongchoice{the electromagnetic interaction}
        \wrongchoice{the strong interaction}
        \wrongchoice{some combination of the above}
    \end{choices}
\end{question}
}

\element{halliday-mc}{
\begin{question}{halliday-ch44-q27}
    Compared to the lifetimes of particles that decay via the weak interaction,
        the lifetimes of particles that decay via the strong interaction are:
    \begin{choices}
      \correctchoice{\num{e-12} times as long}
        \wrongchoice{\num{e-23} times as long}
        \wrongchoice{\num{e24} times as long}
        \wrongchoice{\num{e12} times as long}
        \wrongchoice{about the same}
    \end{choices}
\end{question}
}

\element{halliday-mc}{
\begin{question}{halliday-ch44-q28}
    Strangeness is conserved in:
    \begin{choices}
        \wrongchoice{all particle decays}
        \wrongchoice{no particle decays}
        \wrongchoice{all weak particle decays}
      \correctchoice{all strong particle decays}
        \wrongchoice{some strong particle decays}
    \end{choices}
\end{question}
}

\element{halliday-mc}{
\begin{question}{halliday-ch44-q29}
    Different types of neutrinos can be distinguished from each other by:
    \begin{choices}
        \wrongchoice{the directions of their spins}
      \correctchoice{the leptons with which they interact}
        \wrongchoice{the baryons with which they interact}
        \wrongchoice{the number of photons that accompany them}
        \wrongchoice{their baryon numbers}
    \end{choices}
\end{question}
}

\element{halliday-mc}{
\begin{question}{halliday-ch44-q30}
    All known quarks have:
    \begin{choices}
        \wrongchoice{charges that are multiples of $e$ and integer baryon numbers}
        \wrongchoice{charges that are multiples of $e$ and baryon numbers that are either $+1/3$ or $-1/3$}
        \wrongchoice{charges that are multiples of $e/3$ and integer baryon numbers}
      \correctchoice{charges that are multiples of $e/3$ and baryon numbers that are either $+1/3$ or $-1/3$}
        \wrongchoice{charges that are multiples of $2e/3$ and baryon numbers that are either $+1/3$ or $-1/3$}
    \end{choices}
\end{question}
}

\element{halliday-mc}{
\begin{question}{halliday-ch44-q31}
    The baryon number of a quark is:
    \begin{multicols}{3}
    \begin{choices}
        \wrongchoice{zero}
        \wrongchoice{\num{1/2}}
      \correctchoice{\num{1/3}}
        \wrongchoice{\num{2/3}}
        \wrongchoice{one}
    \end{choices}
    \end{multicols}
\end{question}
}

\element{halliday-mc}{
\begin{question}{halliday-ch44-q32}
    Quarks are the constituents of:
    \begin{choices}
        \wrongchoice{all particles}
        \wrongchoice{all leptons}
      \correctchoice{all strongly interacting particles}
        \wrongchoice{only strange particles}
        \wrongchoice{only mesons}
    \end{choices}
\end{question}
}

\element{halliday-mc}{
\begin{question}{halliday-ch44-q33}
    Any meson is a combination of:
    \begin{choices}
        \wrongchoice{three quarks}
        \wrongchoice{two quarks and an antiquark}
        \wrongchoice{one quark and two antiquarks}
      \correctchoice{one quark and one antiquark}
        \wrongchoice{two quarks}
    \end{choices}
\end{question}
}

\element{halliday-mc}{
\begin{question}{halliday-ch44-q34}
    Any baryon is a combination of:
    \begin{choices}
      \correctchoice{three quarks}
        \wrongchoice{two quarks and an antiquark}
        \wrongchoice{one quark and two antiquarks}
        \wrongchoice{one quark and one antiquark}
        \wrongchoice{two quarks}
    \end{choices}
\end{question}
}

\element{halliday-mc}{
\begin{question}{halliday-ch44-q35}
    The quark content of a proton is:
    \begin{multicols}{3}
    \begin{choices}
        \wrongchoice{uuu}
      \correctchoice{uud}
        \wrongchoice{udd}
        \wrongchoice{ddd}
        \wrongchoice{uds}
    \end{choices}
    \end{multicols}
\end{question}
}

\element{halliday-mc}{
\begin{question}{halliday-ch44-q36}
    The quark content of a $\pi^+$ meson is:
    \begin{multicols}{3}
    \begin{choices}
        \wrongchoice{uu}
        \wrongchoice{uu}
        \wrongchoice{ud}
      \correctchoice{ud}
        \wrongchoice{dd}
    \end{choices}
    \end{multicols}
\end{question}
}

\element{halliday-mc}{
\begin{question}{halliday-ch44-q37}
    In terms of quark content a beta decay can be written:
    \begin{choices}
      \correctchoice{\ce{udd -> uud + e^{−} + \nu}}
        \wrongchoice{\ce{udd -> udd + dd + \nu}}
        \wrongchoice{\ce{udd -> udd + dd + e^{-}}}
        \wrongchoice{\ce{udd -> uud + ud + \nu}}
        \wrongchoice{\ce{udd -> uud + ud + e^{-} + \nu}}
    \end{choices}
\end{question}
}

\element{halliday-mc}{
\begin{question}{halliday-ch44-q38}
    The up quark u has charge $+2e/3$ and strangeness $0$;
        the down quark d has charge $-e/3$ and strangeness $0$;
        the strange quark s has charge $-e/3$ and strangeness $-1$.
    This means there can be no baryon with:
    \begin{choices}
        \wrongchoice{charge $0$ and strangeness $0$}
        \wrongchoice{charge $-e$ and strangeness $-1$}
      \correctchoice{charge $+e$ and strangeness $-1$}
        \wrongchoice{charge $+e$ and strangeness $-2$}
        \wrongchoice{charge $0$ and strangeness $+2$}
    \end{choices}
\end{question}
}

\element{halliday-mc}{
\begin{question}{halliday-ch44-q39}
    The up quark u has charge $+2e/3$ and strangeness $0$;
        the down quark d has charge $-e/3$ and strangeness $0$;
        the strange quark s has charge $-e/3$ and strangeness $-1$.
    This means there can be no meson with:
    \begin{choices}
        \wrongchoice{charge $0$ and strangeness $-1$}
        \wrongchoice{charge $-e$ and strangeness $-1$}
      \correctchoice{charge $+e$ and strangeness $-1$}
        \wrongchoice{charge $+e$ and strangeness $+1$}
        \wrongchoice{charge $0$ and strangeness $+1$}
    \end{choices}
\end{question}
}

\element{halliday-mc}{
\begin{question}{halliday-ch44-q40}
    Messenger particles of the electromagnetic interaction are called:
    \begin{multicols}{2}
    \begin{choices}
        \wrongchoice{gluons}
      \correctchoice{photons}
        \wrongchoice{W and Z}
        \wrongchoice{gravitons}
        \wrongchoice{pions}
    \end{choices}
    \end{multicols}
\end{question}
}

\element{halliday-mc}{
\begin{question}{halliday-ch44-q41}
    Messenger particles of the strong interaction are called:
    \begin{multicols}{2}
    \begin{choices}
      \correctchoice{gluons}
        \wrongchoice{photons}
        \wrongchoice{W and Z}
        \wrongchoice{gravitons}
        \wrongchoice{pions}
    \end{choices}
    \end{multicols}
\end{question}
}

\element{halliday-mc}{
\begin{question}{halliday-ch44-q42}
    Messenger particles of the weak interaction are called:
    \begin{multicols}{2}
    \begin{choices}
        \wrongchoice{gluons}
        \wrongchoice{photons}
      \correctchoice{W and Z}
        \wrongchoice{gravitons}
        \wrongchoice{pions}
    \end{choices}
    \end{multicols}
\end{question}
}

\element{halliday-mc}{
\begin{question}{halliday-ch44-q43}
    A down quark can be changed into an up quark (plus other particles perhaps) by:
    \begin{choices}
        \wrongchoice{the gravitational interaction.}
        \wrongchoice{the electromagnetic interaction.}
      \correctchoice{the weak interaction.}
        \wrongchoice{the strong interaction.}
        \wrongchoice{none of the provided.}
    \end{choices}
\end{question}
}

\element{halliday-mc}{
\begin{question}{halliday-ch44-q44}
    The color theory explains why quarks:
    \begin{choices}
      \correctchoice{form particles in pairs and triplets}
        \wrongchoice{have charge that is a multiple of e/3}
        \wrongchoice{have spin}
        \wrongchoice{have mass}
        \wrongchoice{none of the provided}
    \end{choices}
\end{question}
}

\element{halliday-mc}{
\begin{question}{halliday-ch44-q45}
    Color is carried by:
    \begin{choices}
        \wrongchoice{only quarks.}
        \wrongchoice{only leptons.}
        \wrongchoice{only quarks and leptons.}
      \correctchoice{only quarks and gluons.}
        \wrongchoice{only photons and gluons.}
    \end{choices}
\end{question}
}

\element{halliday-mc}{
\begin{question}{halliday-ch44-q46}
    Hubble's law is evidence that:
    \begin{choices}
        \wrongchoice{the speed of light is increasing.}
      \correctchoice{the universe is expanding.}
        \wrongchoice{the Earth is slowing down in its orbit.}
        \wrongchoice{galaxies have rotational motion.}
        \wrongchoice{none of the provided.}
    \end{choices}
\end{question}
}

\element{halliday-mc}{
\begin{question}{halliday-ch44-q47}
    Objects in the universe are receding from us with a speed that is proportional to:
    \begin{choices}
        \wrongchoice{the reciprocal of their distance from us.}
        \wrongchoice{the reciprocal of the square of their distance from us.}
      \correctchoice{their distance from us.}
        \wrongchoice{the square of their distance from us.}
        \wrongchoice{their distance from the center of the universe.}
    \end{choices}
\end{question}
}

\element{halliday-mc}{
\begin{question}{halliday-ch44-q48}
    The velocities of distant objects in the universe indicate that the time elapsed since the big bang is about:
    \begin{multicols}{3}
    \begin{choices}
        \wrongchoice{\SI{e5}{\year}}
      \correctchoice{\SI{e10}{\year}}
        \wrongchoice{\SI{e15}{\year}}
        \wrongchoice{\SI{e20}{\year}}
        \wrongchoice{\SI{e25}{\year}}
    \end{choices}
    \end{multicols}
\end{question}
}

\element{halliday-mc}{
\begin{question}{halliday-ch44-q49}
    The intensity of the microwave background radiation,
        a remnant of the big bang:
    \begin{choices}
        \wrongchoice{is greatest in directions toward the center of the galaxy.}
        \wrongchoice{is least in directions toward the center of the galaxy.}
        \wrongchoice{is proportional to the reciprocal of the distance from us.}
        \wrongchoice{is proportional to the square of the distance from us.}
      \correctchoice{is nearly the same in all directions.}
    \end{choices}
\end{question}
}

\element{halliday-mc}{
\begin{question}{halliday-ch44-q50}
    As a result of the big bang there is,
        in addition to the microwave background radiation,
        a uniform distribution of background:
    \begin{multicols}{2}
    \begin{choices}
        \wrongchoice{electrons}
        \wrongchoice{quarks}
        \wrongchoice{gluons}
      \correctchoice{neutrinos}
        \wrongchoice{atoms}
    \end{choices}
    \end{multicols}
\end{question}
}

\element{halliday-mc}{
\begin{question}{halliday-ch44-q51}
    Dark matter is suspected to exist in the universe because:
    \begin{choices}
        \wrongchoice{the night sky is dark between stars.}
        \wrongchoice{the orbital period of stars in the outer parts of a galaxy is greater than the orbital period of stars near the galactic center.}
        \wrongchoice{the orbital period of stars in the outer parts of a galaxy is less than the orbital period of stars near the galactic center.}
      \correctchoice{the orbital period of stars in the outer parts of a galaxy is about the same as the orbital period of stars near the galactic center.}
        \wrongchoice{all galaxies have about the same mass.}
    \end{choices}
\end{question}
}

\element{halliday-mc}{
\begin{question}{halliday-ch44-q52}
    If dark matter did not exist it is likely that:
    \begin{choices}
      \correctchoice{the universe would expand forever.}
        \wrongchoice{the universe would begin contracting soon.}
        \wrongchoice{the night sky would be brighter.}
        \wrongchoice{the night sky would be darker.}
        \wrongchoice{we would be able to see the center of the universe.}
    \end{choices}
\end{question}
}


\endinput


