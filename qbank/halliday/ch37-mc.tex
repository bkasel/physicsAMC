
%%--------------------------------------------------
%% Halliday: Fundamentals of Physics
%%--------------------------------------------------


%% Chapter 37: Special Theory of Relativity
%%--------------------------------------------------


%% Learning Objectives
%%--------------------------------------------------

%% 37.01: Identify the two postulates of (special) relativity and the type of frames to which they apply.
%% 37.02: Identify the speed of light as the ultimate speed and give its approximate value.
%% 37.03: Explain how the space and time coordinates of an event can be measured with a three-dimensional array of clocks and measuring rods and how that eliminates the need of a signal's travel time to an observer.
%% 37.04: Identify that the relativity of space and time has to do with transferring measurements between two inertial frames with relative motion but we still use classical kinematics and Newtonian mechanics within a frame.
%% 37.05: Identify that for reference frames with relative motion, simultaneous events in one of the frames will generally not be simultaneous in the other frame.
%% 37.06: Explain what is meant by the entanglement of the spatial and temporal separations between two events.
%% 37.07: Identify the conditions in which a temporal separation of two events is a proper time.
%% 37.08: Identify that if the temporal separation of two events is a proper time as measured in one frame, that separation is greater (dilated) as measured in another frame.
%% 37.09: Apply the relationship between proper time $\Delta t_0$, dilated time $\Delta t$, and the relative speed $v$ between two frames.
%% 37.10: Apply the relationship between the relative speed $v$, the speed parameter $\beta$, and the Lorentz factor $\gamma$.


%% Halliday Multiple Choice Questions
%%--------------------------------------------------
\element{halliday-mc}{
\begin{question}{halliday-ch37-q01}
    A basic postulate of Einstein’s theory of relativity is:
    \begin{choices}
        \wrongchoice{moving clocks run more slowly than when they are at rest}
        \wrongchoice{moving rods are shorter than when they are at rest}
        \wrongchoice{light has both wave and particle properties}
      \correctchoice{the laws of physics must be the same for observers moving with uniform velocity relative to each other}
        \wrongchoice{everything is relative}
    \end{choices}
\end{question}
}

\element{halliday-mc}{
\begin{question}{halliday-ch37-q02}
    A consequence of Einstein's theory of relativity is:
    \begin{choices}
      \correctchoice{moving clocks run more slowly than when they are at rest}
        \wrongchoice{moving rods are longer than when they are at rest}
        \wrongchoice{light has both wave and particle properties}
        \wrongchoice{the laws of physics must appear the same to all observers moving with uniform velocity relative to each other}
        \wrongchoice{everything is relative}
    \end{choices}
\end{question}
}

\element{halliday-mc}{
\begin{question}{halliday-ch37-q03}
    A consequence of Einstein's theory of relativity is:
    \begin{choices}
        \wrongchoice{moving clocks run faster than when they are at rest}
      \correctchoice{moving rods are shorter than when they are at rest}
        \wrongchoice{light has both wave and particle properties}
        \wrongchoice{the laws of physics must appear the same to all observers moving with uniform velocity relative to each other}
        \wrongchoice{everything is relative}
    \end{choices}
\end{question}
}

\element{halliday-mc}{
\begin{question}{halliday-ch37-q04}
    According to the theory of relativity:
    \begin{choices}
        \wrongchoice{moving clocks run fast}
        \wrongchoice{energy is not conserved in high speed collisions}
        \wrongchoice{the speed of light must be measured relative to the ether}
        \wrongchoice{momentum is not conserved in high speed collisions}
      \correctchoice{none of the provided are true}
    \end{choices}
\end{question}
}

\element{halliday-mc}{
\begin{question}{halliday-ch37-q05}
    Two events occur simultaneously on the $x$ axis of reference frame $S$,
        one at $x=-a$ and the other at $x=+a$.
    According to an observer moving in the positive $x$ direction:
    \begin{choices}
      \correctchoice{the event at $x=+a$ occurs first}
        \wrongchoice{the event at $x=-a$ occurs first}
        \wrongchoice{either event might occur first, depending on the value of a and the observer's speed}
        \wrongchoice{the events are simultaneous}
        \wrongchoice{none of the provided}
    \end{choices}
\end{question}
}

\element{halliday-mc}{
\begin{question}{halliday-ch37-q06}
    Two events occur simultaneously at separated points on the $y$ axis of reference frame $S$. 
    According to an observer moving in the positive $x$ direction:
    \begin{choices}
        \wrongchoice{the event with the greater $y$ coordinate occurs first}
        \wrongchoice{the event with the greater $y$ coordinate occurs last}
        \wrongchoice{either event might occur first, depending on the observer's speed}
      \correctchoice{the events are simultaneous}
      \wrongchoice{none of the provided}
    \end{choices}
\end{question}
}

\element{halliday-mc}{
\begin{question}{halliday-ch37-q07}
    A train traveling very fast ($v=0.6c$) has an engineer ($E$) at the front,
        a guard ($G$) at the rear, and an observer ($S^{\prime}$) exactly half way between them. 
    Both $E$ and $G$ are equipped with yellow signaling lamps. 
    The train passes a station, closely observed by the station master ($S$).
    Both $E$ and $G$ use their lamps to send signals. 
    According to both $S$ and $S$ these signals arrive simultaneously at the instant $S^{\prime}$ is passing $S$. 
    According to $S^{\prime}$:
    \begin{choices}
        \wrongchoice{$E$ and $G$ sent their signals simultaneously from different distances}
        \wrongchoice{$G$ sent his signal before $E$ and from farther away}
        \wrongchoice{$G$ sent his signal before $E$ but was the same distance away}
        \wrongchoice{$E$ sent his signal before $G$ and from farther away}
      \correctchoice{none of the provided}
    \end{choices}
\end{question}
}

\element{halliday-mc}{
\begin{question}{halliday-ch37-q08}
    The proper time between two events is measured by clocks at rest in a reference frame in which the two events:
    \begin{choices}
        \wrongchoice{occur at the same time}
      \correctchoice{occur at the same coordinates}
        \wrongchoice{are separated by the distance a light signal can travel during the time interval}
        \wrongchoice{occur in Boston}
        \wrongchoice{satisfy none of the above}
    \end{choices}
\end{question}
}

\element{halliday-mc}{
\begin{question}{halliday-ch37-q09}
    The spaceship U.S.S. Enterprise, traveling through the galaxy,
        sends out a smaller explorer craft that travels to a nearby planet and signals its findings back. 
    The proper time for the trip to the planet is measured by clocks:
    \begin{choices}
        \wrongchoice{on board the Enterprise}
      \correctchoice{on board the explorer craft}
        \wrongchoice{on Earth}
        \wrongchoice{at the center of the galaxy}
        \wrongchoice{none of the provided}
    \end{choices}
\end{question}
}

\element{halliday-mc}{
\begin{question}{halliday-ch37-q10}
    Spaceship $A$, traveling past us at $0.7c$,
        sends a message capsule to spaceship $B$, which is in front of $A$ and is traveling in the same direction as A at 0.8c relative to us. 
    The capsule travels at $0.95c$ relative to us. 
    A clock that measures the proper time between the sending and receiving of the capsule travels:
    \begin{choices}
        \wrongchoice{in the same direction as the spaceships at $0.7c$ relative to us}
        \wrongchoice{in the opposite direction from the spaceships at $0.7c$ relative to us}
        \wrongchoice{in the same direction as the spaceships at $0.8c$ relative to us}
      \correctchoice{in the same direction as the spaceships at $0.95c$ relative to us}
        \wrongchoice{in the opposite direction from the spaceships at $0.95c$ relative to us}
    \end{choices}
\end{question}
}

\element{halliday-mc}{
\begin{question}{halliday-ch37-q11}
    Two events occur on the $x$ axis separated in time by $\Delta t$ and in space by $\Delta X$.
    A reference frame,
        traveling at less than the speed of light,
        in which the two events occur at the same time:
    \begin{choices}
        \wrongchoice{exists no matter what the values of $\Delta x$ and $\Delta t$}
        \wrongchoice{exists only if $\dfrac{\Delta x}{\Delta t} < c$}
      \correctchoice{exists only if $\dfrac{\Delta x}{\Delta t} > c$}
        \wrongchoice{exists only if $\dfrac{\Delta x}{\Delta t} = c$}
        \wrongchoice{does not exist under any condition}
    \end{choices}
\end{question}
}

\element{halliday-mc}{
\begin{question}{halliday-ch37-q12}
    Two events occur on the $x$ axis separated in time by $\Delta t$ and in space by $\Delta x$. 
    A reference frame,
        traveling at less than the speed of light,
        in which the two events occur at the same coordinate:
    \begin{choices}
        \wrongchoice{exists no matter what the values of $\Delta x$ and $\Delta t$}
      \correctchoice{exists only if $\dfrac{\Delta x}{\Delta t} < c$}
        \wrongchoice{exists only if $\dfrac{\Delta x}{\Delta t} > c$}
        \wrongchoice{exists only if $\dfrac{\Delta x}{\Delta t} = c$}
        \wrongchoice{does not exist under any condition}
    \end{choices}
\end{question}
}

\element{halliday-mc}{
\begin{question}{halliday-ch37-q13}
    As we watch, a spaceship passes us in time $\Delta t$. 
    The crew of the spaceship measures the passage time and finds it to be $\Delta t^{\prime}$. 
    Which of the following statements is true?
    \begin{choices}
        \wrongchoice{$\Delta t$ is the proper time for the passage and it is smaller than $\Delta t^{\prime}$}
        \wrongchoice{$\Delta t$ is the proper time for the passage and it is greater than $\Delta t^{\prime}$}
      \correctchoice{$\Delta t$ is the proper time for the passage and it is smaller than $\Delta t$}
        \wrongchoice{$\Delta t$ is the proper time for the passage and it is greater than $\Delta t$}
        \wrongchoice{None of the provided statements are true.}
    \end{choices}
\end{question}
}

\element{halliday-mc}{
\begin{question}{halliday-ch37-q14}
    A millionairess was told in 1992 that she had exactly 15 years to live. 
    However, if she travels away from the Earth at $0.8c$ and then returns at the same speed,
        the last New Year's day the doctors expect her to celebrate is:
    \begin{multicols}{3}
    \begin{choices}
        \wrongchoice{2001}
        \wrongchoice{2003}
        \wrongchoice{2007}
        \wrongchoice{2010}
      \correctchoice{2017}
    \end{choices}
    \end{multicols}
\end{question}
}

\element{halliday-mc}{
\begin{question}{halliday-ch37-q15}
    Two events occur \SI{100}{\meter} apart with an intervening time interval of \SI{0.60}{\micro\second}. 
    The speed of a reference frame in which they occur at the same coordinate is:
    \begin{multicols}{3}
    \begin{choices}
        \wrongchoice{$0$}
        \wrongchoice{$0.25c$}
      \correctchoice{$0.56c$}
        \wrongchoice{$1.1c$}
        \wrongchoice{$1.8c$}
    \end{choices}
    \end{multicols}
\end{question}
}

\element{halliday-mc}{
\begin{question}{halliday-ch37-q16}
    Two independent events occur \SI{100}{\meter} apart with an intervening time interval of \SI{0.42}{\micro\second}. 
    The proper time between the events is:
    \begin{multicols}{3}
    \begin{choices}
        \wrongchoice{zero}
        \wrongchoice{\SI{0.16}{\micro\second}}
      \correctchoice{\SI{0.26}{\micro\second}}
        \wrongchoice{\SI{0.42}{\micro\second}}
        \wrongchoice{\SI{0.69}{\micro\second}}
    \end{choices}
    \end{multicols}
\end{question}
}

\element{halliday-mc}{
\begin{question}{halliday-ch37-q17}
    Two events occur \SI{100}{\meter} apart with an intervening time interval of \SI{0.37}{\micro\second}. 
    The speed of a clock that measures the proper time between the events is:
    \begin{multicols}{3}
    \begin{choices}
        \wrongchoice{zero}
        \wrongchoice{$0.45c$}
        \wrongchoice{$0.56c$}
      \correctchoice{$0.90c$}
        \wrongchoice{$1.8c$}
    \end{choices}
    \end{multicols}
\end{question}
}

\element{halliday-mc}{
\begin{question}{halliday-ch37-q18}
    A rocket traveling with constant velocity makes a \SI{8.4e15}{\meter} trip in exactly one year. 
    The proper time in years between events that mark the beginning and end of the trip is:
    \begin{multicols}{3}
    \begin{choices}
        \wrongchoice{\SI{0.21}{\year}}
        \wrongchoice{\SI{0.46}{\year}}
        \wrongchoice{\SI{1.0}{\year}}
        \wrongchoice{\SI{2.2}{\year}}
        \wrongchoice{\SI{4.7}{\year}}
    \end{choices}
    \end{multicols}
\end{question}
}

\element{halliday-mc}{
\begin{question}{halliday-ch37-q19}
    An observer notices that a moving clock runs slow by a factor of exactly 10. 
    The speed of the clock is:
    \begin{multicols}{3}
    \begin{choices}
        \wrongchoice{$0.100c$}
        \wrongchoice{$0.0100c$}
        \wrongchoice{$0.990c$}
        \wrongchoice{$0.900c$}
      \correctchoice{$0.995c$}
    \end{choices}
    \end{multicols}
\end{question}
}

\element{halliday-mc}{
\begin{question}{halliday-ch37-q20}
    A meson when at rest decays \SI{2}{\micro\second} after it is created. 
    If moving in the laboratory at $0.99c$,
        its lifetime according to laboratory clocks would be:
    \begin{multicols}{2}
    \begin{choices}
        \wrongchoice{the same}
        \wrongchoice{\SI{0.28}{si}}
      \correctchoice{\SI{14}{\micro\second}}
        \wrongchoice{\SI{4.6}{\second}}
        \wrongchoice{none of the provided}
    \end{choices}
    \end{multicols}
\end{question}
}

\element{halliday-mc}{
\begin{question}{halliday-ch37-q21}
    Pi mesons at rest have a half-life of $T$. 
    If a beam of pi mesons is traveling at a speed of $v=\beta c$,
        the distance in which the intensity of the beam is halved is:
    \begin{multicols}{2}
    \begin{choices}
      \correctchoice{$\dfrac{c\beta T}{\sqrt{1-\beta^2}}$}
        \wrongchoice{$c\beta T \sqrt{\dfrac{1+\beta}{1-\beta}}$}
        \wrongchoice{$\beta vT$}
        \wrongchoice{$\sqrt{1-\beta^2} vT$}
        \wrongchoice{none of the provided}
    \end{choices}
    \end{multicols}
\end{question}
}

\element{halliday-mc}{
\begin{question}{halliday-ch37-q22}
    A meson moving through a laboratory of length $x$ at a speed $v$ decays after a lifetime $T$ as measured by an observer at rest in the laboratory. 
    If the meson were at rest in the laboratory its lifetime would be:
    \begin{multicols}{2}
    \begin{choices}
        \wrongchoice{$T\left(1-\dfrac{v}{c}\right)$}
        \wrongchoice{$\dfrac{T}{1-\dfrac{v}{c}}$}
        \wrongchoice{$\dfrac{T}{\sqrt{1-\dfrac{v}{c}}}$}
      \correctchoice{$T\sqrt{1-\dfrac{v}{c}}$}
        \wrongchoice{$\dfrac{T-\dfrac{vx}{c^2}}{\sqrt{1-\dfrac{v^2}{c^2}}}$}
    \end{choices}
    \end{multicols}
\end{question}
}

\element{halliday-mc}{
\begin{question}{halliday-ch37-q23}
    A measurement of the length of an object that is moving relative to the laboratory consists of noting the coordinates of the front and back:
    \begin{choices}
        \wrongchoice{at different times according to clocks at rest in the laboratory}
        \wrongchoice{at the same time according to clocks that move with the object}
      \correctchoice{at the same time according to clocks at rest in the laboratory}
        \wrongchoice{at the same time according to clocks at rest with respect to the fixed stars}
        \wrongchoice{none of the provided}
    \end{choices}
\end{question}
}

\element{halliday-mc}{
\begin{question}{halliday-ch37-q24}
    A meter stick moves in the direction of its length through a laboratory. 
    According to measurements taken in the laboratory,
        its length is \SI{0.31}{\meter}.
    The speed of the meter stick relative to the laboratory is:
    \begin{multicols}{3}
    \begin{choices}
        \wrongchoice{$0.096c$}
        \wrongchoice{$0.31c$}
        \wrongchoice{$0.69c$}
        \wrongchoice{$0.83c$}
      \correctchoice{$0.95c$}
    \end{choices}
    \end{multicols}
\end{question}
}

\element{halliday-mc}{
\begin{question}{halliday-ch37-q25}
    A meter stick moves sideways at $0.95c$. 
    According to measurements taken in the laboratory,
        its length is:
    \begin{multicols}{3}
    \begin{choices}
        \wrongchoice{zero}
        \wrongchoice{\SI{0.098}{\meter}}
        \wrongchoice{\SI{0.31}{\meter}}
        \wrongchoice{\SI{3.2}{\meter}}
      \correctchoice{\SI{1.0}{\meter}}
    \end{choices}
    \end{multicols}
\end{question}
}

\element{halliday-mc}{
\begin{question}{halliday-ch37-q26}
    A rocket ship of rest length \SI{100}{\meter} is moving at speed $0.8c$ past a timing device that records the time interval between the passage of the front and back ends of the ship.
    This time interval is:
    \begin{multicols}{3}
    \begin{choices}
        \wrongchoice{\SI{0.20}{\micro\second}}
      \correctchoice{\SI{0.25}{\micro\second}}
        \wrongchoice{\SI{0.33}{\micro\second}}
        \wrongchoice{\SI{0.52}{\micro\second}}
        \wrongchoice{\SI{0.69}{\micro\second}}
    \end{choices}
    \end{multicols}
\end{question}
}

\element{halliday-mc}{
\begin{question}{halliday-ch37-q27}
    The length of a meter stick moving at $0.95c$ in the direction of its length is measured by simultaneously marking its ends on a stationary axis. 
    As measured by clocks moving with the stick,
        the time interval between the making of the back mark and the making of the front mark is:
    \begin{multicols}{2}
    \begin{choices}
        \wrongchoice{zero}
        \wrongchoice{\SI{3.1e-10}{\second}}
        \wrongchoice{\SI{1.0e-9}{\second}}
      \correctchoice{\SI{3.2e-9}{\second}}
        \wrongchoice{\SI{1.0e-8}{\second}}
    \end{choices}
    \end{multicols}
\end{question}
}

\element{halliday-mc}{
\begin{question}{halliday-ch37-q28}
    A certain automobile is \SI{6}{\meter} long if at rest. 
    If it is measured to be \num{4/5} as long,
        its speed is:
    \begin{multicols}{3}
    \begin{choices}
        \wrongchoice{$0.1c$}
        \wrongchoice{$0.3c$}
      \correctchoice{$0.6c$}
        \wrongchoice{$0.8c$}
        \wrongchoice{$>0.95c$}
    \end{choices}
    \end{multicols}
\end{question}
}

\element{halliday-mc}{
\begin{question}{halliday-ch37-q29}
    As a rocket ship moves by at $0.95c$ a mark is made on a stationary axis at the front end of the rocket and \SI{9e-8}{\second} later a mark is made on the axis at the back end.
    The marks are found to be \SI{100}{\meter} apart. 
    The rest length of the rocket is:
    \begin{multicols}{3}
    \begin{choices}
        \wrongchoice{\SI{31}{\meter}}
        \wrongchoice{\SI{78}{\meter}}
        \wrongchoice{\SI{100}{\meter}}
      \correctchoice{\SI{240}{\meter}}
        \wrongchoice{\SI{320}{\meter}}
    \end{choices}
    \end{multicols}
\end{question}
}

\element{halliday-mc}{
\begin{question}{halliday-ch37-q30}
    A clock is moving along the $x$ axis at $0.6c$. 
    It reads zero as it passes the origin ($x=0$). 
    When it passes the $x=\SI{180}{\meter}$ mark on the $x$ axis the clock reads:
    \begin{multicols}{3}
    \begin{choices}
        \wrongchoice{\SI{0.60}{\micro\second}}
      \correctchoice{\SI{0.80}{\micro\second}}
        \wrongchoice{\SI{1.00}{\micro\second}}
        \wrongchoice{\SI{1.25}{\micro\second}}
        \wrongchoice{\SI{1.67}{\micro\second}}
    \end{choices}
    \end{multicols}
\end{question}
}

\element{halliday-mc}{
\begin{question}{halliday-ch37-q31}
    Relative to reference frame 1,
        reference frame 2 moves with speed $v$ in the negative $x$ direction.
    When the origins of the two frames coincide the clocks in both frames are set to zero. 
    An event occurs at coordinate $x_1$ and time $t_1$ as measured in reference frame 1 and at coordinate $x_2$ and time $t_2$ as measured in frame 2.
    If $\gamma=\dfrac{1}{\sqrt{1-\frac{v^2}{v^2}}}$,
        then the coordinates and times of the event are related by:
    \begin{choices}
        \wrongchoice{$x_2=\gamma\left(x_1-vt_1\right)$ and $t_2 = \gamma\left(t_1-\dfrac{vx_1}{c^2}\right)$}
        \wrongchoice{$x_2=\gamma\left(x_1-vt_1\right)$ and $t_2 = \gamma\left(t_1+\dfrac{vx_1}{c^2}\right)$}
        \wrongchoice{$x_2=\gamma\left(x_1+vt_1\right)$ and $t_2 = \gamma\left(t_1-\dfrac{vx_1}{c^2}\right)$}
      \correctchoice{$x_2=\gamma\left(x_1+vt_1\right)$ and $t_2 = \gamma\left(t_1+\dfrac{vx_1}{c^2}\right)$}
        \wrongchoice{none of the provided are correct}
    \end{choices}
\end{question}
}

\element{halliday-mc}{
\begin{question}{halliday-ch37-q32}
    An event occurs at $x=\SI{500}{\meter}$, $t=\SI{0.90}{\micro\second}$ in one frame of reference. 
    Another frame is moving at $0.90c$ in the positive $x$ direction. 
    The origins coincide at $t=0$ and clocks in the second frame are zeroed when the origins coincide. 
    The coordinate and time of the event in the second frame are:
    \begin{multicols}{2}
    \begin{choices}
        \wrongchoice{\SI{500}{\meter},  \SI{0.90}{\micro\second}}
        \wrongchoice{\SI{1700}{\meter}, \SI{5.5}{\micro\second}}
        \wrongchoice{\SI{740}{\meter},  \SI{2.4}{\micro\second}}
        \wrongchoice{\SI{260}{\meter},  \SI{-0.60}{\micro\second}}
      \correctchoice{\SI{590}{\meter},  \SI{-1.4}{\micro\second}}
    \end{choices}
    \end{multicols}
\end{question}
}

\element{halliday-mc}{
\begin{question}{halliday-ch37-q33}
    An event occurs at $x=\SI{500}{\meter}$, $t=\SI{0.90}{\micro\second}$ in one frame of reference. 
    Another frame is moving at $0.90c$ in the negative $x$ direction. 
    The origins coincide at $t=0$ and clocks in the second frame are zeroed when the origins coincide. 
    The coordinate and time of the event in the second frame are:
    \begin{multicols}{2}
    \begin{choices}
        \wrongchoice{\SI{500}{\meter},  \SI{0.90}{\micro\second}}
      \correctchoice{\SI{1700}{\meter}, \SI{5.5}{\micro\second}}
        \wrongchoice{\SI{740}{\meter},  \SI{2.4}{\micro\second}}
        \wrongchoice{\SI{260}{\meter},  \SI{-0.60}{\micro\second}}
        \wrongchoice{\SI{590}{\meter},  \SI{-1.4}{\micro\second}}
    \end{choices}
    \end{multicols}
\end{question}
}

\element{halliday-mc}{
\begin{question}{halliday-ch37-q34}
    Two flashes of light occur simultaneously at $t=0$ in reference frame $S$,
        one at $x=0$ and the other at $x=\SI{600}{\meter}$.
    They are observed in reference frame $S^{\prime}$,
        which is moving at $0.95c$ in the positive $x$ direction. 
    The origins of the two frames coincide at $t=0$ and the clocks of $S^{\prime}$ are zeroed when the origins coincide. 
    In $S^{\prime}$ the coordinate where the leading edges of the two light flashes meet and the time when they meet are:
    \begin{multicols}{2}
    \begin{choices}
        \wrongchoice{\SI{300}{\meter},  \SI{1.0}{\micro\second}}
        \wrongchoice{\SI{15}{\meter},   \SI{0.050}{\micro\second}}
        \wrongchoice{\SI{585}{\meter},  \SI{1.95}{\micro\second}}
      \correctchoice{\SI{49}{\meter},   \SI{0.16}{\micro\second}}
        \wrongchoice{\SI{1900}{\meter}, \SI{0.16}{\micro\second}}
    \end{choices}
    \end{multicols}
\end{question}
}

\element{halliday-mc}{
\begin{question}{halliday-ch37-q35}
    Frame $S^{\prime}$ moves in the positive $x$ direction at $0.6c$ with respect to frame $S$. 
    A particle moves in the positive $x$ direction at $0.4c$ as measured by an observer in $S^{\prime}$. 
    The speed of the particle as measured by an observer in $S$ is:
    \begin{multicols}{3}
    \begin{choices}
        \wrongchoice{$\dfrac{c}{5}$}
        \wrongchoice{$\dfrac{5c}{19}$}
        \wrongchoice{$\dfrac{8c}{25}$}
      \correctchoice{$\dfrac{25c}{31}$}
        \wrongchoice{$c$}
    \end{choices}
    \end{multicols}
\end{question}
}

\element{halliday-mc}{
\begin{question}{halliday-ch37-q36}
    Star S1 is moving away from us at a speed of $0.8c$. 
    Star S2 is moving away from us in the opposite direction at a speed of $0.5c$.
    The speed of S1 as measured by an observer on S2 is:
    \begin{multicols}{3}
    \begin{choices}
        \wrongchoice{$0.21c$}
        \wrongchoice{$0.5c$}
      \correctchoice{$0.93c$}
        \wrongchoice{$1.3c$}
        \wrongchoice{$2.17c$}
    \end{choices}
    \end{multicols}
\end{question}
}

\element{halliday-mc}{
\begin{question}{halliday-ch37-q37}
    Observer $A$ measures the velocity of a rocket as $v$ and a comet as $u$. 
    Here $u$ and $v$ are parallel and in the direction of the observer's positive $x$ axis. 
    The speed of the comet as measured by observer $B$ on the rocket is:
    \begin{multicols}{2}
    \begin{choices}
      \correctchoice{$\dfrac{u-v}{1-\dfrac{uv}{c^2}}$}
        \wrongchoice{$\dfrac{u-v}{1-\dfrac{v^2}{c^2}}$}
        \wrongchoice{$\dfrac{u-v}{\sqrt{1-\dfrac{uv}{c^2}}}$}
        \wrongchoice{$\dfrac{u-v}{1+\dfrac{uv}{c^2}}$}
        \wrongchoice{$\dfrac{u+v}{1-\dfrac{uv}{c^2}}$}
    \end{choices}
    \end{multicols}
\end{question}
}

\element{halliday-mc}{
\begin{question}{halliday-ch37-q38}
    Two electrons move in opposite directions at $0.70c$ as measured in the laboratory. 
    The speed of one electron as measured from the other is:
    \begin{multicols}{3}
    \begin{choices}
        \wrongchoice{$0.35c$}
        \wrongchoice{$0.70c$}
      \correctchoice{$0.94c$}
        \wrongchoice{$1.00c$}
        \wrongchoice{$1.40c$}
    \end{choices}
    \end{multicols}
\end{question}
}

\element{halliday-mc}{
\begin{question}{halliday-ch37-q39}
    Light from some stars shows an apparent change in frequency because of:
    \begin{choices}
        \wrongchoice{interference}
        \wrongchoice{refraction by layers of air}
        \wrongchoice{diffraction}
        \wrongchoice{reflection}
      \correctchoice{relative motion}
    \end{choices}
\end{question}
}

\element{halliday-mc}{
\begin{question}{halliday-ch37-q40}
    While emitting light of proper frequency $f_0$,
        a source moves to the right with speed $\dfrac{c}{4}$ relative to reference frame $S$. 
    \begin{center}
    \begin{tikzpicture}
        %% NOTE: tikz
    \end{tikzpicture}
    \end{center}
    A detector, to the left of the source, measures the frequency to be $f$,
        which is greater than $f_0$. 
    This means:
    \begin{choices}
      \correctchoice{the detector is moving to the right with a speed that is greater than $\dfrac{c}{4}$ relative to $S$}
        \wrongchoice{the detector is moving to the right with a speed that is less than $\dfrac{c}{4}$ relative to $S$}
        \wrongchoice{the detector is moving to the left with a speed that is greater than $\dfrac{c}{4}$ relative to $S$}
        \wrongchoice{the detector is moving to the left with a speed that is less than $\dfrac{c}{4}$ relative to $S$}
        \wrongchoice{the detector is not moving}
    \end{choices}
\end{question}
}

\element{halliday-mc}{
\begin{question}{halliday-ch37-q41}
    Light from a stationary spaceship is observed,
        then the spaceship moves directly away from the observer at high speed while still emitting the light. 
    As a result, the light seen by the observer has:
    \begin{choices}
        \wrongchoice{a higher frequency and a longer wavelength than before}
        \wrongchoice{a lower frequency and a shorter wavelength than before}
        \wrongchoice{a higher frequency and a shorter wavelength than before}
      \correctchoice{a lower frequency and a longer wavelength than before}
        \wrongchoice{the same frequency and wavelength as before}
    \end{choices}
\end{question}
}

\element{halliday-mc}{
\begin{question}{halliday-ch37-q42}
    A train traveling very fast ($v=0.6c$) has an engineer ($E$) at the front,
        a guard ($G$) at the rear, and a passenger ($S^{\prime}$) exactly halfway between them. 
    Both $E$ and $G$ are equipped with yellow signaling lamps. 
    The train passes a station, closely observed by the station master ($S$).
    Both $E$ and $G$ use their lamps to send signals. 
    According to both $S^{\prime}$ and $S$ these signals arrive simultaneously at the instant $S^{\prime}$ is passing $S$. 
    According to S,
        the signal from $E$ will look \rule[-0.1pt]{4em}{0.1pt} and that from $G$ will look \rule[-0.1pt]{4em}{0.1pt}.
    \begin{multicols}{2}
    \begin{choices}
      \correctchoice{red, blue}
        \wrongchoice{yellow, yellow}
        \wrongchoice{blue, red}
        \wrongchoice{blue, blue}
        \wrongchoice{red, red}
    \end{choices}
    \end{multicols}
\end{question}
}

\element{halliday-mc}{
\begin{question}{halliday-ch37-q43}
    A console lamp in the cabin of a spaceship appears green when the ship and observer are both at rest on Earth. 
    When the ship is moving at $0.90c$ away from Earth,
        passengers on board see:
    \begin{choices}
        \wrongchoice{nothing (the frequency is too high to be seen)}
        \wrongchoice{nothing (the frequency is too low to be seen)}
        \wrongchoice{a red lamp}
        \wrongchoice{a violet lamp}
      \correctchoice{a green lamp}
    \end{choices}
\end{question}
}

\element{halliday-mc}{
\begin{question}{halliday-ch37-q44}
    A spectral line of a certain star is observed to be ``red shifted'' from a wavelength of \SI{500}{\nano\meter} to a wavelength of \SI{1500}{\nano\meter}. 
    Interpreting this as a Doppler effect,
        the speed of recession of this star is:
    \begin{multicols}{3}
    \begin{choices}
        \wrongchoice{$0.33c$}
        \wrongchoice{$0.50c$}
        \wrongchoice{$0.71c$}
      \correctchoice{$0.8c$}
        \wrongchoice{$c$}
    \end{choices}
    \end{multicols}
\end{question}
}

\element{halliday-mc}{
\begin{question}{halliday-ch37-q45}
    A source at rest emits light of wavelength \SI{500}{\nano\meter}. 
    When it is moving at $0.90c$ toward an observer,
        the observer detects light of wavelength:
    \begin{multicols}{3}
    \begin{choices}
        \wrongchoice{\SI{26}{\nano\meter}}
      \correctchoice{\SI{115}{\nano\meter}}
        \wrongchoice{\SI{500}{\nano\meter}}
        \wrongchoice{\SI{2200}{\nano\meter}}
        \wrongchoice{\SI{9500}{\nano\meter}}
    \end{choices}
    \end{multicols}
\end{question}
}

\element{halliday-mc}{
\begin{question}{halliday-ch37-q46}
    A source at rest emits light of wavelength \SI{500}{\nano\meter}. 
    When it is moving at $0.90c$ away from an observer,
        the observer detects light of wavelength:
    \begin{multicols}{3}
    \begin{choices}
        \wrongchoice{\SI{26}{\nano\meter}}
        \wrongchoice{\SI{115}{\nano\meter}}
        \wrongchoice{\SI{500}{\nano\meter}}
      \correctchoice{\SI{2200}{\nano\meter}}
        \wrongchoice{\SI{9500}{\nano\meter}}
    \end{choices}
    \end{multicols}
\end{question}
}

\element{halliday-mc}{
\begin{question}{halliday-ch37-q47}
    Visible light, with a frequency of \SI{6.0e14}{\hertz},
        is reflected from a spaceship moving directly away at a speed of $0.90c$. 
    The frequency of the reflected waves observed at the source is:
    \begin{multicols}{2}
    \begin{choices}
      \correctchoice{\SI{3.2e13}{\hertz}}
        \wrongchoice{\SI{1.4e14}{\hertz}}
        \wrongchoice{\SI{6.0e14}{\hertz}}
        \wrongchoice{\SI{2.6e15}{\hertz}}
        \wrongchoice{\SI{1.1e16}{\hertz}}
    \end{choices}
    \end{multicols}
\end{question}
}

\element{halliday-mc}{
\begin{question}{halliday-ch37-q48}
    How fast should you move away from a \SI{6.0e14}{\hertz} light source to observe waves with a frequency of \SI{4.0e14}{\hertz}?
    \begin{multicols}{3}
    \begin{choices}
        \wrongchoice{$0.20c$}
      \correctchoice{$0.38c$}
        \wrongchoice{$0.45c$}
        \wrongchoice{$0.51c$}
        \wrongchoice{$0.76c$}
    \end{choices}
    \end{multicols}
\end{question}
}

\element{halliday-mc}{
\begin{question}{halliday-ch37-q49}
    A particle with rest mass $m$ moves with speed $0.6c$. 
    Its kinetic energy is:
    \begin{multicols}{3}
    \begin{choices}
        \wrongchoice{$0.18 mc^2$}
        \wrongchoice{$0.22 mc^2$}
      \correctchoice{$0.25 mc^2$}
        \wrongchoice{$mc^2$}
        \wrongchoice{$1.25 mc^2$}
    \end{choices}
    \end{multicols}
\end{question}
}

\element{halliday-mc}{
\begin{question}{halliday-ch37-q50}
    An electron is moving at $0.6c$. 
    If we calculate its kinetic energy using $\dfrac{1}{2} mv^2$,
        we get a result that is:
    \begin{choices}
        \wrongchoice{just right}
        \wrongchoice{just half enough}
        \wrongchoice{twice the correct value}
        \wrongchoice{about \SI{1}{\percent} too low}
      \correctchoice{about \SI{28}{\percent} too low}
    \end{choices}
\end{question}
}

\element{halliday-mc}{
\begin{question}{halliday-ch37-q51}
    The velocity of an electron is changed from $\dfrac{c}{2}$ in the $-x$ direction to $\dfrac{c}{2}$ in the $+x$ direction.
    As a result,
        its kinetic energy changes by:
    \begin{multicols}{3}
    \begin{choices}
        \wrongchoice{$mc^2$}
        \wrongchoice{$0.5 mc^2$}
        \wrongchoice{$\sqrt{2} mc^2$}
        \wrongchoice{$2mc^2$}
      \correctchoice{zero}
    \end{choices}
    \end{multicols}
\end{question}
}

\element{halliday-mc}{
\begin{question}{halliday-ch37-q52}
    The work that must be done to increase the speed of an electron
        ($m=\SI{9.11e-31}{\kilo\gram}$) from $0.90c$ to $0.95c$ is:
    \begin{multicols}{2}
    \begin{choices}
        \wrongchoice{\SI{2.6e-13}{\joule}}
        \wrongchoice{\SI{8.2e-13}{\joule}}
        \wrongchoice{\SI{3.2e-13}{\joule}}
      \correctchoice{\SI{7.4e-14}{\joule}}
        \wrongchoice{\SI{3.8e-15}{\joule}}
    \end{choices}
    \end{multicols}
\end{question}
}

\element{halliday-mc}{
\begin{question}{halliday-ch37-q53}
    An electron ($m=\SI{9.11e-31}{\kilo\gram}$) has a speed of $0.95c$.
    Its kinetic energy is:
    \begin{multicols}{2}
    \begin{choices}
        \wrongchoice{\SI{8.2e-14}{\joule}}
      \correctchoice{\SI{1.8e-13}{\joule}}
        \wrongchoice{\SI{2.0e-13}{\joule}}
        \wrongchoice{\SI{2.2e-13}{\joule}}
        \wrongchoice{\SI{2.6e-13}{\joule}}
    \end{choices}
    \end{multicols}
\end{question}
}

\element{halliday-mc}{
\begin{question}{halliday-ch37-q54}
    An electron ($m=\SI{9.11e-31}{\kilo\gram}$) has a speed of $0.95c$.
    The magnitude of its momentum is:
    \begin{multicols}{2}
    \begin{choices}
        \wrongchoice{\SI{2.6e-22}{\kilo\gram\meter\per\second}}
        \wrongchoice{\SI{2.9e-22}{\kilo\gram\meter\per\second}}
        \wrongchoice{\SI{6.0e-22}{\kilo\gram\meter\per\second}}
      \correctchoice{\SI{8.3e-22}{\kilo\gram\meter\per\second}}
        \wrongchoice{\SI{8.8e-22}{\kilo\gram\meter\per\second}}
    \end{choices}
    \end{multicols}
\end{question}
}

\element{halliday-mc}{
\begin{question}{halliday-ch37-q55}
    According to the special theory of relativity:
    \begin{choices}
      \correctchoice{all forms of energy have mass-like properties}
        \wrongchoice{moving particles lose mass}
        \wrongchoice{momentum is not conserved in high speed collisions}
        \wrongchoice{a rod moving rapidly sideways is shorter along its length}
        \wrongchoice{a rod moving rapidly sideways is longer along its length}
    \end{choices}
\end{question}
}

\element{halliday-mc}{
\begin{question}{halliday-ch37-q56}
    The mass of a particle is $m$.
    In order for its total energy to be twice its rest energy,
        its momentum must be:
    \begin{multicols}{3}
    \begin{choices}
        \wrongchoice{$\dfrac{mc}{2}$}
        \wrongchoice{$\dfrac{mc}{\sqrt{2}}$}
        \wrongchoice{$mc$}
      \correctchoice{$\sqrt{3}mc$}
        \wrongchoice{$2mc$}
    \end{choices}
    \end{multicols}
\end{question}
}

\element{halliday-mc}{
\begin{question}{halliday-ch37-q57}
    If the kinetic energy of a particle is equal to its rest energy then its speed must be:
    %\begin{multicols}{2}
    \begin{choices}
        \wrongchoice{$0.25c$}
        \wrongchoice{$0.50c$}
      \correctchoice{$0.87c$}
        \wrongchoice{$c$}
        \wrongchoice{unknown unless its mass is given}
    \end{choices}
    %\end{multicols}
\end{question}
}

\element{halliday-mc}{
\begin{question}{halliday-ch37-q58}
    If the mass of a particle is zero its speed must be:
    %\begin{multicols}{2}
    \begin{choices}
      \correctchoice{$c$}
        \wrongchoice{infinite}
        \wrongchoice{$0$}
        \wrongchoice{any speed less than $c$}
        \wrongchoice{any speed greater than $c$}
    \end{choices}
    %\end{multicols}
\end{question}
}

\element{halliday-mc}{
\begin{question}{halliday-ch37-q59}
    A particle with zero mass and energy $E$ carries momentum:
    \begin{multicols}{3}
    \begin{choices}
        \wrongchoice{$Ec$}
        \wrongchoice{$Ec^2$}
        \wrongchoice{$\sqrt{Ec}$}
      \correctchoice{$\dfrac{E}{c}$}
        \wrongchoice{$\dfrac{E}{c^2}$}
    \end{choices}
    \end{multicols}
\end{question}
}

\element{halliday-mc}{
\begin{question}{halliday-ch37-q60}
    According to relativity theory a particle of mass $m$ with a momentum of $2mc$ has a speed of:
    \begin{multicols}{3}
    \begin{choices}
        \wrongchoice{$2c$}
        \wrongchoice{$4c$}
        \wrongchoice{$c$}
        \wrongchoice{$\dfrac{c}{2}$}
      \correctchoice{$0.89c$}
    \end{choices}
    \end{multicols}
\end{question}
}

\element{halliday-mc}{
\begin{question}{halliday-ch37-q61}
    If the kinetic energy of a free particle is much greater than its rest energy then its kinetic energy is proportional to:
    \begin{choices}
      \correctchoice{the magnitude of its momentum}
        \wrongchoice{the square of the magnitude of its momentum}
        \wrongchoice{the square root of the magnitude of its momentum}
        \wrongchoice{the reciprocal of the magnitude of its momentum}
        \wrongchoice{none of the above}
    \end{choices}
\end{question}
}

\element{halliday-mc}{
\begin{question}{halliday-ch37-q62}
    If the kinetic energy of a free particle is much less than its rest energy then its kinetic energy is proportional to:
    \begin{choices}
        \wrongchoice{the magnitude of its momentum}
      \correctchoice{the square of the magnitude of its momentum}
        \wrongchoice{the square root of the magnitude of its momentum}
        \wrongchoice{the reciprocal of the magnitude of its momentum}
        \wrongchoice{none of the above}
    \end{choices}
\end{question}
}

\element{halliday-mc}{
\begin{question}{halliday-ch37-q63}
    The magnitude of the momentum of a particle can never exceed:
    \begin{choices}
        \wrongchoice{$mc$, where $m$ is its mass}
      \correctchoice{$\dfrac{E}{c}$, where $E$ is its total energy}
        \wrongchoice{$\dfrac{K}{c}$, where $K$ is its kinetic energy}
        \wrongchoice{none of the above, but there is an upper limit}
        \wrongchoice{none of the above; there is no upper limit}
    \end{choices}
\end{question}
}

\element{halliday-mc}{
\begin{question}{halliday-ch37-q64}
    An electron ($m=\SI{9.11e-31}{\kilo\gram}$) has a momentum of \SI{4.0e-22}{\kilo\gram\meter\per\second}.
    Its kinetic energy is:
    \begin{multicols}{2}
    \begin{choices}
      \correctchoice{\SI{6.3e-14}{\joule}}
        \wrongchoice{\SI{8.2e-14}{\joule}}
        \wrongchoice{\SI{1.2e-13}{\joule}}
        \wrongchoice{\SI{1.5e-13}{\joule}}
        \wrongchoice{\SI{2.7e-13}{\joule}}
    \end{choices}
    \end{multicols}
\end{question}
}

\element{halliday-mc}{
\begin{question}{halliday-ch37-q65}
    A certain particle has a kinetic energy of \SI{3.2e-10}{\joule} and a momentum of \SI{1.7e-18}{\kilo\gram\meter\per\second}.
    Its mass is:
    \begin{multicols}{2}
    \begin{choices}
        \wrongchoice{\SI{9.11e-31}{\kilo\gram}}
      \correctchoice{\SI{2.7e-27}{\kilo\gram}}
        \wrongchoice{\SI{4.5e-27}{\kilo\gram}}
        \wrongchoice{\SI{6.3e-27}{\kilo\gram}}
        \wrongchoice{\SI{8.6e-27}{\kilo\gram}}
    \end{choices}
    \end{multicols}
\end{question}
}

\element{halliday-mc}{
\begin{question}{halliday-ch37-q66}
    An electron ($m=\SI{9.11e-31}{\kilo\gram}$, $q=\SI{1.60e-19}{\coulomb}$) travels at $0.95c$ around a circular orbit perpendicular to a uniform \SI{1.8}{\tesla} magnetic field. 
    The radius of its orbit is:
    \begin{multicols}{3}
    \begin{choices}
        \wrongchoice{\SI{0.28}{\milli\meter}}
        \wrongchoice{\SI{0.90}{\milli\meter}}
        \wrongchoice{\SI{1.1}{\milli\meter}}
      \correctchoice{\SI{2.9}{\milli\meter}}
        \wrongchoice{\SI{4.7}{\milli\meter}}
    \end{choices}
    \end{multicols}
\end{question}
}

\element{halliday-mc}{
\begin{question}{halliday-ch37-q67}
    An electron ($m=\SI{9.11e-31}{\kilo\gram}$, $q=\SI{1.60e-19}{\coulomb}$) travels around a \SI{1.7}{\milli\meter} radius circular orbit perpendicular to a \SI{2.8}{\tesla} magnetic field. 
    Its speed is:
    \begin{multicols}{3}
    \begin{choices}
        \wrongchoice{$0.16c$}
        \wrongchoice{$0.36c$}
      \correctchoice{$0.94c$}
        \wrongchoice{$c$}
        \wrongchoice{$2.8c$}
    \end{choices}
    \end{multicols}
\end{question}
}


\endinput


