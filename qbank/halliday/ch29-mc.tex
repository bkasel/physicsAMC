
%%--------------------------------------------------
%% Halliday: Fundamentals of Physics
%%--------------------------------------------------


%% Chapter 29: Magnetic Fields Due to Currents
%%--------------------------------------------------


%% Learning Objectives
%%--------------------------------------------------

%% 29.01: Sketch a current-length element in a wire and indicate the direction of the magnetic field that it sets up at a given point near the wire.
%% 29.02: For a given point near a wire and a given current-element in the wire, determine the magnitude and direction of the magnetic field due to that element.
%% 29.03: Identify the magnitude of the magnetic field set up by a current-length element at a point in line with the direction of that element.
%% 29.04: For a point to one side of a long straight wire carrying current, apply the relationship between the magnetic field magnitude, the current, and the distance to the point.
%% 29.05: For a point to one side of a long straight wire carrying current, use a right-hand rule to determine the direction of the field vector.
%% 29.06: Identify that around a long straight wire carrying current, the magnetic field lines form circles.
%% 29.07: For a point to one side of the end of a semi-infinite wire carrying current, apply the relationship between the magnetic field magnitude, the current, and the distance to the point.
%% 29.08: For the center of curvature of a circular arc of wire carrying current, apply the relationship between the magnetic field magnitude, the current, the radius of curvature, and the angle subtended by the arc (in radians).
%% 29.09: For a point to one side of a short straight wire carrying current, integrate the Biot--Savart law to find the magnetic field set up at the point by the current.


%% Halliday Multiple Choice Questions
%%--------------------------------------------------
\element{halliday-mc}{
\begin{question}{halliday-ch29-q01}
    Suitable units for $\mu_0$ are:
    \begin{choices}
        \wrongchoice{tesla (\si{\tesla})}
        \wrongchoice{newton per ampere squared (\si{\newton\per\ampere\squared})}
        \wrongchoice{weber per meter (\si{\weber\per\meter})}
        \wrongchoice{kilogram ampere per meter (\si{\kilo\gram\ampere\per\meter})}
      \correctchoice{tesla meter per ampere (\si{\tesla\meter\per\ampere})}
    \end{choices}
\end{question}
}

\element{halliday-mc}{
\begin{question}{halliday-ch29-q02}
    A ``coulomb'' is:
    \begin{choices}
        \wrongchoice{one ampere per second}
        \wrongchoice{the quantity of charge that will exert a force of \SI{1}{\newton} on a similar charge at a distance of \SI{1}{\meter}}
        \wrongchoice{the amount of current in each of two long parallel wires, separated by \SI{1}{\meter}, that produces a force of \SI{2e-7}{\newton\per\meter}}
      \correctchoice{the amount of charge that flows past a point in one second when the current is \SI{1}{\ampere}}
        \wrongchoice{an abbreviation for a certain combination of kilogram, meter and second}
    \end{choices}
\end{question}
}

\element{halliday-mc}{
\begin{question}{halliday-ch29-q03}
    Electrons are going around a circle in a counterclockwise direction as shown. 
    \begin{center}
    \begin{tikzpicture}
        %% NOTE: tikz
    \end{tikzpicture}
    \end{center}
    At the center of the circle they produce a magnetic field that is:
    \begin{multicols}{2}
    \begin{choices}
      \correctchoice{into the page}
        \wrongchoice{out of the page}
        \wrongchoice{to the left}
        \wrongchoice{to the right}
        \wrongchoice{zero}
    \end{choices}
    \end{multicols}
\end{question}
}

\element{halliday-mc}{
\begin{question}{halliday-ch29-q04}
    In the figure, the current element $i\,\mathrm{d}\vec{l}$, the point $P$,
        and the three vectors (1, 2, 3) are all in the plane of the page. 
    \begin{center}
    \begin{tikzpicture}
        %% NOTE: tikz
    \end{tikzpicture}
    \end{center}
    The direction of $\mathrm{d}\vec{B}$,
        due to this current element, at the point $P$ is:
    \begin{choices}
        \wrongchoice{in the direction marked ``1''}
        \wrongchoice{in the direction marked ``2''}
        \wrongchoice{in the direction marked ``3''}
        \wrongchoice{out of the page}
      \correctchoice{into the page}
    \end{choices}
\end{question}
}

\element{halliday-mc}{
\begin{question}{halliday-ch29-q05}
    The magnitude of the magnetic field at point $P$,
    \begin{center}
    \begin{tikzpicture}
        %% NOTE: tikz
    \end{tikzpicture}
    \end{center}
        at the center of the semicircle shown, is given by:
    \begin{multicols}{3}
    \begin{choices}
        \wrongchoice{$\dfrac{2\mu_0 i}{R}$}
        \wrongchoice{$\dfrac{\mu_0 i}{R}$}
        \wrongchoice{$\dfrac{\mu_0 i}{4\pi R}$}
        \wrongchoice{$\dfrac{\mu_0 i}{2R}$}
      \correctchoice{$\dfrac{\mu_0 i}{4R}$}
    \end{choices}
    \end{multicols}
\end{question}
}

\element{halliday-mc}{
\begin{question}{halliday-ch29-q06}
    The diagrams show three circuits consisting of concentric circular arcs (either half or quarter circles of radii $r$, $2r$, and $3r$) and radial lengths. 
    The circuits carry the same current.
    \begin{center}
    \begin{tikzpicture}
        %% NOTE: tikz
    \end{tikzpicture}
    \end{center}
    Rank them according to the magnitudes of the magnetic fields they produce at $C$,
        least to greatest.
    \begin{multicols}{2}
    \begin{choices}
        \wrongchoice{1, 2, 3}
      \correctchoice{3, 2, 1}
        \wrongchoice{1, 3, 2}
        \wrongchoice{2, 3, 1}
        \wrongchoice{2, 1, 3}
    \end{choices}
    \end{multicols}
\end{question}
}

\element{halliday-mc}{
\begin{question}{halliday-ch29-q07}
    Lines of the magnetic field produced by a long straight wire carrying a current are:
    \begin{choices}
        \wrongchoice{in the direction of the current}
        \wrongchoice{opposite to the direction of the current}
        \wrongchoice{radially outward from the wire}
        \wrongchoice{radially inward toward the wire}
      \correctchoice{circles that are concentric with the wire}
    \end{choices}
\end{question}
}

\element{halliday-mc}{
\begin{question}{halliday-ch29-q08}
    In an overhead straight wire, the current is north. 
    The magnetic field due to this current,
        at our point of observation, is:
    \begin{multicols}{3}
    \begin{choices}
        \wrongchoice{east}
        \wrongchoice{up}
        \wrongchoice{north}
        \wrongchoice{down}
      \correctchoice{west}
    \end{choices}
    \end{multicols}
\end{question}
}

\element{halliday-mc}{
\begin{question}{halliday-ch29-q09}
    A wire carrying a large current $i$ from east to west is placed over an ordinary magnetic compass.
    The end of the compass needle marked ``N'' will point:
    \begin{multicols}{2}
    \begin{choices}
        \wrongchoice{north}
      \correctchoice{south}
        \wrongchoice{east}
        \wrongchoice{west}
        \wrongchoice{the compass will act as an electric motor,
            hence the needle will keep rotating}
    \end{choices}
    \end{multicols}
\end{question}
}

\element{halliday-mc}{
\begin{question}{halliday-ch29-q10}
    The magnetic field outside a long straight current-carrying wire depends on the distance $R$ from the wire axis according to:
    \begin{multicols}{3}
    \begin{choices}
        \wrongchoice{$R$}
      \correctchoice{$\dfrac{1}{R}$}
        \wrongchoice{$\dfrac{1}{R^2}$}
        \wrongchoice{$\dfrac{1}{R^3}$}
        \wrongchoice{$\dfrac{1}{R^{3/2}}$}
    \end{choices}
    \end{multicols}
\end{question}
}

\element{halliday-mc}{
\begin{question}{halliday-ch29-q11}
    Which graph correctly gives the magnitude of the magnetic field outside an infinitely long straight current-carrying wire as a function of the distance $r$ from the wire?
    \begin{multicols}{2}
    \begin{choices}
        %% NOTE: ANS is D
        \wrongchoice{
            \begin{tikzpicture}
                %% NOTE: pgfplots
            \end{tikzpicture}
        }
    \end{choices}
    \end{multicols}
\end{question}
}

\element{halliday-mc}{
\begin{question}{halliday-ch29-q12}
    The magnetic field a distance \SI{2}{\centi\meter} from a long straight current-carrying wire is \SI{2.0e-5}{\tesla}.
    The current in the wire is:
    \begin{multicols}{2}
    \begin{choices}
        \wrongchoice{\SI{0.16}{\ampere}}
        \wrongchoice{\SI{1.0}{\ampere}}
      \correctchoice{\SI{2.0}{\ampere}}
        \wrongchoice{\SI{4.0}{\ampere}}
        \wrongchoice{\SI{25}{\ampere}}
    \end{choices}
    \end{multicols}
\end{question}
}

\element{halliday-mc}{
\begin{question}{halliday-ch29-q13}
    Two long parallel straight wires carry equal currents in opposite directions. 
    At a point midway between the wires,
        the magnetic field they produce is:
    \begin{choices}
        \wrongchoice{zero}
        \wrongchoice{non-zero and along a line connecting the wires}
        \wrongchoice{non-zero and parallel to the wires}
      \correctchoice{non-zero and perpendicular to the plane of the two wires}
        \wrongchoice{none of the provided}
    \end{choices}
\end{question}
}

\element{halliday-mc}{
\begin{question}{halliday-ch29-q14}
    Two long straight wires are parallel and carry current in the same direction. 
    The currents are \SI{8.0}{\ampere} and \SI{12}{\ampere} and the wires are separated by \SI{0.40}{\centi\meter}. 
    The magnetic field at a point midway between the wires is:
    \begin{multicols}{2}
    \begin{choices}
        \wrongchoice{zero}
      \correctchoice{\SI{4.0e-4}{\tesla}}
        \wrongchoice{\SI{8.0e-4}{\tesla}}
        \wrongchoice{\SI{12e-4}{\tesla}}
        \wrongchoice{\SI{20e-4}{\tesla}}
    \end{choices}
    \end{multicols}
\end{question}
}

\element{halliday-mc}{
\begin{question}{halliday-ch29-q15}
    Two long straight wires are parallel and carry current in opposite directions. 
    The currents are \SI{8.0}{\ampere} and \SI{12}{\ampere} and the wires are separated by \SI{0.40}{\centi\meter}. 
    The magnetic field at a point midway between the wires is:
    \begin{multicols}{2}
    \begin{choices}
        \wrongchoice{zero}
        \wrongchoice{\SI{4.0e-4}{\tesla}}
        \wrongchoice{\SI{8.0e-4}{\tesla}}
        %% normalized scientific notation
        \wrongchoice{\SI{1.2e-3}{\tesla}}
      \correctchoice{\SI{2.0e-3}{\tesla}}
    \end{choices}
    \end{multicols}
\end{question}
}

\element{halliday-mc}{
\begin{question}{halliday-ch29-q16}
    Two long straight current-carrying parallel wires cross the $x$ axis and carry currents $I$ and $3I$ in the same direction,
        as shown. 
    \begin{center}
    \begin{tikzpicture}
        %% NOTE: tikz
    \end{tikzpicture}
    \end{center}
    At what value of $x$ is the net magnetic field zero?
    \begin{multicols}{3}
    \begin{choices}
        \wrongchoice{\num{0}}
        \wrongchoice{\num{1}}
      \correctchoice{\num{3}}
        \wrongchoice{\num{5}}
        \wrongchoice{\num{7}}
    \end{choices}
    \end{multicols}
\end{question}
}

\element{halliday-mc}{
\begin{question}{halliday-ch29-q17}
    Two long straight wires pierce the plane of the paper at vertices of an equilateral triangle as shown below. 
    \begin{center}
    \begin{tikzpicture}
        %% NOTE: tikz
    \end{tikzpicture}
    \end{center}
    They each carry \SI{2}{\ampere}, out of the paper. 
    The magnetic field at the third vertex ($P$) has magnitude:
    \begin{multicols}{2}
    \begin{choices}
        \wrongchoice{\SI{1.0e-5}{\tesla}}
      \correctchoice{\SI{1.7e-5}{\tesla}}
        \wrongchoice{\SI{2.0e-5}{\tesla}}
        \wrongchoice{\SI{5.0e-6}{\tesla}}
        \wrongchoice{\SI{8.7e-6}{\tesla}}
    \end{choices}
    \end{multicols}
\end{question}
}

\element{halliday-mc}{
\begin{question}{halliday-ch29-q18}
    The diagram shows three equally spaced wires that are perpendicular to the page. 
    The currents are all equal,
        two being out of the page and one being into the page. 
    \begin{center}
    \begin{tikzpicture}
        %% NOTE: tikz
    \end{tikzpicture}
    \end{center}
    Rank the wires according to the magnitudes of the magnetic forces on them,
        from least to greatest.
    \begin{multicols}{2}
    \begin{choices}
        \wrongchoice{1, 2, 3}
      \correctchoice{2, 1 and 3 tie}
        \wrongchoice{2 and 3 tie, then 1}
        \wrongchoice{1 and 3 tie, then 2}
        \wrongchoice{3, 2, 1}
    \end{choices}
    \end{multicols}
\end{question}
}

\element{halliday-mc}{
\begin{question}{halliday-ch29-q19}
    Two parallel wires carrying equal currents of \SI{10}{\ampere} attract each other with a force of \SI{1}{\milli\newton}.
    If both currents are doubled,
        the force of attraction will be:
    \begin{multicols}{3}
    \begin{choices}
        \wrongchoice{\SI{1}{\milli\newton}}
      \correctchoice{\SI{4}{\milli\newton}}
        \wrongchoice{\SI{0.5}{\milli\newton}}
        \wrongchoice{\SI{0.25}{\milli\newton}}
        \wrongchoice{\SI{2}{\milli\newton}}
    \end{choices}
    \end{multicols}
\end{question}
}

\element{halliday-mc}{
\begin{question}{halliday-ch29-q20}
    Two parallel long wires carry the same current and repel each other with a force $F$ per unit length. 
    If both these currents are doubled and the wire separation tripled,
        the force per unit length becomes:
    \begin{multicols}{3}
    \begin{choices}
        \wrongchoice{$\dfrac{2F}{9}$}
        \wrongchoice{$\dfrac{4F}{9}$}
        \wrongchoice{$\dfrac{2F}{3}$}
      \correctchoice{$\dfrac{4F}{3}$}
        \wrongchoice{$6F$}
    \end{choices}
    \end{multicols}
\end{question}
}

\element{halliday-mc}{
\begin{question}{halliday-ch29-q21}
    Two parallel wires, \SI{4}{\centi\meter} apart,
        carry currents of \SI{2}{\ampere} and \SI{4}{\ampere} respectively,
        in the same direction.
    The force per unit length of one wire on the other is:
    \begin{choices}
        \wrongchoice{\SI{1e-3}{\newton\per\meter}, repulsive}
        \wrongchoice{\SI{1e-3}{\newton\per\meter}, attractive}
        \wrongchoice{\SI{4e-5}{\newton\per\meter}, repulsive}
      \correctchoice{\SI{4e-5}{\newton\per\meter}, attractive}
        \wrongchoice{none of the provided}
    \end{choices}
\end{question}
}

\element{halliday-mc}{
\begin{question}{halliday-ch29-q22}
    Two parallel wires, \SI{4}{\centi\meter} apart,
        carry currents of \SI{2}{\ampere} and \SI{4}{\ampere} respectively,
        in opposite directions.
    The force per unit length of one wire on the other is:
    \begin{choices}
        \wrongchoice{\SI{1e-3}{\newton\per\meter}, repulsive}
        \wrongchoice{\SI{1e-3}{\newton\per\meter}, attractive}
      \correctchoice{\SI{4e-5}{\newton\per\meter}, repulsive}
        \wrongchoice{\SI{4e-5}{\newton\per\meter}, attractive}
        \wrongchoice{none of the provided}
    \end{choices}
\end{question}
}

\element{halliday-mc}{
\begin{question}{halliday-ch29-q23}
    Four long straight wires carry equal currents into the page as shown. 
    \begin{center}
    \begin{tikzpicture}
        %% NOTE: tikz
    \end{tikzpicture}
    \end{center}
    The magnetic force exerted on wire $F$ is:
    \begin{multicols}{3}
    \begin{choices}
        \wrongchoice{north}
      \correctchoice{east}
        \wrongchoice{south}
        \wrongchoice{west}
        \wrongchoice{zero}
    \end{choices}
    \end{multicols}
\end{question}
}

\element{halliday-mc}{
\begin{question}{halliday-ch29-q24}
    A constant current is sent through a helical coil. 
    The coil:
    \begin{choices}
      \correctchoice{tends to get shorter}
        \wrongchoice{tends to get longer}
        \wrongchoice{tends to rotate about its axis}
        \wrongchoice{produces zero magnetic field at its center}
        \wrongchoice{none of the provided}
    \end{choices}
\end{question}
}

\element{halliday-mc}{
\begin{question}{halliday-ch29-q25}
    The diagram shows three arrangements of circular loops,
        centered on vertical axes and carrying identical currents in the directions indicated.
    \begin{center}
    \begin{tikzpicture}
        %% NOTE: tikz
    \end{tikzpicture}
    \end{center}
    Rank the arrangements according to the magnitudes of the magnetic fields at the midpoints between the loops on the central axes.
    \begin{multicols}{2}
    \begin{choices}
        \wrongchoice{1, 2, 3}
        \wrongchoice{2, 1, 3}
      \correctchoice{2, 3, 1}
        \wrongchoice{3, 2, 1}
        \wrongchoice{3, 1, 2}
    \end{choices}
    \end{multicols}
\end{question}
}

\element{halliday-mc}{
\begin{question}{halliday-ch29-q26}
    Helmholtz coils are commonly used in the laboratory because the magnetic field between them:
    \begin{choices}
        \wrongchoice{can be varied more easily than the fields of other current arrangements}
        \wrongchoice{is especially strong}
        \wrongchoice{nearly cancels Earth's magnetic field}
        \wrongchoice{is parallel to the plane of the coils}
      \correctchoice{is nearly uniform}
    \end{choices}
\end{question}
}

\element{halliday-mc}{
\begin{question}{halliday-ch29-q27}
    If the radius of a pair of Helmholtz coils is $R$ then the distance between the coils is:
    \begin{multicols}{3}
    \begin{choices}
        \wrongchoice{$\dfrac{R}{4}$}
        \wrongchoice{$\dfrac{R}{2}$}
      \correctchoice{$R$}
        \wrongchoice{$2R$}
        \wrongchoice{$4R$}
    \end{choices}
    \end{multicols}
\end{question}
}

\element{halliday-mc}{
\begin{question}{halliday-ch29-q28}
    If $R$ is the distance from a magnetic dipole,
        then the magnetic field it produces is proportional to:
    \begin{multicols}{3}
    \begin{choices}
        \wrongchoice{$R$}
        \wrongchoice{$\dfrac{1}{R}$}
        \wrongchoice{$R^2$}
        \wrongchoice{$\dfrac{1}{R^2}$}
      \correctchoice{$\dfrac{1}{R^3}$}
    \end{choices}
    \end{multicols}
\end{question}
}

\element{halliday-mc}{
\begin{question}{halliday-ch29-q29}
    A square loop of current-carrying wire with edge length $a$ is in the $xy$ plane,
        the origin being at its center. 
    Along which of the following lines can a charge move without experiencing a magnetic force?
    \begin{choices}
        \wrongchoice{$x=\text{zero}$,   $y=\dfrac{a}{2}$}
        \wrongchoice{$x=\dfrac{a}{2}$,  $y=\dfrac{a}{2}$}
        \wrongchoice{$x=\dfrac{a}{2}$,  $y=\text{zero}$}
        \wrongchoice{$x=\text{zero}$,   $y=\text{zero}$}
        \wrongchoice{$x=\text{zero}$,   $z=\text{zero}$}
    \end{choices}
\end{question}
}

\element{halliday-mc}{
\begin{question}{halliday-ch29-q30}
    In Ampere's law, $\oint \vec{B}\cdot\mathrm{d}\vec{s} = \mu_0 i$,
        the integration must be over any:
    \begin{choices}
        \wrongchoice{surface}
        \wrongchoice{closed surface}
        \wrongchoice{path}
      \correctchoice{closed path}
        \wrongchoice{closed path that surrounds all the current producing $\vec{B}$}
    \end{choices}
\end{question}
}

\element{halliday-mc}{
\begin{question}{halliday-ch29-q31}
    In Ampere's law, $\oint \vec{B}\cdot\mathrm{d}\vec{s} = \mu_0 i$,
        the symbol $\mathrm{d}\vec{s}$ is:
    \begin{choices}
        \wrongchoice{an infinitesimal piece of the wire that carries current $i$}
        \wrongchoice{in the direction of $\vec{B}$}
        \wrongchoice{perpendicular to $\vec{B}$}
        \wrongchoice{a vector whose magnitude is the length of the wire that carries current $i$}
      \correctchoice{none of the provided}
    \end{choices}
\end{question}
}

\element{halliday-mc}{
\begin{question}{halliday-ch29-q32}
    In Ampere's law, $\oint \vec{B}\cdot\mathrm{d}\vec{s} = \mu_0 i$,
        the direction of the integration around the path:
    \begin{choices}
        \wrongchoice{must be clockwise}
        \wrongchoice{must be counterclockwise}
        \wrongchoice{must be such as to follow the magnetic field lines}
        \wrongchoice{must be along the wire in the direction of the current}
        \wrongchoice{none of the provided}
    \end{choices}
\end{question}
}

\element{halliday-mc}{
\begin{question}{halliday-ch29-q33}
    A long straight wire carrying a \SI{3.0}{\ampere} current enters a room through a window \SI{1.5}{\meter} high and \SI{1.0}{\meter} wide. 
    The path integral $\oint \vec{B}\cdot\mathrm{d}\vec{s}$ around the window frame has the value:
    \begin{multicols}{2}
    \begin{choices}
        \wrongchoice{\SI{0.20}{\tesla\meter}}
        \wrongchoice{\SI{2.5e-7}{\tesla\meter}}
        \wrongchoice{\SI{3.0e-7}{\tesla\meter}}
      \correctchoice{\SI{3.8e-6}{\tesla\meter}}
        \wrongchoice{none of the provided}
    \end{choices}
    \end{multicols}
\end{question}
}

\element{halliday-mc}{
\begin{question}{halliday-ch29-q34}
    Two long straight wires enter a room through a door. 
    One carries a current of \SI{3.0}{\ampere} into the room while the other carries a current of \SI{5.0}{\ampere} out. 
    The magnitude of the path integral $\oint \vec{B}\cdot\mathrm{d}\vec{s}$ around the door frame is:
    \begin{multicols}{2}
    \begin{choices}
      \correctchoice{\SI{2.5e-6}{\tesla\meter}}
        \wrongchoice{\SI{3.8e-6}{\tesla\meter}}
        \wrongchoice{\SI{6.3e-6}{\tesla\meter}}
        \wrongchoice{\SI{1.0e-5}{\tesla\meter}}
        \wrongchoice{none of the provided}
    \end{choices}
    \end{multicols}
\end{question}
}

\element{halliday-mc}{
\begin{question}{halliday-ch29-q35}
    If the magnetic field $\vec{B}$ is uniform over the area bounded by a circle with radius $R$,
        the net current through the circle is:
    \begin{multicols}{3}
    \begin{choices}
      \correctchoice{zero}
        \wrongchoice{$\dfrac{2\pi RB}{\mu_0}$}
        \wrongchoice{$\dfrac{\pi R^2 B}{\mu_0}$}
        \wrongchoice{$\dfrac{RB}{2\mu_0}$}
        \wrongchoice{$\dfrac{2RB}{\mu_0}$}
    \end{choices}
    \end{multicols}
\end{question}
}

\element{halliday-mc}{
\begin{question}{halliday-ch29-q36}
    The magnetic field at any point is given by $\vec{B} = A\vec{r}\times\vec{k}$,
        where $\vec{r}$ is the position vector of the point and $A$ is a constant. 
    The net current through a circle of radius $R$,
        in the $xy$ plane and centered at the origin is given by:
    \begin{multicols}{3}
    \begin{choices}
        \wrongchoice{$\dfrac{\pi AR^2}{\mu_0}$}
        \wrongchoice{$\dfrac{2\pi AR}{\mu_0}$}
        \wrongchoice{$\dfrac{4\pi AR^3}{3\mu_0}$}
      \correctchoice{$\dfrac{2\pi AR^2}{\mu_0}$}
        \wrongchoice{$\dfrac{\pi AR^2}{2\mu_0}$}
    \end{choices}
    \end{multicols}
\end{question}
}

\element{halliday-mc}{
\begin{question}{halliday-ch29-q37}
    A hollow cylindrical conductor (inner radius = $a$, outer radius = $b$)
        carries a current $i$ uniformly spread over its cross section. 
    Which graph below correctly gives $B$ as a function of the distance $r$ from the center of the cylinder?
    \begin{multicols}{2}
    \begin{choices}
        %% NOTE: ANS is C
        \wrongchoice{
            \begin{tikzpicture}
                %% NOTE: pgfplots
            \end{tikzpicture}
        }
    \end{choices}
    \end{multicols}
\end{question}
}

\element{halliday-mc}{
\begin{question}{halliday-ch29-q38}
    A long straight cylindrical shell carries current $i$ parallel to its axis and uniformly distributed over its cross section. 
    The magnitude of the magnetic field is greatest:
    \begin{choices}
        \wrongchoice{at the inner surface of the shell}
      \correctchoice{at the outer surface of the shell}
        \wrongchoice{inside the shell near the middle}
        \wrongchoice{in hollow region near the inner surface of the shell}
        \wrongchoice{near the center of the hollow region}
    \end{choices}
\end{question}
}

\element{halliday-mc}{
\begin{question}{halliday-ch29-q39}
    A long straight cylindrical shell has inner radius $R_i$ and outer radius $R_o$.
    It carries current $i$, uniformly distributed over its cross section. 
    A wire is parallel to the cylinder axis,
        in the hollow region ($r<R_i$). 
    The magnetic field is zero everywhere outside the shell ($r>R_o$). 
    We conclude that the wire:
    \begin{choices}
        \wrongchoice{is on the cylinder axis and carries current $i$ in the same direction as the current in the shell}
        \wrongchoice{may be anywhere in the hollow region but must be carrying current $i$ in the direction opposite to that of the current in the shell}
        \wrongchoice{may be anywhere in the hollow region but must be carrying current $i$ in the same direction as the current in the shell}
      \correctchoice{is on the cylinder axis and carries current $i$ in the direction opposite to that of the current in the shell}
        \wrongchoice{does not carry any current}
    \end{choices}
\end{question}
}

\element{halliday-mc}{
\begin{question}{halliday-ch29-q40}
    A long straight cylindrical shell has inner radius $R_i$ and outer radius $R_o$.
    It carries a current $i$,
        uniformly distributed over its cross section. 
    A wire is parallel to the cylinder axis,
        in the hollow region ($r<R_i$). 
    The magnetic field is zero everywhere in the hollow region. 
    We conclude that the wire:
    \begin{choices}
        \wrongchoice{is on the cylinder axis and carries current i in the same direction as the current in the shell}
        \wrongchoice{may be anywhere in the hollow region but must be carrying current i in the direction opposite to that of the current in the shell}
        \wrongchoice{may be anywhere in the hollow region but must be carrying current i in the same direction as the current in the shell}
        \wrongchoice{is on the cylinder axis and carries current i in the direction opposite to that of the current in the shell}
      \correctchoice{does not carry any current}
    \end{choices}
\end{question}
}

\element{halliday-mc}{
\begin{question}{halliday-ch29-q41}
    The magnetic field B inside a long ideal solenoid is independent of:
    \begin{choices}
        \wrongchoice{the current}
        \wrongchoice{the core material}
        \wrongchoice{the spacing of the windings}
      \correctchoice{the cross-sectional area of the solenoid}
        \wrongchoice{the direction of the current}
    \end{choices}
\end{question}
}

\element{halliday-mc}{
\begin{question}{halliday-ch29-q42}
    Two long ideal solenoids 
        (with radii \SI{20}{\milli\meter} and \SI{30}{\milli\meter}, respectively)
        have the same number of turns of wire per unit length. 
    The smaller solenoid is mounted inside the larger,
        along a common axis. 
    The magnetic field within the inner solenoid is zero. 
    The current in the inner solenoid must be:
    \begin{choices}
        \wrongchoice{two-thirds the current in the outer solenoid}
        \wrongchoice{one-third the current in the outer solenoid}
        \wrongchoice{twice the current in the outer solenoid}
        \wrongchoice{half the current in the outer solenoid}
      \correctchoice{the same as the current in the outer solenoid}
    \end{choices}
\end{question}
}

\element{halliday-mc}{
\begin{question}{halliday-ch29-q43}
    Magnetic field lines inside the solenoid shown are:
    \begin{center}
    \begin{tikzpicture}
        %% NOTE: tikz
    \end{tikzpicture}
    \end{center}
    \begin{choices}
        \wrongchoice{clockwise circles as one looks down the axis from the top of the page}
        \wrongchoice{counterclockwise circles as one looks down the axis from the top of the page}
      \correctchoice{toward the top of the page}
        \wrongchoice{toward the bottom of the page}
        \wrongchoice{in no direction since $B=0$}
    \end{choices}
\end{question}
}

\element{halliday-mc}{
\begin{question}{halliday-ch29-q44}
    Solenoid 2 has twice the radius and six times the number of turns per unit length as solenoid 1.
    The ratio of the magnetic field in the interior of 2 to that in the interior of 1 is:
    \begin{multicols}{3}
    \begin{choices}
        \wrongchoice{\num{2}}
        \wrongchoice{\num{4}}
      \correctchoice{\num{6}}
        \wrongchoice{\num{1}}
        \wrongchoice{\num{1/3}}
    \end{choices}
    \end{multicols}
\end{question}
}

\element{halliday-mc}{
\begin{question}{halliday-ch29-q45}
    A solenoid is \SI{3.0}{\centi\meter} long and has a radius of \SI{0.50}{\centi\meter}.
    It is wrapped with \num{500} turns of wire carrying a current of \SI{2.0}{\ampere}.
    The magnetic field at the center of the solenoid is:
    \begin{multicols}{2}
    \begin{choices}
        \wrongchoice{\SI{9.9e-8}{\tesla}}
        \wrongchoice{\SI{1.3e-3}{\tesla}}
      \correctchoice{\SI{4.2e-2}{\tesla}}
        \wrongchoice{\SI{16}{\tesla}}
        \wrongchoice{\SI{20}{\tesla}}
    \end{choices}
    \end{multicols}
\end{question}
}

\element{halliday-mc}{
\begin{question}{halliday-ch29-q46}
    A toroid with a square cross section carries current $i$. 
    The magnetic field has its largest magnitude:
    \begin{choices}
        \wrongchoice{at the center of the hole}
      \correctchoice{just inside the toroid at its inner surface}
        \wrongchoice{just inside the toroid at its outer surface}
        \wrongchoice{at any point inside (the field is uniform)}
        \wrongchoice{none of the above}
    \end{choices}
\end{question}
}

\element{halliday-mc}{
\begin{question}{halliday-ch29-q47}
    A toroid has a square cross section with the length of an edge equal to the radius of the inner surface. 
    The ratio of the magnitude of the magnetic field at the inner surface to the magnitude of the field at the outer surface is:
    \begin{multicols}{3}
    \begin{choices}
        \wrongchoice{\num{1/4}}
        \wrongchoice{\num{1/2}}
        \wrongchoice{\num{1}}
      \correctchoice{\num{2}}
        \wrongchoice{\num{4}}
    \end{choices}
    \end{multicols}
\end{question}
}


\endinput


