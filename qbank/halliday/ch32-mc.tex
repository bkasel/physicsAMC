
%%--------------------------------------------------
%% Halliday: Fundamentals of Physics
%%--------------------------------------------------


%% Chapter 32: Maxwell's Equations;
%%              Magnetism and Matter
%%--------------------------------------------------


%% Learning Objectives
%%--------------------------------------------------

%% 32.01: Identify that the simplest magnetic structure is a magnetic dipole.
%% 32.02: Calculate the magnetic flux  through a surface by integrating the dot product of the magnetic field vector $\vec{B}$ and the area vector dA (for patch elements) over the surface.
%% 32.03: Identify that the net magnetic flux through a Gaussian surface (which is a closed surface) is zero.


%% Halliday Multiple Choice Questions
%%--------------------------------------------------
\element{halliday-mc}{
\begin{question}{halliday-ch32-q01}
    Gauss' law for magnetism:
    \begin{choices}
        \wrongchoice{can be used to find $\vec{B}$ due to given currents provided there is enough symmetry}
        \wrongchoice{is false because there are no magnetic poles}
        \wrongchoice{can be used with open surfaces because there are no magnetic poles}
        \wrongchoice{contradicts Faraday's law because one says $\Phi_B=0$ and the other says $\varepsilon=-\dfrac{\mathrm{d}\Phi_B}{\mathrm{d}t}$}
      \correctchoice{none of the provided}
    \end{choices}
\end{question}
}

\element{halliday-mc}{
\begin{question}{halliday-ch32-q02}
    Gauss' law for magnetism tells us:
    \begin{choices}
        \wrongchoice{the net charge in any given volume}
        \wrongchoice{that the line integral of a magnetic field around any closed loop must vanish}
        \wrongchoice{the magnetic field of a current element}
      \correctchoice{that magnetic monopoles do not exist}
        \wrongchoice{charges must be moving to produce magnetic fields}
    \end{choices}
\end{question}
}

\element{halliday-mc}{
\begin{question}{halliday-ch32-q03}
    The statement that magnetic field lines form closed loops is a direct consequence of:
    \begin{choices}
        \wrongchoice{Faraday's law}
        \wrongchoice{Ampere's law}
        \wrongchoice{Gauss' law for electricity}
      \correctchoice{Gauss' law for magnetism}
        \wrongchoice{the Lorentz force}
    \end{choices}
\end{question}
}

\element{halliday-mc}{
\begin{question}{halliday-ch32-q04}
    A magnetic field parallel to the $x$ axis with a magnitude that decreases with increasing $x$ but does not change with $y$ and $z$ is impossible according to:
    \begin{choices}
        \wrongchoice{Faraday's law}
        \wrongchoice{Ampere's law}
        \wrongchoice{Gauss' law for electricity}
      \correctchoice{Gauss' law for magnetism}
        \wrongchoice{Newton's second law}
    \end{choices}
\end{question}
}

\element{halliday-mc}{
\begin{question}{halliday-ch32-q05}
    According to Gauss' law for magnetism, magnetic field lines:
    \begin{choices}
      \correctchoice{form closed loops}
        \wrongchoice{start at south poles and end at north poles}
        \wrongchoice{start at north poles and end at south poles}
        \wrongchoice{start at both north and south poles and end at infinity}
        \wrongchoice{do not exist}
    \end{choices}
\end{question}
}

\element{halliday-mc}{
\begin{question}{halliday-ch32-q06}
    The magnetic field lines due to an ordinary bar magnet:
    \begin{choices}
        \wrongchoice{form closed curves}
        \wrongchoice{cross one another near the poles}
        \wrongchoice{are more numerous near the N pole than near the S pole}
        \wrongchoice{do not exist inside the magnet}
        \correctchoice{none of the provided}
    \end{choices}
\end{question}
}

\element{halliday-mc}{
\begin{question}{halliday-ch32-q07}
    Four closed surfaces are shown. 
    The areas $A_{top}$ and $A_{bot}$ of the top and bottom faces and the magnitudes $B_{top}$ and $B_{bot}$ of the uniform magnetic fields through the top and bottom faces are given.
    \begin{center}
    \begin{tikzpicture}
        %% NOTE:
    \end{tikzpicture}
    \end{center}
    The fields are perpendicular to the faces and are either inward or outward. 
    Rank the surfaces according to the magnitude of the magnetic flux through the curved sides,
        least to greatest.
    \begin{multicols}{2}
    \begin{choices}
        \wrongchoice{1, 2, 3, 4}
      \correctchoice{3, 4, 1, 2}
        \wrongchoice{1, 2, 4, 3}
        \wrongchoice{4, 3, 2, 1}
        \wrongchoice{2, 1, 4, 3}
    \end{choices}
    \end{multicols}
\end{question}
}

\element{halliday-mc}{
\begin{question}{halliday-ch32-q08}
    Consider the four Maxwell equations:
    \begin{enumerate}
        \item $\oint\,\vec{E}\cdot\mathrm{d}\vec{A} = \dfrac{q}{\epsilon_0}$
        \item $\oint\,\vec{B}\cdot\mathrm{d}\vec{A} = 0$
        \item $\oint\,\vec{E}\cdot\mathrm{d}\vec{s} = -\dfrac{\mathrm{d}\Phi_B}{\mathrm{d}t}$
        \item $\oint\,\vec{B}\cdot\mathrm{d}\vec{s} = \mu_0 i + \mu_0 \epsilon_0 \dfrac{\mathrm{d}\Phi_E}{\mathrm{d}t}$
    \end{enumerate}
    Which of these must be modified if magnetic poles are discovered?
    \begin{multicols}{2}
    \begin{choices}
        %% NOTE: questionmult?
        \wrongchoice{Only 1}
        \wrongchoice{Only 2}
      \correctchoice{Only 2 and 3}
        \wrongchoice{Only 3 and 4}
        \wrongchoice{Only 2, 3, and 4}
    \end{choices}
    \end{multicols}
\end{question}
}

\element{halliday-mc}{
\begin{question}{halliday-ch32-q09}
    One of the Maxwell equations begins with $\oint\,\vec{B}\cdot\mathrm{d}\vec{s} = \ldots$
    The symbol ``$\mathrm{d}\vec{s}$'' means:
    \begin{choices}
        \wrongchoice{an infinitesimal displacement of a charge}
        \wrongchoice{an infinitesimal displacement of a magnetic pole}
        \wrongchoice{an infinitesimal inductance}
        \wrongchoice{an infinitesimal surface area}
      \correctchoice{none of the provided}
    \end{choices}
\end{question}
}

\element{halliday-mc}{
\begin{question}{halliday-ch32-q10}
    One of the Maxwell equations begins with $\oint\,\vec{E}\cdot\mathrm{d}\vec{s} = \ldots$.
    The ``\tikz \draw (0,0) circle (2pt);'' symbol in the integral sign means:
    \begin{choices}
        \wrongchoice{the same as the subscript in $\mu_0$}
        \wrongchoice{integrate clockwise around the path}
        \wrongchoice{integrate counterclockwise around the path}
      \correctchoice{integrate around a closed path}
        \wrongchoice{integrate over a closed surface}
    \end{choices}
\end{question}
}

\element{halliday-mc}{
\begin{question}{halliday-ch32-q11}
    One of the Maxwell equations begins with $\oint\,\vec{B}\cdot\mathrm{d}\vec{A} = \ldots$.
    The ``\tikz \draw (0,0) circle (2pt);'' symbol in the integral sign means:
    \begin{choices}
        \wrongchoice{the same as the subscript in $\mu_0$}
        \wrongchoice{integrate clockwise around the path}
        \wrongchoice{integrate counterclockwise around the path}
        \wrongchoice{integrate around a closed path}
      \correctchoice{integrate over a closed surface}
    \end{choices}
\end{question}
}

\element{halliday-mc}{
\begin{question}{halliday-ch32-q12}
    One of the crucial facts upon which the Maxwell equations are based is:
    \begin{choices}
        \wrongchoice{the numerical value of the electron charge}
        \wrongchoice{charge is quantized}
        \wrongchoice{the numerical value of the charge/mass ratio of the electron}
        \wrongchoice{there are three types of magnetic materials}
      \correctchoice{none of the provided}
    \end{choices}
\end{question}
}

\element{halliday-mc}{
\begin{question}{halliday-ch32-q13}
    Two of Maxwell's equations contain a path integral on the left side and an area integral on the right. 
    For them:
    \begin{choices}
        \wrongchoice{the path must pierce the area}
        \wrongchoice{the path must be well-separated from the area}
        \wrongchoice{the path must be along a field line and the area must be perpendicular to the field line}
      \correctchoice{the path must be the boundary of the area}
        \wrongchoice{the path must lie in the area, away from its boundary}
    \end{choices}
\end{question}
}

\element{halliday-mc}{
\begin{question}{halliday-ch32-q14}
    Two of Maxwell's equations contain an integral over a closed surface. 
    For them the infinitesimal vector area $\mathrm{d}\vec{A}$ is always:
    \begin{choices}
        \wrongchoice{tangent to the surface}
      \correctchoice{perpendicular to the surface and pointing outward}
        \wrongchoice{perpendicular to the surface and pointing inward}
        \wrongchoice{tangent to a field line}
        \wrongchoice{perpendicular to a field line}
    \end{choices}
\end{question}
}

\element{halliday-mc}{
\begin{question}{halliday-ch32-q15}
    Two of Maxwell's equations contain a path integral on the left side and an area integral on the right.
    The directions of the infinitesimal path element ds and infinitesimal area element $\mathrm{d}\vec{A}$ are:
    \begin{choices}
        \wrongchoice{always in the same direction}
        \wrongchoice{always in opposite directions}
        \wrongchoice{always perpendicular to each other}
        \wrongchoice{never perpendicular to each other}
      \correctchoice{none of the provided}
    \end{choices}
\end{question}
}

\element{halliday-mc}{
\begin{question}{halliday-ch32-q16}
    Two of Maxwell's equations contain a path integral on the left side and an area integral on the right.
    Suppose the area is the surface of a piece of paper at which you are looking and $\mathrm{d}\vec{A}$ is chosen to point toward you.
    Then, the path integral is:
    \begin{choices}
        \wrongchoice{clockwise around the circumference of the paper}
      \correctchoice{counterclockwise around the circumference of the paper}
        \wrongchoice{from left to right}
        \wrongchoice{from right to left}
        \wrongchoice{from top to bottom}
    \end{choices}
\end{question}
}

\element{halliday-mc}{
\begin{question}{halliday-ch32-q17}
    Which of the following equations can be used,
        along with a symmetry argument,
        to calculate the electric field of a point charge?
    \begin{choices}
      \correctchoice{$\oint\,\vec{E}\cdot\mathrm{d}\vec{A} = \dfrac{q}{\epsilon_0}$}
        \wrongchoice{$\oint\,\vec{B}\cdot\mathrm{d}\vec{A} = 0$}
        \wrongchoice{$\oint\,\vec{E}\cdot\mathrm{d}\vec{s} = -\dfrac{\mathrm{d}\Phi_B}{\mathrm{d}t}$}
        \wrongchoice{$\oint\,\vec{B}\cdot\mathrm{d}\vec{s} = \mu_0 i + \mu_0 \epsilon_0\dfrac{\mathrm{d}\Phi_E}{\mathrm{d}t}$}
        \wrongchoice{None of the provided}
    \end{choices}
\end{question}
}

\element{halliday-mc}{
\begin{question}{halliday-ch32-q18}
    Which of the following equations can be used, along with a symmetry argument,
        to calculate the magnetic field of a long straight wire carrying current?
    \begin{choices}
        \wrongchoice{$\oint\,\vec{E}\cdot\mathrm{d}\vec{A} = \dfrac{q}{\epsilon_0}$}
        \wrongchoice{$\oint\,\vec{B}\cdot\mathrm{d}\vec{A} = 0$}
        \wrongchoice{$\oint\,\vec{E}\cdot\mathrm{d}\vec{s} = -\dfrac{\mathrm{d}\Phi_B}{\mathrm{d}t}$}
      \correctchoice{$\oint\,\vec{B}\cdot\mathrm{d}\vec{s} = \mu_0 i + \mu_0 \epsilon_0\dfrac{\mathrm{d}\Phi_E}{\mathrm{d}t}$}
        \wrongchoice{None of the provided}
    \end{choices}
\end{question}
}

\element{halliday-mc}{
\begin{question}{halliday-ch32-q19}
    Which of the following equations can be used to show that magnetic field lines form closed loops?
    \begin{choices}
        \wrongchoice{$\oint\,\vec{E}\cdot\mathrm{d}\vec{A} = \dfrac{q}{\epsilon_0}$}
      \correctchoice{$\oint\,\vec{B}\cdot\mathrm{d}\vec{A} = 0$}
        \wrongchoice{$\oint\,\vec{E}\cdot\mathrm{d}\vec{s} = -\dfrac{\mathrm{d}\Phi_B}{\mathrm{d}t}$}
        \wrongchoice{$\oint\,\vec{B}\cdot\mathrm{d}\vec{s} = \mu_0 i + \mu_0 \epsilon_0\dfrac{\mathrm{d}\Phi_E}{\mathrm{d}t}$}
        \wrongchoice{None of the provided}
    \end{choices}
\end{question}
}

\element{halliday-mc}{
\begin{question}{halliday-ch32-q20}
    Which of the following equations, along with a symmetry argument,
        can be used to calculate the magnetic field produced by a uniform time-varying electric field?
    \begin{choices}
        \wrongchoice{$\oint\,\vec{E}\cdot\mathrm{d}\vec{A} = \dfrac{q}{\epsilon_0}$}
        \wrongchoice{$\oint\,\vec{B}\cdot\mathrm{d}\vec{A} = 0$}
        \wrongchoice{$\oint\,\vec{E}\cdot\mathrm{d}\vec{s} = -\dfrac{\mathrm{d}\Phi_B}{\mathrm{d}t}$}
      \correctchoice{$\oint\,\vec{B}\cdot\mathrm{d}\vec{s} = \mu_0 i + \mu_0 \epsilon_0\dfrac{\mathrm{d}\Phi_E}{\mathrm{d}t}$}
        \wrongchoice{None of the provided}
    \end{choices}
\end{question}
}

\element{halliday-mc}{
\begin{question}{halliday-ch32-q21}
    Which of the following equations, along with a symmetry argument,
        can be used to calculate the electric field produced by a uniform time-varying magnetic field?
    \begin{choices}
        \wrongchoice{$\oint\,\vec{E}\cdot\mathrm{d}\vec{A} = \dfrac{q}{\epsilon_0}$}
        \wrongchoice{$\oint\,\vec{B}\cdot\mathrm{d}\vec{A} = 0$}
      \correctchoice{$\oint\,\vec{E}\cdot\mathrm{d}\vec{s} = -\dfrac{\mathrm{d}\Phi_B}{\mathrm{d}t}$}
        \wrongchoice{$\oint\,\vec{B}\cdot\mathrm{d}\vec{s} = \mu_0 i + \mu_0 \epsilon_0\dfrac{\mathrm{d}\Phi_E}{\mathrm{d}t}$}
        \wrongchoice{None of the provided}
    \end{choices}
\end{question}
}

\element{halliday-mc}{
\begin{question}{halliday-ch32-q22}
    Which of the following equations, along with a symmetry argument,
        can be used to calculate the magnetic field between the plates of a charging parallel plate capacitor with circular plates?
    \begin{choices}
        \wrongchoice{$\oint\,\vec{E}\cdot\mathrm{d}\vec{A} = \dfrac{q}{\epsilon_0}$}
        \wrongchoice{$\oint\,\vec{B}\cdot\mathrm{d}\vec{A} = 0$}
        \wrongchoice{$\oint\,\vec{E}\cdot\mathrm{d}\vec{s} = -\dfrac{\mathrm{d}\Phi_B}{\mathrm{d}t}$}
      \correctchoice{$\oint\,\vec{B}\cdot\mathrm{d}\vec{s} = \mu_0 i + \mu_0 \epsilon_0\dfrac{\mathrm{d}\Phi_E}{\mathrm{d}t}$}
        \wrongchoice{None of the provided}
    \end{choices}
\end{question}
}

\element{halliday-mc}{
\begin{question}{halliday-ch32-q23}
    Maxwell's equations, along with an appropriate symmetry argument,
        can be used to calculate:
    \begin{choices}
        \wrongchoice{the electric force on a given charge}
        \wrongchoice{the magnetic force on a given moving charge}
        \wrongchoice{the flux of a given electric field}
        \wrongchoice{the flux of a given magnetic field}
      \correctchoice{none of the provided}
    \end{choices}
\end{question}
}

\element{halliday-mc}{
\begin{question}{halliday-ch32-q24}
    The polarity of an unmarked magnet can be determined using:
    \begin{choices}
        \wrongchoice{a charged glass rod}
      \correctchoice{a compass}
        \wrongchoice{an electroscope}
        \wrongchoice{another unmarked magnet}
        \wrongchoice{iron filings}
    \end{choices}
\end{question}
}

\element{halliday-mc}{
\begin{question}{halliday-ch32-q25}
    A bar magnet is placed vertically with its S pole up and its N pole down. 
    Its $\vec{B}$ field at its center is:
    \begin{choices}
        \wrongchoice{zero}
      \correctchoice{down}
        \wrongchoice{up due to the weight of the magnet}
        \wrongchoice{horizontal}
        \wrongchoice{slightly below the horizontal}
    \end{choices}
\end{question}
}

\element{halliday-mc}{
\begin{question}{halliday-ch32-q26}
    A bar magnet is broken in half.
    Each half is broken in half again, etc. 
    The observation is that each piece has both a north and south pole. 
    This is usually explained by:
    \begin{choices}
      \correctchoice{Ampere's theory that all magnetic phenomena result from electric currents}
        \wrongchoice{our inability to divide the magnet into small enough pieces}
        \wrongchoice{Coulomb's law}
        \wrongchoice{Lenz' law}
        \wrongchoice{conservation of charge.}
    \end{choices}
\end{question}
}

\element{halliday-mc}{
\begin{question}{halliday-ch32-q27}
    A small bar magnet is suspended horizontally by a string. 
    When placed in a uniform horizontal magnetic field, it will:
    \begin{choices}
        \wrongchoice{translate in the direction of $\vec{B}$}
        \wrongchoice{translate in the opposite direction of $\vec{B}$}
        \wrongchoice{rotate so as to be at right angles to $\vec{B}$}
        \wrongchoice{rotate so as to be vertical}
      \correctchoice{none of the provided}
    \end{choices}
\end{question}
}

\element{halliday-mc}{
\begin{question}{halliday-ch32-q28}
    Magnetic dipole $X$ is fixed and dipole $Y$ is free to move. 
    \begin{center}
    \begin{tikzpicture}
        %% NOTE: tikz
    \end{tikzpicture}
    \end{center}
    Dipole $Y$ will initially:
    \begin{choices}
      \correctchoice{move toward $X$ but not rotate}
        \wrongchoice{move away from $X$ but not rotate}
        \wrongchoice{move toward $X$ and rotate}
        \wrongchoice{move away from $X$ and rotate}
        \wrongchoice{rotate but not translate}
    \end{choices}
\end{question}
}

\element{halliday-mc}{
\begin{question}{halliday-ch32-q29}
    Magnetic dipole $X$ is fixed and dipole $Y$ is free to move. 
    \begin{center}
    \begin{tikzpicture}
        %% NOTE: tikz
    \end{tikzpicture}
    \end{center}
    Dipole $Y$ will initially:
    \begin{choices}
        \wrongchoice{move toward $X$ but not rotate}
        \wrongchoice{move away from $X$ but not rotate}
        \wrongchoice{move toward $X$ and rotate}
        \wrongchoice{move away from $X$ and rotate}
      \correctchoice{rotate but not move toward or away from $X$}
    \end{choices}
\end{question}
}

\element{halliday-mc}{
\begin{question}{halliday-ch32-q30}
    The diagram shows the angular momentum vectors of two electrons and two protons in the same external magnetic field. 
    The field points upward in the diagram. 
    \begin{center}
    \begin{tikzpicture}
        %% NOTE: tikz
    \end{tikzpicture}
    \end{center}
    Rank the situations according to the potential energy,
        least to greatest.
    \begin{choices}
        \wrongchoice{1 and 3 tie, then 2 and 4 tie}
      \correctchoice{2 and 3 tie, then 1 and 4 tie}
        \wrongchoice{1 and 2 tie, then 3 and 4 tie}
        \wrongchoice{3 and 4 tie, then 1 and 2 tie}
        \wrongchoice{all tie}
    \end{choices}
\end{question}
}

\element{halliday-mc}{
\begin{question}{halliday-ch32-q31}
    The energy of a magnetic dipole in an external magnetic field is least when:
    \begin{choices}
      \correctchoice{the dipole moment is parallel to the field}
        \wrongchoice{the dipole moment is antiparallel to the field}
        \wrongchoice{the dipole moment is perpendicular to the field}
        \wrongchoice{none of the provided (the same energy is associated with all orientations)}
        \wrongchoice{none of the proivded (no energy is associated with the dipole-field interaction)}
    \end{choices}
\end{question}
}

\element{halliday-mc}{
\begin{question}{halliday-ch32-q32}
    The magnetic properties of materials stem chiefly from:
    \begin{choices}
        \wrongchoice{particles with north poles}
        \wrongchoice{particles with south poles}
        \wrongchoice{motions of protons within nuclei}
        \wrongchoice{proton spin angular momentum}
      \correctchoice{electron magnetic dipole moments}
    \end{choices}
\end{question}
}

\element{halliday-mc}{
\begin{question}{halliday-ch32-q33}
    Magnetization is:
    \begin{choices}
        \wrongchoice{the current density in an object}
        \wrongchoice{the charge density of moving charges in an object}
        \wrongchoice{the magnetic dipole moment of an object}
      \correctchoice{the magnetic dipole moment per unit volume of an object}
        \wrongchoice{the magnetic field per unit volume produced by an object}
    \end{choices}
\end{question}
}

\element{halliday-mc}{
\begin{question}{halliday-ch32-q34}
    The units of magnetization are:
    \begin{choices}
        \wrongchoice{ampere (\si{\ampere})}
        \wrongchoice{ampere meter (\si{\ampere\meter})}
        \wrongchoice{ampere meter squared (\si{\ampere\meter\squared})}
      \correctchoice{ampere per meter (\si{\ampere\per\meter})}
        \wrongchoice{ampere per meter squared (\si{\ampere\per\meter\squared})}
    \end{choices}
\end{question}
}

\element{halliday-mc}{
\begin{question}{halliday-ch32-q35}
    If $\vec{L}$ is the orbital angular momentum of an electron,
        the magnetic dipole moment associated with its orbital motion:
    \begin{choices}
        \wrongchoice{is in the direction of $\vec{L}$ and has magnitude proportional to $L$}
      \correctchoice{is opposite to the direction of $\vec{L}$ and has magnitude proportional to $L$}
        \wrongchoice{is in the direction of $\vec{L}$ and has magnitude proportional to $L^2$}
        \wrongchoice{is opposite to the direction of $\vec{L}$ and has magnitude proportional to $L^2$}
        \wrongchoice{does not depend on $\vec{L}$}
    \end{choices}
\end{question}
}

\element{halliday-mc}{
\begin{question}{halliday-ch32-q36}
    If an electron has an orbital angular momentum with magnitude $L$ the magnitude of the orbital contribution to its magnetic dipole moment is given by:
    \begin{multicols}{3}
    \begin{choices}
        \wrongchoice{$\dfrac{eL}{m}$}
      \correctchoice{$\dfrac{eL}{2m}$}
        \wrongchoice{$\dfrac{2eL}{m}$}
        \wrongchoice{$\dfrac{mL}{e}$}
        \wrongchoice{$\dfrac{mL}{2}$}
    \end{choices}
    \end{multicols}
\end{question}
}

\element{halliday-mc}{
\begin{question}{halliday-ch32-q37}
    An electron traveling with speed $v$ around a circle of radius $r$ is equivalent to a current of:
    \begin{multicols}{3}
    \begin{choices}
        \wrongchoice{$\dfrac{evr}{2}$}
        \wrongchoice{$\dfrac{ev}{r}$}
      \correctchoice{$\dfrac{ev}{2\pi r}$}
        \wrongchoice{$\dfrac{2\pi er}{v}$}
        \wrongchoice{$\dfrac{2\pi ev}{r}$}
    \end{choices}
    \end{multicols}
\end{question}
}

\element{halliday-mc}{
\begin{question}{halliday-ch32-q38}
    The intrinsic magnetic dipole moments of protons and neutrons are much less than that of an electron because:
    \begin{choices}
      \correctchoice{their masses are greater}
        \wrongchoice{their angular momenta are much less}
        \wrongchoice{their angular momenta are much greater}
        \wrongchoice{their charges are much less}
        \wrongchoice{their radii are much less}
    \end{choices}
\end{question}
}

\element{halliday-mc}{
\begin{question}{halliday-ch32-q39}
    The spin magnetic dipole moment of an electron:
    \begin{choices}
        \wrongchoice{is in the same direction as the spin angular momentum}
        \wrongchoice{is zero}
        \wrongchoice{has a magnitude that depends on the orbital angular momentum}
        \wrongchoice{has a magnitude that depends on the applied magnetic field}
        \wrongchoice{none of the provided}
    \end{choices}
\end{question}
}

\element{halliday-mc}{
\begin{question}{halliday-ch32-q40}
    If an electron has zero orbital angular momentum,
        the magnitude of its magnetic dipole moment equals:
    \begin{choices}
        \wrongchoice{zero}
        \wrongchoice{half the Bohr magneton}
        \wrongchoice{a Bohr magneton}
        \wrongchoice{twice a Bohr magneton}
        \wrongchoice{none of the provided}
    \end{choices}
\end{question}
}

\element{halliday-mc}{
\begin{question}{halliday-ch32-q41}
    The magnetic dipole moment of an atomic electron is typically:
    \begin{choices}
        \wrongchoice{much less than a Bohr magneton}
      \correctchoice{a few Bohr magnetons}
        \wrongchoice{much greater than a Bohr magneton}
        \wrongchoice{much greater or much less than a Bohr magneton, depending on the atom}
        \wrongchoice{not related to the value of the Bohr magneton}
    \end{choices}
\end{question}
}

\element{halliday-mc}{
\begin{question}{halliday-ch32-q42}
    The magnitude of the Bohr magneton is about:
    \begin{multicols}{2}
    \begin{choices}
        \wrongchoice{\SI{e-15}{\joule\per\tesla}}
        \wrongchoice{\SI{e-19}{\joule\per\tesla}}
      \correctchoice{\SI{e-23}{\joule\per\tesla}}
        \wrongchoice{\SI{e-27}{\joule\per\tesla}}
        \wrongchoice{\SI{e-31}{\joule\per\tesla}}
    \end{choices}
    \end{multicols}
\end{question}
}

\element{halliday-mc}{
\begin{question}{halliday-ch32-q43}
    The molecular theory of magnetism can explain each of the following \emph{except}:
    \begin{choices}
      \correctchoice{an N pole attracts a S pole}
        \wrongchoice{stroking an iron bar with a magnet will magnetize the bar}
        \wrongchoice{when a bar magnet is broken in two, each piece is a bar magnet}
        \wrongchoice{heating tends to destroy magnetization}
        \wrongchoice{hammering tends to destroy magnetization}
    \end{choices}
\end{question}
}

\element{halliday-mc}{
\begin{question}{halliday-ch32-q44}
    Lenz' law can explain:
    \begin{choices}
        \wrongchoice{paramagnetism only}
      \correctchoice{diamagnetism only}
        \wrongchoice{ferromagnetism only}
        \wrongchoice{only two of the three types of magnetism}
        \wrongchoice{all three of the types of magnetism}
    \end{choices}
\end{question}
}

\element{halliday-mc}{
\begin{question}{halliday-ch32-q45}
    The diagram shows two small diamagnetic spheres,
        one near each end of a bar magnet. 
    \begin{center}
    \begin{tikzpicture}
        %% NOTE: tikz
    \end{tikzpicture}
    \end{center}
    Which of the following statements is true?
    \begin{choices}
        \wrongchoice{The force on 1 is toward the magnet and the force on 2 is away from the magnet}
        \wrongchoice{The force on 1 is away from the magnet and the force on 2 is away from the magnet}
        \wrongchoice{The forces on 1 and 2 are both toward the magnet}
      \correctchoice{The forces on 1 and 2 are both away from the magnet}
        \wrongchoice{The magnet does not exert a force on either sphere}
    \end{choices}
\end{question}
}

\element{halliday-mc}{
\begin{question}{halliday-ch32-q46}
    Paramagnetism is closely associated with:
    \begin{choices}
      \correctchoice{the tendency of electron dipole moments to align with an applied magnetic field}
        \wrongchoice{the tendency of electron dipole moments to align opposite to an applied magnetic field}
        \wrongchoice{the exchange force between electrons}
        \wrongchoice{the force exerted by electron dipole moments on each other}
        \wrongchoice{the torque exerted by electron dipole moments on each other}
    \end{choices}
\end{question}
}

\element{halliday-mc}{
\begin{question}{halliday-ch32-q47}
    The diagram shows two small paramagnetic spheres,
        one near each end of a bar magnet. 
    \begin{center}
    \begin{tikzpicture}
        %% NOTE: tikz
    \end{tikzpicture}
    \end{center}
    Which of the following statements is true?
    \begin{choices}
        \wrongchoice{The force on 1 is toward the magnet and the force on 2 is away from the magnet}
        \wrongchoice{The force on 1 is away from the magnet and the force on 2 is away from the magnet}
      \correctchoice{The forces on 1 and 2 are both toward the magnet}
        \wrongchoice{The forces on 1 and 2 are both away from the magnet}
        \wrongchoice{The magnet does not exert a force on either sphere}
    \end{choices}
\end{question}
}

\element{halliday-mc}{
\begin{question}{halliday-ch32-q48}
    A paramagnetic substance is placed in a weak magnetic field and its absolute temperature $T$ is increased. 
    As a result,
        its magnetization:
    \begin{choices}
        \wrongchoice{increases in proportion to $T$}
        \wrongchoice{increases in proportion to $T^2$}
        \wrongchoice{remains the same}
      \correctchoice{decreases in proportion to $\dfrac{1}{T}$}
        \wrongchoice{decreases in proportion to $\dfrac{1}{T^2}$}
    \end{choices}
\end{question}
}

\element{halliday-mc}{
\begin{question}{halliday-ch32-q49}
    A magnetic field $\vec{B}_0$ is applied to a paramagnetic substance. 
    In the interior the magnetic field produced by the magnetic dipoles of the substance is:
    \begin{choices}
        \wrongchoice{greater than $\vec{B}_0$ and in the opposite direction}
        \wrongchoice{less than $\vec{B}_0$ and in the opposite direction}
        \wrongchoice{greater than $\vec{B}_0$ and in the same direction}
      \correctchoice{less than $\vec{B}_0$ and in the same direction}
        \wrongchoice{the same as $\vec{B}_0$}
    \end{choices}
\end{question}
}

\element{halliday-mc}{
\begin{question}{halliday-ch32-q50}
    A paramagnetic substance, in an external magnetic field,
        is thermally isolated. 
    The field is then removed. 
    As a result:
    \begin{choices}
        \wrongchoice{the magnetic energy of the magnetic dipoles decreases}
        \wrongchoice{the temperature of the substance increases}
        \wrongchoice{the magnetization decreases, but only slightly}
        \wrongchoice{the magnetization reverses direction}
      \correctchoice{none of the provided}
    \end{choices}
\end{question}
}

\element{halliday-mc}{
\begin{question}{halliday-ch32-q51}
    A magnetic field $\vec{B}_0$ is applied to a diamagnetic substance. 
    In the interior the magnetic field produced by the magnetic dipoles of the substance is:
    \begin{choices}
        \wrongchoice{greater than $\vec{B}_0$ and in the opposite direction}
      \correctchoice{less than $\vec{B}_0$ and in the opposite direction}
        \wrongchoice{greater than $\vec{B}_0$ and in the same direction}
        \wrongchoice{less than $\vec{B}_0$ and in the same direction}
        \wrongchoice{the same as $\vec{B}_0$}
    \end{choices}
\end{question}
}

\element{halliday-mc}{
\begin{question}{halliday-ch32-q52}
    Ferromagnetism is closely associated with:
    \begin{choices}
        \wrongchoice{the tendency of electron dipole moments to align with an applied magnetic field}
        \wrongchoice{the tendency of electron dipole moments to align opposite to an applied magnetic field}
        \wrongchoice{the tendency of electron dipole moments to change magnitude in an applied magnetic field}
      \correctchoice{the tendency of electron dipole moments to align with each other}
        \wrongchoice{the force exerted by electron dipole moments on each other}
    \end{choices}
\end{question}
}

\element{halliday-mc}{
\begin{question}{halliday-ch32-q53}
    Of the three chief kinds of magnetic materials 
        (diamagnetic, paramagnetic, and ferromagnetic),
        which are used to make permanent magnets?
    \begin{choices}
        \wrongchoice{Only diamagnetic}
      \correctchoice{Only ferromagnetic}
        \wrongchoice{Only paramagnetic}
        \wrongchoice{Only paramagnetic and ferromagnetic}
        \wrongchoice{All three types}
    \end{choices}
\end{question}
}

\element{halliday-mc}{
\begin{question}{halliday-ch32-q54}
    When a permanent magnet is strongly heated:
    \begin{choices}
        \wrongchoice{nothing happens}
        \wrongchoice{it becomes an induced magnet}
      \correctchoice{it loses its magnetism}
        \wrongchoice{its magnetism increases}
        \wrongchoice{its polarity reverses}
    \end{choices}
\end{question}
}

\element{halliday-mc}{
\begin{question}{halliday-ch32-q55}
    Magnetization vectors in neighboring ferromagnetic domains are:
    \begin{choices}
        \wrongchoice{always in opposite directions}
        \wrongchoice{always in the same direction}
      \correctchoice{always in different directions}
        \wrongchoice{sometimes in different directions and sometimes in the same direction}
        \wrongchoice{sometimes in opposite directions and sometimes in the same direction}
    \end{choices}
\end{question}
}

\element{halliday-mc}{
\begin{question}{halliday-ch32-q56}
    The behavior of ferromagnetic domains in an applied magnetic field gives rise to:
    \begin{choices}
      \correctchoice{hysteresis}
        \wrongchoice{ferromagnetism}
        \wrongchoice{the Curie law}
        \wrongchoice{a lowering of the Curie temperature}
        \wrongchoice{Gauss’ law for magnetism}
    \end{choices}
\end{question}
}

\element{halliday-mc}{
\begin{question}{halliday-ch32-q57}
    Because ferromagnets have ferromagnetic domains,
        the net magnetization:
    \begin{choices}
        \wrongchoice{can never be in the same direction as an applied field}
      \correctchoice{may not vanish when an applied field is reduced to zero}
        \wrongchoice{can never vanish}
        \wrongchoice{is proportional to any applied magnetic field}
        \wrongchoice{is always opposite to the direction of any applied magnetic field}
    \end{choices}
\end{question}
}

\element{halliday-mc}{
\begin{question}{halliday-ch32-q58}
    The soft iron core in the solenoid shown is removable. 
    \begin{center}
    \begin{tikzpicture}
        %% NOTE: tikz
    \end{tikzpicture}
    \end{center}
    Then:
    \begin{choices}
        \wrongchoice{the current will be larger without the core}
        \wrongchoice{the current will be larger with the core}
      \correctchoice{one must do work to remove the core}
        \wrongchoice{the circuit will do work in expelling the core}
        \wrongchoice{the stored energy is the same with or without the core}
    \end{choices}
\end{question}
}

\element{halliday-mc}{
\begin{question}{halliday-ch32-q59}
    An unmagnetized steel bar is placed inside a solenoid. 
    As the current in the solenoid is slowly increased from zero to some large value,
        the magnetization of the bar:
    \begin{choices}
        \wrongchoice{increases proportionally with the current}
        \wrongchoice{remains zero for awhile and then increases linearly with any further increase in current}
      \correctchoice{increases with increasing current at first but later is much less affected by it}
        \wrongchoice{is unaffected by the current}
        \wrongchoice{increases quadratically with the current}
    \end{choices}
\end{question}
}

\element{halliday-mc}{
\begin{question}{halliday-ch32-q60}
    The magnetic field of Earth is roughly the same as that of a magnetic dipole with a dipole moment of about:
    \begin{multicols}{3}
    \begin{choices}
        \wrongchoice{\SI{e17}{\joule\per\tesla}}
        \wrongchoice{\SI{e19}{\joule\per\tesla}}
        \wrongchoice{\SI{e21}{\joule\per\tesla}}
      \correctchoice{\SI{e23}{\joule\per\tesla}}
        \wrongchoice{\SI{e25}{\joule\per\tesla}}
    \end{choices}
    \end{multicols}
\end{question}
}

\element{halliday-mc}{
\begin{question}{halliday-ch32-q61}
    Of the following places,
        one would expect that the horizontal component of Earth's magnetic field is largest in:
    \begin{multicols}{3}
    \begin{choices}
        \wrongchoice{Maine}
      \correctchoice{Florida}
        \wrongchoice{Maryland}
        \wrongchoice{New York}
        \wrongchoice{Iowa}
    \end{choices}
    \end{multicols}
\end{question}
}

\element{halliday-mc}{
\begin{question}{halliday-ch32-q62}
    A positively charged ion, due to a cosmic ray,
        is headed through Earth's atmosphere toward the center of Earth. 
    Due to Earth's magnetic field,
        the ion will be deflected:
    \begin{multicols}{2}
    \begin{choices}
        \wrongchoice{south}
        \wrongchoice{north}
        \wrongchoice{west}
      \correctchoice{east}
        \wrongchoice{not at all since it is a charge and not a pole}
    \end{choices}
    \end{multicols}
\end{question}
}

\element{halliday-mc}{
\begin{question}{halliday-ch32-q63}
    Maxwell's great contribution to electromagnetic theory was his hypothesis that:
    \begin{choices}
        \wrongchoice{work is required to move a magnetic pole through a closed path surrounding a current}
      \correctchoice{a time-varying electric flux acts as a current for purposes of producing a magnetic field}
        \wrongchoice{the speed of light could be determined from simple electrostatic and magnetostatic experiments (finding the values of $\mu_0$ and $\epsilon_0$)}
        \wrongchoice{the magnetic force on a moving charge particle is perpendicular to both $\vec{v}$ and $\vec{B}$}
        \wrongchoice{magnetism could be explained in terms of circulating currents in atoms}
    \end{choices}
\end{question}
}

\element{halliday-mc}{
\begin{question}{halliday-ch32-q64}
    Displacement current is:
    \begin{multicols}{3}
    \begin{choices}
        \wrongchoice{$\dfrac{\mathrm{d}\Phi_E}{\mathrm{d}t}$}
      \correctchoice{$\epsilon_0\dfrac{\mathrm{d}\Phi_E}{\mathrm{d}t}$}
        \wrongchoice{$\mu_0\dfrac{\mathrm{d}\Phi_E}{\mathrm{d}t}$}
        \wrongchoice{$\mu_0\epsilon_0\dfrac{\mathrm{d}\Phi_E}{\mathrm{d}t}$}
        \wrongchoice{$\dfrac{−\mathrm{d}\Phi_B}{\mathrm{d}t}$}
    \end{choices}
    \end{multicols}
\end{question}
}

\element{halliday-mc}{
\begin{question}{halliday-ch32-q65}
    Displacement current exists wherever there is:
    \begin{choices}
        \wrongchoice{a magnetic field}
        \wrongchoice{moving charge}
        \wrongchoice{a changing magnetic field}
        \wrongchoice{an electric field}
      \correctchoice{a changing electric field}
    \end{choices}
\end{question}
}

\element{halliday-mc}{
\begin{question}{halliday-ch32-q66}
    Displacement current exists in the region between the plates of a parallel plate capacitor if:
    \begin{choices}
        \wrongchoice{the capacitor leaks charge across the plates}
      \correctchoice{the capacitor is being discharged}
        \wrongchoice{the capacitor is fully charged}
        \wrongchoice{the capacitor is fully discharged}
        \wrongchoice{none of the provided are true}
    \end{choices}
\end{question}
}

\element{halliday-mc}{
\begin{question}{halliday-ch32-q67}
    An electric field exists in the cylindrical region shown and is parallel to the cylinder axis. 
    The magnitude of the field might vary with time according to any of the four graphs shown. 
    \begin{center}
    \begin{tikzpicture}
        %% NOTE:
    \end{tikzpicture}
    \end{center}
    Rank the four variations according to the magnitudes of the magnetic field induced at the edge of the region,
        least to greatest.
    \begin{multicols}{2}
    \begin{choices}
      \correctchoice{2, 4, 3, 1}
        \wrongchoice{3 and 4 tie, then 1, 2}
        \wrongchoice{4, 3, 2, 1}
        \wrongchoice{4, 3, 1, 2}
        \wrongchoice{2, 1, 3, 4}
    \end{choices}
    \end{multicols}
\end{question}
}

\element{halliday-mc}{
\begin{question}{halliday-ch32-q68}
    The diagram shows one plate of a parallel-plate capacitor from within the capacitor.
    The plate is circular and has radius $R$. 
    The dashed circles are four integration paths and have radii of $r_1=R/4$, $r2=R/2$, $r_3=3R/2$, and $r_4=2R$.
    \begin{center}
    \begin{tikzpicture}
        %% NOTE:
    \end{tikzpicture}
    \end{center}
    Rank the paths according to the magnitude of $\oint\,\vec{B}\cdot\mathrm{d}\vec{s}$ around the paths during the discharging of the capacitor,
        least to greatest.
    \begin{multicols}{2}
    \begin{choices}
        \wrongchoice{1, 2 and 3 tie, then 4}
        \wrongchoice{1, 2, 3, 4}
      \correctchoice{1, 2, 3 and 4 tie}
        \wrongchoice{4, 3, 1, 2}
        \wrongchoice{all tie}
    \end{choices}
    \end{multicols}
\end{question}
}

\element{halliday-mc}{
\begin{question}{halliday-ch32-q69}
    A \SI{1.2}{\meter} radius cylindrical region contains a uniform electric field along the cylinder axis. 
    It is increasing uniformly with time. 
    To obtain a total displacement current of \SI{2.0e-9}{\ampere} through a cross section of the region,
        the magnitude of the electric field should change at a rate of:
    \begin{multicols}{2}
    \begin{choices}
        \wrongchoice{\SI{5.0}{\volt\per\meter\per\second}}
        \wrongchoice{\SI{12}{\volt\per\meter\per\second}}
        \wrongchoice{\SI{37}{\volt\per\meter\per\second}}
      \correctchoice{\SI{50}{\volt\per\meter\per\second}}
        \wrongchoice{\SI{4.0e7}{\volt\per\meter\per\second}}
    \end{choices}
    \end{multicols}
\end{question}
}

\element{halliday-mc}{
\begin{question}{halliday-ch32-q70}
    A current of \SI{1}{\ampere} is used to charge a parallel plate capacitor with square plates. 
    If the area of each plate is \SI{0.6}{\meter\squared} the displacement current through a \SI{0.3}{\meter\squared} area wholly between the capacitor plates and parallel to them is:
    \begin{multicols}{3}
    \begin{choices}
        \wrongchoice{\SI{1}{\ampere}}
        \wrongchoice{\SI{2}{\ampere}}
        \wrongchoice{\SI{0.7}{\ampere}}
      \correctchoice{\SI{0.5}{\ampere}}
        \wrongchoice{\SI{0.25}{\ampere}}
    \end{choices}
    \end{multicols}
\end{question}
}

\element{halliday-mc}{
\begin{question}{halliday-ch32-q71}
    A \SI{1}{\micro\farad} capacitor is connected to an emf that is increasing uniformly with time at a rate of \SI{100}{\volt\per\second}.
    The displacement current between the plates is:
    \begin{multicols}{2}
    \begin{choices}
        \wrongchoice{zero}
        \wrongchoice{\SI{1e-8}{\ampere}}
        \wrongchoice{\SI{1e-6}{\ampere}}
      \correctchoice{\SI{1e-4}{\ampere}}
        \wrongchoice{\SI{100}{\ampere}}
    \end{choices}
    \end{multicols}
\end{question}
}

\element{halliday-mc}{
\begin{question}{halliday-ch32-q72}
    A magnetic field exists between the plates of a capacitor:
    \begin{choices}
        \wrongchoice{always}
        \wrongchoice{never}
        \wrongchoice{when the capacitor is fully charged}
      \correctchoice{while the capacitor is being charged}
        \wrongchoice{only when the capacitor is starting to be charged}
    \end{choices}
\end{question}
}

\element{halliday-mc}{
\begin{question}{halliday-ch32-q73}
    Suppose you are looking into one end of a long cylindrical tube in which there is a uniform electric field,
        pointing away from you. 
    If the magnitude of the field is decreasing with time the direction of the induced magnetic field is:
    \begin{choices}
        \wrongchoice{toward you}
        \wrongchoice{away from you}
        \wrongchoice{clockwise}
      \correctchoice{counterclockwise}
        \wrongchoice{to your right}
    \end{choices}
\end{question}
}

\element{halliday-mc}{
\begin{question}{halliday-ch32-q74}
    Suppose you are looking into one end of a long cylindrical tube in which there is a uniform electric field,
        pointing away from you. 
    If the magnitude of the field is decreasing with time the field lines of the induced magnetic field are:
    \begin{choices}
      \correctchoice{circles}
        \wrongchoice{ellipses}
        \wrongchoice{straight lines parallel to the electric field}
        \wrongchoice{straight lines perpendicular to the electric field}
        \wrongchoice{none of the provided}
    \end{choices}
\end{question}
}

\element{halliday-mc}{
\begin{question}{halliday-ch32-q75}
    A cylindrical region contains a uniform electric field that is along the cylinder axis and is changing with time. 
    If $r$ is distance from the cylinder axis the magnitude of the magnetic field within the region is:
    \begin{choices}
        \wrongchoice{uniform}
        \wrongchoice{proportional to $\dfrac{1}{r}$}
        \wrongchoice{proportional to $r^2$}
        \wrongchoice{proportional to $\dfrac{1}{r^2}$}
      \correctchoice{proportional to $r$}
    \end{choices}
\end{question}
}

\element{halliday-mc}{
\begin{question}{halliday-ch32-q76}
    A cylindrical region contains a uniform electric field that is parallel to the axis and is changing with time. 
    If $r$ is distance from the cylinder axis the magnitude of the magnetic field outside the region is:
    \begin{choices}
        \wrongchoice{zero}
      \correctchoice{proportional to $\dfrac{1}{r}$}
        \wrongchoice{proportional to $r^2$}
        \wrongchoice{proportional to $\dfrac{1}{r^2}$}
        \wrongchoice{proportional to $r$}
    \end{choices}
\end{question}
}

\element{halliday-mc}{
\begin{question}{halliday-ch32-q77}
    A \SI{0.70}{\meter} radius cylindrical region contains a uniform electric field that is parallel to the axis and is decreasing at the rate \SI{5.0e12}{\volt\per\meter\per\second}.
    The magnetic field at a point \SI{0.25}{\meter} from the axis has a magnitude of:
    \begin{multicols}{2}
    \begin{choices}
        \wrongchoice{zero}
      \correctchoice{\SI{7.0e-6}{\tesla}}
        \wrongchoice{\SI{2.8e-5}{\tesla}}
        \wrongchoice{\SI{5.4e-5}{\tesla}}
        \wrongchoice{\SI{7.0e-5}{\tesla}}
    \end{choices}
    \end{multicols}
\end{question}
}

\element{halliday-mc}{
\begin{question}{halliday-ch32-q78}
    A \SI{0.70}{\meter} radius cylindrical region contains a uniform electric field that is parallel to the axis and is decreasing at the rate \SI{5.0e12}{\volt\per\meter\per\second}.
    The magnetic field at a point \SI{1.2}{\meter} from the axis has a magnitude of:
    \begin{multicols}{2}
    \begin{choices}
        \wrongchoice{zero}
        \wrongchoice{\SI{7.0e-6}{\tesla}}
      \correctchoice{\SI{1.1e-5}{\tesla}}
        \wrongchoice{\SI{2.3e-5}{\tesla}}
        \wrongchoice{\SI{2.8e-5}{\tesla}}
    \end{choices}
    \end{multicols}
\end{question}
}

\element{halliday-mc}{
\begin{question}{halliday-ch32-q79}
    A \SI{1}{\ampere} current is used to charge a parallel plate capacitor. 
    A large square piece of paper is placed between the plates and parallel to them so it sticks out on all sides.
    The value of the integral $\oint\,\vec{B}\cdot\mathrm{d}\vec{s}$ around the perimeter of the paper is:
    \begin{multicols}{2}
    \begin{choices}
        \wrongchoice{\SI{2}{\tesla\meter}}
        \wrongchoice{\SI{4\pi e-7}{\tesla\meter}}
        \wrongchoice{\SI{8.85e-12}{\tesla\meter}}
        \wrongchoice{\SI{e-7}{\tesla\meter}}
        \wrongchoice{not determined from the given quantities}
    \end{choices}
    \end{multicols}
\end{question}
}

\element{halliday-mc}{
\begin{question}{halliday-ch32-q80}
    A sinusoidal emf is connected to a parallel plate capacitor. 
    The magnetic field between the plates is:
    \begin{choices}
        \wrongchoice{zero}
        \wrongchoice{constant}
        \wrongchoice{sinusoidal and its amplitude does not depend on the frequency of the source}
      \correctchoice{sinusoidal and its amplitude is proportional to the frequency of the source}
        \wrongchoice{sinusoidal and its amplitude is inversely proportional to the frequency of the source}
    \end{choices}
\end{question}
}

\element{halliday-mc}{
\begin{question}{halliday-ch32-q81}
    An electron is on the $z$ axis moving toward the $xy$ plane but it has not reached that plane yet.
    At that instant:
    \begin{choices}
        \wrongchoice{there is only a true current through the $xy$ plane}
      \correctchoice{there is only a displacement current through the $xy$ plane}
        \wrongchoice{there are both true and displacement currents through the $xy$ plane}
        \wrongchoice{there is neither a true nor a displacement current through the $xy$ plane}
        \wrongchoice{none of the provided are true}
    \end{choices}
\end{question}
}


\endinput


