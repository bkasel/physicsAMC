

%% Centripetal
%%------------------------------------------------


\begin{question}[ID=centripetal-A-Q01,topic=circular-motion,difficulty=A]
    A roller coaster car stops just before it starts down the first hill.
    At the bottom of the hill, which is \SI{65}{\meter} below, the 
        track curves upward with a radius of \SI{48}{\meter}.
    If \SI{1}{G} is equal to the weight of the rider, how many G's
        does the rider experience at the bottom of the curve?
\end{question}
\begin{solution}
    N/A
\end{solution}


\begin{question}[ID=centripetal-A-Q02,topic=circular-motion,difficulty=A]
    A roller coaster car is traveling at \SI{10}{\meter\per\second}
        as it crests a hill.
    At the bottom of the hill, which is \SI{25}{\meter} below,
        the designers want the rider to experience \SI{3.5}{G}'s as
        it heads to the next hill.
    Draw the free-body diagram of the car at the bottom of the hill.
    Calculate the required radius of the curve at the bottom of the hill.
\end{question}
\begin{solution}
    N/A
\end{solution}


\begin{question}[ID=centripetal-A-Q03,topic=circular-motion,difficulty=A]
    A roller coaster has an initial drop of \SI{75}{\meter}
        where the cars start down the hill from rest.
    After the rider survives the first hill, the designers
        want the riders to experience \SI{0.1}{G} as they
        reach the top of the second hill.
    Assuming there is no loss of speed due to friction,
        how high should they make the second hill if it
        is to have a radius of \SI{25}{\meter}?
    And of course you must draw the free-body diagram of
        the car at the top of the hill.
\end{question}
\begin{solution}
    N/A
\end{solution}



\begin{question}[ID=centripetal-A-Q04,topic=circular-motion,difficulty=A]
    A roller coaster car crests a hill at \SI{16}{\meter\per\second}
        and heads down a \SI{35}{\meter} incline.
    The car then goes into an inverted loop (upside-down) with a
        radius of \SI{30}{\meter}.
    The ride is designed so that the rider experiences only
        \SI{0.5}{G} at the top of the loop.
    How far below the first hill is the top of the inverted loop?
    (Assume no loss of speed due to friction.)
\end{question}
\begin{solution}
    N/A
\end{solution}


\begin{question}[ID=centripetal-B-Q01,topic=circular-motion,difficulty=B]
    A roller coaster rises up onto a hill with radius of
        \SI{20}{\meter}.
    You want the riders to experience only \sfrac{1}{4}
        of their weight as they reach the top of the hill.
    How fast should the roller coaster car be going at that
        point to achieve our objective?
\end{question}
\begin{solution}
    N/A
\end{solution}


\begin{question}[ID=centripetal-B-Q02,topic=circular-motion,difficulty=B]
    A roller coaster has an inverted loop with a radius
        of \SI{25}{\meter}.
    You want the riders to be held upside-down in the seat
        with exactly \SI{1}{G} (The same as if they were
        sitting still on the track).
    Draw a free-body diagram of the roller coaster car at
        the top of this loop.
    How fast must the roller coaster car be going at the
        top of the loop to achieve our objective?
\end{question}
\begin{solution}
    N/A
\end{solution}


\begin{question}[ID=centripetal-B-Q03,topic=circular-motion,difficulty=B]
    A roller coaster car is traveling at \SI{23}{\meter\per\second}
        as it enters an upward curve at the bottom of a hill.
    The curve has a radius of \SI{40}{\meter} and the rider has a
        mass of \SI{70}{\kilo\gram}.
    Draw a free-body diagram of the rider.
    How much does his body weight change while he is
        experiencing the curve?
\end{question}
\begin{solution}
    N/A
\end{solution}


\begin{question}[ID=centripetal-B-Q04,topic=circular-motion,difficulty=B]
    A roller coaster is traveling at \SI{25}{\meter\per\second}
        as it enters a horizontal loop at the bottom of a hill.
    The loop has a radius of \SI{45}{\meter} and the rider has
        a mass of \SI{75}{\kilo\gram}.
    Draw a free body diagram of the forces acting on the rider.
    How much does his body weight while he is experiencing the
        curve?
\end{question}
\begin{solution}
    N/A
\end{solution}


\begin{question}[ID=centripetal-C-Q01,topic=circular-motion,difficulty=C]
    A carousel at the county fair has three seats next to each other.
    Closest in is a Chicken seat,
        which is \SI{5.0}{\meter} from the center.
    The next closest is the Unicorn,
        which is \SI{8.0}{\meter} from the center.
    The farthest from the center is the Horsey chair,
        which is \SI{12.0}{\meter} from the center.
    The ride makes a revolution once every \SI{15}{\second}.
    \begin{enumerate*}[label=\arabic*)]
        \item What is the tangential speed of the three seats?
        \item What is the centripetal acceleration on each of the seats?
    \end{enumerate*}
\end{question}
\begin{solution}
    Chicken: \SI{2.09}{\meter\per\second}, \SI{0.877}{\meter\per\second\squared};
    Unicorn: \SI{3.35}{\meter\per\second}, \SI{1.40}{\meter\per\second\squared};
    Horsey: \SI{5.03}{\meter\per\second}, \SI{2.11}{\meter\per\second\squared};
\end{solution}


\begin{question}[ID=centripetal-C-Q02,topic=circular-motion,difficulty=C]
    A customer \SI{11}{\meter} from the center of a revolving
        restaurant has a speed of \SI{1.92e-2}{\meter\per\second}.
    How large a centripetal acceleration acts on the customer?
\end{question}
\begin{solution}
    %% a = v^2 / r
    \SI{3.35e-5}{\meter\per\second\squared}
\end{solution}


\begin{question}[ID=centripetal-C-Q03,topic=circular-motion,difficulty=C]
    A toy train on a circular track has a tangential speed of
        \SI{0.35}{\meter\per\second} and a centripetal acceleration
        of \SI{0.29}{\meter\per\second\squared}.
    What is the radius of the track?
\end{question}
\begin{solution}
    %% a = v^2 / r  -->  r = v^2 / a
    \SI{0.422}{\meter}
\end{solution}


\begin{question}[ID=centripetal-C-Q04,topic=circular-motion,difficulty=C]
    An \SI{1250}{\kilo\gram} automobile with a tangential speed
        of \SI{48.0}{\kilo\meter\per\hour} follows a circular
        road that has a radius of \SI{35.0}{\meter}.
    How large is the centripetal force?
\end{question}
\begin{solution}
    %% convert km/h to m/s
    %% F = m v^2 / r
    \SI{6.35e3}{\newton}
\end{solution}


\endinput

