

\begin{question}[ID=DA01,topic=measurement,difficulty=easy]
    The period of a simple pendulum, defined as the time necessary for one complete oscillation,
        is measured in time units and is given by
    \begin{equation*}
        T = 2\pi \sqrt{\frac{l}{g}}
    \end{equation*}
    where $l$ is the length of the pendulum and $g$ is the acceleration due to gravity,
        in units of length per time squared.
    Show that this formula is dimensionally consistent.
\end{question}
\begin{solution}
    $\si{\second} = \si{\second}$
    %$\si{\second} = \sqrt{\frac{\si{\meter}}{\si{\meter\per\second\sqaured}}}
    %              = \sqrt{ \si{\second\squared}}
    %              = \si{\second} $
\end{solution}

\begin{question}[ID=DA02,topic=measurement,difficulty=easy]
    The period of a mass on a spring, defined as the time necessary for one complete oscillation,
        is measured in time units and is given by
    \begin{equation*}
        T = 2\pi \sqrt{\frac{m}{k}}
    \end{equation*}
    where $m$ is the mass of the object,
    and $k$ is a spring constant.
    A spring constant has dimensionality of force per unit length.
    Force has dimensionality of \si{\kilo\gram\meter\per\second\squared}.
    Show that this formula is dimensionally consistent.
\end{question}
\begin{solution}
    $\si{\second} = \si{\second}$
    %$\si{\second} = \sqrt{\frac{\si{\kilo\gram}}{\si{\newton\per\meter}}}
    %              = \sqrt{\frac{\si{\kilo\gram}}{\si{\kilo\gram\per\second\squared}}}
    %              = \sqrt{ \si{\second\squared}}
    %              = \si{\second} $
\end{solution}

\begin{question}[ID=DA03,topic=measurement,difficulty=easy]
    Newton's law of universal graviation is represented by
    \begin{equation*}
        F = G \frac{Mm}{r^2}
    \end{equation*}
    where $F$ is the gravitational force, $M$ and $m$ are masses, and $r$ is a length.
    Force has the SI units \si{\kilo\gram\meter\per\second\squared}.
    What are the SI units of the proportionality constant, $G$?
\end{question}
\begin{solution}
    $G = \si{\meter\cubed\per\kilo\gram\per\second\squared}$
\end{solution}

\begin{question}[ID=DA04,topic=measurement,difficulty=easy]
    Coulomb's law described the force between two charged objects, and is represented by
    \begin{equation*}
        F = k_e \frac{q_1 q_2}{r^2}
    \end{equation*}
    where $F$ is the electrostatic force, $q_1$ and $q_2$ are charges, and $r$ is a length.
    Force has the SI units \si{\kilo\gram\meter\per\second\squared}.
    What are the SI units of the proportionality constant, $k_e$?
\end{question}
\begin{solution}
    $k_e = \si{\kilo\gram\meter\cubed\per\coulomb\squared\per\second\squared}$
\end{solution}

\begin{question}[ID=DA05,topic=measurement,difficulty=easy]
    Planck's constant describes the relationship between a photon's frequency
        and it's energy by
    \begin{equation*}
        E = h f
    \end{equation*}
    where $E$ is the photon's energy, $f$ is the frequency, and $h$ is Planck's constant.
    Energy has the SI units \si{\kilo\gram\meter\squared\per\second\squared}
        and frequency has SI units of \si{\per\second} or \si{\hertz}.
    What are the SI units of Planck's constant, $h$?
\end{question}
\begin{solution}
    \si{\kilo\gram\meter\squared\per\second}
\end{solution}

\begin{question}[ID=DA06,topic=measurement,difficulty=easy]
    The de Broglie wavelength of a particle is described by
    \begin{equation*}
        \lambda = \frac{h}{p}
    \end{equation*}
    where $\lambda$ is a measure of distance, $p$ is a measure of mass times velocity,
        and $h$ is Planck's constant.
    What are the SI units of Planck's constant, $h$?
\end{question}
\begin{solution}
    \si{\kilo\gram\meter\squared\per\second}
\end{solution}

\begin{question}[ID=DA06,topic=measurement,difficulty=easy]
    The velocity of a wave is related to its frequency and wavelength by
    \begin{equation*}
        v = f \lambda
    \end{equation*}
    where $v$ is velocity and $\lambda$ is measure of distance.
    What are the SI units of the frequency, $f$?
\end{question}
\begin{solution}
    \si{\per\second} or \si{\hertz}
\end{solution}

\begin{question}[ID=DA07,topic=measurement,difficulty=easy]
    The Lorentz force describes the force that a charged particle
        will experience in an electromagnetic field by
    \begin{equation*}
        F = q ( E + v \times B )
    \end{equation*}
    where $F$ is the force, $q$ is the charge, $v$ is velocity,
        and $B$ is the magnetic field.
    Force has dimension of \si{\kilo\gram\meter\per\second\squared}.
    What are the SI units of a magnetic field, $B$?
\end{question}
\begin{solution}
    \si{\kilo\gram\per\coulomb\per\second}
\end{solution}

\begin{question}[ID=DA08,topic=measurement,difficulty=easy]
    The Lorentz force describes the force that a charged particle
        will experience in an electromagnetic field by
    \begin{equation*}
        F = q ( E + v \times B )
    \end{equation*}
    where $F$ is the force, $q$ is the charge, $v$ is velocity,
        and $E$ is the electric field.
    Force has dimension of \si{\kilo\gram\meter\per\second\squared}.
    What are the SI units of an electric field, $E$?
\end{question}
\begin{solution}
    \si{\kilo\gram\meter\per\coulomb\per\second\squared}
\end{solution}

\begin{question}[ID=DA09,topic=measurement,difficulty=easy]
    A magnitic field will produce a force on a moving charged particle by
    \begin{equation*}
        F = q v \times B,
    \end{equation*}
    where $F$ is a force and measured in Newtons (\si{\newton}),
        $q$ is a charge,
        and $v$ is a velocity.
    What are the SI units of a magnititic field?
\end{question}
\begin{solution}
    \si{\kilo\gram\per\coulomb\per\second}
\end{solution}

\begin{question}[ID=DA10,topic=measurement,difficulty=easy]
    The Biot-Savart law relates a current to a produced magnetic field, $B$, by
    \begin{equation*}
        B = \frac{\mu_0 I}{4\pi r}
    \end{equation*}
    where $I$ is current, $r$ is a distance and $\mu_0$ is the
        magnetic permeability of free space and has SI units of
        newtons per ampere squared (\si{\newton\per\ampere\squared}).
    What are the SI units of a magnititic field?
\end{question}
\begin{solution}
    \si{\kilo\gram\per\coulomb\per\second}
\end{solution}

\begin{question}[ID=DA11,topic=measurement,difficulty=easy]
    The fine structure constant is represented by
    \begin{equation*}
        \alpha = \frac{k_e e^2}{p}
    \end{equation*}
    where $\lambda$ is a measure of distance, $p$ is a measure of mass times velocity,
        and $h$ is Planck's constant.
    What are the SI units of Planck's constant, $h$?
\end{question}
\begin{solution}
    \si{\kilo\gram\meter\squared\per\second}
\end{solution}


%% Dimensional Analysis of equations
%%----------------------------------------
\begin{question}[ID=DA21,topic=measurement,difficulty=easy]
    Show why the following equation cannot be correct.
    \begin{equation*}
        \frac{1}{2} m v^2 = \frac{1}{2} mv_0^2 + \sqrt{mgh}
    \end{equation*}
\end{question}
\begin{solution}
    $\si{\kilo\gram\meter\squared\per\second\squared} \neq
        \sqrt{\si{\kilo\gram\meter\squared\per\second\squared}}$
\end{solution}

\begin{question}[ID=DA22,topic=measurement,difficulty=easy]
    Show why the following equation cannot be correct.
    \begin{equation*}
        v = v_0 + at^2 
    \end{equation*}
\end{question}
\begin{solution}
    $\si{\meter\per\second} \neq \si{\meter}$
\end{solution}

\begin{question}[ID=DA23,topic=measurement,difficulty=easy]
    Show why the following equation cannot be correct.
    \begin{equation*}
        ma = v^2 \\
    \end{equation*}
\end{question}
\begin{solution}
    $\si{\kilo\gram\meter\per\second\squared} \neq \si{\meter\squared\per\second\squared}$
\end{solution}

\begin{question}[ID=DA24,topic=measurement,difficulty=easy]
    Show why the following equation cannot be correct.
    \begin{equation*}
        mv = \frac{1}{2}mv^2 \\
    \end{equation*}
\end{question}
\begin{solution}
    $\si{\kilo\gram\meter\per\second} \neq \si{\kilo\gram\meter\squared\per\second\squared}$
\end{solution}

\endinput

