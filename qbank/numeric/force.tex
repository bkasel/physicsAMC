
%% This defines the lua function mycommand()
%\directlua{require('./qbank/lua/common.lua')}


%% Numeric Difficulty A
%%--------------------------------------
\element{numeric}{
\begin{questionmultx}{force-A-Q01}
\luaexec{
    %% Question
    local Q = [[
        A \string\SI{\%0.2f}{\string\kilo\string\gram}
            mass and a \string\SI{\%0.2f}{\string\kilo\string\gram}
            mass are attached to a lightweight cord that passes over
            a frictionless pulley.
        The hanging masses are free to move.
        What is the magnitude of acceleration for the larger mass divided by
            \string\SI{1}{\string\meter\string\per\string\second\string\squared}?
        [$g=\string\SI{9.80665}{\string\meter\string\per\string\second\string\squared}$]
    ]]
    %% Random Values
    local m1 = 1 + round(10 * math.random(),2)
    local m2 = 1 + round(10 * math.random(),2)
    local g  = 9.80665
    local ans = math.abs( g * (m1-m2) / (m1+m2) )
    %% Print Question
    tex.print( string.format(Q, m1, m2))
    %% Print AMCnumeric
    options = [[ digits=4, decimals=3, sign=True, approx=1 ]]
    tex.print( BeginCenter )
    tex.print( string.format(AMCnumeric, ans, options) )
    tex.print( EndCenter )
}
\end{questionmultx}
}

\element{numeric}{
\begin{questionmultx}{force-A-Q02}
\luaexec{
    %% Question
    local Q = [[
        A \string\SI{\%0.2f}{\string\kilo\string\gram}
            mass on a frictionless level surface is attached
            to a lightweight cord that passes over a pulley
            at the edge of the table and is connected to a
            \string\SI{\%0.2f}{\string\kilo\string\gram}
            hanging mass.
        What is the magnitude of the acceleration for the larger mass divided by
            \string\SI{1}{\string\meter\string\per\string\second\string\squared}?
        [$g=\string\SI{9.80665}{\string\meter\string\per\string\second\string\squared}$]
    ]]
    %% Random Values
    local m1 = 1 + round(10 * math.random(),2)
    local m2 = 1 + round(10 * math.random(),2)
    local g  = 9.80665
    local ans = g * m2 / (m1+m2)
    %% Print Question
    tex.print( string.format(Q, m1, m2))
    %% Print AMCnumeric
    options = [[ digits=4, decimals=3, sign=True, approx=1 ]]
    tex.print( BeginCenter )
    tex.print( string.format(AMCnumeric, ans, options) )
    tex.print( EndCenter )
}
\end{questionmultx}
}


%% Numeric Difficulty B
%%--------------------------------------
\element{numeric}{
\begin{questionmultx}{force-B-Q01}
\luaexec{
    %% Question
    local Q = [[
        An elevator filled to capacity has a mass of
            \string\SI{\%d}{\string\kilo\string\gram}.
        The elevator accelerates uniformly upward from
            rest for \string\SI{\%0.1f}{\string\second}
            until it reaches a speed of
            \string\SI{\%0.1f}{\string\meter\string\per\string\second}.
        The elevator is supported by a cable.
        What is the tension in the cable when the elevator
            is accelerating upward divided by
            \string\SI{1}{\string\kilo\string\newton}?
        [$g=\string\SI{9.80665}{\string\meter\string\per\string\second\string\squared}$]
    ]]
    %% Random Values
    local mas = math.random(1500,2500)
    local tim = 1 + round(3 * math.random(),1)
    local vel = 1 + round(4 * math.random(),1)
    local g   = 9.80665
    local f = (mas*g) + (mas * vel / tim)
    local ans = f / 1000
    %% Print Question
    tex.print( string.format(Q, mas, tim, vel))
    %% Print AMCnumeric
    options = [[ digits=4, decimals=2, sign=True, approx=1 ]]
    tex.print( BeginCenter )
    tex.print( string.format(AMCnumeric, ans, options) )
    tex.print( EndCenter )
}
\end{questionmultx}
}


\begin{comment}
\element{numeric}{
\begin{questionmultx}{force-B-Q02}
\luaexec{
    %% Question
    local Q = [[
        The parachute on a race car that weighs
            \string\SI{\%d}{\string\newton}
            opens at the end of a very exciting
            quarter-mile run when the car is traveling
            \string\SI{\%d}{\string\meter\string\per\string\second}.
        What net retarding force divided by
            \string\SI{1}{\string\newton}
            must be suppied by the parachute
            to stop the car in a distance of
            \string\SI{\%d}{\string\meter}?
        [$g=\string\SI{9.80665}{\string\meter\string\per\string\second\string\squared}$]
    ]]
    %% Random Values
    local w = math.random(1500,2500)
    local v = math.random(25,55)
    local d = math.random(800,1200)
    local g = 9.80665
    local ans = w*v*v / g / 2 / d
    %% Print Question
    tex.print( string.format(Q, w, v, d) )
    %% Print AMCnumeric
    options = [[ digits=4, decimals=1, sign=True, approx=1 ]]
    tex.print( BeginCenter )
    tex.print( string.format(AMCnumeric, ans, options) )
    tex.print( EndCenter )
}
\end{questionmultx}
}


\element{numeric}{
\begin{questionmultx}{force-B-Q03}
\luaexec{
    %% Question
    local Q = [[
        A race car has a mass of
            \string\SI{\%d}{\string\kilo\string\gram}.
        It starts from rest and travels
            \string\SI{\%d}{\string\meter}
            in \string\SI{\%0.1f}{\string\second}.
        The car is uniformly accelerated during the entire time.
        What net force is applied to the race car divided by
            \string\SI{1}{\string\kilo\string\newton}?
    ]]
    %% Random Values
    local m = math.random(500,1000)
    local d = math.random(20,50)
    local t = 1 + round(3 * math.random(),1)
    local ans = 2 * m * d / t / t / 1000
    %% Print Question
    tex.print( string.format(Q, m, d, t) )
    %% Print AMCnumeric
    options = [[ digits=4, decimals=2, sign=True, approx=1 ]]
    tex.print( BeginCenter )
    tex.print( string.format(AMCnumeric, ans, options) )
    tex.print( EndCenter )
}
\end{questionmultx}
}
\end{comment}


\element{numeric}{
\begin{questionmultx}{force-B-Q04}
\luaexec{
    %% Question
    local Q = [[
        A \string\SI{\%d}{\string\kilo\string\gram}
            mass starts from rest and slides down
            an inclined plane \string\SI{\%0.2f}{\string\meter}
            long in \string\SI{\%0.2f}{\string\second}.
        What net force is acting on the mass along the incline
            divided by \string\SI{1}{\string\newton}?
    ]]
    %% Random Values
    local m = math.random(1,10)
    local d = round(2*math.random(),2)
    local t = round(2*math.random(),2)
    local ans = 2 * m * d / t / t
    %% Print Question
    tex.print( string.format(Q, m, d, t))
    %% Print AMCnumeric
    options = [[ digits=4, decimals=2, sign=True, approx=1 ]]
    tex.print( BeginCenter )
    tex.print( string.format(AMCnumeric, ans, options) )
    tex.print( EndCenter )
}
\end{questionmultx}
}


\element{numeric}{
\begin{questionmultx}{force-B-Q05}
\luaexec{
    %% Question
    local Q = [[
        In an emergency stop to avoid an accident,
            a shoulder-strap seat-belt held a
            \string\SI{\%d}{\string\kilo\string\gram}
            passenger in place.
        The car was initially traveling at
            \string\SI{\%d}{\string\kilo\string\meter\string\per\string\hour}
            and came to a stop in \string\SI{\%0.1f}{\string\second}.
        What was the average force applied to the passenger by the seat-belt
            divided by \string\SI{1}{\string\kilo\string\newton}?
    ]]
    %% Random Values
    local m = math.random(40,120)
    local v = math.random(40,120)
    local t = 1 + round(4*math.random(),1)
    local ans = m * 1000 * v / 3600 / t
    %% Print Question
    tex.print( string.format(Q, m, v, t))
    %% Print AMCnumeric
    options = [[ digits=4, decimals=3, sign=True, approx=1 ]]
    tex.print( BeginCenter )
    tex.print( string.format(AMCnumeric, ans, options) )
    tex.print( EndCenter )
}
\end{questionmultx}
}

\endinput

