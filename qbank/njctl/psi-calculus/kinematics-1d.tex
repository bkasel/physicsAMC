

%% NJCTL: PSI AP Physics C
%%----------------------------------------


%% Kinematics 1D
%%----------------------------------------
\element{njctl}{
\begin{question}{kinematics-1d-q01}
    In the absence of air resistance,
        a ball dropped near the surface of the Earth experiences a constant acceleration of \SI{9.8}{\meter\per\second\squared}.
    This means that the:
    \begin{choices}
      \correctchoice{speed of the object increases \SI{9.8}{\meter\per\second} during each second}
        \wrongchoice{speed of the object as it falls is \SI{9.8}{\meter\per\second}}
        \wrongchoice{object falls \SI{9.8}{\meter} during each second}
        \wrongchoice{object falls \SI{9.8}{\meter} during the first second only}
        \wrongchoice{derivative of the distance with respect to time for the object equals \SI{9.8}{\meter\per\second\squared}}
    \end{choices}
\end{question}
}

\element{njctl}{
\begin{question}{kinematics-1d-q02}
    A \SI{1000}{\kilo\gram} truck accelerates uniformly from rest,
        reaching a speed of \SI{50}{\meter\per\second} in \SI{10}{\second}.
    During the \SI{10}{\second},
        the truck has traveled a distance of
    \begin{multicols}{2}
    \begin{choices}
        \wrongchoice{\SI{85}{\meter}}
        \wrongchoice{\SI{175}{\meter}}
        \wrongchoice{\SI{200}{\meter}}
      \correctchoice{\SI{250}{\meter}}
        \wrongchoice{\SI{300}{\meter}}
    \end{choices}
    \end{multicols}
\end{question}
}

\element{njctl}{
\begin{question}{kinematics-1d-q03}
    The velocity as a function of time of a moving object is presented by the graph below.
    \begin{center}
    \begin{tikzpicture}
        \begin{axis}[
            axis y line=left,
            axis x line=middle,
            axis line style={->},
            xlabel={time},
            xtick=\empty,
            ylabel={velocity},
            ytick=\empty,
            xmin=0,xmax=12,
            ymin=-6,ymax=6,
            width=0.8\columnwidth,
            height=0.5\columnwidth,
            grid=major,
            very thin,
        ]
        \addplot[line width=1pt,mark=\empty] plot coordinates {(0,-5) (4,-5) (8,5) (12,5)};
        \end{axis}
    \end{tikzpicture}
    \end{center}
    The object moves in one dimension and at time $t=0$, $x_0=0$.
    Which of the following graphs best represents the displacement as a function of time for the same time intervals?
    \begin{multicols}{2}
    \begin{choices}
        \AMCboxDimensions{down=-1.5em}
        \wrongchoice{
            \begin{tikzpicture}
                \begin{axis}[
                    axis y line=left,
                    axis x line=middle,
                    axis line style={->},
                    xlabel={time},
                    xtick=\empty,
                    ylabel={position},
                    ytick=\empty,
                    xmin=0,xmax=12,
                    ymin=-6,ymax=6,
                    width=0.95\columnwidth,
                    very thin,
                ]
                \addplot[line width=1pt,mark=\empty] plot coordinates {(0,0) (4,0) (6,5) (8,0) (12,0)};
                \end{axis}
            \end{tikzpicture}
        }
        \wrongchoice{
            \begin{tikzpicture}
                \begin{axis}[
                    axis y line=left,
                    axis x line=middle,
                    axis line style={->},
                    xlabel={time},
                    xtick=\empty,
                    ylabel={position},
                    ytick=\empty,
                    xmin=0,xmax=12,
                    ymin=-6,ymax=6,
                    width=0.95\columnwidth,
                    very thin,
                ]
                \addplot[line width=1pt,mark=\empty] plot coordinates {(0,0) (2,5) (4,0) (8,0) (10,-5) (12,0)};
                \end{axis}
            \end{tikzpicture}
        }
        \wrongchoice{
            \begin{tikzpicture}
                \begin{axis}[
                    axis y line=left,
                    axis x line=middle,
                    axis line style={->},
                    xlabel={time},
                    xtick=\empty,
                    ylabel={position},
                    ytick=\empty,
                    xmin=0,xmax=12,
                    ymin=-6,ymax=6,
                    width=0.95\columnwidth,
                    very thin,
                ]
                \addplot[line width=1pt,domain=0:4] {0};
                \addplot[line width=1pt,domain=4:8] {(x-4) *(x-8)};
                \addplot[line width=1pt,domain=8:12] {0};
                \end{axis}
            \end{tikzpicture}
        }
        %% ANS is D
        \correctchoice{
            \begin{tikzpicture}
                \begin{axis}[
                    axis y line=left,
                    axis x line=middle,
                    axis line style={->},
                    xlabel={time},
                    xtick=\empty,
                    ylabel={position},
                    ytick=\empty,
                    xmin=0,xmax=12,
                    ymin=-6,ymax=6,
                    width=0.95\columnwidth,
                    very thin,
                ]
                \addplot[line width=1pt,domain=0:4] {-x};
                \addplot[line width=1pt,domain=4:8] {-4 + 0.3*(x-4) *(x-8)};
                \addplot[line width=1pt,domain=8:12] {-4+(x-8)};
                \end{axis}
            \end{tikzpicture}
        }
        \wrongchoice{
            \begin{tikzpicture}
                \begin{axis}[
                    axis y line=left,
                    axis x line=middle,
                    axis line style={->},
                    xlabel={time},
                    xtick=\empty,
                    ylabel={position},
                    ytick=\empty,
                    xmin=0,xmax=12,
                    ymin=-6,ymax=6,
                    width=0.95\columnwidth,
                    very thin,
                ]
                \addplot[line width=1pt,mark=\empty] plot coordinates {(0,0) (4,-5) (8,-5) (12,0)};
                \end{axis}
            \end{tikzpicture}
        }
    \end{choices}
    \end{multicols}
\end{question}
}

\newcommand{\njctlapcKinematicsOneDQFour}{
\begin{tikzpicture}
    \begin{axis}[
        axis y line=left,
        axis x line=bottom,
        axis line style={->},
        ytick={0,5,10},
        yticklabels={,$v_0$,$v$},
        xlabel={time},
        x unit=\si{\second},
        xtick={0,5,10,15,20},
        xmin=0,xmax=21,
        ymin=0,ymax=11,
        width=0.8\columnwidth,
        height=0.5\columnwidth,
        very thin,
    ]
    \addplot[line width=1pt,mark=\empty] plot coordinates {(0,0) (20,10)} node[pos=0.10,anchor=south] {$B$};
    \addplot[line width=1pt,mark=\empty] plot coordinates {(0,5) (20,5)} node[pos=0.14,anchor=south east] {$A$};;
    \end{axis}
\end{tikzpicture}
}

\element{njctl}{
\begin{question}{kinematics-1d-q04}
    At time $t=\text{zero}$, car $A$ traveling with speed $v_0$ passes car $B$,
        which is just starting to move.
    Both cars then travel on two parallel lanes of the same straight road.
    The graphs of speed versus time for both cars are shown below.
    \begin{center}
        \njctlapcKinematicsOneDQFour
    \end{center}
    %% Start question
    Which of the following is true at time $t=\SI{10}{\second}$?
    \begin{choices}
      \correctchoice{Car $B$ is behind car $A$.}
        \wrongchoice{Car $B$ is passing car $A$.}
        \wrongchoice{Car $B$ is in front of car $A$.}
        \wrongchoice{Both cars have the same acceleration.}
        \wrongchoice{Car $A$ is accelerating faster then car $B$.}
    \end{choices}
\end{question}
}

\element{njctl}{
\begin{question}{kinematics-1d-q05}
    At time $t=0$, car $A$ traveling with speed $v_0$ passes car $B$,
        which is just starting to move.
    Both cars then travel on two parallel lanes of the same straight road.
    The graphs of speed versus time for both cars are shown below.
    \begin{center}
        \njctlapcKinematicsOneDQFour
    \end{center}
    %% Start question
    From time $t=0$ to time $t=\SI{20}{\second}$,
        the areas under both curves are equal.
    Therefore, which of the following is true at time $t=\SI{20}{\second}$?
    \begin{choices}
        \wrongchoice{Car $B$ is behind car $A$. }
      \correctchoice{Car $B$ is passing car $A$.}
        \wrongchoice{Car $B$ is in front of car $A$.}
        \wrongchoice{Both cars have the same acceleration.}
        \wrongchoice{Car $A$ is accelerating faster than car $B$.}
    \end{choices}
\end{question}
}

\element{njctl}{
\begin{question}{kinematics-1d-q06}
    A runner moving in the $+x$ direction passes the origin at time $t=0$.
    Between $t=0$ and $t=\SI{2}{\second}$,
    the runner has a constant speed of \SI{8}{\meter\per\second} and at $t=\SI{2}{\second}$ is accelerating at a constant rate of \SI{4}{\meter\per\second\squared} in the $-x$ direction.
    The runner location at $t=\SI{5}{\second}$ is:
    \begin{multicols}{3}
    \begin{choices}
        \wrongchoice{\SI{+14}{\meter}}
        \wrongchoice{\SI{+94}{\meter}}
      \correctchoice{\SI{+22}{\meter}}
        \wrongchoice{\SI{-74}{\meter}}
        \wrongchoice{\SI{-94}{\meter}}
    \end{choices}
    \end{multicols}
\end{question}
}

\element{njctl}{
\begin{question}{kinematics-1d-q07}
    Starting from rest,
        an object accelerates on a straight level surface at the rate of \SI{2}{\meter\per\second\squared} for \SI{10}{\second}.
    %% Start question
    What is the speed of the object at the end of \SI{10}{\second}?
    \begin{multicols}{3}
    \begin{choices}
        \wrongchoice{\SI{5}{\meter\per\second}}
        \wrongchoice{\SI{10}{\meter\per\second}}
      \correctchoice{\SI{20}{\meter\per\second}}
        \wrongchoice{\SI{30}{\meter\per\second}}
        \wrongchoice{\SI{40}{\meter\per\second}}
    \end{choices}
    \end{multicols}
\end{question}
}

\element{njctl}{
\begin{question}{kinematics-1d-q08}
    Starting from rest,
        an object accelerates on a straight level surface at the rate of \SI{2}{\meter\per\second\squared} for \SI{10}{\second}.
    %% Start question
    What is the total distance of the object during \SI{10}{\second}?
    \begin{multicols}{3}
    \begin{choices}
        \wrongchoice{\SI{20}{\meter}}
        \wrongchoice{\SI{40}{\meter}}
        \wrongchoice{\SI{80}{\meter}}
      \correctchoice{\SI{100}{\meter}}
        \wrongchoice{\SI{120}{\meter}}
    \end{choices}
    \end{multicols}
\end{question}
}

\element{njctl}{
\begin{question}{kinematics-1d-q09}
    A stomp rocket is projected from the ground level in two different trials,
        with two different initial velocities.
    It was found that the total time the rocket stays in air is $t$ in the first trial and $2t$ in the second trial.
    What is the difference between the maximum heights in this experiment?
    (Ignore air resistance)
    \begin{choices}
        \wrongchoice{The maximum height in the second trial is two times greater of that in the first trial}
      \correctchoice{The maximum height in the second trial is four times greater of that in the first trial}
        \wrongchoice{The maximum height in the second trial is eight times greater of that in the first trial}
        \wrongchoice{The maximum height in the second trial is sixteen times greater of that in the first trial}
        \wrongchoice{The maximum height is the same in both trials}
    \end{choices}
\end{question}
}

\element{njctl}{
\begin{question}{kinematics-1d-q10}
    A particle moves along the x-axis with an acceleration described by $a=6t$,
        where $a$ is in meters per second squared and $t$ is in seconds.
    If at time $t_0=0$, $x_0=0$ and $v_0=0$,
        how fast does the particle move when $t=\SI{2}{\second}$?
    \begin{multicols}{2}
    \begin{choices}
        \wrongchoice{$v=\SI{4}{\meter\per\second}$}
        \wrongchoice{$v=\SI{8}{\meter\per\second}$}
        \wrongchoice{$v=\SI{16}{\meter\per\second}$}
        \wrongchoice{$v=\SI{24}{\meter\per\second}$}
      \correctchoice{$v=\SI{12}{\meter\per\second}$}
    \end{choices}
    \end{multicols}
\end{question}
}

\element{njctl}{
\begin{question}{kinematics-1d-q11}
    A particle moves along the $x$-axis with an acceleration described by $a=6t$,
        where $a$ is in meters per second squared and $t$ is in seconds.
    If at time $t_0=0$, $x_0=0$ and $v_0=0$,
        where is the particle located when $t=\SI{2}{\second}$?
    \begin{multicols}{2}
    \begin{choices}
        \wrongchoice{$x=\SI{4}{\meter}$}
      \correctchoice{$x=\SI{8}{\meter}$}
        \wrongchoice{$x=\SI{16}{\meter}$}
        \wrongchoice{$x=\SI{24}{\meter}$}
        \wrongchoice{$x=\SI{36}{\meter}$}
    \end{choices}
    \end{multicols}
\end{question}
}

\element{njctl}{
\begin{question}{kinematics-1d-q12}
    An object starts from rest at time $t_0=0$.
    The velocity $v$ as a function of time is given by $v=2t+3t^2$.
    How far does the object travel in first \SI{5}{\second}?
    \begin{multicols}{3}
    \begin{choices}
      \correctchoice{\SI{150}{\meter}}
        \wrongchoice{\SI{300}{\meter}}
        \wrongchoice{\SI{450}{\meter}}
        \wrongchoice{\SI{500}{\meter}}
        \wrongchoice{\SI{625}{\meter}}
    \end{choices}
    \end{multicols}
\end{question}
}

\element{njctl}{
\begin{question}{kinematics-1d-q13}
    %% Questions 13-14
    An object moving in a straight line has a velocity $v$ in meters per second that varies with time $t$ in seconds according to the following function.
    \begin{equation*}
        v = 8 + 2.5 t^2
    \end{equation*}
    %% Start question
    The instantaneous acceleration of the object at $t=\SI{2}{\second}$ is:
    \begin{multicols}{3}
    \begin{choices}
        \wrongchoice{\SI{2}{\meter\per\second\squared}}
        \wrongchoice{\SI{4}{\meter\per\second\squared}}
        \wrongchoice{\SI{6}{\meter\per\second\squared}}
        \wrongchoice{\SI{8}{\meter\per\second\squared}}
      \correctchoice{\SI{10}{\meter\per\second\squared}}
    \end{choices}
    \end{multicols}
\end{question}
}

\element{njctl}{
\begin{question}{kinematics-1d-q14}
    %% Questions 13-14
    An object moving in a straight line has a velocity $v$ in meters per second that varies with time $t$ in seconds according to the following function.
    \begin{equation*}
        v = 8 + 2.5 t^2
    \end{equation*}
    %% Start question
    The displacement of the object between $t=0$ and $t=\SI{6}{\second}$ is:
    \begin{multicols}{3}
    \begin{choices}
        \wrongchoice{\SI{120}{\meter}}
        \wrongchoice{\SI{180}{\meter}}
      \correctchoice{\SI{228}{\meter}}
        \wrongchoice{\SI{242}{\meter}}
        \wrongchoice{\SI{260}{\meter}}
    \end{choices}
    \end{multicols}
\end{question}
}

\element{njctl}{
\begin{question}{kinematics-1d-q15}
    A rocket is shot vertically upward into the air with a positive initial velocity.
    Which of the following correctly describes the velocity and acceleration of the object at its maximum elevation?
    \begin{choices}
        \wrongchoice{$\text{velocity}>0$, $\text{acceleration}<0$}
        \wrongchoice{$\text{velocity}=0$, $\text{acceleration}=0$}
        \wrongchoice{$\text{velocity}<0$, $\text{acceleration}<0$}
      \correctchoice{$\text{velocity}=0$, $\text{acceleration}<0$}
        \wrongchoice{$\text{velocity}>0$, $\text{acceleration}>0$}
    \end{choices}
\end{question}
}

\element{njctl}{
\begin{question}{kinematics-1d-q16}
    %% Questions 16-17
    The acceleration of a moving object is given by the function:
    \begin{equation*}
        a = 8t + 4\, ,
    \end{equation*}
    where all quantities are in base SI units.
    The object is initially \SI{2}{\meter} from the origin and moving to the right with a speed of \SI{2}{\meter\per\second}.
    %% Start question
    Find the velocity at $t=\SI{3}{\second}$.
    \begin{multicols}{3}
    \begin{choices}
        \wrongchoice{\SI{37}{\meter\per\second}}
        \wrongchoice{\SI{43}{\meter\per\second}}
      \correctchoice{\SI{50}{\meter\per\second}}
        \wrongchoice{\SI{62}{\meter\per\second}}
        \wrongchoice{\SI{78}{\meter\per\second}}
    \end{choices}
    \end{multicols}
\end{question}
}

\element{njctl}{
\begin{question}{kinematics-1d-q17}
    %% Questions 16-17
    The acceleration of a moving object is given by the function:
    \begin{equation*}
        a = 8t + 4\, ,
    \end{equation*}
    where all quantities are in base SI units.
    The object is initially \SI{2}{\meter} from the origin and moving to the right with a speed of \SI{2}{\meter\per\second}.
    %% Start question
    Find the displacement of the object during first \SI{3}{\second}.
    \begin{multicols}{3}
    \begin{choices}
        \wrongchoice{\SI{15}{\meter}}
        \wrongchoice{\SI{20}{\meter}}
        \wrongchoice{\SI{25}{\meter}}
        \wrongchoice{\SI{30}{\meter}}
      \correctchoice{\SI{60}{\meter}}
    \end{choices}
    \end{multicols}
\end{question}
}

\element{njctl}{
\begin{question}{kinematics-1d-q18}
    An object in free fall has a velocity of \SI{5}{\meter\per\second} in the upward direction.
    What is the instantaneous velocity of the object one second later?
    \begin{multicols}{2}
    \begin{choices}
        \wrongchoice{\SI{10}{\meter\per\second} down}
        \wrongchoice{\SI{5}{\meter\per\second} up }
        \wrongchoice{\SI{10}{\meter\per\second} up }
      \correctchoice{\SI{5}{\meter\per\second} down }
        \wrongchoice{\SI{14}{\meter\per\second} down}
    \end{choices}
    \end{multicols}
\end{question}
}

\element{njctl}{
\begin{question}{kinematics-1d-q19}
    At what vertical velocity should an object be launched at in order to achieve a height of \SI{20}{\meter}?
    \begin{multicols}{3}
    \begin{choices}
        \wrongchoice{\SI{10}{\meter\per\second}}
        \wrongchoice{\SI{14}{\meter\per\second}}
      \correctchoice{\SI{20}{\meter\per\second}}
        \wrongchoice{\SI{23}{\meter\per\second}}
        \wrongchoice{\SI{29}{\meter\per\second}}
    \end{choices}
    \end{multicols}
\end{question}
}

\element{njctl}{
\begin{question}{kinematics-1d-q20}
    How long will it take for an object accelerating at a constant rate of \SI{5}{\meter\per\second\squared} to change its velocity from \SI{0}{\meter\per\second} to \SI{6}{\meter\per\second}?
    \begin{multicols}{3}
    \begin{choices}
        \wrongchoice{\SI{0.6}{\second}}
      \correctchoice{\SI{1.2}{\second}}
        \wrongchoice{\SI{2.4}{\second}}
        \wrongchoice{\SI{3.6}{\second}}
        \wrongchoice{\SI{4.8}{\second}}
    \end{choices}
    \end{multicols}
\end{question}
}

\element{njctl}{
\begin{question}{kinematics-1d-q21}
    An object is thrown straight downward on Earth with an initial velocity of \SI{10}{\meter\per\second} from a position of \SI{20}{\meter} above the ground.
    The speed of the object when it reaches the ground is about:
    \begin{multicols}{3}
    \begin{choices}
        \wrongchoice{\SI{13.2}{\meter\per\second}}
        \wrongchoice{\SI{16.1}{\meter\per\second}}
        \wrongchoice{\SI{19.5}{\meter\per\second}}
      \correctchoice{\SI{22.2}{\meter\per\second}}
        \wrongchoice{\SI{31.7}{\meter\per\second}}
    \end{choices}
    \end{multicols}
\end{question}
}

\element{njctl}{
\begin{question}{kinematics-1d-q22}
    A ball is dropped from a height $h$ without air resistance.
    If the ball falls a distance of $h/2$ in a time $t$,
        how much time is required to fall the remaining $h/2$?
    \begin{multicols}{3}
    \begin{choices}
        \wrongchoice{$0.25 t$}
      \correctchoice{$0.40 t$}
        \wrongchoice{$0.50 t$}
        \wrongchoice{$0.71 t$}
        \wrongchoice{$1.00 t$}
    \end{choices}
    \end{multicols}
\end{question}
}

\element{njctl}{
\begin{question}{kinematics-1d-q23}
    The following function represents the path taken by an object:
    \begin{equation*}
        x=4t^2 - 3t + 2 \, ,
    \end{equation*}
    where $x$ is the object's position in meters and $t$ is time in seconds.
    How far does the object travel during first \SI{3}{\second}?
    \begin{multicols}{3}
    \begin{choices}
        \wrongchoice{\SI{16}{\meter}}
        \wrongchoice{\SI{23}{\meter}}
        \wrongchoice{\SI{29}{\meter}}
      \correctchoice{\SI{27}{\meter}}
        \wrongchoice{\SI{32}{\meter}}
    \end{choices}
    \end{multicols}
\end{question}
}

\element{njctl}{
\begin{question}{kinematics-1d-q24}
    The following function represents the path taken by an object:
    \begin{equation*}
        x=4t^2 - 3t + 2 \, ,
    \end{equation*}
    where $x$ is the object's position in meters and $t$ is time in seconds.
    How far away from the origin is the object at $t=\SI{3}{\second}$?
    \begin{multicols}{3}
    \begin{choices}
        \wrongchoice{\SI{8}{\meter}}
        \wrongchoice{\SI{12}{\meter}}
        \wrongchoice{\SI{16}{\meter}}
        \wrongchoice{\SI{20}{\meter}}
      \correctchoice{\SI{29}{\meter}}
    \end{choices}
    \end{multicols}
\end{question}
}

\element{njctl}{
\begin{question}{kinematics-1d-q25}
    The following function represents the path taken by an object:
    \begin{equation*}
        x = 4t^2 - 3t + 2\, ,
    \end{equation*}
    where $x$ is the object's position in meters and $t$ is time in seconds.
    What is the instantaneous velocity of the object at $t=\SI{3}{\meter}$?
    \begin{multicols}{3}
    \begin{choices}
      \correctchoice{\SI{21}{\meter\per\second}}
        \wrongchoice{\SI{18}{\meter\per\second}}
        \wrongchoice{\SI{23}{\meter\per\second}}
        \wrongchoice{\SI{28}{\meter\per\second}}
        \wrongchoice{\SI{33}{\meter\per\second}}
    \end{choices}
    \end{multicols}
\end{question}
}

\element{njctl}{
\begin{question}{kinematics-1d-q26}
    The following function represents the path taken by an object:
    \begin{equation*}
        x = 4t^2 - 3t + 2\, ,
    \end{equation*}
    where $x$ is the object's position in meters and $t$ is time in seconds.
    At what time is the velocity of the object about \SI{29}{\meter\per\second}?
    \begin{multicols}{3}
    \begin{choices}
        \wrongchoice{\SI{2}{\second}}
      \correctchoice{\SI{4}{\second}}
        \wrongchoice{\SI{5}{\second}}
        \wrongchoice{\SI{6}{\second}}
        \wrongchoice{\SI{8}{\second}}
    \end{choices}
    \end{multicols}
\end{question}
}

\element{njctl}{
\begin{question}{kinematics-1d-q27}
    The following function represents the path taken by an object:
    \begin{equation*}
        x = 4t^2 - 3t + 2\, ,
    \end{equation*}
    where $x$ is the object's position in meters and $t$ is time in seconds.
    What is the instantaneous acceleration of the object at $t=\SI{3}{\second}$?
    \begin{multicols}{3}
    \begin{choices}
      \correctchoice{\SI{8}{\meter\per\second\squared}}
        \wrongchoice{\SI{12}{\meter\per\second\squared}}
        \wrongchoice{\SI{20}{\meter\per\second\squared}}
        \wrongchoice{\SI{24}{\meter\per\second\squared}}
        \wrongchoice{\SI{30}{\meter\per\second\squared}}
    \end{choices}
    \end{multicols}
\end{question}
}

\element{njctl}{
\begin{question}{kinematics-1d-q28}
    The following function represents the path taken by an object:
    \begin{equation*}
        x = 4t^2 - 3t + 2\, ,
    \end{equation*}
    where $x$ is the object's position in meters and $t$ is time in seconds.
    At what time is the acceleration of the object \SI{10}{\meter\per\second\squared}?
    \begin{multicols}{2}
    \begin{choices}
        \wrongchoice{\SI{0}{\second}}
        \wrongchoice{\SI{2}{\second}}
        \wrongchoice{\SI{4}{\second}}
        \wrongchoice{\SI{8}{\second}}
      \correctchoice{none are correct}
    \end{choices}
    \end{multicols}
\end{question}
}

\element{njctl}{
\begin{question}{kinematics-1d-q29}
    The acceleration of a moving object is represented by the function
    \begin{equation*}
        a = 9t^2 +2 \, .
    \end{equation*}
    where $a$ is the object's acceleration in meter per second squared and $t$ is time in seconds.
    At $t=0$ the object at $x_0=\SI{-2}{\meter}$ and is traveling with a velocity of \SI{3}{\meter\per\second} in the negative $x$ direction.
    How fast is the object moving at $t=\SI{2}{\second}$?
    \begin{multicols}{3}
    \begin{choices}
        \wrongchoice{\SI{13}{\meter\per\second}}
      \correctchoice{\SI{25}{\meter\per\second}}
        \wrongchoice{\SI{38}{\meter\per\second}}
        \wrongchoice{\SI{47}{\meter\per\second}}
        \wrongchoice{\SI{65}{\meter\per\second}}
    \end{choices}
    \end{multicols}
\end{question}
}

\element{njctl}{
\begin{question}{kinematics-1d-q30}
    The acceleration of a moving object is represented by the function
    \begin{equation*}
        a = 9t^2 +2 \, .
    \end{equation*}
    where $a$ is the object's acceleration in meter per second squared and $t$ is time in seconds.
    At $t=0$ the object at $x_0=\SI{-2}{\meter}$ and is traveling with a velocity of \SI{3}{\meter\per\second} in the negative $x$ direction.
    What is the instantaneous acceleration at $t=\SI{3}{\second}$?
    \begin{multicols}{3}
    \begin{choices}
        \wrongchoice{\SI{9}{\meter\per\second\squared}}
        \wrongchoice{\SI{19}{\meter\per\second\squared}}
        \wrongchoice{\SI{41}{\meter\per\second\squared}}
        \wrongchoice{\SI{54}{\meter\per\second\squared}}
      \correctchoice{\SI{83}{\meter\per\second\squared}}
    \end{choices}
    \end{multicols}
\end{question}
}

\element{njctl}{
\begin{question}{kinematics-1d-q31}
    The acceleration of a moving object is represented by the function
    \begin{equation*}
        a = 9t^2 +2 \, .
    \end{equation*}
    where $a$ is the object's acceleration in meter per second squared and $t$ is time in seconds.
    At $t=0$ the object at $x_0=\SI{-2}{\meter}$ and is traveling with a velocity of \SI{3}{\meter\per\second} in the negative $x$ direction.
    At what time is the acceleration approximately \SI{38}{\meter\per\second\squared}?
    \begin{multicols}{2}
    \begin{choices}
        \wrongchoice{\SI{1}{\second}}
      \correctchoice{\SI{2}{\second}}
        \wrongchoice{\SI{3}{\second}}
        \wrongchoice{\SI{4}{\second}}
        \wrongchoice{none (acceleration is constant)}
    \end{choices}
    \end{multicols}
\end{question}
}

\element{njctl}{
\begin{question}{kinematics-1d-q32}
    The acceleration of a moving object is represented by the function
    \begin{equation*}
        a = 9t^2 +2 \, .
    \end{equation*}
    where $a$ is the object's acceleration in meter per second squared and $t$ is time in seconds.
    At $t=0$ the object at $x_0=\SI{-2}{\meter}$ and is traveling with a velocity of \SI{3}{\meter\per\second} in the negative $x$ direction.
    How far from the origin is the object at $t=\SI{2}{\second}$?
    \begin{multicols}{3}
    \begin{choices}
      \correctchoice{\SI{8}{\meter}}
        \wrongchoice{\SI{4}{\meter}}
        \wrongchoice{\SI{6}{\meter}}
        \wrongchoice{\SI{12}{\meter}}
        \wrongchoice{\SI{17}{\meter}}
    \end{choices}
    \end{multicols}
\end{question}
}

%\newcommand{\njctlapcKinematicsOneDQThirtyThree}{
%\begin{tikzpicture}
%    %% NOTE:
%\end{tikzpicture}
%}
%
%\element{njctl}{
%\begin{question}{kinematics-1d-q33}
%    %% Question 33 – 34 tikz
%    A car starting from rest uniformly accelerates in a straight line as shown below
%    \begin{center}
%        \njctlapcKinematicsOneDQThirtyThree
%    \end{center}
%    What was the acceleration during the entire \SI{3}{\second} interval?
%    \begin{multicols}{2}
%    \begin{choices}
%        \wrongchoice{\SI{1}{\meter\per\second\squared}}
%        \wrongchoice{\SI{3}{\meter\per\second\squared}}
%      \correctchoice{\SI{4}{\meter\per\second\squared}}
%        \wrongchoice{\SI{6}{\meter\per\second\squared}}
%        \wrongchoice{\SI{12}{\meter\per\second\squared}}
%    \end{choices}
%    \end{multicols}
%\end{question}
%}
%
%\element{njctl}{
%\begin{question}{kinematics-1d-q34}
%    %% Question 33 – 34 tikz
%    A car starting from rest uniformly accelerates in a straight line as shown below
%    \begin{center}
%        \njctlapcKinematicsOneDQThirtyThree
%    \end{center}
%    What was the average velocity of the car during the first two seconds?
%    \begin{multicols}{2}
%    \begin{choices}
%        \wrongchoice{\SI{2}{\meter\per\second}}
%      \correctchoice{\SI{4}{\meter\per\second}}
%        \wrongchoice{\SI{6}{\meter\per\second}}
%        \wrongchoice{\SI{8}{\meter\per\second}}
%        \wrongchoice{\SI{10}{\meter\per\second}}
%    \end{choices}
%    \end{multicols}
%\end{question}
%}

\element{njctl}{
\begin{question}{kinematics-1d-q35}
    The velocity as a function of time of a moving object along a straight line is presented by the graph below.
    \begin{center}
    \begin{tikzpicture}
        %% NOTE:
        \begin{axis}[
            axis y line=left,
            axis x line=middle,
            axis line style={->},
            xlabel={time},
            xtick=\empty,
            ylabel={velocity},
            ytick=\empty,
            xmin=0,xmax=16,
            ymin=-6,ymax=6,
            width=0.8\columnwidth,
            height=0.5\columnwidth,
            grid=major,
            very thin,
        ]
        \addplot[line width=1pt,mark=\empty] plot coordinates {(0,-4) (3,-4) (6,5) (10,5)};
        \end{axis}
    \end{tikzpicture}
    \end{center}
    For which of the following sections of the graph the does object move at nonzero constant acceleration?
    \begin{multicols}{3}
    \begin{choices}
        \wrongchoice{$AB$}
      \correctchoice{$BC$}
        \wrongchoice{$CD$}
        \wrongchoice{$DE$}
        \wrongchoice{$EF$}
    \end{choices}
    \end{multicols}
\end{question}
}

\element{njctl}{
\begin{question}{kinematics-1d-q36}
    An object is dropped from a height $H$ and it reaches the ground with a velocity $v$.
    The same object is thrown down from the same height $H$ with the initial velocity $v$.
    When it reaches the ground it has a velocity $v_2$.
    \begin{center}
    \begin{tikzpicture}
        %% NOTE:
    \end{tikzpicture}
    \end{center}
    Which of the following is correct relationship between $v_2$ and $v$?
    \begin{multicols}{3}
    \begin{choices}
        \wrongchoice{$v_2=2 v$}
        \wrongchoice{$v_2=4 v$}
        \wrongchoice{$v_2=v$}
      \correctchoice{$v_2=\sqrt{2} v$}
        \wrongchoice{$v_2=\sqrt{3} v$}
    \end{choices}
    \end{multicols}
\end{question}
}

\newcommand{\njctlapcKinematicsOneDQThirtySeven}{
\begin{tikzpicture}
    %% NOTE:
\end{tikzpicture}
}

\element{njctl}{
\begin{question}{kinematics-1d-q37}
    A ball is dropped from a height $H$ and a second ball is thrown from the ground with an initial velocity $v_0$.
    \begin{center}
        \njctlapcKinematicsOneDQThirtySeven
    \end{center}
    At what do time the balls pass each other?
    \begin{multicols}{3}
    \begin{choices}
        \wrongchoice{$\sqrt{\dfrac{H}{2v_0}}$}
        \wrongchoice{$\sqrt{\dfrac{2H}{v_0}}$}
        \wrongchoice{$\dfrac{2H}{v_0}$}
        \wrongchoice{$\dfrac{H}{4v_0}$}
      \correctchoice{$\dfrac{H}{v_0}$}
    \end{choices}
    \end{multicols}
\end{question}
}

\element{njctl}{
\begin{question}{kinematics-1d-q38}
    A ball is dropped from a height $H$ and a second ball is thrown from the ground with an initial velocity $v_0$.
    \begin{center}
        \njctlapcKinematicsOneDQThirtySeven
    \end{center}
    How fast does the first ball move when they pass each other?
    \begin{multicols}{3}
    \begin{choices}
        \wrongchoice{$\sqrt{\dfrac{gH}{2v_0}}$}
        \wrongchoice{$\sqrt{\dfrac{2gH}{v_0}}$}
        \wrongchoice{$\dfrac{2gH}{v_0}$}
      \correctchoice{$\dfrac{gH}{v_0}$}
        \wrongchoice{$\dfrac{gH}{4v_0}$}
    \end{choices}
    \end{multicols}
\end{question}
}

\element{njctl}{
\begin{question}{kinematics-1d-q39}
    The velocity in meters per second as a function of time
        of a moving particle is given:
    \begin{equation*}
        v = \alpha + \beta t^2 \, ,
    \end{equation*}
    where $\alpha$ and $\beta$ are constants and $t$ is time in seconds.
    How far does the particle move during the first \SI{3}{\second}?
    \begin{multicols}{2}
    \begin{choices}
        \wrongchoice{$3\alpha + 3\beta$}
      \correctchoice{$3\alpha + 9\beta$}
        \wrongchoice{$9\alpha + 3\beta$}
        \wrongchoice{$3\alpha + 27\beta$}
        \wrongchoice{$9\alpha + 27\beta$}
    \end{choices}
    \end{multicols}
\end{question}
}

\element{njctl}{
\begin{question}{kinematics-1d-q40}
    The velocity in meters per second as a function of time
        of a moving particle is given:
    \begin{equation*}
        v = \alpha + \beta t^2 \, ,
    \end{equation*}
    where $\alpha$ and $\beta$ are constants and $t$ is time in seconds.
    What is the acceleration of the particle at \SI{3}{\second}?
    \begin{multicols}{3}
    \begin{choices}
        \wrongchoice{$3\beta$}
        \wrongchoice{$9\beta$}
      \correctchoice{$6\beta$}
        \wrongchoice{$27\beta$}
        \wrongchoice{$18\beta$}
    \end{choices}
    \end{multicols}
\end{question}
}


\endinput


