
%%--------------------------------------------------
%% ACT: English Test
%%--------------------------------------------------


%% DIRECTIONS:
%%--------------------------------------------------

% In the five passages that follow, certain words and phrases are underlined and numbered. 
% In the right-hand column, you will find alternatives for the underlined part. 
% In most cases, you are to choose the one that best expresses the idea, makes the statement appropriate for standard written English, or is worded most consistently with the style and tone of the passage as a whole. 
% If you think the original version is best, choose ``NO CHANGE.''
% In some cases, you will find in the right-hand column a question about the underlined part. 
% You are to choose the best answer to the question.  
% You will also find questions about a section of the passage, or about the passage as a whole. 
% These questions do not refer to an underlined portion of the passage, but rather are identified by a number or numbers in a box.
% For each question, choose the alternative you consider best and fill in the corresponding oval on your answer document. 
% Read each passage through once before you begin to answer the questions that accompany it. 
% For many of the questions, you must read several sentences beyond the question to determine the answer. 
% Be sure that you have read far enough ahead each time you choose an alternative.



%% 2005--2006: ACT Multiple Choice Questions
%%--------------------------------------------------

%% page 14
\element{act-mc}{

\begin{figure}
    PASSAGE I
    Notes from Underground

    A lot of people hate to ride the New York City subways, but I love them because I like to get places fast.  
    A musician balancing a cello case, two Buddhist monks in saffron robes, and a group of stockbrokers in crisp,
    charcoal gray suits get on at Wall Street. 
    A passenger placidly sews while the subway train flings and jolts. 
    A teenager whose holding a shoebox containing a kitten as tiny as a gingersnap smiles even if a line of girls in frilly white communion dresses file by. 
    About three and a half million people a day ride the subways I think maybe I might possibly have met them all.
\end{figure}

\begin{question}{english-q01}
    At this point, the writer wants to provide one reason why she likes to ride the subways. 
    Which choice is most relevant to the information provided in this first paragraph?
    \begin{choices}
        \wrongchoice{NO CHANGE}
        \wrongchoice{I never know what I’ll see there.}
        \wrongchoice{they are so much cheaper than taxis.}
        \wrongchoice{they are places of enormous quiet and calm.}
    \end{choices}
\end{question}

\begin{question}{english-q02}
    \begin{choices}
        \wrongchoice{NO CHANGE}
        \wrongchoice{charcoal gray suits,}
        \wrongchoice{charcoal, gray suits}
        \wrongchoice{charcoal gray, suits}
    \end{choices}
\end{question}

\begin{question}{english-q03}
    \begin{choices}
        \wrongchoice{NO CHANGE}
        \wrongchoice{thats}
        \wrongchoice{as}
        \wrongchoice{who’s}
    \end{choices}
\end{question}

\begin{question}{english-q04}
    \begin{choices}
        \wrongchoice{NO CHANGE}
        \wrongchoice{as}
        \wrongchoice{whereas}
        \wrongchoice{such that}
    \end{choices}
\end{question}

\begin{question}{english-q05}
    \begin{choices}
        \wrongchoice{NO CHANGE}
        \wrongchoice{subways, and}
        \wrongchoice{subways, which}
        \wrongchoice{subways actually}
    \end{choices}
\end{question}

\begin{question}{english-q06}
    \begin{choices}
        \wrongchoice{NO CHANGE}
        \wrongchoice{perhaps I've}
        \wrongchoice{I've possibly}
        \wrongchoice{I've}
    \end{choices}
\end{question}
}

%% page 15
\element{act-mc}{
\begin{figure}
    Sometimes a Salvation Army volunteer boards the subway train with sandwiches and juice to give to the needy. 
    ``Put your pride to the side!'' the volunteer shouts, and I’ve seen many people put out their hands. 
    The speaker also raises money. 
    Its impossible to predict which people will dig into their pockets or if they were to open their purses, and I've stopped trying to guess.

    Last week some fellow passengers and I watched an elderly man with a portable chessboard playing chess against himself. 
    Just yesterday I sat across the aisle with a woman who was composing music in pink-tinted glasses in a notebook. 
    She tapped her foot as she reviewed what she'd written and then stopped tapping and jotted more notes as the train hurtled along.

    Today is my mother's birthday. 
    I decided to surprise her with lilac blooms from my backyard, so this morning, carrying a shopping bag full of the flowers, I boarded a crowded ``E'' train and rode it to the very last stop in the Bronx. 
    Strangers smiled and took pains not to crush the flowers, even when the train jerked to a halt.
    I got off at an elevated station and, lifting the splendid bouquet, rushed down to my mother, feeling delighted that I'd brought the blooms all the way from Brooklyn on the subway train.
\end{figure}

\begin{question}{english-q07}
    \begin{choices}
        \wrongchoice{NO CHANGE}
        \wrongchoice{Therefore, the}
        \wrongchoice{In conclusion, the}
        \wrongchoice{In other words, the}
    \end{choices}
\end{question}

\begin{question}{english-q08}
    \begin{choices}
        \wrongchoice{NO CHANGE}
        \wrongchoice{It's}
        \wrongchoice{Its'}
        \wrongchoice{That's}
    \end{choices}
\end{question}

\begin{question}{english-q09}
    \begin{choices}
        \wrongchoice{NO CHANGE}
        \wrongchoice{would have opened}
        \wrongchoice{open}
        \wrongchoice{might be opening}
    \end{choices}
\end{question}

\begin{question}{english-q10}
    Which of the following alternatives to the underlined portion would NOT be acceptable?
    \begin{choices}
        \wrongchoice{who played}
        \wrongchoice{as he played}
        \wrongchoice{played}
        \wrongchoice{who was playing}
    \end{choices}
\end{question}

\begin{question}{english-q11}
    Which of the following alternatives to the underlined portion would NOT be acceptable?
    \begin{choices}
        \wrongchoice{NO CHANGE}
        \wrongchoice{to}
        \wrongchoice{at}
        \wrongchoice{from}
    \end{choices}
\end{question}

\begin{question}{english-q12}
    The best placement for the underlined portion would be:
    \begin{choices}
        \wrongchoice{where it is now.}
        \wrongchoice{after the word woman.}
        \wrongchoice{after the word was.}
        \wrongchoice{after the word composing.}
    \end{choices}
\end{question}

\begin{question}{english-q13}
    Which choice most effectively emphasizes the rapid speed of the train?
    \begin{choices}
        \wrongchoice{NO CHANGE}
        \wrongchoice{continued on its way.}
        \wrongchoice{moved on down the tracks.}
        \wrongchoice{proceeded toward the next stop.}
    \end{choices}
\end{question}

\begin{question}{english-q14}
    If the writer were to delete the preceding sentence, this paragraph would primarily lose a statement that:
    \begin{choices}
        \wrongchoice{provides physical descriptions of people on the subway train.}
        \wrongchoice{supports the opening sentence of the essay.}
        \wrongchoice{provides evidence that people can be friendly on the subway train.}
        \wrongchoice{gives an explanation for the narrator’s actions.}
    \end{choices}
\end{question}

\begin{question}{english-q15}
    Question 15 asks about the preceding passage as a whole.

    Suppose the writer had intended to write a brief essay persuading readers that the subway system is New York City's most economical means of public transportation. Would this essay fulfill the writer’s goal?
    \begin{choices}
        \wrongchoice{Yes, because the essay supplies evidence of the large number of people using the subways.}
        \wrongchoice{Yes, because the essay describes people who are able to give to the needy because they have extra money in their pockets.}
        \wrongchoice{No, because the essay focuses on the kinds of people riding the subways, not on how inexpensive the subways are to ride.}
        \wrongchoice{No, because the essay focuses on the writer's love of all public transportation, not just the subways.}
    \end{choices}
\end{question}
}


%% page 16
\element{act-mc}{
\begin{figure}
    Navajo Code Talkers

    During World War II, a group of Navajo soldiers developed a code that became one of the most successful in U.S. military history. 
    This group, known as the Navajo code talkers, took part in every assault the U.S. Marines conducted in the Pacific from 1942 to 1945, transmitting information, on tactics, troop movements, orders, and other vital communications over telephones and radios.

    American military officials have been using cumbersome machines to encode and relay information during battles.
    In preliminary tests under simulated combat
\end{figure}

\begin{question}{english-q16}
    \begin{choices}
        \wrongchoice{NO CHANGE}
        \wrongchoice{group which was}
        \wrongchoice{group was}
        \wrongchoice{group}
    \end{choices}
\end{question}

\begin{question}{english-q17}
    \begin{choices}
        \wrongchoice{NO CHANGE}
        \wrongchoice{transmitting information on:}
        \wrongchoice{transmitting information on}
        \wrongchoice{transmitting: information on}
    \end{choices}
\end{question}

\begin{question}{english-q18}
    \begin{choices}
        \wrongchoice{NO CHANGE}
        \wrongchoice{had}
        \wrongchoice{would have}
        \wrongchoice{will have}
    \end{choices}
\end{question}

\begin{question}{english-q19}
    \begin{choices}
         \wrongchoice{NO CHANGE}
         \wrongchoice{thorny}
         \wrongchoice{strenuous}
         \wrongchoice{gawky}
    \end{choices}
\end{question}
}


conditions, the Navajo encoded, transmitted, and decoded
a three-line message in twenty seconds as the machines
20
required thirty minutes to perform the same job.
Nevertheless, these tests convinced the
21
officials of the value, of using the Navajo
22
language in a code.
20. F.
G.
H.
J. NO CHANGE
seconds so
seconds,
seconds, whereas
21. A.
B.
C.
D. NO CHANGE
Similarly, these
Still, these
These
22. F.
G.
H.
J. NO CHANGE
officials, of the value
officials of the value
officials, of the value,
23. A.
B.
C.
D. NO CHANGE
makes it
make it
make them
24. F.
G.
H.
J. NO CHANGE
from
with
of
The Navajo language is complex, with a structure and
sounds that makes them unintelligible to anyone without
23
extensive exposure to it. Outside Navajo communities,
24
such exposure is rare, which greatly contributed to
25. Which of the following alternatives to the underlined
portion would NOT be acceptable?
A. rare; this
B. rare this
C. rare. This
D. rare, a factor that
25
it’s success.
26
The Navajo developed and memorized the code. Since
27
their language did not have words for common U.S.
26. F.
G.
H.
J. NO CHANGE
that
this
the Navajo code’s
27. A.
B.
C.
D. NO CHANGE
The Navajo, who were various heights and weights,
Being of various heights and weights, the Navajo
The Navajo of different sizes
28. F.
G.
H.
J. NO CHANGE
hazardous
risky
OMIT the underlined portion.
military equipment, they turned to nature. They named
planes after birds and ships after fish. Dive bombers
became gini (chicken hawk) and destroyers were called
ca-lo (shark). The skilled Japanese code breakers remained
baffled by the Navajo language. The code was never
broken.
Unfortunately, the code talkers sometimes faced
dangerous peril from their own side. Many code talkers
28
needed bodyguards to protect them from other American
soldiers, some of whom mistook the Navajo for Japanese




\endinput

