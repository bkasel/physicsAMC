Page 1 of 8

PAGE \* MERGEFORMAT

7Copyright © 2010 Pearson Education, Inc.

Essential College Physics (Rex/Wolfson)

Chapter 1 Measurements in Physics

Quantitative

1) The following conversion equivalents are given:

1 gal = 231 in3 1 ft = 12 in 1 min = 60 s

A pipe delivers water at the rate of 95 gal/min. The rate, in ft3/s, is closest to:

A) 0.21

B) 0.19

C) 0.17

D) 0.15

E) 0.14

Answer: A

Var: 50+

2) The following conversion equivalents are given:

1 m = 100 cm 1 in = 2.54 cm 1 ft = 12 in

A bin has a volume of 1.5 m3. The volume of the bin, in ft3, is closest to:

A) 35

B) 41

C) 47

D) 53

E) 59

Answer: D

Var: 1

3) The following conversion equivalents are given:

1 mile = 5280 ft 1 ft = 12 in 1 m = 39.37 in 1 hour = 60 min 1 min = 60 s

A particle has a velocity of 4.7 miles per hour. The velocity, in m/s, is closest to:

A) 2.1

B) 1.7

C) 1.9

D) 2.3

E) 2.5

Answer: A

Var: 50+
Page 2 of 8

PAGE \* MERGEFORMAT

7Copyright © 2010 Pearson Education, Inc.

4) The following conversion equivalents are given:

1 kg = 1000 g 1 l = 1000 cm3 1 l = 0.0353 ft3

The density of a liquid is 0.83 g/cm3. The density of the liquid, in kg/ft3, is closest to:

A) 24

B) 19

C) 21

D) 26

E) 28

Answer: A

Var: 50+

5) A weight lifter can bench press 171 kg. How many milligrams (mg) is this?

A) 1.71 × 108 mg

B) 1.71 × 109 mg

C) 1.71 × 107 mg

D) 1.71 × 106 mg

Answer: A

Var: 50+

6) How many nanoseconds does it take for a computer to perform one calculation if it performs

6.7 × 107

calculations

second?

A) 15 ns

B) 67 ns

C) 11 ns

D) 65 ns

Answer: A

Var: 50+

7) A CD-ROM disk can store approximately 600.0 megabytes of information. If an average word

requires 9.0 bytes of storage, how many words can be stored on one disk?

A) 6.7 × 107 words

B) 5.4 × 109 words

C) 2.1 × 107 words

D) 2.0 × 109 words

Answer: A

Var: 9
Page 3 of 8

PAGE \* MERGEFORMAT

7Copyright © 2010 Pearson Education, Inc.

8) Your car gets 34.7 mi/gal on a trip. How many kilometers/liter did it get?

(3.79 = 1 gal; 1 mi = 1.61 km)

A) 14.7 km/l

B) 9.16 km/l

C) 55.9 km/l

D) 32.4 km/l

Answer: A

Var: 50+

9) The wavelength of a certain laser is 0.66 microns, where 1 micron = 1 × 10-6 m.

What is this wavelength in nanometers? (1 nm = 10-9m)

A) 6.6 × 102 nm

B) 6.6 × 103 nm

C) 6.6 × 101 nm

D) 6.6 × 104 nm

Answer: A

Var: 50+

10) Express [ 2.2 × 106]-1/2 in scientific notation.

A) 6.7 × 10-4

B) 1.5 × 103

C) 1.5 × 10-5

D) 1.5 × 104

Answer: A

Var: 40

11) An oak tree was planted 22 years ago. How many seconds does this correspond to? (Do not

take leap days into account.)

A) 6.9 × 108

B) 1.2 × 107

C) 2.9 × 107

D) 2.8 × 108

Answer: A

Var: 12

12) A plot of land contains 5.8 acres. How many square meters does it contain?

[1 acre = 43,560 ft2].

A) 2.3 × 104 m2

B) 7.1 × 103 m2

C) 7.0 × 104 m2

D) 5.0 × 104 m2

Answer: A

Var: 50+

13) A person on a diet loses 1.6 kg in a week. How many micrograms/second (µg/s) are lost?

A) 2.6 × 103 μg/s

B) 1.6 × 105μg/s
Page 4 of 8

PAGE \* MERGEFORMAT

7Copyright © 2010 Pearson Education, Inc.

C) 44 μg/s

D) 6.4 × 104 μg/s

Answer: A

Var: 11

14) Which of the following is a reasonable estimate of the number of characters (typed letters or

numbers) in a 609 page book? Assume an average of 194 words/page and a reasonable average

number of letters/word.

A) 5 × 105 char

B) 5 × 107 char

C) 5 × 106 char

D) 5 × 104 char

Answer: A

Var: 50+

15) Add 3685 g and 66.8 kg and express your answer in milligrams (mg).

A) 7.05 × 107 mg

B) 7.05 × 104 mg

C) 7.05 × 105 mg

D) 7.05 × 106 mg

Answer: A

Var: 50+

16) What is

0.674

0.74 to the proper number of significant figures?

A) 0.91

B) 0.911

C) 0.9108

D) 0.9

Answer: A

Var: 50+

17) A marathon is 26 mi and 385 yd long. Estimate how many strides would be required to run a

marathon. Assume a reasonable value for the average number of feet/stride.

A) 4.5 × 104 strides

B) 4.5 × 103 strides

C) 4.5 × 105 strides

D) 4.5 × 106 strides

Answer: A

Var: 1

18) Estimate the number of times an average person's heart beats in a lifetime. Assume the

average heart rate is 69 beats/min and a life span of 75 yr.

A) 3 × 109 beats

B) 3 × 108 beats

C) 3 × 1010 beats

D) 3 × 107 beats
Page 5 of 8

PAGE \* MERGEFORMAT

7Copyright © 2010 Pearson Education, Inc.

Answer: A

Var: 50+

True/False

1) Dimensional analysis can tell you whether an equation is physically correct.

Answer: FALSE

Var: 1

2) A 2-L bottle of soda gives you more for your money than a 2-qt bottle would, at the same

price.

Answer: TRUE

Var: 1

3) Zeros between non-zero numbers are significant.

Answer: TRUE

Var: 1

4) Zeros at the beginning of a number are significant.

Answer: FALSE

Var: 1

5) Zeros at the end of a number after the decimal point are not significant.

Answer: FALSE

Var: 1

Conceptual

1) The length of a certain piece of wood is stated as 12.946 cm. Thinking in terms of accuracy

and significant figures, what value would you obtain if you measured this board with a ruler

whose smallest division is the millimeter?

Answer: Since 1 cm = 10 mm, you could only measure this board to 1/10 of a centimeter,

therefore, 12.9 cm.

Var: 1

2) A possible unit for the measure of time would be the period of your own pulse. Why might

this unit not be a good choice for a time standard?

Answer: In general, the human pulse rate often changes. Also, other people would, in general,

have different pulse rates, so it would be difficult to reproduce the standard.

Var: 1

3) What are some advantages of using scientific notation to express very large or very small

numbers?

Answer: The main advantage is the ability to express these numbers in a minimum amount of

space. For example:

1 trillion = 1,000,000,000,000

= 1 × 1012
Page 6 of 8

PAGE \* MERGEFORMAT

7Copyright © 2010 Pearson Education, Inc.

Var: 1

4) If you needed to estimate the number of gallons of gasoline burned in the United States each

year by automobiles, what quantities would you need to assume? What formula would you need

to use?

Answer: If you assumed

N = total number of cars

M = average number of miles driven

each year per car

R = average number of miles per gallon

Then

Total number of gallons =

NM

R



Var: 1

5) Why so much emphasis upon units? Why are units considered to be as important as the

quantity, "magnitude", of something?

Answer: Just knowing the magnitude of something is not complete information (in fact it is

ambiguous) unless one also knows the units. For example, learning that you will earn 150 for a

certain task might mean 150 dollars, or perhaps 150 cents, or 150 pesos. The magnitude is almost

worthless without the unit in which it is expressed.

Var: 1

6) Why is MASS considered a more basic property than WEIGHT?

Answer: A mass may have weight but the weight is a property which changes as the mass is

moved around in the universe. Mass is not a property which depends upon position.

Var: 1

7) Define and explain the meaning of DENSITY.

Answer: Density expresses how much mass is contained in a unit volume:

i.e., Density = Mass/Volume.

Var: 1

8) Determine the number of significant figures in 100.01 x 103 meters.

Answer: 5

Var: 1
Page 7 of 8

PAGE \* MERGEFORMAT

7Copyright © 2010 Pearson Education, Inc.

9) Express 0.0015671 kg to three significant figures.

Answer: 0.00157 kg

Var: 1

10) If you are measuring the thickness of a strand of human hair, the most appropriate SI unit is

the

A) kilometer.

B) meter.

C) centimeter.

D) millimeter.

E) micrometer.

Answer: E

Var: 1



11) If you are measuring the mass of an elephant, the most appropriate SI unit is the

A) megagram.

B) kilogram.

C) gram.

D) milligram.

E) microgram.

Answer: B

Var: 1



12) If you are measuring the mass of a small glass of milk, the most appropriate SI unit is the

A) megagram.

B) kilogram.

C) gram.

D) milligram.

E) microgram.

Answer: C

Var: 1
Page 8 of 8

PAGE \* MERGEFORMAT

7Copyright © 2010 Pearson Education, Inc.
8 of 8
Displaying essential-college-physics-rex-1st-tb.doc.
