
\AMCopenOpts{
    lineheight=1.0em,
    dots=true,
    hspace=0.5em,         % sets the space between boxes in the marking area
    backgroundcol=white,  % sets the background color of the marking area.
    foregroundcol=black,  % sets the foreground color of the marking area.
    width=0.95\linewidth, % sets the width of the frame enclosing the answering area
    framerule=1pt,        % sets the line width for the frame enclosing the answering area.
    framerulecol=black,   % sets the frame color for the answering area.
    boxmargin=0.25em,     % sets the margin around the scoring boxes.
    boxframerule=1pt,     % sets the line width for the frame around the scoring boxes.
    boxframerulecol=black,% sets the color of the frame around the scoring boxes.
}

\element{openNumber}{
\begin{question}{first}
    Suppose you are a doctor faced with a patient who has a malignant tumor in his stomach.
    It is impossible to operate on the patient,
         but unless the tumor is destroyed, the patient will die. 
    There is a kind of ray that can be used to destroy the tumor.
    If the rays reach the tumor all at once at a sufficiently high intensity,
        the tumor will be destroyed. 
    Unfortunately at this intensity the healthy tissue that the rays pass
        through on the way to the tumor will also be destroyed. 
    At lower intensities the rays are harmless to healthy tissue,
        but they will not affect the tumor either. 
    What type of procedure might be used to destroy the tumor with the rays,
        and at the same time avoid destroying the healthy tissue?
    \AMCOpen{lines=3}{
        \wrongchoice[W]{w}\scoring{0}
        \wrongchoice[P]{p}\scoring{1}
        \correctchoice[C]{c}\scoring{2}
    }
    %% ANS: use multiple directions
\end{question}
}

\element{openNumber}{
\begin{question}{second}
    A dictator ruled a small country from a fortress.
    The fortress was situated in the middle of the country and many roads
        radiated outward from it, like spokes on a wheel. 
    A great general vowed to capture the fortress and free the country
        of the dictator. 
    The general knew that if his entire army could attack the fortress at once,
        it could be captured. 
    But a spy reported that the dictator had planted mines on each of the roads. 
    The mines were set so that small bodies of men could pass over them safely,
        since the dictator needed to be able to move troops and workers about;
        however, any large force would detonate the mines. 
    Not only would this blow up the road,
        but the dictator would destroy many villages in retaliation. 
    How could the general attack the fortress?
    \AMCOpen{lines=2}{
        \wrongchoice[W]{w}\scoring{0}
        \wrongchoice[P]{p}\scoring{1}
        \correctchoice[C]{c}\scoring{2}
    }
    %% ANS: use multiple fronts
\end{question}
}

