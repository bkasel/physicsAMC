

%% http://ctan.mirrorcatalogs.com/macros/latex/contrib/physics/physics.pdf
%%--------------------------------------------------------------------------------


%% GRE Physics 9277 Practice Exam
%%----------------------------------------

%% Page 47
\element{gre}{
\begin{question}{GRE9277-Q01}
    The wave function of a particle is $\mathrm{e}^{i\left(kx-\omega t\right)}$,
        where $x$ is distance, $t$ is time, and $k$ and $\omega$ are positive real numbers.
    The $x$-component of the momentum of the particle is:
    \begin{multicols}{3}
    \begin{choices}
        \wrongchoice{zero}
        \wrongchoice{$\hbar\omega$}
        \wrongchoice{$\hbar k$}
        \wrongchoice{$\dfrac{\hbar\omega}{c}$}
        \wrongchoice{$\dfrac{\hbar k}{\omega}$}
    \end{choices}
    \end{multicols}
\end{question}
}

\element{gre}{
\begin{question}{GRE9277-Q02}
    The longest wavelength x-ray that can undergo Bragg diffraction in a crystal for a given family of planes of spacing $d$ is:
    \begin{multicols}{3}
    \begin{choices}
        \wrongchoice{$\dfrac{d}{4}$}
        \wrongchoice{$\dfrac{d}{2}$}
        \wrongchoice{$d$}
        \wrongchoice{$2d$}
        \wrongchoice{$4d$}
    \end{choices}
    \end{multicols}
\end{question}
}

\element{gre}{
\begin{question}{GRE9277-Q03}
    The ratio of the energies of the $K$ characteristic x-rays of carbon ($Z=6$) to those of magnesium ($Z=12$) is most nearly:
    \begin{multicols}{3}
    \begin{choices}
        \wrongchoice{$\dfrac{1}{4}$}
        \wrongchoice{$\dfrac{2}{2}$}
        \wrongchoice{$1$}
        \wrongchoice{$2$}
        \wrongchoice{$4$}
    \end{choices}
    \end{multicols}
\end{question}
}

\element{gre}{
\begin{question}{GRE9277-Q04}
    The magnitude of the Earth's gravitational force on a point mass is $F(r)$, where $r$ is the distance from the Earth's center to the point mass.
    Assume teh Earth is a homogeneous sphere of radius $R$.
    %% Inverse square law
    What is $\dfrac{F(R)}{F(2R)}$?
    \begin{multicols}{3}
    \begin{choices}
        \wrongchoice{32}
        \wrongchoice{8}
        \wrongchoice{4}
        \wrongchoice{2}
        \wrongchoice{1}
    \end{choices}
    \end{multicols}
\end{question}
}

\element{gre}{
\begin{question}{GRE9277-Q05}
    Suppose there is a very small shaft in the Earth such that the point mass can be placed at a radius of $R/2$.
    What is $\dfrac{F(R)}{F\left(\dfrac{R}{2}\right)}$?
    %% NOTE: use equation* ??
    \begin{multicols}{3}
    \begin{choices}
        \wrongchoice{8}
        \wrongchoice{4}
        \wrongchoice{2}
        \wrongchoice{$\dfrac{1}{2}$}
        \wrongchoice{$\dfrac{1}{4}$}
    \end{choices}
    \end{multicols}
\end{question}
}

%% Page 48
\element{gre}{
\begin{question}{GRE9277-Q06}
    \begin{center}
    \begin{tikzpicture}
        %% NOTE: TODO: draw triangles and square
    \end{tikzpicture}
    \end{center}
    Two wedges, each of mass $m$, are placed next to each other on a flat floor.
    A cube of mass $M$ is balanced on the wedges as shown above.
    Assume no friction between the cube and the wedges,
        but a coefficient of static friction $\mu<1$ between the wedges and the floor.
    What is the largest $M$ that can be balanced as shown without motion of the wedges?
    \begin{multicols}{2}
    \begin{choices}
        \wrongchoice{$\dfrac{m}{\sqrt{2}}$}
        \wrongchoice{$\dfrac{\mu m}{\sqrt{2}}$}
        \wrongchoice{$\dfrac{\mu m}{1-\mu}$}
        \wrongchoice{$\dfrac{2\mu m}{1-\mu}$}
        \wrongchoice{All $M$ will balance}
    \end{choices}
    \end{multicols}
\end{question}
}

\element{gre}{
\begin{question}{GRE9277-Q07}
    \begin{center}
    \begin{tikzpicture}
        %% NOTE: TODO: draw triangles and square
    \end{tikzpicture}
    \end{center}
    A cylindrical tube of mass $M$ can slide on a horizontal wire.
    Two identical pendulums, each of mass $m$ and length $l$,
        hang from the ends of the tube, as shown above.
    For small oscilations of the pendulum in the plane of the paper,
        the eigenfrequencies of the normal modes of oscillation of this system
        are $0$, $\sqrt{\dfrac{g\left(M+2m\right)}{lM}}$, and
    \begin{multicols}{2}
    \begin{choices}
        \wrongchoice{$\sqrt{\dfrac{g}{l}}$}
        \wrongchoice{$\sqrt{\dfrac{g}{l}\dfrac{M+m}{M}}$}
        \wrongchoice{$\sqrt{\dfrac{g}{l}\dfrac{m}{M}}$}
        \wrongchoice{$\sqrt{\dfrac{g}{l}\dfrac{m}{M+m}}$}
        \wrongchoice{$\sqrt{\dfrac{g}{l}\dfrac{m}{M+2m}}$}
    \end{choices}
    \end{multicols}
\end{question}
}

\element{gre}{
\begin{question}{GRE9277-Q08}
    \begin{center}
    \begin{tikzpicture}
        %% NOTE: TODO: draw triangles and square
    \end{tikzpicture}
    \end{center}
    A solid cone hangs from a frictionless pivot at the origin $O$,
        as shown above.
    If $\hat{\imath}$, $\hat{\jmath}$, and $\hat{k}$ ae unit vectors,
        and $a$, $b$, and $c$ are positive constants,
        which of the following forces $\mathbf{F}$ applied to the rim of the cone at a point $P$ results in a torque $\tau$ on the cone with a negative component $\tau_z$?
    \begin{choices}
        %% NOTE: TODO: \mathbf ??
        \wrongchoice{$\mathbf{F} = a\hat{k}$,       $P$ is $\left(0,b,-c\right)$}
        \wrongchoice{$\mathbf{F} = -a\hat{k}$,      $P$ is $\left(0,-b,-c\right)$}
        \wrongchoice{$\mathbf{F} = a\hat{\jmath}$,  $P$ is $\left(-b,0,-c\right)$}
        \wrongchoice{$\mathbf{F} = a\hat{\jmath}$,  $P$ is $\left(b,0,-c\right)$}
        \wrongchoice{$\mathbf{F} = -a\hat{k}$,      $P$ is $\left(-b,0,-c\right)$}
    \end{choices}
\end{question}
}

\element{gre}{
\begin{question}{GRE9277-Q09}
    \begin{center}
    \begin{tikzpicture}
        %% NOTE: TODO: draw triangles and square
    \end{tikzpicture}
    \end{center}
    A coaxial cable having radii $a$, $b$, and $c$ carries equal and opposite currents of magnitude $i$ on the inner and outer conductors.
    What is the magnetic induction at point $P$ outisde of the cable at a distance $r$ from the axis?
    \begin{multicols}{2}
    \begin{choices}
        \wrongchoice{Zero}
        \wrongchoice{$\dfrac{\mu_0 i r}{2\pi a^2}$}
        \wrongchoice{$\dfrac{\mu_0 i}{2\pi r}$}
        \wrongchoice{$\dfrac{\mu_0 i}{2\pi r} \dfrac{c^2-r^2}{c^2-b^2}$}
        \wrongchoice{$\dfrac{\mu_0 i}{2\pi r} \dfrac{r^2-b^2}{c^2-b^2}$}
    \end{choices}
    \end{multicols}
\end{question}
}

\element{gre}{
\begin{question}{GRE9277-Q10}
    \begin{center}
    \begin{tikzpicture}
        %% NOTE: TODO: draw triangles and square
    \end{tikzpicture}
    \end{center}
    Two positive charges of $q$ and $2q$ coulombs are located on the $x$-axis at $x=0.5a$ and $1.5a$,
        respectively, as shown above.
    There is an infinite, grounded conducting plane at $x=0$.
    What is the magnitude of the net force on the charge $q$?
    \begin{multicols}{2}
    \begin{choices}
        \wrongchoice{$\dfrac{1}{4\pi \epsilon_0} \dfrac{q^2}{a^2}$}
        \wrongchoice{$\dfrac{1}{4\pi \epsilon_0} \dfrac{3q^2}{2a^2}$}
        \wrongchoice{$\dfrac{1}{4\pi \epsilon_0} \dfrac{2q^2}{a^2}$}
        \wrongchoice{$\dfrac{1}{4\pi \epsilon_0} \dfrac{3q^2}{a^2}$}
        \wrongchoice{$\dfrac{1}{4\pi \epsilon_0} \dfrac{7q^2}{2a^2}$}
    \end{choices}
    \end{multicols}
\end{question}
}

\element{gre}{
\begin{question}{GRE9277-Q11}
    \begin{center}
    \begin{tikzpicture}
        %% NOTE: TODO: draw triangles and square
    \end{tikzpicture}
    \end{center}
    The capacitor in the circuit shown above is initially charged.
    After closing the switch,
        how much time elapses until one-half of the capacitor's initial stored \emph{energy} is dissipated?
    \begin{multicols}{2}
    \begin{choices}
        \wrongchoice{$RC$}
        \wrongchoice{$\dfrac{RC}{2}$}
        \wrongchoice{$\dfrac{RC}{4}$}
        \wrongchoice{$2RC\ln 2$}
        \wrongchoice{$\dfrac{2RC\ln 2}{2}$}
    \end{choices}
    \end{multicols}
\end{question}
}

\element{gre}{
\begin{question}{GRE9277-Q12}
    \begin{center}
    \begin{tikzpicture}
        %% NOTE: TODO: draw triangles and square
    \end{tikzpicture}
    \end{center}
    Two large conducting plates form a wedge of angle $\alpha$ as shown in the diagram above.
    The plates are insulated from each other;
        one has a potential $V_0$ and the other is grounded.
    Assuming that the plates are large enough so that the potential difference between them is independent of the cylindrical coordinates $z$ and $\rho$,
        the potential  anywhere between the plates as a function of the angle $\phi$ is:
    \begin{multicols}{3}
    \begin{choices}
        \wrongchoice{$\dfrac{V_0}{\alpha}$}
        \wrongchoice{$\dfrac{V_0 \phi}{\alpha}$}
        \wrongchoice{$\dfrac{V_0 \alpha}{\phi}$}
        \wrongchoice{$\dfrac{V_0 \alpha^2}{\phi}$}
        \wrongchoice{$\dfrac{V_0 \alpha}{\phi^2}$}
    \end{choices}
    \end{multicols}
\end{question}
}

\element{gre}{
\begin{question}{GRE9277-Q13}
    Listed below are Maxwell's equations of electromagnetism.
    If magnetic monopoles exists,
        which of these equations would be \emph{incorrect}?
    \begin{enumerate}
        \item[I.] $\mathrm{Curl}\,\mathbf{H}  = \mathbf{J} + \dfrac{\partial\mathbf{D}}{\partial t}$
        \item[II.] $\mathrm{Curl}\,\mathbf{E}  = - \dfrac{\partial\mathbf{B}}{\partial t}$
        \item[III.] $\mathrm{div}\,\mathbf{D}  = - \rho$
        \item[IV.] $\mathrm{div}\,\mathbf{B}  = 0$
    \end{enumerate}
    \begin{multicols}{2}
    \begin{choices}
        \wrongchoice{I only}
        \wrongchoice{I and II}
        \wrongchoice{I and III}
        \wrongchoice{II and IV}
        \wrongchoice{III and IV}
    \end{choices}
    \end{multicols}
\end{question}
}

\element{gre}{
\begin{question}{GRE9277-Q14}
    The total energy of a blackbody radiation source is collected for one minute and used to heat water.
    The temperature of the water increases from \SI{20}{\degreeCelsius} to \SI{20.5}{\degreeCelsius}.
    If the absolute temperature of the blackbody are doubled and the experiment repeated,
        which of the following statements would be most nearly correct?
    \begin{choices}
        \wrongchoice{The temperature of the water would increase from \SI{20}{\degreeCelsius} to a final temperature of \SI{21}{\degreeCelsius}}
        \wrongchoice{The temperature of the water would increase from \SI{20}{\degreeCelsius} to a final temperature of \SI{24}{\degreeCelsius}}
        \wrongchoice{The temperature of the water would increase from \SI{20}{\degreeCelsius} to a final temperature of \SI{28}{\degreeCelsius}}
        \wrongchoice{The temperature of the water would increase from \SI{20}{\degreeCelsius} to a final temperature of \SI{36}{\degreeCelsius}}
        \wrongchoice{The water would boil within the one-minute time period}
    \end{choices}
\end{question}
}

\element{gre}{
\begin{question}{GRE9277-Q15}
    \begin{center}
    \begin{tikzpicture}
        %% NOTE: TODO: draw dumbbell
    \end{tikzpicture}
    \end{center}
    A classical model of a diatomic molecule is a springy dumbbell, as shown above,
        where the dumbbell is free to rotate about axes perpendicular to the spring.
    In the limit of high temperature,
        what is the specific heat per mole at constant volume?
    \begin{multicols}{3}
    \begin{choices}
        \wrongchoice{$\dfrac{3}{2}R$}
        \wrongchoice{$\dfrac{5}{2}R$}
        \wrongchoice{$\dfrac{7}{2}R$}
        \wrongchoice{$\dfrac{9}{2}R$}
        \wrongchoice{$\dfrac{11}{2}R$}
    \end{choices}
    \end{multicols}
\end{question}
}

\element{gre}{
\begin{question}{GRE9277-Q16}
    \begin{center}
    \begin{tikzpicture}
        %% NOTE: TODO: draw dumbbell
    \end{tikzpicture}
    \end{center}
    An engine absorbs heat a temperature of \SI{727}{\degreeCelsius} and exhausts heat at a temperature of \SI{527}{\degreeCelsius}.
    If the engine operates at maximum possible efficiency, for \SI{2000}{\joule} of heat input the amount of work the engine performs is most nearly:
    \begin{multicols}{2}
    \begin{choices}
        \wrongchoice{\SI{400}{\joule}}
        \wrongchoice{\SI{1400}{\joule}}
        \wrongchoice{\SI{1600}{\joule}}
        \wrongchoice{\SI{2000}{\joule}}
        \wrongchoice{\SI{2760}{\joule}}
    \end{choices}
    \end{multicols}
\end{question}
}

\element{gre}{
\begin{question}{GRE9277-Q17}
    The outpus of two electrical oscillators are compared on an osciloscope screen.
    The oscilloscope spot is initially at the center of the screen.
    Oscillator $Y$ is connected to the vertical terminals of the oscilloscope and oscillator $X$ to the horizontal terminals.
    Which of the following patterns could appear on the oscilloscope screen,
        if the frequency of oscillator $Y$ is twice that of oscillator $X$?
    \begin{multicols}{2}
    \begin{choices}
        \AMCboxDimensions{down=-2.0em}
        \wrongchoice{
            \begin{tikzpicture}
                %% NOTE: TODO: draw curve
            \end{tikzpicture} 
        }
    \end{choices}
    \end{multicols}
\end{question}
}

\element{gre}{
\begin{question}{GRE9277-Q18}
    In transmitting high frequency signals on a coaxial cable,
        it is important that the cable be terminated at an end with its characteristics impedance in order to avoid:
    \begin{choices}
        \wrongchoice{leakage of the signal out of the cable}
        \wrongchoice{overheating of the cable}
        \wrongchoice{reflection of signals from the terminated end of the cable}
        \wrongchoice{attenuation of the signal propagating in the cable}
        \wrongchoice{production of image currents in the outer conductor}
    \end{choices}
\end{question}
}

\element{gre}{
\begin{question}{GRE9277-Q19}
    Which of the following is most nearly the mass of the Earth?
    (The radius of the Earth is about \SI{6.4e6}{\meter})
    \begin{multicols}{2}
    \begin{choices}
        \wrongchoice{\SI{6e24}{\kilo\gram}}
        \wrongchoice{\SI{6e27}{\kilo\gram}}
        \wrongchoice{\SI{6e30}{\kilo\gram}}
        \wrongchoice{\SI{6e33}{\kilo\gram}}
        \wrongchoice{\SI{6e36}{\kilo\gram}}
    \end{choices}
    \end{multicols}
\end{question}
}

\element{gre}{
\begin{question}{GRE9277-Q20}
    \begin{center}
    \begin{tikzpicture}
        %% NOTE: TODO: draw triangles and square
    \end{tikzpicture}
    \end{center}
    In a double-slit interference experiment,
        $d$ is the distance between the centers of the slits and $w$ is the width of each slit,
        as shown in the figure above.
    For incident plane waves, an interference maximum on a distant screen will be ``missing'' when:
    \begin{multicols}{2}
    \begin{choices}
        \wrongchoice{$d = \sqrt{2} w$}
        \wrongchoice{$d = \sqrt{3} w$}
        \wrongchoice{$2d = w$}
        \wrongchoice{$2d = 3w$}
        \wrongchoice{$3d = 2w$}
    \end{choices}
    \end{multicols}
\end{question}
}

\element{gre}{
\begin{question}{GRE9277-Q21}
    A soap film with index of refraction greater than air is formed on a circular wire fram that is held in a vertical plane.
    The film is viewed by reflected light from a white-light source.
    Bands of color are observed at the lower parts of the soap film,
        but the area near the top appears black.
    A correct explanation for this phenoment would involve which of the following?
    \begin{enumerate}
        \item[I.] The top of the soap film absorbs all of the light incident on it;
            none is transmitted.
        \item[II.] The thickness of the top part of the soap film has become much less than a wavelength of visible light.
        \item[III.] There is a phase change of \ang{180} for all wavelengths of light reflected from the front surface of the soap film.
        \item[IV.] There is no phase change for any wavelength of light reflected from the back surface of the soap film.
    \end{enumerate}
    \begin{multicols}{2}
    \begin{choices}
        \wrongchoice{I only}
        \wrongchoice{II and II only}
        \wrongchoice{I, II, and III}
        \wrongchoice{II, III, and IV}
    \end{choices}
    \end{multicols}
\end{question}
}

\element{gre}{
\begin{question}{GRE9277-Q22}
    \begin{center}
    \begin{tikzpicture}
        %% NOTE: TODO: draw triangles and square
    \end{tikzpicture}
    \end{center}
    A simple telescope consists of two convex lenses,
        the objective and the eyepiece,
        which have a common focal point $P$,
        as shown in the figure above.
    If the focal length of the objective is 1.0 meter and the angular magnification of the telescope is 10,
        what is the optical path length between objective and eyepiece?
    \begin{multicols}{3}
    \begin{choices}
        \wrongchoice{\SI{0.1}{\meter}}
        \wrongchoice{\SI{0.9}{\meter}}
        \wrongchoice{\SI{1.0}{\meter}}
        \wrongchoice{\SI{1.1}{\meter}}
        \wrongchoice{\SI{10}{\meter}}
    \end{choices}
    \end{multicols}
\end{question}
}

\element{gre}{
\begin{question}{GRE9277-Q23}
    The Fermi temperature of Cu is about \SI{80 000}{\kelvin}.
    Which of the following is most nearly equal to the average speed of a conduction electron in Cu?
    \begin{multicols}{2}
    \begin{choices}
        \wrongchoice{\SI{2e-2}{\meter\per\second}}
        \wrongchoice{\SI{2}{\meter\per\second}}
        \wrongchoice{\SI{2e2}{\meter\per\second}}
        \wrongchoice{\SI{2e4}{\meter\per\second}}
        \wrongchoice{\SI{2e6}{\meter\per\second}}
    \end{choices}
    \end{multicols}
\end{question}
}

\element{gre}{
\begin{question}{GRE9277-Q24}
    Solid argon is held together by which of the following bonding mechanisms?
    \begin{choices}
        \wrongchoice{Ionic bond only}
        \wrongchoice{Covalent bond only}
        \wrongchoice{Partly covalent and partly ionic bond}
        \wrongchoice{Metallic bond}
        \wrongchoice{van der Waals bond}
    \end{choices}
\end{question}
}

\element{gre}{
\begin{question}{GRE9277-Q25}
    In experiments located deep underground,
        the two types of cosmic rays that most commonly reach the experimental apparatus are:
    \begin{choices}
        \wrongchoice{alpha particles and neutrons}
        \wrongchoice{protons and electrons}
        \wrongchoice{iron nuclei and carbon nuclei}
        \wrongchoice{muons and neutrinos}
        \wrongchoice{positrons and electrons}
    \end{choices}
\end{question}
}

\element{gre}{
\begin{question}{GRE9277-Q26}
    ($\log_{10} 2 = 0.30$; $\log_{10} e = 0.43$)
    \begin{center}
    \begin{tikzpicture}
        %% NOTE: TODO: pgfplots logaxis
    \end{tikzpicture}
    \end{center}
    A radioactive nucleus decays,
        with the activity shown in the graph above.
    The half-lift of the nucleus is:
    \begin{multicols}{3}
    \begin{choices}
        \wrongchoice{\SI{2}{\minute}}
        \wrongchoice{\SI{7}{\minute}}
        \wrongchoice{\SI{11}{\minute}}
        \wrongchoice{\SI{18}{\minute}}
        \wrongchoice{\SI{23}{\minute}}
    \end{choices}
    \end{multicols}
\end{question}
}

\element{gre}{
\begin{question}{GRE9277-Q27}
    If a freely moving electron is localized in space to within $\Delta x_0$ of $x_0$,
        its wave function can be described by a wave packet
        $\psi \left(x,t\right) =
            \int^{\infty}_{-\infty} \mathrm{e}^{i\left(kx-\omega t\right)} f(k)\,\mathrm{d}k$,
        where $f(k)$ is peaked around a central value $k_0$.
    Which of the following is most nearly with width of the peak in $k$?
    \begin{multicols}{2}
    \begin{choices}
        \wrongchoice{$\Delta k = \dfrac{1}{x_0}$}
        \wrongchoice{$\Delta k = \dfrac{1}{\Delta x_0}$}
        \wrongchoice{$\Delta k = \dfrac{\Delta x_01}{x_0^2}$}
        \wrongchoice{$\Delta k = \left(\dfrac{\Delta x_0}{x_0}\right) k_0$}
        \wrongchoice{$\Delta k = \sqrt{k_0^2 + \left(\dfrac{1}{x_0}\right)^2}$}
    \end{choices}
    \end{multicols}
\end{question}
}

\element{gre}{
\begin{question}{GRE9277-Q28}
    A system is known to be in the normalized state described by the wave function
    \begin{equation*}
        \psi\left(\theta,\phi\right) =
            \dfrac{1}{\sqrt{30}} \left(5 Y_4^{3} + Y_6^{3} - 2 Y_6^{0}\right),
    \end{equation*}
    where the $Y_{l}^{m} (\theta,\phi)$ are the spherical harmonics.
    The probability of finding the system in a state with aximuthal orbital quantum number $m=3$ is:
    \begin{multicols}{3}
    \begin{choices}
        \wrongchoice{$0$}
        \wrongchoice{$\dfrac{1}{15}$}
        \wrongchoice{$\dfrac{1}{6}$}
        \wrongchoice{$\dfrac{1}{3}$}
        \wrongchoice{$\dfrac{13}{15}$}
    \end{choices}
    \end{multicols}
\end{question}
}

\element{gre}{
\begin{question}{GRE9277-Q29}
    \begin{center}
    \begin{tikzpicture}
        %% NOTE: TODO: pgfplots
    \end{tikzpicture}
    \end{center}
    An attractive, one-dimensional square well has depth $V_0$ as shown above.
    Which of the following best shows a possible wave function for a bound state?
    \begin{multicols}{3}
    \begin{choices}
        \wrongchoice{
            \begin{tikzpicture}
                %% NOTE: TODO: pgfplots
            \end{tikzpicture}
        }
    \end{choices}
    \end{multicols}
\end{question}
}

\element{gre}{
\begin{question}{GRE9277-Q30}
    Given that the binding energy of the hydrogen atom ground state is $E_0 = \SI{13.6}{\eV}$,
        the binding energy of the $n=2$ state of positronium (positron-electron system) is:
    \begin{multicols}{3}
    \begin{choices}
        \wrongchoice{$8 E_0$}
        \wrongchoice{$4 E_0$}
        \wrongchoice{$E_0$}
        \wrongchoice{$\dfrac{E_0}{4}$}
        \wrongchoice{$\dfrac{E_0}{8}$}
    \end{choices}
    \end{multicols}
\end{question}
}

\element{gre}{
\begin{question}{GRE9277-Q31}
    In a $^{3}S$  state of the helium atom,
        the possible values of the total electronic angular momentum quantum number are:
    \begin{multicols}{2}
    \begin{choices}
        \wrongchoice{$0$ only}
        \wrongchoice{$1$ only}
        \wrongchoice{$0$ and $1$ only}
        \wrongchoice{$0$, $\dfrac{1}{2}$, and 1}
        \wrongchoice{$0$, $1$, and $2$}
    \end{choices}
    \end{multicols}
\end{question}
}

\newcommand{\GRENinetyTwoQthirtyTwo}{
\begin{circuitikz}
\end{circuitikz}
}

\element{gre}{
\begin{question}{GRE9277-Q32}
    \begin{center}
        \GRENinetyTwoQthirtyTwo
    \end{center}
    In the circuit shown above, the resistances are given in ohms and the battery is assumed ideal with emf equal to \SI{3.0}{\volt}.
    %% begin question
    The reistor that dissipates the most power is:
    \begin{multicols}{3}
    \begin{choices}
        \wrongchoice{$R_1$}
        \wrongchoice{$R_2$}
        \wrongchoice{$R_3$}
        \wrongchoice{$R_4$}
        \wrongchoice{$R_5$}
    \end{choices}
    \end{multicols}
\end{question}
}

\element{gre}{
\begin{question}{GRE9277-Q33}
    \begin{center}
        \GRENinetyTwoQthirtyTwo
    \end{center}
    In the circuit shown above, the resistances are given in ohms and the battery is assumed ideal with emf equal to \SI{3.0}{\volt}.
    %% begin question
    The voltage across resistor $R_4$ is:
    \begin{multicols}{3}
    \begin{choices}
        \wrongchoice{\SI{0.4}{\volt}}
        \wrongchoice{\SI{0.6}{\volt}}
        \wrongchoice{\SI{1.2}{\volt}}
        \wrongchoice{\SI{1.5}{\volt}}
        \wrongchoice{\SI{3.0}{\volt}}
    \end{choices}
    \end{multicols}
\end{question}
}

\element{gre}{
\begin{question}{GRE9277-Q34}
    A conducting cavity is driven as an electromagnetic resonator.
    If perfect conductivity is assumed,
        the transverse and normal field component must obey which of the following conditions at the inner cavity walls?
    \begin{choices}[o]
        \wrongchoice{$E_n=0$, $B_n=0$}
        \wrongchoice{$E_n=0$, $B_t=0$}
        \wrongchoice{$E_t=0$, $B_t=0$}
        \wrongchoice{$E_t=0$, $B_n=0$}
        \wrongchoice{None of the above}
    \end{choices}
\end{question}
}

\element{gre}{
\begin{question}{GRE9277-Q35}
    Light of wavelength 5200 \r{a}ngstroms is incident normally on a transmission diffraction grating with 2000 lines per centimeter.
    The first-order diffraction maximum is at an angle,
        with respect to the incident beam,
        that is most nearly:
    \begin{multicols}{3}
    \begin{choices}
        \wrongchoice{\ang{3}}
        \wrongchoice{\ang{6}}
        \wrongchoice{\ang{9}}
        \wrongchoice{\ang{12}}
        \wrongchoice{\ang{15}}
    \end{choices}
    \end{multicols}
\end{question}
}

\element{gre}{
\begin{question}{GRE9277-Q36}
    A plane-polarized electromagnetic wave is incident normally on a flat,
        perfectly conducting surface.
    Upon reflection at the surface,
        which of the following is true?
    \begin{choices}
        \wrongchoice{Both the electric vector and magnetic vector are reversed.}
        \wrongchoice{Neither the electric nor the magnetic vector is reversed.}
        \wrongchoice{The electric vector is reversed; the magnetic vector is not.}
        \wrongchoice{The magnetic vector is reversed; the electric vector is not.}
        \wrongchoice{The direction of the electric and magnetic vectors are interchanged.}
    \end{choices}
\end{question}
}

\element{gre}{
\begin{question}{GRE9277-Q37}
    \begin{center}
    \begin{tikzpicture}
        %% NOTE: TODO: draw triangles and square
    \end{tikzpicture}
    \end{center}
    %% K vector bfseries??
    A $\pi^0$ meson (rest-mass energy \SI{135}{\mega\eV}) is moving with velocity $0.8c\,\hat{k}$ in the laboratory rest frame when it decays into two photons, $\gamma_1$ and $\gamma_2$.
    In the $\pi^0$ rest frame, $\gamma_1$ is emitted forward and $\gamma_2$ is emitted backward relative to the $\pi^0$ direction of flight.
    The velocity of $\gamma_2$ in the laboratory rest frame is:
    \begin{multicols}{3}
    \begin{choices}
        \wrongchoice{$-1.0c\,\hat{k}$}
        \wrongchoice{$-0.2c\,\hat{k}$}
        \wrongchoice{$-0.8c\,\hat{k}$}
        \wrongchoice{$+1.0c\,\hat{k}$}
        \wrongchoice{$+1.8c\,\hat{k}$}
    \end{choices}
    \end{multicols}
\end{question}
}

\element{gre}{
\begin{question}{GRE9277-Q38}
    Tau leptons are observed to have an average half-life of $\Delta t_1$ in the frame $S_1$ in which the leptons are at rest.
    In an inertial frame $S_2$,
        which is moving at a speed $v_{12}$ relative to $S_1$,
        the leptons are observed to have an average half-life of $\Delta t_2$.
    In another inertial reference frame $S_3$,
        which is moving at a speed $v_{13}$ relative to $S_1$ and $v_{23}$ relative to $S_2$,
        the leptons have an observed half-life of $\Delta t_3$.
    Which of the following is a correct relationship among two of the half-lives,
        $\Delta t_1$, $\Delta t_2$, and $\Delta t_3$?
    \begin{choices}
        \wrongchoice{$\Delta t_2 = \Delta t_1 \sqrt{1-\dfrac{v_{12}^2}{c^2}}$}
        \wrongchoice{$\Delta t_1 = \Delta t_3 \sqrt{1-\dfrac{v_{13}^2}{c^2}}$}
        \wrongchoice{$\Delta t_2 = \Delta t_3 \sqrt{1-\dfrac{v_{23}^2}{c^2}}$}
        \wrongchoice{$\Delta t_3 = \Delta t_2 \sqrt{1-\dfrac{v_{23}^2}{c^2}}$}
        \wrongchoice{$\Delta t_1 = \Delta t_2 \sqrt{1-\dfrac{v_{23}^2}{c^2}}$}
    \end{choices}
\end{question}
}

%% page 52
\element{gre}{
\begin{question}{GRE9277-Q39}
    \begin{center}
    \begin{tikzpicture}
        %% NOTE: TODO: pgfplots
    \end{tikzpicture}
    \end{center}
    If $n$ is an integer ranging from $1$ to infinity, $\omega$ is an agular frequency,
        and $t$ is time, the the Fourier series for a squared wave, as shown above,
        is given by which of the following?
    \begin{choices}
        \wrongchoice{$V(t) = \dfrac{4}{\pi} \sum_{1}^{\infty} \dfrac{1}{n}\sin\left(n\omega{}t\right)$}
        \wrongchoice{$V(t) = \dfrac{4}{\pi} \sum_{0}^{\infty} \dfrac{1}{2n+1}\sin\left(\left(2n+1\right)n\omega{}t\right)$}
        \wrongchoice{$V(t) = \dfrac{4}{\pi} \sum_{1}^{\infty} \dfrac{1}{n}\cos\left(n\omega{}t\right)$}
        \wrongchoice{$V(t) = \dfrac{4}{\pi} \sum_{0}^{\infty} \dfrac{1}{2n+1}\cos\left(\left(2n+1\right)\omega{}t\right)$}
        \wrongchoice{$V(t) = \dfrac{4}{\pi} + \dfrac{4}{\pi} \sum_{0}^{\infty} \dfrac{1}{n^2}\cos\left(n\omega{}t\right)$}
    \end{choices}
\end{question}
}

\element{gre}{
\begin{question}{GRE9277-Q40}
    A rigid cylinder rolls at constant speed without slipping on top of a horizontal plane surface.
    The acceleration of a point on the circumference of the cylinder at the moment when the point touches the plane is:
    \begin{choices}
        \wrongchoice{directed forward}
        \wrongchoice{directed backward}
        \wrongchoice{directed up}
        \wrongchoice{directed down}
        \wrongchoice{zero}
    \end{choices}
\end{question}
}

%% Questions 41-42
\element{gre}{
\begin{question}{GRE9277-Q41}
    A cylinder with momentum of inertia \SI{4}{\kilo\gram\meter\squared} about a fixed axis initially rotates at \SI{80}{\radian\per\second} about this axis.
    A constant torque is applied to slow it down to \SI{40}{\radian\per\second}.
    %% start question
    The kinetic energy lost by the clinder is:
    \begin{multicols}{2}
    \begin{choices}
        \wrongchoice{\SI{80}{\joule}}
        \wrongchoice{\SI{800}{\joule}}
        \wrongchoice{\SI{4000}{\joule}}
        \wrongchoice{\SI{9600}{\joule}}
        \wrongchoice{\SI{19 200}{\joule}}
    \end{choices}
    \end{multicols}
\end{question}
}

\element{gre}{
\begin{question}{GRE9277-Q42}
    A cylinder with momentum of inertia \SI{4}{\kilo\gram\meter\squared} about a fixed axis initially rotates at \SI{80}{\radian\per\second} about this axis.
    A constant torque is applied to slow it down to \SI{40}{\radian\per\second}.
    %% start question
    If the cylinder takes 10 seconds to reach \SI{40}{\radian\per\second},
        the magnitude of the applied torque is:
    \begin{multicols}{2}
    \begin{choices}
        \wrongchoice{\SI{80}{\newton\meter}}
        \wrongchoice{\SI{40}{\newton\meter}}
        \wrongchoice{\SI{32}{\newton\meter}}
        \wrongchoice{\SI{16}{\newton\meter}}
        \wrongchoice{\SI{8}{\newton\meter}}
    \end{choices}
    \end{multicols}
\end{question}
}

\element{gre}{
\begin{question}{GRE9277-Q43}
    If $\dfrac{\partial L}{\partial q_n}=0$, where $L$ is the Langrangian for a conservative system without constraints and $q_n$ is a generalized coordinate,
        then the generalized momentum $p_n$ is:
    \begin{choices}
        \wrongchoice{an ignorable coordinate}
        \wrongchoice{constant}
        \wrongchoice{undefined}
        \wrongchoice{Equal to $\dfrac{\dd}{\dd t}\left(\dfrac{\partial L}{\partial q_n}\right)$}
        \wrongchoice{equal to the Hamiltonian for the system}
    \end{choices}
\end{question}
}

%% Page 57
\element{gre}{
\begin{question}{GRE9277-Q44}
    A particle of mass $m$ on the Earth's surface is confined to move on the parabolic curve $y=ax^2$ where $y$ is up.
    Which of the following is a Langrangian for the particle?
    \begin{choices}
        \wrongchoice{$L = \dfrac{1}{2}m\dot{y}^2\left(1+\dfrac{1}{4ay}\right) - mgy$}
        \wrongchoice{$L = \dfrac{1}{2}m\dot{y}^2\left(1-\dfrac{1}{4ay}\right) - mgy$}
        \wrongchoice{$L = \dfrac{1}{2}m\dot{x}^2\left(1+\dfrac{1}{4ax}\right) - mgx$}
        \wrongchoice{$L = \dfrac{1}{2}m\dot{x}^2\left(1+4a^2x^2\right) - mgx$}
        \wrongchoice{$L = \dfrac{1}{2}m\dot{x}^2 + \dfrac{1}{2}m\dot{y}^2 + mgy$}
    \end{choices}
\end{question}
}

\element{gre}{
\begin{question}{GRE9277-Q45}
    A ball is dropped from a height $h$.
    As it bounces off the floor, its speed is 80 percent of what it was just before it hit the floor.
    The ball will then rise to a height of most nearly:
    \begin{multicols}{2}
    \begin{choices}
        \wrongchoice{$0.94 h$}
        \wrongchoice{$0.80 h$}
        \wrongchoice{$0.75 h$}
        \wrongchoice{$0.64 h$}
        \wrongchoice{$0.50 h$}
    \end{choices}
    \end{multicols}
\end{question}
}

%% Questions 46-47
\element{gre}{
\begin{question}{GRE9277-Q46}
    \begin{center}
        % \newcommand
    \end{center}
    Isotherms and coexistence curves are shown in the $pV$ diagram above for a liquid-gas system.
    The dashed lines are the boundaries of the labeled regions.
    %% start question
    Which numbered curve is the critical isotherm?
    \begin{multicols}{3}
    \begin{choices}
        \wrongchoice{$1$}
        \wrongchoice{$2$}
        \wrongchoice{$3$}
        \wrongchoice{$4$}
        \wrongchoice{$5$}
    \end{choices}
    \end{multicols}
\end{question}
}

\element{gre}{
\begin{question}{GRE9277-Q47}
    \begin{center}
        % \newcommand
    \end{center}
    Isotherms and coexistence curves are shown in the $pV$ diagram above for a liquid-gas system.
    The dashed lines are the boundaries of the labeled regions.
    %% start question
    In which region are the liquid and the vapor in equilibrium with each other?
    \begin{multicols}{3}
    \begin{choices}[o]
        \wrongchoice{$A$}
        \wrongchoice{$B$}
        \wrongchoice{$C$}
        \wrongchoice{$D$}
        \wrongchoice{$E$}
    \end{choices}
    \end{multicols}
\end{question}
}

%% page 58
\element{gre}{
\begin{question}{GRE9277-Q48}
    \begin{center}
        % \newcommand
    \end{center}
    Isotherms and coexistence curves are shown in the $pV$ diagram above for a liquid-gas system.
    The dashed lines are the boundaries of the labeled regions.
    %% start question
    In which region are the liquid and the vapor in equilibrium with each other?
    \begin{multicols}{3}
    \begin{choices}[o]
        \wrongchoice{$A$}
        \wrongchoice{$B$}
        \wrongchoice{$C$}
        \wrongchoice{$D$}
        \wrongchoice{$E$}
    \end{choices}
    \end{multicols}
\end{question}
}


\endinput


