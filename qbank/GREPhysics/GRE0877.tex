

%% http://ctan.mirrorcatalogs.com/macros/latex/contrib/physics/physics.pdf
%%--------------------------------------------------------------------------------


%% GRE Physics 0877 Practice Exam
%%----------------------------------------

%% Page 12
\element{GREphysics}{
\begin{question}{GRE0877-Q01}
    A ball is thrown out of the passenger window of a car moving
        to the right (ignore air resistance). 
    If the ball is thrown out perpendicular to the velocity of the car,
        which of the following best depicts the path the ball takes,
        as viewed from above?
    \begin{choices}
        \wrongchoice{\includegraphics[keepaspectratio,scale=0.95]{GRE0877-Q01-A}}
        \wrongchoice{\includegraphics[keepaspectratio,scale=0.95]{GRE0877-Q01-B}}
      \correctchoice{\includegraphics[keepaspectratio,scale=0.95]{GRE0877-Q01-C}}
        \wrongchoice{\includegraphics[keepaspectratio,scale=0.95]{GRE0877-Q01-D}}
        \wrongchoice{\includegraphics[keepaspectratio,scale=0.95]{GRE0877-Q01-E}}
    \end{choices}
\end{question}
}

\element{GREphysics}{
\begin{question}{GRE0877-Q02}
    An object is thrown horizontally from the open window of a building. 
    If the initial speed of the object is \SI{20}{\meter\per\second}
        and it hits the ground \SI{2.0}{\second} later,
        from what height was it thrown? 
    (Neglect air resistance and assume the ground is level.)
    \begin{multicols}{2}
    \begin{choices}
        \wrongchoice{\SI{4.9}{\meter}}
        \wrongchoice{\SI{9.8}{\meter}}
        \wrongchoice{\SI{10.0}{\meter}}
        \wrongchoice{\SI{19.6}{\meter}}
        \wrongchoice{\SI{39.2}{\meter}}
    \end{choices}
    \end{multicols}
\end{question}
}

\element{GREphysics}{
\begin{question}{GRE0877-Q03}
    A resistor in a circuit dissipates energy   
        at a rate of \SI{1}{\watt}. 
    If the voltage across the resistor is doubled,
        what will be the new rate of energy dissipation?
    \begin{multicols}{2}
    \begin{choices}
        \wrongchoice{\SI{0.25}{\watt}}
        \wrongchoice{\SI{0.5}{\watt}}
        \wrongchoice{\SI{1}{\watt}}
        \wrongchoice{\SI{2}{\watt}}
        \wrongchoice{\SI{4}{\watt}}
    \end{choices}
    \end{multicols}
\end{question}
}


%% Page 14
\element{GREphysics}{
\begin{question}{GRE0877-Q04}
    An infinitely long, straight wire carrying current $I_1$
        passes through the center of a circular loop of
        wire carrying current $I_2$, as shown above. 
    \begin{center}
        %% NOTE: tikz
    \end{center}
    The long wire is perpendicular to the plane of the loop.
    Which of the following describes the magnetic force on the loop?
    \begin{choices}
        \wrongchoice{Outward, along a radius of the loop.}
        \wrongchoice{Inward, along a radius of the loop.}
        \wrongchoice{Upward, along the axis of the loop.}
        \wrongchoice{Downward, along the axis of the loop.}
        \wrongchoice{There is no magnetic force on the loop.}
    \end{choices}
\end{question}
}

\element{GREphysics}{
\begin{question}{GRE0877-Q05}
    De Broglie hypothesized that the linear momentum
        and wavelength of a free massive particle are
        related by which of the following constants?
    \begin{choices}
        \wrongchoice{Planck's constant}
        \wrongchoice{Boltzmann's constant}
        \wrongchoice{The Rydberg constant}
        \wrongchoice{The speed of light}
        \wrongchoice{Avogadro's number}
    \end{choices}
\end{question}
}

\element{GREphysics}{
\begin{question}{GRE0877-Q06}
    An atom has filled $n=1$ and $n=2$ levels. 
    How many electrons does the atom have?
    \begin{multicols}{3}
    \begin{choices}
        \wrongchoice{\num{2}}
        \wrongchoice{\num{4}}
        \wrongchoice{\num{6}}
        \wrongchoice{\num{8}}
        \wrongchoice{\num{10}}
    \end{choices}
    \end{multicols}
\end{question}
}

\element{GREphysics}{
\begin{question}{GRE0877-Q07}
    The root-mean-square speed of molecules of
        mass $m$ in an ideal gas at temperature $T$ is
    \begin{multicols}{2}
    \begin{choices}
        \wrongchoice{$0$}
        \wrongchoice{$\sqrt\frac{2kT}{m}}$}
        \wrongchoice{$\sqrt\frac{3kT}{m}}$}
        \wrongchoice{$\sqrt\frac{8kT}{\pim}}$}
        \wrongchoice{$\frac{kT}{m}}$}
    \end{choices}
    \end{multicols}
\end{question}
}

\element{GREphysics}{
\begin{question}{GRE0877-Q08}
    The energy from electromagnetic waves in equilibrium
        in a cavity is used to melt ice. 
    If the Kelvin temperature of the cavity is increased
        by a factor of two, the mass of ice that can be melted in
        a fixed amount of time is increased by a factor of
    \begin{multicols}{2}
    \begin{choices}
        \wrongchoice{\num{2}}
        \wrongchoice{\num{4}}
        \wrongchoice{\num{8}}
        \wrongchoice{\num{16}}
        \wrongchoice{\num{32}}
    \end{choices}
    \end{multicols}
\end{question}
}


%% Page 16
\element{GREphysics}{
\begin{question}{GRE0877-Q09}
    \begin{center}
        %% NOTE: tikz
    \end{center}
    The figure above represents the orbit of a planet around a star,
        $S$, and the marks divide the orbit into 14 equal time intervals,
        $t = \frac{T}{14}$, where $T$ is the orbital period. 
    If the only force acting on the planet is Newtonian gravitation,
        then true statements about the situation include which of the following?
    \begin{itemize}
        \item[I.] Area $A$ = area $B$
        \item[II.] The star $S$ is at one focus of an elliptically shaped orbit.
        \item[III.] $T^2 = Ca^3$, where $a$ is the semimajor axis of the ellipse and $C$ is a constant.
    \end{itemize}
    \begin{multicols}{2}
    \begin{choices}
        \wrongchoice{I only}
        \wrongchoice{II only}
        \wrongchoice{I and II only}
        \wrongchoice{II and III only}
        \wrongchoice{I, II, and III}
    \end{choices}
    \end{multicols}
\end{question}
}

\element{GREphysics}{
\begin{question}{GRE0877-Q10}
    A massless spring with force constant $k$ launches a ball of mass $m$.
    In order for the ball to reach a speed $u$,
        by what displacement $s$ should the spring be compressed
    \begin{multicols}{2}
    \begin{choices}
        \wrongchoice{$s = v\sqrt{\frac{k}{m}}$}
        \wrongchoice{$s = v\sqrt{\frac{m}{v}}$}
        \wrongchoice{$s = v\sqrt{\frac{2k}{m}}$}
        \wrongchoice{$s = v\frac{m}{k}$}
        \wrongchoice{$s = v^2\frac{m}{2k}$}
    \end{choices}
    \end{multicols}
\end{question}
}

\element{GREphysics}{
\begin{question}{GRE0877-Q11}
    A quantum mechanical harmonic oscillator has an angular frequency $\omega$.
    The Schr\"{o}dinger equation predicts that the ground state energy of the oscillator will be
    \begin{multicols}{2}
    \begin{choices}
        \wrongchoice{$-\frac{1}{2}\hbar\omega$}
        \wrongchoice{$0$}
        \wrongchoice{$\hbar\omega$}
        \wrongchoice{$\hbar\omega$}
        \wrongchoice{$\frac{3}{2}\hbar\omega$}
    \end{choices}
    \end{multicols}
\end{question}
}

\element{GREphysics}{
\begin{question}{GRE0877-Q12}
    In the Bohr model of the hydrogen atom,
        the linear momentum of the electron at radius $r_n$
        is given by which of the following? 
    ($n$ is the principal quantum number.)
    \begin{multicols}{2}
    \begin{choices}
        \wrongchoice{$n\hbar$}
        \wrongchoice{$nr_n\hbar$}
        \wrongchoice{$\frac{n^2\hbar}{r_n}$}
        \wrongchoice{$n^2 r_n \hbar$}
        \wrongchoice{$\frac{n^2\hbar}{r_n}$}
    \end{choices}
    \end{multicols}
\end{question}
}


%% Page 18
\element{GREphysics}{
\begin{question}{GRE0877-Q13}
    The figure below represents a log-log plot
        of variable $y$ versus variable $x$.
    \begin{center}
        %% NOTE: pgfplots
    \end{center}
    The origin represents the point $x=1$ and $y=1$.
    Which of the following gives the approximate
        functional relationship between y and x ?
    \begin{multicols}{2}
    \begin{choices}
        \wrongchoice{$y = 6 x$}
        \wrongchoice{$y = \frac{1}{2}x + 6$}
        \wrongchoice{$y = 6 x + \frac{1}{2}$}
        \wrongchoice{$y = \frac{1}{6}x^2$}
        \wrongchoice{$y = 6 x^2$}
    \end{choices}
    \end{multicols}
\end{question}
}


%% Page 20
\element{GREphysics}{
\begin{question}{GRE0877-Q14}
    Two experimental techniques determine the
    mass of an object to be 11 ± 1 kg and 10 ± 2 kg.
    These two measurements can be combined to
    give a weighted average. The uncertainty of the
    weighted average is equal to which of the
    following?
    \begin{multicols}{2}
    \begin{choices}
        \wrongchoice{\SI[parse-numbers=false]{\frac{1}{2}}{\kilo\gram}}
        \wrongchoice{\SI[parse-numbers=false]{\frac{2}{\sqrt{5}}}{\kilo\gram}}
        \wrongchoice{\SI[parse-numbers=false]{\frac{2}{\sqrt{3}}}{\kilo\gram}}
        \wrongchoice{\SI[parse-numbers=false]{2}{\kilo\gram}}
        \wrongchoice{\SI[parse-numbers=false]{\sqrt{5}}{\kilo\gram}}
    \end{choices}
    \end{multicols}
\end{question}
}

\element{GREphysics}{
\begin{question}{GRE0877-Q15}
    If the five lenses shown below are made of the same material,
        which lens has the shortest positive focal length?
    \begin{multicols}{2}
    \begin{choices}
        %% NOTE: add tikz options
        \wrongchoice{}
    \end{choices}
    \end{multicols}
\end{question}
}


%% Page 22
\element{GREphysics}{
\begin{question}{GRE0877-Q16}
    Unpolarized light is incident on a pair of ideal linear polarizers
        whose transmission axes make an angle of \ang{45} with each other. 
    The transmitted light intensity through both polarizers is what
        percentage of the incident intensity?
    \begin{multicols}{2}
    \begin{choices}
        \wrongchoice{\SI{100}{\percent}}
        \wrongchoice{\SI{75}{\percent}}
        \wrongchoice{\SI{50}{\percent}}
        \wrongchoice{\SI{25}{\percent}}
        \wrongchoice{\SI{0}{\percent}}
    \end{choices}
    \end{multicols}
\end{question}
}

\element{GREphysics}{
\begin{question}{GRE0877-Q17}
    A very long, thin, straight wire carries a uniform
    charge density of l per unit length. Which of the
    following gives the magnitude of the electric field
    at a radial distance r from the wire?
    \begin{multicols}{2}
    \begin{choices}
        \wrongchoice{$\frac{1}{2\pi\epsilon_0}\frac{\lambda}{r}$}
        \wrongchoice{$\frac{1}{2\pi\epsilon_0}\frac{r}{\lambdar}$}
        \wrongchoice{$\frac{1}{2\pi\epsilon_0}\frac{\lambda}{r^2}$}
        \wrongchoice{$\frac{1}{4\pi\epsilon_0}\frac{\lambda^2}{r^2}$}
        \wrongchoice{$\frac{1}{4\pi\epsilon_0}\lambda\ln{}r$}
    \end{choices}
    \end{multicols}
\end{question}
}

\element{GREphysics}{
\begin{question}{GRE0877-Q18}
    The bar magnet shown in the figure below is moved
        completely through the loop. 
    \begin{center}
        %% NOTE: tikz?
    \end{center}
    Which of the following is a true statement about the
        direction of the current flow between the two
        points $a$ and $b$ in the circuit?
    \begin{choices}
        \wrongchoice{No current flows between $a$ and $b$ as the magnet passes through the loop.}
        \wrongchoice{Current flows from $a$ to $b$ as the magnet passes through the loop.}
        \wrongchoice{Current flows from $b$ to $a$ as the magnet passes through the loop.}
        \wrongchoice{Current flows from $a$ to $b$ as the magnet enters the loop and from $b$ to $a$ as the magnet leaves the loop.}
        \wrongchoice{Current flows from $b$ to $a$ as the magnet enters the loop and from $a$ to $b$ as the magnet leaves the loop}
    \end{choices}
\end{question}
}


%% Page 24
\element{GREphysics}{
\begin{question}{GRE0877-Q19}
    The surface of the Sun has a temperature close to
        \SI{6 000}{\kelvin} and it emits a blackbody (Planck)
        spectrum that reaches a maximum near \SI{500}{\nano\meter}.
    For a body with a surface temperature close to \SI{300}{\kelvin},
        at what wavelength would the thermal spectrum reach a maximum?
    \begin{mutlicols}{2}
    \begin{choices}
        \wrongchoice{\SI{10}{\micro\meter}}
        \wrongchoice{\SI{100}{\micro\meter}}
        \wrongchoice{\SI{10}{\milli\meter}}
        \wrongchoice{\SI{100}{\milli\meter}}
        \wrongchoice{\SI{10}{\meter}}
    \end{choices}
    \end{mutlicols}
\end{question}
}

\element{GREphysics}{
\begin{question}{GRE0877-Q20}
    At the present time, the temperature of the universe
        (i.e., the microwave radiation background)
        is about \SI{3}{\kelvin}.
    When the temperature was \SI{12}{\kelvin},
        typical objects in the universe, such as galaxies, were
    \begin{choices}
        \wrongchoice{one-quarter as distant as they are today}
        \wrongchoice{one-half as distant as they are today}
        \wrongchoice{separated by about the same distances as they are today}
        \wrongchoice{two times as distant as they are today}
        \wrongchoice{four times as distant as they are today}
    \end{choices}
\end{question}
}

\element{GREphysics}{
\begin{question}{GRE0877-Q21}
    For an adiabatic process involving an ideal gas having
        volume $V$ and temperature $T$, which
        of the following is constant?
    ($\gamma = \frac{C_p}{C_V}$)
    \begin{multicols}{2}
    \begin{choices}
        \wrongchoice{$TV$}
        \wrongchoice{$TV^{\gamma}$}
        \wrongchoice{$TV^{\gamma-1}$}
        \wrongchoice{$T^{\gamma}V$}
        \wrongchoice{$T^{\gamma}V^{\gamma-1}$}
    \end{choices}
    \end{multicols}
\end{question}
}

\element{GREphysics}{
\begin{question}{GRE0877-Q22}
    An electron has total energy equal to four times its rest energy. 
    The momentum of the electron is
    \begin{multicols}{2}
    \begin{choices}
        \wrongchoice{$m_e c$}
        \wrongchoice{$\sqrt{2} m_e c$}
        \wrongchoice{$\sqrt{15} m_e c$}
        \wrongchoice{$4 m_e c$}
        \wrongchoice{$2\sqrt{15} m_e c$}
    \end{choices}
    \end{multicols}
\end{question}
}

\element{GREphysics}{
\begin{question}{GRE0877-Q23}
    Two spaceships approach Earth with equal speeds,
        as measured by an observer on Earth,
        but from opposite directions. 
    A meterstick on one spaceship is measured to be
        \SI{60}{\centi\meter} long by an occupant
        of the other spaceship. 
    What is the speed of each spaceship,
        as measured by the observer on Earth?
    \begin{multicols}{2}
    \begin{choices}
        \wrongchoice{$0.4 c$}
        \wrongchoice{$0.5 c$}
        \wrongchoice{$0.6 c$}
        \wrongchoice{$0.7 c$}
        \wrongchoice{$0.8 c$}
    \end{choices}
    \end{multicols}
\end{question}
}

\element{GREphysics}{
\begin{question}{GRE0877-Q24}
    A meter stick with a speed of $0.8c$ moves past an observer. 
    In the observer's reference frame, how long does it take
        the stick to pass the observer?
    \begin{multicols}{2}
    \begin{choices}
        \wrongchoice{\SI{1.6}{\nano\second}}
        \wrongchoice{\SI{2.5}{\nano\second}}
        \wrongchoice{\SI{4.2}{\nano\second}}
        \wrongchoice{\SI{6.9}{\nano\second}}
        \wrongchoice{\SI{8.3}{\nano\second}}
    \end{choices}
    \end{multicols}
\end{question}
}


%% Page 26


\endinput


