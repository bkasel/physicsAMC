
%%--------------------------------------------------
%% Schaum's Outline of Applied Physics
%%--------------------------------------------------


%% Chapter 6: Friction
%%--------------------------------------------------


%% Schaum's Numeric Problems
%%--------------------------------------------------
\element{schaums-num}{
\begin{questionmultx}{ch06-num-Q01}
    Find the length, $l$, of the hypotenuse of a right triangle whose legs are \SI{483}{\meter} and \SI{620}{\meter} long.
    Report $l/\si{\meter}$ below.
    \AMCnumericChoices{785.93}{
        vertical=false,
        digits=5,decimals=2,sign=true,
        borderwidth=0pt,backgroundcol=white,approx=5
    }
\end{questionmultx}
}



6.1. Ships are often built on ways that slope down to a nearby body of water. Often a ship is launched before most of its
interior and superstructure have been installed, and is completed when afloat. Is this done because the added weight
would cause the ship to slide down the ways prematurely?
6.2. A box is being pushed across a floor. Can the frictional force on the box ever exceed its own weight? Can the frictional
force ever exceed the applied force?
6.3. An 80-lb wooden crate rests on a horizontal wooden floor. If the coefficient of static friction is 0.5, how much force
is needed to set the crate in motion?
6.4. A force of 300 N is sufficient to keep a 100-kg wooden crate moving at constant velocity across a wooden floor. What
is the coefficient of kinetic friction?
6.5. A force of 1000 N is applied to a 1200-kg car. If the coefficient of rolling friction is 0.04, what is the car’s acceleration?
6.6. The coefficients of static and kinetic friction for stone on wood are, respectively, 0.5 and 0.4. If a 150-kg stone statue is
pushed with just enough force to start it moving across a wooden floor and the same force continues to act afterward,
find the statue’s acceleration.
6.7. A car whose brakes are locked skids to a stop 70 m from an initial velocity of 80 km/h. Find the coefficient of kinetic
friction.
6.8. A driver sees a horse on the road and applies the brakes so hard that they lock and the car skids to a stop in 24 m. The
road is level, and the coefficient of kinetic friction between tires and road is 0.7. How fast was the car going when the
brakes were applied?
6.9. A truck moving at 100 km/h carries a steel girder that rests on its wooden floor. What is the minimum time in which
the truck can come to a stop without the girder moving forward? The coefficient of static friction between steel and
wood is 0.5.

6.10. A car with its brakes locked will remain stationary on an inclined plane of dry concrete when the plane is at an angle
of less than 45 ◦ with the horizontal. What is the coefficient of static friction of rubber tires on dry concrete?
6.11. A steel ramp is to be built for sliding blocks of ice from a refrigeration plant down to ground level. If μ = 0.05, find
the angle with the horizontal at which the ice will slide at constant velocity.
6.12. A box slides down a plane 8 m long that is inclined at an angle of 30 ◦ with the horizontal. If the box starts from rest
and μ = 0.25, find (a) the acceleration of the box, (b) its velocity at the bottom of the plane, and (c) the time required
for it to reach the bottom.
6.13. If the box of Prob. 6.10 has a mass of 60 kg, how much force is needed to move it up the plane (a) at constant velocity
and (b) with an acceleration of 2 m/s 2 ?
6.14. A skier stands on a 5 ◦ slope. If the coefficient of static friction is 0.1, does the skier start to slide down?

6.1. No. The frictional force that prevents the ship from sliding down the ways increases in proportion as the ship's weight increases.



6.2. Yes; no
6.9. 5.67 s
6.3. 40 lb
6.4. 0.306 
6.5. 0.44 m/s 2
6.6. 0.98 m/s 2
6.7. 0.36
6.8. 18 m/s = 65 km/h
6.9. 5.67 s
6.10. 1.0
6.11. 3 ◦
6.12. (a) 2.78 m/s 2 (b) 6.67 m/s (c) 2.40 s
6.13. (a) 421 N (b) 541 N
6.14. No



\endinput

