
%%--------------------------------------------------
%% Schaum's Outline of Applied Physics
%%--------------------------------------------------


%% Chapter 7: Energy
%%--------------------------------------------------


%% Schaum's Numeric Problems
%%--------------------------------------------------
\element{schaums-num}{
\begin{questionmultx}{ch07-num-Q01}
    Find the length, $l$, of the hypotenuse of a right triangle whose legs are \SI{483}{\meter} and \SI{620}{\meter} long.
    Report $l/\si{\meter}$ below.
    \AMCnumericChoices{785.93}{
        vertical=false,
        digits=5,decimals=2,sign=true,
        borderwidth=0pt,backgroundcol=white,approx=5
    }
\end{questionmultx}
}


7.1. Under what circumstances (if any) is no work done on a moving object even though a net force acts on it?
7.2. A golf ball and a Ping-Pong ball are dropped in a vacuum chamber. When they have fallen halfway to the bottom,
how do their velocities compare? Their kinetic energies? Their potential energies?
7.3. The earth exerts a gravitational force of 2 × 10 20 N on the moon, and the moon travels 2.4 × 10 9 m each time it orbits
the earth. How much work does the earth do on the moon in each orbit?
7.4. (a) How much work must be done to raise a 1100-kg car 2 m above the ground? (b) What is the car’s potential energy
afterward?
7.5. A 20-lb object is raised to a height of 40 ft above the ground. (a) How much work was done? (b) What is the potential
energy of the object? (c) If the object is dropped, what will its kinetic energy be just before it strikes the ground?


7.6. A boy pulls a wagon with a force of 45 N by means of a rope that makes an angle of 40 ◦ with the ground. How much
work does he do in moving the wagon 50 m?
7.7. A horse exerts a force of 200 lb while pulling a sled for 3 mi. (a) How much work does the horse do? (b) If the trip
takes 30 min, what is the power output of the horse in horsepower?
7.8. A road slopes upward so that it climbs 1 m for each 12 m of distance covered. A car whose weight is 10 kN moves
up the road at a constant velocity of 24 m/s. Find the minimum power the car’s engine is delivering.
7.9. A certain 80-kg mountain climber has an average power output of 0.1 hp. (a) How much work does she perform
in climbing a mountain 2000 m high? (b) How long does she take to climb the mountain? (c) What is her potential
energy at the top?
7.10. A man uses a rope and a system of pulleys to raise a 200-lb box to a height of 10 ft. He exerts a force of 60 lb on the
rope and pulls a total of 40 ft of rope through the pulleys. (a) How much work does he perform? (b) By how much is
the potential energy of the box increased? (c) If these answers are different, what do you think the reason is?
7.11. A total of 10 4 kg of water per second flows over a waterfall 25 m high. If 50 percent of the power this flow represents
could be converted into electricity, how many 100-W light bulbs could be supplied?
7.12. The four engines of a DC-8 airplane develops a total of 22 MW when its velocity is 240 m/s. How much force do the
engines exert?
7.13. Neglecting friction and air resistance, is more work needed to accelerate a car from 10 to 20 km/h or from 20 to
30 km/h?
7.14. A 3000-lb car has an engine which can deliver 80 hp to the driving wheels. What is the maximum velocity at which
the car can climb a 15 ◦ hill?
7.15. Find the kinetic energy of a 2-g (0.002-kg) insect when it is flying at 0.4 m/s.
7.16. The electrons in a television picture tube whose impacts on the screen produce the flashes of light that make up the
image have masses of 9.1 × 10 −31 kg and typical velocities of 3 × 10 7 m/s. What is the kinetic energy of such an
electron?
7.17. A 15-kg object initially at rest is raised to a height of 8 m by a force of 200 N. What is the velocity of the object at
this height?
7.18. A 7-kg iron shot is thrown 18 m. What was its minimum initial kinetic energy?
7.19. An 800-kg car moving at 70 km/h is carrying two 75-kg people. If the power output of the car’s engine is 30 kW, how
much time is needed for the car to reach a velocity of 110 km/h? Neglect friction and air resistance.
7.20. (a) What velocity does a 1-slug object have when its kinetic energy is 1 ft·lb? (b) What velocity does a 1-lb object
have when its kinetic energy is 1 ft·lb?
7.21. A stone is dropped from a height of 100 m. At what height is half of its energy potential and half kinetic?
7.22. A 10-g bullet has a velocity of 600 m/s when it leaves the barrel of a rifle. If the barrel is 60 cm long, find the average
force on the bullet while it is in the barrel.
7.23. A 16-lb shell has a velocity of 2000 ft/s when it leaves the barrel of a cannon. If the barrel is 10 ft long, find the
average force on the shell while it is in the barrel.
7.24. (a) An 8-N force pushes a 0.5-kg ball on a horizontal table for 3 m, starting from rest. If there is no friction, what is
the final KE of the ball? (b) The same force is used to raise the ball a height of 3 m, starting from rest. What is its
final KE now?

7.25. An 800-kg car moving at 6 m/s begins to coast down a hill 40 m high with its engine off. The driver applies the brakes
so that the car’s speed at the bottom of the hill is 20 m/s. How much energy was lost to friction?
7.26. One kilogram of water at 0 ◦ C contains 335 kJ of energy more than 1 kg of ice at 0 ◦ C. What is the mass equivalent of
this amount of energy?
7.27. Approximately 12 MJ of energy is liberated when 1 kg of dynamite explodes. How much matter is converted to energy
in this process?
7.28. A sedentary person uses energy at an average rate of about 70 W. (a) How many joules of energy does this person
use per day? (b) All this energy originates in the sun. How much matter is converted to energy per day to supply such
a person?



7.1. No work is done by a net force acting on a moving object when the force is perpendicular to the direction of the
object’s motion.
7.2. Their velocities are the same. The golf ball, which has the greater mass, has the greater KE and PE.
7.3. No work is done because the force on the moon is perpendicular to its direction of motion.
7.4. (a) 21.6 kJ (b) 21.6 kJ
7.5. (a) 800 ft·lb (b) 800 ft·lb (c) 800 ft·lb
7.6. 1.72 kJ


7.7. (a) 3.17 × 10 6 ft·lb (b) 3.2 hp
7.8. 20 kW
7.9.  (a) 1.57 MJ (b) 2000 ft·lb (c) 1.57 MJ
7.10. (a) 2400 ft·lb (b) 5 h 50 min (c) 400 ft·lb was used in doing work against frictional forces in the pulleys
7.11. 12,250 bulbs
7.12. 92 kN
7.13. From 20 to 30 km/h
7.14. 56.7 ft/s = 38.6 mi/h
7.15. 1.6 × 10 −4 J
7.16. 4.1 × 10 −16 J
7.17. 7.5 m/s
7.18. 0.62 kJ
7.19. 8.8 s
7.20. (a) 1.4 ft/s (b) 8 ft/s
7.21. 50 m
7.22. 3 kN
7.23. 10 5 lb
7.24. (a) 24 J (b) 9.3 J
7.25. 168 kJ
7.26. 3.7 × 10 −12 kg
7.27. 1.3 × 10 −10 kg
7.28. (a) 6.05 × 10 6 J (b) 6.72 × 10 −11 kg












\endinput

