
%%--------------------------------------------------
%% Schaum's Outline of Applied Physics
%%--------------------------------------------------


%% Chapter 5: Laws of Motion
%%--------------------------------------------------


%% Schaum's Numeric Problems
%%--------------------------------------------------
\element{schaums-num}{
\begin{questionmultx}{ch05-num-Q01}
    Find the length, $l$, of the hypotenuse of a right triangle whose legs are \SI{483}{\meter} and \SI{620}{\meter} long.
    Report $l/\si{\meter}$ below.
    \AMCnumericChoices{785.93}{
        vertical=false,
        digits=5,decimals=2,sign=true,
        borderwidth=0pt,backgroundcol=white,approx=5
    }
\end{questionmultx}
}


5.1. Since action and reaction forces are always equal in magnitude and opposite in direction, how can anything ever be
accelerated?
5.2. A horse is pulling a cart. (a) What is the force that causes the horse to move forward? (b) What is the force that causes
the cart to move forward?

5.3. Is it possible for something to have a downward acceleration greater than g? If so, how can this be accomplished?
5.4. (a) When a horizontal force equal to its weight is applied to an object on a frictionless surface, what is its acceleration?
(b) What is its acceleration when the force is applied vertically upward?
5.5. Compare the tension in the coupling between the first two cars in a train with the tension in the coupling between the
last two cars when (a) the train’s speed is constant and (b) the train is accelerating.
5.6. (a) What is the weight of 6 kg of potatoes? (b) What is the mass of 6 N of potatoes?
5.7. A force of 10 N is applied to (a) a body of mass 5 kg and (b) a body of weight 5 N. Find their accelerations.
5.8. (a) What is the weight of 2 slugs of salami? (b) What is the mass of 2 lb of salami?
5.9. (a) How much upward force is needed to support a 20-kg object at rest? (b) To give it an upward acceleration of
2 m/s 2 ? (c) To give it a downward acceleration of 2 m/s 2 ?
5.10. What is the acceleration of a 5-kg object suspended by a string when an upward force of (a) 39 N, (b) 49 N, and (c) 59
N is applied to the string?
5.11. How much applied force is needed to give an 8-N object (a) an upward acceleration of 2 m/s 2 and (b) a downward
acceleration of 2 m/s 2 ? (c) In what direction must the latter force act?
5.12. A net force of 12 N gives an object an acceleration of 4 m/s 2 . (a) What net force is needed to give it an acceleration
of 1 m/s 2 ? (b) An acceleration of 10 m/s 2 ?
5.13. A certain net force gives a 2-kg object an acceleration of 0.5 m/s 2 . What acceleration would the same force give a
10-kg object?
5.14. A 12,000-kg airplane launched by a catapult from an aircraft carrier is accelerated from 0 to 200 km/h in 3 s. (a) How
many times the acceleration of gravity is the airplane’s acceleration? (b) What is the average force that the catapult
exerts on the airplane?
5.15. When a 5-kg rifle is fired, the 9-g bullet receives an acceleration of 3 × 10 4 m/s 2 while it is in the barrel. (a) How
much force acts on the bullet? (b) Does any force act on the rifle? If so, how much and in what direction? (c) The
bullet is accelerated for 0.007 s. How fast does it leave the barrel of the rifle?
5.16. How much force is needed to accelerate a train whose mass is 1000 metric tons (1 metric ton = 1000 kg) from rest
to a velocity of 6 m/s in 2 min?
5.17. (a) How much force is needed to increase the velocity of a 6400-lb truck from 20 to 30 ft/s in 5 s? (b) How far does
the truck travel in this time?
5.18. (a) How much force is needed to decrease the velocity of a 6400-lb truck from 30 to 20 ft/s in 5 s? (b) How far does
the truck travel in this time?
5.19. A car strikes a stone wall at a velocity of 12 m/s. (a) The car is rigidly built, and the 60-kg driver comes to a stop in
a time of 0.05 s. How much force acts on her? (b) The car is built so that its front end collapses gradually, and the
driver comes to a stop in 0.2 s. How much force acts on her in this case?
5.20. A 0.05-kg snail goes from rest to a velocity of 0.01 m/s in 5 s. (a) How much force does it exert? (b) How far does it
go during this time?
5.21. A 430-g soccer ball moving toward a player at 8 m/s is kicked and flies off in the opposite direction at 12 m/s. If the
ball is in contact with the player’s foot for 0.01 s, find the average force on the ball.


5.22. The cable supporting a 2000-kg elevator can safely withstand a tension of 25 kN. What is the maximum upward
acceleration the elevator can have if the tension in the cable is not to exceed this figure?
5.23. An 800-N man stands on a scale in an elevator. What does the scale read when the elevator is (a) ascending at a
constant velocity of 3 m/s, (b) ascending at a constant acceleration of 0.8 m/s 2 , (c) descending at a constant velocity
of 3 m/s, (d) descending at a constant acceleration of 0.8 m/s 2 , and (e) in free fall because the cable has broken?
5.24. An 80-kg woman stands on a scale in an elevator. When it starts to move, the scale reads 700 N. (a) Is the elevator
moving upward or downward? (b) Is its velocity constant? If so, what is it? If not, what is the elevator’s acceleration?
5.25. Two boxes, one of mass 20 kg and the other of mass 30 kg, are sliding down a frictionless inclined plane that makes
an angle of 25 ◦ with the horizontal. Find their respective accelerations.
5.26. A force of 50 lb is used to pull a 50-lb crate up a frictionless plane that is inclined at 30 ◦ with the horizontal. Find the
acceleration of the crate.
5.27. An 800-kg car is towed up an 8 ◦ hill by a rope attached to a truck. The tension in the rope is 2000 N, and there is no
frictional resistance to the car’s motion. How much time is needed to tow the car for 50 m starting from rest?




5.1. The action and reaction forces always act on different bodies.
5.2. (a) The reaction force the ground exerts on its feet.
     (b) The force the horse exerts on it.
5.3. Yes, by an applied downward force in addition to the downward force of gravity.
5.4. (a) g (b) 0
5.5. (a) The tensions are the same.
     (b) The front coupling is under the greater tension because it has a larger mass behind it to accelerate.


5.6. (a) 59 N (b) 0.61 kg
5.7. (a) 2 m/s 2 (b) 19.6 m/s 2
5.8. (a) 64 lb (b) 0.0625 slug
5.9. (a) 196 N (b) 236 N (c) 156 N
5.10. (a) 2 m/s 2 downward (b) 0 (c) 2 m/s 2 upward
5.11. (a) 9.63 N (b) 6.37 N (c) Upward
5.12. (a) 3N (b) 30 N
5.13. 0.1 m/s 2
5.14. (a) 1.89 g (b) 222 kN
5.15. (a) 270 N (b) 270 N backward (c) 210 m/s
5.16. 5 × 10 4 N
5.17. (a) 400 lb (b) 125 ft
5.18. (a) 400 lb (b) 125 ft
5.19. (a) 14.4 kN (b) 3.6 kN
5.20. (a) 10 −4 N (b) 0.025 m
5.21. 860 N
5.22. 2.7 m/s 2
5.23. (a) 800 N (b) 865 N (c) 800 N (d) 735 N (e) 0
5.24. (a) Downward (b) No; 1.05 m/s 2
5.25. 4.14 m/s 2; 4.14 m/s 2
5.26. 16 ft/s 2
5.27. 9.4 s




\endinput

