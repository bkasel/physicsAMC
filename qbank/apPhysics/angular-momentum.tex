

%% AP Physics MC Questions Archive
%%----------------------------------------


%% Angular Momentum
%%----------------------------------------
\element{ap}{
\begin{question}{angular-momentum-q01}
    A \SI{4}{\kilo\gram} object moves in a circle of radius \SI{8}{\meter} at a constant speed of \SI{2}{\meter\per\second}.
    What is the angular momentum of the object with respect to an axis perpendicular to the circle and through its center?
    \begin{multicols}{3}
    \begin{choices}
        \wrongchoice{\SI{2}{\newton\second}}
        \wrongchoice{\SI{6}{\newton\meter\per\kilo\gram}}
        \wrongchoice{\SI{12}{\kilo\gram\meter\per\second}}
        \wrongchoice{\SI{24}{\meter\squared\per\second}}
      \correctchoice{\SI{10}{\kilo\gram\meter\squared\per\second}}
    \end{choices}
    \end{multicols}
\end{question}
}

\element{ap}{
\begin{question}{angular-momentum-q02}
    Angular momentum can be calculated by taking the cross product of:
    \begin{choices}
        \wrongchoice{Radius and force}
      \correctchoice{Radius and linear momentum}
        \wrongchoice{Radius and rotational velocity}
        \wrongchoice{Force and rotational velocity}
        \wrongchoice{Torque and rotational velocity}
    \end{choices}
\end{question}
}

\element{ap}{
\begin{question}{angular-momentum-q03}
    A uniform sphere has a moment of inertia \SI{2}{\kilo\gram\meter\squared} and an angular velocity or \SI{5}{\radian\per\second}.
    What is its angular momentum?
    \begin{multicols}{2}
    \begin{choices}
      \correctchoice{\SI{10}{\kilo\gram\meter\squared\per\second}}
        \wrongchoice{\SI{25}{\kilo\gram\meter\squared\per\second}}
        \wrongchoice{\SI{50}{\kilo\gram\meter\squared\per\second}}
        \wrongchoice{\SI{100}{\kilo\gram\meter\squared\per\second}}
        \wrongchoice{\SI{500}{\kilo\gram\meter\squared\per\second}}
    \end{choices}
    \end{multicols}
\end{question}
}

\element{ap}{
\begin{question}{angular-momentum-q04}
    A slender rod with mass \SI{0.6}{\kilo\gram} and length \SI{10}{\centi\meter} has an angular velocity of \SI{2}{\meter\per\second}.
    The moment of inertia for a slender rod is calculated by $\dfrac{1}{12}ML^2$.
    What is its angular momentum?
    \begin{multicols}{3}
    \begin{choices}
      \correctchoice{\SI{1}{\gram\meter\squared\per\second}}
        \wrongchoice{\SI{3}{\gram\meter\squared\per\second}}
        \wrongchoice{\SI{6}{\gram\meter\squared\per\second}}
        \wrongchoice{\SI{9}{\gram\meter\squared\per\second}}
        \wrongchoice{\SI{64}{\gram\meter\squared\per\second}}
    \end{choices}
    \end{multicols}
\end{question}
}

\element{ap}{
\begin{question}{angular-momentum-q05}
    A solid cylinder with diameter \SI{20}{\centi\meter} has an angular velocity of \SI{10}{\meter\per\second} and angular momentum of \SI{2}{\kilo\gram\meter\squared\per\second}.
    What is its mass?
    \begin{multicols}{3}
    \begin{choices}
        \wrongchoice{\SI{0.1}{\kilo\gram}}
        \wrongchoice{\SI{1}{\kilo\gram}}
        \wrongchoice{\SI{2}{\kilo\gram}}
        \wrongchoice{\SI{5}{\kilo\gram}}
      \correctchoice{\SI{10}{\kilo\gram}}
    \end{choices}
    \end{multicols}
\end{question}
}

\element{ap}{
\begin{question}{angular-momentum-q06}
    What is the ratio of angular momentum for a solid sphere and a solid cylinder with the same angular velocities,
        masses and radii?
    \begin{multicols}{3}
    \begin{choices}
        \wrongchoice{$2:5$}
      \correctchoice{$4:5$}
        \wrongchoice{$1:1$}
        \wrongchoice{$5:4$}
        \wrongchoice{$5:2$}
    \end{choices}
    \end{multicols}
\end{question}
}

\element{ap}{
\begin{question}{angular-momentum-q07}
    A unicycle travels at \SI{5.0}{\meter\per\second}.
    Its wheel has a radius of \SI{0.4}{\meter} and a moment of inertia of \SI{20}{\kilo\gram\meter\squared}.
    What is its angular momentum?
    \begin{multicols}{2}
    \begin{choices}
        \wrongchoice{\SI{125}{\kilo\gram\meter\squared\per\second}}
        \wrongchoice{\SI{150}{\kilo\gram\meter\squared\per\second}}
        \wrongchoice{\SI{200}{\kilo\gram\meter\squared\per\second}}
        \wrongchoice{\SI{225}{\kilo\gram\meter\squared\per\second}}
      \correctchoice{\SI{250}{\kilo\gram\meter\squared\per\second}}
    \end{choices}
    \end{multicols}
\end{question}
}

\element{ap}{
\begin{question}{angular-momentum-q08}
    The angular position of a sphere is given by the equation $\theta(t) = 2t^2 + 3t + 6$,
        where $\theta$ is in radians and $t$ is in seconds.
    If the moment of inertia for the sphere is \SI{4}{\kilo\gram\meter\squared},
        then what is its angular momentum at $t=\SI{3}{\second}$?
    \begin{multicols}{2}
    \begin{choices}
        \wrongchoice{\SI{15}{\kilo\gram\meter\squared\per\second}}
        \wrongchoice{\SI{27}{\kilo\gram\meter\squared\per\second}}
        \wrongchoice{\SI{30}{\kilo\gram\meter\squared\per\second}}
        \wrongchoice{\SI{54}{\kilo\gram\meter\squared\per\second}}
      \correctchoice{\SI{60}{\kilo\gram\meter\squared\per\second}}
    \end{choices}
    \end{multicols}
\end{question}
}

\element{ap}{
\begin{question}{angular-momentum-q09}
    A unicycle travels at \SI{6.0}{\meter\per\second}.
    Its wheel has an angular momentum of \SI{300}{\kilo\gram\meter\squared\per\second} and a moment of inertia of \SI{10}{\kilo\gram\meter\squared}.
    What is its radius?
    \begin{multicols}{3}
    \begin{choices}
        \wrongchoice{\SI{0.1}{\meter}}
      \correctchoice{\SI{0.2}{\meter}}
        \wrongchoice{\SI{0.3}{\meter}}
        \wrongchoice{\SI{0.4}{\meter}}
        \wrongchoice{\SI{0.5}{\meter}}
    \end{choices}
    \end{multicols}
\end{question}
}

\element{ap}{
\begin{question}{angular-momentum-q10}
    The cross product of radius and linear momentum is equivalent to all of the following \emph{except}:
    \begin{choices}
        \wrongchoice{the product of rotational inertia and angular velocity}
        \wrongchoice{the product of moment of inertia and angular velocity}
        \wrongchoice{angular momentum}
      \correctchoice{kinetic energy}
        \wrongchoice{the derivative of rotational kinetic energy with respect to angular velocity}
    \end{choices}
\end{question}
}

\element{ap}{
\begin{question}{angular-momentum-q11}
    What is the angular momentum of a wheel with rotational inertia \SI{5}{\kilo\gram\meter\squared} and angular velocity \SI{4}{\radian\per\second}?
    \begin{multicols}{2}
    \begin{choices}
        \wrongchoice{\SI{10}{\kilo\gram\meter\squared\per\second}}
        \wrongchoice{\SI{10}{\kilo\gram\meter\cubed\per\second}}
      \correctchoice{\SI{20}{\kilo\gram\meter\squared\per\second}}
        \wrongchoice{\SI{20}{\kilo\gram\meter\cubed\per\second}}
        \wrongchoice{\SI{40}{\kilo\gram\meter\squared\per\second}}
    \end{choices}
    \end{multicols}
\end{question}
}

\element{ap}{
\begin{question}{angular-momentum-q12}
    What is the angular momentum of a hoop with radius \SI{1}{\meter} and mass \SI{5}{\kilo\gram} that moves with angular velocity \SI{4}{\radian\per\second}?
    \begin{multicols}{2}
    \begin{choices}
        \wrongchoice{\SI{5}{\kilo\gram\meter\squared\per\second}}
        \wrongchoice{\SI{10}{\kilo\gram\meter\squared\per\second}}
      \correctchoice{\SI{20}{\kilo\gram\meter\squared\per\second}}
        \wrongchoice{\SI{25}{\kilo\gram\meter\squared\per\second}}
        \wrongchoice{\SI{30}{\kilo\gram\meter\squared\per\second}}
    \end{choices}
    \end{multicols}
\end{question}
}

\element{ap}{
\begin{question}{angular-momentum-q13}
    What is the angular velocity of a hoop of radius \SI{1}{\meter} and mass \SI{3}{\kilo\gram} that has an angular momentum of \SI{18}{\kilo\gram\meter\squared\per\second}?
    \begin{multicols}{3}
    \begin{choices}
        \wrongchoice{\SI{3}{\radian\per\second}}
      \correctchoice{\SI{6}{\radian\per\second}}
        \wrongchoice{\SI{9}{\radian\per\second}}
        \wrongchoice{\SI{12}{\radian\per\second}}
        \wrongchoice{\SI{54}{\radian\per\second}}
    \end{choices}
    \end{multicols}
\end{question}
}

\element{ap}{
\begin{question}{angular-momentum-q14}
    What is the angular velocity of a cylinder of radius \SI{2}{\meter} and mass \SI{6}{\kilo\gram} that has an angular momentum of \SI{24}{\kilo\gram\meter\squared\per\second}?
    \begin{multicols}{3}
    \begin{choices}
        \wrongchoice{\SI{1}{\radian\per\second}}
      \correctchoice{\SI{2}{\radian\per\second}}
        \wrongchoice{\SI{3}{\radian\per\second}}
        \wrongchoice{\SI{4}{\radian\per\second}}
        \wrongchoice{\SI{6}{\radian\per\second}}
    \end{choices}
    \end{multicols}
\end{question}
}

\element{ap}{
\begin{question}{angular-momentum-q15}
    What is the angular momentum of a sphere with radius \SI{2}{\meter} and mass \SI{5}{\kilo\gram} that rotates at a speed of \SI{12}{\radian\per\second}?
    \begin{multicols}{2}
    \begin{choices}
        \wrongchoice{\SI{48}{\kilo\gram\meter\squared\per\second}}
      \correctchoice{\SI{96}{\kilo\gram\meter\squared\per\second}}
        \wrongchoice{\SI{120}{\kilo\gram\meter\squared\per\second}}
        \wrongchoice{\SI{144}{\kilo\gram\meter\squared\per\second}}
        \wrongchoice{\SI{240}{\kilo\gram\meter\squared\per\second}}
    \end{choices}
    \end{multicols}
\end{question}
}

\element{ap}{
\begin{question}{angular-momentum-q16}
    A sphere and a cylinder of equal mass and equal radius are rolling with equal angular momentum.
    What is the ratio of the angular speed of the sphere to the angular speed of the cylinder?
    \begin{multicols}{3}
    \begin{choices}
        \wrongchoice{$1:1$}
        \wrongchoice{$2:5$}
        \wrongchoice{$4:5$}
      \correctchoice{$5:4$}
        \wrongchoice{$5:2$}
    \end{choices}
    \end{multicols}
\end{question}
}

\element{ap}{
\begin{question}{angular-momentum-q17}
    If a cylinder and a hoop of the same mass and same radius both roll with the same angular velocity,
        what is the ratio of the angular momentum of the cylinder to the angular momentum of the hoop?
    \begin{multicols}{3}
    \begin{choices}
        \wrongchoice{$1:1$}
      \correctchoice{$1:2$}
        \wrongchoice{$2:5$}
        \wrongchoice{$2:1$}
        \wrongchoice{$5:4$}
    \end{choices}
    \end{multicols}
\end{question}
}

\element{ap}{
\begin{question}{angular-momentum-q18}
    If a wheel of moment of inertia \SI{24}{\kilo\gram\meter\squared} moves at a velocity of \SI{6}{\meter\per\second} and has an angular momentum of \SI{36}{\kilo\gram\meter\squared\per\second},
        what is its radius?
    \begin{multicols}{3}
    \begin{choices}
        \wrongchoice{\SI{2}{\meter}}
        \wrongchoice{\SI{3}{\meter}}
      \correctchoice{\SI{4}{\meter}}
        \wrongchoice{\SI{6}{\meter}}
        \wrongchoice{\SI{9}{\meter}}
    \end{choices}
    \end{multicols}
\end{question}
}

\element{ap}{
\begin{question}{angular-momentum-q19}
    What is the mass of a hoop that a radius of \SI{2.5}{\meter} and has an angular momentum of \SI{25}{\kilo\gram\meter\squared} when it rolls with at a speed of \SI{10}{\meter\per\second}?
    \begin{multicols}{3}
    \begin{choices}
        \wrongchoice{\SI{0.5}{\kilo\gram}}
      \correctchoice{\SI{1}{\kilo\gram}}
        \wrongchoice{\SI{2}{\kilo\gram}}
        \wrongchoice{\SI{4}{\kilo\gram}}
        \wrongchoice{\SI{5}{\kilo\gram}}
    \end{choices}
    \end{multicols}
\end{question}
}

\element{ap}{
\begin{question}{angular-momentum-q20}
    A cylinder of mass $M$ and radius $r$ rolls with a linear velocity $v$.
    What is its angular momentum?
    \begin{multicols}{3}
    \begin{choices}
        \wrongchoice{$Mv$}
        \wrongchoice{$\dfrac{Mv}{r}$}
      \correctchoice{$\dfrac{Mrv}{2}$}
        \wrongchoice{$Mrv$}
        \wrongchoice{$2Mrv$}
    \end{choices}
    \end{multicols}
\end{question}
}

\element{ap}{
\begin{question}{angular-momentum-q21}
    Which of the following is true of the angular momentum vector?
    \begin{choices}
      \correctchoice{It is perpendicular to both the radius vector and the linear momentum vector.}
        \wrongchoice{It is in the same direction as the radius vector.}
        \wrongchoice{It is in the same direction as the linear momentum vector.}
        \wrongchoice{It is in the same plane as both the radius vector and the linear momentum vector.}
        \wrongchoice{Angular momentum is not a vector quantity.}
    \end{choices}
\end{question}
}

\element{ap}{
\begin{question}{angular-momentum-q22}
    A figure skater spins with an angular velocity of $\omega_0$ on the ice holding his arms out perpendicular to his body.
    He then brings his arm in towards his body causing his rotational inertia to decrease by a factor of 3.
    What is his new angular velocity?
    \begin{multicols}{3}
    \begin{choices}
        \wrongchoice{$\dfrac{\omega_0}{9}$}
        \wrongchoice{$\dfrac{\omega_0}{3}$}
        \wrongchoice{$\omega_0$}
      \correctchoice{$3\omega_0$}
        \wrongchoice{$9\omega_0$}
    \end{choices}
    \end{multicols}
\end{question}
}

\element{ap}{
\begin{question}{angular-momentum-q23}
    A solid disk of mass $M$ and radius $R$ is spinning with an angular velocity
        of $\omega$ when a disk of the same radius but mass $m$ is placed on top.
    What is the new angular velocity in terms of $M$, $R$ and $\omega$?
    \begin{multicols}{2}
    \begin{choices}
        \wrongchoice{$\dfrac{\omega M}{2\left(m+M\right)}$}
      \correctchoice{$\dfrac{\omega M}{m+M}$}
        \wrongchoice{$\dfrac{\omega M}{2M}$}
        \wrongchoice{$\dfrac{\omega M}{2m}$}
        \wrongchoice{$\dfrac{2\omega M}{m+M}$}
    \end{choices}
    \end{multicols}
\end{question}
}

\element{ap}{
\begin{question}{angular-momentum-q24}
    An object spins with angular velocity $\omega$.
    If the object's moment of inertia increases by a factor of 4 with no added torque,
        what is the object's new angular velocity?
    \begin{multicols}{3}
    \begin{choices}
      \correctchoice{$\dfrac{\omega}{4}$}
        \wrongchoice{$\dfrac{\omega}{2}$}
        \wrongchoice{$\omega$}
        \wrongchoice{$2\omega$}
        \wrongchoice{$4\omega$}
    \end{choices}
    \end{multicols}
\end{question}
}

\element{ap}{
\begin{question}{angular-momentum-q25}
    Which of the following is equal to the time rate-of-change of angular momentum?
    \begin{choices}
        \wrongchoice{translational velocity}
        \wrongchoice{linear momentum}
      \correctchoice{torque}
        \wrongchoice{inertia}
        \wrongchoice{angular velocity}
    \end{choices}
\end{question}
}

\element{ap}{
\begin{question}{angular-momentum-q26}
    An object is spinning with an initial velocity of $v_0$.
    What is its final velocity if its radius is halved but its inertia stays the same?
    \begin{multicols}{3}
    \begin{choices}
        \wrongchoice{$\dfrac{v_0}{4}$}
        \wrongchoice{$\dfrac{v_0}{2}$}
        \wrongchoice{$v_0$}
      \correctchoice{$2v_0$}
        \wrongchoice{$4v_0$}
    \end{choices}
    \end{multicols}
\end{question}
}

\element{ap}{
\begin{question}{angular-momentum-q27}
    An spinning object has moment of inertia $I_0$.
    What is its inertia if its radius is halved but its translational velocity stays the same?
    \begin{multicols}{3}
    \begin{choices}
        \wrongchoice{$\dfrac{I_0}{4}$}
      \correctchoice{$\dfrac{I_0}{2}$}
        \wrongchoice{$I_0$}
        \wrongchoice{$2I_0$}
        \wrongchoice{$4I_0$}
    \end{choices}
    \end{multicols}
\end{question}
}

\element{ap}{
\begin{question}{angular-momentum-q28}
    An object spins with velocity $v_0$.
    Its radius is halved and its inertia is doubled.
    What is its new velocity?
    \begin{multicols}{3}
    \begin{choices}
        \wrongchoice{$\dfrac{v_0}{4}$}
        \wrongchoice{$\dfrac{v_0}{2}$}
        \wrongchoice{$v_0$}
        \wrongchoice{$2v_0$}
      \correctchoice{$4v_0$}
    \end{choices}
    \end{multicols}
\end{question}
}

\element{ap}{
\begin{question}{angular-momentum-q29}
    A figure skater in the middle of a spin moves her arms farther from her body.
    Which of the following will remain the same?
    \begin{choices}
        \wrongchoice{her rotational inertia}
        \wrongchoice{her angular velocity}
        \wrongchoice{her linear speed}
        \wrongchoice{her angular displacement}
      \correctchoice{her angular momentum}
    \end{choices}
\end{question}
}

\element{ap}{
\begin{question}{angular-momentum-q30}
    Assuming no torque acts on the object,
        what will happen to the angular velocity if its moment of inertia is halved?
    \begin{choices}
        \wrongchoice{it will be halved}
        \wrongchoice{it will remain the same}
      \correctchoice{it will be doubled}
        \wrongchoice{it will decrease by a factor of four}
        \wrongchoice{it will increase by a factor of four}
    \end{choices}
\end{question}
}

\element{ap}{
\begin{question}{angular-momentum-q31}
    %% Base your answers to questions 31 and 32 on the information below.
    A figure skater of moment of inertia \SI{40}{\kilo\gram\meter\squared} is initially spinning at a speed of \SI{6}{\radian\per\second}.
    If she moves her arms outward,
        her new moment of inertia is \SI{60}{\kilo\gram\meter\squared}.
    %% Start question
    What is her new angular velocity?
    \begin{multicols}{2}
    \begin{choices}
        \wrongchoice{\SI{2}{\radian\per\second}}
      \correctchoice{\SI{4}{\radian\per\second}}
        \wrongchoice{\SI{6}{\radian\per\second}}
        \wrongchoice{\SI{9}{\radian\per\second}}
        \wrongchoice{\SI{12}{\radian\per\second}}
    \end{choices}
    \end{multicols}
\end{question}
}

\element{ap}{
\begin{question}{angular-momentum-q32}
    %% Base your answers to questions 31 and 32 on the information below.
    A figure skater of moment of inertia \SI{40}{\kilo\gram\meter\squared} is initially spinning at a speed of \SI{6}{\radian\per\second}.
    If she moves her arms outward,
        her new moment of inertia is \SI{60}{\kilo\gram\meter\squared}.
    %% Start question
    How much work does the skater do in order to move her hands outward to this position?
    \begin{multicols}{3}
    \begin{choices}
        \wrongchoice{\SI{120}{\joule}}
        \wrongchoice{\SI{180}{\joule}}
      \correctchoice{\SI{240}{\joule}}
        \wrongchoice{\SI{360}{\joule}}
        \wrongchoice{\SI{480}{\joule}}
    \end{choices}
    \end{multicols}
\end{question}
}

\element{ap}{
\begin{question}{angular-momentum-q33}
    If an object is initially spinning at an angular velocity of $\omega$,
        what is its new angular velocity if its rotational inertia decreases by a factor of 4?
    \begin{multicols}{3}
    \begin{choices}
        \wrongchoice{$\dfrac{\omega}{16}$}
        \wrongchoice{$\dfrac{\omega}{4}$}
        \wrongchoice{$2\omega$}
      \correctchoice{$4\omega$}
        \wrongchoice{$16\omega$}
    \end{choices}
    \end{multicols}
\end{question}
}

\element{ap}{
\begin{question}{angular-momentum-q34}
    If an object is initially spinning at an angular velocity of $\omega$,
        what is its new angular velocity if its rotational inertia increases by a factor of 2?
    \begin{multicols}{3}
    \begin{choices}
        \wrongchoice{$\dfrac{\omega}{4}$}
      \correctchoice{$\dfrac{\omega}{2}$}
        \wrongchoice{$\omega$}
        \wrongchoice{$2\omega$}
        \wrongchoice{$4\omega$}
    \end{choices}
    \end{multicols}
\end{question}
}

\element{ap}{
\begin{question}{angular-momentum-q35}
    Base your answer to this question on the diagram below.
    \begin{center}
    \begin{tikzpicture}
        %% Block and point P
        \draw (0,0) rectangle (7cm,1em);
        \draw[fill] (3.5cm,0.5em) circle (1.5pt) node[anchor=north,yshift=-0.5em] {$P$};
        %% mass m, height h
        \node[draw,circle,fill=white!90!black] (M) at (-1em,4cm) {$m$};
        \draw[thick,<->] (M.south) -- (-1em,1em) node[fill=white,pos=0.5,anchor=center] {$h$};
    \end{tikzpicture}
    \end{center}
    If the mass $m$ is attached to the moving system without imparting a force on the system,
        what is the effect on the total angular momentum of the system?
    \begin{choices}
        \wrongchoice{It will increase.}
      \correctchoice{It will remain the same.}
        \wrongchoice{It will decrease.}
        \wrongchoice{The effect depends where the mass is attached to the system.}
        \wrongchoice{The effect depends on the ratio of m to the mass of the rod.}
    \end{choices}
\end{question}
}

\element{ap}{
\begin{question}{angular-momentum-q36}
    A \SI{3}{\newton} force is exerted on an infinitely large wheel of moment of inertia \SI{12}{\kilo\gram\meter\squared}.
    At what distance from the center of the wheel should the force be exerted in order for the wheel to accelerate with a angular acceleration of \SI{2}{\radian\per\second\squared},
        assuming the force is exerted perpendicular to the radius of the wheel in the plane of the wheel?
    \begin{multicols}{3}
    \begin{choices}
        \wrongchoice{\SI{6}{\meter}}
      \correctchoice{\SI{8}{\meter}}
        \wrongchoice{\SI{12}{\meter}}
        \wrongchoice{\SI{16}{\meter}}
        \wrongchoice{\SI{18}{\meter}}
    \end{choices}
    \end{multicols}
\end{question}
}


\endinput


