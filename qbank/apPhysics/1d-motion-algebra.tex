

%% AP Physics MC Questions Archive
%%----------------------------------------


%% One Dimensional Motion with Algebra
%%----------------------------------------
\element{ap}{
\begin{question}{1d-motion-algebra-q01}
    A car traveling at \SI{30}{\meter\per\second} decelerates uniformly to \SI{20}{\meter\per\second} in \SI{300}{\meter}.
    The amount of time it took to decelerate is closest to:
    \begin{multicols}{3}
    \begin{choices}
        \wrongchoice{\SI{5}{\second}}
        \wrongchoice{\SI{6}{\second}}
        \wrongchoice{\SI{10}{\second}}
      \correctchoice{\SI{12}{\second}}
        \wrongchoice{\SI{30}{\second}}
    \end{choices}
    \end{multicols}
\end{question}
}

\element{ap}{
\begin{question}{1d-motion-algebra-q02}
    A motorist driving at \SI{50}{\meter\per\second} applies the brakes so that the car decelerates at a rate of \SI{2}{\meter\per\second\squared}.
    The time for the car to stop is closest to:
    \begin{multicols}{3}
    \begin{choices}
        \wrongchoice{\SI{5}{\second}}
        \wrongchoice{\SI{10}{\second}}
        \wrongchoice{\SI{13}{\second}}
      \correctchoice{\SI{25}{\second}}
        \wrongchoice{\SI{50}{\second}}
    \end{choices}
    \end{multicols}
\end{question}
}

\element{ap}{
\begin{question}{1d-motion-algebra-q03}
    An airplane initially at rest accelerates at a constant rate of \SI{6}{\meter\per\second\squared}.
    When its speed is \SI{30}{\meter\per\second},
        the distance it has traveled is closest to:
    \begin{multicols}{3}
    \begin{choices}
        \wrongchoice{\SI{5}{\meter}}
        \wrongchoice{\SI{38}{\meter}}
      \correctchoice{\SI{75}{\meter}}
        \wrongchoice{\SI{150}{\meter}}
        \wrongchoice{\SI{180}{\meter}}
    \end{choices}
    \end{multicols}
\end{question}
}

\element{ap}{
\begin{question}{1d-motion-algebra-q04}
    A train starts at rest and accelerates at a uniform rate of \SI{4}{\meter\per\second\squared}.
    Which of the following is closest to its speed after it has traveled \SI{25}{\meter}?
    \begin{multicols}{3}
    \begin{choices}
        \wrongchoice{\SI{10}{\meter\per\second}}
      \correctchoice{\SI{14}{\meter\per\second}}
        \wrongchoice{\SI{70}{\meter\per\second}}
        \wrongchoice{\SI{100}{\meter\per\second}}
        \wrongchoice{\SI{200}{\meter\per\second}}
    \end{choices}
    \end{multicols}
\end{question}
}

\element{ap}{
\begin{question}{1d-motion-algebra-q05}
    An airplane traveling at \SI{52}{\meter\per\second} slows to \SI{32}{\meter\per\second} in a time span of \SI{4}{\second}.
    What is the magnitude of its average acceleration?
    \begin{multicols}{3}
    \begin{choices}
      \correctchoice{\SI{5}{\meter\per\second\squared}}
        \wrongchoice{\SI{8}{\meter\per\second\squared}}
        \wrongchoice{\SI{13}{\meter\per\second\squared}}
        \wrongchoice{\SI{20}{\meter\per\second\squared}}
        \wrongchoice{\SI{52}{\meter\per\second\squared}}
    \end{choices}
    \end{multicols}
\end{question}
}

\element{ap}{
\begin{question}{1d-motion-algebra-q06}
    A ball is thrown straight up with an initial velocity of \SI{14}{\meter\per\second}.
    The maximum height attained by the ball is closest to:
    \begin{multicols}{3}
    \begin{choices}
        \wrongchoice{\SI{0.7}{\meter}}
        \wrongchoice{\SI{1.4}{\meter}}
      \correctchoice{\SI{10}{\meter}}
        \wrongchoice{\SI{20}{\meter}}
        \wrongchoice{\SI{30}{\meter}}
    \end{choices}
    \end{multicols}
\end{question}
}

\element{ap}{
\begin{questionmult}{1d-motion-algebra-q07}
    Which of the following statements are true?
    \begin{choices}
      \correctchoice{An object with zero acceleration moves at a constant speed.}
        \wrongchoice{An object with constant acceleration must move in a straight line.}
      \correctchoice{An object with an acceleration always perpendicular to its velocity must move in a circle.}
        %% (A) I, only (B) II, only (C) I and III, only (D) I, II, and III (E) None of the above
    \end{choices}
\end{questionmult}
}

\element{ap}{
\begin{question}{1d-motion-algebra-q08}
    A gun fires a bullet directly upward.
    Which of the following best characterizes its motion at the highest point of its trajectory?
    \begin{choices}
        \wrongchoice{Its velocity is downward and its acceleration is upward.}
        \wrongchoice{Its velocity is upward and its acceleration is downward.}
        \wrongchoice{Both its velocity is and its acceleration are downward.}
      \correctchoice{Its velocity is zero and its acceleration is downward.}
        \wrongchoice{Both its velocity and its acceleration are zero.}
    \end{choices}
\end{question}
}

\element{ap}{
\begin{question}{1d-motion-algebra-q09}
    A car is traveling along a straight road with a velocity of \SI{10}{\meter\per\second}.
    It begins to accelerate uniformly at time $t=0$ and covers a distance of \SI{300}{\meter} in \SI{5}{\second}.
    What is the magnitude of the acceleration?
    \begin{multicols}{3}
    \begin{choices}
        \wrongchoice{\SI{10}{\meter\per\second\squared}}
        \wrongchoice{\SI{12}{\meter\per\second\squared}}
      \correctchoice{\SI{20}{\meter\per\second\squared}}
        \wrongchoice{\SI{24}{\meter\per\second\squared}}
        \wrongchoice{\SI{60}{\meter\per\second\squared}}
    \end{choices}
    \end{multicols}
\end{question}
}

\element{ap}{
\begin{question}{1d-motion-algebra-q10}
    A car is traveling with a velocity of \SI{5}{\meter\per\second} along a straight road.
    Another car starting at rest \SI{250}{\meter} behind it begins to accelerate uniformly.
    What is the magnitude of its acceleration if it catches the first car after \SI{10}{\second}?
    \begin{multicols}{3}
    \begin{choices}
        \wrongchoice{\SI{0.5}{\meter\per\second\squared}}
        \wrongchoice{\SI{1}{\meter\per\second\squared}}
        \wrongchoice{\SI{2}{\meter\per\second\squared}}
        \wrongchoice{\SI{2.5}{\meter\per\second\squared}}
      \correctchoice{\SI{6}{\meter\per\second\squared}}
    \end{choices}
    \end{multicols}
\end{question}
}

\element{ap}{
\begin{question}{1d-motion-algebra-q11}
    An object is thrown off a cliff with an initial upward velocity of \SI{15}{\meter\per\second}.
    It strikes the ground with a downward velocity of \SI{25}{\meter\per\second}.
    If air resistance is negligible,
        the height of the cliff is most nearly:
    \begin{multicols}{3}
    \begin{choices}
        \wrongchoice{\SI{10}{\meter}}
      \correctchoice{\SI{20}{\meter}}
        \wrongchoice{\SI{30}{\meter}}
        \wrongchoice{\SI{40}{\meter}}
        \wrongchoice{\SI{50}{\meter}}
    \end{choices}
    \end{multicols}
\end{question}
}

\element{ap}{
\begin{question}{1d-motion-algebra-q12}
    An object is released from rest on a planet that has no atmosphere.
    The object falls \SI{8}{\meter} in the first second.
    What is the acceleration due to gravity on the planet?
    \begin{multicols}{3}
    \begin{choices}
        \wrongchoice{\SI{2}{\meter\per\second\squared}}
        \wrongchoice{\SI{4}{\meter\per\second\squared}}
        \wrongchoice{\SI{8}{\meter\per\second\squared}}
        \wrongchoice{\SI{10}{\meter\per\second\squared}}
      \correctchoice{\SI{16}{\meter\per\second\squared}}
    \end{choices}
    \end{multicols}
\end{question}
}

\element{ap}{
\begin{question}{1d-motion-algebra-q13}
    A street car driver is traveling at a speed of \SI{25}{\meter\per\second} when he sees a road block.
    He applies the brakes causing a magnitude of deceleration of \SI{5}{\meter\per\second\squared}.
    If the roadblock is \SI{70}{\meter} away,
        which of the following will happen?
    \begin{choices}
        \wrongchoice{The car will hit the road block.}
        \wrongchoice{The car will stop immediately before the roadblock.}
        \wrongchoice{The car will stop \SI{2.5}{\meter} before the roadblock.}
        \wrongchoice{The car will stop \SI{5}{\meter} before the roadblock.}
      \correctchoice{The car will stop \SI{7.5}{\meter} before the roadblock.}
    \end{choices}
\end{question}
}

\element{ap}{
\begin{question}{1d-motion-algebra-q14}
    An object is dropped from Planet $E$.
    After \SI{10}{\second},
        the speed of the object is \SI{40}{\meter\per\second}.
    What is the acceleration due to gravity on this planet?
    \begin{multicols}{3}
    \begin{choices}
        \wrongchoice{\SI{2}{\meter\per\second\squared}}
      \correctchoice{\SI{4}{\meter\per\second\squared}}
        \wrongchoice{\SI{6}{\meter\per\second\squared}}
        \wrongchoice{\SI{8}{\meter\per\second\squared}}
        \wrongchoice{\SI{10}{\meter\per\second\squared}}
    \end{choices}
    \end{multicols}
\end{question}
}

\element{ap}{
\begin{question}{1d-motion-algebra-q15}
    A model aircraft requires a minimum speed of \SI{15}{\meter\per\second} in order to take off.
    If accelerating from rest,
        the minimum acceleration the airplane must have in order to lift off of a \SI{375}{\meter} long runway is closest to:
    \begin{multicols}{3}
    \begin{choices}
        \wrongchoice{\SI{0.1}{\meter\per\second\squared}}
        \wrongchoice{\SI{0.2}{\meter\per\second\squared}}
      \correctchoice{\SI{0.3}{\meter\per\second\squared}}
        \wrongchoice{\SI{1.33}{\meter\per\second\squared}}
        \wrongchoice{\SI{9.8}{\meter\per\second\squared}}
    \end{choices}
    \end{multicols}
\end{question}
}

\element{ap}{
\begin{question}{1d-motion-algebra-q16}
    A train is traveling at a speed of \SI{100}{\kilo\meter\per\hour}.
    It then brakes and comes to a halt over a distance of \SI{70}{\meter}.
    The magnitude of the deceleration of the train is nearest to:
    \begin{multicols}{3}
    \begin{choices}
        \wrongchoice{\SI{1.4}{\meter\per\second\squared}}
        \wrongchoice{\SI{4.4}{\meter\per\second\squared}}
      \correctchoice{\SI{5.5}{\meter\per\second\squared}}
        \wrongchoice{\SI{7.1}{\meter\per\second\squared}}
        \wrongchoice{\SI{0.71}{\meter\per\second\squared}}
    \end{choices}
    \end{multicols}
\end{question}
}

\element{ap}{
\begin{question}{1d-motion-algebra-q17}
    A car traveling at a speed of \SI{10}{\meter\per\second} slows down to a speed of \SI{2}{\meter\per\second} over a period of \SI{4}{\second}.
    The distance that the car has traveled is:
    \begin{multicols}{3}
    \begin{choices}
        \wrongchoice{\SI{8}{\meter}}
        \wrongchoice{\SI{16}{\meter}}
      \correctchoice{\SI{24}{\meter}}
        \wrongchoice{\SI{32}{\meter}}
        \wrongchoice{\SI{40}{\meter}}
    \end{choices}
    \end{multicols}
\end{question}
}

\element{ap}{
\begin{question}{1d-motion-algebra-q18}
    Which of the following is a scalar unit?
    \begin{choices}
        \wrongchoice{acceleration}
      \correctchoice{speed}
        \wrongchoice{displacement}
        \wrongchoice{average velocity}
        \wrongchoice{instantaneous velocity}
    \end{choices}
\end{question}
}

\element{ap}{
\begin{question}{1d-motion-algebra-q19}
    A ball is released at rest, from a height of \SI{5}{\meter}.
    About how long will it take for the ball to reach the ground?
    \begin{multicols}{3}
    \begin{choices}
        \wrongchoice{\SI{0.5}{\second}}
        \wrongchoice{\SI{0.66}{\second}}
      \correctchoice{\SI{1}{\second}}
        \wrongchoice{\SI{1.33}{\second}}
        \wrongchoice{\SI{1.66}{\second}}
    \end{choices}
    \end{multicols}
\end{question}
}

\element{ap}{
\begin{questionmult}{1d-motion-algebra-q20}
    An archer shoots an arrow straight up from the surface of the Earth.
    When the arrow reaches its maximum height,
        which of the following would be zero?
    \begin{choices}
      \correctchoice{speed}
      \correctchoice{velocity}
        \wrongchoice{acceleration}
        %% (A) III only (B) I and II only (C) I and III only (D) II and III only (E) I, II, and III
    \end{choices}
\end{questionmult}
}

\element{ap}{
\begin{question}{1d-motion-algebra-q21}
    Two runners are facing each other \SI{100}{\meter} apart.
    They both run toward each other.
    One runner runs at a rate of \SI{4}{\meter\per\second} and the other runs at a rate of \SI{6}{\meter\per\second}.
    How far has the \SI{6}{\meter\per\second} runner traveled when he passes the other runner?
    \begin{multicols}{3}
    \begin{choices}
        \wrongchoice{\SI{20}{\meter}}
        \wrongchoice{\SI{30}{\meter}}
        \wrongchoice{\SI{40}{\meter}}
        \wrongchoice{\SI{50}{\meter}}
      \correctchoice{\SI{60}{\meter}}
    \end{choices}
    \end{multicols}
\end{question}
}

\element{ap}{
\begin{questionmult}{1d-motion-algebra-q22}
    An object has translational equilibrium if:
    \begin{choices}
      \correctchoice{It is at rest.}
      \correctchoice{It has a constant velocity.}
        \wrongchoice{It has a constant acceleration.}
        \wrongchoice{It has a constant net force applied to it.}
        %% (A) I (B) III (C) I and II only (D) III and IV only (E) I, II, III and IV
    \end{choices}
\end{questionmult}
}

\element{ap}{
\begin{questionmult}{1d-motion-algebra-q23}
    An object has a non-zero acceleration.
    Of the following,
        which could be constant?
    \begin{choices}
      \correctchoice{speed}
      \correctchoice{kinetic energy}
        \wrongchoice{linear momentum}
        %% (A) I only (B) I and II only (C) I and III only (D) II only (E) II and III only
    \end{choices}
\end{questionmult}
}

\element{ap}{
\begin{questionmult}{1d-motion-algebra-q24}
    An object is moving with constant acceleration.
    Which of the following is necessarily true?
    \begin{choices}
      \correctchoice{The velocity is changing.}
      \correctchoice{The kinetic energy is changing.}
      \correctchoice{The linear momentum is changing.}
        %% (A) I only (B) II only (C) I and II only (D) I, II, and III (E) None of these
    \end{choices}
\end{questionmult}
}

\element{ap}{
\begin{questionmult}{1d-motion-algebra-q25}
    An object is at rest.
    Which of the following must be true?
    \begin{choices}
      \correctchoice{Its linear momentum is zero.}
        \wrongchoice{Its acceleration is zero.}
        \wrongchoice{Its potential energy is not zero.}
        %% (A) I only (B) I and II only (C) I and III only (D) I, II, and III (E) None of these
    \end{choices}
\end{questionmult}
}

\element{ap}{
\begin{question}{1d-motion-algebra-q26}
    An object is dropped of a cliff with a height of \SI{8}{\meter}.
    When the rock has been falling for \SI{4}{\meter},
        its velocity is most nearly
    \begin{multicols}{3}
    \begin{choices}
      \correctchoice{\SI{9}{\meter\per\second}}
        \wrongchoice{\SI{12}{\meter\per\second}}
        \wrongchoice{\SI{18}{\meter\per\second}}
        \wrongchoice{\SI{80}{\meter\per\second}}
        \wrongchoice{\SI{320}{\meter\per\second}}
    \end{choices}
    \end{multicols}
\end{question}
}

\element{ap}{
\begin{question}{1d-motion-algebra-q27}
    An object is dropped off a cliff of height $h$.
    When the object has fallen for \SI{4}{\second},
        it has fallen a distance $h/2$.
    How long will it take to fall the rest of the way?
    (Neglect air resistance.)
    \begin{multicols}{3}
    \begin{choices}
        \wrongchoice{\SI{2.0}{\second}}
        \wrongchoice{\SI{2.50}{\second}}
        \wrongchoice{\SI{3.0}{\second}}
      \correctchoice{\SI{1.66}{\second}}
        \wrongchoice{\SI{4.50}{\second}}
    \end{choices}
    \end{multicols}
\end{question}
}

\element{ap}{
\begin{question}{1d-motion-algebra-q28}
    Which of the following is a scalar quantity?
    \begin{choices}
        \wrongchoice{linear momentum}
        \wrongchoice{velocity}
        \wrongchoice{acceleration}
      \correctchoice{kinetic energy}
        \wrongchoice{force}
    \end{choices}
\end{question}
}

\element{ap}{
\begin{question}{1d-motion-algebra-q29}
    Which of the following is a vector quantity?
    \begin{choices}
        \wrongchoice{speed}
        \wrongchoice{potential energy}
        \wrongchoice{time}
      \correctchoice{displacement}
        \wrongchoice{none of the provided}
    \end{choices}
\end{question}
}

\element{ap}{
\begin{question}{1d-motion-algebra-q30}
    A hammer and a feather are brought to the surface of the moon,
        where air resistance is negligible,
        and are released from the same height at rest.
    When the hammer hits the ground,
        the two objects will have the same:
    \begin{choices}
        \wrongchoice{momentum}
        \wrongchoice{kinetic energy}
        \wrongchoice{change in potential energy}
        \wrongchoice{inertia}
      \correctchoice{speed}
    \end{choices}
\end{question}
}

\element{ap}{
\begin{question}{1d-motion-algebra-q31}
    A softball is thrown straight up,
        reaching a maximum height of \SI{20}{\meter}.
    Neglecting air resistance,
        the vertical speed of the ball when it hits the ground is:
    \begin{multicols}{3}
    \begin{choices}
        \wrongchoice{\SI{10}{\meter\per\second}}
        \wrongchoice{\SI{15}{\meter\per\second}}
      \correctchoice{\SI{20}{\meter\per\second}}
        \wrongchoice{\SI{30}{\meter\per\second}}
        \wrongchoice{\SI{40}{\meter\per\second}}
    \end{choices}
    \end{multicols}
\end{question}
}

\element{ap}{
\begin{question}{1d-motion-algebra-q32}
    Object $A$ and $B$ are near the surface of the moon.
    When object $A$ is released from rest,
        it falls a distance $d$ in \SI{2.0}{\second}.
    Object $B$ is twice the mass of object $A$.
    How far would object $B$ fall in \SI{1.0}{\second} if released from rest?
    \begin{multicols}{3}
    \begin{choices}
        \wrongchoice{$18 d$}
      \correctchoice{$14 d$}
        \wrongchoice{$12 d$}
        \wrongchoice{$d$}
        \wrongchoice{$2 d$}
    \end{choices}
    \end{multicols}
\end{question}
}


\endinput


