

%% AP Physics MC Questions Archive
%%----------------------------------------


%% B-field due to wire
%%----------------------------------------
\element{ap}{
\begin{question}{b-field-wire-q01}
    The magnetic field inside a solenoid of length $l$ is $B$.
    A second solenoid has twice as many turns as the first one and is the same length.
    Both solenoids have the same current passing through them.
    What is the magnetic field inside the second solenoid?
    \begin{multicols}{3}
    \begin{choices}
        \wrongchoice{$\dfrac{B}{4}$}
        \wrongchoice{$\dfrac{B}{2}$}
        \wrongchoice{$B$}
      \correctchoice{$2B$}
        \wrongchoice{$4B$}
    \end{choices}
    \end{multicols}
\end{question}
}


1. Two long straight intersecting wires carry currents I in the
directions shown.
Which direction is the magnetic field pointed at the
point P?
A) into the page
B) out of the page
C) towards the top of the page
D) towards the bottom of the page
E) the magnetic field at point P is zero.


2.
%% NOTE:
The direction of the magnetic field at point P due to the
current I in the wire show above is
A) to the left
B) to the right
C) into the page
D) out of the page
E) toward the top of the page



3.
%% NOTE:
The direction of the magnetic field at point P due to the
current I in the wire shown above is
A) to the left
B) to the right
C) toward the top of the page
D) toward the bottom of the page
E) into the page


4. %% NOTE:
Two long, straight, parallel wires are separated by a
distance d, as shown above. They each carry a steady
current I into the page. At what points in the plane of the
page and outside the wires, besides the points at infinity is
the magnetic field due to the currents zero.
A) Only at point P
B) At all points on the line AA'
C) At all points on the line connecting the two wires
D) At all points on a circle of radius 2d centered at point P
E) At no points




5. The diagram below shows two wires running parallel to the
z-axis. One carries a current I towards the top of the page
and intersects the x-axis at a distance of 3.0 meters from the
origin. The other carries a current I towards the bottom of
the page and intersects the y-axis at a distance of 4.0 meters
from the origin.
%% NOTE:
What is the magnitude of the magnetic field at the origin?
A) 0
B) 1/ 12 ƒ  
C) 25/144 ƒ  
D) 1/5 ƒ  
E) 7/12 ƒ  


6. The magnetic field due to a long straight wire at a distance d
from it has a magnitude B. If the current in the wire is
doubled, the magnetic field at a distance d would be.
A) 1/4 B
B) 1/2 B
C) B
D) 2B
E) 4B



7. Which of the following are true about magnetic forces and
fields
I. Magnetic field lines are always perpendicular to electric field lines.
II. Magnetic field lines are always perpendicular to magnetic force lines.
III. Magnetic field lines are always parallel to magnetic force lines.
A) I only
B) II only
C) III only
D) I and II only
E) I and III only

8. Which of the following are true about electromagnetic forces and fields?

I. The magnetic field lines due to a current-carrying wire radiate away from the wire.
II. Electric field lines due to a current-carrying wire circle the wire and their direction is determined by the right hand rule.
III. Magnetic force vectors and electric force vectors for a charged particle always point in opposite directions.
A) III only
B) I and II only
C) I and III only
D) I, II, and III
E) none of the above are true




9. Which of the following statements is true about magnetic
forces and fields?
12.
A) The magnetic field lines are always parallel to the
magnetic force lines.
B) The magnetic field lines are always parallel to the
velocity vector.
As shown above, point P is midway between two long,
straight, parallel wires a distance d apart. Wire A has
current I running to the right, Wire B has current I running
to the left. Find the magnetic field at point P.
C) The magnetic force can never change the velocity
vector of a particle.
D) The magnetic field from a current-carrying wire is
related to the inverse square of the distance from the
wire.
A)
E) A charged particle can move through a magnetic field
without feeling a magnetic force.
μ 0 I out of the page
2d
B) 2μ 0 I out of the page
pd
C) 0
10. A long straight wire carries a current of 3 A. Find the
magnitude of the magnetic field 6 cm from the wire.
D)
–6
A) 1 × 10 T
μ 0 I into the page
2pd
E) 2μ 0 I into the page
pd
B) 2 × 10 –6 T
C) 1 × 10 –5 T
D) 2 × 10 –5 T
13.
E) 1 × 10 –4 T

11. A long straight wire carries a current of 1.5 A. Find the
force on a proton traveling at a distance of 10 cm from the
wire in the direction of current flow, if its velocity is
2 ] 10 5 m/s.
A) 9.6 × 10 –26 T
B) 4.8 × 10 –20 T


13:
%% NOTE:
Two parallel, straight, long wires a distance d apart each
carry a current I in the same direction. A particle with
charge +q is traveling midway between and parallel to the
wires at velocity v. The force on the particle is equal to
C) 9.6 × 10 –20 T A) 2μ 0 Iqv × 1 p 1 upward, in the plane of the page
d
D) 4.8 × 10 –13 T B) 2μ 0 Iqv × 1 p 1 downward, in the plane of the page
d
E) 9.6 × 10 –13 T C) 2μ 0 Iqv × 1 p 1 out of the page
d
D) 2μ 0 Iqv × 1 p 1 into the page
d
E) zero







\endinput


