

%% AP Physics MC Questions Archive
%%----------------------------------------


%% Oscillating Spring
%%----------------------------------------
\element{ap}{
\begin{question}{oscillating-spring-q01}
    If a mass attached to a spring is doubled,
        but the spring constant remains constant,
        then the period of oscillations is multiplied by:
    \begin{multicols}{3}
    \begin{choices}
        \wrongchoice{$\dfrac{1}{2}$}
        \wrongchoice{$\dfrac{1}{\sqrt{2}}$}
        \wrongchoice{$1$}
      \correctchoice{$\sqrt{2}$}
        \wrongchoice{$2$}
    \end{choices}
    \end{multicols}
\end{question}
}

\element{ap}{
\begin{question}{oscillating-spring-q02}
    If the spring constant of a spring is doubled,
        but the mass attached to it is remains constant,
        then the period of oscillations is multiplied by:
    \begin{multicols}{3}
    \begin{choices}
        \wrongchoice{$\dfrac{1}{2}$}
      \correctchoice{$\dfrac{1}{\sqrt{2}}$}
        \wrongchoice{$1$}
        \wrongchoice{$\sqrt{2}$}
        \wrongchoice{$2$}
    \end{choices}
    \end{multicols}
\end{question}
}

\element{ap}{
\begin{question}{oscillating-spring-q03}
    A mass oscillating on a spring has a period of \SI{3.0}{\second} on earth.
    On the moon,
        where the acceleration due to gravity is approximately one-sixth of its value on earth,
        its period would be most nearly:
    \begin{multicols}{3}
    \begin{choices}
        \wrongchoice{\SI{0.50}{\second}}
        \wrongchoice{\SI{1.2}{\second}}
      \correctchoice{\SI{3.0}{\second}}
        \wrongchoice{\SI{7.3}{\second}}
        \wrongchoice{\SI{18}{\second}}
    \end{choices}
    \end{multicols}
\end{question}
}

\element{ap}{
\begin{question}{oscillating-spring-q04}
    A \SI{10}{\kilo\gram} mass suspended on Earth by a spring oscillates with a period of \SI{12}{\second}.
    This mass-spring system is now moved to the moon where the gravity is one-sixth that of the Earth.
    What is the new period of motion?
    \begin{multicols}{3}
    \begin{choices}
        \wrongchoice{\SI{2}{\second}}
        \wrongchoice{\SI{4}{\second}}
        \wrongchoice{\SI{8}{\second}}
      \correctchoice{\SI{12}{\second}}
        \wrongchoice{\SI{72}{\second}}
    \end{choices}
    \end{multicols}
\end{question}
}

\element{ap}{
\begin{question}{oscillating-spring-q05}
    When a \SI{2}{\kilo\gram} object is suspended on a particular spring,
        the spring stretches \SI{40}{\meter}.
    What is the period of oscillation for this system?
    \begin{multicols}{3}
    \begin{choices}
        \wrongchoice{\SI{2\pi}{\second}}
      \correctchoice{\SI{4\pi}{\second}}
        \wrongchoice{\SI{\pi}{\second}}
        \wrongchoice{\SI{\pi/2}{\second}}
        \wrongchoice{\SI{\pi/4}{\second}}
    \end{choices}
    \end{multicols}
\end{question}
}

\element{ap}{
\begin{question}{oscillating-spring-q06}
    A block is attached to a vertical ideal spring,
        and is undergoing simple harmonic motion.
    If this same block and spring are brought to the moon,
        where all conditions are the same except that the acceleration due to gravity is \num{1/6} that of the Earth,
        and are undergoing simple harmonic motion of the same amplitude as on Earth,
        which of the following will change?
    \begin{choices}
        \wrongchoice{period}
        \wrongchoice{frequency}
        \wrongchoice{maximum kinetic energy}
      \correctchoice{equilibrium point}
        \wrongchoice{minimum kinetic energy}
    \end{choices}
\end{question}
}

\element{ap}{
\begin{question}{oscillating-spring-q07}
    A block is attached to a vertical spring and undergoes simple harmonic motion with frequency $f$.
    If the block is removed and replaced by a block with 4 times the mass of the first block,
        what will the new frequency be?
    \begin{multicols}{3}
    \begin{choices}
        \wrongchoice{$\dfrac{f}{4}$}
      \correctchoice{$\dfrac{f}{2}$}
        \wrongchoice{$f$}
        \wrongchoice{$2f$}
        \wrongchoice{$4f$}
    \end{choices}
    \end{multicols}
\end{question}
}

\element{ap}{
\begin{question}{oscillating-spring-q08}
    A pendulum and a mass-spring system each have a period $T$.
    If the mass in each system is halved,
        what is the new period of each system?
    \begin{choices}
        \wrongchoice{pendulum: $\dfrac{T}{\sqrt{2}}$; mass on spring: $\sqrt{2}T$}
        \wrongchoice{pendulum: $T$; mass on spring: $\sqrt{2}T$}
        \wrongchoice{pendulum: $T$; mass on spring: $T$}
      \correctchoice{pendulum: $T$; mass on spring: $\dfrac{T}{\sqrt{2}}$}
        \wrongchoice{pendulum: $\sqrt{2}T$; mass on spring: $\dfrac{T}{\sqrt{2}}$}
    \end{choices}
\end{question}
}

\element{ap}{
\begin{question}{oscillating-spring-q09}
    An object moves up and down the $y$-axis with position $s$ given as a function of time $t$ by the expression $s=A\sin\omega t$.
    What is the period of this motion?
    \begin{multicols}{3}
    \begin{choices}
        \wrongchoice{$\omega$}
        \wrongchoice{$2\pi\omega$}
        \wrongchoice{$A\omega$}
      \correctchoice{$\dfrac{2\pi}{\omega}$}
        \wrongchoice{$\dfrac{\omega}{2\pi}$}
    \end{choices}
    \end{multicols}
\end{question}
}

\element{ap}{
\begin{question}{oscillating-spring-q10}
    A mass attached to an ideal spring moves up and down the $y$-axis with position s given as a function of time $t$ by the expression $s=A\sin\omega t$.
    What is the maximum displacement from equilibrium of this motion?
    \begin{multicols}{3}
    \begin{choices}
      \correctchoice{$A$}
        \wrongchoice{$2\pi A$}
        \wrongchoice{$A\omega$}
        \wrongchoice{$\dfrac{2\pi}{A}$}
        \wrongchoice{$\dfrac{A}{2\pi}$}
    \end{choices}
    \end{multicols}
\end{question}
}

\element{ap}{
\begin{question}{oscillating-spring-q11}
    A mass-spring system oscillates with a period $T$ on Earth.
    If it is brought to a planet where acceleration due to gravity is $4g$,
        what is the new period of the mass-spring system?
    \begin{multicols}{3}
    \begin{choices}
        \wrongchoice{$\dfrac{T}{4}$}
        \wrongchoice{$\dfrac{T}{2}$}
      \correctchoice{$T$}
        \wrongchoice{$2T$}
        \wrongchoice{$4T$}
    \end{choices}
    \end{multicols}
\end{question}
}

\element{ap}{
\begin{question}{oscillating-spring-q12}
    A certain spring-mass system initially has a period of $T$.
    If the mass remains the same,
        what must happen to the spring in order for the period to become $2T$?
    \begin{choices}
        \wrongchoice{The spring must be double in length.}
        \wrongchoice{The acceleration due to gravity must be doubled.}
        \wrongchoice{The acceleration due to gravity must be halved.}
        \wrongchoice{The spring constant must decrease by a factor of 2.}
      \correctchoice{The spring constant must decrease by a factor of 4.}
    \end{choices}
\end{question}
}

\element{ap}{
\begin{question}{oscillating-spring-q13}
    Which of the following is true of both of a simple pendulum and a Hookean spring-mass system?
    \begin{choices}
      \correctchoice{The period is unaffected by the amplitude of motion.}
        \wrongchoice{The motion is unaffected by a change in mass.}
        \wrongchoice{The period is unaffected by a change in gravity.}
        \wrongchoice{The motion is unaffected by a change in the length of the system.}
        \wrongchoice{The period is unaffected by a change in the constant inherent to the system.}
    \end{choices}
\end{question}
}

\element{ap}{
\begin{question}{oscillating-spring-q14}
    A \SI{0.01}{\kilo\gram} bullet traveling at \SI{50}{\meter\per\second} hits a \SI{0.04}{\kilo\gram} wood block attached to a spring with spring constant \SI{5}{\newton\per\meter},
        and embeds itself in the block.
    What is the maximum displacement of the block after the bullet is fired at it?
    \begin{multicols}{3}
    \begin{choices}
        \wrongchoice{\SI{0.2}{\meter}}
        \wrongchoice{\SI{0.25}{\meter}}
        \wrongchoice{\SI{0.4}{\meter}}
        \wrongchoice{\SI{0.5}{\meter}}
      \correctchoice{\SI{1.0}{\meter}}
    \end{choices}
    \end{multicols}
\end{question}
}

\element{ap}{
\begin{question}{oscillating-spring-q15}
    A mass $m$ oscillating on a spring has a period of $T$ and a maximum displacement of $x$.
    What is the total energy in the system?
    \begin{multicols}{2}
    \begin{choices}
        \wrongchoice{$\dfrac{2\pi x^2 m}{T}$}
        \wrongchoice{$\dfrac{2\pi x m}{T^2}$}
      \correctchoice{$\dfrac{2\pi^2 x^2 m}{T^2}$}
        \wrongchoice{$\dfrac{2\pi^2 x^2 m^2}{T^2}$}
        \wrongchoice{$\dfrac{2\pi x m^2}{T^2}$}
    \end{choices}
    \end{multicols}
\end{question}
}

\newcommand{\apOscillatingSpringQSixteen}{
\begin{tikzpicture}
\end{tikzpicture}
}

\element{ap}{
\begin{question}{oscillating-spring-q16}
    %% Base your answers to questions 16 through 18 on the following.
    Two blocks are attached by a stretched spring and held on a frictionless surface as shown below.
    The blocks are then released simultaneously.
    Block I has one-fourth the mass of block II.
    \begin{center}
        \apOscillatingSpringQSixteen
    \end{center}
    What is the ratio of the magnitude of the initial restoring force on mass I to the initial magnitude of the restoring force on mass II?
    \begin{multicols}{3}
    \begin{choices}
        \wrongchoice{\num{1/16}}
        \wrongchoice{\num{1/4}}
      \correctchoice{\num{1}}
        \wrongchoice{\num{4}}
        \wrongchoice{\num{16}}
    \end{choices}
    \end{multicols}
\end{question}
}

\element{ap}{
\begin{question}{oscillating-spring-q17}
    %% Base your answers to questions 16 through 18 on the following.
    Two blocks are attached by a stretched spring and held on a frictionless surface as shown below.
    The blocks are then released simultaneously.
    Block I has one-fourth the mass of block II.
    \begin{center}
        \apOscillatingSpringQSixteen
    \end{center}
    What is the ratio of the maximum acceleration of block I to the maximum acceleration of block II?
    %% NOTE: what is the magnitude of the maximum acceleration fo block I divided by the maximum
    \begin{multicols}{3}
    \begin{choices}
        \wrongchoice{$11/16$}
        \wrongchoice{$1/4$}
        \wrongchoice{$1:1$}
      \correctchoice{$4:1$}
        \wrongchoice{$16:1$}
    \end{choices}
    \end{multicols}
\end{question}
}

\element{ap}{
\begin{question}{oscillating-spring-q18}
    %% Base your answers to questions 16 through 18 on the following.
    Two blocks are attached by a stretched spring and held on a frictionless surface as shown below.
    The blocks are then released simultaneously.
    Block I has one-fourth the mass of block II.
    \begin{center}
        \apOscillatingSpringQSixteen
    \end{center}
    What is the ratio of the maximum kinetic energy of block I to the maximum kinetic energy of block II?
    %% NOTE: what is the magnitude of the maximum acceleration fo block I divided by the maximum
    \begin{multicols}{3}
    \begin{choices}
        \wrongchoice{\num{1/4}}
        \wrongchoice{\num{1/2}}
        \wrongchoice{\num{1}}
        \wrongchoice{\num{2}}
      \correctchoice{\num{4}}
    \end{choices}
    \end{multicols}
\end{question}
}

\element{ap}{
\begin{question}{oscillating-spring-q19}
    When an object oscillating in simple harmonic motion passes through its equilibrium position,
        which of the following is true regarding the values of its speed and the magnitude of the restoring force?
    \begin{choices}
      \correctchoice{The speed is at a maximum and the restoring force is zero.}
        \wrongchoice{Both the speed and the restoring force are zero.}
        \wrongchoice{Both the speed and the magnitude of the restoring force are at a maximum.}
        \wrongchoice{The speed is zero and the magnitude of the restoring force is at a maximum.}
        \wrongchoice{Both the speed and the magnitude of the restoring force are at one-half their maximum values.}
    \end{choices}
\end{question}
}

\element{ap}{
\begin{question}{oscillating-spring-q20}
    Two bodies of masses four and nine kilograms are initially at rest on a horizontal frictionless surface.
    A light spring is compressed between the bodies which are held together by a thin thread.
    After the thread is cut, the nine kilogram body has a speed of one meter per second.
    The speed of the four kilogram body is:
    \begin{multicols}{3}
    \begin{choices}
        \wrongchoice{\SI{4/9}{\meter\per\second}}
        \wrongchoice{\SI{2/3}{\meter\per\second}}
        \wrongchoice{\SI{1}{\meter\per\second}}
        \wrongchoice{\SI{3/2}{\meter\per\second}}
      \correctchoice{\SI{9/4}{\meter\per\second}}
    \end{choices}
    \end{multicols}
\end{question}
}

\newcommand{\apOscillatingSpringQTwentyOneA}{
\begin{tikzpicture}
\end{tikzpicture}
}

\element{ap}{
\begin{question}{oscillating-spring-q21}
    %% Base your answers to questions 21 through 24 on the situation below.
    A block oscillates without friction on the end of a spring.
    The minimum and maximum lengths of the spring as it oscillates are,
        respectively, $x_{min}$ and $x_{max}$.
    The graphs below can represent quantities associated with the oscillation as functions of the length $x$ of the spring.
    Graphs $C$ and $D$ are parabolic.
    %% NOTE: pgfplots
    Which graph can represent the magnitude of the restoring force on the block as a function of $x$?
    \begin{multicols}{3}
    \begin{choices}
      \correctchoice{A}
        \wrongchoice{B}
        \wrongchoice{C}
        \wrongchoice{D}
        \wrongchoice{None of the above}
    \end{choices}
    \end{multicols}
\end{question}
}

\element{ap}{
\begin{question}{oscillating-spring-q22}
    %% Base your answers to questions 21 through 24 on the situation below.
    A block oscillates without friction on the end of a spring.
    The minimum and maximum lengths of the spring as it oscillates are,
        respectively, $x_{min}$ and $x_{max}$.
    The graphs below can represent quantities associated with the oscillation as functions of the length $x$ of the spring.
    Graphs $C$ and $D$ are parabolic.
    %% NOTE: pgfplots
    Which graph can represent the magnitude of the acceleration of the block as a function of $x$?
    \begin{multicols}{3}
    \begin{choices}
      \correctchoice{A}
        \wrongchoice{B}
        \wrongchoice{C}
        \wrongchoice{D}
        \wrongchoice{None of the above}
    \end{choices}
    \end{multicols}
\end{question}
}

\element{ap}{
\begin{question}{oscillating-spring-q23}
    %% Base your answers to questions 21 through 24 on the situation below.
    A block oscillates without friction on the end of a spring.
    The minimum and maximum lengths of the spring as it oscillates are,
        respectively, $x_{min}$ and $x_{max}$.
    The graphs below can represent quantities associated with the oscillation as functions of the length $x$ of the spring.
    Graphs $C$ and $D$ are parabolic.
    %% NOTE: pgfplots
    Which graph can represent the potential energy stored in the spring as a function of $x$?
    \begin{multicols}{3}
    \begin{choices}
        \wrongchoice{A}
        \wrongchoice{B}
      \correctchoice{C}
        \wrongchoice{D}
        \wrongchoice{None of the above}
    \end{choices}
    \end{multicols}
\end{question}
}

\element{ap}{
\begin{question}{oscillating-spring-q24}
    %% Base your answers to questions 21 through 24 on the situation below.
    A block oscillates without friction on the end of a spring.
    The minimum and maximum lengths of the spring as it oscillates are,
        respectively, $x_{min}$ and $x_{max}$.
    The graphs below can represent quantities associated with the oscillation as functions of the length $x$ of the spring.
    Graphs $C$ and $D$ are parabolic.
    %% NOTE: pgfplots
    Which graph can represent the velocity of the block as a function of $x$?
    \begin{multicols}{3}
    \begin{choices}
        \wrongchoice{A}
        \wrongchoice{B}
        \wrongchoice{C}
        \wrongchoice{D}
      \correctchoice{None of the above}
    \end{choices}
    \end{multicols}
\end{question}
}

\element{ap}{
\begin{questionmult}{oscillating-spring-q25}
    Which of the following are necessarily properties of simple harmonic motion?
    \begin{choices}
        \wrongchoice{Acceleration is constant.}
      \correctchoice{The restoring force is dependent on position.}
        \wrongchoice{The equilibrium position is dependent upon amplitude.}
        %\wrongchoice{I only}
        %\correctchoice{II only}
        %\wrongchoice{III only}
        %\wrongchoice{I and III only}
        %\wrongchoice{II and III only}
    \end{choices}
\end{questionmult}
}

\element{ap}{
\begin{question}{oscillating-spring-q26}
    A block attached to an ideal spring undergoes simple harmonic motion.
    The restoring force has its highest value where:
    \begin{choices}
        \wrongchoice{the speed is highest.}
        \wrongchoice{the kinetic energy is the highest.}
        \wrongchoice{the acceleration is the lowest.}
      \correctchoice{the speed is the lowest.}
        \wrongchoice{the potential energy is lowest.}
    \end{choices}
\end{question}
}

\element{ap}{
\begin{question}{oscillating-spring-q27}
    An object is attached to an ideal spring and is undergoing simple harmonic motion.
    Which of the following statements correctly describes this setup?
    \begin{choices}
        \wrongchoice{The force vector is constant.}
        \wrongchoice{The velocity vector and acceleration vector are always directly opposed.}
      \correctchoice{The displacement vector and the acceleration vector are always directly opposed.}
        \wrongchoice{The velocity vector is always in the same direction as the displacement vector.}
        \wrongchoice{The displacement vector is always in the same direction as the acceleration vector.}
    \end{choices}
\end{question}
}

\element{ap}{
\begin{question}{oscillating-spring-q28}
    A block attached to an ideal spring undergoes simple harmonic motion about its equilibrium position with amplitude \SI{8}{\centi\meter}.
    When the block is at position \SI{-4}{\centi\meter},
        what is the ratio of kinetic energy to total energy?
    \begin{multicols}{3}
    \begin{choices}
        \wrongchoice{$1:4$}
        \wrongchoice{$1:3$}
        \wrongchoice{$1:2$}
        \wrongchoice{$2:3$}
      \correctchoice{$3:4$}
    \end{choices}
    \end{multicols}
\end{question}
}

\element{ap}{
\begin{question}{oscillating-spring-q29}
    A block of mass $m=\SI{8}{\kilo\gram}$ oscillates on a spring with a spring constant of \SI{0.5}{\newton\per\meter}.
    The maximum speed the block attains in this motion is \SI{2}{\meter\per\second}.
    The amplitude of the oscillation is:
    \begin{multicols}{3}
    \begin{choices}
        \wrongchoice{\SI{4}{\meter}}
      \correctchoice{\SI{8}{\meter}}
        \wrongchoice{\SI{16}{\meter}}
        \wrongchoice{\SI{32}{\meter}}
        \wrongchoice{\SI{64}{\meter}}
    \end{choices}
    \end{multicols}
\end{question}
}

\element{ap}{
\begin{question}{oscillating-spring-q30}
    A block of mass $m$ is attached to an ideal spring with spring constant $k$.
    The block is at rest at the equilibrium position when a force is quickly applied to the block that gives it an initial speed $v$.
    What is the amplitude of the resulting oscillations?
    \begin{multicols}{3}
    \begin{choices}
      \correctchoice{$\sqrt{\dfrac{mv^2}{k}}$}
        \wrongchoice{$\dfrac{mv^2}{k}$}
        \wrongchoice{$\dfrac{mv^2}{2k}$}
        \wrongchoice{$\dfrac{1}{2}\sqrt{\dfrac{mv^2}{k}}$}
        \wrongchoice{$\sqrt{\dfrac{mv}{k}}$}
    \end{choices}
    \end{multicols}
\end{question}
}

\newcommand{\apOscillatingSpringQThirtyOne}{
\begin{tikzpicture}
    %% NOTE: tikz
\end{tikzpicture}
}

\element{ap}{
\begin{question}{oscillating-spring-q31}
    %% Base your answers to questions 31 through 33 on the following situation.
    A ball of mass \SI{2.0}{\kilo\gram} oscillates on a spring with a force constant of \SI{50}{\newton\per\meter} between points I and IV on the track shown below.
    Point II is the midpoint of its path, and Point III is located half way between points II and IV.
    Friction is negligible.
    \begin{center}
        \apOscillatingSpringQThirtyOne
    \end{center}
    If the potential energy is zero at point II,
        where will the kinetic and potential energies of the ball be equal?
    \begin{choices}
        \wrongchoice{At point II}
        \wrongchoice{At some point between II and III}
        \wrongchoice{At point III}
      \correctchoice{At some point between III and IV}
        \wrongchoice{At point IV}
    \end{choices}
\end{question}
}

\element{ap}{
\begin{question}{oscillating-spring-q32}
    %% Base your answers to questions 31 through 33 on the following situation.
    A ball of mass \SI{2.0}{\kilo\gram} oscillates on a spring with a force constant of \SI{50}{\newton\per\meter} between points I and IV on the track shown below.
    Point II is the midpoint of its path, and Point III is located half way between points II and IV.
    Friction is negligible.
    \begin{center}
        \apOscillatingSpringQThirtyOne
    \end{center}
    The maximum velocity of the ball is most nearly:
    \begin{multicols}{3}
    \begin{choices}
        \wrongchoice{\SI{1.0}{\meter\per\second}}
        \wrongchoice{\SI{2.5}{\meter\per\second}}
      \correctchoice{\SI{5.0}{\meter\per\second}}
        \wrongchoice{\SI{7.0}{\meter\per\second}}
        \wrongchoice{\SI{10}{\meter\per\second}}
    \end{choices}
    \end{multicols}
\end{question}
}

\element{ap}{
\begin{question}{oscillating-spring-q33}
    %% Base your answers to questions 31 through 33 on the following situation.
    A ball of mass \SI{2.0}{\kilo\gram} oscillates on a spring with a force constant of \SI{50}{\newton\per\meter} between points I and IV on the track shown below.
    Point II is the midpoint of its path, and Point III is located half way between points II and IV.
    Friction is negligible.
    \begin{center}
        \apOscillatingSpringQThirtyOne
    \end{center}
    The velocity of the ball at point III is most nearly
    \begin{multicols}{3}
    \begin{choices}
        \wrongchoice{\SI{1.0}{\meter\per\second}}
        \wrongchoice{\SI{2.3}{\meter\per\second}}
      \correctchoice{\SI{4.3}{\meter\per\second}}
        \wrongchoice{\SI{5.0}{\meter\per\second}}
        \wrongchoice{\SI{6.7}{\meter\per\second}}
    \end{choices}
    \end{multicols}
\end{question}
}

\element{ap}{
\begin{question}{oscillating-spring-q34}
    The below above shows the potential energy $U$ of an object on the end of a spring undergoing simple harmonic motion with amplitude \SI{10}{\centi\meter} as a function of $x$,
        the displacement of the object from equilibrium.
    \begin{center}
    \begin{tikzpicture}
        %% NOTE:
    \end{tikzpicture}
    \end{center}
    At what point(s) is the kinetic energy of the object maximized?
    \begin{choices}
        \wrongchoice{$x=\SI{-10}{\centi\meter}$}
        \wrongchoice{$x=\SI{10}{\centi\meter}$}
      \correctchoice{$x=\text{zero}$}
        \wrongchoice{$x=\SI{10}{\centi\meter}$ and $x=\SI{-10}{\centi\meter}$}
        \wrongchoice{$x=\text{zero}$, $x=\SI{10}{\centi\meter}$, and $x=\SI{-10}{\centi\meter}$}
    \end{choices}
\end{question}
}

\newcommand{\apOscillatingSpringQThirtyFive}{
\begin{tikzpicture}
    %% NOTE: TODO: draw tikz
\end{tikzpicture}
}

\element{ap}{
\begin{question}{oscillating-spring-q35}
    %% Base your answers to questions 35 and 36 on
    The graph below shows the kinetic energy in joules as a function of displacement $x$ for an object on the end of a spring oscillating in simple harmonic motion with amplitude $A$.
    \begin{center}
        \apOscillatingSpringQThirtyFive
    \end{center}
    The maximum potential energy of the object during its oscillation is:
    \begin{multicols}{3}
    \begin{choices}
        \wrongchoice{\SI{2.5}{\joule}}
        \wrongchoice{\SI{5}{\joule}}
      \correctchoice{\SI{10}{\joule}}
        \wrongchoice{\SI{25}{\joule}}
        \wrongchoice{\SI{50}{\joule}}
    \end{choices}
    \end{multicols}
\end{question}
}

\element{ap}{
\begin{questionmult}{oscillating-spring-q36}
    %% Base your answers to questions 35 and 36 on
    The graph below shows the kinetic energy in joules as a function of displacement $x$ for an object on the end of a spring oscillating in simple harmonic motion with amplitude $A$.
    \begin{center}
        \apOscillatingSpringQThirtyFive
    \end{center}
    The potential energy is maximized at which of the following displacements?
    \begin{multicols}{3}
    \begin{choices}
        \wrongchoice{$x=\text{zero}$}
      \correctchoice{$x=A$}
      \correctchoice{$x=-A$}
        %\wrongchoice{I only}
        %\wrongchoice{II only}
        %\wrongchoice{III only}
        %\correctchoice{II and III only}
        %\wrongchoice{I, II, and III}
    \end{choices}
    \end{multicols}
\end{questionmult}
}

\element{ap}{
\begin{question}{oscillating-spring-q37}
    If the period of a mass $m$ on a spring is $t$,
        then the period of a mass $m$ on a spring with 3 times the spring constant as the first spring would be:
    \begin{multicols}{3}
    \begin{choices}
        \wrongchoice{$3t$}
        \wrongchoice{$\dfrac{t}{3}$}
      \correctchoice{$\dfrac{\sqrt{3}t}{3}$}
        \wrongchoice{$\sqrt{3}t$}
        \wrongchoice{$t$}
    \end{choices}
    \end{multicols}
\end{question}
}

\element{ap}{
\begin{question}{oscillating-spring-q38}
    Which of the following is true of an oscillating spring-mass system?
    \begin{choices}
        \wrongchoice{The velocity is greatest when the potential energy is greatest.}
        \wrongchoice{The force is greatest when the velocity is greatest.}
        \wrongchoice{The kinetic energy is greatest when the force is greatest.}
      \correctchoice{The potential energy is greatest when the acceleration is greatest.}
        \wrongchoice{The potential energy is greatest when the displacement is least.}
    \end{choices}
\end{question}
}


\endinput


