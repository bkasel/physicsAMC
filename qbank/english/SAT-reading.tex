
%%--------------------------------------------------
%% SAT Practice Test One
%%--------------------------------------------------

%% https://collegereadiness.collegeboard.org/pdf/sat-practice-test-1.pdf

%% Directions
%%--------------------------------------------------

%% Each passage or pair of passages below is followed by a number of questions. 
%% After reading each passage or pair, choose the best answer to each question based on what is stated or implied in the passage or passages and in any accompanying graphics (such as a table or graph).

%% NOTE: Unauthorized copying or reuse of any part of this page is illegal.


%% SAT Reading Test
%%--------------------------------------------------
\element{sat-mc}{

Questions 1--10 are based on the following passage.

This passage is from Lydia Minatoya, The Strangeness of Beauty.
\copright{} 1999 by Lydia Minatoya. 
The setting is Japan in 1920. Chie and her daughter Naomi are members of the House of Fuji, a noble family.


Akira came directly, breaking all tradition. 
Was that it? Had he followed form---had he asked his mother to speak to his father to approach a go-between---would Chie have been more receptive?


He came on a winter's eve. 
He pounded on the door while a cold rain beat on the shuttered veranda, so at first Chie thought him only the wind. 
The maid knew better. 
Chie heard her soft scuttling footsteps, the creak of the door. 
Then the maid brought a calling card to the drawing room, for Chie.

Chie was reluctant to go to her guest; perhaps she was feeling too cozy.
She and Naomi were reading at a low table set atop a charcoal brazier. 
A thick quilt spread over the sides of the table so their legs were tucked inside with the heat.


``Who is it at this hour, in this weather?''
Chie questioned as she picked the name card off the maid's lacquer tray.


``Shinoda, Akira. Kobe Dental College,'' she read.
Naomi recognized the name. 
Chie heard a soft intake of air.


``I think you should go,'' said Naomi.


Akira was waiting in the entry. 
He was in his early twenties, slim and serious, wearing the black military-style uniform of a student. 
As he bowed---his hands hanging straight down, a black cap in one, a yellow oil-paper umbrella in the other—Chie glanced beyond him. 
In the glistening surface of the courtyard's rain-drenched paving stones, she saw his reflection like a dark double.


``Madame,'' said Akira, ``forgive my disruption, but I come with a matter of urgency.''


His voice was soft, refined. 
He straightened and stole a deferential peek at her face.


In the dim light his eyes shone with sincerity.
Chie felt herself starting to like him.


``Come inside, get out of this nasty night. 
Surely your business can wait for a moment or two.''


``I don't want to trouble you. 
Normally I would approach you more properly but I've received word of a position. 
I've an opportunity to go to America, as dentist for Seattle's Japanese community.''


``Congratulations,'' Chie said with amusement.
``That is an opportunity, I'm sure. But how am I involved?''


Even noting Naomi's breathless reaction to the name card, Chie had no idea. 
Akira's message, delivered like a formal speech, filled her with maternal amusement. 
You know how children speak so earnestly, so hurriedly, so endearingly about things that have no importance in an adult's mind?  
That's how she viewed him, as a child


It was how she viewed Naomi. 
Even though Naomi was eighteen and training endlessly in the arts needed to make a good marriage, Chie had made no effort to find her a husband.


Akira blushed.

``Depending on your response, I may stay in Japan. 
I've come to ask for Naomi's hand.''


Suddenly Chie felt the dampness of the night.
``Does Naomi know anything of your \ldots{} ambitions?''


``We have an understanding. 
Please don't judge my candidacy by the unseemliness of this proposal. 
I ask directly because the use of a go-between takes much time. 
Either method comes down to the same thing: a matter of parental approval. 
If you give your consent, I become Naomi's yoshi.\footnote{A man who marries a woman of higher status and takes her family's name}

We'll live in the House of Fuji. 
Without your consent, I must go to America, to secure a new home for my bride.''


Eager to make his point, he'd been looking her full in the face. 
Abruptly, his voice turned gentle.
``I see I've startled you. My humble apologies. 
I'll take no more of your evening. 
My address is on my card. 
If you don't wish to contact me, I'll reapproach you in two weeks' time. 
Until then, good night.''


He bowed and left. 
Taking her ease, with effortless grace, like a cat making off with a fish.


``Mother?'' Chie heard Naomi's low voice and turned from the door. 
``He has asked you?''

The sight of Naomi's clear eyes, her dark brows gave Chie strength. 
Maybe his hopes were preposterous.


``Where did you meet such a fellow? 
Imagine! He thinks he can marry the Fuji heir and take her to America all in the snap of his fingers!''
Chie waited for Naomi's ripe laughter.


Naomi was silent. 
She stood a full half minute looking straight into Chie's eyes. 
Finally, she spoke.  ``I met him at my literary meeting.''


Naomi turned to go back into the house, then stopped.

``Mother.''

``Yes?''

``I mean to have him.''


\begin{question}{reading-t01-q01}
    Which choice best describes what happens in the passage?
    \begin{choices}
        \wrongchoice{One character argues with another character who intrudes on her home.}
        \wrongchoice{One character receives a surprising request from another character.}
        \wrongchoice{One character reminisces about choices she has made over the years.}
        \wrongchoice{One character criticizes another character for pursuing an unexpected course of action.}
    \end{choices}
\end{question}

\begin{question}{reading-t01-q02}
    Which choice best describes the developmental pattern of the passage?
    \begin{choices}
        \wrongchoice{A careful analysis of a traditional practice}
        \wrongchoice{A detailed depiction of a meaningful encounter}
        \wrongchoice{A definitive response to a series of questions}
        \wrongchoice{A cheerful recounting of an amusing anecdote}
    \end{choices}
\end{question}

\begin{question}{reading-t01-q03}
    %%  NOTE: line referencing ??
    As used in line 1 and line 65, ``directly'' most nearly means:
    \begin{choices}
        \wrongchoice{frankly}
        \wrongchoice{confidently}
        \wrongchoice{without mediation}
        \wrongchoice{with precision}
    \end{choices}
\end{question}

\begin{question}{reading-t01-q04}
    Which reaction does Akira most fear from Chie?
    \begin{choices}
        \wrongchoice{She will consider his proposal inappropriate.}
        \wrongchoice{She will mistake his earnestness for immaturity.}
        \wrongchoice{She will consider his unscheduled visit an imposition.}
        \wrongchoice{She will underestimate the sincerity of his emotions.}
    \end{choices}
\end{question}

\begin{question}{reading-t01-q05}
    Which choice provides the best evidence for the answer to the previous question?
    \begin{choices}
        \wrongchoice{Line 33 (``His voice \ldots{} refined'')}
        \wrongchoice{Lines 49-51 (``You \ldots{} mind'')}
        \wrongchoice{Lines 63-64 (``Please \ldots{} proposal'')}
        \wrongchoice{Lines 71-72 (``Eager \ldots{} face'')}
    \end{choices}
\end{question}

\begin{question}{reading-t01-q06}
    In the passage, Akira addresses Chie with:
    \begin{choices}
        \wrongchoice{affection but not genuine love.}
        \wrongchoice{objectivity but not complete impartiality.}
        \wrongchoice{amusement but not mocking disparagement.}
        \wrongchoice{respect but not utter deference.}
    \end{choices}
\end{question}

\begin{question}{reading-t01-q07}
    The main purpose of the first paragraph is to:
    \begin{choices}
        \wrongchoice{describe a culture.}
        \wrongchoice{criticize a tradition.}
        \wrongchoice{question a suggestion.}
        \wrongchoice{analyze a reaction.}
    \end{choices}
\end{question}

\begin{question}{reading-t01-q08}
    As used in line 2, ``form'' most nearly means:
    \begin{choices}
        \wrongchoice{appearance}
        \wrongchoice{custom}
        \wrongchoice{structure}
        \wrongchoice{nature}
    \end{choices}
\end{question}

\begin{question}{reading-t01-q09}
    Why does Akira say his meeting with Chie is ``a matter of urgency'' (line 32)?
    \begin{choices}
        \wrongchoice{He fears that his own parents will disapprove of Naomi.}
        \wrongchoice{He worries that Naomi will reject him and marry someone else.}
        \wrongchoice{He has been offered an attractive job in another country.}
        \wrongchoice{He knows that Chie is unaware of his feelings for Naomi.}
    \end{choices}
\end{question}

\begin{question}{reading-t01-q10}
    Which choice provides the best evidence for the answer to the previous question?
    \begin{choices}
        \wrongchoice{Line 39 (``I don't \ldots{} you'')}
        \wrongchoice{Lines 39-42 (``Normally \ldots{} community'')}
        \wrongchoice{Lines 58-59 (``Depending \ldots{} Japan'')}
        \wrongchoice{Lines 72-73 (``I see \ldots{} you'')}
    \end{choices}
\end{question}
}


\element{sat-mc}{

\begin{question}{reading-t01-q11}
    Which choice provides the best evidence for the answer to the previous question?
    \begin{choices}
        \wrongchoice{Line 39 (``I don't \ldots{} you'')}
        \wrongchoice{Lines 39-42 (``Normally \ldots{} community'')}
        \wrongchoice{Lines 58-59 (``Depending \ldots{} Japan'')}
        \wrongchoice{Lines 72-73 (``I see \ldots{} you'')}
    \end{choices}
\end{question}
}


Questions 11-21 are based on the following
passage and supplementary material.
This passage is adapted from Francis J. Flynn and Gabrielle
S. Adams, "Money Can't Buy Love: Asymmetric Beliefs about
Gift Price and Feelings of Appreciation." ©2008 by Elsevier
Inc.
Line
5
10
15
20
25
30
35
40
Every day, millions of shoppers hit the stores in
full force—both online and on foot—searching
frantically for the perfect gift. Last year, Americans
spent over $30 billion at retail stores in the month of
December alone. Aside from purchasing holiday
gifts, most people regularly buy presents for other
occasions throughout the year, including weddings,
birthdays, anniversaries, graduations, and baby
showers. This frequent experience of gift-giving can
engender ambivalent feelings in gift-givers. Many
relish the opportunity to buy presents because
gift-giving offers a powerful means to build stronger
bonds with one’s closest peers. At the same time,
many dread the thought of buying gifts; they worry
that their purchases will disappoint rather than
delight the intended recipients.
Anthropologists describe gift-giving as a positive
social process, serving various political, religious, and
psychological functions. Economists, however, offer
a less favorable view. According to Waldfogel (1993),
gift-giving represents an objective waste of resources.
People buy gifts that recipients would not choose to
buy on their own, or at least not spend as much
money to purchase (a phenomenon referred to as
‘‘the deadweight loss of Christmas”). To wit, givers
are likely to spend $100 to purchase a gift that
receivers would spend only $80 to buy themselves.
This ‘‘deadweight loss” suggests that gift-givers are
not very good at predicting what gifts others will
appreciate. That in itself is not surprising to social
psychologists. Research has found that people often
struggle to take account of others’ perspectives—
their insights are subject to egocentrism, social
projection, and multiple attribution errors.
What is surprising is that gift-givers have
considerable experience acting as both gift-givers and
gift-recipients, but nevertheless tend to overspend
each time they set out to purchase a meaningful gift.
In the present research, we propose a unique
psychological explanation for this overspending
problem—i.e., that gift-givers equate how much they


spend with how much recipients will appreciate the
gift (the more expensive the gift, the stronger a
gift-recipient’s feelings of appreciation). Although a
link between gift price and feelings of appreciation
might seem intuitive to gift-givers, such an
assumption may be unfounded. Indeed, we propose
that gift-recipients will be less inclined to base their
feelings of appreciation on the magnitude of a gift
than givers assume.
Why do gift-givers assume that gift price is closely
linked to gift-recipients’ feelings of appreciation?
Perhaps givers believe that bigger (i.e., more
expensive) gifts convey stronger signals of
thoughtfulness and consideration. According to
Camerer (1988) and others, gift-giving represents a
symbolic ritual, whereby gift-givers attempt to signal
their positive attitudes toward the intended recipient
and their willingness to invest resources in a future
relationship. In this sense, gift-givers may be
motivated to spend more money on a gift in order to
send a “stronger signal” to their intended recipient.
As for gift-recipients, they may not construe smaller
and larger gifts as representing smaller and larger
signals of thoughtfulness and consideration.
The notion of gift-givers and gift-recipients being
unable to account for the other party’s perspective
seems puzzling because people slip in and out of
these roles every day, and, in some cases, multiple
times in the course of the same day. Yet, despite the
extensive experience that people have as both givers
and receivers, they often struggle to transfer
information gained from one role (e.g., as a giver)
and apply it in another, complementary role (e.g., as
a receiver). In theoretical terms, people fail to utilize
information about their own preferences and
experiences in order to produce more efficient
outcomes in their exchange relations. In practical
terms, people spend hundreds of dollars each year on
gifts, but somehow never learn to calibrate their gift
expenditures according to personal insight.




The authors most likely use the examples in lines 1-9
of the passage (“Every . . . showers”) to highlight the
A) regularity with which people shop for gifts.
B) recent increase in the amount of money spent on
gifts.
C) anxiety gift shopping causes for consumers.
D) number of special occasions involving
gift-giving.
12
In line 10, the word “ambivalent” most nearly means
A) unrealistic.
B) conflicted.
C) apprehensive.
D) supportive.



The authors indicate that people value gift-giving
because they feel it
A) functions as a form of self-expression.
B) is an inexpensive way to show appreciation.
C) requires the gift-recipient to reciprocate.
D) can serve to strengthen a relationship.
14
Which choice provides the best evidence for the
answer to the previous question?
A) Lines 10-13 (“Many . . . peers”)
B) Lines 22-23 (“People . . . own”)
C) Lines 31-32 (“Research . . . perspectives”)
D) Lines 44-47 (“Although . . . unfounded”)
15
The “social psychologists” mentioned in paragraph 2
(lines 17-34) would likely describe the “deadweight
loss” phenomenon as
A) predictable.
B) questionable.
C) disturbing.
D) unprecedented.
16
The passage indicates that the assumption made by
gift-givers in lines 41-44 may be
A) insincere.
B) unreasonable.
C) incorrect.
D) substantiated.



Which choice provides the best evidence for the
answer to the previous question?
A) Lines 53-55 (“Perhaps . . . consideration”)
B) Lines 55-60 (“According . . . relationship”)
C) Lines 63-65 (“As . . . consideration”)
D) Lines 75-78 (“In . . . relations”)
18
As it is used in line 54, “convey” most nearly means
A) transport.
B) counteract.
C) exchange.
D) communicate.
19
The authors refer to work by Camerer and others
(line 56) in order to
A) offer an explanation.
B) introduce an argument.
C) question a motive.
D) support a conclusion.


The graph following the passage offers evidence that
gift-givers base their predictions of how much a gift
will be appreciated on
A) the appreciation level of the gift-recipients.
B) the monetary value of the gift.
C) their own desires for the gifts they purchase.
D) their relationship with the gift-recipients.
21
The authors would likely attribute the differences in
gift-giver and recipient mean appreciation as
represented in the graph to
A) an inability to shift perspective.
B) an increasingly materialistic culture.
C) a growing opposition to gift-giving.
D) a misunderstanding of intentions.




\endinput


