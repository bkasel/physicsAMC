
%%--------------------------------------------------
%% SAT: Quantitative Ability
%%--------------------------------------------------

%%--------------------------------------------------
%% References
%%--------------------------------------------------

%% http://www.bestsamplequestions.com/sat-sample-questions/quantitative/quantitative.html


%% SAT Multiple Choice Questions
%%--------------------------------------------------
\element{sat-mc}{
\begin{question}{quantitative-ability-Q01}
    If $a/b = 0.625$, then $b/a$ is equal to which of the following?
    \begin{multicols}{3}
    \begin{choices}
        %% What !! ?? WHAT !!!!!!!!!!!!!!!!
        \wrongchoice{1.50}
      \correctchoice{2.67}
        \wrongchoice{2.80}
        \wrongchoice{3.53}
        \wrongchoice{4.85}
    \end{choices}
    \end{multicols}
\end{question}
}

\element{sat-mc}{
\begin{question}{quantitative-ability-Q02}
    \begin{multicols}{3}
    \begin{choices}
        \wrongchoice{1.50}
    \end{choices}
    \end{multicols}
\end{question}
}

\begin{comment}
    Explanation: Make the base on both sides of the equation the same. 
    81 is similar as 34. So 34x  = 34, which means that x = 1

    If x - 3 = 3 (1 - x), then what is the value of x?

        0.33
        0.57
        1.40
        1.97
        2.15 

    Answer (a)

    Explanation: The fraction b/a is a reciprocal of a/b. To find the numerical value of b/a, just flip the numerical value of a/b, which is 0.625. Your calculation will tell you that 1/0.625 = 1.6.

    Points A, B C and D are arranged in a line in the same order. If AC = 13, BD = 14 and AD = 21 then BC =

        12
        9
        8
        6
        3 

    Answer (c)

    Explanation: To solve the problem algebraically isolate x one step at a time. First multiply through by 3 on the right, x - 3 = 3 - 3x. Next add 3x to each side and then add 3 to each side to get 4x = 6. Divide each side by 4 to get x = 6/4 or 1.5

    The distance between the points (-3, 5) and (-3, -12) is

        √13
        48
        8
        17
        √50 

    Answer (d)

    Explanation: Notice that the x coordinate in both points is the same. So you just have to find the difference in the y coordinates. The difference is 5 - (-12) = 17

    If the cube root of the square root of a number is 2, what is the number?

        19
        64
        148
        286
        1,094 

    Answer (b)

    Explanation: Translate the whole into math's 3√x = 2. Now take away. First cube both sides. √x = 8. Now square both sides; x = 64

    From a pack of 52 cards find the number of ways in which

        A ruler or the queen can be drawn
        Both a ruler and the Queen can be drawn 

        18 ways
        25 ways
        16 ways
        23 ways
        85 ways 

    Answer (c)

    Explanation: A ruler can be drawn in 4 dissimilar ways. A queen can be drawn in 4 different ways. By fundamental principle of addition a ruler or a queen can be drawn in 4 + 4 = 8 ways. By fundamental principles of multiplication a ruler and queen can be drawn in 4 * 4 = 16 ways

    How many words can be made with the letters of the word Mathematics? In how many of them do the vowels occur together?

        120960 ways
        125789 ways
        147874 ways
        568784 ways
        789654 ways 

    Answer (a)

    Explanation: Total alphabets is equal to 11, M, A, T Occur twice. Total arrangements is equal to 11 (2/2/2) = 4989600. Treat the four vowels A, A, E and I as one unit. This with remaining 7 letters of which two are repeated can be arranged among themselves in 4/2 ways. Therefore total number of ways = (8/2/2) (4/2) = 120960

    Twenty people are invited to a party. In how many ways can they and the host be seated at a rounded table? In how many of these ways will two particular people be seated on either sides of the host.

        21 ways
        18 ways
        15 ways
        23 ways
        85 ways 

    Answer (b)

    Explanation: There are 20 + 1 = 21 people to be seated at the table. Fixing the seat of one person the remaining twenty can be seated in twenty ways. Two particular can  sit on both sides of the host in 2 ways. Remaining 18 can arrange themselves in 18 ways. Therefore the answer is 18.

    Ten different letters are given. Words with five letters are to be formed from these given letters. Find the number of words which have at least one letter repeated

        69760
        45123
        78541
        12365
        78954 

    Answer (a)

    Explanation: When there is no repetition of alphabets. Total number of words with 5 letters = 10p5 = 30240. When any alphabet is repeated any number of times, total number of words = 105

    Therefore required numbers of words = 100000 - 30240 = 69760

    A photograph of 4 players is to be taken from 11 players of a cricket team. How many different photographs can be taken if the captain and vice captain have to be present in each photograph

        Must be included
        Are never included 

        568
        864
        125
        478
        899 

    Answer (b)

    Explanation: Two players can be chosen in 9c2 ways.

        Total number of photographs = 9c2 * 4 = 864
        Total number of photographs is equal to 9p4 

%% page 2 prob 11


If 7a + 2b = 11 and a - 2b = 5, then what is the value of a?

    - 5.0
    - 8.5
    - 0.8
    1.9
    2.0 

Answer (e)

Explanation: When you are given two equations extremely similar in form you are probably looking at classic ETS style simultaneous equations. Adding the two equations cancels out the b term leaving you with the equation 8a = 16, so a = 2

If f(x) = x2 - 3X, then f(-3) =

    0
    2.3
    9.0
    5.9
    18.0 

Answer (e)

Explanation: Plug - 3 into the function: (-3)2 - 3 (-3) = 18

If m varies directly as n and m/n = 5, then what is the value of m when n = 2.2?

    0.74
    9.25
    8.60
    5.80
    11.00 

Answer (e)

Explanation: Direct variation between two quantities means that they always have the same quotient. In this case, it means that m/n must always be equal to 5. To find the value of m when n = 2.2, set up the equation m/2.2 = 5 and solve for m. You will find that m = 11

What is the slope of the line given by the equation 3y - 5 = 7 - 2x?

    - 8
    - 2/3
    3/8
    3
    9 

Answer (b)

Explanation: To find the slope of the line easily, get its equation into the form y = mx + b, where m will be the value of the slope. To express 3y - 5 = 7 - 2x in this form, just isolate y. You will find that y = -2/3 + 4. Here the slope of the line (m) is - 2/3

(n3)6 * (n4)5/n2 =

    N9
    N 45
    N 14
    N 36
    N 23 

Answer (d)

Explanation: A quick review of exponent rules. When raising powers to powers; multiply exponents. When multiplying powers of the same base add exponents and when dividing powers of the same base, subtract exponents. For this problem you have to do all three.

(N3)6 * (N4)5/N2 = N18 * N20/N2 = N38/N2 = N36

If f(x) = 5 - 2x and g (x) = x2 + 7, then f (g(2)) =

    - 17
    - 8
    9
    12
    28 

Answer (a)

Explanation: Work from the inside out, g(2) = 22 + 7 = 11. Now place 11 for x f(11) = 5 - 2 (11) = - 17

If the perimeter of a square is 60, what is the area of the square?

    13√3
    25√2
    56
    185
    225 

Answer (e)

Explanation: The sides of the square are the same. So each side must be 15. Since the area of a square is side 2 , therefore it is (15)2 = 225.

Which of the following lines is perpendicular to the line 3x - 2y =16?

    3x - 2y = 25
    3x + 2y = 16
    2x - 3y = 7
    6x  + 9y = 16
    6x - 9y = 32 

Answer (d)

Explanation: Move the equation around so that it is in y = mx + b formula: y = 3/2x - 8. So an equation perpendicular would have a slope of - 2/3. Therefore the answer would be (d)

The surface area of a sphere is 75 square centimeters. What is the volume of the sphere in cubic centimeters?

    2.253
    8.788
    17.231
    12.85
    61.075 

Answer (e) Use the formula for surface area of a sphere with radius r. In case you forget the equation, it is given in the reference information at the beginning of the test: S = 4Πr2. You are given S, so write 75 = 4Πr2 and therefore r = 2.443. The formula for volume of a sphere with radius r is also given in the reference info. Its V = 4/3Πr3. Now, plug r = 2.443 into the formula, therefore the correct answer V = 61.075

Where defined, {x2 - 4/4} {8/2x + 4} =

    6
    5X
    X - 2
    X + 8
    9x2 - 2 

Answer (c)

Explanation: You can just factor this one and then cancel

{x2 - 4/4} {8/2x + 4} = {(x + 2) (x - 2)/4} {2 * 4/2(x + 2)} = x - 2 


%% page 3 prob 21


    At what coordinates does the graph of 3y+ 5 = x - 1 intersects the y axis?

        (0 , -2)
        (0 , -1)
        (0, 1/3)
        (-2, 0)
        (-6,0) 

    Answer (a)

    Explanation: The value at which a line intersects the y axis is called the y intercept; that is the y coordinate of the point of intersection. This means that the  x coordinate (or the value of x) should be  zero at each point on the y axis, so eliminate answer (d) and (e). If you put the line 3y + 5 = x - 1 into the form of y = mx + b then b will be the y intercept. The rearranged equation looks like the form y = 1/3x - 2. The y intercept is - 2 and you know the x coordinate must be zero, so the point of intersection has the coordinates (0, -2)

    Ronaldo is starting a small pump selling business. If he spends $200 on supplies and sells his pumps for $4 each, which of the following functions correctly shows the amount of money Ronaldo has gained or lost when he has sold x pumps?

        F(x) = 800x
        F(x) = 200x + 4
        F(x) = 200x - 4
        F(x) = 4x + 200
        F(x) = 4x - 200 

    Answer (e)

    Explanation: Plug in x = 1,000. Then Ronaldo has earned 4x*1,000 = $4000 and spent $200 so he has made a total of $3,800, therefore the answer is (e)

    If 0 < n < 1 then all of the following must be true except:

        N2 < n
        N < square root of n
        N < n
        -n < n
        N < 1/n 

    Answer (c)

    Explanation: Plug in 0.5 for n. Now n2 = 0.25, square root of n = 0.707, n = 0.5, - n = - 0.5, -n = -0.5, and 1/n = 2. This makes A B D and E true and (c) is false.

    If f(x) =2x2+ 2 then what is the value of f(x + 4)?

        2x2+4
        2x2 + 6
        2x2+ x +6
        2x2+ 16 +32
        2x2+ 16x +34 

    Answer (e)

    Explanation: Since there are variables in the answer choices you should insert in. Try x = 3. We are trying to find f(3 + 4) = f(7) = 2(7)2 + 2 which is 100 our target number. Now insert 3 in for x in the answer choices to see which answer choice goes with the target its (e)

    If set S consists of ten distinct positive integers, which of the following could be a member of S?

        The mean of the members of S
        The median of the members of S
        The mode of the members of S 

        None
        1 only
        2 only
        1 and 3
        2 and 3 

    Answer (b)

    Explanation: The median of the 10 numbers will be the average of the fifth and sixth numbers, because the numbers are distinct, the fifth and sixth numbers cannot be the same so the medium will be between them. Therefore the answer is b.

    If a cube and sphere intersect at exactly eight points then which of the following must be true?

        The sphere is inscribed in the cube
        The cube is inscribed in the sphere
        The diameter of the sphere is equal in length to an edge of the cube
        The sphere and the cube have equal volumes.
        The sphere and the cube have equal surface areas 

    Answer (b)

    Explanation: When a cube is inscribed in a sphere each of the cube's 8 corners touch the inside of the sphere, therefore the answer is (b)

    Six congruent circles are arranged so that each circle is tangent externally to at least two other circles. The centers of the six circles are then connected to form a polygon. If each circle has a radius of 2 then what is the perimeter of this polygon?

        6
        12
        24
        36
        48 

    Answer (c)

    Explanation: Arrange the circles and connect them the resulting polygon will be a hexagon which always has a perimeter of 24. That's because the polygon is always made up of two radii for a total of 12 radii each with the length of 2.

    How many distinct three digit number contain only nonzero digit?

        934
        828
        746
        729
        567 

    Answer (d)

    Explanation: In a three digit number containing no 0. There are nine possibilities for the first digit 1 to 9, 9 possibilities for the 2nd digit 1 to 9, 9th possibilities for the third digit 1 to 9. This makes the total of 9*9*9 possible 3 digit numbers or 729

    If z = log, (yx) then xz =

        xx
        yx
        yx
        xy
        y 

    Answer(b)

    Explanation: If (yx) = z is a component then z that turn x into yz. You can compute the value of z = 3.0103. The only answer choice that equals to 1,024. Therefore (b) is the answer.

    What is the slope of the line given by the equation 3y - 5 = 7 - 2x?

        - 5
        - 2/3
        3/13
        67
        34 

    Answer (b)

    Explanation: To find the slope of the line simply, get its equation into the form y = yx + b, where x will be the value of the slope. To find 3y - 5 = 7 - 2x in this form, just take away y. You will find that y = -2/3 + 4. Therefore the answer is (b) 

%% page 4 31

If log 5 = 7 then y =

    45/23
    78.45
    1.09
    4j
    32.45 

Answer (c)

Explanation: The equation log 5 = 7 in exponential form, j7= 5. Now you have to find the value of j, you are required to take the 7th root of both the sides. You will see that j = 1.09. The correct option is (c).

If 0 < p < 1 then all of the following must be true except:

    78/1
    px
    p < p
    p/4
    p <5 

Answer (c)

Explanation: First place the amount 0.5 in place of g. Now p3 = 0.25, cube root of n = 0.707, p = 0.5, - p = - 0.5, -p = -0.5, and 1/p = 2. This makes (a), (b) (d ) and (e) true and (c) is false.

What is the distance between the c intercept and the h intercept of the line given by the equation

2h = 6 - c?

    12/78
    159
    6.71
    4561
    45/2 

Answer (c)

Explanation: To find the value h you have to intercept and make c = 0 and solve for h. h= 3. Now to find the c intercept, make h = 0 and solve for c. You will see that they form a right angle triangle with length of 3 and 6, in which the hypotenuse stand for the distance in between the two different points. You are required to use  Pythagoras theorem to find the length of the hypotenuse, which will be equivalent to 6.7156 approximately; therefore the answer is (c).

If z = log, (fx) then bz =

    zf
    zx
    z/1
    fz
    b 

Answer (b)

Explanation: If (fx) = z (which is a component) you can calculate the value of z = 3.0203. The only answer choice that equals to 1,034. Therefore (b) is the answer.

If p/m = 0.625, then m/p is equal to which of the following?

    146
    2.67
    45/3
    951
    357 

Answer (b)

Explanation: Make the base on both sides of the equation the same. 8 is similar as 23. So 23x  = 23, which means that y = 1

The distance between the points (-3, 5) and (-3, -18) is

    78/5
    13
    753
    23
    784 

%% page 5 41


    A fair cube is one that is labeled with the numbers 1, 2, 3, 4, 5, and 6 such that there is an equal probability of rolling each of those numbers. If jade rolls two fair cubes at the same time then what is the probability that the product of the two numbers she rolls will be greater than 18?

        0.222
        0.568
        0.7
        0.3
        0.658 

    Answer (a)

    Explanation: If you want to find the probability first find out the total number of possibilities and then figure out how many meet the condition you want. Since there are 6 possible rolls on a fair cube, the total number of possibilities for two rolls is 6 * 6 = 36. Now you are required to figure out all the ways to get product greater than 18. The rolls that will work are 4 * 5, 4 * 6, 5 *5, 5 * 4, 6 * 5, 6 * 6, 6 *4, 5 * 6. That's 8 rolls out of 36 which is a probability of 8/36 = 0.222

    If f(x) = 4x2+4x+4, which of the following is equal to f(-3.5)?

        f9
        f89
        f(2.5)
        f78
        f(5.7) 

    Answer (c)

    Explanation: You require a graphing calculator, you need to press the Y key and enter the function. You will check the value of the Table you can find that f(-3.5) and f(2.5) both are equal to 39. But if you do not possess graphing calculator, you can make the use of PITA.

    Which of the following has the greatest value?

        4400
        290600
        1.798
        36
        78 

    Answer (a)

    Explanation: The important fact about this interesting question is that it not essential to find the accurate value of the expression. You need to rearrange as many answer choices as possible so that they possess the exponents of 100. Take a close look on each an expression 290600, 36, 78, 1.798, 4400. It will make clear that (a) is bigger than (c), therefore the answer is (a)

    Which of the following represent the solution set of Ιx3 - 8Ι ≤ 5?

        2.5 ≤x ≤ 1.98
        8.6  ≤x ≤ 1.23
        8.5 ≤x ≤ 1.21
        1.44 ≤ x ≤ 2.35
        0.5 ≤x ≤ 1.81 

    Answer (d)

    Explanation: In this problem choose a simple number that is in some, but not all of the ranges in the answer choices. Assume x = 1 this gives you [(1)3- 8] ≤ 5, which means 7 ≤ 5, but this is not correct. Remove any answer choice that includes one. Now use x = 2, this comes to [(2)2 - 8]≤ 5, which simplifies to 0 ≤ 5, therefore the answer is (d)

    The researcher finds that an ant colony's population increases by exactly 8% each month. If the colony has an initial population of 1,250 insects, which of the following is the nearest approximation of the population of the colony two years later?

        7,926
        5,879
        4,213
        7,984
        2,125 

    Answer (a)

    Explanation: You are require to use the formula Final = Original * (1 + rate)# of changes This colony has a original size of 1,250 and increases by 8% each month. In 2 years it will make 24 of these increases. If you will pay attention you will find that final = 7,926.456 which is very close to the option (a).

    What is the distance between the x intercept and the y intercept of the line given by the equation

    2y = 6 - x?

        8.56
        2.45
        6.71
        5.64
        1.89 

    Answer (c)

    Explanation: To find the y - intercept just make x = 0 and solve for y. Y= 3. Now to find the x intercept, make y = 0 and solve for x. You will see that they form a right angle triangle with length of 3 and 6, in which the hypotenuse represents the distance in between the  two distinct points. You are required to used Pythagoras theorem to find the length of the hypotenuse, which will be equal to 6.7156 approximately; therefore the answer is (c).

    If 0 < q < 1 then all of the following must be true except:

        Q3 < n
        q < cube root of q
        q < q
        -q < q
        q < 1/q 

    Answer (c)

    Explanation: Apply 0.5 in place of q and try to solve the equation. Now q2 = 0.25, square root of q = 0.707, q = 0.5, - q = - 0.5, -q = -0.5, and 1/q = 2. This makes A B D and E true, therefore (c) is false.

    If x - 3 = 3 (2- x), then what is the value of x?

        0.33
        0.87
        8.80
        1.87
        2.55 

    Answer (a)

    Explanation: The fraction b/a is a reciprocal of a/b. To find the numerical value of b/a, just change the numerical value of a/b, which comes to 0.625. Your calculation will tell you that 1/0.625 = 1.6.

    If the cube root of the square root of a number is 3, what is the number?

        19
        54
        148
        286
        1,094 

    Answer (b)

    Explanation: Translate the whole into math's 3√x = 3. Now take away. First cube both sides. √x = 8. Now square both sides; x = 54

    What is the slope of the line given by the equation

    Y = 3 = 5/4(x - 7)?

        7/4
        8/1
        -3/4
        2/4
        5/4 

    Explanation: You might observe that this is the point slope form and that the slope is 5/4, if you are still not clear you can rewrite the equation into the slope intercept form. 

%% page 6 51


    If 0 < p < 1 then all of the following must be true except:

        p2 < n
        p < cube root of p
        p < p
        -p < p
        p < 1/p 

    Answer (c)

    Explanation: Place 0.5 for p. Now p2 = 0.25, square root of p = 0.707, p = 0.5, - p = - 0.5, -p = -0.5, and 1/p = 2. This makes A B D and E true and (c) is false.

    If set Q consists of eleven different positive integers, which of the following could be a member of Q?

        The mean of the members of Q
        The median of the members of Q
        The mode of the members of Q 

        None of the above
        First statement
        Second statement
        First and Second statement
        Third and first statement 

    Answer (b)

    Explanation: The median of the 11 numbers will be the average of the fifth and sixth numbers, because the numbers are different, the fifth and sixth numbers cannot be the similar so the medium will be between them. Therefore the answer  would be (b).

    How many different three digit numbers include only non-zero digit?

        126
        767
        989
        729
        565 

    Answer (d)

    Explanation: In a three digit number containing no 0, there are actually nine possibilities for the first digit from 1 to 9, 9 possibilities for the 2nd digit from 1 to 9, 9 possibilities for the third digit from 1 to 9. This makes the total of 9*9*9 possible 3 digit numbers or the option (d) that is 729.

    After 8:00pm a ride in a taxi costs $2.50 plus $0.30 for every fifth of a mile traveled. If the passenger travels b miles, then what is the cost of the trip, in dollars in terms of b?

        6.7 + 8.9b
        2.5 + 1.5b
        5.3b
        27 + 370b
        457 + 50b 

    Answer (b)

    Explanation: Assume that you are travelling 6 miles (b = 6). To find the right answer place 5 in place of b and see which one gives you value of 9 and you will find that the correct answer is (b).

    What is the slope of the line given by the equation

    Y + 3 = 5/4 (x - 7)?

        7/9
        5/8
        4/9
        4/6
        5/4 

    Answer (e)

    Explanation: You may see that this is the point slope form and that the slope is 5/4. If you don't see that rewrite the equation into slope form, i.e., y=mx+c form.

    If a = cos Θ and b = sin Θ, then for all Θ, a2 + b2 =

        7
        1
        67
        54
        8 

    Answer (b)

    Explanation: Substitute for a and b. So a2 + b2 = cos2 Θ + sin 2 Θ. But cos2 Θ + sin 2 Θ = 1. Therefore the option is (b)

    If (y - 3) = 4y - 7, then which of the following could be the value of y?

        8/6
        56
        24
        2
        6/9 

    Answer (d)

    Explanation: The first equation is y - 3 = 4y - 7, which simplifies to y = 4/3 but this is not the answer. Let's look at the second equation -(y - 3) = 4y - 7. This simplifies to 3 - y = 4y - 7 that means 5y = 10 and y = 2. Therefore, (d) is the right answer.

    If r = 2/3 and s = 6 then s/r + 4/r2 =

        45
        62
        75
        12
        18 

    Answer (e)

    Explanation: You have to just insert the values. Insert the values given for r and s into the equation

    s/r + 4/r2 =

    6/2 + 4/(2/3)2 =6/2/3 + 4/4/9 =

    (6*3/2) + (4 * 9/4) =

    18/2 + 36/4 = 9 + 9 = 18 therefore the answer is (d)

    Michel and Donaldo together weigh around 300 pounds. Donaldo and Hitler together weigh around 240 pounds. If all of them weigh around 430 pounds then what is Donaldo's weight in pounds?

        320
        115
        980
        130
        140 

    Answer (d)

    Explanation: If Donaldo's weight is around 120 pounds, that makes Michel's weight around 180 pounds, and Hitler's weight  120 pounds, these three weights doesn't add up to 410 pounds, so you have to choose the option (d).

    As a saleswoman, Kelie receives a \$10,00 commission for each unit she sells more than 300 units, she receives an additional bonus of \$1,000,00. What was the total amount Kelie received as bonuses in 1996?

        \$2,00,789
        \$4,00,000
        \$9,00,000
        \$7,00,980
        \$8,30,900 

    Answer (b)

    Explanation: Kelie sold more than 300 units in only 4 months in the year 1996; therefore, 4 months bonus will be equal to the option \$4,00,000, which is option (b) 


%% Page 7 61

Below are the few questions on Multiple Choices. Choose the correct answer from the following questions

    In the above figure sin angle BAC =

        5/13
        8/45
        14/15
        34/12
        12/18 

    Answer (a)

    Explanation: In a right angle triangle the sine of an angle equals the length of the opposite side over the length of the hypotenuse. You can easily find the length of the opposite side over the length of hypotenuse. You can easily find the length of sides AB and BC. The hypotenuse length can be found with the Pythagorous theorem. To find the sine of angle BAC, then you just need to find the value of BC/AC which equals to 5/13.

    15!/13!2! =

        54
        78
        91
        105
        312 

    Answer (d)

    Explanation: You need to expand the factorials in the fraction 15!/13!2!, then you get

    15*14*13*11*12*9*10*7*6*5*3*4*2*1

    --------------------------------------------

    (13*12*11*9*10*8*7*6*5*4*2*3*1)(1*2)

     

    You will observe that every factor from 1 to 13 in the numerator is also in the denominator, it is only left with 15*14/2*1 or 210/2 which comes to 105.

     

    If f(x) = |x| + 10 for which of the following values of x does f(x) = f(-x)?

        15 f
        18x
        All real x
        10 except x
        10x 

    Answer (c)

     

    Explanation: To start with the equation the statement f(x) = f(-x) when x = -10 and 10. A simple number like zero works best. f(0) = Ι0Ι + 10 = 10. You can see that f(0) = f(-0) so zero must be the part of correct answer, therefore (c) is the correct option.

    If f(x) = 3√x and g(x) = ½ √x + 1, then f (g(2.3)) =

        5.4
        1.2
        8.6
        4.3
        2.8 

    Answer (b)

    This is a compound question, in which you are required to apply two functions in combination. You require to place the numbers 2 and 3 in place of x is the definition of g(x) g(2,3) = ½ √2.3 is equal to f (1.76), which can easily be solved by f(1.76) = 2√1.76 = 1.21. The correct option is (b).

    If x mod y is the remainder when x is divided by y, then (61 mod 7) - (5 mod 5) =

        9
        10
        6
        5
        63 

    Answer (d)

    Explanation: To find the value of y and x just take the number as the x position and divide it by the number in y position. The remainder is the value of x and y of those numbers. The value of 61 and 7 is 5. The expression (61 mod 7) - (5 mod 5) can be rewritten as 5 - 0 which equals 5. Therefore the correct answer is (d).

    Which of the following must be true?
        Sin(- ?) = -Sin ?
        Cos (- ?) = -cos ?
        Tan (- ?) = - tan ? (Where tan ? is defined
     

        1 and 2 only
        1 only
        3 and 2 only
        1 and 3 only
        2 only 

    Answer (d)

    Explanation: You need to solve this question by using the trignometry theory. Place the few numbers for example angles between 0° and 90°, you will soon find out that statement 1 and 3 always will be true, but it's easy to find an exception to statement 2, therefore the right answer is (d)

     

    If the ratio of sec x to csc x is 1:4, then the ratio of tan x to cot x is

        1:16
        2:7
        5:6
        8:14
        3:8 

    Answer (a)

    Explanation: The ratio which is given can be written in fractional form like sec x/csc x = ¼. The secant and cosecant can also be expressed in terms of sine and cosine. 1/cos x/1/sin x = ¼. The cotangent is the reciprocal of the tangent so cot x = 4. Therefore the correct answer would be (a)

    If a/b = 0.595, then b/a is equal to which of the following?

        6.45
        2.67
        8.41
        9.13
        5.15 

    Answer (b)

    Explanation: Make the base on both sides of the equation the same. 243 is similar as 35. So 35x  = 35, which means that x = 1

    The distance between the points (-3, 6) and (-3, -14) is

        √15
        56
        4
        17
        √25 

    Answer (d)

    Explanation: Notice that the x coordinate in both points is the same. So you just have to find the difference in the y coordinates. The difference is 6 - (-14) = 17

    If x - 5 = 5 (1 - x), then what is the value of x?

        0.33
        2.78
        9.45
        8.45
        6.45 

    Answer (a)

    Explanation: The fraction b/a is a equal to a/b. To find the numerical value of b/a just turn the numerical value of a/b; this  comes to 0.625. Your calculation will tell you that 1/0.625 = 1.6. 

%% page 8 71


    If the ratio of sec x to cos x is 1:4, then the ratio of tan x to cot x is

        1:16
        2:78
        5:12
        3:47
        1:55 

    Answer (a)

    Explanation: The ratio which is given can be written in fractional form like sec x/csc x = ¼. The secant and cosecant can also be expressed in terms of sine and cosine. i/cos x/1/sin x = ¼. The fraction simplifies to sin x/cos x = ¼ and since tan x = sin x/cos x, this can also be written as tan x = ¼. Therefore the correct answer is (a)

    If log 2 = 8 then y =

        5.65
        8.74
        1.09
        8.45
        6.12 

    Answer (c )

    Explanation: The equation log 2 = 8 in exponential form, y8= 2. Now you have to find the value of y, you are required to take the 8th root of both the sides. You will see that y = 1.09. The correct option is (c).

    Points E, F G and H are arranged in a line in the same order. If EG = 13, FH = 14 and EH = 21 then FG =

        98
        5
        8
        14
        52 

    Answer (c)

    Explanation: To answer the problem algebraically separate x one step at a time. First you are required to multiply by 4 on the right, x - 4 = 4 - 4x. Next add 3x to each side and then add 4 to each side to get 4x = 6. Divide each side by 4 to get x = 8

    If 0 < g < 1 then all of the following must be true except:

        g2 < g
        g < cube root of g
        g < g
        -g < g
        g < 1/g 

    Answer (c)

    Explanation: First place the amount 0.5 in place of g. Now g3 = 0.25, cube root of n = 0.707, g = 0.5, - g = - 0.5, -g = -0.5, and 1/g = 2. This makes A B D and E true and (c) is false.

    If r = log, (qx) then kz=

        kr
        yx
        rq
        qk
        k 

    Answer (b)

    Explanation: If (qx) = r (which is a component) You can compute the value of r = 3.0203. The only answer choice that equals to 1,034. Therefore (b) is the answer.

    What is the slope of the line given by the equation 7z - 5 = 7 - 2m?

        - 45
        - 2/3
        23
        5/4
        -25 

    Answer (b)

    Explanation: To find the slope of the line you have to just, get its equation into the form z = zm + b, where m will be the value of the slope. To find 7z - 5 = 7 - 2m in this form, just take away y. You will find that z = -2/7 + 4. Therefore the answer is (b)

    If m - q > m + q, then which of the following must be true?

        q/m
        q < 0
        m > 0
        q > 0
        m / q 

    Answer (b)

    Explanation: Algebraic manipulation is the simplest way to solve this problem. You need to add q to each side which will give inequality m > m+ 2q. Now you need to subtract from each side to get 0 > 2q and therefore the answer is (b)

    If mn is not equal to 0 then mn - m/n / m/n =

        mn - 12
        mn + 9
        m - 7
        n + 8
        n2 - 1 

    Answer (e)

    Explanation: In this problem it is significant to choose the number that makes the fraction m/n work out conveniently. Take m = 4 and n = 2 as it will make m/n = 2. Now put the values in the equation and you will find the correct option is (e)

    If for all the actual numbers y a function g(y) is defined by g(y) = {2, y ≠ 13 4, y = 13}

    Then g(15) - g(14) =

        54
        0
        78
        12
        89 

    Answer (b)

    Explanation: The equation g(y) = {2, y ≠ 13 4, y = 13} no values are given which is equal to 13, the functions will always come out to 2. g(15) - g(14) = 2 - 2 = 0, therefore the right answer is (b)

    If j5/35 = 35, then what is the value of j

        23.7
        3.45
        12.6
        3.62
        21.64 

    Answer (d)

    Explanation: First you are required to multiply both the sides by 35 so that j5/35 = 35 becomes j5 = 625. Then take the fifth root of each side to get j alone 5√625 = 3.62, therefore the right option is (d). 

%% page 9 81

Let us discuss few important questions on Multiple Choice with answers and explanation.

    In the above diagram, the rectangle j contains all points that is x, y. What is the area of a rectangle that contains all points (2x, y - 1)?

        52
        87
        65
        36
        12 

    Answer (d)

    Explanation: If you look at the diagram carefully the x coordinates of the points in region j contain everything from x = 0 to x = 6. A rectangular region containing all points (2x, y - 1) would therefore stretch from X = 0 TO X = 12. Therefore the diagram will look like this.

    It would have length around 12 and breadth of 3 and an area of 36. Therefore the correct answer is (d)

    If sin Θ = 1/3 and - Π/4 ≤ Θ ≤ Π/4, then cos (2 Θ) =

          9/7
          15/7
          -23/7
          7/9
          5 

    Answer (d)

    Explanation: An angle of Π/4 radians is equivalent to an angle of 45°. Now you should look for an angle between 45° and -45° and sin = 1/3. The simplest way to find this number is to place the quantity 1/3 and you are required to reverse the sine.

    If u - t > u + t, then which of the following must be true?

        5/u
        t < 8
        u > 9
        t > 0
        4 / t 

    Answer (b)

    Explanation: Algebra manipulation is the easiest way to solve this problem. You need to add t to each side which will give inequality u > u + 2t. Now you are required to subtract from each side to get 0 > 2t and you will find your answer is (b)

    Points J, K L and M are arranged in a line in the similar order. If JL = 13, KM = 14 and JM = 21 then KL =

        45
        7
        8
        23
        32 

    Answer (c)

    Explanation: To answer this question algebraically separate b one step at a time. First you  need to multiply by 4 on the right, b - 4 = 4 - 4b. Next add 3b to each side and then add 4 to each side to get 4b = 6. Divide each side by 4 to get b = 8

    If r - o > r + o, then which of the following must be true?

        0/r
        q < 0
        9 > 0
        7 > 0
        o / 10 

    Answer (b)

    Explanation: Algebraic manipulation is the best way to answer this problem. You have to add o to each side which will give dissimilarity r > r+ 2o. Now you have to subtract from each side to get 0 > 2o and therefore the answer is (b)

     

    If for all the real numbers y, a function i(h) is defined by i(y) = {2, h ≠ 13 4, h = 13}

    Then g(16) - g(15) =

        31
        0
        91
        1
        43 

    Answer (b)

    Explanation: The equation i(y) = {2, h ≠ 13 4, h = 13} no values are given which is equal to 13, the functions will always come out to 2. i(16) - i(15) = 2 - 2 = 0, therefore the right answer is (b)

    If t(x) = 5m2+5m+5, which of the following is equal to t(-3.5)?

        F9
        F89
        F(2.5)
        F78
        F(5.7) 

    Answer (c)

    Explanation: You will need to check the value of the Table you can find that t( -3.5) and t(2.5) both are equal to 39. But if you do not possess graphing calculator, you can make the use of PITA.

    If 0 < b < 1 then all of the following must be true except:

        b3 < n
        b < square root of b
        b < b
        -q < q
        q < 6/q 

    Answer (c )

    Explanation: Apply 0.5 in place of b and try to solve the equation. Now b2 = 0.25, square root of b = 0.707, b = 0.5, - b = - 0.5, -b = -0.5, and 1/b = 2. This makes A B D and E true and (c) is false.

    If the cube root of the square root of a number is 8, what is the number?

        23
        512
        451
        784
        102 

    Answer (b)

    Explanation: Translate the whole into math's 3√x = 8. Now take away. First cube both sides. √x = 8. Now square both sides; x = 512

    If d - 5 = 5 (2- d), then what is the value of d?

        0.33
        0.87
        8.80
        1.87
        2.55 

    Answer (a)

    Explanation: The fraction m/n is a reciprocal of a/m. To find the numerical value of m/n, just change the numerical value of n/m, which comes to 0.625. Your calculation will tell you that 1/0.625 = 1.6. 


%% page 10 91

If g(y) = |y| + 10 for which of the following values of x does g(y) = g(-y)?

    25 y
    34y
    All real y
    60 except y
    52y 

Answer (c)

Explanation: To start with the equation the above given statement g(y) = g(-y) when y = -10 and 10. A simple number like zero works best. g(0) = Ι0Ι + 10 = 10. You can see that g(0) = f(-0) so zero must be the part of correct answer, therefore (c) is the correct option.

If b/c = 0.455, then b/a is equal to which of the following?

    9.67
    2.67
    5.78
    0.45
    6.45 

Answer (b)

Explanation: Make the base on both sides of the equation the same. 233 is similar as 35. So 45x  = 45, which means that x = 1. Therefore the correct option would be (b)

If y - 4 = 4 (1 - y), then what is the value of y?

    0.33
    0.85
    1.40
    5.45
    9.41 

Answer (a)

Explanation: The fraction c/a is a reciprocal of a/c. To find the numerical value of c/a, just flip the numerical value of a/c, which is 0.625. Your calculation will tell you that 1/0.625 = 0.33.

The distance between the points (-8, 6) and (-8, -14) is

    √21
    78
    52
    17
    √14 

Answer (d)

Explanation: Notice that the x coordinate in both points is the same. So you just have to find the difference in the y coordinates. The difference is 6 - (-14) = 17

If 7c + 2d = 11 and c - 2d = 5, then what is the value of c?

    - 5.0
    - 8.5
    - 0.8
    5.12
    2.0 

Answer (e)

Explanation: When you have two equations which is very similar in form you are probably looking at classic ETS style simultaneous equations. Adding the two equations cancels out the d term leaving you with the equation 8c = 16, so c = 2

If m(y) = y2 - 5y, then m(-5) =

    0
    2.3
    9.0
    5.9
    50.0 

Answer (e)

Explanation: Plug - 5 into the function: (-5)2 - 5 (-5) = 50

What is the slope of the line given by the equation 4n - 5 = 7 - 2m?

    - 23
    - 2/3
    9/5
    7
    10 

Answer (b)

Explanation: To find the slope of the line without difficulty, get its equation into the form n = ma + b, where m will be the value of the slope. To express 4n - 5 = 7 - 2m in this form, just isolate n. You will find that n = -2/4 + 4. Here the slope of the line (m) is - 2/4

In a right triangle ABC Angle B measures 90°, angle C measures 27° and AB = 9. What is the length of the hypotenuse of triangle ABC?

    7.84
    9.56
    2.78
    19.8
    85.6 

Answer (d)

Explanation: The triangle described here will look like this

Here you can see h is the unknown length of the hypotenuse. You can see you're given the measure of an angle and the length of the opposite side of the right angle triangle. According to the formula sin = opposite/hypotenuse. You have to just place the given information in the formula and solve the missing number. Sin 27° = 9/h. h = 9/sin 27°. H = 9/sin 27°= 9/0.454 = 19.82. Therefore the option is (d).

Which of the following is 0 of

f(x) = x2 + 6x - 12?

    -8.45
    -7.58
    5.85
    3.78
    1.52 

Answer (b)

Explanation: Place each answer option for x in the statement f(x) = 0 . You can also graph y = x2 + 6x - 12 = - 7.58. Therefore the option is (b)

If sin x = m and 0 < x < 90 than tan x =

    m/52
    m/√1 - m2
    45/m
    5 - m
    m/12 

Answer (b)

Explanation: Take x = 30°. Therefore sin 30°= 0.5 = m tan 30° = 0.577. Now place 0.5 in the place of m and you will see the correct option is (b) 


%% page 11 101


    If (s - 3) = 4s - 7, then which of the following could be the value of s?

        32
        1231
        19
        2
        29 

    Answer (d)

    Explanation: The first equation is s - 3 = 4s - 7, which simplifies to s = 4/3 but this is not the answer. Let us look at the second equation -(s - 3) = 4s - 7. This simplifies to 3 - s = 4s - 7 that means 5s = 10 and s = 2.

    If d = 2/3 and e = 6 then e/d + 4/d2 =

        323
        76
        10
        12
        2346 

    Answer (e)

    Explanation: Insert the values given for d and e into the equation

    e/d + 4/d2 =

    6/2 + 4/(2/3)2 =6/2/3 + 4/4/9 =

    (6*3/2) + (4 * 9/4) =

    18/2 + 36/4 = 9 + 9 = 18 therefore the correct answer is (d).

    As an agent, Miranda receives a $10,00 commission for each unit she sells more than 400 units, she receives an additional bonus of $1,000,00. What was the total amount Miranda received in bonuses in 2011?

        $568
        $4,00,000
        $129
        $289
        $123 

    Answer (b)

    Explanation: Miranda sold more than 300 units in only 4 month in the year 2011; therefore, 4 months bonus will be quiet equal to the option (b).

    If (k - 3) = 4k - 7, then which of the following could be the value of k?

        32.56
        3256
        15
        2
        43 

    Answer (d)

    Explanation: The first equation is k - 3 = 4k - 7, which simplifies to k = 4/3 but this is not the answer. Let's look at the second equation - (k - 3) = 4k - 7. This simplifies to 3 - k = 4k - 7 that means 5k = 10 and k = 2.

    If m = cos θ and n = sin θ, then for all θ, m2 + n2 =

        54
        1
        342
        1000
        97 

    Answer (b)

    Explanation: Substitute for m and n. So m2 + n2 = cos2θ + sin 2θ. But you should the correct identity of cos2θ + sin 2θ = 1. Therefore the correct option is (b).

    If log 3 = 9 then m =

        6.8
        45.98
        1.09
        89.0
        1.90 

    Answer (c)

    Explanation: The equation log 3 = 9 in exponential form, m8= 3. Now you have to find the value of m, you are required to take the 9th root of both the sides. You will see that m = 1.09. The correct option is (c).

    If 0 < z < 1 then all of the following must be true except:

        z
        z2
        z < z
        37z
        54z 

    Answer (c)

    Explanation: First place the amount 0.5 in place of z. Now z3 = 0.25, cube root of z = 0.707, z = 0.5, - z = - 0.5, -z = -0.5, and 1/z = 2. This makes A B D and E true and (c) is false.

    Points I, J L and F are arranged in a line in the same order. If IL = 13, JF = 14 and IF = 21 then JL =

        34.21
        342
        8
        56
        97 

    Answer (c)

    Explanation: To answer the problem algebraically separate y one step at a time. First you are required to multiply by 4 on the right, y - 4 = 4 - 4y. Next add 3y to each side and then add 4 to each side to get 4y = 6. Divide each side by 4 to get y = 8.Therefore the correct answer is (c).

    If c = log, (jx) then zz =

        cj
        yx
        56
        jz
        34 

    Answer (b)

    Explanation: If (jx) = c is a component then c will turn into z and it will turn into jz you can compute the value of c = 3.0203. The only answer choice that equals to 1,034. Therefore (b) is the answer.

    If t - r > t + r, then which of the following must be true?

        t
        r < 0
        t > 9
        r > 0
        tr 

    Answer (b)

    Explanation: Algebraic manipulation is the simplest way to solve this problem. You need to add r to each side which will give inequality t > t+ 2r. Now you need to subtract from each side to get 0 > 2r and therefore the answer is (b). 


%% page 11 111


    Seven blue marbles and six red marbles are held in a single container. Marbles are randomly selected one at a time and not returned to the container. If the first two marbles selected are blue, what is the probability that at least two red marbles will be chosen in the next three selections?

        8/45
        7/13
        45/100
        19/33
        78/5 

    Answer (d)

    Explanation: As you know the probability of every outcome that produces at least two red marbles in three drawings. To find the total overall probability adds up all of the individual probabilities.

    5/33 + 5/33 + 5/33 + 4/33 = 19/33

    Therefore the correct option would be option (d).

    Y varies directly as the square of x. When y = 2.5, x = 0.5 If y = 80, then x would be equal to

        - 2 √2
        - 85
        -45
        - 12
        - 56 

    Answer (a)

    Explanation: If y varies directly as the square of the x, it means y / x2 will always have the same value. So set up a proportion 2.5/(0.5)2 = 80/x2. You require to cross multiply and solve x2 = 8. Therefore x = +2√2

    If v - h > v + h, then which of the following must be true?

        V - h
        h < 8
        h - 1
        45/8
        76/9 

    Answer (b)

    Explanation: Algebraic manipulation is the simplest way to answer this problem. You require to add to each side which will give inequality v > v + 2h. Now you are required to subtract from each side to get 0 > 2h and you will find your answer is (b).

    If t - q > t + q, then which of the following must be true?

        Q = t
        t < q
        56 > 0
        37 > 9
        rt 

    Answer (b)

    Explanation: You have to add 0 to each side which will give dissimilarity t > t+ 2q. Now you have to subtract from each side to get 0 > 2q and therefore the answer is (b)

    Which of the following has the greatest value?

        500500
        90100
        1.528
        25
        70 

    Answer (a)

    Explanation: The significant fact about this question is that it is not necessary to find the correct value of the expression. You require rearranging as many answer choices as possible so that they have the exponents of 100. Take a close glance on each of the expression 500500, 25, 70, 1.528, 90100. It will make clear that (a) is bigger than (c), therefore the answer is (a)

    What is the distance between the w intercept and the z intercept of the line given by the equation?

    2z = 6 - w?

        7.56
        3.78
        5.81
        2.79
        3.56 

    Answer (c)

    Explanation: To find the z - intercept just make w = 0 and solve for z. z= 3. Now to find the w intercept, make z = 0 and solve for w. You will see that they form a right angle triangle with length of 3 and 6, in which the hypotenuse represents the distance in between the two distinct points. You are required to used Pythagoras theorem to find the length of the hypotenuse, which will be equal to 5.8156 approximately; therefore the answer is (c).

    How many dissimilar three digit numbers include only non-zero numbers?

        432
        78
        325
        729
        867 

    Answer (d)

    Explanation: If there is a three digit no and it has the no 0. Then there are in fact nine possibilities for the first numbers from 1 to 9, 9 possibilities for the 2nd numbers from 1 to 9, 9th possibilities for the third number from 1 to 9. This comes to the total of 9*9*9 possible 3 digit numbers therefore the right answer is 729.

    After 5:00 am ride in a bus costs $1.60 plus $0.50 for every sixth of a mile traveled. If the traveler travels x miles then what is the price of the trip, in dollars, in terms of x?

        456x
        2.5 + 1.5x
        4678x
        45 + 232x
        321 + 45x 

    Answer (b)

    Explanation: Suppose that you are travelling 6 miles (x = 6). To find the right answer place 5 in place of x and see which one gives you value of 9 and you will find the correct answer would be option (b).

    If x = cos θ and y = sin θ, then for all θ, x2 + y2 =

        456
        1
        32
        21
        0 

    Answer (b)

    Explanation: Substitute for x and y. So x2 + y2 = cos2θ + sin 2θ. But you should the correct identity of cos2θ+ sin 2θ= 1. Therefore the option is (b).

    Find the slope of the line given by the equation

    b + 3 = 5/4 (a - 7)?

        45/2
        32/5
        33/5
        21/9
        5/4 

    Answer (e)

    Explanation: You may see that this is the point slope form and that the slope is 5/4. If you don't see that rewrite the equations into slope form. 

%% page 13 121


    All the following can be formed by the intersection of a cube and a plane EXCEPT

        A cube
        A rectangle
        A square
        A line
        A circle 

    Answer (e)

    Explanation: This is a visual perception question and there is no easy technique to follow to solve this problem. You need to draw the sketches of the diagram.

    The intersection of a cube and a plane can be triangle

    A point

    A rectangle

    Or a line segment

    But if you see the above diagrams carefully you can see that there is no way to produce a circle, as none of the cube's faces is curved. Therefore the correct option would be (e).

    The polar equation r sin θ = 1 defines the graph of

        A line
        A Rectangle
        A point
        A Hyperbole
        An ellipse 

    Answer (a)

    Explanation: This equation is very simple than it appears to be. If you observe the equation carefully x = r cos θ and y = r sin θ. This clearly states that equation r sin θ = 1 which simply shows y = 1 in which rectangular coordinates and it as a horizontal line. Therefore the correct option would be (a).

    If the 20th term of an arithmetic sequence is 20 and the 50th term is 100. What is the first term of the sequence?

        - 23.455
        -30.67
        47.00
        4.45
        9.15 

    Answer (b)

    The arithmetic sequence is one that one goes up by adding a constant quantity again. As the 20th term in the sequence of 20th and the 50th term is 100. That will make each step worth 80/30. You can solve the value of a1

    20 = a1 + 19 * 2.666

    20 = a1 + 50.666

    a1 = - 30.666

    Therefore the correct option would be (b).

    m varies directly as the square of n. When m = 2.5, n = 0.5 If m = 80, then x would be equal to

        - 2 √2
        -9.56
        210
        89/5
        78/2 

    Answer (a)

    Explanation: If m varies directly as the square of the n, it means m / n3 will always have the same value. So set up a proportion 2.5/(0.5)3 = 80/n3. You require to cross multiply and solve n3 = 8. Therefore n = +2√2

    What is the distance between the u intercept and the h intercept of the line given by the equation?

    2h = 6 - u?

        89/5
        45/2
        5.81
        2/5
        3.56 

    Answer (c)

    Explanation: To find the h - intercept just make u = 0 and solve for h. h= 3. Now to find the w intercept, make h = 0 and solve for u. You will see that they form a right angle triangle with length of 3 and 6, in which the hypotenuse represents the distance in between the two distinct points. You are required to use here the Pythagoras theorem to find the length, which will be equal to 5.8156 approximately; therefore the answer is (c).

    If s(m) = |m| + 10 for which of the following values of x does s(m) = s(-m)?

        8/9m
        45s
        All real m
        89.8m
        8/6s 

    Answer (c)

     

    Explanation: To start with the equation the above given statement s(m) = s(-m) when m = -10 and 10. A simple number like zero works best. s(0) = Ι0Ι + 10 = 10. You can see that s(0) = m(-0) so zero must be the part of correct answer, therefore the correct option is (c).

     

    The distance between the points (-5, 7) and (-5, -15) is

        25.5
        83
        8/6
        17
        4.5 

    Answer (d)

    Explanation: Notice that the t coordinate in both points is the same. So you just have to find the difference in the j coordinates. The difference is 7 - (-15) = 17

    If 7r + 2l = 11 and r - 2l = 5, then what is the value of r?

        45
        8/3
        45.6
        78.23
        2.0 

    Answer (e)

    Explanation: When you have two equations, then you would be most likely looking at classic ETS style simultaneous equations. Adding the two equations cancels out the l term leaving you with the equation 8r = 16, so r = 2

    What is the slope of the line given by the equation 4d - 5 = 7 - 2k?

        1/5
        - 2/3
        12
        7\87
        45 

    Answer (b)

    Explanation: To find the slope of the line without difficulty, get its equation into the form d = ka + b, where m will be the value of the slope. To express 4d - 5 = 7 - 2m in this form, just isolate d. You will find that d = -2/4 + 4. Here the slope of the line (k) is - 2/4

    Which of the following is 0 of

    v(h) = h2 + 6h - 12?

        45/12
        -7.58
        45/5
        78
        98 

    Answer (b)

    Explanation: Place each answer option for h in the statement v(h) = 0 . You can also graph y = h2 + 6h - 12 = - 7.58. Therefore the option is (b) 


%% page 14 131


    If z(j) =2j2+ 2 then what is the value of z(j + 5)?

        78p2+8
        71p2 + 8
        10p2+ m +5
        14p2+ 5 +87
        2j2+ 16j +34 

    Answer (e)

    Explanation: Since there are variables in the answer choices you should insert in. Try j = 3. We are trying to find z(3 + 5) = z(8) = 2(8)2 + 2 which is 100 our target number. Now insert 3 in for j in the answer choices to see which answer choice goes with the target its (e)

    How many different three digit number contain only non-zero digit?

        7841
        701
        521
        729
        124 

    Answer (d)

    Explanation: In a three digit numeral containing no 0. There are nine possibilities for the first number from 1 to 9, 9 possibilities for the 2nd numeral 1 to 9, 9th possibilities for the third number 1 to 9. This makes the total of 9*9*9 possible 3 numeral numbers, therefore the correct option would be (d)

    If e(r) = 3√x and p(r) = ½ √r + 1, then e(p(5.3)) =

        12.78
        1.2
        2.1
        5.6
        9.14 

    Answer (b)

    This is a compound question, in which you are required to apply two functions in combination. You require to place the numbers 5 and 3 in place of r is the definition of p(r) p(5,3) = ½ √5.3 is equal to e(1.76), which can easily be solved by e(1.76) = 2√1.76 = 1.21. The correct option is (b).

    If u mod z is the remainder when u is divided by z, then (81 mod 7) - (9 mod 9) =

        78
        9
        12
        5
        78/2 

    Answer (d)

    Explanation: To find the value of z and u just take the number as the u position and divide it by the number in z position. The remainder is the value of u and z of those numbers. The value of 81 and 7 is 9. The expression (81 mod 7) - (9 mod 9) can be rewritten as 9 - 0 which equals 9. Therefore the correct answer is (d).

    If the ratio of sec p to cos p is 1:4, then the ratio of tan p to cot p is

        1:16
        7:78
        56.7
        1:76
        5:25 

    Answer (a)

    Explanation: The ratio which is given can be written in fractional form like sec p/cosec p = ¼. The secant and cosecant can also be expressed in terms of sine and cosine. 1/cos p/1/sin p = ¼. The cotangent is the reciprocal of the tangent so cot p = 4. Therefore the correct answer would be (a)

    If e/f = 0.595, then f/e is equal to which of the following?

        78/1
        2.67
        418
        32.5
        9/6 

    Answer (b)

    Explanation: Make the base on together sides of the equation the same. 243 is alike as 45. So 45x  = 45, which means that p = 1

    If l - 7 = 7 (1 - l), then what is the value of l?

        0.33
        4.78
        87/2
        3.78
        0.45 

    Answer (a)

    Explanation: The fraction w/v is an equal to v/w. To find the numerical value of w/a just turn the numerical value of v/w; this comes to 0.735. Your calculation will tell you that 1/0.735 = 0.33.

    The distance between the points (-8, 9) and (-8, -14) is

        82
        7.45
        8/5
        23
        41 

    Answer (d)

    Explanation: Notice that the b coordinate in both points is the same. So you just have to find the difference in the d coordinates. The difference is 9 - (-14) = 23

    If p - 4 = 4 (1 - p), then what is the value of p?

        0.33
        0.85
        1.40
        5.45
        9.41 

    Answer (a)

    Explanation: The fraction t/s is a reciprocal of a/t. To find the numerical value of t/s, just reverse the numerical value of s/t, which is 0.625. Your calculation will tell you that 1/0.625 = 0.33.

    h varies directly as the square of k. When h = 3.2, k = 0.7 If h = 80, then k would be equal to

        - 2 √2
        56/5
        78.2
        4.5
        21 

    Answer (a)

    Explanation: If h varies directly as the square of the k, it means h / k2 will always have the same value. So set up a proportion 3.2/(0.7)2 = 80/k2. You require to cross multiply and solve k2 = 8. Therefore k = +2√2 

%% page 15 141


    Seven pink marbles and six yellow marbles are held in a single container. Marbles are randomly selected one at a time and not returned to the container. If the first two marbles selected are pink. What is the probability that at least two marbles will be chosen in the next three selections?

        712
        78/56
        87.78
        19/33
        8/6 

    Answer (d)

    Explanation: You are given a container that holds five pink marbles and six yellow marbles. Here is the probability 6/11 * 5/10 * 5/9 * = 150/990 = 5/33

    There are six yellow marbles out of the total of eleven on the first drawing and 5 red marbles out of ten and five pink marbles out of nine. Therefore the total probability of drawing

    6/11 * 5/10 * 5/9 = 150/990 = 5/33

    Therefore the correct answer would be (d)

    If (s - 3) = 4s - 7, then which of the following could be the value of s?

        12
        843
        78
        2
        29 

    Answer (d)

    Explanation: The first equation is s - 3 = 4s - 7, which simplifies to s = 4/3 but this is not the answer. Let us look at the second equation -(s - 3) = 4s - 7. This simplifies to 3 - s = 4s - 7 that means 5s = 10 and s = 2.

    If (k - 3) = 4k - 7, then which of the following could be the value of k?

        324
        124
        13
        2
        43 

    Answer (d)

    Explanation: The first equation is k - 3 = 4k - 7, which simplifies to k = 4/3 but this is not the answer. Let's look at the second equation - (k - 3) = 4k - 7. This simplifies to 3 - k = 4k - 7 that means 5k = 10 and k = 2.

    If d = 2/3 and e = 6 then e/d + 4/d2 =

        323
        76
        10
        12
        2346 

    Answer (e)

    Explanation: Insert the values given for d and e into the equation

    e/d + 4/d2 =

    6/2 + 4/(2/3)2 =6/2/3 + 4/4/9 =

    (6*3/2) + (4 * 9/4) =

    18/2 + 36/4 = 9 + 9 = 18 therefore the correct answer is (d).

    If sin θ = 1/2 and - Π/4 ≤ θ ≤ Π/4, then cos (3θ) =

          6/3
          89
          25.78
          7/9
          87/12 

    Answer (d)

    Explanation: An angle of Π/4 radians is equivalent to an angle of 35°. Now you should look for an angle between 35° and -35° and sin = 1/2. The simplest way to find this number is to place the quantity 1/2 and you are required to reverse the sine.

    If m - h > m + h, then which of the following must be true?

        89/m
        h < 8
        h/m
        mh
        12/3 

    Answer (b)

    Explanation: Algebra manipulation is the easiest way to solve this problem. You need to add h to each side which will give inequality m > m + 2h. Now you are required to subtract from each side to get 0 > 2h and you will find your answer is (b)

    Points T, N, Y and M are arranged in a line in the similar order. If TY = 18, NM = 15 and TM = 31 then NY =

        7/8
        21.45
        8
        92
        6.45 

    Answer (c)

    Explanation: To answer this question algebraically separate g one step at a time. First you need to multiply by 4 on the right, g - 5 = 5 - 4g. Next add 3g to each side and then add 5 to each side to get 5g = 6. Divide each side by 5 to get g = 8

    If for all the real numbers d, a function p(b) is defined by p(d) = {2, b ≠ 15 4, b = 15}

    Then s(16) - s(15) =

        23
        0
        74
        5/6
        7/9 

    Answer (b)

    Explanation: The equation p(d) = {2, b ≠ 15 4, b = 15} no values are given which is equal to 15, the functions will always come out to 2. p(16) - p(15) = 2 - 2 = 0, therefore the right answer is (b)

    If p < t < 1 then all of the following must be true except:

        p3 < t
        t < 8
        t < t
        -56
        6/5t 

    Answer (c)

    Explanation: Apply 0.5 in place of t and try to solve the equation. Now t2 = 0.25, square root of t = 0.707, t = 0.6, - t = - 0.6, -t = -0.6, and 1/t = 2. This makes A B D and E true and (c) is false.

    If j - 7 = 7 (2- j), then what is the value of j?

        0.33
        87.12
        12.56
        3.78
        785 

    Answer (a)

    Explanation: The fraction m/n is a reciprocal of h/r. To find the numerical value of r/h, just change the numerical value of h/r, which comes to 0.625. Your calculation will tell you that 1/0.625 = 1.6. 

%% page 16 151


    If c = 5/3 and m = 6 then m/c + 4/c2 =

        1200
        87
        2
        56
        2346 

    Answer (e)

    Explanation: Insert the values given for c and m into the equation

    m/c + 4/c2 =

    7/2 + 4/(5/3)2 =7/5/3 + 4/4/9 =

    (7*3/2) + (4 * 9/4) =

    18/2 + 36/4 = 9 + 9 = 18 therefore the correct answer is (d).

    If (v - 5) = 4v - 8, then which of the following could be the value of v?

        23/41
        12v
        45v
        3
        1.754 

    Answer (d)

    Explanation: The first equation is v - 5 = 4v - 8, which simplifies to v = 4/5 but this is not the answer. Let us look at the second equation - (v - 5) = 4v - 8. This simplifies to 5 - v = 4v - 8 that means 5v = 10 and s = 3.

    If (e - 6) = 4e - 7, then which of the following could be the value of e?

        32.56
        3256
        15
        2
        43 

    Answer (d)

    Explanation: The first equation is e - 6 = 4e - 7, which simplifies to e = 4/3 but this is not the answer. Let's look at the second equation - (e - 6) = 4e - 7. This simplifies to 6 - e = 4e - 7 that means 5e = 10 and e = 2.

    If p = cos θ and a = sin θ, then for all θ, p3 + a3 =

        1.456
        1
        12/23
        20/10
        36 

    Answer (b)

    Explanation: Substitute for m and n. So p3 + a3 = cos3θ + sin 3θ. But you should the correct identity of cos3θ + sin 3θ = 1. Therefore the correct option is (b).

    If n = log, (rx) then kz =

        23k
        yx
        87/2
        45r
        146 

    Answer (b)

    Explanation: If (rx) = n is a component then n will turn into k and it will turn into rz you can compute the value of n = 3.0203. The only answer choice that equals to 1,034. Therefore (b) is the answer.

    z varies directly as the square of c. When z = 3.5, c = 0.5 If z = 80, then c would be equal to

        - 2 √2
        15/74
        -12.56
        148
        23 

    Answer (a)

    Explanation: If z varies directly as the square of the c, it means z /c2 will always have the same value. So set up a proportion 3.5/(0.5)2 = 80/c2. You require to cross multiply and solve c2 = 8. Therefore c = +2√2

    If d - m > d + m, then which of the following must be true?

        d - m
        m < 8
        d3
        15.24
        83.41 

    Answer (b)

    Explanation: Algebraic manipulation is the simplest way to answer this problem. You require to add to each side which will give inequality d > d + 2m. Now you are required to subtract from each side to get 0 > 2m and you will find your answer is (b).

    Which of the following has the maximum value?

        800500
        45100
        2.728
        14
        18 

    Answer (a)

    Explanation: The major fact about this question is that it is not essential to find the right value of the expression. You need to rearrange as many answer options as possible so that they have the examples of 100. Take a close look on every expression 800500, 18, 14, 2.728, 45100. It will make clear that (a) is bigger than (c), therefore the answer is (a)

    How many different three digit numbers comprise only non-zero numbers?

        432
        78
        325
        739
        74/2 

    Answer (d)

    Explanation: If there is a three digit number and it has  0, then there are in fact nine possibilities for the first numbers from 1 to 9, 9 possibilities for the 2nd numbers from 1 to 9, 9th possibilities for the third number from 1 to 9. This comes to the total of 9*9*9 possible 3 digit numbers therefore the right answer is 739.

    If z = cos θ and q = sin θ, then for all θ, z2 + q2 =

        12
        1
        78/12
        36
        18 

    Answer (b)

    Explanation: Substitute for x and y. So z2 + q2 = cos2θ + sin 2θ. But you should the correct identity of cos2θ+ sin 2θ= 1. Therefore the option is (b). 


%% page 17 161


    If z(g) = |g| + 20 for which of the following values of g does z(g) = z(-g)?

        15 f
        18x
        All real g
        10 except x
        10x 

    Answer (c)

    Explanation: To start with the equation the statement z(g) = z(-g) when z = -20 and 20. A simple number like zero works best. z(0) = Ι0Ι + 20 = 20. You can see that z(0) = z(-0) so zero must be the part of correct answer, therefore (c) is the correct option.

    If q/c = 0.595, then c/q is equal to which of the following?

        34/7
        2.67
        12.01
        7/5
        32.87 

    Answer (b)

    Explanation: Make the base on both sides of the equation the same. 243 is similar as 45. So 45x = 45,which means that x = 1

    If u - 5 = 5 (1 - u), then what is the value of u?

        0.33
        2.78
        9.45
        8.45
        6.45 

    Answer (a)

    Explanation: The fraction s/a is a equal to n/s. To find the numerical value of s/n just turn the numerical value of n/s; this comes to 0.724. Your calculation will tell you that 1/0.724 = 1.6.

    The distance between the points (-7, 5) and (-7, -18) is

        32/6
        23.98
        7.8
        25
        12.98 

    Answer (d)

    Explanation: Notice that the x coordinate in both points is the same. So you just have to find the difference in the y coordinates. The difference is 7 - (-18) = 25

    If t/n = 0.455, then t/n is equal to which of the following?

        9.67
        2.67
        5.78
        0.45
        6.45 

    Answer (b)

    Explanation: Make the base on both sides of the equation the same. 673 is similar as 85. So 85x = 85,which means that x = 1. Therefore the correct option would be (b)

    If m - 4 = 4 (1 - m), then what is the value of m?

        0.33
        56
        43.09
        43.98
        43/89 

    Answer (a)

    Explanation: The fraction j/p is a reciprocal of p/c. To find the numerical value of j/p, just flip the numerical value of p/j, which is 0.625. Your calculation will tell you that 1/0.625 = 0.33.

    If 9c + 2d = 11 and z - 3d = 5, then what is the value of z?

        45/7
        32.45
        45/3
        9.45
        2.0 

    Answer (e)

    Explanation: When you have two equations which is very similar in form you are probably looking at classic ETS style simultaneous equations. Adding the two equations cancels out the d term leaving you with the equation 8c = 16, so z = 2

    What is the slope of the line given by the equation 3w - 8 = 7 - 2z?

        - 43
        - 2/3
        3/5
        8
        23/4 

    Answer (b)

    Explanation: To find the slope of the line without difficulty, get its equation into the form w = zc + d, where m will be the value of the slope. To express 3w - 8 = 7 - 2m in this form, just isolate n. You will find that w = -2/4 + 4. Here the slope of the line (z) is - 2/4

    If v(o) =2o2+ 2 then what is the value of v(o + 4)?

        7v2+8
        2h2 + 6
        4y2+ b +43
        6g2+ 12 +56
        2x2+ 16x +34 

    Answer (e)

    Explanation: Assume that o = 3. We are trying to find v(3 + 4) = v(7) = 2(7)2 + 2 which is 100 our target number. Now insert 3 in for o in the answer choices to see which answer choice goes with the target its (e)

    If w = log, (wx) then pz =

        25p
        fx
        45w
        13y
        w 

    Answer (b)

    Explanation: If (fx) = w is a component then w that turn p into yz. You can compute the value of w = 3.0103. The only answer choice that equals to 1,024. Therefore (b) is the answer. 

%% page 18 171


    If h(p) = |p| + 20 for which of the following values of p does h(p) = h(-p)?

        hp
        2p
        All real g
        89p
        87h 

    Answer (c)

    Explanation: To start with the equation the statement h(p) = h(-p) when h = -20 and 20. A simple number like zero works best. h(0) = Ι0Ι + 20 = 20. You can see that h(0) = h(-0) so zero must be the part of correct answer, therefore (c) is the correct option.

    What is the distance between the e intercept and the c intercept of the line given by the equation?

    2c = 6 - w?

        6.78
        56.13
        5.81
        85
        15.23 

    Answer (c)

    Explanation: To find the c - intercept just make e = 0 and solve for c. c= 3. Now to find the e intercept, make c = 0 and solve for e. You will see that they form a right angle triangle with length of 3 and 6, in which the hypotenuse represents the distance in between the two distinct points. You are required to use the Pythagoras theorem to find the length of the hypotenuse, which will be equal to 5.8178 approximately; therefore the answer is (c).

    If b(k) =2k2+ 2 then what is the value of b(k + 5)?

        8k
        89/41
        8/5p
        89k
        2j2+ 16j +34 

    Answer (e)

    Explanation: Since there are variables in the answer choices you should insert in. Try k = 3. We are trying to find b(3 + 5) = b(8) = 2(8)2 + 2 which is 100 our target number. Now insert 3 in for k in the answer choices to see which answer choice goes with the target its (e)

    If f(o) = 3√y and u(o) = ½ √o + 1, then f(u(4.5)) =

        12.78
        1.2
        2.1
        5.6
        9.14 

    Answer (b)

    This is a compound question, in which you are required to apply two functions in combination. You require to place the numbers 4 and 5 in place of r is the definition of u(o) u(4,5) = ½ √4.5 is equal to f(1.76), which can easily be solved by f(1.76) = 2√1.76 = 1.21. The correct option is (b).

    If the ratio of sec p to cos p is 1:8, then the ratio of tan p to cot p is

        1:16
        56.12
        89.12
        23.45
        89/2 

    Answer (a)

    Explanation: The ratio which is given can be written in fractional form like sec p/cosec p = 1/8. The secant and cosecant can also be expressed in terms of sine and cosine. 1/cos p/1/sin p = 1/8. The cotangent is the reciprocal of the tangent so cot p = 4. Therefore the correct answer would be (a)

    If i - 7 = 7 (i - i), then what is the value of i?

        0.33
        78/5
        157
        314
        2.45 

    Answer (a)

    Explanation: The fraction u/h is an equal to h/u. To find the numerical value of u/h just turn the numerical value of h/u; this comes to 0.735. Your calculation will tell you that 1/0.735 = 0.33.

    If p varies directly as n and p/d = 10, then what is the value of p when d = 2.2?

        16
        0.23
        7/6
        45
        11.00 

    Answer (e)

    Explanation: Direct variation between two quantities means that they always have the same quotient. In this case, it means that p/d must always be equal to 10. To find the value of p when d = 2.2, set up the equation p/2.2 = 10 and solve for p. You will find that p = 11

    (r3)6 * (r4)5/r2 =

        R6
        r 18
        r 48
        r 36
        r 65 

    Answer (d)

    Explanation: A quick evaluation of exponent rules. When increasing the powers to powers; you need to multiply exponents. When multiplying the powers of the similar base include exponents and when dividing powers of the similar base, subtract exponents. For this problem you have to do all three to get correct answer.

    If the perimeter of a rectangle is 80, what is the area of the rectangle?

        13√3
        25√2
        56
        185
        324 

    Answer (e)

    Explanation: The opposite sides of the rectangle are the same. So the opposite side must be 18. Since the area of a rectangle is side2, therefore it is (18)2 = 324.

    Where defined, {j2 - 3/3} {8/2j + 3} =

        7m
        56
        j - 2
        m+ 9
        4m2 /8 

    Answer (c)

    Explanation: You can just factor this one and then cancel

    {j2 - 3/3} {8/2j + 3} = {(j + 2) (j - 2)/3} {2 * 3/2(j + 2)} = j - 2 

%% page 19 181


    If b(x) = 5x2+5x+5, which of the following is equal to b(-4.5)?

        23b
        89/4
        b(2.5)
        86b3
        5.78k 

    Answer (c)

    Explanation: You require a graphing calculator, you need to press the Y key and enter the function. You will check the value of the Table you can find that b(-4.5) and b(4.5) both are equal to 39. But if you do not possess graphing calculator, you can make the use of PITA.

    What is the distance between the z intercept and the s intercept of the line given by the equation

    2s = 6 - z?

        78/96
        56.124
        6.71
        89.014
        74.137 

    Answer (c)

    Explanation: To find the s - intercept just make z = 0 and solve for s. s= 3. Now to find the z intercept, make s = 0 and solve for z. You will see that they form a right angle triangle with length of 3 and 6, in which the hypotenuse represents the distance in between the two distinct points. You are required to used Pythagoras theorem to find the length of the hypotenuse, which will be equal to 6.7156 approximately; therefore, the answer is (c).

    b varies directly as the square of d. When b = 3.5, d = 0.5 If b = 80, then d would be equal to

        - 2 √2
        45.78
        -89
         32/5
        23/41 

    Answer (a)

    Explanation: If b varies directly as the square of the d, it means b / d2 will always have the same value. So set up a proportion 3.5/(0.5)2 = 80/d2. You require to cross multiply and solve d2 = 8. Therefore d = +2√2

    If w - t > w + t, then which of the following must be true?

        t < 8
        h - 1
        45/8
        76/9
        86 

    Answer (a)

    Explanation: Algebraic manipulation is the simplest way to answer this problem. You require to add to each side which will give inequality w > w + 2t. Now you are required to subtract from each side to get 0 > 2t and you will find your answer is (b).

    Find the slope of the line given by the equation

    o + 3 = 4/7 (p - 7)?

        46/8
        12/7
        8/9
        45/12
        5/4 

    Answer (e)

    Explanation: You may see that this is the point slope form and that the slope is 4/7. If you don't see that rewrite the equations into slope form.

    If g(b) =2b2+ 2 then what is the value of g(b + 8)?

        51/2g
        86g + 52
        78g - 56
        15b + 85
        2j2+ 16j +34 

    Answer (e)

    Explanation: Since there are variables in the answer choices you should insert in. Try b = 3. We are trying to find g(3 + 8) = g(8) = 2(8)2 + 2 which is 100 our target number. Now insert 3 in for j in the answer choices to see which answer choice goes with the target its (e)

    If r mod a is the remainder when r is divided by a, then (91 mod 7) - (9 mod 9) =

        9/6
        6
        78/2
        5
        7/5 

    Answer (d)

    Explanation: To find the value of a and r just take the number as the r position and divide it by the number in a position. The remainder is the value of r and a of those numbers. The value of 81 and 7 is 9. The expression (91 mod 7) - (9 mod 9) can be rewritten as 9 - 0 which equals 9. Therefore, the correct answer is (d).

    If the ratio of sec z to cos z is 1:8, then the ratio of tan z to cot z is

        1:16
        1:23
        7:15
        8:13
        9:23 

    Answer (a)

    Explanation: The ratio which is given can be written in fractional form like sec p/cosec z = 1/8. The secant and cosecant can also be expressed in terms of sine and cosine. 1/cos z/1/sin z = 1/8. The cotangent is the reciprocal of the tangent so cot z = 8. Therefore, the correct answer would be (a)

    The distance between the points (-5, 10) and (-5, -14) is

        16
        77
        89
        24
        85 

    Answer (d)

    Explanation: Notice that the b coordinate in both points is the same. So you just have to find the difference in the d coordinates. The difference is 10 - (-14) = 24

    If j = 5/2 and q = 6 then q/j + 4/j2 =

        124
        95
        9
        74.12
        2346 

    Answer (e)

    Explanation: Insert the values given for j and q into the equation

    q/j + 4/j2 =

    6/2 + 4/(5/2)2 =6/5/2 + 4/4/9 =

    (6*2/5) + (4 * 9/4) =

    18/2 + 36/4 = 9 + 9 = 18 therefore the correct answer is (d). 

%% page 20 191

Below we have given a few questions and answers on Multiple Choice. These questions will help you solve questions on multiple choice

    If (m - 2) = 4m - 7, then which of the following could be the value of m?

        65
        98
        57.98
        2
        78 

    Answer (d)

    Explanation: The first equation is m - 2 = 4m - 7, which simplifies to m = 4/2 but this is not the answer. Let's look at the second equation -(m - 2) = 4m - 7. This simplifies to 2 - m = 4m - 7 that means 5m = 10 and m = 2. Therefore the answer is (d)

    If b = 2/3 and j = 6 then j/b + 4/b2 =

        986
        57
        75
        89
        2346 

    Answer (e)

    Explanation: You have to just insert the values. Insert the values given for b and j into the equation

    j/b + 4/b2 =

    6/2 + 4/(2/3)2 =6/2/3 + 4/4/8 =

    (6*3/2) + (4 * 8/4) =

    18/2 + 36/4 = 9 + 9 = 18 therefore the correct answer is (d)

    As an agent, Robbie receives a $10,00 commission for each unit she sells more than 500 units, she receives an additional bonus of $ 2,000,00. What was the total amount Robbie received in bonuses in 2011?

        $675
        $4,00,000
        $897
        $990
        $764 

    Answer (b)

    Explanation: Robbie sold more than 500 units in only 4 month in the year 2011; therefore 4 months bonus will be quiet equal to the option (b)

    If (z - 3) = 4z - 7, then which of the following could be the value of z?

        876
        745
        89
        2
        78 

    Answer (d)

    Explanation: The first equation is z - 3 = 4z - 7, which simplifies to z = 4/3 but this is not the answer. Let's look at the second equation - (z - 3) = 4z - 7. This simplifies to 3 - z = 4z - 7 that means 5z = 10 and z = 2. Therefore the correct answer is option (d)

    If b = cos θ and c = sin θ, then for all θ, b2 + c2 =

        81
        1
        89
        54
        60 

    Answer (b)

    Explanation: Substitute for m and c. So b2 + c2 = cos2 θ + sin 2 θ. But you should the correct identity of cos2 θ + sin 2 θ = 1. Therefore the correct option is (b)

    If log 3 = 9 then g =

        8.90
        98
        1.09
        89.096
        3.098 

    Answer (c)

    Explanation: The equation log 3 = 9 in exponential form, g8= 3. Now you have to find the value of g, you are required to take the 9th root of both the sides. You will see that g = 1.09. The correct option is (c).

    If 0 < b < 1 then all of the following must be true except:

        z
        z2
        b < b
        37z
        54z 

    Answer (c)

    Explanation: First place the amount 0.5 in place of b. Now b3 = 0.25, cube root of b = 0.707, b = 0.5, - b = - 0.5, -b = -0.5, and 1/b = 2. This makes (a), (b), (d) and (e) true and (c) false.

    Points P, O A and V are arranged in a line in the same order. If PA = 13, OF = 14 and PV = 21 then OA =

        987
        89
        8
        53
        25 

    Answer (c)

    Explanation: To answer the problem algebraically separate y one step at a time. First you are required to multiply by 4 on the right, y - 4 = 4 - 4y. Next add 3y to each side and then add 4 to each side to get 4y = 6. Divide each side by 4 to get y = 8.Therefore the correct answer is (c)

    If u = log, (mx) then wz =

        2m
        wx
        78
        3w
        67 

    Answer (b)

    Explanation: If (mx) = a is a component then a that turn w into mz. You can compute the value of u = 3.0203. The only answer choice that equals to 1,034. Therefore (b) is the answer.

    If w - a > w + a, then which of the following must be true?

        w
        a < 0
        w> 9
        a > 0
        78 

    Answer (b)

    Explanation: Algebraic manipulation is the simplest way to solve this problem. You need to add a to each side which will give inequality w > w+ 2a. Now you need to subtract from each side to get 0 > 2a and therefore the answer is (b). 

%% page 21 201


    If d(q) = |q| + 30 for which of the following values of q does d(q) = d(-q)?

        456q
        78d
        All real q
        67/98
        45.78 

    Answer (c)

    Explanation: To start with the equation the statement d(q) = d(-q) when d = -30 and 30. A simple number like zero works best. d(0) = Ι0Ι + 30 = 30. You can see that d(0) = d(-0) so zero must be the part of correct answer, therefore (c) is the correct option.

    If v/h = 0.595, then h/v is equal to which of the following?

        6.987
        2.67
        67.90
        78.987
        65 

    Answer (b)

    Explanation: Make the base on both sides of the equation  same. 243 is similar as 55. So 55x  = 55, which means that t = 1

    If w - 5 = 5 (1 - w), then what is the value of w?

        0.33
        2.78
        9.45
        67.98
        6.45 

    Answer (a)

    Explanation: The fraction r/m is a equal to m/r. To find the numerical value of m/r just turn the numerical value of m/r; this comes to 0.724. Your calculation will tell you that 1/0.724 = 1.6.

    The distance between the points (-9, 5) and (-9, -20) is

        9.01
        5/8
        89/65
        29
        67/9 

    Answer (d)

    Explanation: Notice that the a coordinate in both points is the same. So you just have to find the difference in the b coordinates. The difference is 9 - (-20) = 29

    If p/k = 0.455, then p/k is equal to which of the following?

        89/7
        2.67
        98.076
        45/9
        76.098 

    Answer (b)

    Explanation: Make the base on both sides of the equation the same. 673 is similar as 65. So 65x  = 65, which means that c = 1. Therefore the correct option would be (b)

    If d - 8 = 8 (1 - d), then what is the value of d?

        0.33
        65
        3/4
        12.678
        89.987 

    Answer (a)

    Explanation: The fraction m/o is a reciprocal of o/m. To find the numerical value of m/0, just flip the numerical value of o/m, which is 0.625. Your calculation will tell you that 1/0.625 = 0.33.

    If 9h + 2i = 11 and z - 3i = 5, then what is the value of f?

        67
        7d
        56
        89
        2.0 

    Answer (e)

    Explanation: When you have two equations which is very similar in form you are probably looking at classic ETS style simultaneous equations. Adding the two equations cancels out the i term leaving you with the equation 8h = 16, so f = 2

    What is the slope of the line given by the equation 3z - 8 = 7 - 2p?

        - 67
        - 2/3
        7.98
        98
        56.90 

    Answer (b)

    Explanation: To find the slope of the line without difficulty, get its equation into the form z = pc + d, where m will be the value of the slope. To express 3z - 8 = 7 - 2m in this form, just isolate n. You will find that z = -2/4 + 4. Here the slope of the line (p) is - 2/4

    If a(p) =2p2+ 2 then what is the value of a(p + 4)?

        87/43
        9/2
        9/78
        89p
        2x2+ 16x +34 

    Answer (e)

    Explanation: Since there are variables in the answer choices you should insert in  p = 3. We are trying to find a(3 + 4) = a(7) = 2(7)2 + 2 which is 100 our target number. Now insert 3 in  answer choices to see which answer choice goes with the target. You will find that the answer is (e)

    If d = log, (dx) then kz =

        78d
        gx
        65.87
        7.98d
        f 

    Answer (b)

    Explanation: If (gx) = d is a component then d that turn p into yz. You can compute the value of d = 3.0103. The only answer choice that equals to 1,024. Therefore (b) is the right answer. 

%% page 22 211


    If j/m = 0.625, then m/j is equal to which of the following?

        1.50
        2.67
        0.678
        9.76
        675 

    Answer (b)

    Explanation: Make the base on both sides of the equation same. 8 is similar as 24. So 24x  = 24,which means that x = 8

    If h - 5 = 5 (1 - h), then what is the value of h?

        0.33
        9.56
        5.987
        3.876
        1.34 

    Answer (a)

    Explanation: The fraction d/n is a reciprocal of n/d. To find the numerical value of d/n, just flip the numerical value of n/d, which is 0.625. Your calculation will tell you that 1/0.625 = 1.6.

    Points P, N Y and R are arranged in a line in the same order. If PY = 13, NR = 14 and PR = 21 then NY =

        43
        9
        8
        89
        4 

    Answer (c)

    Explanation: To solve the problem algebraically isolate u one step at a time. First multiply through by 4 on the right, u - 4 = 4 - 4u. Next add 3x to each side and then add 3 to each side to get 4u = 6. Divide each side by 4 to get u = 6/4 or 1.5

    The distance between the points (-5, 8) and (-5, -12) is

        90
        76
        15
        23
        78.50 

    Answer (d)

    Explanation: Notice that the m coordinate in both points is the same. So you just have to find the difference in the n coordinates. The difference is 5 - (-18) = 23

    If the cube root of the square root of a number is 4, what is the number?

        87
        64
        148
        10
        157 

    Answer (b)

    Explanation: Translate the whole into math's 3√x = 4. Now take away. First cube both sides. √x = 8. Now square both sides; x = 64

    If the ratio of sec p to cosec p is 1:4, then the ratio of tan p to cot p is

        1:16
        5:48
        9:78
        3:45
        4:45 

    Answer (a)

    Explanation: The ratio which is given can be written in fractional form like sec p/cosec p = 1/5. The secant and cosecant can also be expressed in terms of sine and cosine. 1/cos p/1/sin p = 1/5. The cotangent is the reciprocal of the tangent so cot p = 5. Therefore the correct answer would be (a)

    If i/j = 0.694, then j/i is equal to which of the following?

        8.45
        2.67
        3.45
        9.41
        4.26 

    Answer (b)

    Explanation: Make the base on both sides of the equation  same. 243 is similar as 25. So 25x  = 25,which means that g = 1

    If 0 < w < 1 then all of the following must be true except:

        w2 < n
           28w
        w < w
           24w
           89w 

    Answer (c)

    Explanation: Plug in 0.4 for w. Now w2 = 0.25, square root of w = 0.707, w = 0.4, - w = - 0.4, -w = -0.4, and 1/w = 2. This makes A B D and E true and (c) is false.

    If v(p) =2p2+ 2 then what is the value of v(p + 4)?

        2v
        98/76
        43
        56v
        2p2+ 16p +34 

    Answer (e)

    Explanation: Since there are variables in the answer choices you should insert in. Try p = 3. We are trying to find v(3 + 4) = v(7) = 2(7)2 + 2 which is 100 our target number. Now insert 3 in for p in the answer choices to see which answer choice goes with the target its (e)

    If (b - 3) = 4b - 7, then which of the following could be the value of b?

        15
        987
        18/87
        2
        89 

    Answer (d)

    Explanation: The first equation is b - 3 = 4b - 7, which simplifies to b = 4/3 but this is not the answer. Let us look at the second equation - (b - 3) = 4b - 7. This simplifies to 3 - b = 4b - 7 that means 5b = 10 and b = 2. 

%% page 23 221


    h varies directly as the square of i. When h = 2.5, i = 0.5 If h = 80, then i would be equal to

        - 2 √2
        98h
        48
        1/8
        78/4 

    Answer (a)

    Explanation: If h varies directly as the square of i, it means that h / i2 will always have the same value. So set up a proportion 2.5/(0.5)2 = 80/i2. You require to cross multiply and solve i2 = 8. Therefore i = +2√2

    If o - p > o + p, then which of the following must be true?

        -45
        p < 8
        78p
        78:45
        9 

    Answer (b)

    Explanation: Algebraic manipulation is the simplest way to answer this problem. You have to add to each side which will give inequality o > o + 2p. Now you are required to subtract from each side to get 0 > 2p and you will find your answer is (b).

    If d - f > d + f, then which of the following must be true?

        59t
        d < f
        8/5
        df
        87 

    Answer (b)

    Explanation: You have to add 0 to each side which will give dissimilarity d > d+ 2f. Now you have to subtract from each side to get 0 > 2f and therefore the answer is (b)

    Which of the following has the greatest value?

        400500
        8100
        0.458
        35
        7 

    Answer (a)

    Explanation: The significant fact about this question is that it is not necessary to find the correct value of the expression. You need to rearrange as many answer choices as possible so that they have the exponents of 100. Take a close glance on each of the expression 400500, 35, 7, 0.458, 8100.It will make clear that (a) is bigger than (c), therefore the answer is (a)

    What is the distance between the c intercept and the u intercept of the line given by the equation?

    2z = 6 - w?

        23
        33
        5.81
        0.45
        789 

    Answer (c)

    Explanation: To find the u - intercept just make c = 0 and solve for u. u= 3. Now to find the c intercept, make u = 0 and solve for c. You will see that they form a right angle triangle with length of 3 and 6, in which the hypotenuse represents the distance in between the two distinct points. You are required to used Pythagoras theorem to find the length of the hypotenuse, which will be equal to 5.8156 approximately; therefore the answer is (c).

    How many dissimilar three digit numbers include only non-zero numbers?

        124
        78
        148
        729
        54 

    Answer (d)

    Explanation: If there is a three digit number and it has the no. 0, then there are in fact nine possibilities for the first numbers from 1 to 9, 9 possibilities for the 2nd numbers from 1 to 9, 9th possibilities for the third number from 1 to 9. This comes to the total of 9*9*9 possible 3 digit numbers therefore the right answer is 729.

    After 8:00 am ride in a bike costs $1.50 plus $0.50 for every sixth of a mile traveled. If the traveler travels a miles then what is the price of the trip, in dollars, in terms of a?

        90
        2.5 + 1.5x
        45a
        847
        7/9 

    Answer (b)

    Explanation: Suppose that you are travelling 6 miles (a = 6). To find the right answer place 5 in place of a and see which one gives you value of 9 and you will find the correct answer would be option (b).

    If t = cos θ and p = sin θ, then for all θ, t2 + p2 =

        87
        1
        78
        27
        7 

    Answer (b)

    Explanation: Substitute for t and y. So t2 + p2 = cos2θ + sin 2θ. But you should the correct identity of cos2θ+ sin 2θ= 1. Therefore the option is (b).

    Find the slope of the line given by the equation

    h + 4 = 3/4 (g - 7)?

        23
        96
        87
        56
        5/4 

    Answer (e)

    Explanation: You may see that this is the point slope form and that the slope is 5/4. If you don't see that rewrite the equations into slope form. 

%% page 24 231


    p varies directly as the square of r. When p = 2.5, r = 0.5 If p = 80, then r would be equal to

        - 2 √2
        65
        124
        89/5
        12 

    Answer (a)

    Explanation: If p varies directly as the square of the r, it means p / r2 will always have the same value. So set up a proportion 2.5/(0.5)2 = 80/r2. You require to cross multiply and solve r2 = 8. Therefore r = +2√2

    If c - n > c + n, then which of the following must be true?

        23v
        n < 8
        45
        989
        124 

    Answer (b)

    Explanation: Algebraic manipulation is the simplest way to answer this problem. You have to add to each side which will give inequality c > c + 2n. Now you are required to subtract from each side to get 0 > 2n and you will find your answer is (b).

    If a - w > a + w, then which of the following must be true?

        89a
        a < w
        45.42
        78/5
        45 

    Answer (b)

    Explanation: You have to add 0 to each side which will give dissimilarity a> a+ 2q. Now you have to subtract from each side to get 0 > 2w and therefore the answer is (b)

    Which of the following has the greatest value?

        800600
        10200
        2.425
        30
        80 

    Answer (a)

    Explanation: The significant fact about this question is that it is not necessary to find the correct value of the expression. You require rearranging as many answer choices as possible so that they have the exponents of 100. Take a close glance on each of the expression 800600, 30, 80, 2.425, 10200. It will make clear that (a) is bigger than (c), therefore the answer is (a)

    What is the distance between the m intercept and the k intercept of the line given by the equation?

    2z = 6 - m?

        63.00
        89.12
        5.81
        56.78
        2.78 

    Answer (c)

    Explanation: To find the k - intercept just make m = 0 and solve for k. k= 3. Now to find the m intercept, make k = 0 and solve for m. You will see that they form a right angle triangle with length of 3 and 6, in which the hypotenuse represents the distance in between the two distinct points. You are required to use Pythagoras theorem to find the length of the hypotenuse, which will be equal to 5.8156 approximately; therefore the answer is (c).

    How many dissimilar three digit numbers include only non-zero numbers?

        894
        124
        56
        729
        234 

    Answer (d)

    Explanation: If there is a three digit no and it has the no 0. Then there are in fact nine possibilities for the first numbers from 1 to 9, 9 possibilities for the 2nd numbers from 1 to 9, 9th possibilities for the third number from 1 to 9. This comes to the total of 9*9*9 possible 3 digit numbers therefore the right answer is 729.

    If b = cos θ and l = sin θ, then for all θ, b2 + l2 =

        235
        1
        65
        127
        78 

    Answer (b)

    Explanation: Substitute for x and y. So b2 + l2 = cos2θ + sin 2θ. But you should the correct identity of cos2θ+ sin 2θ= 1. Therefore the option is (b).

    Find the slope of the line given by the equation

    o + 3 = 3/2 (k - 3)?

        5/6
        59
        89
        45
        5/4 

    Answer (e)

    Explanation: You may see that this is the point slope form and that the slope is 3/2. If you don't see that rewrite the equations into slope form.

    If log 6 = 9 then a =

        12.354
        89.456
        1.09
        78.456
        12.45 

    Answer (c)

    Explanation: The equation log 6 = 9 in exponential form, k7= 6. Now you have to find the value of k, you are required to take the 9th root of both the sides. You will see that k = 1.09. The correct option is (c).

    If t < w < 1 then all of the following must be true except:

        89
        34w
        w < w
        789
        145 

    Answer (c)

    Explanation: First place the amount 0.5 in place of g. Now w3 = 0.25, cube root of n = 0.707, w = 0.5, - w = - 0.5, -w = -0.5, and 1/w = 2. This makes (a), (b) (d) and (e) true and (c) false. 

%% page 25 241


    If (a + b)2 = 15 and ab = 5 then a2 +b2 =

        45
        78
        8
        23
        5 

    Answer (5)

    Explanation: (a + b)2 = 15

    (a + b) (a = b) = 15

    A2 + 2ab + b2 = 15

    Since ab = 5

    A2 2(5) + b2 = 15

    A2 + 10 + b2 = 15

    A2 + b2 = 5 Therefore the correct option is (e)

    If w(a) = 5a2+5a+5, which of the following is equal to w(-4.5)?

        23b
        89/4
        b(2.5)
        86b3
        5.78k 

    Answer (c)

    If you see carefully then you can notice w(-4.5) and w(4.5) both are equal to 39. Therefore the correct answer is option (c)

    What is the distance between the p intercept and the v intercept of the line given by the equation?

    2v = 6 - p?

        784
        125
        6.71
        2.784
        365 

    Answer (c)

    Explanation: To find the v - intercept just make p = 0 and solve for v. v= 3. Now to find the p intercept, make v = 0 and solve for p. You will see that they form a right angle triangle with length of 3 and 6, in which the hypotenuse represents the distance in between the two distinct points. You are required to used Pythagoras theorem to find the length of the hypotenuse, which will be equal to 6.7156 approximately; therefore, the answer is (c).

    If a - p > a + p, then which of the following must be true?

        p < 8
        25p
        564
        147
        45 

    Answer (a)

    Explanation: Algebraic manipulation is the simplest way to answer this problem. You require to add to each side which will give inequality a > a + 2p. Now you are required to subtract from each side to get 0 > 2p and you will find your answer is (b).

    If k(u) =2u2+ 2 then what is the value of k(u + 8)?

        32
        41k
        872
        25u
        2y2+ 16y +34 

    Answer (e)

    Explanation: Since there are variables in the answer choices you should insert in. Try u = 3. We are trying to find k(3 + 8) = k(8) = 2(8)2 + 2 which is 100 our target number. Now insert 3 in for y in the answer choices to see which answer choice goes with the target its (e)

    Find the slope of the line given by the equation

    d + 3 = 3/2 (b - 7)?

        897
        23
        15/45
        478
        5/4 

    Answer (e)

    Explanation: You may see that this is the point slope form and that the slope is 3/2. If you don't see that rewrite the equations into slope form.

    If the ratio of sec b to cos b is 1:8, then the ratio of tan b to cot b is

        1:16
        6:45
        89:12
        9:16
        2:78 

    Answer (a)

    Explanation: The ratio which is given can be written in fractional form like sec p/cosec b = 1/8. The secant and cosecant can also be expressed in terms of sine and cosine. 1/cos b/1/sin b = 1/8. The cotangent is the reciprocal of the tangent so cot b = 8. Therefore, the correct answer would be (a)

    If p = 5/2 and a = 6 then ap + 4/p2 =

        2
        56
        78
        18
        235 

    Answer (e)

    Explanation: Insert the values given for p and a into the equation

    a/p + 4/p2 =

    6/2 + 4/(5/2)2 =6/5/2 + 4/4/9 =

    (6*2/5) + (4 * 9/4) =

    18/2 + 36/4 = 9 + 9 = 18 therefore the correct answer is (d).

    If b - 5 = 5 (1 - b), then what is the value of b?

        0.33
        456
        784
        12.74
        32 

    Answer (a)

    Explanation: The fraction w/m is a equal to n/w. To find the numerical value of w/m just turn the numerical value of m/w; this comes to 0.724. Your calculation will tell you that 1/0.724 = 0.33.

    If 9q + 2p = 11 and w - 3p = 5, then what is the value of w?

        124
        98
        78
        89
        2.0 

    Answer (e)

    Explanation: When you have two equations which are very similar in form you are probably looking at classic ETS style simultaneous equations. Adding the two equations cancels out the d term leaving you with the equation 8q = 16, so w = 2. 

\end{comment}


\endinput



