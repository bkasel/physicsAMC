
%% university of northern iowa
%%  morgans astronomy exams
%%--------------------------------------------------

%% this section contains 20 problems

\element{morgans}{
\begin{question}{test4F-q01}
    Which of the following statements about Jovian planets is \emph{false}?
    \begin{choices}
        \wrongchoice{they have low densities}
        \wrongchoice{they have no solid surface}
        \wrongchoice{they are located in the outer solar system}
      \correctchoice{they have small masses and sizes}
    \end{choices}
\end{question}
}

\element{morgans}{
\begin{question}{test4F-q02}
    A volcano found along a region of subduction is a \rule[-0.1pt]{4em}{0.1pt} volcano.
    \begin{choices}
        \wrongchoice{andesitic.}
        \wrongchoice{basaltic.}
        \wrongchoice{metamorphic.}
        \wrongchoice{pyroclastic.}
    \end{choices}
\end{question}
}

\element{morgans}{
\begin{question}{test4F-q03}
    Which of the following represents the Earth's present atmosphere?
    \begin{choices}
        \wrongchoice{N (\SI{21}{\percent}), O (\SI{78}{\percent}}), \ce{CO2} (\SI{0.5}{\percent}}), \ce{H2O} (\SI{0.5}{\percent}})}
        \wrongchoice{N (\SI{21}{\percent}), O (\SI{21}{\percent}}), \ce{CO2} (\SI{5}{\percent}}), \ce{H2O} (\SI{53}{\percent}})}
        \wrongchoice{N (\SI{28}{\percent}), O (\SI{21}{\percent}}), \ce{CO2} (\SI{53}{\percent}}), \ce{H2O} (\SI{5}{\percent}})}
      \correctchoice{N (\SI{78}{\percent}), O (\SI{21}{\percent}}), \ce{CO2} (\SI{0.03}{\percent}), \ce{H2O} (\SI{0.1}{\percent})}
    \end{choices}
\end{question}
}

\element{morgans}{
\begin{question}{test4F-q04}
    The Moon lacks a permanent atmosphere because:
    \begin{choices}
        \wrongchoice{it never had one to begin with.}
      \correctchoice{its gravity is too small to hold a permanent atmosphere.}
        \wrongchoice{the Earth's strong gravity pulled it away.}
        \wrongchoice{its atmosphere condensed and fell to the surface because it was so cold.}
    \end{choices}
\end{question}
}

\element{morgans}{
\begin{question}{test4F-q05}
    The maria are thought to have originated as:
    \begin{choices}
        \wrongchoice{bodies of water which are now dried-up.}
      \correctchoice{lava flows.}
        \wrongchoice{uplifted plains (highlands).}
        \wrongchoice{areas eroded by wind before the Moon lost its atmosphere.}
    \end{choices}
\end{question}
}

\element{morgans}{
\begin{question}{test4F-q06}
    Why does it get very cold at night on Mercury? 
    \begin{choices}
      \correctchoice{there is no atmosphere, so heat easily escapes from the planet.}
        \wrongchoice{the atmosphere is too thick to let heat (light) in.}
        \wrongchoice{the atmosphere is not thick enough to hold heat in, though it does hold in some.}
        \wrongchoice{this is a trick question---it doesn't get cold at night on Mercury since it is so close to the Sun.}
    \end{choices}
\end{question}
}

\element{morgans}{
\begin{question}{test4F-q07}
    Venus' atmospheric chemical composition is mostly:
    \begin{choices}
        \wrongchoice{nitrogen.}
        \wrongchoice{carbon dioxide.}
        \wrongchoice{oxygen.}
        \wrongchoice{water vapor.}
        \wrongchoice{hydrogen.}
    \end{choices}
\end{question}
}

\element{morgans}{
\begin{question}{test4F-q08}
    Why is the surface of Venus so hot?
    \begin{choices}
        \wrongchoice{atmospheric carbon monoxide allows visible light in but doesn't let any heat (infrared light) out.}
      \correctchoice{atmospheric carbon dioxide allows visible light in but doesn't let any heat (infrared light) out.}
        \wrongchoice{atmospheric carbon monoxide lets heat (infrared light) in but doesn't let any visible light out.}
        \wrongchoice{atmospheric carbon dioxide lets heat (infrared light) in but doesn't let visible light out.}
    \end{choices}
\end{question}
}

\element{morgans}{
\begin{question}{test4F-q09}
    Although Mars' tilt on its axis is the same as that of the Earth's (23.5 degrees),
        its seasons are more complex than those of the Earth because of:
    \begin{choices}
        \wrongchoice{the influence of Mars' internal energy source.}
        \wrongchoice{the planet's strong magnetic field.}
        \wrongchoice{extensive current volcanic activity.}
        \wrongchoice{the tidal effects of its two satellites.}
      \correctchoice{the elliptical shape of its orbit.}
    \end{choices}
\end{question}
}

\element{morgans}{
\begin{question}{test4F-q10}
    Which planet has been the least explored?
    \begin{choices}
        \wrongchoice{Mercury}
        \wrongchoice{Venus}
        \wrongchoice{Earth}
        \wrongchoice{Mars}
    \end{choices}
\end{question}
}

\element{morgans}{
\begin{question}{test4F-q11}
    Jupiter's mass is approximately \rule[-0.1pt]{4em}{0.1pt} times that of the Earth.
    \begin{choices}
        \wrongchoice{10}
        \wrongchoice{50}
        \wrongchoice{100}
      \correctchoice{300}
        \wrongchoice{700}
    \end{choices}
\end{question}
}

\element{morgans}{
\begin{question}{test4F-q12}
    Jupiter's magnetic field is flattened because of the solar wind and the:
    \begin{choices}
        \wrongchoice{influence of Io.}
      \correctchoice{planet's rotation.}
        \wrongchoice{influence of its ring.}
        \wrongchoice{influence of passing comets.}
    \end{choices}
\end{question}
}

\element{morgans}{
\begin{question}{test4F-q13}
    How many moons is Jupiter known to possess?
    \begin{choices}
        \wrongchoice{4}
        \wrongchoice{8}
        \wrongchoice{12}
        \wrongchoice{17}
      \correctchoice{More than 20}
    \end{choices}
\end{question}
}

\element{morgans}{
\begin{question}{test4F-q14}
    The Roche limit is the distance from a planet:
    \begin{choices}
        \wrongchoice{beyond which the planet's gravitational tidal forces are so strong that satellites cannot form.}
        \wrongchoice{at which tidal forces are at their maximum.}
      \correctchoice{at which the planet's tidal forces balance the self gravity of objects.}
        \wrongchoice{beyond which only ring systems can form.}
    \end{choices}
\end{question}
}

\element{morgans}{
\begin{question}{test4F-q15}
    Which of the giants planets has the least atmospheric activity?
    \begin{choices}
        \wrongchoice{Jupiter}
        \wrongchoice{Saturn}
      \correctchoice{Uranus}
        \wrongchoice{Neptune}
    \end{choices}
\end{question}
}

\element{morgans}{
\begin{question}{test4F-q16}
    Which of the following is \emph{not} a satellite of Uranus?
    \begin{choices}
        \wrongchoice{Titania}
      \correctchoice{Titan}
        \wrongchoice{Oberon}
        \wrongchoice{Ariel}
        \wrongchoice{Miranda}
    \end{choices}
\end{question}
}

\element{morgans}{
\begin{question}{test4F-q17}
    Approximately, what is the current number of known asteroids?
    \begin{choices}
        \wrongchoice{less than 100}
        \wrongchoice{500}
        \wrongchoice{1000}
      \correctchoice{more than 5,000}
    \end{choices}
\end{question}
}

\element{morgans}{
\begin{question}{test4F-q18}
    In general, short period comets:
    \begin{choices}
        \wrongchoice{have orbits which are highly elongated.}
        \wrongchoice{have orbits which are inclined at large angles to the ecliptic.}
        \wrongchoice{formed in the spherically shaped Oort cloud.}
        \wrongchoice{are really asteroids.}
      \correctchoice{come from the Kuiper belt.}
    \end{choices}
\end{question}
}

\element{morgans}{
\begin{question}{test4F-q19}
    A meteorite is:
    \begin{choices}
        \wrongchoice{a rock or grain of sand passing through the Earth's atmosphere.}
        \wrongchoice{the trail left by a rock or piece of sand as it passes through the Earth's atmosphere.}
      \correctchoice{a rock from space that strikes the ground.}
        \wrongchoice{most often a part of a comet's nucleus.}
    \end{choices}
\end{question}
}

\element{morgans}{
\begin{question}{test4F-q20}
    Asteroid-sized bodies which formed the cores of the larger planets are known as:
    \begin{choices}
        \wrongchoice{asteroids.}
        \wrongchoice{the solar nebula.}
      \correctchoice{planetesimals.}
        \wrongchoice{planetoids.}
        \wrongchoice{minor bodies.}
    \end{choices}
\end{question}
}

\begin{comment}
    Fill In
    Place the most appropriate word or words in the blank. You may have to click on the blank to activate it before you start typing in your answer.
    was the probe which has visited the four Jovian planets.

    Titan's atmosphere is made up mainly of .

    The interaction of charged particles from the Sun with the Earth's atmosphere produce which tend to occur near the Earth's poles.

    waves can travel through liquids.

    A(n) results from rapid planetary rotation coupled with a metallic conducting core.

    The region of the electromagnetic spectrum is used to study the surface of Venus by bouncing waves off the surface.

    The point near a massive body such as a planet inside of which the tidal forces acting on orbiting particles exceed the gravitational force holding the particles together is called the .

    Uranus was discovered by .

    A bright streak of light produced by a particle passing through the Earth's atmosphere is called a(n) .

    The primordial gas and dust cloud from which the Sun and the planets condensed was enriched by a .
\end{comment}


\endinput


