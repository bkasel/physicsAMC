
%% Astronomy:
%%--------------------------------------------------

%% Chapter 02: Learning Objectives and Study Questions for
%%------------------------------------------------------------

% 1. Recognize an empirically meaningful statement based on the criterion discussed in class.
% 2. Explain, in light of Occam's razor, why the heliocentric model for the solar system replaced the geocentric model.
% 3. State Kepler's laws and explain:
%     (1) how the first is related to the orbital motions of two bodies around a common center of mass;
%     (2) how the second is related to the conservation of angular momentum (mvr) by an orbiting body; and
%     (3) how to use the third to calculate the period of a planet's orbit given the length of its semimajor axis and vice versa.
% 4. Describe how the motions of bodies (velocities and accelerations) are influenced by the forces that act upon them using the concepts of mass and inertia.
% 5. Describe how the force of gravity between two bodies depends upon their masses and the distance between their centers of gravity and use this relationship to predict how a person's weight will change if they travel to a planet whose mass and diameter are give relative to those of Earth.

\element{astr}{
\begin{question}{ch02-q01}
    An important aspect of all scientific theories is that they \rule[-0.1pt]{4em}{0.1pt}.
    \begin{choices}
        \wrongchoice{make testable predictions}
        \wrongchoice{are bound to tradition}
        \wrongchoice{are unchanging}
        \wrongchoice{routinely ignore selected observations}
        \wrongchoice{are based on human authority}
    \end{choices}
\end{question}
}

\element{astr}{
\begin{question}{ch02-q02}
    A key distinction between scientific and revealed knowledge is that any scientific theory can,
        at least potentially, be \rule[-0.1pt]{4em}{0.1pt} using observations.
    \begin{choices}
        \wrongchoice{proven}
        \wrongchoice{disproven}
        \wrongchoice{confirmed}
        \wrongchoice{verified}
        \wrongchoice{established}
    \end{choices}
\end{question}
}

\element{astr}{
\begin{question}{ch02-q03}
    If several hypotheses are put forward to explain a phenomenon Occam's razor holds that the \rule[-0.1pt]{4em}{0.1pt} is most likely to be correct.
    \begin{choices}
        \wrongchoice{newest}
        \wrongchoice{oldest}
        \wrongchoice{most authoritative}
        \wrongchoice{simplest}
        \wrongchoice{most radical}
    \end{choices}
\end{question}
}

\element{astr}{
\begin{question}{ch02-q04}
    Although Copernicus’ \rule[-0.1pt]{4em}{0.1pt} cosmology did not predict planetary positions much more accurately than the earlier geocentric cosmology,
        its greater simplicity led to its widespread acceptance.
    \begin{choices}
        \wrongchoice{egocentric}
        \wrongchoice{cosmocentric}
        \wrongchoice{eccentric}
        \wrongchoice{astrocentric}
        \wrongchoice{heliocentric}
    \end{choices}
\end{question}
}

\element{astr}{
\begin{question}{ch02-q05}
    Brahe’s conclusion that the 1572 supernova occurred far from Earth was based on his failure to observe \rule[-0.1pt]{4em}{0.1pt},
        or an apparent shift in its position relative to distant background stars.
    \begin{choices}
        \wrongchoice{Doppler shift}
        \wrongchoice{parallax}
        \wrongchoice{prograde motion}
        \wrongchoice{retrograde motion}
        \wrongchoice{transform motion}
    \end{choices}
\end{question}
}

\element{astr}{
\begin{question}{ch02-q06}
    \rule[-0.1pt]{4em}{0.1pt}
    Like Brahe,
        Galileo’s greatest contribution to our understanding of the cosmos probably came from his \rule[-0.1pt]{4em}{0.1pt}.
    \begin{choices}
        \wrongchoice{revolutionary theories}
        \wrongchoice{detailed observations}
        \wrongchoice{elegant mathematical models}
        \wrongchoice{sophisticated technology}
        \wrongchoice{gently persuasive arguments}
    \end{choices}
\end{question}
}

\element{astr}{
\begin{question}{ch02-q07}
    According to Kepler’s first law the orbits of all of the planets are \rule[-0.1pt]{4em}{0.1pt} with the Sun at one focus.
    \begin{choices}
        \wrongchoice{circles}
        \wrongchoice{ellipses}
        \wrongchoice{hyperbolas}
        \wrongchoice{ovals}
        \wrongchoice{parabolas}
    \end{choices}
\end{question}
}

\element{astr}{
\begin{question}{ch02-q08}
    \rule[-0.1pt]{4em}{0.1pt}
    According to Kepler's third law a planet at an orbital distance of about 12 AU from the Sun will have an orbital period of about \rule[-0.1pt]{4em}{0.1pt} years.
    \begin{choices}
        \wrongchoice{2.3}
        \wrongchoice{3.5}
        \wrongchoice{41.6}
        \wrongchoice{144}
        \wrongchoice{1728}
    \end{choices}
\end{question}
}

\element{astr}{
\begin{question}{ch02-q09}
    A corollary of Kepler's second law is that planets travel \rule[-0.1pt]{4em}{0.1pt}.
    \begin{choices}
        \wrongchoice{at the same speeds throughout their orbits}
        \wrongchoice{at variable speeds independent of their distances from the Sun}
        \wrongchoice{fastest when they are farthest from the Sun}
        \wrongchoice{fastest when they are closest to the Sun}
        \wrongchoice{slowest when they are closest to the Sun}
    \end{choices}
\end{question}
}

\element{astr}{
\begin{question}{ch02-q10}
    Conservation of angular momentum causes a spinning body to \rule[-0.1pt]{4em}{0.1pt} as it contracts.
    \begin{choices}
        \wrongchoice{spin more rapidly}
        \wrongchoice{maintain its rotational speed}
        \wrongchoice{spin more slowly}
        \wrongchoice{change speed irregularly}
        \wrongchoice{stop spinning}
    \end{choices}
\end{question}
}

\element{astr}{
\begin{question}{ch02-q11}
    According to Newton’s second law of motion,
        the change in speed or direction an object undergoes when a force is applied to it is inversely proportional to its \rule[-0.1pt]{4em}{0.1pt}.
    \begin{choices}
        \wrongchoice{length}
        \wrongchoice{age}
        \wrongchoice{charge}
        \wrongchoice{mass}
        \wrongchoice{velocity}
    \end{choices}
\end{question}
}

\element{astr}{
\begin{question}{ch02-q12}
    Suppose that you weigh 120 pounds on Earth and travel to a planet that has the same mass but twice the radius of our world. 
    How much will you weigh if you stand on the surface of this planet?
    \begin{choices}
        \wrongchoice{240 lbs.}
        \wrongchoice{180 lbs.}
        \wrongchoice{120 lbs.}
        \wrongchoice{60 lbs.}
        \wrongchoice{30 lbs.}
    \end{choices}
\end{question}
}


\endinput



