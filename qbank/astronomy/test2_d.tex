
%% University of Northern Iowa
%%  Morgans Astronomy Exams
%%--------------------------------------------------

%% this section contains 15 problems

\element{morgans}{
\begin{question}{test2D-Q01}
    Hydrostatic equilibrium is an equilibrium between:
    \begin{choices}
        \wrongchoice{hydrogen and carbon.}
        \wrongchoice{water, hydrogen, and oxygen.}
      \correctchoice{gravity and outward pressure.}
        \wrongchoice{water and electric (static) charge.}
    \end{choices}
\end{question}
}

\element{morgans}{
\begin{question}{test2D-Q02}
    Fusion in the Sun occurs between:
    \begin{choices}
      \correctchoice{protons.}
        \wrongchoice{electrons and protons.}
        \wrongchoice{electrons and neutrons.}
        \wrongchoice{neutrons and protons.}
        \wrongchoice{electrons.}
    \end{choices}
\end{question}
}

\element{morgans}{
\begin{question}{test2D-Q03}
    The absorption line spectrum of the Sun comes from the:
    \begin{choices}
        \wrongchoice{interior.}
      \correctchoice{photosphere.}
        \wrongchoice{chromosphere.}
        \wrongchoice{corona.}
    \end{choices}
\end{question}
}

\element{morgans}{
\begin{question}{test2D-Q04}
    What is the period of the Sun's oscillation?
    \begin{choices}
        \wrongchoice{1 minute}
      \correctchoice{5 minutes}
        \wrongchoice{1 hour}
        \wrongchoice{5 hours}
        \wrongchoice{25 days}
    \end{choices}
\end{question}
}

\element{morgans}{
\begin{question}{test2D-Q05}
    If a star has a parallax of 0.75 seconds of arc,
        what is its approximate distance in parsecs?
    \begin{choices}
        \wrongchoice{0.75}
      \correctchoice{1.3}
        \wrongchoice{4.3}
        \wrongchoice{18}
    \end{choices}
\end{question}
}

\element{morgans}{
\begin{question}{test2D-Q06}
    What information can be obtained from the motion of an optical binary?
    \begin{choices}
        \wrongchoice{only the masses of the stars}
        \wrongchoice{the masses and radii of the stars}
        \wrongchoice{only the radii of the stars}
        \wrongchoice{no information, since it is not a true binary system}
    \end{choices}
\end{question}
}

\element{morgans}{
\begin{question}{test2D-Q07}
    An H-R diagram is a plot of:
    \begin{choices}
        \wrongchoice{heat versus radius.}
        \wrongchoice{luminosity versus radius.}
        \wrongchoice{mass versus temperature.}
      \correctchoice{luminosity versus temperature.}
        \wrongchoice{mass versus luminosity.}
    \end{choices}
\end{question}
}

\element{morgans}{
\begin{question}{test2D-Q08}
    From knowledge of only a star's temperature and luminosity,
        we can determine its:
    \begin{choices}
        \wrongchoice{mass.}
      \correctchoice{radius.}
        \wrongchoice{distance.}
        \wrongchoice{period of rotation.}
        \wrongchoice{rotational velocity.}
    \end{choices}
\end{question}
}

\element{morgans}{
\begin{question}{test2D-Q09}
    If the interior temperature of a star decreases,
        the star will:
    \begin{choices}
        \wrongchoice{do nothing---that is, remain the same.}
      \correctchoice{contract.}
        \wrongchoice{expand.}
        \wrongchoice{cease to exist as a star but cool off to become a planet.}
    \end{choices}
\end{question}
}

\element{morgans}{
\begin{question}{test2D-Q10}
    Stars evolve because of changes in:
    \begin{choices}
      \correctchoice{chemical composition of the core.}
        \wrongchoice{luminosity of the star.}
        \wrongchoice{mass of the star.}
        \wrongchoice{chemical composition of the surface.}
        \wrongchoice{surface temperature.}
    \end{choices}
\end{question}
}

\element{morgans}{
\begin{question}{test2D-Q11}
    H-H objects are associated with:
    \begin{choices}
        \wrongchoice{Giant Molecular clouds}
        \wrongchoice{supernova}
        \wrongchoice{nova}
      \correctchoice{T Tauri stars}
    \end{choices}
\end{question}
}

\element{morgans}{
\begin{question}{test2D-Q12}
    A main sequence star of spectral type A2 will be on the main sequence \rule[-0.1pt]{4em}{0.1pt} than a F2 main sequence star.
    \begin{choices}
      \correctchoice{shorter}
        \wrongchoice{longer}
        \wrongchoice{for the same time}
        \wrongchoice{It is not possible to answer this question without more information}
    \end{choices}
\end{question}
}

\element{morgans}{
\begin{question}{test2D-Q13}
    Bipolar outflow can occur during:
    \begin{choices}
      \correctchoice{star formation.}
        \wrongchoice{the main sequence life of a star.}
        \wrongchoice{supernova.}
        \wrongchoice{the red giant phase.}
    \end{choices}
\end{question}
}

\element{morgans}{
\begin{question}{test2D-Q14}
    Helium burning requires a temperature of:
    \begin{choices}
        \wrongchoice{100 thousand K.}
        \wrongchoice{1 million K.}
        \wrongchoice{10 million K.}
      \correctchoice{100 million K.}
        \wrongchoice{1 billion K.}
    \end{choices}
\end{question}
}

\element{morgans}{
\begin{question}{test2D-Q15}
    A planetary nebula is:
    \begin{choices}
        \wrongchoice{a collapsing gas cloud out of which planets will form.}
      \correctchoice{an expanding gas cloud that was ejected by a star.}
        \wrongchoice{the cloud of gas blown off by a T Tauri type star.}
        \wrongchoice{a gas cloud which goes into the formation of red giants.}
    \end{choices}
\end{question}
}

\element{morgans}{
\begin{question}{test2D-Q16}
    Once iron has formed in the core of a star:
    \begin{choices}
        \wrongchoice{there is an increased production of energy as atoms fuse together.}
        \wrongchoice{more and more nuclear energy is produced as fission occurs.}
        \wrongchoice{the core expands and cools.}
      \correctchoice{no further energy producing reactions can occur.}
        \wrongchoice{it soon becomes a white dwarf.}
    \end{choices}
\end{question}
}

\element{morgans}{
\begin{question}{test2D-Q17}
    The most massive stars are thought to end up as:
    \begin{choices}
      \correctchoice{black holes.}
        \wrongchoice{planetary nebulae.}
        \wrongchoice{neutron stars.}
        \wrongchoice{white dwarfs.}
    \end{choices}
\end{question}
}

\element{morgans}{
\begin{question}{test2D-Q18}
    A white dwarf is about the diameter of:
    \begin{choices}
        \wrongchoice{the Sun.}
        \wrongchoice{a red supergiant.}
        \wrongchoice{a red dwarf.}
      \correctchoice{the Earth.}
        \wrongchoice{the Moon.}
    \end{choices}
\end{question}
}

\element{morgans}{
\begin{question}{test2D-Q19}
    Stellar remnants with masses between 2 and 3 solar masses will be:
    \begin{choices}
        \wrongchoice{white dwarfs.}
      \correctchoice{neutron stars.}
        \wrongchoice{black holes.}
        \wrongchoice{planetary nebulae.}
    \end{choices}
\end{question}
}

\element{morgans}{
\begin{question}{test2D-Q20}
    If the Sun were suddenly to be replaced by a solar-mass black hole the gravitational force on the Earth (1 A. U. away) would:
    \begin{choices}
        \wrongchoice{double.}
        \wrongchoice{become so strong that the Earth would be ``sucked'' into the black hole.}
        \wrongchoice{decrease because black holes cause gravity at large distances to disappear.}
      \correctchoice{remain the same.}
    \end{choices}
\end{question}
}


\begin{comment}
    Fill In
    Place the most appropriate word or words in the blank. You may have to click on the blank to activate it before you start typing in your answer.
     
    The word for the transfer of energy by mass motion is .

    The spotty appearance of the solar surface caused by convection in the layers just below is called .

    Binary stars discovered by means of their Doppler shifts are known as .

    classified over 200,000 stars in the Henry Draper Catalog.

    The movement of energy from one location to another by means of photon movement is called .

    A young star still associated with the interstellar material from which it formed, typically exhibiting brightness variations and a stellar wind, is called a(n) .

    The hydrogen reaction that forms helium is known as .

    The end product of the Sun's evolution will be a(n).

    A star that temporarily flares up in brightness, most likely as the result of nuclear reactions caused by the deposition of new nuclear fuel on the surface of a white dwarf in a binary system is called a(n) .

    The "surface'' of a black hole; the boundary of the region from within which no light can escape is called the
\end{comment}


\endinput



