
%% Astronomy II Practice for EXAM 1
%%--------------------------------------------------

%% This is a dup of Astronomy I practice Exm

%% this section contains 16 problems

\element{astr}{
\begin{question}{ASTRII-1-Q01}
    Measurements made on the celestial sphere are made in units of:
    \begin{multicols}{2}
    \begin{choices}
        \wrongchoice{miles}
        \wrongchoice{kilometers}
        \wrongchoice{degrees}
        \wrongchoice{light years}
    \end{choices}
    \end{multicols}
\end{question}
}

\element{astr}{
\begin{question}{ASTRII-1-Q02}
    Measurements made on the celestial sphere are made in units of:
    \begin{multicols}{2}
    \begin{choices}
2. An astronomical unit is the

a. distance from the Earth to the Moon.
b. distance from the Earth to the Sun.
c. distance from the Earth to the nearest star.
d. distance light travels in one year.
e. circumference of the Earth.

    \end{choices}
    \end{multicols}
\end{question}
}

3. Which of the following types of electromagnetic radiation has
a wavelength adjacent to but longer than visible light?

a. radio
b. infrared
c. X ray
d. ultraviolet
e. gamma ray

4. The energy of a photon is

a. proportional to the wavelength and inversely proportional to the frequency.
b. proportional to the wavelength and proportional to the frequency.
c. inversely proportional to the wavelength and inversely proportional to the frequency.
d. inversely proportional to the wavelength and proportional to the frequency.

5. Radio waves have

a. high energy and long wavelength.
b. low energy and long wavelength.
c. low energy and short wavelength.
d. high energy and short wavelength.

6. If Star A is hotter than Star B, and Star A is emitting most of its light at a wavelength
corresponding to yellow light, which of the following statements is true?

a. Star B will emit most of its light at a wavelength longer than yellow
b. Star B will emit most of its light at a wavelength shorter than yellow
c. Star B will emit most of its light at the same wavelength as Star A
d. more information is required to answer this question

7. If two stars have the same surface area but one has 3 times the temperature of the other,
how many times more energy is radiated by the more luminous star?

a. 3
b. 9
c. 12
d. 27
e. 81

8. Star A is radiating two times as much energy as Star B, but it is two times the distance
from us. Which star will appear brighter, and by how much?

a. Star A will be 2 times brighter
b. Star B will be 2 times brighter
c. Star A will be 4 times brighter
d. Star B will be 4 times brighter
e. they will both have the same observed brightness

9. Which one of the following types of spectrum always occurs with an
absorption (or dark line) spectrum?

a. bright line
b. emission line
c. continuous

10. Spectral lines unique to each type of atom are caused by

a. each atom having a unique set of protons.
b. the unique sets of electron orbits.
c. the neutron-electron interaction being unique for each atom.
d. each type of photon emitted by the atom being unique.
e. none of the above; spectral lines are not unique to each type of atom.

11. A star has an absorption spectrum showing many lines corresponding to silicon.
Before it reaches an observer, the light from this star passes through a cool gas cloud
containing a large amount of silicon. What will the observer detect?

a. an absorption spectrum with many silicon lines
b. an absorption and emission spectrum with lines corresponding to silicon
c. an emission spectrum of many silicon lines
d. a continuous spectrum

12. Consider a cloud of (cool) gas between a star and an observer to be moving away
from a source of continuous radiation (and towards the observer). Suppose the atoms
in the gas have two energy levels separated by an energy corresponding to 5000 Angstroms.
The observer will see a spectrum with absorption at a wavelength

a. less than 5000 Angstroms.
b. equal to 5000 Angstroms.
c. greater than 5000 Angstroms.
d. no absorption will take place.

13. If an electron moves from a lower energy level to the next higher energy level, then

a. the atom has become excited.
b. the atom has become ionized.
c. the atom's light will be blue shifted.
d. the atom's light will be red shifted.

14. The degree of ionization of an atom (i.e. the number of electrons lost) depends on

a. the level of the ground state.
b. the temperature of the gas.
c. the energy difference between the ground state and the first excited state.
d. the distance to the observer.

15. Which of the following types of light can only be observed from space?

a. visible
b. radio
c. infrared
d. gamma ray

16. The determination of stellar parallax is important because it allows the direct determination of

a. mass.
b. distance.
c. diameter.
d. velocity.


Astronomy II Practice EXAM 1

Answer Key

1. c
2. b
3. b
4. d
5. b
6. a
7. e
8. b
9. c
10. b
11. a
12. a
13. a
14. b
15. d
16. b 


