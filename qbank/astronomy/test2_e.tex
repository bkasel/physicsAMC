
%% University of Northern Iowa
%%  Morgans Astronomy Exams
%%--------------------------------------------------

%% this section contains 15 problems

\element{morgans}{
\begin{question}{test2E-Q01}
    Einstein's famous equation, $E=mc^2$, says that:
    \begin{choices}
        \wrongchoice{electromagnetism is the same as mass.}
      \correctchoice{energy is equivalent to mass.}
        \wrongchoice{kinetic energy is determined by mass.}
        \wrongchoice{the speed of light never changes.}
        \wrongchoice{energy is equivalent to light.}
    \end{choices}
\end{question}
}

\element{morgans}{
\begin{question}{test2E-Q02}
    The difference between helium-3 and helium-4 is:
    \begin{choices}
      \correctchoice{a neutron}
        \wrongchoice{a proton}
        \wrongchoice{an electron}
        \wrongchoice{a neutrino}
        \wrongchoice{a positron}
    \end{choices}
\end{question}
}

\element{morgans}{
\begin{question}{test2E-Q03}
    Prominences are associated with:
    \begin{choices}
      \correctchoice{sunspots.}
        \wrongchoice{granulation.}
        \wrongchoice{absorption lines.}
        \wrongchoice{the interior.}
    \end{choices}
\end{question}
}

\element{morgans}{
\begin{question}{test2E-Q04}
    Which of the following statements concerning solar neutrinos is correct?
    \begin{choices}
      \correctchoice{less have been observed than predicted}
        \wrongchoice{observed and predicted numbers are equal}
        \wrongchoice{more have been observed than predicted}
    \end{choices}
\end{question}
}

\element{morgans}{
\begin{question}{test2E-Q05}
    Which one of the following is the primary difference between the observed spectra of most stars?
    \begin{choices}
        \wrongchoice{the presence or absence of a continuous spectrum}
      \correctchoice{the differing strengths and patterns of the absorption lines}
        \wrongchoice{the differing strengths and patterns of the emission lines}
        \wrongchoice{spectra of all stars have approximately the same appearance}
    \end{choices}
\end{question}
}

\element{morgans}{
\begin{question}{test2E-Q06}
    The reason astronomers use the concept of the absolute magnitude is to:
    \begin{choices}
        \wrongchoice{make life difficult for introductory astronomy students!}
      \correctchoice{allow stars to be compared with the effects of differing distance removed}
        \wrongchoice{allow stars to be compared with the effects of differing mass removed}
        \wrongchoice{allow stars to be compared with the effects of differing temperature removed}
        \wrongchoice{allow stars to be compared with the effects of differing radius removed.}
    \end{choices}
\end{question}
}

\element{morgans}{
\begin{question}{test2E-Q07}
    The main sequence extends from:
    \begin{choices}
      \correctchoice{high luminosity, high temperature to low luminosity, low temperature.}
        \wrongchoice{high luminosity, low temperature to low luminosity, low temperature.}
        \wrongchoice{high luminosity, low temperature to low luminosity, high temperature.}
        \wrongchoice{high luminosity, high temperature to low luminosity high temperature.}
    \end{choices}
\end{question}
}

\element{morgans}{
\begin{question}{test2E-Q08}
    Binary stars are particularly important for determining stellar:
    \begin{choices}
        \wrongchoice{temperature.}
        \wrongchoice{luminosity.}
        \wrongchoice{distance.}
        \wrongchoice{radius.}
      \correctchoice{mass.}
    \end{choices}
\end{question}
}

\element{morgans}{
\begin{question}{test2E-Q09}
    The masses of stars on the main sequence \rule[-0.1pt]{4em}{0.1pt} from the lower right to the upper left.
    \begin{choices}
      \correctchoice{increase}
        \wrongchoice{decrease}
        \wrongchoice{are all the same}
        \wrongchoice{are randomly distributed}
    \end{choices}
\end{question}
}

\element{morgans}{
\begin{question}{test2E-Q10}
    What is the rarest type of main sequence star?
    \begin{choices}
      \correctchoice{O}
        \wrongchoice{F}
        \wrongchoice{G}
        \wrongchoice{M}
    \end{choices}
\end{question}
}

\element{morgans}{
\begin{question}{test2E-Q11}
    The temperature of a Giant Molecular Cloud is about:
    \begin{choices}
      \correctchoice{5 K.}
        \wrongchoice{50 K.}
        \wrongchoice{500 K.}
        \wrongchoice{5000 K.}
    \end{choices}
\end{question}
}

\element{morgans}{
\begin{question}{test2E-Q12}
    Which part of a star changes composition while the star is on the Main Sequence?
    \begin{choices}
      \correctchoice{The core}
        \wrongchoice{The convective zone}
        \wrongchoice{The radiative zone}
        \wrongchoice{The photosphere}
    \end{choices}
\end{question}
}

\element{morgans}{
\begin{question}{test2E-Q13}
    A star like the Sun will have \rule[-0.1pt]{4em}{0.1pt} percent of its mass as hydrogen.
    \begin{choices}
        \wrongchoice{10}
        \wrongchoice{30}
        \wrongchoice{50}
      \correctchoice{70}
        \wrongchoice{100}
    \end{choices}
\end{question}
}

\element{morgans}{
\begin{question}{test2E-Q14}
    As a degenerate gas is compressed, it will:
    \begin{choices}
      \correctchoice{heat up but not get denser.}
        \wrongchoice{heat up and get denser.}
        \wrongchoice{only get denser.}
        \wrongchoice{do nothing.}
    \end{choices}
\end{question}
}

\element{morgans}{
\begin{question}{test2E-Q15}
    The stellar remnant of a one solar mass star is a:
    \begin{choices}
      \correctchoice{white dwarf.}
        \wrongchoice{neutron star.}
        \wrongchoice{pulsar.}
        \wrongchoice{black hole.}
        \wrongchoice{main sequence star.}
    \end{choices}
\end{question}
}

\element{morgans}{
\begin{question}{test2E-Q16}
    The most mass a white dwarf can have is about:
    \begin{choices}
        \wrongchoice{1 M.}
      \correctchoice{1.4 M.}
        \wrongchoice{3 M.}
        \wrongchoice{10 M.}
        \wrongchoice{there is no limit to the mass.}
    \end{choices}
\end{question}
}

\element{morgans}{
\begin{question}{test2E-Q17}
    Heavy elements which are mixed into the material from which new generations of stars may come primarily from:
    \begin{choices}
        \wrongchoice{the big bang.}
        \wrongchoice{planetary nebulae.}
      \correctchoice{supernovae.}
        \wrongchoice{neutron stars.}
        \wrongchoice{Wolf-Rayet stars.}
    \end{choices}
\end{question}
}

\element{morgans}{
\begin{question}{test2E-Q18}
    All novae are thought to involve a:
    \begin{choices}
      \correctchoice{white dwarf.}
        \wrongchoice{main sequence star.}
        \wrongchoice{supergiant.}
        \wrongchoice{neutron star.}
        \wrongchoice{black hole.}
    \end{choices}
\end{question}
}

\element{morgans}{
\begin{question}{test2E-Q19}
    Pulsars are known to be:
    \begin{choices}
        \wrongchoice{pulsating white dwarfs.}
        \wrongchoice{pulsating neutron stars.}
        \wrongchoice{rotating white dwarfs.}
      \correctchoice{rotating neutron stars.}
        \wrongchoice{rotating black holes.}
    \end{choices}
\end{question}
}

\element{morgans}{
\begin{question}{test2E-Q20}
    As a massive star collapses, the gravitational field on the stellar surface:
    \begin{choices}
        \wrongchoice{doubles.}
      \correctchoice{increases strongly.}
        \wrongchoice{decreases with the square of the decreasing size.}
        \wrongchoice{remains the same.}
    \end{choices}
\end{question}
}

\begin{comment}
    Fill In
    Place the most appropriate word or words in the blank. You may have to click on the blank to activate it before you start typing in your answer.
     
    The general type of nuclear reactions in which low mass particles form higher mass particles is .

    An explosive outburst of ionized gas from the Sun, usually accompanied by X-ray emission and the injection of large quantities of charged particles into the solar wind is called a(n) .

    Binary stars discovered by means of their varying light are known as .

    Absorption lines in a star's spectrum are produced in the .

    The movement of energy from one location to another by means of mass motion is called .

    A star in the process of formation, specifically one that has entered the slow gravitational contraction phase, is called a(n) .

    The evolution of a star depends primarily on the star's .

    A compact electron degenerate object approximately the size of the Earth is known as a(n) .

    Radiation produced by electrons moving with high velocities in a magnetic field is called .

    The distance over which the gravitational field of an object is so strong that light cannot escape is called the
\end{comment}


\endinput



