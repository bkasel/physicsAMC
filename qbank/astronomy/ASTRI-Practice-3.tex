
%% SAMPLE EXAM 3: FOR STELLAR EVOLUTION
%%--------------------------------------------------

%% this section contains 50 problems

\element{astr}{
\begin{question}{ASTRI-3-Q01}
    The masses of stars on the main sequence \rule[-0.1pt]{4em}{0.1pt} from the lower right to the upper left.
    \begin{choices}
      \correctchoice{increase}
        \wrongchoice{decrease}
        \wrongchoice{are all the same}
        \wrongchoice{are randomly distributed}
    \end{choices}
\end{question}
}

\element{astr}{
\begin{question}{ASTRI-3-Q02}
    The fundamental quantity which determines a star's central pressure and temperature is its
    \begin{choices}
      \correctchoice{mass.}
        \wrongchoice{luminosity.}
        \wrongchoice{surface temperature.}
        \wrongchoice{chemical composition.}
        \wrongchoice{radius.}
    \end{choices}
\end{question}
}

\element{astr}{
\begin{question}{ASTRI-3-Q03}
    The Russell-Vogt theorem (about chemical composition) states that the properties of a star at any given age depend completely upon the star's:
    \begin{choices}
        \wrongchoice{luminosity and radius.}
        \wrongchoice{mass and luminosity.}
        \wrongchoice{mass, chemical composition, and radius.}
        \wrongchoice{radius, and chemical composition.}
      \correctchoice{mass and chemical composition.}
    \end{choices}
\end{question}
}

\element{astr}{
\begin{question}{ASTRI-3-Q04}
    The lifetime of a star is determined by its initial:
    \begin{choices}
        \wrongchoice{temperature and luminosity.}
        \wrongchoice{temperature and radius.}
        \wrongchoice{temperature and mass.}
      \correctchoice{mass and luminosity.}
        \wrongchoice{mass and radius.}
    \end{choices}
\end{question}
}

\element{astr}{
\begin{question}{ASTRI-3-Q05}
    Stars evolve because of changes in:
    \begin{choices}
      \correctchoice{chemical composition of the core.}
        \wrongchoice{luminosity of the star.}
        \wrongchoice{mass of the star.}
        \wrongchoice{chemical composition of the surface.}
        \wrongchoice{surface temperature.}
    \end{choices}
\end{question}
}

\element{astr}{
\begin{question}{ASTRI-3-Q06}
    The triple alpha reaction converts \rule[-0.1pt]{4em}{0.1pt} into \rule[-0.1pt]{4em}{0.1pt}.
    \begin{choices}
        \wrongchoice{hydrogen, helium
        \wrongchoice{helium, hydrogen
      \correctchoice{helium, carbon
        \wrongchoice{carbon, nitrogen
        \wrongchoice{hydrogen, nitrogen
    \end{choices}
\end{question}
}

\element{astr}{
\begin{question}{ASTRI-3-Q07}
    A nuclear reaction which occurs at sometime after hydrogen is exhausted in the core is the:
    \begin{choices}
        \wrongchoice{CNO cycle.}
      \correctchoice{triple alpha reaction.}
        \wrongchoice{helium-iron chain.}
        \wrongchoice{electron-positron cycle.}
        \wrongchoice{tritium-alpha cycle.}
    \end{choices}
\end{question}
}

\element{astr}{
\begin{question}{ASTRI-3-Q08}
    The temperature of a star's core will \rule[-0.1pt]{4em}{0.1pt} as the star fuses heavier elements.
    \begin{choices}
      \correctchoice{increase}
        \wrongchoice{decrease}
        \wrongchoice{remain the same}
    \end{choices}
\end{question}
}

\element{astr}{
\begin{question}{ASTRI-3-Q09}
    Which one of the following is \emph{not} a class of star cluster?
    \begin{choices}
        \wrongchoice{galactic clusters}
        \wrongchoice{open clusters}
      \correctchoice{closed clusters}
        \wrongchoice{globular clusters}
        \wrongchoice{stellar associations}
    \end{choices}
\end{question}
}

\element{astr}{
\begin{question}{ASTRI-3-Q10}
    Which of the following is \emph{not} an assumption made in the study of star clusters?
    \begin{choices}
        \wrongchoice{all the stars are at the same distance}
        \wrongchoice{all the stars formed at the same time}
        \wrongchoice{all the stars have the same chemical composition}
      \correctchoice{all the stars have the same mass}
    \end{choices}
\end{question}
}

\element{astr}{
\begin{question}{ASTRI-3-Q11}
    Which of the following can \emph{not} be obtained from a color-magnitude diagram?
    \begin{choices}
        \wrongchoice{a cluster's age}
        \wrongchoice{a cluster's distance}
        \wrongchoice{the relative brightnesses of stars in a cluster}
      \correctchoice{the cluster velocity}
    \end{choices}
\end{question}
}

\element{astr}{
\begin{question}{ASTRI-3-Q12}
    %% NOTE: need diagram
    What would be the correct order of the clusters shown above,
        if they are arranged from youngest to oldest?
    \begin{multicols}{2}
    \begin{choices}
        \wrongchoice{ABC}
        \wrongchoice{ACB}
        \wrongchoice{BAC}
        \wrongchoice{BCA}
      \correctchoice{CAB}
        \wrongchoice{CBA}
    \end{choices}
    \end{multicols}
\end{question}
}

\element{astr}{
\begin{question}{ASTRI-3-Q13}
    The evolution of a star depends primarily on the star's:
    \begin{multicols}{2}
    \begin{choices}
        \wrongchoice{radius.}
      \correctchoice{mass.}
        \wrongchoice{luminosity.}
        \wrongchoice{density.}
        \wrongchoice{temperature.}
    \end{choices}
    \end{multicols}
\end{question}
}

\element{astr}{
\begin{question}{ASTRI-3-Q14}
    Stars on the upper end of the main sequence next evolve into:
    \begin{choices}
        \wrongchoice{red dwarfs.}
        \wrongchoice{lower main sequence stars.}
        \wrongchoice{solar-type stars.}
        \wrongchoice{white dwarfs.}
      \correctchoice{red giants.}
    \end{choices}
\end{question}
}

\element{astr}{
\begin{question}{ASTRI-3-Q15}
    The main sequence turn-off is useful in determining a cluster's:
    \begin{choices}
        \wrongchoice{mass.}
      \correctchoice{age.}
        \wrongchoice{distance.}
        \wrongchoice{apparent magnitude.}
        \wrongchoice{velocity.}
    \end{choices}
\end{question}
}

\element{astr}{
\begin{question}{ASTRI-3-Q16}
    The H-R diagram for cluster $A$ consists of stars on the main sequence,
        while the diagram for cluster $B$ has stars both on the main sequence and just above the main sequence at lower temperatures. 
    Which cluster is older?
    \begin{choices}
      \correctchoice{cluster $A$}
        \wrongchoice{cluster $B$}
        \wrongchoice{both clusters are the same age}
        \wrongchoice{more information is required}
    \end{choices}
\end{question}
}

\element{astr}{
\begin{question}{ASTRI-3-Q17}
    A T Tauri star is one which is:
    \begin{choices}
        \wrongchoice{like the Sun.}
      \correctchoice{variable and shedding mass.}
        \wrongchoice{old and shedding mass.}
        \wrongchoice{becoming a white dwarf.}
        \wrongchoice{a main sequence star.}
    \end{choices}
\end{question}
}

\element{astr}{
\begin{question}{ASTRI-3-Q18}
    Protostars in dark, dusty regions may be studied in the \rule[-0.1pt]{4em}{0.1pt} spectral region.
    \begin{choices}
        \wrongchoice{X-ray}
        \wrongchoice{ultraviolet}
        \wrongchoice{visual}
      \correctchoice{infrared}
        \wrongchoice{gamma-ray}
    \end{choices}
\end{question}
}

\element{astr}{
\begin{question}{ASTRI-3-Q19}
    During the formation of a star,
        the contraction stops when:
    \begin{choices}
        \wrongchoice{the star collapses into a black hole.}
        \wrongchoice{the star collapses into a white dwarf.}
      \correctchoice{hydrogen burning becomes the dominant energy source.}
        \wrongchoice{helium burning becomes the dominant energy source.}
        \wrongchoice{the star becomes a T Tauri star.}
    \end{choices}
\end{question}
}

\element{astr}{
\begin{question}{ASTRI-3-Q20}
    Hydrogen burning for a Sun-like star lasts approximately:
    \begin{choices}
        \wrongchoice{one million years.}
        \wrongchoice{ten million years.}
        \wrongchoice{one hundred million years.}
        \wrongchoice{one billion years.}
      \correctchoice{ten billion years.}
    \end{choices}
\end{question}
}

\element{astr}{
\begin{question}{ASTRI-3-Q21}
    What types of stars are found on the zero-age main sequence (ZAMS)?
    \begin{choices}
        \wrongchoice{red giants}
        \wrongchoice{protostar}
      \correctchoice{newly formed stars}
        \wrongchoice{stars that have not yet started to burn hydrogen}
        \wrongchoice{white dwarfs}
    \end{choices}
\end{question}
}

\element{astr}{
\begin{question}{ASTRI-3-Q22}
    When the hydrogen in the core of a star has been converted to helium,
        the core will next:
    \begin{choices}
      \correctchoice{contract.}
        \wrongchoice{expand.}
        \wrongchoice{burn helium.}
        \wrongchoice{decrease in temperature.}
        \wrongchoice{explode.}
    \end{choices}
\end{question}
}

\element{astr}{
\begin{question}{ASTRI-3-Q23}
    Which of the following occurs during and immediately after the phase of the hydrogen burning shell?
    \begin{choices}
        \wrongchoice{the core shrinks until the star becomes a white dwarf}
        \wrongchoice{the helium flash occurs}
        \wrongchoice{the core temperature decreases while the envelope temperature increases}
        \wrongchoice{the star becomes a supernova}
      \correctchoice{the envelope expands and cools, and the star becomes a red giant}
    \end{choices}
\end{question}
}

\element{astr}{
\begin{question}{ASTRI-3-Q24}
    As a degenerate gas is heated, it will:
    \begin{choices}
        \wrongchoice{expand.}
        \wrongchoice{contract.}
      \correctchoice{neither expand nor contract.}
        \wrongchoice{oscillate.}
    \end{choices}
\end{question}
}

\element{astr}{
\begin{question}{ASTRI-3-Q25}
    The ignition of helium in the degenerate core of a one solar mass star produces:
    \begin{choices}
        \wrongchoice{a supernova.}
        \wrongchoice{a black hole.}
        \wrongchoice{the formation of new heavy elements.}
      \correctchoice{a helium flash.}
        \wrongchoice{nothing much---the core expands, cools, and continues burning.}
    \end{choices}
\end{question}
}

\element{astr}{
\begin{question}{ASTRI-3-Q26}
    In a degenerate electron gas the outward pressure which keeps the star from collapsing is:
    \begin{choices}
        \wrongchoice{dependent upon temperature.}
        \wrongchoice{dependent upon mass.}
      \correctchoice{independent of temperature.}
        \wrongchoice{independent of mass.}
    \end{choices}
\end{question}
}

\element{astr}{
\begin{question}{ASTRI-3-Q27}
    Which of the following is \emph{not} a characteristic of a red giant?
    \begin{choices}
        \wrongchoice{extended outer layers}
        \wrongchoice{degenerate core}
      \correctchoice{cool core}
        \wrongchoice{cool surface temperature}
    \end{choices}
\end{question}
}

\element{astr}{
\begin{question}{ASTRI-3-Q28}
    The horizontal branch is:
    \begin{choices}
        \wrongchoice{a region on the H-R diagram where stars have roughly the same temperature.}
      \correctchoice{a region on the H-R diagram where stars have roughly the same luminosity.}
        \wrongchoice{seen only in young clusters where protostars are evolving towards the main sequence.}
        \wrongchoice{the track that white dwarfs follow.}
    \end{choices}
\end{question}
}

\element{astr}{
\begin{question}{ASTRI-3-Q29}
    After a Sun-like star enters its second red giant phase,
        its internal structure would consist of:
    \begin{choices}
      \correctchoice{a carbon core surrounded by a helium burning shell, which is surrounded by a hydrogen burning shell.}
        \wrongchoice{a helium core surrounded by a hydrogen burning shell.}
        \wrongchoice{a iron core surrounded by many different layers of shell burning.}
        \wrongchoice{a oxygen core surrounded by a carbon burning shell, a helium burning shell, and a hydrogen burning shell.}
    \end{choices}
\end{question}
}

\element{astr}{
\begin{question}{ASTRI-3-Q30}
    A planetary nebula is:
    \begin{choices}
        \wrongchoice{a collapsing gas cloud out of which planets will form.}
      \correctchoice{an expanding gas cloud that was ejected by a star.}
        \wrongchoice{the cloud of gas blown off by a T Tauri type star.}
        \wrongchoice{a gas cloud which goes into the formation of red giants.}
    \end{choices}
\end{question}
}

\element{astr}{
\begin{question}{ASTRI-3-Q31}
    The stellar remnant of a one solar mass star is a:
    \begin{choices}
      \correctchoice{white dwarf.}
        \wrongchoice{neutron star.}
        \wrongchoice{pulsar.}
        \wrongchoice{black hole.}
        \wrongchoice{main sequence star.}
    \end{choices}
\end{question}
}

\element{astr}{
\begin{question}{ASTRI-3-Q32}
    High mass stars evolve more rapidly than low mass ones because the high mass stars:
    \begin{choices}
        \wrongchoice{are larger}
      \correctchoice{have higher core temperatures.}
        \wrongchoice{have higher core densities.}
        \wrongchoice{are made of more massive elements.}
        \wrongchoice{are in the lower right corner of the H-R diagram.}
    \end{choices}
\end{question}
}

\element{astr}{
\begin{question}{ASTRI-3-Q33}
    In the most massive stars,
        the heaviest element which will be produced in the core will be:
    \begin{choices}
        \wrongchoice{helium.}
        \wrongchoice{oxygen.}
        \wrongchoice{silicon.}
      \correctchoice{iron.}
        \wrongchoice{uranium.}
    \end{choices}
\end{question}
}

\element{astr}{
\begin{question}{ASTRI-3-Q34}
    Each time a form of nuclear fuel is exhausted in the core of a star,
        the star:
    \begin{choices}
        \wrongchoice{returns to the main sequence.}
      \correctchoice{returns to the red giant branch.}
        \wrongchoice{returns to the white dwarf region.}
        \wrongchoice{explodes in a supernova explosion.}
        \wrongchoice{ejects a planetary nebula.}
    \end{choices}
\end{question}
}

\element{astr}{
\begin{question}{ASTRI-3-Q35}
    The most mass a white dwarf can have is about:
    \begin{choices}
        \wrongchoice{1 Solar Mass.}
      \correctchoice{1.4 Solar Mass.}
        \wrongchoice{3 Solar Masses.}
        \wrongchoice{10 Solar Masses.}
        \wrongchoice{there is no limit to the mass.}
    \end{choices}
\end{question}
}

\element{astr}{
\begin{question}{ASTRI-3-Q36}
    What element is observed in the spectra of Type II supernova?
    \begin{choices}
      \correctchoice{hydrogen}
        \wrongchoice{helium}
        \wrongchoice{iron}
        \wrongchoice{uranium}
    \end{choices}
\end{question}
}

\element{astr}{
\begin{question}{ASTRI-3-Q37}
    A star which has a main sequence mass of 10 solar masses will most likely end up as:
    \begin{choices}
        \wrongchoice{a T Tauri star.}
        \wrongchoice{a white dwarf.}
      \correctchoice{a neutron star.}
        \wrongchoice{a black hole.}
    \end{choices}
\end{question}
}

\element{astr}{
\begin{question}{ASTRI-3-Q38}
    If a neutron star has more mass than its mass limit, it will:
    \begin{choices}
        \wrongchoice{expand catastrophically.}
      \correctchoice{contract catastrophically.}
        \wrongchoice{begin a new phase of nuclear reactions.}
        \wrongchoice{nothing will happen.}
    \end{choices}
\end{question}
}

\element{astr}{
\begin{question}{ASTRI-3-Q39}
    A black hole is called that because:
    \begin{choices}
        \wrongchoice{its color is black.}
        \wrongchoice{its energy output depends on its temperature.}
      \correctchoice{photons can not be emitted from it.}
        \wrongchoice{they exist only in the dark reaches of space where there are no stars.}
    \end{choices}
\end{question}
}

\element{astr}{
\begin{question}{ASTRI-3-Q40}
    A neutron star's size is that of:
    \begin{choices}
        \wrongchoice{the Sun.}
        \wrongchoice{the Earth.}
        \wrongchoice{the Earth's orbit.}
        \wrongchoice{the State of Iowa.}
      \correctchoice{a typical city.}
    \end{choices}
\end{question}
}

\element{astr}{
\begin{question}{ASTRI-3-Q41}
    The most massive stars are thought to end up as:
    \begin{choices}
      \correctchoice{black holes.}
        \wrongchoice{planetary nebulae.}
        \wrongchoice{neutron stars.}
        \wrongchoice{white dwarfs.}
    \end{choices}
\end{question}
}

\element{astr}{
\begin{question}{ASTRI-3-Q42}
    Heavy elements which are mixed into the material from which new generations of stars may come primarily from:
    \begin{choices}
        \wrongchoice{the big bang.}
        \wrongchoice{planetary nebulae.}
      \correctchoice{supernovae.}
        \wrongchoice{super-neutron stars.}
        \wrongchoice{Wolf-Rayet stars.}
    \end{choices}
\end{question}
}

\element{astr}{
\begin{question}{ASTRI-3-Q43}
    After the initial outburst of Supernova 1987A had diminished,
        the main source of luminosity was energy released from
    \begin{choices}
      \correctchoice{radioactive decay.}
        \wrongchoice{the neutron star formed during the explosion.}
        \wrongchoice{the high temperature core which remains.}
        \wrongchoice{the expanding envelope.}
        \wrongchoice{the ejection of slowing moving neutrinos.}
    \end{choices}
\end{question}
}

\element{astr}{
\begin{question}{ASTRI-3-Q44}
    What was the most energetic component of Supernova 1987A?
    \begin{choices}
        \wrongchoice{the visible light}
        \wrongchoice{the kinetic energy of the expanding gas cloud}
      \correctchoice{the energy released in neutrinos}
        \wrongchoice{the X-ray energy}
    \end{choices}
\end{question}
}

\element{astr}{
\begin{question}{ASTRI-3-Q45}
    Which of the following objects would have the highest gravitational redshift?
    \begin{choices}
        \wrongchoice{a white dwarf}
        \wrongchoice{the Sun}
        \wrongchoice{a neutron star}
      \correctchoice{a black hole}
    \end{choices}
\end{question}
}

\element{astr}{
\begin{question}{ASTRI-3-Q46}
    A white dwarf will cool to become a black dwarf in several:
    \begin{choices}
        \wrongchoice{thousand years.}
        \wrongchoice{million years.}
      \correctchoice{billion years.}
    \end{choices}
\end{question}
}

\element{astr}{
\begin{question}{ASTRI-3-Q47}
    White dwarfs are composed mostly of:
    \begin{choices}
        \wrongchoice{normal (perfect) gases.}
      \correctchoice{degenerate gases.}
        \wrongchoice{equal amounts of normal (perfect) and degenerate gases.}
        \wrongchoice{hot, solid material.}
    \end{choices}
\end{question}
}

\element{astr}{
\begin{question}{ASTRI-3-Q48}
    All novae are thought to involve a:
    \begin{choices}
      \correctchoice{white dwarf.}
        \wrongchoice{main sequence star.}
        \wrongchoice{supergiant.}
        \wrongchoice{neutron star.}
        \wrongchoice{black hole.}
    \end{choices}
\end{question}
}

\element{astr}{
\begin{question}{ASTRI-3-Q49}
    Novae explosions are caused by:
    \begin{choices}
        \wrongchoice{exploding white dwarfs.}
        \wrongchoice{interstellar matter falling onto the surface of a star, usually a white dwarf.}
        \wrongchoice{material falling into a black hole.}
      \correctchoice{mass lost from a normal star falling onto a white dwarf companion.}
        \wrongchoice{strong flares on a stellar surface (large-scale ``solar'' flares).}
    \end{choices}
\end{question}
}

\element{astr}{
\begin{question}{ASTRI-3-Q50}
    How can astronomers determine which type of supernovae they are observing?
    \begin{choices}
        \wrongchoice{Type I supernovae fade much more quickly than Type II}
        \wrongchoice{the brightnesses are different}
      \correctchoice{the spectral features are different}
        \wrongchoice{by determining exactly which object produced the supernova}
    \end{choices}
\end{question}
}

\element{astr}{
\begin{question}{ASTRI-3-Q51}
    The internal properties of a neutron star are most similar to those of a \rule[-0.1pt]{4em}{0.1pt} star.
    \begin{choices}
        \wrongchoice{main sequence}
        \wrongchoice{red giant}
        \wrongchoice{red dwarf}
      \correctchoice{white dwarf}
        \wrongchoice{solar-type}
        \wrongchoice{blue supergiant}
    \end{choices}
\end{question}
}

\element{astr}{
\begin{question}{ASTRI-3-Q52}
    Stellar remnants with masses between 2 and 3 solar masses will be:
    \begin{choices}
        \wrongchoice{white dwarfs.}
      \correctchoice{neutron stars.}
        \wrongchoice{black holes.}
        \wrongchoice{planetary nebulae.}
    \end{choices}
\end{question}
}

\element{astr}{
\begin{question}{ASTRI-3-Q53}
    Pulsars are known to be:
    \begin{choices}
        \wrongchoice{pulsating white dwarfs.}
        \wrongchoice{pulsating neutron stars.}
        \wrongchoice{rotating white dwarfs.}
      \correctchoice{rotating neutron stars.}
        \wrongchoice{rotating black holes.}
    \end{choices}
\end{question}
}

\element{astr}{
\begin{question}{ASTRI-3-Q54}
    Binary X-ray sources are known to be binary because
    \begin{choices}
        \wrongchoice{the two stars are observed visually as visual binary stars.}
        \wrongchoice{they are astrometric binaries.}
      \correctchoice{eclipses are observed.}
        \wrongchoice{the name is a misnomer since no X-ray objects are known to be binary.}
    \end{choices}
\end{question}
}

\element{astr}{
\begin{question}{ASTRI-3-Q55}
    The accretion disk surrounding a neutron star is very hot due to compression caused by gravitational forces. 
      This implies the object will emit strongly in which spectral region?
    \begin{choices}
      \correctchoice{X ray}
        \wrongchoice{ultraviolet}
        \wrongchoice{visual}
        \wrongchoice{infrared}
        \wrongchoice{radio}
    \end{choices}
\end{question}
}

\element{astr}{
\begin{question}{ASTRI-3-Q56}
    If the Sun were suddenly to be replaced by a solar-mass black hole the gravitational force on the Earth (1 A. U. away) would:
    \begin{choices}
        \wrongchoice{double.}
        \wrongchoice{become so strong that the Earth would be ``sucked'' into the black hole.}
        \wrongchoice{decrease because black holes cause gravity at large distances to disappear.}
      \correctchoice{remain the same.}
    \end{choices}
\end{question}
}

\element{astr}{
\begin{question}{ASTRI-3-Q57}
    The Schwarzschild radius of a black hole is:
    \begin{choices}
        \wrongchoice{the radius of the star when it is on the main sequence.}
      \correctchoice{the distance from a black hole inside of which light cannot escape.}
        \wrongchoice{the theoretical size of the smallest possible white dwarf.}
        \wrongchoice{the size of a star when it begins hydrogen burning just prior to reaching the main sequence.}
        \wrongchoice{the size of the early protosun.}
    \end{choices}
\end{question}
}

\element{astr}{
\begin{question}{ASTRI-3-Q58}
    From the outsider's point of view,
        in watching a star collapse to form a black hole,
        the collapse would appear to take:
    \begin{choices}
        \wrongchoice{only a fraction of a second.}
        \wrongchoice{a few hours.}
      \correctchoice{forever.}
    \end{choices}
\end{question}
}

\element{astr}{
\begin{question}{ASTRI-3-Q59}
    Which of the following lists the stellar remnants in order of decreasing maximum mass?
    \begin{choices}
        \wrongchoice{neutron star, white dwarf, black hole}
      \correctchoice{black hole, neutron star, white dwarf}
        \wrongchoice{white dwarf, black hole, neutron star}
        \wrongchoice{black hole, white dwarf, neutron star}
        \wrongchoice{they all have approximately the same mass}
    \end{choices}
\end{question}
}

\element{astr}{
\begin{question}{ASTRI-3-Q60}
    A black hole is really:
    \begin{choices}
        \wrongchoice{a star of temperature \SI{0}{\kelvin}.}
        \wrongchoice{densely packed matter.}
      \correctchoice{strongly curved space.}
        \wrongchoice{at the center of most stars and provides the star's energy.}
    \end{choices}
\end{question}
}

\element{astr}{
\begin{question}{ASTRI-3-Q61}
    How is it possible to detect the presence of a black hole?
    \begin{choices}
        \wrongchoice{using a large optical telescope to see its surface}
        \wrongchoice{using a large radio telescope to see its surface}
      \correctchoice{by its effect upon other objects around it}
        \wrongchoice{by watching material cross its event horizon}
    \end{choices}
\end{question}
}
      

% Answer Key for Sample Exam 3
% 1. a  
% 2. a  
% 3. e  
% 4. d  
% 5. a  
% 6. c  
% 7. b  
% 8. a  
% 9. c
% 10. d   
% 11. d   
% 12. e   
% 13. b   
% 14. e   
% 15. b    
% 16. a   
% 17. b        
% 18. d        
% 19. c        
% 20. e        
% 21. c        
% 22. a        
% 23. e        
% 24. c         
% 25. d         
% 26. c         
% 27. c         
% 28. b         
% 29. a         
% 30. b     
% 31. a         
% 32. b         
% 33. d         
% 34. b         
% 35. b         
% 36. a         
% 37. c         
% 38. b         
% 39. c         
% 40. e         
% 41. a         
% 42. c         
% 43. a    
% 44. c   
% 45. d   
% 46. c   
% 47. b   
% 48. a   
% 49. d   
% 50. c   
% 51. d   
% 52. b   
% 53. d   
% 54. c   
% 55. a   
% 56. d   
% 57. b   
% 58. c   
% 59. b
% 60. c   
% 61. c    


\endinput


