
%% University of Northern Iowa
%%  Morgans Astronomy Exams
%%--------------------------------------------------

%% this section contains 15 problems

\element{morgans}{
\begin{question}{test2F-Q01}
    The proton-proton chain produces enough energy to keep the Sun shining for:
    \begin{choices}
        \wrongchoice{thousands of years.}
        \wrongchoice{hundreds of thousands of years.}
        \wrongchoice{millions of years.}
        \wrongchoice{hundreds of millions of years.}
      \correctchoice{billions of years.}
    \end{choices}
\end{question}
}

\element{morgans}{
\begin{question}{test2F-Q02}
    High temperatures are required for hydrogen fusion reactions to occur because:
    \begin{choices}
        \wrongchoice{high potential energy is required.}
      \correctchoice{like charges repel.}
        \wrongchoice{neutrinos can be explained only with high temperatures.}
        \wrongchoice{opposite charges attract.}
    \end{choices}
\end{question}
}

\element{morgans}{
\begin{question}{test2F-Q03}
    The solar wind is mostly associated with:
    \begin{choices}
        \wrongchoice{sunspots.}
        \wrongchoice{prominences.}
      \correctchoice{coronal holes.}
        \wrongchoice{Jupiter's magnetic field.}
    \end{choices}
\end{question}
}

\element{morgans}{
\begin{question}{test2F-Q04}
    Which of the following is \emph{not} a reason to determine stellar positions accurately?
    \begin{choices}
        \wrongchoice{stellar motions can tell us about the motions of the galaxy}
        \wrongchoice{accurate positions are needed for telescope guidance}
      \correctchoice{accurate positions are needed for mass determinations}
        \wrongchoice{stellar motions can be used to make parallax measurements}
    \end{choices}
\end{question}
}

\element{morgans}{
\begin{question}{test2F-Q05}
    The observed differences between stellar spectra are caused primarily by differences in stellar:
    \begin{choices}
        \wrongchoice{luminosity.}
        \wrongchoice{chemical composition.}
      \correctchoice{temperature.}
        \wrongchoice{motions.}
        \wrongchoice{location in space.}
    \end{choices}
\end{question}
}

\element{morgans}{
\begin{question}{test2F-Q06}
    Star $A$ has an apparent magnitude of $+5.0$,
        while Star $B$ has an apparent magnitude of $+4.0$. 
    Which star is more luminous?
    \begin{choices}
        \wrongchoice{A}
        \wrongchoice{B}
        \wrongchoice{they have the same luminosity}
      \correctchoice{not possible to determine from the information given}
    \end{choices}
\end{question}
}

\element{morgans}{
\begin{question}{test2F-Q07}
    Very luminous stars with low temperatures:
    \begin{choices}
        \wrongchoice{have small diameters compared with that of the Sun.}
        \wrongchoice{have diameters comparable with that of the Sun.}
        \wrongchoice{have large diameters compared with that of the Sun.}
        \wrongchoice{are white dwarfs.}
    \end{choices}
\end{question}
}

\element{morgans}{
\begin{question}{test2F-Q08}
    Stellar masses are most easily determined with use of:
    \begin{choices}
        \wrongchoice{Newton's first law.}
        \wrongchoice{Newton's second law}
        \wrongchoice{Kepler's second law.}
      \correctchoice{Kepler's third law}
    \end{choices}
\end{question}
}

\element{morgans}{
\begin{question}{test2F-Q09}
    The fundamental quantity which determines a star's central pressure and temperature is its:
    \begin{choices}
      \correctchoice{mass.}
        \wrongchoice{luminosity.}
        \wrongchoice{surface temperature.}
        \wrongchoice{chemical composition.}
        \wrongchoice{radius.}
    \end{choices}
\end{question}
}

\element{morgans}{
\begin{question}{test2F-Q10}
    The triple alpha reaction converts \rule[-0.1pt]{4em}{0.1pt} into \rule[-0.1pt]{4em}{0.1pt}.
    \begin{choices}
        \wrongchoice{hydrogen, helium}
        \wrongchoice{helium, hydrogen}
      \correctchoice{helium, carbon}
        \wrongchoice{carbon, nitrogen}
        \wrongchoice{hydrogen, nitrogen}
    \end{choices}
\end{question}
}

\element{morgans}{
\begin{question}{test2F-Q11}
    The evolution of a star depends primarily on the star's:
    \begin{choices}
        \wrongchoice{radius.}
      \correctchoice{mass.}
        \wrongchoice{luminosity.}
        \wrongchoice{density.}
        \wrongchoice{temperature.}
    \end{choices}
\end{question}
}

\element{morgans}{
\begin{question}{test2F-Q12}
    A T Tauri star is one which is:
    \begin{choices}
        \wrongchoice{like the Sun.}
      \correctchoice{variable and shedding mass.}
        \wrongchoice{old and shedding mass.}
        \wrongchoice{becoming a white dwarf.}
        \wrongchoice{a main sequence star.}
    \end{choices}
\end{question}
}

\element{morgans}{
\begin{question}{test2F-Q13}
    A star like the Sun will have \rule[-0.1pt]{4em}{0.1pt} percent
        of its mass in the form of heavy elements (elements heavier than hydrogen or helium).
    \begin{choices}
      \correctchoice{1 or 2 (traces of heavier elements)}
        \wrongchoice{10}
        \wrongchoice{30}
        \wrongchoice{50}
        \wrongchoice{70}
    \end{choices}
\end{question}
}

\element{morgans}{
\begin{question}{test2F-Q14}
    As a degenerate gas is heated, it will:
    \begin{choices}
        \wrongchoice{expand.}
        \wrongchoice{contract.}
        \wrongchoice{neither expand nor contract.}
        \wrongchoice{oscillate.}
    \end{choices}
\end{question}
}

\element{morgans}{
\begin{question}{test2F-Q15}
    A white dwarf has a radius the size of:
    \begin{choices}
        \wrongchoice{the Sun's radius.}
        \wrongchoice{the orbit of the Earth.}
      \correctchoice{the Earth's radius.}
        \wrongchoice{a large city.}
    \end{choices}
\end{question}
}

\element{morgans}{
\begin{question}{test2F-Q16}
    What is the Chandrasekhar limit?
    \begin{choices}
        \wrongchoice{the maximum mass that a main sequence star can have}
        \wrongchoice{the maximum mass that a red giant can have}
      \correctchoice{the maximum mass that a white dwarf can have}
        \wrongchoice{the maximum mass that a neutron star can have}
    \end{choices}
\end{question}
}

\element{morgans}{
\begin{question}{test2F-Q17}
    The observations of Supernova 1987A has proven that:
    \begin{choices}
        \wrongchoice{supernova do not produce heavy elements.}
      \correctchoice{neutrinos are produced in supernova.}
        \wrongchoice{white dwarf stars can explode into supernovae.}
        \wrongchoice{only stars in our galaxy become supernovae.}
    \end{choices}
\end{question}
}

\element{morgans}{
\begin{question}{test2F-Q18}
    Novae explosions are caused by:
    \begin{choices}
        \wrongchoice{exploding white dwarfs.}
        \wrongchoice{interstellar matter falling onto the surface of a star, usually a white dwarf.}
        \wrongchoice{material falling into a black hole.}
      \correctchoice{mass lost from a normal star falling onto a white dwarf companion.}
        \wrongchoice{strong flares on a stellar surface (large-scale ``solar'' flares).}
    \end{choices}
\end{question}
}

\element{morgans}{
\begin{question}{test2F-Q19}
    The rapid changes in the radiation received from a pulsar are caused by rapid:
    \begin{choices}
        \wrongchoice{temperature changes.}
        \wrongchoice{size changes which produce rapid luminosity changes.}
        \wrongchoice{temperature changes which produce rapid luminosity changes.}
      \correctchoice{changes in the magnetic properties of the star.}
        \wrongchoice{changes in the viewing angle of the magnetic pole.}
    \end{choices}
\end{question}
}

\element{morgans}{
\begin{question}{test2F-Q20}
    The Schwarzschild radius of a black hole is:
    \begin{choices}
        \wrongchoice{the radius of the star when it is on the main sequence.}
      \correctchoice{the distance from a black hole inside of which light cannot escape.}
        \wrongchoice{the theoretical size of the smallest possible white dwarf.}
        \wrongchoice{the size of a star when it begins hydrogen burning just prior to reaching the main sequence.}
        \wrongchoice{the size of the early protosun.}
    \end{choices}
\end{question}
}


\begin{comment}
    Fill In
    Place the most appropriate word or words in the blank. You may have to click on the blank to activate it before you start typing in your answer.
     
    The specific name for a hydrogen atom having a proton and a neutron is .

    The specific temperature of the center of the Sun is K.

    The maximum parallax observed for a star is approximately seconds of arc.

    Masses of stars on the main sequence (increase/decrease) with increasing temperature.

    is the energy transport mechanism in the outer layers of the Sun's interior.

    The main sequence in the H-R diagram formed by stars that have just begun their hydrogen-burning life-times, and have not yet converted any significant fraction of their core mass into helium; it forms the lower left boundary of the main sequence and is called the .

    Protostars in dark, dusty regions may be studied in the spectral region.

    A star with a main sequence mass of 5 solar masses will most likely end up as a(n) .

    Radiation which is associated with magnetic fields and is non-thermal in origin is called radiation.

    A very compact, dense stellar remnant having a mass between 1.5 and 3 M is called a .
\end{comment}


