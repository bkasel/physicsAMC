
%% Astronomy I Practice Exam 1
%%--------------------------------------------------

%% this section contains 16 problems

\element{astr}{
\begin{question}{ASTRI-1-Q01}
    Measurements made on the celestial sphere are made in units of:
    \begin{multicols}{2}
    \begin{choices}
        \wrongchoice{miles.}
        \wrongchoice{kilometers.}
      \correctchoice{degrees.}
        \wrongchoice{light years.}
    \end{choices}
    \end{multicols}
\end{question}
}


SAMPLE EXAM 2: FOR Chapters 16, 17, & 18 in Chaisson

1. The masses of stars on the main sequence __________ from the lower right to the upper left.
    a. increase
    b. decrease
    c. are all the same
    d. are randomly distributed

2. The fundamental quantity which determines a star's central pressure and temperature is its
    a. mass.
    b. luminosity.
    c. surface temperature.
    d. chemical composition.
    e. radius.

3. The determination of stellar parallax is important because it allows the direct determination of
    a. mass.
    b. distance.
    c. diameter.
    d. velocity.

4. The observed differences between stellar spectra are caused primarily by differences in stellar
    a. luminosity.
    b. chemical composition.
    c. temperature.
    d. motions.
    e. location in space.

5. Which of the following is the correct order for the spectral classification system?
    a. OBAFKGM
    b. OBFAGKM
    c. OBAFGKM
    d. ABFGKMO

6. If a stellar spectrum shows strong lines produced by ionized helium, the star is of spectral type
    a. O.
    b. B.
    c. A.
    d. F.
    e. M.

7. The reason astronomers use the concept of the absolute magnitude is to
    a. make life difficult for introductory astronomy students!
    b. allow stars to be compared with the effects of differing distance removed
    c. allow stars to be compared with the effects of differing mass removed
    d. allow stars to be compared with the effects of differing temperature removed
    e. allow stars to be compared with the effects of differing radius removed.

8. Which of the following is NOT by itself useful for determining stellar temperature?
    a. spectral class
    b. color index
    c. absolute magnitude
    d. degree of ionization
    e. wavelength of maximum intensity of the underlying spectrum

9. The purpose of the spectroscopic parallax technique is to determine
    a. distance.
    b. temperature.
    c. apparent magnitude.
    d. bolometric magnitude.
    e. color index.

10. An H-R diagram is a plot of
    a. heat versus radius.
    b. luminosity versus radius.
    c. mass versus temperature.
    d. luminosity versus temperature.
    e. mass versus luminosity.

11. Very luminous stars with low temperatures
    a. have small diameters compared with that of the Sun.
    b. have diameters comparable with that of the Sun.
    c. have large diameters compared with that of the Sun.
    d. are white dwarfs.

12. The interior of the Sun is a
      a.  gas.
      b.  liquid.
      c.  solid.
       
13. The temperature of the center of the Sun is roughly
      a.  10,000 K.
      b.  100,000 K.
      c.  1,000,000 K.
      d.  10,000,000 K.
      e.  100,000,000 K.
       
14. How does a gamma-ray photon, produced in the core of the Sun, emerge at the surface as a visible light photon?
      a.  it loses energy through absorptions and re-emissions
      b.  it gains energy through absorptions and re-emissions
      c.  it does not actually change, all photons from the surface of the Sun are gamma-ray photons
      d.  it is not a gamma-ray photon, since all photons produced in the core of the Sun are visible light photons
       
15. The two most abundant elements in the Sun, with the most abundant given first, are
      a.  carbon and oxygen.
      b.  iron and hydrogen.
      c.  helium and nitrogen.
      d.  nitrogen and helium.
      e.  hydrogen and helium.
       
16. Hydrostatic equilibrium is an equilibrium between
      a.  hydrogen and carbon.
      b.  water, hydrogen, and oxygen.
      c.  gravity and outward pressure.
      d.  water and electric (static) charge.
       
 17. Fusion in the Sun occurs between
      a.  protons.
      b.  electrons and protons.
      c.  electrons and neutrons.
      d.  neutrons and protons.
      e.  electrons.
       
 18. What we see visually as the surface of the Sun is the
      a.  photosphere.
      b.  chromosphere.
      c.  corona.
       
 19. The temperature of the solar corona is approximately
      a.  5,000 K.
      b.  10,000 K.
      c.  50,000 K.
      d.  100,000 K.
      e.  1,000,000 K.
       
 20. The absorption line spectrum of the Sun comes from the
      a.  interior.
      b.  photosphere.
      c.  chromosphere.
      d.  corona.
       
 21. The solar wind is mostly associated with
      a.  sunspots.
      b.  prominences.
      c.  coronal holes.
      d.  Jupiter's magnetic field.
       
 22. A complete period of the solar magnetic field cycle (including polarity) takes
      a.  one year.
      b.  11 years.
      c.  22 years.
      d.  33 years.
      e.  44 years.
       
 23. Solar oscillations are studied by means of
      a.  observing periodic changes in the Sun's angular size.
      b.  Doppler shift observations.
      c.  observing strong color changes in the Sun.
      d.  solar eclipses.
       
 24. Which of the following statements concerning solar neutrinos is correct?
      a.  less have been observed than predicted
      b.  observed and predicted numbers are equal
      c.  more have been observed than predicted

 25. Stellar diameters may be determined from studies of __________ stars.
    a. visual binary
    b. astrometric binary
    c. spectroscopic binary
    d. eclipsing binary

 26. From knowledge of only a star's temperature and luminosity, we can determine its
    a. mass.
    b. radius.
    c. distance.
    d. period of rotation.
    e. rotational velocity. 

 27. Stellar masses are most easily determined with use of
    a. Newton's first law.
    b. Newton's second law as modified by Kepler.
    c. Kepler's second law.
    d. Kepler's third law as modified by Newton.

 28. The mass-luminosity relation for main sequence stars says:
    a. high mass, high luminosity.
    b. high mass, low luminosity.
    c. luminosity is constant for all masses.
    d. luminosity is independent of mass.

 29. What is the average temperature of interstellar gas and dust?
    a. 3 K
    b. 100 K
    c. 1,000 K
    d. 0 K

 30. Which two ingredients are needed to make an emission nebula or HII region?
    a.  Interstellar gas and dust
    b. Cool stars and interstellar dust
    c. Hot stars and interstellar gas
    d. Cool stars and interstellar gas


ANSWERS TO PRACTICE EXAM 2

1. a
2. a
3. b
4. c
5. c
6. a
7. b
8. c
9. a
10. d
11. c
12. a
13. d
14. a
15. e
16. c
17. a
18. a
19. e
20. b
21. c
22. c
23. b
24. a
25. d
26. b
27. d
28. a
29. b
30. c

