

1. Relative to the horizon, as seen from the earth's southern hemisphere, the sun daily

    rises in the east and sets in the west.
    rises in the west and sets in the east.
    moves mainly in a northward direction.
    moves mainly in a southward direction. 

Answer: A, Even from the earth's southern hemisphere, the entire sky turns once a day from east to west RELATIVE TO THE HORIZON.

Level: Near transfer

2.When Venus has reached its maximum eastern elongation from the sun as viewed from the earth, it is visible in the sky

    in opposition to the sun.
    as an evening "star".
    as a morning "star".
    in conjunction with the sun. 

Answer: B, When Venus lies east of the, it sets after the sun sets. So it is an evening "star".

Level: Near transfer

3. As seen from the earth, when Jupiter undergoes its retrograde motion, it is __________ the sun in the sky and moving __________ with respect to the stars.

    in conjunction/eastward
    in conjunction/westward
    in opposition/eastward
    in opposition/westward 

Answer: D, Jupiter must be in opposition to the sun to undergo its retrograde motion. During this time, it moves westward (east=>west) relative to the stars.

Level: Recall

4. In order to have a solar eclipse, you need to have

    a full moon.
    a new moon.
    the moon on or close to the ecliptic.
    (A) and (C).
    (B) and (C). 

Answer: E, Both a new moon and the moon on or close to the ecliptic.

Level: Recall

5. One evening, just after sunset, you see Mars, Jupiter, and Saturn spread out across the sky. How could you trace out the rough position of the ecliptic in the sky?

    Draw an imaginary arc through the sunset point and Jupiter.
    Draw an imaginary arc through the sunset point and the three visible planets.
    Wait for the moon to rise later; it lies right on the ecliptic.
    Trace an arc around the horizon below each visible planet. 

Answer: B, The planets move through the zodiac on or near the ecliptic. So if you traced an imaginary arc from the sunset point through the visible planets, you would have a rough idea of the location of the ecliptic.

Level: Near transfer

Indicate whether the statements 6-8 about astronomical events seen from the earth are: always true (answer A), sometimes true (answer B), or never true (answer C)

6. Saturn is rising as the sun sets.

    Always true
    Sometimes true
    Never true 

Answer: B

Level: Near transfer

7. A month after an eclipse of the moon, the moon is full.

    Always true
    Sometimes true
    Never true 

Answer: A

Level: Near transfer

8. Planets move eastward with respect to the stars of the zodiac.

    Always true
    Sometimes true
    Never true 

Answer: B

Level: Near transfer

9. Which of the following statements about scientific models is false?

    They are testable by observations.
    They are aesthetically pleasing by the standards of the time.
    They use physical ideas to explain phenomena.
    They contain geometrical elements.
    They do not change in the light of new observations. 

Answer: E, Scientific models must be subject to change.

Level: Recall

10. According to the physical ideas of a basic geocentric model, the motions of the celestial bodies took place because of

    a force between the planets and the sun.
    a force between the planets and the earth.
    the natural motion of the earth in space.
    the natural spinning motion of the celestial spheres. 

Answer: D, The natural motion (no forces) of celestial spheres to spin.

Level: Recall

11. In a basic geocentric model of the cosmos, the daily motion of the sky from east to west was explained by the

    rotation of the earth from west to east.
    rotation of all the heavenly spheres from west to east.
    rotation of all the heavenly spheres from east to west.
    motion of the sun's sphere yearly from east to west. 

Answer: C, The earth had no motions in the geocentric model, so the sky had to rotate once a day from east to west (clockwise) to account for that motion relative to the horizon.

Level: Near transfer

12. In a basic geocentric model, the monthly motion of the moon eastward with respect to the stars of the zodiac is explained by moving the

    entire celestial sphere westward in one day.
    moon eastward around the earth once a day.
    moon eastward around the earth once a year.
    moon eastward around the earth once a month. 

Answer: D, In the geocentric model, the earth is stationary. So to explain the eastward motion of the moon relative to the stars, the moon moves eastward (counterclockwise) once a month.

Level: Near transfer

13. In a basic geocentric model, the annual motion of the sun eastward with respect to the stars of the zodiac is explained by moving the

    entire celestial sphere westward in one day.
    sun eastward around the earth once a day.
    sun eastward around the earth once a year.
    sun westward around the earth once a year. 

Answer: C, The sun was seen as attached to a celestial sphere that turned west to east (counterclockwise) once in a year to explain the annual motion of the sun relative to the stars of the zodiac.

Level: Near transfer

14. Consider that you are stranded on a desert island and have lots of time for naked-eye observations. You see three planets complete one circuit (360 degrees) relative to the stars. The durations of these motions are:

    Planet A: 2 years
    Planet B: 1 year
    Planet C: 12 years

In a basic geocentric model, you can estimate the relative distances of the planets from the earth and the correct order from nearest to farthest is:

    A, B, C
    C, B, A
    B, A, C
    B, C, A 

Answer: C, Use angular speeds; the faster are the closer, the slower, farther away. So the most distant planet takes the longest time to circuit the sky; the least distance, the shortest time.

Level: Near transfer

15. A basic geocentric model predicts that stellar parallax occurs

    once a year.
    once a month.
    once a day.
    not at all. 

Answer: D, A geocentric model predicts no stellar parallax, because the earth does not move.

Level: Near transfer

Indicate whether the statements 16-18 about astronomical events seen from Mars in a geocentric model are: always true (answer A), sometimes true (answer B), or never true (answer C)

16. The earth undergoes retrograde motion.

    Always true.
    Sometimes true.
    Never true. 

Answer: C, Never true; the earth does not have a small circle (epicycle).

Level: Far transfe

17. Mercury is in opposition to the sun.

    Always true.
    Sometimes true.
    Never true. 

Answer: C, Never true, Mercury must stay close to the sun.

Level: Far transfer

18. Saturn is at opposition during retrograde motion.

    Always true.
    Sometimes true.
    Never true. 

Answer: A, Always true; Saturn must retrograde at opposition as seen from Mars.

Level: Far transfer

19. In a heliocentric model, the earth rotates eastward once a day to account for the

    sun's motion eastward through the zodiac.
    retrograde motion of the planets.
    moon's motion eastward through the zodiac.
    westward motion of the entire sky daily with respect to the horizon.
    eastward motion of the entire sky daily with respect to the horizon. 

Answer: D, The earth's eastward rotation makes the sky appear to turn westward relative to the horizon.

Level: Near transfer

20. In a heliocentric model, the observed annual eastward motion of the sun relative to the stars of the zodiac is explained by the

    daily westward rotation of the earth on its axis.
    yearly eastward revolution of the earth around the sun.
    daily eastward revolution of the earth around the sun.
    monthly eastward revolution of the earth around the sun. 

Answer: B, The earth revolves eastward (west to east) YEARLY around the sun.

Level: Near transfer

21. In a heliocentric model, simplicity is achieved by

    placing the sun in the center of the cosmos.
    having the earth move around the sun.
    placing the stars far away compared to the earth-sun distance.
    Both (A) and (B).
    Both (B) and (C). 

Answer: D, By having the sun in the center of the cosmos AND the earth revolving around the sun.

Level: Near transfer

22. In comparing the basic geocentric and heliocentric solar system models, naked-eye observations ALONE show

    that the heliocentric model is the better, in terms predicting observations.
    that the geocentric model is the better, in terms of predicting observations.
    no difference between the two models, in terms of predicting observations.
    one model is more beautiful than the other, and that one is the heliocentric one.
    that we need to devise an alternative to just these two models. 

Answer: C, By naked-eye observations alone, we cannot tell the difference between the two models.

Level: Near transfer

23. According to Kepler's first and third laws, a planet positioned between the earth and Venus and orbiting the sun would move

    more slowly than the earth and retrograde at opposition.
    in a circular orbit and go the fastest when closest to the sun.
    in an elliptical orbit and have an orbital period of less than one year.
    in an elliptical orbit and go the fastest at aphelion. 

Answer: C, The orbit would be elliptical and the period less than one year.

Level: Near transfer

    24. Kepler's model for the solar system discarded ideas that was (were) common to the classical heliocentric models. It (They) wereCircular paths for celestial objects.
    Uniform speeds of the planets in their motions about circles.
    The earth was at the center of the solar system.
    Both (A) and (B)
    Both (B) and (C) 

Answer: D, Kepler had elliptical orbits and non-uniform speeds along them. So he discarded both circles and uniform motion.

Level: Near transfer

25. Observed from the sun, Jupiter takes about 12 years to circuit the zodiac once eastward. Now imagine that you lived on Jupiter and observed the sun. It would take the sun _______ to circuit the zodiac once in a(n) __________ direction in a heliocentric model.

    1 year/eastward
    1 year/westward
    12 years/eastward
    12 years/westward 

Answer: C, As seen from Jupiter, the earth would circuit the zodiac eastward in 12 years. The situation is symmetrical in a heliocentric model.

Level: Far transfer

26. As seen from the earth, it takes Saturn about 30 years to circuit the zodiac once and the time interval between retrograde motions is about 378 days. Imagine that you lived on Saturn and observed the earth. In a heliocentric model, the time interval between retrograde motions of the earth would be

    1 year (365 days).
    30 years.
    378 days.
    Earth does not retrograde. 

Answer: C, Same as that of Saturn as seen from the earth; situation is symmetrical.

Level: Far transfer

27. A heliocentric model of the solar system predicts that stellar parallax

    does not occur at all.
    occurs on an annual basis and is large enough to be seen by the naked eye.
    occurs on an annual basis and is too small to be seen by the naked eye.
    occurs on a daily basis and is too small to be seen by the naked eye. 

Answer: C, In a heliocentric model, the earth moves around the sun, causing parallax to occur on an annual basis. But the angular amount is too small to be seen by the naked eye.

Level: Near transfer

28. In the heliocentric model, retrograde motion

    comes from circular motion around small circles (epicycles).
    arises for planets farther from the sun than the earth.
    occurs when one planet passes another.
    occurs only once a year for each planet. 

Answer: C, With the assumption (later confirmed) that the farther a planet lies from the sun, the slower it moves in its path around the sun, then retrograde occurs whenever one planet passes another on the same side of the sun.

Level: Recall

29. Imagine a new planet appears in the sky. After some time of naked-eye observations, we find that the planet stays close to the sun with a maximum elongation of 30 degrees. For comparison, the maximum elongation angle of Venus is about 46 degrees; for Mercury, 23 degrees. In a basic heliocentric model, we can conclude that

    the planet is closer to the sun than Mercury.
    the planet lies between Mercury and Venus.
    the planet lies between Venus and the earth.
    we cannot tell the relative distance of the planet from the sun. 

Answer: B, With a maximum elongation angle between that of Mercury and Venus, the planet lies between Mercury and Venus in a heliocentric model.

Level: Near transfer

Indicate whether the statements 31-32 about astronomical events seen from Jupiter in a heliocentric model are: always true (answer A), sometimes true (answer B), or never true (answer C).

30. The earth is undergoing retrograde motion.

    Always true.
    Sometimes true.
    Never true. 

Answer: B, Sometimes true; occurs when the earth passes Jupiter.

Level: Far transfer.

31. Mars is at maximum elongation.

    Always true.
    Sometimes true.
    Never true. 

Answer: B, Sometimes true; occurs when Mars, which is interior to Jupiter, reaches its maximum distance from the sun as seen from Jupiter.

Level: Far transfer.

32. Saturn is in opposition during its retrograde motion.

    Always true.
    Sometimes true.
    Never true. 

Answer: A, Always true; occurs when Jupiter, which is interior to Saturn, passes Saturn.

Level: Far transfer.

33. You are carried away by an alien spacecraft to a different star planetary system. You are set down on a planet with cloudless skies. After some time, you notice five planets in the sky. Three retrograde after greatest eastern elongation with the "sun"; two at opposition. From this observation, you infer that, in a heliocentric model, you are on the _____ planet outward from the "sun".

    first
    second
    third
    fourth
    fifth 

Answer: D, The two planets that retrograde at opposition lie farther away from the "sun" than the planet from which you are observing. The other three planets lie closer to the "sun". So you are on the fourth planet out from the "sun".

Level: Far transfer

Analyze the statements 34-36 to judge whether they are MODEL DEPENDENT or MODEL INDEPENDENT. An inference or conclusion based on a particular model is model dependent; a direct observation (with little interpretation) is not.

34. The earth is in the center of the solar system; the sun, moon and planets move around the earth.

    Model dependent.
    Model independent. 

Answer: A, This statement depends on the acceptance of a geocentric model of the solar system. So it is model dependent.

Level: Near transfer

35. The sun is in the center of the solar system; the earth and other planets move around the sun.

    Model dependent.
    Model independent. 

Answer: A, This statement depends on the acceptance of a heliocentric model of the solar system. So it is model dependent.

Level: Near transfer

36. Write your answer to this question on the FRONT page of the test!

Is the solar system geocentric or heliocentric?

    How do you know?
    Give at least two reasons for "how do you know"? 

Answer: Heliocentric. Reasons: 1) Observations of stellar parallax falsify a geocentric model and verifies heliocentric one; 2) Newton's laws of motion of gravitation provide physical basis for a heliocentric model.

Level: Near transfer

37. Imagine you are viewing an evening scene in which the sun is just setting, Venus is near the sun, Mars is at opposition, and Jupiter lies about halfway between Venus and Mars. Consider making a geocentric model and a heliocentric model of this situation. When you compare the two models, the angular relationships between the planets are explained

    better in a heliocentric model.
    better in a geocentric model.
    equally well in both models. 

Answer: C, Equally well, as all angular relationships are the same.

Level: Near transfer
