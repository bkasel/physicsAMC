
%% NASA Astronomy 101 Exam 1
%%--------------------------------------------------

\element{nasa}{
\begin{question}{exam01a1-q01}
    Which of the following statements is true about the location of the Sun at sunset during the middle of winter?
    \begin{choices}
        \wrongchoice{It will be near the horizon, north of west.}
        \wrongchoice{It will be near the horizon, north of east.}
        \wrongchoice{It will be near the horizon, south of west.}
        \wrongchoice{It will be near the horizon, south of east.}
        \wrongchoice{None of the above is correct.}
    \end{choices}
\end{question}
}

\element{nasa}{
\begin{question}{exam01a1-q02}
    Which of the following groups of moon phases is above the horizon at 4 pm?
    \begin{choices}
        \wrongchoice{Full, Waning Crescent, and Waxing Gibbous}
        \wrongchoice{New Moon, First Quarter, and Waxing Gibbous}
        \wrongchoice{Waxing Gibbous, Full Moon, Waning Gibbous}
        \wrongchoice{Waxing Crescent, Third Quarter, Waxing Gibbous}
        \wrongchoice{None. The moon is only visible above the horizon during the night time.}
    \end{choices}
\end{question}
}

\element{nasa}{
\begin{question}{exam01a1-q03}
    If the Moon is in the Waxing Gibbous phase today,
        approximately how long will it be until the Moon is in the Waxing Crescent phase?
    \begin{choices}
        \wrongchoice{a month}
        \wrongchoice{one week}
        \wrongchoice{one day}
        \wrongchoice{three weeks }
        \wrongchoice{two weeks}
    \end{choices}
\end{question}
}

\element{nasa}{
\begin{question}{exam01a1-q04}
    Facing \rule[-0.1pt]{4em}{0.1pt} you would observe the stars of the big dipper to trace out \rule[-0.1pt]{4em}{0.1pt} over a period of \rule[-0.1pt]{4em}{0.1pt}. 
    \begin{choices}
        \wrongchoice{north, half a circle, 6 hours}
        \wrongchoice{south, half a circle, 12 hours}
        \wrongchoice{north, one fourth of a circle, 6 hours}
        \wrongchoice{south, one fourth of a circle, 6 hours}
        \wrongchoice{none of the provided are correct}
    \end{choices}
\end{question}
}

\element{nasa}{
\begin{question}{exam01a1-q05}
    %% COPIED
    What do we call the day(s) of the year when the Sun rises directly in the East and sets directly in the West? 
    \begin{choices}
        \wrongchoice{Solstices }
        \wrongchoice{Circumpolar}
        \wrongchoice{Equinoxes }
        \wrongchoice{Celestial }
    \end{choices}
\end{question}
}

\element{nasa}{
\begin{question}{exam01a1-q06}
    Use the drawing below to answer the question below.
    \begin{center}
        %% NOTE: includegraphic
    \end{center}
    If you could see stars during the day,
        the drawing above shows what the sky would look like at noon on a given day. 
    The Sun is near the stars of the constellation Taurus. 
    Near which constellation would you expect the Sun to be located at 6am on this day?
    \begin{multicols}{2}
    \begin{choices}
        \wrongchoice{ Pisces}
        \wrongchoice{Taurus}
        \wrongchoice{Aries }
        \wrongchoice{Cancer }
        \wrongchoice{Gemini}
    \end{choices}
    \end{multicols}
\end{question}
}

\element{nasa}{
\begin{question}{exam01a1-q07}
    Use the drawing below to answer the question below.
    \begin{center}
        %% NOTE: includegraphic
    \end{center}
    Which constellation in the drawing above will be highest in the sky at sunset?
    \begin{multicols}{2}
    \begin{choices}
        \wrongchoice{Pisces}
        \wrongchoice{Cancer}
        \wrongchoice{Gemini}
        \wrongchoice{Taurus}
        \wrongchoice{Aries}
    \end{choices}
    \end{multicols}
\end{question}
}

\element{nasa}{
\begin{question}{exam01a1-q08}
    Which phase of the Moon is on the western horizon when the Sun is setting?
    \begin{multicols}{2}
    \begin{choices}
        \wrongchoice{waxing crescent}
        \wrongchoice{waning gibbous}
        \wrongchoice{new}
        \wrongchoice{third quarter}
        \wrongchoice{full}
    \end{choices}
    \end{multicols}
\end{question}
}

\element{nasa}{
\begin{question}{exam01a1-q09}
    Which of the following rankings is correct for the size (width) of the following objects from smallest to largest?
    \begin{choices}
        \wrongchoice{star, planet, solar system, galaxy, universe}
        \wrongchoice{planet, star, galaxy, solar system, universe }
        \wrongchoice{solar system, Planet, star, galaxy universe}
        \wrongchoice{planet, star, solar system, galaxy, universe}
        \wrongchoice{star, solar system, planet, galaxy, universe}
    \end{choices}
\end{question}
}

\element{nasa}{
\begin{question}{exam01a1-q10}
    What time is it when the Waning Crescent Moon is highest in the sky? 
    \begin{choices}
        \wrongchoice{morning}
        \wrongchoice{noon}
        \wrongchoice{early afternoon}
        \wrongchoice{evening}
        \wrongchoice{midnight}
    \end{choices}
\end{question}
}

\element{nasa}{
\begin{question}{exam01a1-q11}
    For an observer in the continental U.S.,
        which of the $x$’s (a, b, c or d) in the figure at right correctly show where the end of the stick’s shadow would be located at noon for different times throughout the year?
    \begin{center}
        %% NOTE: includegraphic
    \end{center}
    Note that the positions of the Sun's shadow at noon on the solstices are labeled for you.
    \begin{choices}
        \wrongchoice{only position a}
        \wrongchoice{only position b}
        \wrongchoice{only position c}
        \wrongchoice{only position d}
        \wrongchoice{more than one of the positions (a, b, c, or d) is possible}
    \end{choices}
\end{question}
}

\element{nasa}{
\begin{question}{exam01a1-q12}
    Which of the following describes one reason that the northern and the southern hemispheres have different seasons at the same time?
    \begin{choices}
        \wrongchoice{The Earth is closer to the Sun during summer in the southern hemisphere and is farther from the sun during winter in the northern hemisphere.}
        \wrongchoice{During the time of the year when the Sun is high in the sky in the northern hemisphere it will be low in the sky in the southern hemisphere.}
        \wrongchoice{The Earth is tilted, so the Sun is closer to one hemisphere than the other, which causes one hemisphere to be in winter and the other in summer.}
        \wrongchoice{The energy received at Earth from the Sun changes throughout the year providing more energy to one hemisphere than the other.}
        \wrongchoice{The amount the Earth is tilted changes over the course of the year and causes the amount of sunlight that reaches each hemisphere to be different which causes the seasons to be opposites. }
    \end{choices}
\end{question}
}

\element{nasa}{
\begin{question}{exam01a1-q13}
    Which of the following sequences of moon phases will occur one after the other within one cycle of phases: 
    \begin{choices}
        \wrongchoice{third quarter, waxing gibbous, full, waning crescent}
        \wrongchoice{waxing gibbous, full, waning gibbous, third quarter}
        \wrongchoice{new, waning crescent, first quarter, waxing crescent}
        \wrongchoice{full, waning gibbous, first quarter, waxing gibbous}
        \wrongchoice{new, waxing crescent, full, waning gibbous}
    \end{choices}
\end{question}
}

\element{nasa}{
\begin{question}{exam01a1-q14}
    During the new moon phase,
        how much of the Moon's total surface is being illuminated by sunlight?
    \begin{choices}
        \wrongchoice{none}
        \wrongchoice{less than half}
        \wrongchoice{half}
        \wrongchoice{more than half}
    \end{choices}
\end{question}
}

\element{nasa}{
\begin{question}{exam01a1-q15}
    In what phase and location described below will the Moon be when a solar eclipse occurs?
    \begin{choices}
        \wrongchoice{New phase and above the plane of Earth’s orbit }
        \wrongchoice{Full phase and above the plane of Earth’s orbit}
        \wrongchoice{New phase and crossing Earth’s orbital plane.}
        \wrongchoice{Full phase and crossing Earth’s orbital plane.}
        \wrongchoice{None of the provided}
    \end{choices}
\end{question}
}

\element{nasa}{
\begin{question}{exam01a1-q16}
    You look to the western horizon as the Moon is setting and discover that it is in the full moon phase. 
    Earlier that same day when the moon was first rising,
        which of the moon phases shown at right would the Moon have looked like?
    \begin{multicols}{2}
    \begin{choices}
        \wrongchoice{
            %% NOTE: moon phases by tikzpictures??
            \begin{tikzpicture}
            \end{tikzpicture}
        }
    \end{choices}
    \end{multicols}
\end{question}
}

\element{nasa}{
\begin{question}{exam01a1-q17}
    Which ``X'' could represent the position of the end of the stick's shadow made shortly before sunset during the summer?
    \begin{center}
        %% NOTE: includegraphic
    \end{center}
    \begin{multicols}{2}
    \begin{choices}
        \wrongchoice{$a$}
        \wrongchoice{$b$}
        \wrongchoice{$c$}
        \wrongchoice{$d$}
        \wrongchoice{None of the above}
    \end{choices}
    \end{multicols}
\end{question}
}

\element{nasa}{
\begin{question}{exam01a1-q18}
    %% NOTE: change wording
    For the next four questions, use the two figures provided below,
        which show the motion of Stars A and B in the sky.  
    Note that Star A is first visible above the horizon at noon.
    \begin{center}
        %% NOTE: includegraphic
    \end{center}
    At what time will Star B be located high in the Northeastern sky?
    \begin{multicols}{2}
    \begin{choices}
        \wrongchoice{11:00 pm}
        \wrongchoice{9:00 pm}
        \wrongchoice{11:00 am}
        \wrongchoice{6:00 am}
        \wrongchoice{6:00 pm}
    \end{choices}
    \end{multicols}
\end{question}
}

\element{nasa}{
\begin{question}{exam01a1-q19}
    %% NOTE: change wording
    For the next four questions, use the two figures provided below,
        which show the motion of Stars A and B in the sky.  
    Note that Star A is first visible above the horizon at noon.
    \begin{center}
        %% NOTE: includegraphic
    \end{center}
    In what direction is Star B moving at 3:00 am?
    \begin{choices}
        \wrongchoice{west (to the left)}
        \wrongchoice{east (to the right)}
        \wrongchoice{south(out of the page)}
        \wrongchoice{away from the horizon (up)}
        \wrongchoice{toward the horizon (down)}
    \end{choices}
\end{question}
}

\element{nasa}{
\begin{question}{exam01a1-q20}
    %% NOTE: change wording
    For the next four questions, use the two figures provided below,
        which show the motion of Stars A and B in the sky.  
    Note that Star A is first visible above the horizon at noon.
    \begin{center}
        %% NOTE: includegraphic
    \end{center}
    At what time would you see Star A high in the southern part of the sky? 
    \begin{multicols}{2}
    \begin{choices}
       \wrongchoice{3:00 pm}
       \wrongchoice{6:00 pm}
       \wrongchoice{9:00 pm}
       \wrongchoice{Midnight}
       \wrongchoice{3:00am}
    \end{choices}
    \end{multicols}
\end{question}
}

\element{nasa}{
\begin{question}{exam01a1-q21}
    %% NOTE: change wording
    For the next four questions, use the two figures provided below,
        which show the motion of Stars A and B in the sky.  
    Note that Star A is first visible above the horizon at noon.
    \begin{center}
        %% NOTE: includegraphic
    \end{center}
    At what time would you see Star A on the horizon in the west? 
    \begin{multicols}{2}
    \begin{choices}
        \wrongchoice{6:00 am}
        \wrongchoice{Noon}
        \wrongchoice{6:00 pm}
        \wrongchoice{9:00 pm}
        \wrongchoice{Midnight}
    \end{choices}
    \end{multicols}
\end{question}
}

\element{nasa}{
\begin{question}{exam01a1-q22}
    If Earth were upright with no tilt,
        would the temperature at your location in January be colder,
        warmer or the same as it is currently during the month of January? 
    \begin{multicols}{2}
    \begin{choices}
        \wrongchoice{cooler}
        \wrongchoice{warmer}
        \wrongchoice{the same}
    \end{choices}
    \end{multicols}
\end{question}
}

\element{nasa}{
\begin{question}{exam01a1-q23}
    Which Moon position (a-e), shown in the diagram at right,
        best corresponds with the moon phase shown below? 
    %% NOTE: tikzpicture of moon
    \begin{multicols}{2}
    \begin{choices}
        \wrongchoice{
            \begin{tikzpicture}
                %% Earth sun and Moon Diagram
            \end{tikzpicture}
        }
        \wrongchoice{warmer}
        \wrongchoice{the same}
    \end{choices}
    \end{multicols}
\end{question}
}

\element{nasa}{
\begin{question}{exam01a1-q24}
    At Noon on the day when the Sun goes highest in the sky for the entire year,
        how will your shadow appear: 
    \begin{choices}
        \wrongchoice{short and pointing to the south.}
        \wrongchoice{short and pointing to the north.}
        \wrongchoice{long and pointing to the north.}
        \wrongchoice{long and pointing to the south.}
        \wrongchoice{at noon on this day you will not cast a shadow.}
    \end{choices}
\end{question}
}

\element{nasa}{
\begin{question}{exam01a1-q25}
    Which of the following locations experiences the least amount of change in sunlight over the course of a year?
    \begin{choices}
        \wrongchoice{north pole}
        \wrongchoice{south pole}
        \wrongchoice{equator}
        \wrongchoice{They all experience the same amount of change in sunlight over a year.}
    \end{choices}
\end{question}
}

\element{nasa}{
\begin{question}{exam01a1-q26}
    Imagine that Earth moved 1 million miles closer to the Sun during its orbit than it currently does, and 6 months later it moved 1 million miles further away than it currently does.
    How would this affect the seasons?
    \begin{choices}
        \wrongchoice{We would no longer experience a difference between the seasons.}
        \wrongchoice{We would continue to experience seasons in essentially the same way we do now.}
        \wrongchoice{We would still experience seasons, but the difference would be much more noticeable.}
        \wrongchoice{We would still experience seasons, but the difference would be much less noticeable.}
    \end{choices}
\end{question}
}

\element{nasa}{
\begin{question}{exam01a1-q27}
    What time is it when the moon phase shown at right first begins to rise above the horizon?
    %% NOTE: include moon tikzpicture?
    \begin{choices}
        \wrongchoice{in the evening}
        \wrongchoice{at noon}
        \wrongchoice{in the mid-afternoon}
        \wrongchoice{at midnight}
        \wrongchoice{in the early morning}
    \end{choices}
\end{question}
}

\element{nasa}{
\begin{question}{exam01a1-q28}
    %% COPIED
    During the full moon phase,
        how much of the illuminated portion of the Moon's surface is visible from Earth?
    \begin{multicols}{2}
    \begin{choices}
        \wrongchoice{none}
        \wrongchoice{all}
        \wrongchoice{less than half}
        \wrongchoice{more than half}
        \wrongchoice{half}
    \end{choices}
    \end{multicols}
\end{question}
}

\element{nasa}{
\begin{question}{exam01a1-q29}
    If you are located in the continental U.S. on the first day of July,
        how will the position of the Sun at noon be different two weeks later?
    \begin{choices}
        \wrongchoice{It will have moved toward the North.}
        \wrongchoice{It will have moved to a position higher in the sky.}
        \wrongchoice{It will stay in the same position.}
        \wrongchoice{It will have moved to a position closer to the horizon.}
        \wrongchoice{It will have moved toward the west.}
    \end{choices}
\end{question}
}

\element{nasa}{
\begin{question}{exam01a1-q30}
    In what phase and location described below will the Moon be,
        when a lunar eclipse occurs? 
    \begin{choices}
        \wrongchoice{Full phase and below the plane of Earth's orbit.}
        \wrongchoice{Full phase and crossing Earth’s orbital plane. }
        \wrongchoice{New phase and below the plane of Earth's orbit.}
        \wrongchoice{New phase and crossing Earth’s orbital plane.}
        \wrongchoice{None of the provided}
    \end{choices}
\end{question}
}

\element{nasa}{
\begin{question}{exam01a1-q31}
    Which of the following best describes why the Moon goes through phases?   
    \begin{choices}
        \wrongchoice{Earth's shadow falls on different parts of the Moon at different times.}
        \wrongchoice{Earth’s clouds cover portions of the Moon at various times resulting in the changing phases that we see.}
        \wrongchoice{We see only part of the lit-up surface of the Moon depending on its position relative to Earth and the Sun.}
        \wrongchoice{The sunlight reflected from Earth lights up the Moon but is less effective when the Moon is lower in the sky than when it is higher in the sky.}
    \end{choices}
\end{question}
}

\element{nasa}{
\begin{question}{exam01a1-q32}
    If Earth were tilted less (\ang{10} rather than \ang{23.5}),
        then during winter at your location you would: 
    \begin{choices}
        \wrongchoice{experience cooler temperatures.}
        \wrongchoice{experience warmer temperatures.}
        \wrongchoice{not experience a change in temperature.}
    \end{choices}
\end{question}
}

\element{nasa}{
\begin{question}{exam01a1-q33}
    Which letter on the map at right shows: 
    Spica?
    \begin{choices}
        %% NOTE: include star charts
        \wrongchoice{}
    \end{choices}
\end{question}
}

\element{nasa}{
\begin{question}{exam01a1-q34}
    Which letter on the map at right shows: 
    Bo\"{o}tes? 
    \begin{choices}
        %% NOTE: include star charts
        \wrongchoice{}
    \end{choices}
\end{question}
}

\element{nasa}{
\begin{question}{exam01a1-q35}
    Which letter on the map at right shows: 
    Leo?
    \begin{choices}
        %% NOTE: include star charts
        \wrongchoice{}
    \end{choices}
\end{question}
}

\element{nasa}{
\begin{question}{exam01a1-q36}
    Which letter on the map at right shows: 
    Arcturus?
    \begin{choices}
        %% NOTE: include star charts
        \wrongchoice{}
    \end{choices}
\end{question}
}

\element{nasa}{
\begin{question}{exam01a1-q37}
    Imagine you see Mars very low in the west at 3am.
    Six hours earlier what direction would you face (look) to see Mars when it was highest in the sky?
    \begin{choices}
        \wrongchoice{toward the north}
        \wrongchoice{toward the south}
        \wrongchoice{toward the east }
        \wrongchoice{toward the west}
        \wrongchoice{directly overhead}
    \end{choices}
\end{question}
}

\element{nasa}{
\begin{question}{exam01a1-q38}
    Which shadow plot (A or B) most closely corresponds to the Sun's path through the sky during the summer? 
    %% NOTE: include graphic
    \begin{choices}
        \wrongchoice{Shadow Plot A}
        \wrongchoice{Shadow Plot B}
    \end{choices}
\end{question}
}

\element{nasa}{
\begin{question}{exam01a1-q39}
    Use the figure below to answer the next three questions.  
    In this Earth-Sun system drawing we have indicated the direction of both the daily rotation of Earth about its own axis and its annual orbit about the Sun.
    Imagine you are the observer shown on Earth in the northern hemisphere.
    %% NOTE: include solar system diagram
    For the time shown,
        which constellation will be highest in the sky at noon?  
    \begin{multicols}{2}
    \begin{choices}
        \wrongchoice{Aquarius }
        \wrongchoice{Pisces}
        \wrongchoice{Leo}
        \wrongchoice{Taurus}
        \wrongchoice{Scorpius}
    \end{choices}
    \end{multicols}
\end{question}
}

\element{nasa}{
\begin{question}{exam01a1-q40}
    Use the figure below to answer the next three questions.  
    In this Earth-Sun system drawing we have indicated the direction of both the daily rotation of Earth about its own axis and its annual orbit about the Sun.
    Imagine you are the observer shown on Earth in the northern hemisphere.
    %% NOTE: include solar system diagram
    What will be an individual's birth sign,
        5 months after the time shown?
    \begin{multicols}{2}
    \begin{choices}
        \wrongchoice{Libra}
        \wrongchoice{Aries}
        \wrongchoice{Sagittarius}
        \wrongchoice{Gemini}
        \wrongchoice{Scorpius}
    \end{choices}
    \end{multicols}
\end{question}
}

\element{nasa}{
\begin{question}{exam01a1-q41}
    Use the figure below to answer the next three questions.  
    In this Earth-Sun system drawing we have indicated the direction of both the daily rotation of Earth about its own axis and its annual orbit about the Sun.
    Imagine you are the observer shown on Earth in the northern hemisphere.
    %% NOTE: include solar system diagram
    Which constellation will be high in the sky at midnight in 9 months? 
    \begin{multicols}{2}
    \begin{choices}
        \wrongchoice{Aquarius}
        \wrongchoice{Pisces}
        \wrongchoice{Leo}
        \wrongchoice{Cancer}
        \wrongchoice{Capricornus}
    \end{choices}
    \end{multicols}
\end{question}
}

\element{nasa}{
\begin{question}{exam01a1-q42}
    %% COPIED
    Which of the following descriptions of Zodiacal constellations best defines the birth sign of a person?  
    \begin{itemize}
        \item Scorpius is in the east at sunset.
        \item Cancer is high in the southern sky at sunrise.
        \item Libra is on the eastern horizon at noon.
        \item Taurus is on the western horizon at sunset.
    \end{itemize}
    \begin{multicols}{2}
    \begin{choices}
        \wrongchoice{Taurus}
        \wrongchoice{Libra}
        \wrongchoice{Cancer}
        \wrongchoice{Scorpius}
    \end{choices}
    \end{multicols}
\end{question}
}

\element{nasa}{
\begin{question}{exam01a1-q43}
    The Sun appears to rise and set in our sky because \rule[-0.1pt]{4em}{0.1pt},
        and you are one year older each time \rule[-0.1pt]{4em}{0.1pt}.
    \begin{choices}
        \wrongchoice{Earth rotates on its axis; Earth completes one orbit of the Sun}
        \wrongchoice{the Sun moves across the orbit of Earth; the Sun completes one rotation on its axis}
        \wrongchoice{Earth's rotational axis is tilted, Earth completes one rotation on its axis}
        \wrongchoice{the Sun rotates on its axis; Earth completes one orbit of the Sun}
        \wrongchoice{Earth rotates on its axis; the Sun completes one rotation on its axis}
    \end{choices}
\end{question}
}

\element{nasa}{
\begin{question}{exam01a1-q44}
    Which one lettered position (A-E), in the image below,
        best represents the location on Earth that is experiencing summer in the Southern Hemisphere? 
    %% NOTE: include sun and earth picture
    \begin{choices}
        %% Options are linked to picture
        \wrongchoice{}
    \end{choices}
\end{question}
}

\element{nasa}{
\begin{question}{exam01a1-q45}
    Which of the following correctly ranks locations from closest to Earth to farthest from Earth?
    \begin{choices}
        \wrongchoice{the Sun, the edge of our solar system, the nearby star Alpha Centauri, far edge of Milky Way galaxy, near side of Andromeda Galaxy}
        \wrongchoice{the nearby star Alpha Centauri, the Sun, edge of our solar system, near side of Andromeda Galaxy, far edge of Milky Way galaxy}
        \wrongchoice{the edge of our solar system, the Sun, the nearby star Alpha Centauri, far edge of Milky Way galaxy, near side of Andromeda Galaxy}
        \wrongchoice{the Sun, the nearby star Alpha Centauri, edge of our solar system, near side of Andromeda Galaxy, far edge of Milky Way galaxy }
        \wrongchoice{the Sun, the edge of our solar system, the nearby star Alpha Centauri, near side of Andromeda Galaxy, far edge of Milky Way galaxy}
    \end{choices}
\end{question}
}

\element{nasa}{
\begin{question}{exam01a1-q46}
    For the next two questions,
        consider the five situations shown below (A--E) in which Earth is drawn along with the light coming from the Sun.
    %% NOTE: sun and earth diagrams
    How many of the situations shown above are physically possible in terms of how the Earth’s tilt is represented relative to daytime,
        nighttime, and the direction of sunlight.
    \begin{choices}
        \wrongchoice{only one}
        \wrongchoice{two}
        \wrongchoice{three}
        \wrongchoice{four }
        \wrongchoice{all of the above are possible.}
    \end{choices}
\end{question}
}

\element{nasa}{
\begin{question}{exam01a1-q47}
    For the next two questions,
        consider the five situations shown below (A--E) in which Earth is drawn along with the light coming from the Sun.
    %% NOTE: sun and earth diagrams
    For the situation(s) that you identified in the previous question as being physically possible,
        how many of the locations marked with an ``X'' would be experiencing Summer? 
    \begin{choices}
        \wrongchoice{only one}
        \wrongchoice{two}
        \wrongchoice{three}
        \wrongchoice{four or more}
        \wrongchoice{none}
    \end{choices}
\end{question}
}

\element{nasa}{
\begin{question}{exam01a1-q48}
    If the moon is in the waxing gibbous phase today,
        how many of the moon phases shown above (A--E) would the moon go through during the next 11 days.
    %% NOTE: sun and earth diagrams
    \begin{choices}
        \wrongchoice{only one}
        \wrongchoice{two}
        \wrongchoice{three}
        \wrongchoice{more than three}
        \wrongchoice{none}
    \end{choices}
\end{question}
}

\element{nasa}{
\begin{question}{exam01a1-q49}
    How many of the Earth locations shown above (A--F) would be experiencing Summer?
    %% NOTE: sun earth diagrams, tikz earth?
    \begin{choices}
        \wrongchoice{only one }
        \wrongchoice{two}
        \wrongchoice{three}
        \wrongchoice{four}
        \wrongchoice{all the positions are experiencing Summer.}
    \end{choices}
\end{question}
}

\element{nasa}{
\begin{question}{exam01a1-q50}
    The figure below shows the evening sky as it would appear for an observer in the northern hemisphere.  Notice that Polaris,
        the North Star, appears fairly high in the sky while other stars (labeled A--E) appear to slowly move around the North Star.
    %% NOTE: include graphic
    Which of the following is the best ranking for the amount of time that each of the stars shown above (A - E),
        will be above the horizon during a 24 hour period,
        from least amount of time to greatest.
    \begin{choices}
        \wrongchoice{$C<A<B<D<E$}
        \wrongchoice{$E<D<B<A<C$}
        \wrongchoice{$B=D<A<C=E$}
        \wrongchoice{$A<B=C<D<E$}
        \wrongchoice{$A=E<C=B=D$}
    \end{choices}
\end{question}
}


\endinput


