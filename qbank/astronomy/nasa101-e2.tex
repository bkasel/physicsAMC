
%% NASA Astronomy 101 Exam 2
%%--------------------------------------------------

\element{nasa}{
\begin{question}{exam01a2-q01}
    Which of the following is not a form of light?
    \begin{choices}[o]
        \wrongchoice{radio waves}
        \wrongchoice{microwaves}
        \wrongchoice{x-rays}
        \wrongchoice{All of the above are a form of light.}
        \wrongchoice{None of the above is a form of light}
    \end{choices}
\end{question}
}

\element{nasa}{
\begin{question}{exam01a2-q02}
    The planet in the orbit shown in the drawing at right obeys Kepler’s 2nd Law.  
    Use this drawing to answer the next four questions.
    %% NOTE: diagram
    %% start questions
    According to Kepler's Second Law, during which segment of the planet's orbit ``B'', ``C'', or ``D'',
        would the planet take the same amount of time as it took for the segment of the orbit identified with letter ``A''? 
    Answer ``E'' if you think the planet took the same amount of time to travel through \emph{all} of the segments of the motion (A, B, C, and D).
    \begin{choices}
        \wrongchoice{ }
    \end{choices}
\end{question}
}

\element{nasa}{
\begin{question}{exam01a2-q03}
    The planet in the orbit shown in the drawing at right obeys Kepler’s 2nd Law.  
    Use this drawing to answer the next four questions.
    %% NOTE: diagram
    %% start questions
    During which segment of the planet’s orbit (A, B, C, or D) would the planet move with the greatest speed?  
    Answer ``E'' if you think the planet travels with the same speed during \emph{all} of the segments of the motion (A, B, C, and D).
    \begin{choices}
        \wrongchoice{ }
    \end{choices}
\end{question}
}

\element{nasa}{
\begin{question}{exam01a2-q04}
    The planet in the orbit shown in the drawing at right obeys Kepler’s 2nd Law.  
    Use this drawing to answer the next four questions.
    %% NOTE: diagram
    %% start question
    During how many segments of the planet's orbit (A, B, C and D) would the planet be slowing down the entire time?
    \begin{choices}[o]
        \wrongchoice{Only during one of the portions shown.}
        \wrongchoice{During two of the portions shown.}
        \wrongchoice{During three of the portions shown.}
        \wrongchoice{During four of the portions shown.}
        \wrongchoice{None of the above.}
    \end{choices}
\end{question}
}

\element{nasa}{
\begin{question}{exam01a2-q05}
    The planet in the orbit shown in the drawing at right obeys Kepler’s 2nd Law.  
    Use this drawing to answer the next four questions.
    %% NOTE: diagram
    %% start question
    During which of the segments of the planet's orbit would the planet experience an increase in speed for at least one moment? 
    \begin{choices}[o]
        \wrongchoice{Only during one of the portions shown.}
        \wrongchoice{During two of the portions shown.}
        \wrongchoice{During three of the portions shown.}
        \wrongchoice{During four of the portions shown.}
        \wrongchoice{None of the above.}
    \end{choices}
\end{question}
}

\element{nasa}{
\begin{question}{exam01a2-q06}
    The planets in the orbits shown in the drawings at right obey Kepler's Laws. 
    Use these drawings to answer the next two questions.  
    %% start question
    Which of the planets shown would experience the greatest change in speed?
    \begin{choices}
        \wrongchoice{
            \begin{tikzpicture}
                %% NOTE: diamgrams, newcommand for 7 also
            \end{tikzpicture}
        }
    \end{choices}
\end{question}
}

\element{nasa}{
\begin{question}{exam01a2-q07}
    The planets in the orbits shown in the drawings at right obey Kepler's Laws. 
    Use these drawings to answer the next two questions.  
    %% start question
    Which of the planet orbits is least like Earth's orbit around the Sun?
    \begin{choices}
        \wrongchoice{
            \begin{tikzpicture}
                %% NOTE: diamgrams, newcommand for 7 also
            \end{tikzpicture}
        }
    \end{choices}
\end{question}
}

\element{nasa}{
\begin{question}{exam01a2-q08}
    Which of the following is the type of spectrum you would observe if you could receive the light from the Sun before it passes through Earth’s atmosphere?   
    \begin{choices}
        \wrongchoice{dark line absorption spectrum}
        \wrongchoice{bright line emission spectrum}
        \wrongchoice{continuous spectrum}
        \wrongchoice{none of the above}
    \end{choices}
\end{question}
}

\element{nasa}{
\begin{question}{exam01a2-q09}
    Use the drawings below to answer the next two questions. 
    %% start question
    Which drawing (not to scale) represents the process by which an emission line is formed?
    \begin{multicols}{2}
    \begin{choices}
        \wrongchoice{
            \begin{tikzpicture}
                %% NOTE: diamgrams, newcommand for 9 also
            \end{tikzpicture}
        }
    \end{choices}
    \end{multicols}
\end{question}
}

\element{nasa}{
\begin{question}{exam01a2-q10}
    Use the drawings below to answer the next two questions. 
    %% start question
    Which drawing (not to scale) represents the process by which an absorption line is formed?
    \begin{multicols}{2}
    \begin{choices}
        \wrongchoice{
            \begin{tikzpicture}
                %% NOTE: diamgrams, newcommand for 9 also
            \end{tikzpicture}
        }
    \end{choices}
    \end{multicols}
\end{question}
}

\element{nasa}{
\begin{question}{exam01a2-q11}
    Imagine that you are the head of a funding agency that can afford to build only one telescope.
    Which of the four proposed telescopes below would be best to support?
    \begin{choices}
        \wrongchoice{An ultraviolet telescope located in the Mojave desert}
        \wrongchoice{An Radio telescope in orbit above the Earth}
        \wrongchoice{A gamma ray telescope in orbit above the Earth}
        \wrongchoice{An x-ray telescope located on a mountain in Peru}
    \end{choices}
\end{question}
}

\element{nasa}{
\begin{question}{exam01a2-q12}
    Imagine you are comparing the four stars shown at right.  
    The temperature of each star is indicated by a shade of gray (as shown at right),
        such that the lighter the shade of gray,
        the higher the temperature of the star. 
    %% start question
    How many of the stars shown above could have the same luminosity as the star shown at right?
    \begin{choices}
        \wrongchoice{only one}
        \wrongchoice{two}
        \wrongchoice{three or more}
        \wrongchoice{none}
    \end{choices}
\end{question}
}

\element{nasa}{
\begin{question}{exam01a2-q13}
    How many of the following: gamma, x-ray, UV, Visible, IR and radio, coming from space can NOT be detected on the surface of Earth \emph{at all}?  
    \begin{choices}
        \wrongchoice{Only one }
        \wrongchoice{Two}
        \wrongchoice{Three}
        \wrongchoice{More than Three}
        \wrongchoice{None of the above}
    \end{choices}
\end{question}
}

\element{nasa}{
\begin{question}{exam01a2-q14}
    Enzo star gives off the same amount of energy as Ferdinand star.  
    But Enzo star is much, much hotter than Ferdinand star. 
    Which star has the greater surface area?
    \begin{choices}
        \wrongchoice{They have the same surface area}
        \wrongchoice{Enzo}
        \wrongchoice{Ferdinand}
        \wrongchoice{There is insufficient information to answer this question.}
    \end{choices}
\end{question}
}

\element{nasa}{
\begin{question}{exam01a2-q15}
    Energy is released from atoms in the form of light when electrons:
    \begin{choices}
        \wrongchoice{are emitted by the atom.}
        \wrongchoice{move from high energy levels to low energy levels.}
        \wrongchoice{are absorbed by atoms.}
        \wrongchoice{move in their orbit around the nucleus.}
        \wrongchoice{move from low energy levels to high energy levels.}
    \end{choices}
\end{question}
}

\element{nasa}{
\begin{question}{exam01a2-q16}
    Kepler's Laws of planetary motion are based on detailed observations made by which astronomer?
    \begin{choices}
        \wrongchoice{Isaac Newton}
        \wrongchoice{Tycho Brahe}
        \wrongchoice{Galileo Galilei}
        \wrongchoice{Nicholas Copernicus}
    \end{choices}
\end{question}
}

\element{nasa}{
\begin{question}{exam01a2-q17}
    Which of the following has the shortest wavelength?  
    \begin{choices}
        \wrongchoice{An X-ray.}
        \wrongchoice{A photon of ultraviolet light.}
        \wrongchoice{Infrared radiation.}
        \wrongchoice{Blue electromagnetic radiation.}
        \wrongchoice{A radio wave.}
    \end{choices}
\end{question}
}

\element{nasa}{
\begin{question}{exam01a2-q18}
    Which of the following has the shortest wavelength?  
    \begin{choices}
        \wrongchoice{An X-ray.}
        \wrongchoice{A photon of ultraviolet light.}
        \wrongchoice{Infrared radiation.}
        \wrongchoice{Blue electromagnetic radiation.}
        \wrongchoice{A radio wave.}
    \end{choices}
\end{question}
}

\newcommand{\nasaExamTwoQEighteen}{
\begin{tabu}{X[c]X[c]X[c]}
    \toprule
    Planet & Distance from the Sun (in Astronomical Units---AU) & Planet Mass (in terms of Earth's mass) \\
    \midrule
    Mercury & 0.38 & 0.06 \\
    Venus & 0.72 & 0.82 \\
    Earth & 1.0 & 1.0 \\
    Mars & 1.52 & 0.11 \\
    Jupiter & 5.20 & 318 \\
    \bottomrule
\end{tabu}
}

\element{nasa}{
\begin{question}{exam01a2-q18}
    For the next two questions, consider the information in the table and the student comment,
        both provided below. 
    \begin{center}
        \nasaExamTwoQEighteen
    \end{center}
    A student in your class makes the following comment about the relationship between the location of planets in our solar system and their orbital period and mass:
    \begin{quote}
        ``As we look at planets farther away from the Sun than Mercury we see that their distance gets bigger and that the mass of the planets is also getting larger.
        So I think that the farther away a planet is the longer it takes to go around the Sun and the more massive the planet will be too.'' 
    \end{quote}
    %% Start questions
    Which planet listed in the table illustrates the fact that the student is incorrect in their reasoning.
    \begin{choices}
        \wrongchoice{Jupiter}
        \wrongchoice{Mercury}
        \wrongchoice{Earth}
        \wrongchoice{Mars }
        \wrongchoice{Venus}
    \end{choices}
\end{question}
}

\element{nasa}{
\begin{question}{exam01a2-q19}
    For the next two questions, consider the information in the table and the student comment,
        both provided below. 
    \begin{center}
        \nasaExamTwoQEighteen
    \end{center}
    A student in your class makes the following comment about the relationship between the location of planets in our solar system and their orbital period and mass:
    \begin{quote}
        ``As we look at planets farther away from the Sun than Mercury we see that their distance gets bigger and that the mass of the planets is also getting larger.
        So I think that the farther away a planet is the longer it takes to go around the Sun and the more massive the planet will be too.'' 
    \end{quote}
    %% Start questions
    Which of the following Laws would you use to explain to this student that their reasoning is incorrect?
    \begin{choices}
        \wrongchoice{Newton's Law of Gravity}
        \wrongchoice{Newton's 2nd}
        \wrongchoice{Newton's 3rd}
        \wrongchoice{Kepler's 2nd}
        \wrongchoice{Kepler's 3rd}
    \end{choices}
\end{question}
}

\element{nasa}{
\begin{question}{exam01a2-q20}
    Imagine that you throw a ball directly upward. 
    Which of the following statements best describes how Newton's Second Law accounts for the motion of the ball when it reaches its maximum height?
    \begin{choices}
        \wrongchoice{The ball has a net force that is downward and a velocity that is downward.}
        \wrongchoice{The ball has a velocity that is upward and an acceleration that is downward.}
        \wrongchoice{The ball has a net force that is downward and an acceleration that is downward. }
        \wrongchoice{The ball has a velocity that is zero and an acceleration that is zero.}
        \wrongchoice{The ball has a net force that is downward and an acceleration of zero.}
    \end{choices}
\end{question}
}

\element{nasa}{
\begin{question}{exam01a2-q21}
    Some telescopes are placed in space above Earth's atmosphere primarily for which of the following reasons?
    \begin{choices}
        \wrongchoice{Astronomers want to observe continuous spectra from objects rather than emission spectra.}
        \wrongchoice{Some of the light being sent out from the telescopes can be blocked by Earth’s atmosphere.}
        \wrongchoice{Astronomers want to observe continuous spectra from objects rather than absorption spectra.}
        \wrongchoice{Some of the light from objects is absorbed by Earth’s atmosphere.}
    \end{choices}
\end{question}
}

\element{nasa}{
\begin{question}{exam01a2-q22}
    The three spectral curves shown in the graphs below illustrate the energy output versus wavelength for three unknown stars X, Y, and Z.?
    \begin{center}
    \begin{tikzpicture}
        %% NOTE:
    \end{tikzpicture}
    \end{center}
    Which of the following is the correct ranking for the temperature of the stars, from hottest to coldest.
    \begin{multicols}{2}
    \begin{choices}
        \wrongchoice{$Y>X>Z$}
        \wrongchoice{$X>Z>Y$}
        \wrongchoice{$Z>Y>X$}
        \wrongchoice{$X>Y>Z$}
        \wrongchoice{$Y>Z>X$}
    \end{choices}
    \end{multicols}
\end{question}
}

\element{nasa}{
\begin{question}{exam01a2-q23}
    Kepler's second law says ``a line joining a planet and the Sun sweeps out equal areas in equal amounts of time.''
    Which of the following statements means nearly the same thing?
    \begin{choices}
        \wrongchoice{Planets move equal distances throughout their orbit of the Sun. }
        \wrongchoice{Planets move farther in each unit of time when they are closer to the Sun.}
        \wrongchoice{Planets move the same speed at all points during their orbit of the Sun. }
        \wrongchoice{Planets move slowest when they are moving away from the Sun. }
        \wrongchoice{Planets move fastest when they are moving toward the Sun.}
    \end{choices}
\end{question}
}

\element{nasa}{
\begin{question}{exam01a2-q24}
    What happens to the light that is missing in an absorption spectrum?
    \begin{choices}
        \wrongchoice{It is absorbed by atoms in a cool, low density cloud.}
        \wrongchoice{It is emitted by atoms in a hot dense cloud.}
        \wrongchoice{It is absorbed by the atoms inside the hot, dense object.}
        \wrongchoice{It is reflected by atoms on the surface of the hot, dense object.}
    \end{choices}
\end{question}
}

\element{nasa}{
\begin{question}{exam01a2-q25}
    Which of the following astronomers first proposed the heliocentric model of the solar system?
    \begin{choices}
        \wrongchoice{Isaac Newton}
        \wrongchoice{Claudius Ptolemy}
        \wrongchoice{Tycho Brahe}
        \wrongchoice{Nicholas Copernicus}
    \end{choices}
\end{question}
}

\element{nasa}{
\begin{question}{exam01a2-q26}
    If an electron in an atom moves from an energy level of 3 to an energy level of 6, 
    \begin{choices}
        \wrongchoice{a photon of energy 3 is absorbed.}
        \wrongchoice{a photon of energy 9 is emitted.}
        \wrongchoice{a photon of energy 3 is emitted.}
        \wrongchoice{a photon of energy 9 is absorbed.}
    \end{choices}
\end{question}
}

\element{nasa}{
\begin{question}{exam01a2-q27}
    Consider the dark line absorption spectra shown below for Star E and Star F. 
    What can you determine about the color of the two stars?
    \begin{center}
    \begin{tikzpicture}
        %% Star E                          Star F
    \end{tikzpicture}
    \end{center}
    [Assume that the left end of each spectrum corresponds to shorter wavelengths and that the right end of each spectrum corresponds with longer wavelengths.]
    \begin{choices}
        \wrongchoice{The color of the stars cannot be determined from this information.}
        \wrongchoice{Star E would appear blue and Star F would appear red.}
        \wrongchoice{Star E would appear red and Star F would appear blue.}
        \wrongchoice{Both stars would appear the same color.}
    \end{choices}
\end{question}
}

\element{nasa}{
\begin{question}{exam01a2-q28}
    Use the energy output versus wavelength graphs, for objects A--D,
        shown below to answer the next three questions.
    %% NOTE: energy vs. wavelenght graphs
    %% start question
    Which, if any, of the other objects has the same temperature as object A?
    \begin{choices}
        \wrongchoice{Object B}
        \wrongchoice{Object C}
        \wrongchoice{Object D}
        \wrongchoice{They are all the same temperature.}
        \wrongchoice{There is insufficient information to answer this question}
    \end{choices}
\end{question}
}

\element{nasa}{
\begin{question}{exam01a2-q29}
    Use the energy output versus wavelength graphs, for objects A--D,
        shown below to answer the next three questions.
    %% NOTE: energy vs. wavelenght graphs
    %% start question
    Which of these objects is the smallest?
    \begin{choices}
        \wrongchoice{Object A}
        \wrongchoice{Object B}
        \wrongchoice{Object C}
        \wrongchoice{Object D}
        \wrongchoice{There is insufficient information to answer this question.}
    \end{choices}
\end{question}
}

\element{nasa}{
\begin{question}{exam01a2-q30}
    Use the energy output versus wavelength graphs, for objects A--D,
        shown below to answer the next three questions.
    %% NOTE: energy vs. wavelenght graphs
    %% start question
    Which, if any, of the objects could be approximately the same size as object D?
    \begin{choices}
        \wrongchoice{Object A}
        \wrongchoice{Object B}
        \wrongchoice{Object C}
        \wrongchoice{More than one of the objects could be the same size as object D.}
        \wrongchoice{None of the above.}
    \end{choices}
\end{question}
}

\element{nasa}{
\begin{question}{exam01a2-q31}
    Use the four spectra for objects A--D, shown below,
        to answer the next two questions.  
    Note that one of the spectra is from an object at rest
        (not moving relative to Earth) and the remaining spectra come from objects that are all moving away from the observer.
    [Assume that the left end of the spectrum corresponds with short wavelengths and the right end corresponds with long wavelengths.]
    %% nOTE: diagrams
    %% start question
    Which object would be at rest?
    \begin{choices}
        \wrongchoice{Object A}
        \wrongchoice{Object B}
        \wrongchoice{Object C }
        \wrongchoice{Object D}
        \wrongchoice{More than one objects is at rest.}
    \end{choices}
\end{question}
}

\element{nasa}{
\begin{question}{exam01a2-q32}
    Use the four spectra for objects A--D, shown below,
        to answer the next two questions.  
    Note that one of the spectra is from an object at rest
        (not moving relative to Earth) and the remaining spectra come from objects that are all moving away from the observer.
    [Assume that the left end of the spectrum corresponds with short wavelengths and the right end corresponds with long wavelengths.]
    %% nOTE: diagrams
    %% start question
    Of the objects that are moving,
        which is moving with the fastest speed?
    \begin{choices}
        \wrongchoice{Object A}
        \wrongchoice{Object B}
        \wrongchoice{Object C}
        \wrongchoice{Object D}
        \wrongchoice{They are moving the same speed, the speed of light.}
    \end{choices}
\end{question}
}

\element{nasa}{
\begin{question}{exam01a2-q33}
    Which of the following would be true about comparing gamma rays and radio waves? 
    \begin{choices}
        \wrongchoice{The radio waves would have a longer wavelength and travel the same speed as gamma rays.}
        \wrongchoice{The radio waves would have a lower energy and would travel slower than gamma rays.}
        \wrongchoice{The gamma rays would have a shorter wavelength and a lower energy than radio waves.}
        \wrongchoice{The gamma rays would have a lower frequency and travel the same speed as radio waves. }
        \wrongchoice{The radio waves would have a shorter wavelength and higher energy than gamma rays.}
    \end{choices}
\end{question}
}

\element{nasa}{
\begin{question}{exam01a2-q34}
    Which of the following has the least energy?
    \begin{choices}
        \wrongchoice{radio waves}
        \wrongchoice{visible light}
        \wrongchoice{x-rays}
        \wrongchoice{infrared light}
        \wrongchoice{They all have the same energy.}
    \end{choices}
\end{question}
}

\element{nasa}{
\begin{question}{exam01a2-q35}
    What kind of spectrum is given off by a red colored, neon ``OPEN'' sign? 
    \begin{choices}
        \wrongchoice{a spectral curve with a peak in the red part of the spectrum}
        \wrongchoice{an absorption spectrum with more dark absorption lines in the blue part of the spectrum}
        \wrongchoice{a continuous spectrum that is brighter in the red than the blue}
        \wrongchoice{an emission spectrum with more bright emission lines at the red end of the spectrum}
    \end{choices}
\end{question}
}

\element{nasa}{
\begin{question}{exam01a2-q36}
    A bright blue star is moving away from Earth. 
    Which of the choices best completes the following statement describing the spectrum of this star?  
    You would observe a(n) \rule[-0.1pt]{4em}{0.1pt} spectrum that is \rule[-0.1pt]{4em}{0.1pt} compared to a star that is not moving.
    \begin{choices}
        \wrongchoice{absorption; redshifted }
        \wrongchoice{emission; redshifted }
        \wrongchoice{continuous; blueshifted }
        \wrongchoice{absorption; blueshifted }
        \wrongchoice{continuous; redshifted}
    \end{choices}
\end{question}
}

\element{nasa}{
\begin{question}{exam01a2-q37}
    Use the picture below to answer the next three questions.  
    In this picture the Earth-Moon system is shown (not to scale) along with three possible positions (A-C) for a spacecraft traveling from Earth to the Moon.  
    Note that position B is exactly halfway between Earth and the Moon. 
    %% NOTE: diagram
    %% start questions
    In what direction would the net (total) force point if the space ship were coasting very quickly toward the Moon when at position ``B''?
    \begin{choices}
        \wrongchoice{toward the Moon}
        \wrongchoice{toward Earth}
        \wrongchoice{Since the force on the spacecraft by Earth is equal to the force on the spacecraft by the Moon the net (total) force would be zero and not point in either direction.}
    \end{choices}
\end{question}
}

\element{nasa}{
\begin{question}{exam01a2-q38}
    Use the picture below to answer the next three questions.  
    In this picture the Earth-Moon system is shown (not to scale) along with three possible positions (A-C) for a spacecraft traveling from Earth to the Moon.  
    Note that position B is exactly halfway between Earth and the Moon. 
    %% NOTE: diagram
    %% start questions
    At which position (A, B or C) would the spacecraft feel the greatest acceleration?
    \begin{choices}
        \wrongchoice{at position A}
        \wrongchoice{at position B}
        \wrongchoice{at position C}
        \wrongchoice{The acceleration would be the same at all the positions.}
    \end{choices}
\end{question}
}

\element{nasa}{
\begin{question}{exam01a2-q39}
    Use the picture below to answer the next three questions.  
    In this picture the Earth-Moon system is shown (not to scale) along with three possible positions (A-C) for a spacecraft traveling from Earth to the Moon.  
    Note that position B is exactly halfway between Earth and the Moon. 
    %% NOTE: diagram
    %% start questions
    What would the spacecraft do next if it were moving toward the moon when at position “C”?
    \begin{choices}
        \wrongchoice{speed up}
        \wrongchoice{slow down}
        \wrongchoice{travel with a constant acceleration}
        \wrongchoice{travel with a constant speed}
    \end{choices}
\end{question}
}

\newcommand{\nasaExamTwoQForty}{
\begin{tabu}{X[c]X[2c]}
    \toprule
    STAR & Wavelength of Absorption line \\
    \midrule
    A & 649 nm \\
    B & 660 nm \\
    C & 656 nm \\
    D & 658 nm \\
    E & 647 nm \\
    \bottomrule
\end{tabu}
}

\element{nasa}{
\begin{question}{exam01a2-q39}
    An important line in the absorption spectrum of stars occurs at a wavelength of 656nm for stars at rest.  
    Imagine that you observe five stars (A-E) from Earth and discover that this absorption line is at the wavelength shown in the table below for each of the five stars. 
    \begin{center}
        \nasaExamTwoQForty
    \end{center}
    Based on the information in the table above,
        which of the following is the most accurate ranking of the speed of the stars from moving fastest toward Earth to moving fastest away from Earth.
    \begin{choices}
        \wrongchoice{A, B, C, D, E  }
        \wrongchoice{E, D, C, B, A  }
        \wrongchoice{C, E, A, D, B  }
        \wrongchoice{B, D, C, A, E   }
        \wrongchoice{E, A, C, D, B  }
    \end{choices}
\end{question}
}

\newcommand{\nasaExamTwoQFortyOne}{
\begin{tikzpicture}
    %% NOTE: 
\end{tikzpicture}
}

\element{nasa}{
\begin{question}{exam01a2-q41}
    Use the graph at right, showing the Luminosity versus Temperature of objects A--E,
        to answer the next two questions.
    \begin{center}
        \nasaExamTwoQFortyOne
    \end{center}
    Which object(s) is giving off just as much energy as object ``B''.
    \begin{choices}
        \wrongchoice{object D}
        \wrongchoice{object C}
        \wrongchoice{Both D and C}
        \wrongchoice{Neither D or C}
        \wrongchoice{None of the above}
    \end{choices}
\end{question}
}

\element{nasa}{
\begin{question}{exam01a2-q42}
    Use the graph at right, showing the Luminosity versus Temperature of objects A--E,
        to answer the next two questions.
    \begin{center}
        \nasaExamTwoQFortyOne
    \end{center}
    Which of the following is the correct ranking for the size of the objects A--E,
        from largest to smallest.
    \begin{choices}
        \wrongchoice{$E=A>C=B>D$}
        \wrongchoice{$D=B>C>A=E$}
        \wrongchoice{$D>B=C>A>E$}
        \wrongchoice{$E>A>C=B>D$}
        \wrongchoice{None of the above}
    \end{choices}
\end{question}
}

\element{nasa}{
\begin{question}{exam01a2-q43}
    If a small weather satellite and a larger communications satellite with twice the mass of the weather satellite are orbiting Earth at the same altitude above Earth’s surface, which of the following is true?
    \begin{choices}
        \wrongchoice{The large communications satellite will have an orbital period that is more than twice as long as the weather satellite.}
        \wrongchoice{The large communications satellite will have an orbital period that is less than half as long as weather satellite.}
        \wrongchoice{The large communications satellite will have an orbital period that is twice as long as the weather satellite.}
        \wrongchoice{The large communications satellite will have an orbital period that is half as long as weather satellite.}
        \wrongchoice{None of the above.}
    \end{choices}
\end{question}
}

\element{nasa}{
\begin{question}{exam01a2-q44}
    In each figure below two rocky asteroids are shown with masses ($m$),
        expressed in arbitrary units, separated by a distance ($d$),
        also expressed in arbitrary units.  
    Three of the asteroids are identified with the letters A, B, and C.  
    Use these figures to answer the next two questions.
    %% start question
    Which of the following is the correct ranking for the acceleration that asteroids A and C would experience as a result of the gravitational force exerted on them? 
    \begin{choices}
        \wrongchoice{A equal to C}
        \wrongchoice{A greater than C}
        \wrongchoice{C greater than A}
    \end{choices}
\end{question}
}

\element{nasa}{
\begin{question}{exam01a2-q45}
    In each figure below two rocky asteroids are shown with masses ($m$),
        expressed in arbitrary units, separated by a distance ($d$),
        also expressed in arbitrary units.  
    Three of the asteroids are identified with the letters A, B, and C.  
    Use these figures to answer the next two questions.
    %% start question
    Which of the following correctly describes how the gravitational force exerted BY asteroid A on its “partner” asteroid compares to the gravitational force exerted BY asteroid B on its “partner” asteroid.
    \begin{choices}
        \wrongchoice{The force of A on its partner is greater than the force of B on its partner.}
        \wrongchoice{The force of B on its partner is greater than the force of A on its partner.}
        \wrongchoice{The force of A on its partner is equal to the force of B on its partner.}
    \end{choices}
\end{question}
}

\element{nasa}{
\begin{question}{exam01a2-q46}
    Use the drawings below to answer the next two questions. 
    %% start question
    Which atom would emit light with the shortest wavelength? 
    \begin{choices}
        \wrongchoice{
            \begin{tikzpicture}
                %% NOTE: 
            \end{tikzpicture}
        }
    \end{choices}
\end{question}
}

\element{nasa}{
\begin{question}{exam01a2-q47}
    Use the drawings below to answer the next two questions. 
    %% start question
    Which atom would be absorbing light with the greatest energy? 
    \begin{choices}
        \wrongchoice{
            \begin{tikzpicture}
            \end{tikzpicture}
        }
    \end{choices}
\end{question}
}

\element{nasa}{
\begin{question}{exam01a2-q48}
    Which of Galileo's discoveries provided the greatest evidence that the Sun must be at the center of the solar system?
    \begin{choices}
        \wrongchoice{that Venus goes through phases}
        \wrongchoice{that Mars moves with retrograde motion}
        \wrongchoice{that the Sun rotates on its axis}
        \wrongchoice{that Saturn has rings }
        \wrongchoice{that Jupiter has several moons orbiting it}
    \end{choices}
\end{question}
}

\element{nasa}{
\begin{question}{exam01a2-q49}
    If you were looking at four Energy Output versus Wavelength graphs that were all the same height, which of the graphs would be from an object giving off the largest amount of Indigo light? 
    \begin{choices}
        \wrongchoice{A graph which peaks at IR wavelengths.}
        \wrongchoice{A graph which peaks at UV wavelengths.}
        \wrongchoice{A graph which peaks in the red part of the visible spectrum}
        \wrongchoice{The graph which peaks at Radio wavelengths.}
    \end{choices}
\end{question}
}

\element{nasa}{
\begin{question}{exam01a2-q50}
    Which of the following would cause the force on the Moon by the Earth to increase by the largest amount?
    \begin{choices}
        \wrongchoice{double the mass of the Moon. }
        \wrongchoice{double the mass of Earth.}
        \wrongchoice{move the moon two times closer to Earth.}
        \wrongchoice{Due to Newton’s third law, the Moon’s force on Earth will always be the same size as the Earth’s force on the Moon so none of the changes listed in choices a-c could cause the force to increase.}
    \end{choices}
\end{question}
}



\endinput


