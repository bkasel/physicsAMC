
%% SAMPLE EXAM 2: FOR Chapters 16, 17, & 18 in Chaisson
%%--------------------------------------------------

%% this section contains 50 problems

\element{astr}{
\begin{question}{ASTRI-2-Q01}
    The masses of stars on the main sequence \rule[-0.1pt]{4em}{0.1pt} from the lower right to the upper left.
    \begin{choices}
      \correctchoice{increase}
        \wrongchoice{decrease}
        \wrongchoice{are all the same}
        \wrongchoice{are randomly distributed}
    \end{choices}
\end{question}
}

\element{astr}{
\begin{question}{ASTRI-2-Q02}
    The fundamental quantity which determines a star's central pressure and temperature is its:
    \begin{choices}
      \correctchoice{mass.}
        \wrongchoice{luminosity.}
        \wrongchoice{surface temperature.}
        \wrongchoice{chemical composition.}
        \wrongchoice{radius.}
    \end{choices}
\end{question}
}

\element{astr}{
\begin{question}{ASTRI-2-Q03}
    The determination of stellar parallax is important because it allows the direct determination of:
    \begin{choices}
        \wrongchoice{mass.}
      \correctchoice{distance.}
        \wrongchoice{diameter.}
        \wrongchoice{velocity.}
    \end{choices}
\end{question}
}

\element{astr}{
\begin{question}{ASTRI-2-Q04}
    The observed differences between stellar spectra are caused primarily by differences in stellar:
    \begin{choices}
        \wrongchoice{luminosity.}
        \wrongchoice{chemical composition.}
      \correctchoice{temperature.}
        \wrongchoice{motions.}
        \wrongchoice{location in space.}
    \end{choices}
\end{question}
}

\element{astr}{
\begin{question}{ASTRI-2-Q05}
    Which of the following is the correct order for the spectral classification system?
    \begin{multicols}{2}
    \begin{choices}
        \wrongchoice{OBAFKGM}
        \wrongchoice{OBFAGKM}
      \correctchoice{OBAFGKM}
        \wrongchoice{ABFGKMO}
    \end{choices}
    \end{multicols}
\end{question}
}

\element{astr}{
\begin{question}{ASTRI-2-Q06}
    If a stellar spectrum shows strong lines produced by ionized helium,
        the star is of spectral type:
    \begin{multicols}{3}
    \begin{choices}
      \correctchoice{O.}
        \wrongchoice{B.}
        \wrongchoice{A.}
        \wrongchoice{F.}
        \wrongchoice{M.}
    \end{choices}
    \end{multicols}
\end{question}
}

\element{astr}{
\begin{question}{ASTRI-2-Q07}
    The reason astronomers use the concept of the absolute magnitude is to:
    \begin{choices}
        \wrongchoice{make life difficult for introductory astronomy students!}
      \correctchoice{allow stars to be compared with the effects of differing distance removed}
        \wrongchoice{allow stars to be compared with the effects of differing mass removed}
        \wrongchoice{allow stars to be compared with the effects of differing temperature removed}
        \wrongchoice{allow stars to be compared with the effects of differing radius removed.}
    \end{choices}
\end{question}
}

\element{astr}{
\begin{question}{ASTRI-2-Q08}
    Which of the following is \emph{not} by itself useful for determining stellar temperature?
    \begin{choices}
        \wrongchoice{spectral class}
        \wrongchoice{color index}
      \correctchoice{absolute magnitude}
        \wrongchoice{degree of ionization}
        \wrongchoice{wavelength of maximum intensity of the underlying spectrum}
    \end{choices}
\end{question}
}

\element{astr}{
\begin{question}{ASTRI-2-Q09}
    The purpose of the spectroscopic parallax technique is to determine:
    \begin{choices}
      \correctchoice{distance.}
        \wrongchoice{temperature.}
        \wrongchoice{apparent magnitude.}
        \wrongchoice{bolometric magnitude.}
        \wrongchoice{color index.}
    \end{choices}
\end{question}
}

\element{astr}{
\begin{question}{ASTRI-2-Q10}
    An H-R diagram is a plot of:
    \begin{choices}
        \wrongchoice{heat versus radius.}
        \wrongchoice{luminosity versus radius.}
        \wrongchoice{mass versus temperature.}
      \correctchoice{luminosity versus temperature.}
        \wrongchoice{mass versus luminosity.}
    \end{choices}
\end{question}
}

\element{astr}{
\begin{question}{ASTRI-2-Q11}
    Very luminous stars with low temperatures:
    \begin{choices}
        \wrongchoice{have small diameters compared with that of the Sun.}
        \wrongchoice{have diameters comparable with that of the Sun.}
      \correctchoice{have large diameters compared with that of the Sun.}
        \wrongchoice{are white dwarfs.}
    \end{choices}
\end{question}
}

\element{astr}{
\begin{question}{ASTRI-2-Q12}
    The interior of the Sun is a:
    \begin{choices}
      \correctchoice{gas.}
        \wrongchoice{liquid.}
        \wrongchoice{solid.}
    \end{choices}
\end{question}
}

\element{astr}{
\begin{question}{ASTRI-2-Q13}
    The temperature of the center of the Sun is roughly:
    \begin{choices}
        \wrongchoice{\SI{10 000}{\kelvin}}
        \wrongchoice{\SI{100 000}{\kelvin}}
        \wrongchoice{\SI{1 000 000}{\kelvin}}
      \correctchoice{\SI{10 000 000}{\kelvin}}
        \wrongchoice{\SI{100 000 000}{\kelvin}}
    \end{choices}
\end{question}
}

\element{astr}{
\begin{question}{ASTRI-2-Q14}
    How does a gamma-ray photon,
        produced in the core of the Sun,
        emerge at the surface as a visible light photon?
    \begin{choices}
      \correctchoice{it loses energy through absorptions and re-emissions}
        \wrongchoice{it gains energy through absorptions and re-emissions}
        \wrongchoice{it does not actually change, all photons from the surface of the Sun are gamma-ray photons}
        \wrongchoice{it is not a gamma-ray photon, since all photons produced in the core of the Sun are visible light photons}
    \end{choices}
\end{question}
}

\element{astr}{
\begin{question}{ASTRI-2-Q15}
    The two most abundant elements in the Sun,
        with the most abundant given first, are:
    \begin{choices}
        \wrongchoice{carbon and oxygen.}
        \wrongchoice{iron and hydrogen.}
        \wrongchoice{helium and nitrogen.}
        \wrongchoice{nitrogen and helium.}
      \correctchoice{hydrogen and helium.}
    \end{choices}
\end{question}
}

\element{astr}{
\begin{question}{ASTRI-2-Q16}
    Hydrostatic equilibrium is an equilibrium between:
    \begin{choices}
        \wrongchoice{hydrogen and carbon.}
        \wrongchoice{water, hydrogen, and oxygen.}
      \correctchoice{gravity and outward pressure.}
        \wrongchoice{water and electric (static) charge.}
    \end{choices}
\end{question}
}

\element{astr}{
\begin{question}{ASTRI-2-Q17}
    Fusion in the Sun occurs between:
    \begin{choices}
      \correctchoice{protons.}
        \wrongchoice{electrons and protons.}
        \wrongchoice{electrons and neutrons.}
        \wrongchoice{neutrons and protons.}
        \wrongchoice{electrons.}
    \end{choices}
\end{question}
}

\element{astr}{
\begin{question}{ASTRI-2-Q18}
    What we see visually as the surface of the Sun is the:
    \begin{choices}
      \correctchoice{photosphere.}
        \wrongchoice{chromosphere.}
        \wrongchoice{corona.}
    \end{choices}
\end{question}
}

\element{astr}{
\begin{question}{ASTRI-2-Q19}
    The temperature of the solar corona is approximately:
    \begin{choices}
        \wrongchoice{\SI{5 000}{\kelvin}}
        \wrongchoice{\SI{10 000}{\kelvin}}
        \wrongchoice{\SI{50 000}{\kelvin}}
        \wrongchoice{\SI{100 000}{\kelvin}}
      \correctchoice{\SI{1 000 000}{\kelvin}}
    \end{choices}
\end{question}
}

\element{astr}{
\begin{question}{ASTRI-2-Q20}
    The absorption line spectrum of the Sun comes from the:
    \begin{choices}
        \wrongchoice{interior.}
      \correctchoice{photosphere.}
        \wrongchoice{chromosphere.}
        \wrongchoice{corona.}
    \end{choices}
\end{question}
}

\element{astr}{
\begin{question}{ASTRI-2-Q21}
    The solar wind is mostly associated with:
    \begin{choices}
        \wrongchoice{sunspots.}
        \wrongchoice{prominences.}
      \correctchoice{coronal holes.}
        \wrongchoice{Jupiter's magnetic field.}
    \end{choices}
\end{question}
}

\element{astr}{
\begin{question}{ASTRI-2-Q22}
    A complete period of the solar magnetic field cycle (including polarity) takes:
    \begin{choices}
        \wrongchoice{one year.}
        \wrongchoice{11 years.}
      \correctchoice{22 years.}
        \wrongchoice{33 years.}
        \wrongchoice{44 years.}
    \end{choices}
\end{question}
}

\element{astr}{
\begin{question}{ASTRI-2-Q23}
    Solar oscillations are studied by means of:
    \begin{choices}
        \wrongchoice{observing periodic changes in the Sun's angular size.}
      \correctchoice{Doppler shift observations.}
        \wrongchoice{observing strong color changes in the Sun.}
        \wrongchoice{solar eclipses.}
    \end{choices}
\end{question}
}

\element{astr}{
\begin{question}{ASTRI-2-Q24}
    Which of the following statements concerning solar neutrinos is correct?
    \begin{choices}
      \correctchoice{less have been observed than predicted.}
        \wrongchoice{observed and predicted numbers are equal.}
        \wrongchoice{more have been observed than predicted.}
    \end{choices}
\end{question}
}

\element{astr}{
\begin{question}{ASTRI-2-Q25}
    Stellar diameters may be determined from studies of \rule[-0.1pt]{4em}{0.1pt} stars.
    \begin{choices}
        \wrongchoice{visual binary}
        \wrongchoice{astrometric binary}
        \wrongchoice{spectroscopic binary}
      \correctchoice{eclipsing binary}
    \end{choices}
\end{question}
}

\element{astr}{
\begin{question}{ASTRI-2-Q26}
    From knowledge of only a star's temperature and luminosity,
        we can determine its:
    \begin{choices}
        \wrongchoice{mass.}
      \correctchoice{radius.}
        \wrongchoice{distance.}
        \wrongchoice{period of rotation.}
        \wrongchoice{rotational velocity. }
    \end{choices}
\end{question}
}

\element{astr}{
\begin{question}{ASTRI-2-Q27}
    Stellar masses are most easily determined with use of:
    \begin{choices}
        \wrongchoice{Newton's first law.}
        \wrongchoice{Newton's second law as modified by Kepler.}
        \wrongchoice{Kepler's second law.}
      \correctchoice{Kepler's third law as modified by Newton.}
    \end{choices}
\end{question}
}

\element{astr}{
\begin{question}{ASTRI-2-Q28}
    The mass-luminosity relation for main sequence stars says:
    \begin{choices}
      \correctchoice{high mass, high luminosity.}
        \wrongchoice{high mass, low luminosity.}
        \wrongchoice{luminosity is constant for all masses.}
        \wrongchoice{luminosity is independent of mass.}
    \end{choices}
\end{question}
}

\element{astr}{
\begin{question}{ASTRI-2-Q29}
    What is the average temperature of interstellar gas and dust?
    \begin{multicols}{2}
    \begin{choices}
        \wrongchoice{\SI{3}{\kelvin}}
      \correctchoice{\SI{100}{\kelvin}}
        \wrongchoice{\SI{1,000}{\kelvin}}
        \wrongchoice{\SI{0}{\kelvin}}
    \end{choices}
    \end{multicols}
\end{question}
}

\element{astr}{
\begin{question}{ASTRI-2-Q30}
    Which two ingredients are needed to make an emission nebula or HII region?
    \begin{choices}
        \wrongchoice{Interstellar gas and dust}
        \wrongchoice{Cool stars and interstellar dust}
      \correctchoice{Hot stars and interstellar gas}
        \wrongchoice{Cool stars and interstellar gas}
    \end{choices}
\end{question}
}
       
% ANSWERS TO PRACTICE EXAM 2
% 1. a
% 2. a
% 3. b
% 4. c
% 5. c
% 6. a
% 7. b
% 8. c
% 9. a
% 10. d
% 11. c
% 12. a
% 13. d
% 14. a
% 15. e
% 16. c
% 17. a
% 18. a
% 19. e
% 20. b
% 21. c
% 22. c
% 23. b
% 24. a
% 25. d
% 26. b
% 27. d
% 28. a
% 29. b
% 30. c


\endinput


