

%% Physics 101 Sample Test Questions by Dr. James Pierce
%%------------------------------------------------------------


%% JP's Physics 101 Test Bank 2
%%--------------------------------------------------


%% Topic: Collisions
\element{jpierce}{
\begin{question}{pt101tb2-Q01}
    The total momentum before a collision is equal to the total momentum after the collision.
    \begin{choices}
        \wrongchoice{This is true only for collisions between objects moving in the same direction.}
        \wrongchoice{This is true for elastic collisions but not for inelastic collisions.}
      \correctchoice{This is true for any collision.}
        \wrongchoice{This is not true for any collisions.}
        \wrongchoice{This is true for inelastic collisions but not for elastic collisions.}
    \end{choices}
\end{question}
}

\element{jpierce}{
\begin{question}{pt101tb2-Q02}
    An elastic collision is one in which:
    \begin{choices}
      \correctchoice{no lasting deformation of either object occurs.}
        \wrongchoice{both of the two colliding objects are made of a rubbery material.}
        \wrongchoice{lasting deformation occurs for both of the colliding objects.}
        \wrongchoice{lasting deformation occurs for one of the two colliding objects.}
        \wrongchoice{one of the two colliding objects is made of a rubbery material.}
    \end{choices}
\end{question}
}

\element{jpierce}{
\begin{question}{pt101tb2-Q03}
    An inelastic collision is one in which:
    \begin{choices}
        \wrongchoice{both of the two colliding objects are made of a very rigid material.}
        \wrongchoice{both of the two colliding objects are made of a rubbery material.}
      \correctchoice{the two colliding objects deform, generate heat, or stick together.}
        \wrongchoice{no lasting deformation of either object occurs.}
        \wrongchoice{one of the two colliding objects is made of a rubbery material.}
    \end{choices}
\end{question}
}

\element{jpierce}{
\begin{question}{pt101tb2-Q04}
    Which of these is the most accurate statement about momentum in a collision between two objects?
    \begin{choices}
        \wrongchoice{Momentum is only conserved if the colliding objects bounce apart.}
        \wrongchoice{Momentum is only conserved if the collision is elastic.}
        \wrongchoice{Momentum is never conserved.}
      \correctchoice{Momentum is always conserved.}
        \wrongchoice{Momentum is only conserved if the collision is inelastic.}
    \end{choices}
\end{question}
}

\element{jpierce}{
\begin{question}{pt101tb2-Q05}
    A freight car moving at \SI{20}{\meter\per\second} to the right strikes
        a stationary freight car of the same mass.
    If the collision is inelastic,
    \begin{choices}
        \wrongchoice{the first car will move left and the second car will move right, both at \SI{20}{\meter\per\second}.}
        \wrongchoice{the first car will stop and the second car will move away at \SI{20}{\meter\per\second} to the right.}
        \wrongchoice{the first car will move left and the second car will move right, both at \SI{10}{\meter\per\second}.}
        \wrongchoice{both cars will move together to the right at \SI{20}{\meter\per\second}.}
      \correctchoice{both cars will move together to the right at \SI{10}{\meter\per\second}.}
    \end{choices}
\end{question}
}

\element{jpierce}{
\begin{question}{pt101tb2-Q06}
    A freight car moving at \SI{20}{\meter\per\second} to the right strikes
        a stationary freight car of the same mass.
    If the collision is elastic,
    \begin{choices}
        \wrongchoice{the first car will move left and the second car will move right, both at \SI{10}{\meter\per\second}.}
      \correctchoice{the first car will stop and the second car will move away at \SI{20}{\meter\per\second} to the right.}
        \wrongchoice{both cars will move together to the right at \SI{10}{\meter\per\second}.}
        \wrongchoice{both cars will move together to the right at \SI{20}{\meter\per\second}.}
        \wrongchoice{the first car will move left and the second car will move right, both at \SI{20}{\meter\per\second}.}
    \end{choices}
\end{question}
}

\element{jpierce}{
\begin{question}{pt101tb2-Q07}
    A green ball moving to the right at \SI{3}{\meter\per\second} strikes a
        yellow ball moving to the left at \SI{2}{\meter\per\second}.
    If the balls are equally massive and the collision is elastic,
    \begin{choices}
      \correctchoice{the green ball will move to the left at \SI{2}{\meter\per\second} while the yellow ball moves right at \SI{3}{\meter\per\second}.}
        \wrongchoice{Both balls will stick together and move to the right at \SI{1}{\meter\per\second}.}
        \wrongchoice{The yellow ball will stop while the green ball moves left at \SI{3}{\meter\per\second}.}
        \wrongchoice{the green ball will move to the left at \SI{3}{\meter\per\second} while the yellow ball moves right at \SI{2}{\meter\per\second}.}
        \wrongchoice{The green ball will stop while the yellow ball moves right at \SI{2}{\meter\per\second}.}
    \end{choices}
\end{question}
}

\element{jpierce}{
\begin{question}{pt101tb2-Q08}
    A green ball moving to the right at \SI{3}{\meter\per\second} strikes a
        yellow ball moving to the left at \SI{2}{\meter\per\second}.
    If the balls are equally massive and the collision is inelastic,
    \begin{choices}
      \correctchoice{Both balls will stick together and move to the right at \SI{1}{\meter\per\second}.}
        \wrongchoice{The yellow ball will stop while the green ball moves left at \SI{3}{\meter\per\second}.}
        \wrongchoice{The green ball will stop while the yellow ball moves right at \SI{2}{\meter\per\second}.}
        \wrongchoice{the green ball will move to the left at \SI{2}{\meter\per\second} while the yellow ball moves right at \SI{3}{\meter\per\second}.}
        \wrongchoice{the green ball will move to the left at \SI{3}{\meter\per\second} while the yellow ball moves right at \SI{2}{\meter\per\second}.}
    \end{choices}
\end{question}
}

\element{jpierce}{
\begin{question}{pt101tb2-Q09}
    A freight car moving at \SI{30}{\meter\per\second} to the right strikes
        a stationary freight car of the same mass. 
    If the two cars couple together,
        what will be their velocity after the collision?
    \begin{choices}
        \wrongchoice{\SI{30}{\meter\per\second} to the right}
        \wrongchoice{\SI{15}{\meter\per\second} to the left}
      \correctchoice{\SI{15}{\meter\per\second} to the right}
        \wrongchoice{\SI{30}{\meter\per\second} to the left}
        \wrongchoice{\SI{0}{\meter\per\second}}
    \end{choices}
\end{question}
}

\element{jpierce}{
\begin{question}{pt101tb2-Q10}
    A green ball moving to the right at \SI{2}{\meter\per\second}
        strikes a yellow ball moving to the left at \SI{3}{\meter\per\second}.
    If the balls are equally massive and the collision is elastic,
    \begin{choices}
        \wrongchoice{the green ball will move to the left at \SI{2}{\meter\per\second} while the yellow ball moves right at \SI{3}{\meter\per\second}.}
        \wrongchoice{the yellow ball will stop while the green ball moves left at \SI{1}{\meter\per\second}.}
      \correctchoice{the green ball will move to the left at \SI{3}{\meter\per\second} while the yellow ball moves right at \SI{2}{\meter\per\second}.}
        \wrongchoice{the yellow ball will stop while the green ball moves left at \SI{3}{\meter\per\second}.}
        \wrongchoice{the green ball will stop while the yellow ball moves right at \SI{2}{\meter\per\second}.}
    \end{choices}
\end{question}
}

%% Topic: Impulse
\element{jpierce}{
\begin{question}{pt101tb2-Q11}
    Impulse is the product of:
    \begin{choices}
        \wrongchoice{force and velocity.}
        \wrongchoice{velocity and acceleration.}
        \wrongchoice{mass and acceleration.}
      \correctchoice{force and time.}
        \wrongchoice{force and inertia.}
    \end{choices}
\end{question}
}

\element{jpierce}{
\begin{question}{pt101tb2-Q12}
    Impulse is equal to the change in \rule[-0.1pt]{4em}{0.1pt} of the
        object on which the impulse acts.
    \begin{multicols}{2}
    \begin{choices}
        \wrongchoice{force.}
        \wrongchoice{acceleration.}
        \wrongchoice{velocity.}
      \correctchoice{momentum.}
        \wrongchoice{mass.}
    \end{choices}
    \end{multicols}
\end{question}
}

\element{jpierce}{
\begin{question}{pt101tb2-Q13}
    In order to minimize the force acting on your bare hand when you catch a baseball,
        you should:
    \begin{choices}
        \wrongchoice{move your hand toward the ball as you catch it.}
        \wrongchoice{think happy thoughts while you catch the ball.}
        \wrongchoice{keep your hand as motionless as possible as you catch the ball.}
        \wrongchoice{let the ball bounce off your hand as you catch it.}
      \correctchoice{move your hand away from the ball as you catch it.}
    \end{choices}
\end{question}
}

\element{jpierce}{
\begin{question}{pt101tb2-Q14}
    Impulse has the same units as:
    \begin{multicols}{2}
    \begin{choices}
        \wrongchoice{time.}
        \wrongchoice{force.}
      \correctchoice{momentum.}
        \wrongchoice{acceleration.}
        \wrongchoice{mass.}
    \end{choices}
    \end{multicols}
\end{question}
}

\element{jpierce}{
\begin{question}{pt101tb2-Q15}
    An impulse of \SI{100}{\newton\second} is applied to an object.
    If this same impulse is delivered over a longer time interval,
    \begin{choices}
        \wrongchoice{the acceleration involved will be increased.}
      \correctchoice{the force involved will be decreased.}
        \wrongchoice{the momentum transferred will be decreased.}
        \wrongchoice{the momentum transferred will be increased.}
        \wrongchoice{the force involved will be increased.}
    \end{choices}
\end{question}
}

\element{jpierce}{
\begin{question}{pt101tb2-Q16}
    An impulse of \SI{100}{\newton\second} is applied to an object.
    If this same impulse is delivered over a shorter time interval,
    \begin{choices}
        \wrongchoice{the acceleration involved will be decreased.}
      \correctchoice{the force involved will be increased.}
        \wrongchoice{the momentum transferred will be increased.}
        \wrongchoice{the momentum transferred will be decreased.}
        \wrongchoice{the force involved will be decreased.}
    \end{choices}
\end{question}
}

\element{jpierce}{
\begin{question}{pt101tb2-Q17}
    Whether you slam on the brakes or apply a steady,
        moderate pressure to the brake pedal,
        the \rule[-0.1pt]{4em}{0.1pt} required to bring your
        car to a stop will be the same.
    \begin{multicols}{2}
    \begin{choices}
        \wrongchoice{time.}
      \correctchoice{impulse.}
        \wrongchoice{distance.}
        \wrongchoice{force.}
        \wrongchoice{acceleration.}
    \end{choices}
    \end{multicols}
\end{question}
}

\element{jpierce}{
\begin{question}{pt101tb2-Q18}
    \rule[-0.1pt]{4em}{0.1pt} is equal to the change in momentum
        of the object on which it acts.
    \begin{multicols}{2}
    \begin{choices}
        \wrongchoice{Mass}
        \wrongchoice{Velocity}
        \wrongchoice{Acceleration}
      \correctchoice{Impulse}
        \wrongchoice{Force}
    \end{choices}
    \end{multicols}
\end{question}
}

\element{jpierce}{
\begin{question}{pt101tb2-Q19}
    When a bullet is fired from a rifle,
    \begin{choices}
        \wrongchoice{the rifle exerts a lesser impulse on the bullet than the bullet exerts on the rifle.}
        \wrongchoice{the rifle exerts a lesser force on the bullet than the bullet exerts on the rifle.}
        \wrongchoice{the rifle and the bullet exert impulses of equal magnitude on each other.}
      \correctchoice{the rifle exerts a greater force on the bullet than the bullet exerts on the rifle.}
        \wrongchoice{the rifle exerts a greater impulse on the bullet than the bullet exerts on the rifle.}
    \end{choices}
\end{question}
}


%% Topic: Impulse*
\element{jpierce}{
\begin{question}{pt101tb2-Q20}
    Case 1: A net force of \SI{10}{\newton} acts on a mass of \SI{2}{\kilo\gram} for a time of \SI{0.1}{\second}.
    Case 2: A net force of \SI{10}{\newton} acts on a mass of \SI{2}{\kilo\gram} for a time of \SI{0.2}{\second}.
    Both cases result in acceleration of the mass.
    In comparison, Case 1 and Case 2 will:
    \begin{choices}
        \wrongchoice{involve the same impulse and produce different accelerations.}
      \correctchoice{involve different impulses and produce the same acceleration.}
        \wrongchoice{produce the same change of momentum.}
        \wrongchoice{involve different impulses and produce different accelerations.}
        \wrongchoice{involve the same impulse and produce the same acceleration.}
    \end{choices}
\end{question}
}

\element{jpierce}{
\begin{question}{pt101tb2-Q21}
    Case 1: A net force of \SI{10}{\newton} acts on a mass of \SI{1}{\kilo\gram} for a time of \SI{0.2}{\second}.
    Case 2: A net force of \SI{10}{\newton} acts on a mass of \SI{2}{\kilo\gram} for a time of \SI{0.2}{\second}.
    Both cases result in acceleration of the mass.
    In comparison, Case 1 and Case 2 will:
    \begin{choices}
        \wrongchoice{involve different impulses and produce different accelerations.}
        \wrongchoice{involve different impulses and produce the same acceleration.}
      \correctchoice{involve the same impulse and produce different accelerations.}
        \wrongchoice{produce different changes of momentum.}
        \wrongchoice{involve the same impulse and produce the same acceleration.}
    \end{choices}
\end{question}
}

\element{jpierce}{
\begin{question}{pt101tb2-Q22}
    Case 1: A net force of \SI{10}{\newton} acts on a mass of \SI{1}{\kilo\gram} for a time of \SI{0.2}{\second}.
    Case 2: A net force of \SI{20}{\newton} acts on a mass of \SI{1}{\kilo\gram} for a time of \SI{0.2}{\second}.
    Both cases result in acceleration of the mass. 
    In comparison, Case 1 and Case 2 will:
    \begin{choices}
        \wrongchoice{produce the same change of momentum.}
      \correctchoice{involve different impulses and produce different accelerations.}
        \wrongchoice{involve the same impulse and produce different accelerations.}
        \wrongchoice{involve different impulses and produce the same acceleration.}
        \wrongchoice{involve the same impulse and produce the same acceleration.}
    \end{choices}
\end{question}
}

\element{jpierce}{
\begin{question}{pt101tb2-Q23}
    Case 1: A net force of \SI{10}{\newton} acts on a mass of \SI{1}{\kilo\gram} for a time of \SI{0.2}{\second}.
    Case 2: A net force of \SI{20}{\newton} acts on a mass of \SI{2}{\kilo\gram} for a time of \SI{0.1}{\second}.
    Both cases result in acceleration of the mass. 
    In comparison, Case 1 and Case 2 will:
    \begin{choices}
        \wrongchoice{involve the same impulse and produce different accelerations.}
        \wrongchoice{involve different impulses and produce the same acceleration.}
        \wrongchoice{involve different impulses and produce different accelerations.}
        \wrongchoice{produce different changes of momentum.}
      \correctchoice{involve the same impulse and produce the same acceleration.}
    \end{choices}
\end{question}
}

\element{jpierce}{
\begin{question}{pt101tb2-Q24}
    Case 1: A net force of \SI{10}{\newton} acts on a mass of \SI{1}{\kilo\gram} for a time of \SI{0.2}{\second}.
    Case 2: A net force of \SI{20}{\newton} acts on a mass of \SI{1}{\kilo\gram} for a time of \SI{0.1}{\second}.
    Both cases result in acceleration of the mass. 
    In comparison, Case 1 and Case 2 will:
    \begin{choices}
        \wrongchoice{involve different impulses and produce the same acceleration.}
        \wrongchoice{involve the same impulse and produce the same acceleration.}
      \correctchoice{involve the same impulse and produce different accelerations.}
        \wrongchoice{produce different changes of momentum.}
        \wrongchoice{involve different impulses and produce different accelerations.}
    \end{choices}
\end{question}
}

\element{jpierce}{
\begin{question}{pt101tb2-Q25}
    Case 1: A net force of \SI{10}{\newton} acts on a mass of \SI{1}{\kilo\gram} for a time of \SI{0.1}{\second}.
    Case 2: A net force of \SI{20}{\newton} acts on a mass of \SI{2}{\kilo\gram} for a time of \SI{0.1}{\second}.
    Both cases result in acceleration of the mass. 
    In comparison, Case 1 and Case 2 will:
    \begin{choices}
        \wrongchoice{involve different impulses and produce different accelerations.}
      \correctchoice{involve different impulses and produce the same acceleration.}
        \wrongchoice{involve the same impulse and produce different accelerations.}
        \wrongchoice{involve the same impulse and produce the same acceleration.}
        \wrongchoice{produce different changes of momentum.}
    \end{choices}
\end{question}
}

%% Topic: Momentum
\element{jpierce}{
\begin{question}{pt101tb2-Q26}
    Momentum is the product of:
    \begin{choices}
        \wrongchoice{force and velocity.}
        \wrongchoice{force and inertia.}
        \wrongchoice{velocity and acceleration.}
      \correctchoice{mass and velocity.}
        \wrongchoice{mass and acceleration.}
    \end{choices}
\end{question}
}

\element{jpierce}{
\begin{question}{pt101tb2-Q27}
    If two speeding trucks have the same momentum,
    \begin{choices}
        \wrongchoice{they must have the same velocity.}
        \wrongchoice{the more massive truck must have a greater speed.}
        \wrongchoice{they must have the same acceleration.}
        \wrongchoice{they must have the same mass.}
      \correctchoice{the more massive truck must have a lower speed.}
    \end{choices}
\end{question}
}

\element{jpierce}{
\begin{question}{pt101tb2-Q28}
    When a bullet is fired from a rifle,
    \begin{choices}
        \wrongchoice{the momentum of the bullet is zero.}
      \correctchoice{the momentum of the rifle is equal and opposite to the momentum of the bullet.}
        \wrongchoice{the momentum of the bullet is greater than the momentum of the rifle.}
        \wrongchoice{the momentum of the rifle is zero.}
        \wrongchoice{the momentum of the rifle is greater than the momentum of the bullet.}
    \end{choices}
\end{question}
}

\element{jpierce}{
\begin{question}{pt101tb2-Q29}
    When a bullet is fired from a rifle,
    \begin{choices}
        \wrongchoice{the velocity of the rifle is zero.}
        \wrongchoice{the velocity of the rifle is greater than the velocity of the bullet.}
        \wrongchoice{the velocity of the bullet is zero.}
        \wrongchoice{the velocity of the rifle is equal and opposite to the velocity of the bullet.}
      \correctchoice{the velocity of the bullet is greater than the velocity of the rifle.}
    \end{choices}
\end{question}
}

\element{jpierce}{
\begin{question}{pt101tb2-Q30}
    If two speeding trucks have the same momentum,
    \begin{choices}
        \wrongchoice{the less massive truck must have a lower speed.}
      \correctchoice{the less massive truck must have a greater speed.}
        \wrongchoice{they must have the same mass.}
        \wrongchoice{they must have the same acceleration.}
        \wrongchoice{they must have the same velocity.}
    \end{choices}
\end{question}
}

\element{jpierce}{
\begin{question}{pt101tb2-Q31}
    When a bullet is fired from a rifle, the rifle and the bullet have
    \begin{choices}
        \wrongchoice{the same momentum, but the bullet has a greater inertia.}
        \wrongchoice{the same inertia and the same momentum.}
        \wrongchoice{the same inertia, but the rifle has a greater momentum.}
        \wrongchoice{the same inertia, but the bullet has a greater momentum.}
      \correctchoice{the same momentum, but the rifle has a greater inertia.}
    \end{choices}
\end{question}
}

\element{jpierce}{
\begin{question}{pt101tb2-Q32}
    The product of mass and velocity is called:
    \begin{multicols}{2}
    \begin{choices}
        \wrongchoice{kinetic energy}
        \wrongchoice{force}
        \wrongchoice{impulse}
      \correctchoice{momentum}
        \wrongchoice{collision}
    \end{choices}
    \end{multicols}
\end{question}
}

%% Topic: Momentum*
\element{jpierce}{
\begin{question}{pt101tb2-Q33}
    If a moving object doubles its speed, how much momentum will it have?
   The product of mass and velocity is called:
    \begin{choices}
        \wrongchoice{the same amount as before}
        \wrongchoice{four times as much as before}
      \correctchoice{twice as much as before}
        \wrongchoice{one half as much as before}
        \wrongchoice{six times as much as before}
    \end{choices}
\end{question}
}

\element{jpierce}{
\begin{question}{pt101tb2-Q34}
    If a Amoving object triples its speed, how much momentum will it have?
    \begin{choices}
         \wrongchoice{six times as much as before}
         \wrongchoice{one third as much as before}
         \wrongchoice{the same amount as before}
         \wrongchoice{nine times as much as before}
       \correctchoice{three times as much as before}
    \end{choices}
\end{question}
}

\element{jpierce}{
\begin{question}{pt101tb2-Q35}
    If a Amoving object cuts its speed in half, how much momentum will it have?
    \begin{choices}
        \wrongchoice{four times as much as before}
        \wrongchoice{one fourth as much as before}
        \wrongchoice{the same amount as before}
      \correctchoice{one half as much as before}
        \wrongchoice{twice as much as before}
    \end{choices}
\end{question}
}

\element{jpierce}{
\begin{question}{pt101tb2-Q36}
    A \SI{1}{\kilo\gram} ball moving horizontally to the right at \SI{3}{\meter\per\second}
        strikes a wall and rebounds, moving horizontally to the left at the same speed.
    What is the magnitude of the change in momentum of the ball?
    \begin{multicols}{2}
    \begin{choices}
        \wrongchoice{\SI{4}{\kilo\gram\meter\per\second}}
      \correctchoice{\SI{6}{\kilo\gram\meter\per\second}}
        \wrongchoice{\SI{0}{\kilo\gram\meter\per\second}}
        \wrongchoice{\SI{2}{\kilo\gram\meter\per\second}}
        \wrongchoice{\SI{3}{\kilo\gram\meter\per\second}}
    \end{choices}
    \end{multicols}
\end{question}
}

\element{jpierce}{
\begin{question}{pt101tb2-Q37}
    A \SI{2}{\kilo\gram} ball moving horizontally to the right at \SI{3}{\meter\per\second}
        strikes a wall and rebounds, moving horizontally to the left at the same speed. 
    What is the magnitude of the change in momentum of the ball?
    \begin{multicols}{2}
    \begin{choices}
      \correctchoice{\SI{12}{\kilo\gram\meter\per\second}}
        \wrongchoice{\SI{0}{\kilo\gram\meter\per\second}}
        \wrongchoice{\SI{4}{\kilo\gram\meter\per\second}}
        \wrongchoice{\SI{6}{\kilo\gram\meter\per\second}}
        \wrongchoice{\SI{18}{\kilo\gram\meter\per\second}}
    \end{choices}
    \end{multicols}
\end{question}
}

\element{jpierce}{
\begin{question}{pt101tb2-Q38}
    A mass of \SI{12}{\kilo\gram} moving to the right with a speed of
        \SI{4}{\meter\per\second} would have a momentum of:
        %\rule[-0.1pt]{4em}{0.1pt} \si{\kilo\gram\meter\per\second}.
    \begin{multicols}{3}
    \begin{choices}
        \wrongchoice{\SI{8}{\kilo\gram\meter\per\second}}
        \wrongchoice{\SI{16}{\kilo\gram\meter\per\second}}
        \wrongchoice{\SI{1/3}{\kilo\gram\meter\per\second}}
      \correctchoice{\SI{48}{\kilo\gram\meter\per\second}}
        \wrongchoice{\SI{3}{\kilo\gram\meter\per\second}}
    \end{choices}
    \end{multicols}
\end{question}
}

\element{jpierce}{
\begin{question}{pt101tb2-Q39}
    A mass of \SI{2}{\kilo\gram} moving to the right with a speed of
        \SI{6}{\meter\per\second} would have a momentum of:
        %\rule[-0.1pt]{4em}{0.1pt} \si{\kilo\gram\meter\per\second}.
    \begin{multicols}{3}
    \begin{choices}
        \wrongchoice{\SI{4}{\kilo\gram\meter\per\second}}
      \correctchoice{\SI{12}{\kilo\gram\meter\per\second}}
        \wrongchoice{\SI{1/3}{\kilo\gram\meter\per\second}}
        \wrongchoice{\SI{8}{\kilo\gram\meter\per\second}}
        \wrongchoice{\SI{3}{\kilo\gram\meter\per\second}}
    \end{choices}
    \end{multicols}
\end{question}
}

\element{jpierce}{
\begin{question}{pt101tb2-Q40}
    A mass of \SI{2}{\kilo\gram} moving to the right with a momentum of
        \SI{6}{\kilo\gram\meter\per\second} would have a speed of:
        %\rule[-0.1pt]{4em}{0.1pt} \si{\meter\per\second}.
    \begin{multicols}{3}
    \begin{choices}
        \wrongchoice{\SI{4}{\meter\per\second}}
        \wrongchoice{\SI{1/3}{\meter\per\second}}
        \wrongchoice{\SI{12}{\meter\per\second}}
      \correctchoice{\SI{3}{\meter\per\second}}
        \wrongchoice{\SI{8}{\meter\per\second}}
    \end{choices}
    \end{multicols}
\end{question}
}

\element{jpierce}{
\begin{question}{pt101tb2-Q41}
    A mass of \SI{3}{\kilo\gram} moving to the right with a momentum of
        \SI{12}{\kilo\gram\meter\per\second} would have a speed of:
        %\rule[-0.1pt]{4em}{0.1pt} \si{\meter\per\second}.
    \begin{multicols}{3}
    \begin{choices}
        \wrongchoice{\SI{36}{\meter\per\second}}
        \wrongchoice{\SI{15}{\meter\per\second}}
      \correctchoice{\SI{4}{\meter\per\second}}
        \wrongchoice{\SI{9}{\meter\per\second}}
        \wrongchoice{\SI{1/4}{\meter\per\second}}
    \end{choices}
    \end{multicols}
\end{question}
}

\element{jpierce}{
\begin{question}{pt101tb2-Q42}
    A mass of \SI{2}{\kilo\gram} moving to the right with a momentum of
        \SI{8}{\kilo\gram\meter\per\second} would have a speed of:
        %\rule[-0.1pt]{4em}{0.1pt} \si{\meter\per\second}.
    \begin{multicols}{3}
    \begin{choices}
        \wrongchoice{\SI{10}{\meter\per\second}}
      \correctchoice{\SI{4}{\meter\per\second}}
        \wrongchoice{\SI{16}{\meter\per\second}}
        \wrongchoice{\SI{2}{\meter\per\second}}
        \wrongchoice{\SI{8}{\meter\per\second}}
    \end{choices}
    \end{multicols}
\end{question}
}

\element{jpierce}{
\begin{question}{pt101tb2-Q43}
    A mass of \SI{4}{\kilo\gram} moving to the right with a momentum of
        \SI{12}{\kilo\gram\meter\per\second} would have a speed of:
        %\rule[-0.1pt]{4em}{0.1pt} \si{\meter\per\second}.
    \begin{multicols}{3}
    \begin{choices}
        \wrongchoice{\SI{12}{\meter\per\second}}
        \wrongchoice{\SI{4}{\meter\per\second}}
        \wrongchoice{\SI{48}{\meter\per\second}}
        \wrongchoice{\SI{8}{\meter\per\second}}
      \correctchoice{\SI{3}{\meter\per\second}}
    \end{choices}
    \end{multicols}
\end{question}
}

%% Topic: Collision
\element{jpierce}{
\begin{question}{pt101tb2-Q44}
    Which of these is the most accurate statement about
        kinetic energy in a collision between two objects?
    \begin{choices}
        \wrongchoice{Kinetic energy is only conserved if the colliding objects stick together.}
        \wrongchoice{Kinetic energy is only conserved if the collision is inelastic.}
      \correctchoice{Kinetic energy is only conserved if the collision is elastic.}
        \wrongchoice{Kinetic energy is always conserved.}
        \wrongchoice{Kinetic energy is never conserved.}
    \end{choices}
\end{question}
}

\element{jpierce}{
\begin{question}{pt101tb2-Q45}
    Which of these is the most accurate statement about
        kinetic energy in a collision between two objects?
    \begin{choices}
        \wrongchoice{Kinetic energy is only conserved if the collision is inelastic.}
        \wrongchoice{Kinetic energy is always conserved.}
        \wrongchoice{Kinetic energy is never conserved.}
      \correctchoice{Kinetic energy is only conserved if the colliding objects bounce apart.}
        \wrongchoice{Kinetic energy is only conserved if the colliding objects stick together.}
    \end{choices}
\end{question}
}

\element{jpierce}{
\begin{question}{pt101tb2-Q46}
    If a collision between two bodies is elastic,
    \begin{choices}
        \wrongchoice{the total momentum will be unchanged, but the total kinetic energy will be reduced.}
        \wrongchoice{the total kinetic energy will be unchanged, but the total momentum will be reduced.}
      \correctchoice{both the total momentum and the total kinetic energy will be unchanged.}
        \wrongchoice{each body will retain its original momentum after the collision.}
        \wrongchoice{each body will retain its original kinetic energy after the collision.}
    \end{choices}
\end{question}
}


%% Topic: Energy
\element{jpierce}{
\begin{question}{pt101tb2-Q47}
    Potential energy is the energy possessed by an object due to its
    \begin{choices}
        \wrongchoice{acceleration.}
        \wrongchoice{momentum.}
        \wrongchoice{velocity.}
        \wrongchoice{shape.}
      \correctchoice{position.}
    \end{choices}
\end{question}
}

\element{jpierce}{
\begin{question}{pt101tb2-Q48}
    Gravitational potential energy is the product of
    \begin{choices}
        \wrongchoice{mass and acceleration.}
        \wrongchoice{power and time.}
      \correctchoice{weight and height.}
        \wrongchoice{force and distance.}
        \wrongchoice{momentum and impulse.}
    \end{choices}
\end{question}
}

\element{jpierce}{
\begin{question}{pt101tb2-Q49}
    The formula for kinetic energy is ${KE = \rule[-0.1pt]{4em}{0.1pt}}$.
    \begin{multicols}{3}
    \begin{choices}
        \wrongchoice{$mv$}
      \correctchoice{$\dfrac{1}{2} mv^2$}
        \wrongchoice{$ma$}
        \wrongchoice{$Fd$}
        \wrongchoice{$mgh$}
        %% NOTE: add one for symmetry
        \wrongchoice{$Ft$}
    \end{choices}
    \end{multicols}
\end{question}
}

\element{jpierce}{
\begin{question}{pt101tb2-Q50}
    The formula for gravitational potential energy is ${PE = \rule[-0.1pt]{4em}{0.1pt}}$.
    \begin{multicols}{3}
    \begin{choices}
      \correctchoice{$mgh$}
        \wrongchoice{$ma$}
        \wrongchoice{$mv$}
        \wrongchoice{$\dfrac{1}{2} mv^2$}
        \wrongchoice{$Fd$}
        %% NOTE: add one for symmetry
        \wrongchoice{$Ft$}
    \end{choices}
    \end{multicols}
\end{question}
}

\element{jpierce}{
\begin{question}{pt101tb2-Q51}
    Which of the following is true?
    \begin{choices}
        \wrongchoice{A body with zero velocity cannot have any potential energy.}
        \wrongchoice{A body with zero acceleration cannot have any kinetic energy.}
        \wrongchoice{A body with zero potential energy cannot have any velocity.}
        \wrongchoice{A body with zero acceleration cannot have any potential energy.}
      \correctchoice{A body with zero velocity cannot have any kinetic energy.}
    \end{choices}
\end{question}
}

\element{jpierce}{
\begin{question}{pt101tb2-Q52}
    The unit of energy is the joule, which is equal to a:
    \begin{choices}
        \wrongchoice{kilogram meter (\si{\kilo\gram\meter}).}
        \wrongchoice{kilogram per second (\si{\kilo\gram\per\second}).}
        \wrongchoice{newton per second (\si{\newton\per\second}).}
        \wrongchoice{newton meter per second (\si{\newton\meter\per\second}).}
      \correctchoice{newton meter (\si{\newton\meter}).}
    \end{choices}
\end{question}
}

\element{jpierce}{
\begin{question}{pt101tb2-Q53}
    The kinetic energy of a body depends on its
    \begin{choices}
        \wrongchoice{mass and volume.}
        \wrongchoice{shape and acceleration.}
        \wrongchoice{shape and speed.}
        \wrongchoice{acceleration and volume.}
      \correctchoice{mass and speed.}
    \end{choices}
\end{question}
}

\element{jpierce}{
\begin{question}{pt101tb2-Q54}
    The gravitational potential energy of a body depends on its
    \begin{choices}
        \wrongchoice{speed and position.}
        \wrongchoice{mass and volume.}
      \correctchoice{weight and position.}
        \wrongchoice{speed and mass.}
        \wrongchoice{weight and volume.}
    \end{choices}
\end{question}
}

\element{jpierce}{
\begin{question}{pt101tb2-Q55}
    A skydiver weighing \SI{500}{\newton} jumps from an airplane
        at a height of \SI{2 000}{\meter}. 
    If there is no air resistance,
        the skydiver's kinetic energy will equal his potential energy
        relative to the ground when he is at a height of:
    \begin{multicols}{2}
    \begin{choices}
      \correctchoice{\SI{1000}{\meter}}
        \wrongchoice{\SI{500}{\meter}}
        \wrongchoice{\SI{1}{\meter}}
        \wrongchoice{\SI{2000}{\meter}}
        \wrongchoice{\SI{1500}{\meter}}
    \end{choices}
    \end{multicols}
\end{question}
}

\element{jpierce}{
\begin{question}{pt101tb2-Q56}
    A skydiver weighing \SI{500}{\newton} jumps from an airplane
        at a height of \SI{2 000}{\meter}. 
    At the start of the jump,
        the skydiver's kinetic energy is:
    \begin{multicols}{2}
    \begin{choices}
        \wrongchoice{\SI{1 500}{\joule}}
        \wrongchoice{\SI{2 500}{\joule}}
      \correctchoice{\SI{0}{\joule}}
        \wrongchoice{\SI{100 000}{\joule}}
        \wrongchoice{\SI{1 000 000}{\joule}}
    \end{choices}
    \end{multicols}
\end{question}
}

\element{jpierce}{
\begin{question}{pt101tb2-Q57}
    If a moving object doubles its speed,
        how much kinetic energy will it have?
    \begin{choices}
        \wrongchoice{one half as much as before}
        \wrongchoice{the same amount as before}
      \correctchoice{four times as much as before}
        \wrongchoice{twice as much as before}
        \wrongchoice{six times as much as before}
    \end{choices}
\end{question}
}

\element{jpierce}{
\begin{question}{pt101tb2-Q58}
    If a moving object triples its speed,
        how much kinetic energy will it have?
    \begin{choices}
        \wrongchoice{six times as much as before}
        \wrongchoice{three times as much as before}
        \wrongchoice{the same amount as before}
        \wrongchoice{one third as much as before}
      \correctchoice{nine times as much as before}
    \end{choices}
\end{question}
}

\element{jpierce}{
\begin{question}{pt101tb2-Q59}
    If a moving object cuts its speed in half,
        how much kinetic energy will it have?
    \begin{choices}
        \wrongchoice{the same amount as before}
        \wrongchoice{twice as much as before}
      \correctchoice{one fourth as much as before}
        \wrongchoice{four times as much as before}
        \wrongchoice{one half as much as before}
    \end{choices}
\end{question}
}

\element{jpierce}{
\begin{question}{pt101tb2-Q60}
    A car traveling \SI{90}{\kilo\meter\per\hour} has \rule[-0.1pt]{4em}{0.1pt} times
        the kinetic energy of the same car traveling \SI{30}{\kilo\meter\per\hour}.
    \begin{multicols}{3}
    \begin{choices}
        \wrongchoice{\num{30}}
        \wrongchoice{\num{3}}
      \correctchoice{\num{9}}
        \wrongchoice{\num{15}}
        \wrongchoice{\num{6}}
        %% NOTE: add one for symmetry
        \wrongchoice{$\sqrt{3}$}
    \end{choices}
    \end{multicols}
\end{question}
}

\element{jpierce}{
\begin{question}{pt101tb2-Q61}
    A car traveling \SI{80}{\kilo\meter\per\hour} has \rule[-0.1pt]{4em}{0.1pt} times
        the kinetic energy of the same car traveling \SI{20}{\kilo\meter\per\hour}.
    \begin{multicols}{3}
    \begin{choices}
        \wrongchoice{\num{2}}
      \correctchoice{\num{16}}
        \wrongchoice{\num{6}}
        \wrongchoice{\num{4}}
        \wrongchoice{\num{8}}
        %% NOTE: add one for symmetry
        \wrongchoice{$\sqrt{2}$}
    \end{choices}
    \end{multicols}
\end{question}
}

\element{jpierce}{
\begin{question}{pt101tb2-Q62}
    A skydiver weighing \SI{500}{\newton} jumps from an airplane at a height of \SI{2 000}{\meter}. 
    At the start of the jump,
        the skydiver's potential energy is \rule[-0.1pt]{4em}{0.1pt} relative to the ground.
    \begin{multicols}{2}
    \begin{choices}
        \wrongchoice{\SI{100 000}{\joule}}
        \wrongchoice{\SI{4}{\joule}}
        \wrongchoice{\SI{2 500}{\joule}}
      \correctchoice{\SI{1 000 000}{\joule}}
        \wrongchoice{\SI{1 500}{\joule}}
    \end{choices}
    \end{multicols}
\end{question}
}

%% Topic: Energy Conservation
\element{jpierce}{
\begin{question}{pt101tb2-Q63}
    A swinging pendulum has \rule[-0.1pt]{4em}{0.1pt} at
        the bottom (middle) of its arc.
    \begin{choices}
        \wrongchoice{maximum total energy}
        \wrongchoice{maximum potential energy}
        \wrongchoice{minimum kinetic energy}
      \correctchoice{minimum potential energy}
        \wrongchoice{minimum total energy}
    \end{choices}
\end{question}
}

\element{jpierce}{
\begin{question}{pt101tb2-Q64}
    A swinging pendulum has \rule[-0.1pt]{4em}{0.1pt} at
        the bottom (middle) of its arc.
    \begin{choices}
        \wrongchoice{maximum potential energy}
        \wrongchoice{maximum total energy}
        \wrongchoice{minimum total energy}
        \wrongchoice{minimum kinetic energy}
      \correctchoice{maximum kinetic energy}
    \end{choices}
\end{question}
}

\element{jpierce}{
\begin{question}{pt101tb2-Q65}
    A swinging pendulum has \rule[-0.1pt]{4em}{0.1pt} at
        the top (end) of its arc.
    \begin{choices}
        \wrongchoice{maximum kinetic energy}
      \correctchoice{minimum kinetic energy}
        \wrongchoice{minimum total energy}
        \wrongchoice{maximum total energy}
        \wrongchoice{minimum potential energy}
    \end{choices}
\end{question}
}

\element{jpierce}{
\begin{question}{pt101tb2-Q66}
    A swinging pendulum has \rule[-0.1pt]{4em}{0.1pt} at
        the top (end) of its arc.
    \begin{choices}
      \correctchoice{maximum potential energy}
        \wrongchoice{minimum total energy}
        \wrongchoice{minimum potential energy}
        \wrongchoice{maximum kinetic energy}
        \wrongchoice{maximum total energy}
    \end{choices}
\end{question}
}

\element{jpierce}{
\begin{question}{pt101tb2-Q67}
    If a swinging pendulum has \SI{4}{\joule} of kinetic energy
        at the bottom (middle) of its arc,
        its potential energy at the top (end) of its arc will
        be \rule[-0.1pt]{4em}{0.1pt} its potential energy at the bottom (middle) of the arc.
    \begin{choices}
      \correctchoice{\SI{4}{\joule} more than}
        \wrongchoice{the same as}
        \wrongchoice{\SI{2}{\joule} more than}
        \wrongchoice{\SI{2}{\joule} less than}
        \wrongchoice{\SI{4}{\joule} less than}
    \end{choices}
\end{question}
}

\element{jpierce}{
\begin{question}{pt101tb2-Q68}
    If a swinging pendulum has 4 joules of kinetic energy at the bottom (middle) of its arc,
        its total energy at top (end) of its arc will be \rule[-0.1pt]{4em}{0.1pt} its
        total energy at the bottom (middle) of the arc.
    \begin{choices}
        \wrongchoice{2 joules less than}
        \wrongchoice{4 joules more than}
        \wrongchoice{4 joules less than}
        \wrongchoice{2 joules more than}
      \correctchoice{the same as}
    \end{choices}
\end{question}
}

\element{jpierce}{
\begin{question}{pt101tb2-Q69}
    If a swinging pendulum has 2 joules of kinetic energy at the bottom (middle) of its arc,
        its potential energy at the top (end) of its arc will be \rule[-0.1pt]{4em}{0.1pt} its
        potential energy at the bottom (middle) of the arc.
    \begin{choices}
        \wrongchoice{the same as}
        \wrongchoice{4 joules more than}
        \wrongchoice{2 joules less than}
        \wrongchoice{4 joules less than}
      \correctchoice{2 joules more than}
    \end{choices}
\end{question}
}

\element{jpierce}{
\begin{question}{pt101tb2-Q70}
    If a swinging pendulum has 2 joules of kinetic energy at the bottom (middle) of its arc,
        its total energy at top (end) of its arc will be \rule[-0.1pt]{4em}{0.1pt} its
        total energy at the bottom (middle) of the arc.
    \begin{choices}
        \wrongchoice{4 joules less than}
        \wrongchoice{2 joules more than}
        \wrongchoice{2 joules less than}
      \correctchoice{the same as}
        \wrongchoice{4 joules more than}
    \end{choices}
\end{question}
}


%% Topic: KE/Momentum
\element{jpierce}{
\begin{question}{pt101tb2-Q71}
    Two identical balls of clay rolling in opposite directions collide,
        stick together, and stop. 
    In this collision
    \begin{choices}
        \wrongchoice{neither momentum nor kinetic energy were conserved.}
        \wrongchoice{momentum and kinetic energy were both conserved.}
      \correctchoice{momentum was conserved, but kinetic energy was not.}
        \wrongchoice{kinetic energy was conserved, but momentum was not.}
        \wrongchoice{none of the above are true.}
    \end{choices}
\end{question}
}

\element{jpierce}{
\begin{question}{pt101tb2-Q72}
    If two objects of different mass have the same non-zero momentum,
    \begin{choices}
      \correctchoice{the one with less mass will have the greater kinetic energy.}
        \wrongchoice{the one with more mass will have the greater kinetic energy.}
        \wrongchoice{they will have the same kinetic energy.}
        \wrongchoice{the one with the lower speed will have the greater kinetic energy.}
        \wrongchoice{the one with the higher speed will have the greater mass.}
    \end{choices}
\end{question}
}

\element{jpierce}{
\begin{question}{pt101tb2-Q73}
    If two objects of different mass have the same non-zero momentum,
    \begin{choices}
        \wrongchoice{the one with the higher speed will have the greater mass.}
        \wrongchoice{the one with less mass will have less kinetic energy.}
        \wrongchoice{they will have the same kinetic energy.}
        \wrongchoice{the one with the lower speed will have the greater kinetic energy.}
      \correctchoice{the one with more mass will have less kinetic energy.}
    \end{choices}
\end{question}
}

\element{jpierce}{
\begin{question}{pt101tb2-Q74}
    A car traveling at \SI{60}{\kilo\meter\per\hour} passes a truck going \SI{30}{\kilo\meter\per\hour}.
    If the truck has twice the mass of the car,
        which of the following is true?
    \begin{choices}
        \wrongchoice{The car has the same momentum and four times as much kinetic energy as the truck.}
        \wrongchoice{The car has the same kinetic energy and twice as much momentum as the truck.}
        \wrongchoice{The car and the truck have the same momentum and the same kinetic energy.}
        \wrongchoice{The car has the same kinetic energy and half as much momentum as the truck.}
      \correctchoice{The car has the same momentum and twice as much kinetic energy as the truck.}
    \end{choices}
\end{question}
}

\element{jpierce}{
\begin{question}{pt101tb2-Q75}
    A car traveling at \SI{60}{\kilo\meter\per\hour} passes a truck going
        \SI{30}{\kilo\meter\per\hour} that has four times the mass of the car.
    Which of the following is true?
    \begin{choices}
        \wrongchoice{The car and the truck have the same momentum and the same kinetic energy.}
        \wrongchoice{The car has the same kinetic energy and twice as much momentum as the truck.}
        \wrongchoice{The car has the same momentum and twice as much kinetic energy as the truck.}
      \correctchoice{The car has the same kinetic energy and half as much momentum as the truck.}
        \wrongchoice{The car has the same momentum and four times as much kinetic energy as the truck.}
    \end{choices}
\end{question}
}

\element{jpierce}{
\begin{question}{pt101tb2-Q76}
    If two objects of different mass have the same non-zero kinetic energy,
    \begin{choices}
      \correctchoice{the one with more mass will have the greater momentum.}
        \wrongchoice{they will have the same momentum.}
        \wrongchoice{the one with the higher speed will have the greater mass.}
        \wrongchoice{the one with the higher speed will have the greater momentum.}
        \wrongchoice{the one with less mass will have the greater momentum.}
    \end{choices}
\end{question}
}

\element{jpierce}{
\begin{question}{pt101tb2-Q77}
    A car traveling at \SI{60}{\kilo\meter\per\hour} passes a truck going
        \SI{30}{\kilo\meter\per\hour} that has twice the mass of the car. 
    Which of the following is true?
    \begin{choices}
        \wrongchoice{The car has the same momentum and four times as much kinetic energy as the truck.}
        \wrongchoice{The car and the truck have the same momentum and the same kinetic energy.}
        \wrongchoice{The car has the same kinetic energy and twice as much momentum as the truck.}
      \correctchoice{The car has the same momentum and twice as much kinetic energy as the truck.}
        \wrongchoice{The car has the same kinetic energy and half as much momentum as the truck.}
    \end{choices}
\end{question}
}


%% Topic: Machines
\element{jpierce}{
\begin{question}{pt101tb2-Q78}
    When using a simple lever to raise a heavy object,
        the \rule[-0.1pt]{4em}{0.1pt} input must equal
        the \rule[-0.1pt]{4em}{0.1pt} output if frictional forces are neglected.
    \begin{choices}
      \correctchoice{work; work}
        \wrongchoice{momentum; momentum}
        \wrongchoice{impulse; impulse}
        \wrongchoice{force; force}
        \wrongchoice{acceleration; acceleration}
    \end{choices}
\end{question}
}

\element{jpierce}{
\begin{question}{pt101tb2-Q79}
    When using a jack as a lever to raise one end of a car off the ground,
        you are applying a relatively \rule[-0.1pt]{4em}{0.1pt} force
        to raise the car a relatively \rule[-0.1pt]{4em}{0.1pt} distance
        for each push of the jack handle.
    \begin{choices}
        \wrongchoice{large; large}
      \correctchoice{small; small}
        \wrongchoice{small; large}
        \wrongchoice{large; small}
        \wrongchoice{None of the above---a jack cannot be used as a lever.}
    \end{choices}
\end{question}
}

\element{jpierce}{
\begin{question}{pt101tb2-Q80}
    When using a jack as a lever to raise one end of a car off the ground,
        the relatively \rule[-0.1pt]{4em}{0.1pt} force applied to the jack handle
        translates into a relatively \rule[-0.1pt]{4em}{0.1pt} force on the car.
    \begin{choices}
        \wrongchoice{large; large}
      \correctchoice{small; large}
        \wrongchoice{small; small}
        \wrongchoice{large; small}
        \wrongchoice{None of the above---a jack cannot be used as a lever.}
    \end{choices}
\end{question}
}

\element{jpierce}{
\begin{question}{pt101tb2-Q81}
    When using a jack as a lever to raise one end of a car off the ground,
        the jack handle is moved a relatively \rule[-0.1pt]{4em}{0.1pt} distance
        in order to lift the car a relatively \rule[-0.1pt]{4em}{0.1pt} distance.
    \begin{choices}
         \wrongchoice{large; large}
       \correctchoice{large; small}
         \wrongchoice{small; large}
         \wrongchoice{small; small}
         \wrongchoice{None of the above---a jack cannot be used as a lever.}
    \end{choices}
\end{question}
}

\element{jpierce}{
\begin{question}{pt101tb2-Q82}
    Efficiency is the ratio of:
    \begin{choices}
      \correctchoice{useful energy output to total energy input.}
        \wrongchoice{useful energy input to total energy input.}
        \wrongchoice{total energy input to total energy output.}
        \wrongchoice{useful energy input to total energy output.}
        \wrongchoice{useful energy output to total energy output.}
    \end{choices}
\end{question}
}

\element{jpierce}{
\begin{question}{pt101tb2-Q83}
    Real machines are not \SI{100}{\percent} efficient because
    \begin{choices}
        \wrongchoice{that would require the work output to be 100 times the work input, which is impossible.}
        \wrongchoice{the energy input is always less than the energy output.}
      \correctchoice{some of the energy input is always transformed into thermal energy.}
        \wrongchoice{some of the energy input is always transformed into gravitational potential energy.}
        \wrongchoice{that would require the work input to be 100 times the work output, which is impossible.}
    \end{choices}
\end{question}
}

\element{jpierce}{
\begin{question}{pt101tb2-Q84}
    A physicist does \SI{100}{\joule} of work on a simple machine that raises
        a box of books through a height of \SI{0.4}{\meter}. 
    If the efficiency of the machine is \SI{80}{\percent},
        how much work is converted to thermal energy by this process?
    \begin{multicols}{3}
    \begin{choices}
      \correctchoice{\SI{20}{\joule}}
        \wrongchoice{\SI{60}{\joule}}
        \wrongchoice{\SI{40}{\joule}}
        \wrongchoice{\SI{80}{\joule}}
        \wrongchoice{\SI{100}{\joule}}
    \end{choices}
    \end{multicols}
\end{question}
}

\element{jpierce}{
\begin{question}{pt101tb2-Q85}
    A physicist does \SI{100}{\joule} of work on a simple machine that raises
        a box of books through a height of \SI{0.6}{\meter}.
    If the efficiency of the machine is \SI{20}{\percent},
        how much work is converted to thermal energy by this process?
    \begin{multicols}{3}
    \begin{choices}
        \wrongchoice{\SI{60}{\joule}}
      \correctchoice{\SI{80}{\joule}}
        \wrongchoice{\SI{100}{\joule}}
        \wrongchoice{\SI{20}{\joule}}
        \wrongchoice{\SI{40}{\joule}}
    \end{choices}
    \end{multicols}
\end{question}
}

\element{jpierce}{
\begin{question}{pt101tb2-Q86}
    A physicist does \SI{100}{\joule} of work on a simple machine that raises
        a box of books through a height of \SI{0.8}{\meter}.
    If the efficiency of the machine is \SI{40}{\percent},
        how much work is converted to thermal energy by this process?
    \begin{multicols}{3}
    \begin{choices}
      \correctchoice{\SI{60}{\joule}}
        \wrongchoice{\SI{80}{\joule}}
        \wrongchoice{\SI{20}{\joule}}
        \wrongchoice{\SI{100}{\joule}}
        \wrongchoice{\SI{40}{\joule}}
    \end{choices}
    \end{multicols}
\end{question}
}

\element{jpierce}{
\begin{question}{pt101tb2-Q87}
    A physicist does \SI{100}{\joule} of work on a simple machine that raises
        a box of books through a height of \SI{0.2}{\meter}.
    If the efficiency of the machine is \SI{60}{\percent},
        how much work is converted to thermal energy by this process?
    \begin{multicols}{3}
    \begin{choices}
        \wrongchoice{\SI{60}{\joule}}
        \wrongchoice{\SI{20}{\joule}}
      \correctchoice{\SI{40}{\joule}}
        \wrongchoice{\SI{100}{\joule}}
        \wrongchoice{\SI{80}{\joule}}
    \end{choices}
    \end{multicols}
\end{question}
}


%% Topic: Work/Power
\element{jpierce}{
\begin{question}{pt101tb2-Q88}
    Work is equal to the product of:
    \begin{choices}
        \wrongchoice{velocity and time.}
        \wrongchoice{mass and velocity.}
        \wrongchoice{mass and acceleration.}
        \wrongchoice{force and time.}
      \correctchoice{force and distance.}
    \end{choices}
\end{question}
}

\element{jpierce}{
\begin{question}{pt101tb2-Q89}
    \rule[-0.1pt]{4em}{0.1pt} is the rate at which work is done.
    \begin{choices}
        \wrongchoice{Kinetic energy}
        \wrongchoice{Impulse}
      \correctchoice{Power}
        \wrongchoice{Potential energy}
        \wrongchoice{Momentum}
    \end{choices}
\end{question}
}

\element{jpierce}{
\begin{question}{pt101tb2-Q90}
    When you run up two flights of stairs instead of walking up them,
        you feel more tired because
    \begin{choices}
      \correctchoice{your power output is greater when you run than when you walk.}
        \wrongchoice{a running person has more inertia than a walking person.}
        \wrongchoice{you do more work when you run than when you walk.}
        \wrongchoice{the gravitational force is greater on a running person than on a walking person.}
        \wrongchoice{the gravitational acceleration is greater on a running person than on a walking person.}
    \end{choices}
\end{question}
}

\element{jpierce}{
\begin{question}{pt101tb2-Q91}
    The work required to move a bowling ball from the sidewalk to the top of a tall building is:
    \begin{choices}
      \correctchoice{equal to the weight of the ball times the height of the building.}
        \wrongchoice{equal to the mass of the ball times the acceleration of gravity.}
        \wrongchoice{equal to the mass of the ball times the speed at which it is moved to the top of the building.}
        \wrongchoice{equal to the impulse applied to the ball.}
        \wrongchoice{equal to the mass of the ball times the height of the building.}
    \end{choices}
\end{question}
}

\element{jpierce}{
\begin{question}{pt101tb2-Q92}
    \rule[-0.1pt]{4em}{0.1pt} is the rate at
        which \rule[-0.1pt]{4em}{0.1pt} is done.
    \begin{multicols}{2}
    \begin{choices}
        \wrongchoice{Friction; power}
        \wrongchoice{Inertia; acceleration}
        \wrongchoice{Work; power}
        \wrongchoice{Energy; work}
      \correctchoice{Power; work}
    \end{choices}
    \end{multicols}
\end{question}
}

\element{jpierce}{
\begin{question}{pt101tb2-Q93}
    The work done against gravity in moving a box with a mass of
        \SI{3}{\kilo\gram} through a horizontal distance of \SI{5}{\meter} is:
    \begin{multicols}{3}
    \begin{choices}
        \wrongchoice{\SI{15}{\joule}.}
        \wrongchoice{\SI{15}{\newton}.}
        \wrongchoice{\SI{150}{\joule}.}
      \correctchoice{\SI{0}{\joule}.}
        \wrongchoice{\SI{150}{\newton}.}
    \end{choices}
    \end{multicols}
\end{question}
}

\element{jpierce}{
\begin{question}{pt101tb2-Q94}
    The work done against gravity in moving a box with a mass of
        \SI{20}{\kilo\gram} through a horizontal distance of \SI{5}{\meter} is:
    \begin{multicols}{3}
    \begin{choices}
        \wrongchoice{\SI{1000}{\newton}.}
        \wrongchoice{\SI{100}{\joule}.}
      \correctchoice{\SI{0}{\joule}.}
        \wrongchoice{\SI{100}{\newton}.}
        \wrongchoice{\SI{1000}{\joule}.}
    \end{choices}
    \end{multicols}
\end{question}
}

\element{jpierce}{
\begin{question}{pt101tb2-Q95}
    Max pushed on a heavy crate ($\text{mass}=\SI{250}{\kilo\gram}$) for \SI{5}{\second}
        with a force of \SI{200}{\newton}, but the crate did not move at all. 
    How much work did Max do on the crate?
    \begin{multicols}{2}
    \begin{choices}
        \wrongchoice{\SI{250 000}{\joule}}
      \correctchoice{none}
        \wrongchoice{\SI{1000}{\joule}}
        \wrongchoice{\SI{1250}{\joule}}
        \wrongchoice{\SI{200}{\joule}}
    \end{choices}
    \end{multicols}
\end{question}
}

\element{jpierce}{
\begin{question}{pt101tb2-Q96}
    The work done against gravity in moving a box with a mass of
        \SI{20}{\kilo\gram} through a height of \SI{5}{\meter} is:
    \begin{multicols}{2}
    \begin{choices}
        \wrongchoice{\SI{100}{\newton}}
      \correctchoice{\SI{1000}{\joule}}
        \wrongchoice{\SI{4}{\joule}}
        \wrongchoice{\SI{1000}{\newton}}
        \wrongchoice{\SI{100}{\joule}}
    \end{choices}
    \end{multicols}
\end{question}
}

\element{jpierce}{
\begin{question}{pt101tb2-Q97}
    The work done against gravity in moving a box with a weight of
        \SI{20}{\newton} through a height of \SI{5}{\meter} is:
    \begin{multicols}{2}
    \begin{choices}
        \wrongchoice{\SI{100}{\newton}}
      \correctchoice{\SI{100}{\joule}}
        \wrongchoice{\SI{1000}{\joule}}
        \wrongchoice{\SI{4}{\joule}}
        \wrongchoice{\SI{1000}{\newton}}
    \end{choices}
    \end{multicols}
\end{question}
}

\element{jpierce}{
\begin{question}{pt101tb2-Q98}
    The work done against gravity in moving a box with a weight of
        \SI{5}{\newton} through a height of \SI{3}{\meter} is:
    \begin{multicols}{3}
    \begin{choices}
        \wrongchoice{\SI{5/3}{\joule}}
      \correctchoice{\SI{15}{\joule}}
        \wrongchoice{\SI{15}{\newton}}
        \wrongchoice{\SI{150}{\joule}}
        \wrongchoice{\SI{150}{\newton}}
    \end{choices}
    \end{multicols}
\end{question}
}

\element{jpierce}{
\begin{question}{pt101tb2-Q99}
    The work done against gravity in moving a box with a mass of
        \SI{3}{\kilo\gram} through a height of \SI{5}{\meter} is:
    \begin{multicols}{3}
    \begin{choices}
        \wrongchoice{\SI{15}{\joule}}
        \wrongchoice{\SI{0.6}{\joule}}
        \wrongchoice{\SI{150}{\newton}}
      \correctchoice{\SI{150}{\joule}}
        \wrongchoice{\SI{15}{\newton}}
    \end{choices}
    \end{multicols}
\end{question}
}

\element{jpierce}{
\begin{question}{pt101tb2-Q100}
    The work done against gravity in moving a box with a mass of
        \SI{5}{\kilo\gram} through a height of \SI{3}{\meter} is:
    \begin{multicols}{3}
    \begin{choices}
        \wrongchoice{\SI{15}{\newton}}
      \correctchoice{\SI{150}{\joule}}
        \wrongchoice{\SI{150}{\newton}}
        \wrongchoice{\SI{5/3}{\joule}}
        \wrongchoice{\SI{15}{\joule}}
    \end{choices}
    \end{multicols}
\end{question}
}

\element{jpierce}{
\begin{question}{pt101tb2-Q101}
    The work done against gravity in moving a box with a weight of
        \SI{3}{\newton} through a height of \SI{5}{\meter} is:
    \begin{multicols}{3}
    \begin{choices}
      \correctchoice{\SI{15}{\joule}}
        \wrongchoice{\SI{15}{\newton}}
        \wrongchoice{\SI{150}{\joule}}
        \wrongchoice{\SI{0.6}{\joule}}
        \wrongchoice{\SI{150}{\newton}}
    \end{choices}
    \end{multicols}
\end{question}
}


%% Topic: Angular Momentum
\element{jpierce}{
\begin{question}{pt101tb2-Q102}
    Angular momentum is the product of
    \begin{choices}
        \wrongchoice{mass and velocity.}
        \wrongchoice{acceleration and time.}
        \wrongchoice{linear momentum and angle.}
      \correctchoice{rotational inertia and rotational velocity.}
        \wrongchoice{force and impulse.}
    \end{choices}
\end{question}
}

\element{jpierce}{
\begin{question}{pt101tb2-Q103}
    A moving bicycle is more stable against tipping than a non-moving bicycle
    \begin{choices}
        \wrongchoice{because of the kinetic energy of the bicycle and rider.}
        \wrongchoice{because of the linear momentum of the bicycle and rider.}
        \wrongchoice{because of the gravitational potential energy of the rider.}
        \wrongchoice{because of the friction of the bicycle tires with the ground.}
      \correctchoice{because of the angular momentum of the spinning wheels.}
    \end{choices}
\end{question}
}

\element{jpierce}{
\begin{question}{pt101tb2-Q104}
    An unbalanced external torque acting on an object will cause a change in the object's
    \begin{choices}
        \wrongchoice{potential energy.}
        \wrongchoice{mass.}
      \correctchoice{angular momentum.}
        \wrongchoice{linear momentum.}
        \wrongchoice{weight.}
    \end{choices}
\end{question}
}

\element{jpierce}{
\begin{question}{pt101tb2-Q105}
    A platform diver performing a somersault maneuver is changing
        her \rule[-0.1pt]{4em}{0.1pt} but not her \rule[-0.1pt]{4em}{0.1pt}.
    \begin{choices}
        \wrongchoice{angular momentum; gravitational potential energy}
        \wrongchoice{kinetic energy; linear momentum}
        \wrongchoice{linear momentum; rotational speed}
        \wrongchoice{gravitational potential energy; kinetic energy}
      \correctchoice{rotational speed; angular momentum}
    \end{choices}
\end{question}
}

\element{jpierce}{
\begin{question}{pt101tb2-Q106}
    As a spinning ice skater pulls her arms in toward her body,
    \begin{choices}
        \wrongchoice{her angular momentum increases, due to conservation of rotational speed.}
        \wrongchoice{her rotational speed decreases, due to conservation of angular momentum.}
        \wrongchoice{her rotational speed remains constant, due to conservation of angular momentum.}
      \correctchoice{her rotational speed increases, due to conservation of angular momentum.}
        \wrongchoice{her angular momentum decreases, due to conservation of rotational speed.}
    \end{choices}
\end{question}
}

\element{jpierce}{
\begin{question}{pt101tb2-Q107}
    As a spinning ice skater moves her arms out away from her body,
    \begin{choices}
        \wrongchoice{her angular momentum increases, due to conservation of rotational speed.}
        \wrongchoice{her rotational speed increases, due to conservation of angular momentum.}
        \wrongchoice{her angular momentum decreases, due to conservation of rotational speed.}
      \correctchoice{her rotational speed decreases, due to conservation of angular momentum.}
        \wrongchoice{her rotational speed remains constant, due to conservation of angular momentum.}
    \end{choices}
\end{question}
}

\element{jpierce}{
\begin{question}{pt101tb2-Q108}
    When angular momentum is conserved, rotational speed
    \begin{choices}
        \wrongchoice{must be constant.}
      \correctchoice{increases if the mass moves closer to the axis of rotation.}
        \wrongchoice{decreases if the mass moves closer to the axis of rotation.}
        \wrongchoice{may increase, but can never decrease.}
        \wrongchoice{may decrease, but can never increase.}
    \end{choices}
\end{question}
}

\element{jpierce}{
\begin{question}{pt101tb2-Q109}
    When angular momentum is conserved, rotational speed
    \begin{choices}
        \wrongchoice{may increase, but can never decrease.}
      \correctchoice{decreases if the mass moves farther from the axis of rotation.}
        \wrongchoice{increases if the mass moves farther from the axis of rotation.}
        \wrongchoice{must be constant.}
        \wrongchoice{may decrease, but can never increase.}
    \end{choices}
\end{question}
}


%% Topic: Center of Mass
\element{jpierce}{
\begin{question}{pt101tb2-Q110}
    The center of mass of an object
    \begin{choices}
        \wrongchoice{must always coincide with some of the object's mass.}
        \wrongchoice{must lie inside the object's surface.}
      \correctchoice{is the average position of the object's mass.}
        \wrongchoice{is always at its midpoint.}
        \wrongchoice{always moves in a straight line when the object is thrown into the air.}
    \end{choices}
\end{question}
}

\element{jpierce}{
\begin{question}{pt101tb2-Q111}
    When you stand in equilibrium on only one foot,
    \begin{choices}
        \wrongchoice{your center of mass will be directly above the other foot.}
        \wrongchoice{your rotational inertia will be zero.}
      \correctchoice{your center of mass will be directly above that foot.}
        \wrongchoice{you will always fall over.}
        \wrongchoice{your center of mass will be directly above a point equidistant between your two feet.}
    \end{choices}
\end{question}
}

\element{jpierce}{
\begin{question}{pt101tb2-Q112}
    The center of mass of a meter stick is
        approximately \rule[-0.1pt]{4em}{0.1pt} from the end of the stick.
    \begin{multicols}{3}
    \begin{choices}
        \wrongchoice{\SI{0}{\centi\meter}}
        \wrongchoice{\SI{500}{\centi\meter}}
        \wrongchoice{\SI{100}{\centi\meter}}
        \wrongchoice{\SI{10}{\centi\meter}}
      \correctchoice{\SI{50}{\centi\meter}}
    \end{choices}
    \end{multicols}
\end{question}
}

\element{jpierce}{
\begin{question}{pt101tb2-Q113}
    The center of mass of a meter stick is 
        approximately \rule[-0.1pt]{4em}{0.1pt} from the end of the stick.
    \begin{multicols}{3}
    \begin{choices}
        \wrongchoice{\SI{50}{\milli\meter}}
        \wrongchoice{\SI{0}{\milli\meter}}
        \wrongchoice{\SI{100}{\milli\meter}}
      \correctchoice{\SI{500}{\milli\meter}}
        \wrongchoice{\SI{10}{\milli\meter}}
    \end{choices}
    \end{multicols}
\end{question}
}

%% Topic: Centrifugal
\element{jpierce}{
\begin{question}{pt101tb2-Q114}
    A centrifugal force is an apparent force that is:
    \begin{choices}
        \wrongchoice{against the direction of motion of an object.}
        \wrongchoice{directed toward the center of curvature of the path of a moving object.}
        \wrongchoice{directed toward the center of the Earth.}
        \wrongchoice{in the direction of motion of an object.}
      \correctchoice{directed away from the center of curvature of the path of a moving object.}
    \end{choices}
\end{question}
}

\element{jpierce}{
\begin{question}{pt101tb2-Q115}
    When a car rounds a curve to the left at high speed,
        the passengers experience the illusion of being acted upon by:
    \begin{choices}
        \wrongchoice{an upward-directed centrifugal force.}
        \wrongchoice{a centrifugal force directed to the left.}
        \wrongchoice{a centripetal force directed to the right.}
      \correctchoice{a centrifugal force directed to the right.}
        \wrongchoice{an upward-directed centripetal force.}
    \end{choices}
\end{question}
}

\element{jpierce}{
\begin{question}{pt101tb2-Q116}
    A centrifugal force is:
    \begin{choices}
        \wrongchoice{a real force caused by gravity.}
      \correctchoice{an apparent force caused by rotational motion.}
        \wrongchoice{a real force caused by rotational motion.}
        \wrongchoice{a real force that is the reaction force to a centripetal force.}
        \wrongchoice{an apparent force caused by gravity.}
    \end{choices}
\end{question}
}

\element{jpierce}{
\begin{question}{pt101tb2-Q117}
    When a car rounds a curve to the right at high speed,
        the passengers experience the illusion of being acted upon by:
    \begin{choices}
        \wrongchoice{a centrifugal force directed to the right.}
      \correctchoice{a centrifugal force directed to the left.}
        \wrongchoice{an upward-directed centrifugal force.}
        \wrongchoice{an upward-directed centripetal force.}
        \wrongchoice{a centripetal force directed to the left.}
    \end{choices}
\end{question}
}

\element{jpierce}{
\begin{question}{pt101tb2-Q118}
    A centripetal force is one that is:
    \begin{choices}
        \wrongchoice{against the direction of motion of an object.}
        \wrongchoice{directed toward the center of the Earth.}
      \correctchoice{directed toward the center of curvature of the path of a moving object.}
        \wrongchoice{directed away from the center of curvature of the path of a moving object.}
        \wrongchoice{in the direction of motion of an object.}
    \end{choices}
\end{question}
}

\element{jpierce}{
\begin{question}{pt101tb2-Q119}
    When you whirl a rock tied to a string in a horizontal circle around your head,
    \begin{choices}
        \wrongchoice{the rock exerts a centripetal force on the string.}
      \correctchoice{the string exerts a centripetal force on the rock.}
        \wrongchoice{the Earth exerts a centripetal force on the rock.}
        \wrongchoice{the string exerts a centripetal force on your hand.}
        \wrongchoice{there are no centripetal forces involved.}
    \end{choices}
\end{question}
}

\element{jpierce}{
\begin{question}{pt101tb2-Q120}
    When a car rounds a curve at high speed,
    \begin{choices}
      \correctchoice{the road exerts a centripetal force on the tires.}
        \wrongchoice{the car body exerts a centripetal force on the tires.}
        \wrongchoice{there are no centripetal forces involved.}
        \wrongchoice{the car exerts a centripetal force on the road.}
        \wrongchoice{the tires exert a centripetal force on the road.}
    \end{choices}
\end{question}
}

\element{jpierce}{
\begin{question}{pt101tb2-Q121}
    When a car rounds a curve at high speed,
    \begin{choices}
        \wrongchoice{there are no centripetal forces involved.}
        \wrongchoice{the car exerts a centripetal force on the road.}
        \wrongchoice{the tires exert a centripetal force on the road.}
      \correctchoice{the car exerts a centripetal force on the driver.}
        \wrongchoice{the car body exerts a centripetal force on the tires.}
    \end{choices}
\end{question}
}

\element{jpierce}{
\begin{question}{pt101tb2-Q122}
    As you whirl a rock tied to a string in a horizontal circle around your head,
        the string suddenly breaks; what happens?
    \begin{choices}
        \wrongchoice{The rock will move inward and strike you on the head.}
      \correctchoice{The rock will move along a straight line tangent to the circle while curving toward the ground.}
        \wrongchoice{The rock will move outward directly away from your head while curving toward the ground.}
        \wrongchoice{The rock will continue to move in a circle about your head.}
        \wrongchoice{The rock will fall straight to the ground.}
    \end{choices}
\end{question}
}


%% Topic: Cicular
\element{jpierce}{
\begin{question}{pt101tb2-Q123}
    On a spinning disk, points closer to the outer edge will
        have \rule[-0.1pt]{4em}{0.1pt} points near the center.
    \begin{choices}
        \wrongchoice{the same tangential speed as and greater rotational speed than}
      \correctchoice{the same rotational speed as and greater tangential speed than}
        \wrongchoice{lower rotational speed and higher tangential speed than}
        \wrongchoice{the same tangential speed as and lower rotational speed than}
        \wrongchoice{the same rotational speed as and lower tangential speed than}
    \end{choices}
\end{question}
}

\element{jpierce}{
\begin{question}{pt101tb2-Q124}
    On a spinning disk, points closer to the center will
        have \rule[-0.1pt]{4em}{0.1pt} points near the outer edge.
    \begin{choices}
        \wrongchoice{the same tangential speed as and greater rotational speed than}
        \wrongchoice{the same tangential speed as and lower rotational speed than}
        \wrongchoice{the same rotational speed as and greater tangential speed than}
        \wrongchoice{lower rotational speed and higher tangential speed than}
      \correctchoice{the same rotational speed as and lower tangential speed than}
    \end{choices}
\end{question}
}

\element{jpierce}{
\begin{question}{pt101tb2-Q125}
    On the rotating Earth,
        points that have the highest tangential speed will be located at
    \begin{choices}
        \wrongchoice{Mankato.}
      \correctchoice{the Equator.}
        \wrongchoice{the South Pole.}
        \wrongchoice{the North Pole.}
        \wrongchoice{(None of these---all points on the Earth have the same tangential speed.)}
    \end{choices}
\end{question}
}

\element{jpierce}{
\begin{question}{pt101tb2-Q126}
    On the rotating Earth,
        points that have the highest rotational speed will be located at
    \begin{choices}
        \wrongchoice{Mankato.}
        \wrongchoice{the Equator.}
        \wrongchoice{the North Pole.}
        \wrongchoice{the South Pole.}
      \correctchoice{(None of these---all points on the Earth have the same rotational speed.)}
    \end{choices}
\end{question}
}

\element{jpierce}{
\begin{question}{pt101tb2-Q127}
    An object moving in a circular path
    \begin{choices}
        \wrongchoice{must be slowing down.}
        \wrongchoice{must be moving at a constant velocity.}
        \wrongchoice{must be moving at a constant speed.}
      \correctchoice{must be accelerating.}
        \wrongchoice{must be speeding up.}
    \end{choices}
\end{question}
}

\element{jpierce}{
\begin{question}{pt101tb2-Q128}
    A merry-go-round rotates 9 times each minute such that a point on its
        rim moves at a rate of \SI{3}{\meter\per\second}. 
    At a point $\frac{2}{3}$ of the way out from the center to the rim,
        the rotational speed would be \rule[-0.1pt]{4em}{0.1pt}.
    \begin{multicols}{2}
    \begin{choices}
        \wrongchoice{\SI{6}{\rotation\per\minute}}
      \correctchoice{\SI{9}{\rotation\per\minute}}
        \wrongchoice{\SI{2}{\meter\per\second}}
        \wrongchoice{\SI{3}{\rotation\per\minute}}
        \wrongchoice{\SI{3}{\meter\per\second}}
    \end{choices}
    \end{multicols}
\end{question}
}

\element{jpierce}{
\begin{question}{pt101tb2-Q129}
    A merry-go-round rotates 8 times each minute such that a point on its
        rim moves at a rate of \SI{4}{\meter\per\second}.
    At a point halfway out from the center to the rim,
        the rotational speed would be \rule[-0.1pt]{4em}{0.1pt}.
    \begin{multicols}{2}
    \begin{choices}
        \wrongchoice{\SI{2}{\rotation\per\minute}}
        \wrongchoice{\SI{4}{\rotation\per\minute}}
        \wrongchoice{\SI{2}{\meter\per\second}}
      \correctchoice{\SI{8}{\rotation\per\minute}}
        \wrongchoice{\SI{4}{\meter\per\second}}
    \end{choices}
    \end{multicols}
\end{question}
}

\element{jpierce}{
\begin{question}{pt101tb2-Q130}
    A merry-go-round rotates 9 times each minute such that a point on its
        rim moves at a rate of \SI{3}{\meter\per\second}.
    At a point $\frac{1}{3}$ of the way out from the center to the rim,
        the rotational speed would be \rule[-0.1pt]{4em}{0.1pt}.
    \begin{multicols}{2}
    \begin{choices}
      \correctchoice{\SI{9}{\rotation\per\minute}}
        \wrongchoice{\SI{3}{\rotation\per\minute}}
        \wrongchoice{\SI{6}{\rotation\per\minute}}
        \wrongchoice{\SI{1}{\meter\per\second}}
        \wrongchoice{\SI{3}{\meter\per\second}}
    \end{choices}
    \end{multicols}
\end{question}
}

\element{jpierce}{
\begin{question}{pt101tb2-Q131}
    A merry-go-round rotates 8 times each minute such that a point on its
        rim moves at a rate of \SI{4}{\meter\per\second}.
    At a point $\frac{3}{4}$ of the way out from the center to the rim,
        the rotational speed would be \rule[-0.1pt]{4em}{0.1pt}.
    \begin{multicols}{2}
    \begin{choices}
        \wrongchoice{\SI{6}{\rotation\per\minute}}
        \wrongchoice{\SI{6}{\meter\per\second}}
        \wrongchoice{\SI{3/4}{\rotation\per\minute}}
      \correctchoice{\SI{8}{\rotation\per\minute}}
        \wrongchoice{\SI{3/4}{\meter\per\second}}
    \end{choices}
    \end{multicols}
\end{question}
}

\element{jpierce}{
\begin{question}{pt101tb2-Q132}
    A merry-go-round rotates 9 times each minute such that a point on its
        rim moves at a rate of \SI{3}{\meter\per\second}. 
    At a point $\frac{2}{3}$ of the way out from the center to the rim,
        the tangential speed would be \rule[-0.1pt]{4em}{0.1pt}.
    \begin{multicols}{2}
    \begin{choices}
        \wrongchoice{\SI{3}{\meter\per\second}}
      \correctchoice{\SI{2}{\meter\per\second}}
        \wrongchoice{\SI{3}{\rotation\per\minute}}
        \wrongchoice{\SI{9}{\rotation\per\minute}}
        \wrongchoice{\SI{6}{\rotation\per\minute}}
    \end{choices}
    \end{multicols}
\end{question}
}

\element{jpierce}{
\begin{question}{pt101tb2-Q133}
    A merry-go-round rotates 8 times each minute such that a point on its
        rim moves at a rate of \SI{4}{\meter\per\second}.
    At a point halfway out from the center to the rim,
        the tangential speed would be \rule[-0.1pt]{4em}{0.1pt}.
    \begin{multicols}{2}
    \begin{choices}
        \wrongchoice{\SI{8}{\rotation\per\minute}}
        \wrongchoice{\SI{2}{\rotation\per\minute}}
        \wrongchoice{\SI{4}{\meter\per\second}}
      \correctchoice{\SI{2}{\meter\per\second}}
        \wrongchoice{\SI{4}{\rotation\per\minute}}
    \end{choices}
    \end{multicols}
\end{question}
}

\element{jpierce}{
\begin{question}{pt101tb2-Q134}
    A merry-go-round rotates 9 times each minute such that a point on its
        rim moves at a rate of \SI{3}{\meter\per\second}.
    At a point $\frac{1}{3}$ of the way out from the center to the rim,
        the tangential speed would be \rule[-0.1pt]{4em}{0.1pt}.
    \begin{multicols}{2}
    \begin{choices}
        \wrongchoice{\SI{3}{\rotation\per\minute}}
      \correctchoice{\SI{1}{\meter\per\second}}
        \wrongchoice{\SI{3}{\meter\per\second}}
        \wrongchoice{\SI{2}{\meter\per\second}}
        \wrongchoice{\SI{9}{\rotation\per\minute}}
    \end{choices}
    \end{multicols}
\end{question}
}

\element{jpierce}{
\begin{question}{pt101tb2-Q135}
    A merry-go-round rotates 8 times each minute such that a point on its
        rim moves at a rate of \SI{4}{\meter\per\second}.
    At a point $\frac{3}{4}$ of the way out from the center to the rim,
        the tangential speed would be \rule[-0.1pt]{4em}{0.1pt}.
    \begin{multicols}{2}
    \begin{choices}
        \wrongchoice{\SI{8}{\rotation\per\minute}}
        \wrongchoice{\SI{6}{\meter\per\second}}
        \wrongchoice{\SI{0.75}{\meter\per\second}}
      \correctchoice{\SI{3}{\meter\per\second}}
        \wrongchoice{\SI{6}{\rotation\per\minute}}
    \end{choices}
    \end{multicols}
\end{question}
}

%% Topic: Rotational Inertia
\element{jpierce}{
\begin{question}{pt101tb2-Q136}
    The rotational inertia of an object depends on
    \begin{choices}
        \wrongchoice{the rotational speed of the object.}
        \wrongchoice{the volume of the object.}
        \wrongchoice{the weight of the object.}
      \correctchoice{the mass of the object and its distribution with respect to the axis of rotation.}
        \wrongchoice{the color of the object.}
    \end{choices}
\end{question}
}

\element{jpierce}{
\begin{question}{pt101tb2-Q137}
    A tightrope walker carries a long pole because
    \begin{choices}
        \wrongchoice{the pole is filled with helium gas and tends to float in the air.}
        \wrongchoice{he can use it to break his fall if he loses his balance.}
        \wrongchoice{the pole decreases his rotational inertia.}
        \wrongchoice{the pole is made of a material that is not affected by gravity.}
      \correctchoice{the pole increases his rotational inertia.}
    \end{choices}
\end{question}
}

\element{jpierce}{
\begin{question}{pt101tb2-Q138}
    An empty soup can and a full one are rolled side-by-side down an incline. 
    If they start together, which one will reach the bottom first?
    \begin{choices}
        \wrongchoice{The empty can arrives first.}
        \wrongchoice{It depends on the diameters of the cans.}
      \correctchoice{The full can arrives first.}
        \wrongchoice{They will arrive together.}
        \wrongchoice{It depends on the kind of soup.}
    \end{choices}
\end{question}
}

\element{jpierce}{
\begin{question}{pt101tb2-Q139}
    An empty soup can and a full one are rolled side-by-side down an incline. 
    If they start together, which one will roll more slowly?
    \begin{choices}
        \wrongchoice{It depends on the diameters of the cans.}
      \correctchoice{The empty can will be slower.}
        \wrongchoice{They will roll at the same rate.}
        \wrongchoice{It depends on the kind of soup.}
        \wrongchoice{The full can will be slower.}
    \end{choices}
\end{question}
}

\element{jpierce}{
\begin{question}{pt101tb2-Q140}
    A mass $m$ is tied to a string and swung in a horizontal circle of radius $r$,
        the rotational inertia of this system is:
        %\rule[-0.1pt]{4em}{0.1pt}.
    \begin{multicols}{3}
    \begin{choices}
      \correctchoice{$\dfrac{m}{r^2}$}
        \wrongchoice{$m r^2$}
        \wrongchoice{$\dfrac{m}{r}$}
        \wrongchoice{$r m^2$}
        \wrongchoice{$m r$}
    \end{choices}
    \end{multicols}
\end{question}
}

\element{jpierce}{
\begin{question}{pt101tb2-Q141}
    The rotational inertia of a sphere of mass $m$ and
        radius $r$ is proportional to:
    %\rule[-0.1pt]{4em}{0.1pt}.
    \begin{multicols}{3}
    \begin{choices}
        \wrongchoice{$\dfrac{m}{r^2}$}
      \correctchoice{$m r^2$}
        \wrongchoice{$\dfrac{m}{r}$}
        \wrongchoice{$r m^2$}
        \wrongchoice{$m r$}
    \end{choices}
    \end{multicols}
\end{question}
}

\element{jpierce}{
\begin{question}{pt101tb2-Q142}
    A mass of 1 kilogram is tied to a string and swung in a
        horizontal circle of radius \SI{1}{\meter};
        if the radius of the circle is then increased to \SI{2}{\meter},
        the rotational inertia of this new system will be \rule[-0.1pt]{4em}{0.1pt} as before.
    \begin{multicols}{2}
    \begin{choices}
        \wrongchoice{the same}
        \wrongchoice{twice as much}
      \correctchoice{four times as much}
        \wrongchoice{one fourth as much}
        \wrongchoice{one half as much}
    \end{choices}
    \end{multicols}
\end{question}
}

\element{jpierce}{
\begin{question}{pt101tb2-Q143}
    A mass of \SI{1}{\kilo\gram} is tied to a string and swung in a
        horizontal circle of radius \SI{1}{\meter};
        if the radius of the circle is then decreased to \SI{0.5}{\meter},
        the rotational inertia of this new system will be \rule[-0.1pt]{4em}{0.1pt} as before.
    \begin{multicols}{2}
    \begin{choices}
      \correctchoice{one fourth as much}
        \wrongchoice{the same}
        \wrongchoice{four times as much}
        \wrongchoice{twice as much}
        \wrongchoice{one half as much}
    \end{choices}
    \end{multicols}
\end{question}
}

\element{jpierce}{
\begin{question}{pt101tb2-Q144}
    A mass of \SI{1}{\kilo\gram} is tied to a string and swung in a
        horizontal circle of radius \SI{1}{\meter};
        if the mass is then decreased to \SI{0.5}{\kilo\gram},
        the rotational inertia of this new system will be \rule[-0.1pt]{4em}{0.1pt} as before.
    \begin{multicols}{2}
    \begin{choices}
        \wrongchoice{one fourth as much}
        \wrongchoice{four times as much}
        \wrongchoice{twice as much}
      \correctchoice{one half as much}
        \wrongchoice{the same}
    \end{choices}
    \end{multicols}
\end{question}
}

\element{jpierce}{
\begin{question}{pt101tb2-Q145}
    A mass of 1 kilogram is tied to a string and swung in a
        horizontal circle of radius \SI{1}{\meter};
        if the mass is then increased to \SI{2}{\kilo\gram},
        the rotational inertia of this new system will be \rule[-0.1pt]{4em}{0.1pt} as before.
    \begin{choices}
      \correctchoice{twice as much}
        \wrongchoice{one half as much}
        \wrongchoice{the same}
        \wrongchoice{one fourth as much}
        \wrongchoice{four times as much}
    \end{choices}
\end{question}
}

\element{jpierce}{
\begin{question}{pt101tb2-Q146}
    Torque is the product of
    \begin{choices}
        \wrongchoice{mass and radius.}
      \correctchoice{lever arm and force.}
        \wrongchoice{rotational inertia and velocity.}
        \wrongchoice{force and velocity.}
        \wrongchoice{lever arm and rotational inertia.}
    \end{choices}
\end{question}
}

\element{jpierce}{
\begin{question}{pt101tb2-Q147}
    A doorknob is normally placed near the edge of the door opposite the hinges. 
    This is because
    \begin{choices}
        \wrongchoice{the door would look funny with the knob in any other position.}
        \wrongchoice{that is where the door's rotational inertia will be lowest.}
        \wrongchoice{a force at this position will produce a minimum torque.}
      \correctchoice{a force at this position will produce a maximum torque.}
        \wrongchoice{that is where the door's center of mass is located.}
    \end{choices}
\end{question}
}

\element{jpierce}{
\begin{question}{pt101tb2-Q148}
    To obtain the maximum torque for a given force when using a wrench,
        the force should be applied at a \rule[-0.1pt]{4em}{0.1pt} angle
        to the handle of the wrench.
    \begin{multicols}{3}
    \begin{choices}
        \wrongchoice{\ang{30}}
      \correctchoice{\ang{90}}
        \wrongchoice{\ang{180}}
        \wrongchoice{\ang{60}}
        \wrongchoice{\ang{45}}
    \end{choices}
    \end{multicols}
\end{question}
}

%% Topic: Torque
\element{jpierce}{
\begin{question}{pt101tb2-Q149}
    A \SI{60}{\kilo\gram} grandfather and his \SI{30}{\kilo\gram} granddaughter
        are balanced on a seesaw.
    If the grandfather is sitting \SI{1}{\meter} from the pivot point,
        the granddaughter must be sitting \rule[-0.1pt]{4em}{0.1pt} from it.
    \begin{multicols}{3}
    \begin{choices}
      \correctchoice{\SI{2}{\meter}}
        \wrongchoice{\SI{1}{\meter}}
        \wrongchoice{\SI{3}{\meter}}
        \wrongchoice{\SI{6}{\meter}}
        \wrongchoice{\SI{0.5}{\meter}}
    \end{choices}
    \end{multicols}
\end{question}
}

\element{jpierce}{
\begin{question}{pt101tb2-Q150}
    A \SI{75}{\kilo\gram} grandfather and his \SI{30}{\kilo\gram} granddaughter
        are balanced on a seesaw. 
    If the grandfather is sitting \SI{1}{\meter} from the pivot point,
        the granddaughter must be sitting \rule[-0.1pt]{4em}{0.1pt} from it.
    \begin{multicols}{3}
    \begin{choices}
      \correctchoice{\SI{2.5}{\meter}}
        \wrongchoice{\SI{0.5}{\meter}}
        \wrongchoice{\SI{7.5}{\meter}}
        \wrongchoice{\SI{5}{\meter}}
        \wrongchoice{\SI{1.5}{\meter}}
    \end{choices}
    \end{multicols}
\end{question}
}

\element{jpierce}{
\begin{question}{pt101tb2-Q151}
    A \SI{60}{\kilo\gram} grandfather and his \SI{30}{\kilo\gram} granddaughter
        are balanced on a seesaw. 
    If the granddaughter is sitting \SI{2}{\meter} from the pivot point,
        the grandfather must be sitting \rule[-0.1pt]{4em}{0.1pt} from it.
    \begin{multicols}{3}
    \begin{choices}
        \wrongchoice{\SI{2}{\meter}}
      \correctchoice{\SI{1}{\meter}}
        \wrongchoice{\SI{0.5}{\meter}}
        \wrongchoice{\SI{3}{\meter}}
        \wrongchoice{\SI{4}{\meter}}
    \end{choices}
    \end{multicols}
\end{question}
}

\element{jpierce}{
\begin{question}{pt101tb2-Q152}
    A \SI{75}{\kilo\gram} grandfather and his \SI{30}{\kilo\gram} granddaughter
        are balanced on a seesaw. 
    If the granddaughter is sitting \SI{2}{\meter} from the pivot point,
        the grandfather must be sitting \rule[-0.1pt]{4em}{0.1pt} from it.
    \begin{multicols}{3}
    \begin{choices}
        \wrongchoice{\SI{2}{\meter}}
        \wrongchoice{\SI{0.5}{\meter}}
        \wrongchoice{\SI{5}{\meter}}
      \correctchoice{\SI{0.8}{\meter}}
        \wrongchoice{\SI{2.5}{\meter}}
    \end{choices}
    \end{multicols}
\end{question}
}

\element{jpierce}{
\begin{question}{pt101tb2-Q153}
    A \SI{60}{\kilo\gram} grandfather and his \SI{20}{\kilo\gram} granddaughter
        are balanced on a seesaw. 
    If the grandfather is sitting \SI{1}{\meter} from the pivot point,
        the granddaughter must be sitting \rule[-0.1pt]{4em}{0.1pt} from it.
    \begin{multicols}{3}
    \begin{choices}
      \correctchoice{\SI{3}{\meter}}
        \wrongchoice{\SI{1}{\meter}}
        \wrongchoice{\SI{2}{\meter}}
        \wrongchoice{\SI{20}{\meter}}
        \wrongchoice{\SI{6}{\meter}}
    \end{choices}
    \end{multicols}
\end{question}
}

\element{jpierce}{
\begin{question}{pt101tb2-Q154}
    A \SI{60}{\kilo\gram} grandfather and his \SI{20}{\kilo\gram} granddaughter
        are balanced on a seesaw. 
    If the granddaughter is sitting \SI{3}{\meter} from the pivot point,
        the grandfather must be sitting \rule[-0.1pt]{4em}{0.1pt} from it.
    \begin{multicols}{3}
    \begin{choices}
        \wrongchoice{\SI{18}{\meter}}
        \wrongchoice{\SI{9}{\meter}}
      \correctchoice{\SI{1}{\meter}}
        \wrongchoice{\SI{20}{\meter}}
        \wrongchoice{\SI{3}{\meter}}
    \end{choices}
    \end{multicols}
\end{question}
}

\element{jpierce}{
\begin{question}{pt101tb2-Q155}
    A \SI{60}{\kilo\gram} grandfather and his \SI{15}{\kilo\gram} granddaughter
        are balanced on a seesaw. 
    If the grandfather is sitting \SI{0.5}{\meter} from the pivot point,
        the granddaughter must be sitting \rule[-0.1pt]{4em}{0.1pt} from it.
    \begin{multicols}{3}
    \begin{choices}
        \wrongchoice{\SI{7.5}{\meter}}
        \wrongchoice{\SI{4}{\meter}}
        \wrongchoice{\SI{15}{\meter}}
        \wrongchoice{\SI{1}{\meter}}
      \correctchoice{\SI{2}{\meter}}
    \end{choices}
    \end{multicols}
\end{question}
}

\element{jpierce}{
\begin{question}{pt101tb2-Q156}
    A \SI{60}{\kilo\gram} grandfather and his \SI{15}{\kilo\gram} granddaughter
        are balanced on a seesaw. 
    If the granddaughter is sitting \SI{2}{\meter} from the pivot point,
        the grandfather must be sitting \rule[-0.1pt]{4em}{0.1pt} from it.
    \begin{multicols}{3}
    \begin{choices}
        \wrongchoice{\SI{4}{\meter}}
        \wrongchoice{\SI{1}{\meter}}
      \correctchoice{\SI{0.5}{\meter}}
        \wrongchoice{\SI{15}{\meter}}
        \wrongchoice{\SI{30}{\meter}}
    \end{choices}
    \end{multicols}
\end{question}
}

\element{jpierce}{
\begin{question}{pt101tb2-Q157}
    A meter stick is balanced on a fulcrum positioned at the \SI{50}{\centi\meter} mark. 
    If a \SI{100}{\gram} weight is hung at the \SI{20}{\centi\meter} mark,
        where should another \SI{100}{\gram} weight be hung to balance the stick?
    \begin{choices}
      \correctchoice{at the \SI{80}{\centi\meter} mark}
        \wrongchoice{at the \SI{20}{\centi\meter} mark}
        \wrongchoice{at the \SI{70}{\centi\meter} mark}
        \wrongchoice{at the \SI{30}{\centi\meter} mark}
        \wrongchoice{at the \SI{50}{\centi\meter} mark}
    \end{choices}
\end{question}
}

\element{jpierce}{
\begin{question}{pt101tb2-Q158}
    A meter stick is balanced on a fulcrum positioned at the \SI{50}{\centi\meter} mark.
    If a \SI{100}{\gram} weight is hung at the \SI{30}{\centi\meter} mark,
        where should another \SI{100}{\gram} weight be hung to balance the stick?
    \begin{choices}
      \correctchoice{at the \SI{70}{\centi\meter} mark}
        \wrongchoice{at the \SI{50}{\centi\meter} mark}
        \wrongchoice{at the \SI{30}{\centi\meter} mark}
        \wrongchoice{at the \SI{80}{\centi\meter} mark}
        \wrongchoice{at the \SI{20}{\centi\meter} mark}
    \end{choices}
\end{question}
}

\element{jpierce}{
\begin{question}{pt101tb2-Q159}
    A meter stick is balanced on a fulcrum positioned at the \SI{50}{\centi\meter} mark. 
    If a \SI{100}{\gram} weight is hung at the \SI{70}{\centi\meter} mark,
        where should another 100 gram weight be hung to balance the stick?
    \begin{choices}
        \wrongchoice{at the \SI{70}{\centi\meter} mark}
      \correctchoice{at the \SI{30}{\centi\meter} mark}
        \wrongchoice{at the \SI{80}{\centi\meter} mark}
        \wrongchoice{at the \SI{20}{\centi\meter} mark}
        \wrongchoice{at the \SI{50}{\centi\meter} mark}
    \end{choices}
\end{question}
}

\element{jpierce}{
\begin{question}{pt101tb2-Q160}
    A meter stick is balanced on a fulcrum positioned at the \SI{50}{\centi\meter} mark.
    If a \SI{100}{\gram} weight is hung at the \SI{80}{\centi\meter} mark,
        where should another \SI{100}{\gram} weight be hung to balance the stick?
    \begin{choices}
        \wrongchoice{at the \SI{50}{\centi\meter} mark}
        \wrongchoice{at the \SI{30}{\centi\meter} mark}
      \correctchoice{at the \SI{20}{\centi\meter} mark}
        \wrongchoice{at the \SI{80}{\centi\meter} mark}
        \wrongchoice{at the \SI{70}{\centi\meter} mark}
    \end{choices}
\end{question}
}

\element{jpierce}{
\begin{question}{pt101tb2-Q161}
    A meter stick is balanced on a fulcrum positioned at the \SI{50}{\centi\meter} mark.
    If a \SI{100}{\gram} weight is hung at the \SI{70}{\centi\meter} mark,
        where should a \SI{200}{\gram} weight be hung to balance the stick?
    \begin{choices}
        \wrongchoice{at the \SI{20}{\centi\meter} mark}
      \correctchoice{at the \SI{40}{\centi\meter} mark}
        \wrongchoice{at the \SI{60}{\centi\meter} mark}
        \wrongchoice{at the \SI{30}{\centi\meter} mark}
        \wrongchoice{at the \SI{80}{\centi\meter} mark}
    \end{choices}
\end{question}
}

\element{jpierce}{
\begin{question}{pt101tb2-Q162}
    A meter stick is balanced on a fulcrum positioned at the \SI{50}{\centi\meter} mark. 
    If a \SI{100}{\gram} weight is hung at the \SI{20}{\centi\meter} mark,
        where should a \SI{300}{\gram} weight be hung to balance the stick?
    \begin{choices}
        \wrongchoice{at the \SI{70}{\centi\meter} mark}
        \wrongchoice{at the \SI{10}{\centi\meter} mark}
        \wrongchoice{at the \SI{80}{\centi\meter} mark}
      \correctchoice{at the \SI{60}{\centi\meter} mark}
        \wrongchoice{at the \SI{30}{\centi\meter} mark}
    \end{choices}
\end{question}
}


\endinput


