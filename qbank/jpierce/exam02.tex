


%% Physics 101 Sample Test Questions by Dr. James Pierce
%%------------------------------------------------------------


%% Exam #2 ---- (momentum, work, energy, rotation, etc.)
%%------------------------------------------------------------
\element{cpo-mc}{
\begin{question}{exam02-Q24}
    A green ball moving to the right at \SI{3}{\meter\per\second}
        strikes a yellow ball moving to the left at \SI{2}{\meter\per\second}.
    If the balls are equally massive and the collision is elastic,
    \begin{choices}
        \wrongchoice{the green ball will move to the left at \SI{3}{\meter\per\second} while the yellow ball moves right at \SI{2}{\meter\per\second}.}
      \correctchoice{the green ball will move to the left at \SI{2}{\meter\per\second} while the yellow ball moves right at \SI{3}{\meter\per\second}.}
        \wrongchoice{The green ball will stop while the yellow ball moves right at \SI{2}{\meter\per\second}.}
        \wrongchoice{The yellow ball will stop while the green ball moves left at \SI{3}{\meter\per\second}.}
        \wrongchoice{Both balls will stick together and move to the right at \SI{1}{\meter\per\second}.}
    \end{choices}
\end{question}
}

\element{cpo-mc}{
\begin{question}{exam02-Q25}
    An impulse of \SI{100}{\newton\second} is applied to an object.
    If this same impulse is delivered over a longer time interval,
    \begin{choices}
      \correctchoice{the force involved will be decreased.}
        \wrongchoice{the force involved will be increased.}
        \wrongchoice{the momentum transferred will be increased.}
        \wrongchoice{the momentum transferred will be decreased.}
        \wrongchoice{the acceleration involved will be increased.}
    \end{choices}
\end{question}
}

\element{cpo-mc}{
\begin{question}{exam02-Q26}
    Case 1: A net force of \SI{10}{\newton} acts on a mass of \SI{1}{\kilo\gram} for a time of \SI{0.2}{\second}.
    Case 2: A net force of \SI{20}{\newton} acts on a mass of \SI{1}{\kilo\gram} for a time of \SI{0.2}{\second}.
    Both cases result in acceleration of the mass.
    In comparison, Case 1 and Case 2 will
    \begin{choices}
        \wrongchoice{involve the same impulse and produce the same acceleration.}
        \wrongchoice{involve the same impulse and produce different accelerations.}
      \correctchoice{involve different impulses and produce different accelerations.}
        \wrongchoice{involve different impulses and produce the same acceleration.}
        \wrongchoice{produce the same change of momentum.}
    \end{choices}
\end{question}
}

\element{cpo-mc}{
\begin{question}{exam02-Q27}
    Momentum is the product of
    \begin{choices}
      \correctchoice{mass and velocity.}
        \wrongchoice{mass and acceleration.}
        \wrongchoice{velocity and acceleration.}
        \wrongchoice{force and inertia.}
        \wrongchoice{force and velocity.}
    \end{choices}
\end{question}
}

\element{cpo-mc}{
\begin{question}{exam02-Q28}
    If a moving object cuts its speed in half,
        how much momentum will it have?
    \begin{choices}
        \wrongchoice{the same amount as before}
        \wrongchoice{twice as much as before}
      \correctchoice{one half as much as before}
        \wrongchoice{four times as much as before}
        \wrongchoice{one fourth as much as before}
    \end{choices}
\end{question}
}

\element{cpo-mc}{
\begin{question}{exam02-Q29}
    A \SI{1}{\kilo\gram} ball moving horizontally to the right at \SI{3}{\meter\per\second} strikes a wall and rebounds,
        moving horizontally to the left at the same speed.
    What is the magnitude of the change in momentum of the ball?
    \begin{multicols}{2}
    \begin{choices}
        \wrongchoice{\SI{0}{\kilo\gram\meter\per\second}} 
        \wrongchoice{\SI{2}{\kilo\gram\meter\per\second}} 
        \wrongchoice{\SI{3}{\kilo\gram\meter\per\second}} 
        \wrongchoice{\SI{4}{\kilo\gram\meter\per\second}} 
      \correctchoice{\SI{6}{\kilo\gram\meter\per\second}} 
    \end{choices}
    \end{multicols}
\end{question}
}

\element{cpo-mc}{
\begin{question}{exam02-Q30}
    Potential energy is the energy possessed by an object due to
    \begin{multicols}{2}
    \begin{choices}
        \wrongchoice{its momentum.}
      \correctchoice{its position.}
        \wrongchoice{its velocity.}
        \wrongchoice{its acceleration.}
        \wrongchoice{its shape.}
    \end{choices}
    \end{multicols}
\end{question}
}

\element{cpo-mc}{
\begin{question}{exam02-Q31}
    Which of the following is true?
    \begin{choices}
        \wrongchoice{A body with zero velocity cannot have any potential energy.}
        \wrongchoice{A body with zero acceleration cannot have any kinetic energy.}
        \wrongchoice{A body with zero acceleration cannot have any potential energy.}
      \correctchoice{A body with zero velocity cannot have any kinetic energy.}
        \wrongchoice{A body with zero potential energy cannot have any velocity.}
    \end{choices}
\end{question}
}

\element{cpo-mc}{
\begin{question}{exam02-Q32}
    If two objects of different mass have the same non-zero momentum,
    \begin{choices}
      \correctchoice{the one with less mass will have the greater kinetic energy.}
        \wrongchoice{the one with more mass will have the greater kinetic energy.}
        \wrongchoice{they will have the same kinetic energy.}
        \wrongchoice{the one with the higher speed will have the greater mass.}
        \wrongchoice{the one with the lower speed will have the greater kinetic energy.}
    \end{choices}
\end{question}
}

\element{cpo-mc}{
\begin{question}{exam02-Q33}
    A car traveling at \SI{60}{\kilo\meter\per\hour} passes a truck going
        \SI{30}{\kilo\meter\per\hour} that has four times the mass of the car.
    Which of the following is true?
    \begin{choices}
        \wrongchoice{The car and the truck have the same momentum and the same kinetic energy.}
        \wrongchoice{The car has the same momentum and twice as much kinetic energy as the truck.}
        \wrongchoice{The car has the same momentum and four times as much kinetic energy as the truck.}
        \wrongchoice{The car has the same kinetic energy and twice as much momentum as the truck.}
      \correctchoice{The car has the same kinetic energy and half as much momentum as the truck.}
    \end{choices}
\end{question}
}

\element{cpo-mc}{
\begin{question}{exam02-Q34}
    A swinging pendulum has \rule[-0.1pt]{4em}{0.1pt} at the bottom (middle) of its arc.
    \begin{choices}
        \wrongchoice{minimum kinetic energy}
        \wrongchoice{minimum total energy}
      \correctchoice{minimum potential energy}
        \wrongchoice{maximum total energy}
        \wrongchoice{maximum potential energy}
    \end{choices}
\end{question}
}

\element{cpo-mc}{
\begin{question}{exam02-Q35}
    Real machines are not \SI{100}{\percent} efficient because
    \begin{choices}
      \correctchoice{some of the energy input is always transformed into thermal energy.}
        \wrongchoice{some of the energy input is always transformed into gravitational potential energy.}
        \wrongchoice{the energy input is always less than the energy output.}
        \wrongchoice{that would require the work output to be 100 times the work input, which is impossible.}
        \wrongchoice{that would require the work input to be 100 times the work output, which is impossible.}
    \end{choices}
\end{question}
}

\element{cpo-mc}{
\begin{question}{exam02-Q36}
    A physicist does \SI{100}{\joule} of work on a simple machine that
        raises a box of books through a height of \SI{0.2}{\meter}.
    If the efficiency of the machine is \SI{60}{\percent},
        how much work is converted to thermal energy by this process?
    \begin{multicols}{2}
    \begin{choices}
      \correctchoice{\SI{40}{\joule}}
        \wrongchoice{\SI{60}{\joule}}
        \wrongchoice{\SI{80}{\joule}}
        \wrongchoice{\SI{20}{\joule}}
        \wrongchoice{\SI{100}{\joule}}
    \end{choices}
    \end{multicols}
\end{question}
}

\element{cpo-mc}{
\begin{question}{exam02-Q37}
    When you run up two flights of stairs instead of walking up them,
        you feel more tired because
    \begin{choices}
        \wrongchoice{you do more work when you run than when you walk.}
      \correctchoice{your power output is greater when you run than when you walk.}
        \wrongchoice{the gravitational force is greater on a running person than on a walking person.}
        \wrongchoice{the gravitational acceleration is greater on a running person than on a walking person.}
        \wrongchoice{a running person has more inertia than a walking person.}
    \end{choices}
\end{question}
}

\element{cpo-mc}{
\begin{question}{exam02-Q38}
    The work done against gravity in moving a box with a mass of \SI{5}{\kilo\gram}
        through a height of \SI{3}{\meter} is
    \begin{multicols}{2}
    \begin{choices}
      \correctchoice{\SI{150}{\joule}}
        \wrongchoice{\SI{150}{\newton}}
        \wrongchoice{\SI{15}{\joule}}
        \wrongchoice{\SI{15}{\newton}}
        \wrongchoice{\SI{5/3}{\joule}}
    \end{choices}
    \end{multicols}
\end{question}
}

\element{cpo-mc}{
\begin{question}{exam02-Q39}
    Angular momentum is the product of
    \begin{choices}
      \correctchoice{rotational inertia and rotational velocity.}
        \wrongchoice{linear momentum and angle.}
        \wrongchoice{mass and velocity.}
        \wrongchoice{force and impulse.}
        \wrongchoice{acceleration and time.}
    \end{choices}
\end{question}
}

\element{cpo-mc}{
\begin{question}{exam02-Q40}
    When you stand in equilibrium on only one foot,
    \begin{choices}
      \correctchoice{your center of mass will be directly above that foot.}
        \wrongchoice{your center of mass will be directly above the other foot.}
        \wrongchoice{your center of mass will be directly above a point equidistant between your two feet.}
        \wrongchoice{your rotational inertia will be zero.}
        \wrongchoice{you will always fall over.}
    \end{choices}
\end{question}
}

\element{cpo-mc}{
\begin{question}{exam02-Q41}
    When a car rounds a curve at high speed,
    \begin{choices}
        \wrongchoice{the tires exert a centripetal force on the road.}
      \correctchoice{the road exerts a centripetal force on the tires.}
        \wrongchoice{the car exerts a centripetal force on the road.}
        \wrongchoice{the car body exerts a centripetal force on the tires.}
        \wrongchoice{there are no centripetal forces involved.}
    \end{choices}
\end{question}
}

\element{cpo-mc}{
\begin{question}{exam02-Q42}
    On a spinning disk,
        points closer to the outer edge will have \rule[-0.1pt]{4em}{0.1pt} points near the center.
    \begin{choices}
      \correctchoice{the same rotational speed as and greater tangential speed than}
        \wrongchoice{the same rotational speed as and lower tangential speed than}
        \wrongchoice{the same tangential speed as and greater rotational speed than}
        \wrongchoice{the same tangential speed as and lower rotational speed than}
        \wrongchoice{lower rotational speed and higher tangential speed than}
    \end{choices}
\end{question}
}

\element{cpo-mc}{
\begin{question}{exam02-Q43}
    A merry-go-round rotates \num{9} times each minute such that a point on its
        rim moves at a rate of \SI{3}{\meter\per\second}.
    At a point \num{2/3} of the way out from the center to the rim,
        the tangential speed would be:
    \begin{multicols}{2}
    \begin{choices}
        \wrongchoice{\SI{6}{\rotation\per\minute}} 
      \correctchoice{\SI{2}{\meter\per\second}} 
        \wrongchoice{\SI{3}{\meter\per\second}} 
        \wrongchoice{\SI{9}{\rotation\per\minute}} 
        \wrongchoice{\SI{3}{\rotation\per\minute}} 
    \end{choices}
    \end{multicols}
\end{question}
}

\element{cpo-mc}{
\begin{question}{exam02-Q44}
    An empty soup can and a full one are rolled side-by-side down an incline.
    If they start together, which one will reach the bottom first?
    \begin{choices}
        \wrongchoice{The empty can arrives first.}
      \correctchoice{The full can arrives first.}
        \wrongchoice{They will arrive together.}
        \wrongchoice{It depends on the diameters of the cans.}
        \wrongchoice{It depends on the kind of soup.}
    \end{choices}
\end{question}
}

\element{cpo-mc}{
\begin{question}{exam02-Q45}
    A mass of \SI{1}{\kilo\gram} is tied to a string and swung in a
        horizontal circle of radius \SI{1}{\meter};
    if the mass is then decreased to \SI{0.5}{\kilo\gram},
        the rotational inertia of this new system will be \rule[-0.1pt]{4em}{0.1pt} as before.
    \begin{choices}
        \wrongchoice{twice as much}
        \wrongchoice{four times as much}
        \wrongchoice{the same}
      \correctchoice{one half as much}
        \wrongchoice{one fourth as much}
    \end{choices}
\end{question}
}

\element{cpo-mc}{
\begin{question}{exam02-Q46}
    Torque is the product of
    \begin{choices}
      \correctchoice{lever arm and force.}
        \wrongchoice{mass and radius.}
        \wrongchoice{rotational inertia and velocity.}
        \wrongchoice{force and velocity.}
        \wrongchoice{lever arm and rotational inertia.}
    \end{choices}
\end{question}
}

\element{cpo-mc}{
\begin{question}{exam02-Q47}
    A \SI{60}{\kilo\gram} grandfather and his \SI{30}{\kilo\gram}
        granddaughter are balanced on a seesaw.
    If the granddaughter is sitting \SI{2}{\meter} from the pivot point,
        the grandfather must be sitting \rule[-0.1pt]{4em}{0.1pt} from it.
    \begin{choices}
        \wrongchoice{\SI{4}{\meter}}
        \wrongchoice{\SI{2}{\meter}}
        \wrongchoice{\SI{3}{\meter}}
      \correctchoice{\SI{1}{\meter}} 
        \wrongchoice{\SI{0.5}{\meter}}
    \end{choices}
\end{question}
}

\endinput


