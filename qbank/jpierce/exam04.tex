

%% Physics 101 Sample Test Questions by Dr. James Pierce
%%------------------------------------------------------------


%% Exam #4 ---- (heat transfer, waves, sound, music, etc.)
%%------------------------------------------------------------
\element{cpo-mc}{
\begin{question}{exam04-Q73}
    In general, when a liquid is heated
    \begin{choices}
        \wrongchoice{it will neither expand nor contract.}
      \correctchoice{it will expand at a greater rate than a solid.}
        \wrongchoice{it will expand at a lesser rate than a solid.}
        \wrongchoice{it will contract at a greater rate than a solid.}
        \wrongchoice{it will contract at a lesser rate than a solid.}
    \end{choices}
\end{question}
}

\element{cpo-mc}{
\begin{question}{exam04-Q74}
    If a flat metal plate with a circular hole cut through it is heated,
    \begin{choices}
        \wrongchoice{the hole gets smaller.}
      \correctchoice{the hole gets larger.}
        \wrongchoice{the hole stays exactly the same size.}
        \wrongchoice{the hole may get larger or smaller, depending on the material of the plate.}
        \wrongchoice{the hole may get larger or smaller, depending on how much the plate is heated.}
    \end{choices}
\end{question}
}

\element{cpo-mc}{
\begin{question}{exam04-Q75}
    Water has a higher specific heat capacity than iron.
    This means that
    \begin{choices}
        \wrongchoice{water is more dense than iron.}
        \wrongchoice{water is hotter than iron.}
        \wrongchoice{water heats more rapidly than iron.}
      \correctchoice{water heats more slowly than iron.}
        \wrongchoice{water boils at a higher temperature than iron.}
    \end{choices}
\end{question}
}

\element{cpo-mc}{
\begin{question}{exam04-Q76}
    The specific heat capacity of water is \SI{1}{\calorie\per\gram\per\degreeCelsius}.
    This means that it will take \rule[-0.1pt]{4em}{0.1pt} to increase
        the temperature of \SI{10}{\gram} of water by \SI{10}{\degreeCelsius}.
    \begin{multicols}{2}
    \begin{choices}
        \wrongchoice{\SI{20}{\calorie}}
        \wrongchoice{\SI{0.1}{\calorie}}
        \wrongchoice{\SI{1}{\calorie}}
        \wrongchoice{\SI{10}{\calorie}}
      \correctchoice{\SI{100}{\calorie}}
    \end{choices}
    \end{multicols}
\end{question}
}

\element{cpo-mc}{
\begin{question}{exam04-Q77}
    Water reaches its highest density at a temperature of:
    \begin{multicols}{2}
    \begin{choices}
        \wrongchoice{\SI{0}{\degreeCelsius}}
      \correctchoice{\SI{4}{\degreeCelsius}}
        \wrongchoice{\SI{10}{\degreeCelsius}}
        \wrongchoice{\SI{-10}{\degreeCelsius}}
        \wrongchoice{\SI{-4}{\degreeCelsius}}
    \end{choices}
    \end{multicols}
\end{question}
}

\element{cpo-mc}{
\begin{question}{exam04-Q78}
    Which of these is an example of heat transfer by conduction?
    \begin{choices}
      \correctchoice{The handle of a metal spoon becomes hot when you use it to stir a pot of soup on the stove.}
        \wrongchoice{The air near the ceiling is normally warmer than air near the floor.}
        \wrongchoice{You can boil water in a microwave oven.}
        \wrongchoice{You feel the heat from a bonfire even though you are several meters away from it.}
        \wrongchoice{Smoke rises up a chimney.}
    \end{choices}
\end{question}
}

\element{cpo-mc}{
\begin{question}{exam04-Q79}
    Rising air tends to
    \begin{choices}
      \correctchoice{expand and become cooler.}
        \wrongchoice{expand and become warmer.}
        \wrongchoice{become denser and warmer.}
        \wrongchoice{become denser and cooler.}
        \wrongchoice{maintain a constant density and temperature.}
    \end{choices}
\end{question}
}

\element{cpo-mc}{
\begin{question}{exam04-Q80}
    Radiation is heat transfer by
    \begin{choices}
        \wrongchoice{molecular and electronic collisions.}
      \correctchoice{electromagnetic waves.}
        \wrongchoice{bulk fluid motions.}
        \wrongchoice{atmospheric currents.}
        \wrongchoice{direct contact.}
    \end{choices}
\end{question}
}

\element{cpo-mc}{
\begin{question}{exam04-Q81}
    The pattern formed by overlapping waves in a bow wave is in the shape of the letter:
    \begin{multicols}{3}
    \begin{choices}
        \wrongchoice{B}
        \wrongchoice{U}
      \correctchoice{V}
        \wrongchoice{I}
        \wrongchoice{T}
    \end{choices}
    \end{multicols}
\end{question}
}

\element{cpo-mc}{
\begin{question}{exam04-Q82}
    The Doppler effect causes
    \begin{choices}
        \wrongchoice{the observed pitch of a sound to be lower if the source of sound is approaching the observer.}
        \wrongchoice{the observed pitch of a sound to be higher if the source of sound is moving away from the observer.}
      \correctchoice{the observed pitch of a sound to be lower if the source of sound is moving away from the observer.}
        \wrongchoice{the speed of sound to increase if the source of sound is moving away from the observer.}
        \wrongchoice{the speed of sound to decrease if the source of sound is moving away from the observer.}
    \end{choices}
\end{question}
}

\element{cpo-mc}{
\begin{question}{exam04-Q83}
    In a \rule[-0.1pt]{4em}{0.1pt} wave,
        the medium vibrates in a direction that is perpendicular to the direction the wave travels.
    \begin{multicols}{2}
    \begin{choices}
        \wrongchoice{sound}
        \wrongchoice{longitudinal}
        \wrongchoice{perpendicular}
      \correctchoice{transverse}
        \wrongchoice{normal}
    \end{choices}
    \end{multicols}
\end{question}
}

\element{cpo-mc}{
\begin{question}{exam04-Q84}
    The period of a pendulum depends on
    \begin{choices}
        \wrongchoice{the mass of the pendulum and the size of the arc it swings through.}
        \wrongchoice{the length of the pendulum and the size of the arc it swings through.}
        \wrongchoice{the mass of the pendulum and the acceleration of gravity.}
      \correctchoice{the length of the pendulum and the acceleration of gravity.}
        \wrongchoice{the weight of the pendulum and the material it is made from.}
    \end{choices}
\end{question}
}

\element{cpo-mc}{
\begin{question}{exam04-Q85}
    A wave that has a relatively long wavelength will also have a relatively
    \begin{multicols}{2}
    \begin{choices}
        \wrongchoice{high frequency.}
      \correctchoice{long period.}
        \wrongchoice{large amplitude.}
        \wrongchoice{high speed.}
        \wrongchoice{small amplitude.}
    \end{choices}
    \end{multicols}
\end{question}
}

\element{cpo-mc}{
\begin{question}{exam04-Q86}
    A train of freight cars, each \SI{10}{\meter} long,
        rolls by at the rate of \num{2} cars each second.
    What is the speed of the train?
    \begin{multicols}{2}
    \begin{choices}
        \wrongchoice{\SI{10}{\meter\per\second}} 
        \wrongchoice{\SI{2}{\meter\per\second}} 
        \wrongchoice{\SI{5}{\meter\per\second}} 
      \correctchoice{\SI{20}{\meter\per\second}} 
        \wrongchoice{\SI{12}{\meter\per\second}} 
    \end{choices}
    \end{multicols}
\end{question}
}

\element{cpo-mc}{
\begin{question}{exam04-Q87}
    Compared to a \SI{500}{\hertz} sound,
        a \SI{300}{\hertz} sound would have
    \begin{choices}
      \correctchoice{a longer wavelength and the same speed.}
        \wrongchoice{a longer wavelength and a lower speed.}
        \wrongchoice{a longer wavelength and a higher speed.}
        \wrongchoice{a shorter wavelength and a lower speed.}
        \wrongchoice{a shorter wavelength and the same speed.}
    \end{choices}
\end{question}
}

\element{cpo-mc}{
\begin{question}{exam04-Q88}
    A vibrating string is being tuned to match a tuning fork with a frequency of \SI{256}{\hertz}.
    When \num{3} beats per second are heard, the vibration frequency of the string must be
    \begin{multicols}{2}
    \begin{choices}
        \wrongchoice{\SI{256}{\hertz}}
        \wrongchoice{\SI{253}{\hertz}}
        \wrongchoice{\SI{259}{\hertz}}
      \correctchoice{either \SI{253}{\hertz} or \SI{259}{\hertz}.} 
        \wrongchoice{\SI{3}{\hertz}}
    \end{choices}
    \end{multicols}
\end{question}
}

\element{cpo-mc}{
\begin{question}{exam04-Q89}
    Constructive interference of sound waves occurs
    \begin{choices}
        \wrongchoice{whenever there is an echo.}
      \correctchoice{when two waves arrive at the same point in phase with each other.}
        \wrongchoice{when two waves arrive at the same point out of phase with each other.}
        \wrongchoice{whenever sound waves are refracted by air layers of different temperatures.}
        \wrongchoice{whenever sound waves are reflected off distant buildings.}
    \end{choices}
\end{question}
}

\element{cpo-mc}{
\begin{question}{exam04-Q90}
    Pushing a person on a swing at the same rate as the natural frequency
        of the swing/pendulum is an example of
    \begin{choices}
        \wrongchoice{destructive interference.}
        \wrongchoice{constructive interference.}
      \correctchoice{resonance.}
        \wrongchoice{the Doppler effect.}
        \wrongchoice{refraction.}
    \end{choices}
\end{question}
}

\element{cpo-mc}{
\begin{question}{exam04-Q91}
    Sound travels faster in air at
    \begin{choices}
        \wrongchoice{lower temperatures because the molecules move faster and collide more frequently.}
        \wrongchoice{lower temperatures because the molecules are closer together and collide more frequently.}
      \correctchoice{higher temperatures because the molecules move faster and collide more frequently.}
        \wrongchoice{higher temperatures because the molecules are closer together and collide more frequently.}
        \wrongchoice{lower temperatures because the air is more solid then.}
    \end{choices}
\end{question}
}

\element{cpo-mc}{
\begin{question}{exam04-Q92}
    An intensity of \SI{60}{\deci\bel} is \rule[-0.1pt]{4em}{0.1pt} times
        as intense as an intensity of \SI{30}{\deci\bel}.
    \begin{multicols}{2}
    \begin{choices}
        \wrongchoice{\num{2}}
        \wrongchoice{\num{30}}
        \wrongchoice{\num{60}}
        \wrongchoice{\num{90}}
      \correctchoice{\num{1000}}
    \end{choices}
    \end{multicols}
\end{question}
}

\element{cpo-mc}{
\begin{question}{exam04-Q93}
    The ``highness'' or ``lowness'' of a musical tone is called the:
    \begin{multicols}{2}
    \begin{choices}
        \wrongchoice{loudness}
        \wrongchoice{rhythm}
        \wrongchoice{scale}
      \correctchoice{pitch}
        \wrongchoice{intensity}
    \end{choices}
    \end{multicols}
\end{question}
}

\element{cpo-mc}{
\begin{question}{exam04-Q94}
    Partial tones whose frequencies are whole number multiples of the
        fundamental frequency are called
    \begin{multicols}{2}
    \begin{choices}
        \wrongchoice{noise.}
        \wrongchoice{integers.}
        \wrongchoice{radicals.}
      \correctchoice{harmonics.}
        \wrongchoice{tonics.}
    \end{choices}
    \end{multicols}
\end{question}
}

\element{cpo-mc}{
\begin{question}{exam04-Q95}
    When a guitar string vibrates at the frequency of its third harmonic,
        it will have a node at each end and \rule[-0.1pt]{4em}{0.1pt} in between.
    \begin{multicols}{2}
    \begin{choices}
        \wrongchoice{no nodes}
        \wrongchoice{one node}
      \correctchoice{two nodes}
        \wrongchoice{three nodes}
        \wrongchoice{four nodes}
    \end{choices}
    \end{multicols}
\end{question}
}

\endinput


