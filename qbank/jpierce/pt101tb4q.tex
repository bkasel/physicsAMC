

%% Physics 101 Sample Test Questions by Dr. James Pierce
%%------------------------------------------------------------


%% JP's Physics 101 Test Bank 4
%%--------------------------------------------------


%% Topic: expansion
\element{jpierce}{
\begin{question}{pt101tb4-Q01}
    In general, when a solid is heated,
    \begin{choices}
        \wrongchoice{it will contract at a lesser rate than a liquid.}
        \wrongchoice{it will contract at a greater rate than a liquid.}
        \wrongchoice{it will neither expand nor contract.}
        \wrongchoice{it will expand at a greater rate than a liquid.}
      \correctchoice{it will expand at a lesser rate than a liquid.}
    \end{choices}
\end{question}
}

\element{jpierce}{
\begin{question}{pt101tb4-Q02}
    In general, when a solid is cooled,
    \begin{choices}
      \correctchoice{it will contract at a lesser rate than a liquid.}
        \wrongchoice{it will expand at a greater rate than a liquid.}
        \wrongchoice{it will neither expand nor contract.}
        \wrongchoice{it will contract at a greater rate than a liquid.}
        \wrongchoice{it will expand at a lesser rate than a liquid.}
    \end{choices}
\end{question}
}

\element{jpierce}{
\begin{question}{pt101tb4-Q03}
    In general, when a liquid is heated,
    \begin{choices}
        \wrongchoice{it will contract at a lesser rate than a solid.}
        \wrongchoice{it will expand at a lesser rate than a solid.}
        \wrongchoice{it will neither expand nor contract.}
      \correctchoice{it will expand at a greater rate than a solid.}
        \wrongchoice{it will contract at a greater rate than a solid.}
    \end{choices}
\end{question}
}

\element{jpierce}{
\begin{question}{pt101tb4-Q04}
    In general, when a liquid is cooled,
    \begin{choices}
        \wrongchoice{it will contract at a lesser rate than a solid.}
        \wrongchoice{it will expand at a greater rate than a solid.}
      \correctchoice{it will contract at a greater rate than a solid.}
        \wrongchoice{it will neither expand nor contract.}
        \wrongchoice{it will expand at a lesser rate than a solid.}
    \end{choices}
\end{question}
}

\element{jpierce}{
\begin{question}{pt101tb4-Q05}
    Liquids \rule[-0.1pt]{4em}{0.1pt} their volume with
        higher temperature due to \rule[-0.1pt]{4em}{0.1pt}.
    \begin{choices}
        \wrongchoice{increase; decreased crystal structure}
        \wrongchoice{increase; less molecular motion}
        \wrongchoice{decrease; greater molecular motion}
        \wrongchoice{decrease; less molecular motion}
      \correctchoice{increase; greater molecular motion}
    \end{choices}
\end{question}
}

\element{jpierce}{
\begin{question}{pt101tb4-Q06}
    Incandescent light bulbs are made of very thin glass because
    \begin{choices}
        \wrongchoice{thicker glass would expand less as the light bulb heats up, causing cracking.}
      \correctchoice{thicker glass would expand more as the light bulb heats up, causing cracking.}
        \wrongchoice{thin glass has a higher specific heat capacity than thick glass.}
        \wrongchoice{thin glass has a lower specific heat capacity than thick glass.}
        \wrongchoice{thicker glass is more expensive.}
    \end{choices}
\end{question}
}

\element{jpierce}{
\begin{question}{pt101tb4-Q07}
    If a flat metal plate with a circular hole cut through it is heated,
    \begin{choices}
        \wrongchoice{the hole gets smaller.}
        \wrongchoice{the hole may get larger or smaller, depending on how much the plate is heated.}
        \wrongchoice{the hole may get larger or smaller, depending on the material of the plate.}
      \correctchoice{the hole gets larger.}
        \wrongchoice{the hole stays exactly the same size.}
    \end{choices}
\end{question}
}

\element{jpierce}{
\begin{question}{pt101tb4-Q08}
    When a mercury thermometer is warmed,
        the mercury level momentarily goes down before it rises. 
    Why?
    \begin{choices}
        \wrongchoice{The mercury crystals must first be melted before expansion can occur.}
        \wrongchoice{The thermometer glass initially expands, decreasing the diameter of the mercury capillary.}
      \correctchoice{The thermometer glass initially expands, increasing the diameter of the mercury capillary.}
        \wrongchoice{The ice crystals must first be melted before expansion can occur.}
        \wrongchoice{The thermometer glass initially contracts, increasing the diameter of the mercury capillary.}
    \end{choices}
\end{question}
}

\element{jpierce}{
\begin{question}{pt101tb4-Q09}
    When a mercury thermometer is warmed,
        the mercury level rises in the glass capillary. 
    Why?
    \begin{choices}
        \wrongchoice{The increased temperature makes the mercury more fluid, permitting it to flow more readily.}
      \correctchoice{The glass expands, increasing the capillary diameter, but mercury expands even more and rises.}
        \wrongchoice{The glass contracts, increasing the capillary diameter, allowing the mercury to rise more easily.}
        \wrongchoice{The glass contracts, decreasing the capillary diameter and squeezing the mercury up.}
        \wrongchoice{The glass expands, decreasing the capillary diameter and squeezing the mercury up.}
    \end{choices}
\end{question}
}

\element{jpierce}{
\begin{question}{pt101tb4-Q10}
    Some thermometers consist essentially of a pointer attached to a coiled,
        bimetallic strip of metal. 
    As the strip is heated,
    \begin{choices}
        \wrongchoice{the two metals on each side of the strip contract at different rates.}
        \wrongchoice{one of the metals expands while the other metal contracts.}
        \wrongchoice{one of the metals contracts while the other metal remains the same length.}
      \correctchoice{the two metals on each side of the strip expand at different rates.}
        \wrongchoice{one of the metals expands while the other metal remains the same length.}
    \end{choices}
\end{question}
}

%% Topic: Heat Capacity
\element{jpierce}{
\begin{question}{pt101tb4-Q11}
    Water has a higher specific heat capacity than iron. 
    This means that
    \begin{choices}
        \wrongchoice{water boils at a higher temperature than iron.}
        \wrongchoice{water is hotter than iron.}
        \wrongchoice{water heats more rapidly than iron.}
      \correctchoice{water heats more slowly than iron.}
        \wrongchoice{water is more dense than iron.}
    \end{choices}
\end{question}
}

\element{jpierce}{
\begin{question}{pt101tb4-Q12}
    Water is a useful cooling agent for automobile engines because it has a relatively
    \begin{choices}
        \wrongchoice{low density.}
      \correctchoice{high specific heat capacity.}
        \wrongchoice{low specific heat capacity.}
        \wrongchoice{low temperature.}
        \wrongchoice{high density.}
    \end{choices}
\end{question}
}

\element{jpierce}{
\begin{question}{pt101tb4-Q13}
    Because the specific heat capacity of water is \rule[-0.1pt]{4em}{0.1pt} that of land,
        water temperatures fluctuate \rule[-0.1pt]{4em}{0.1pt} land temperatures.
    \begin{choices}
      \correctchoice{higher than; less rapidly than}
        \wrongchoice{higher than; more rapidly than}
        \wrongchoice{equal to; at the same rate as}
        \wrongchoice{lower than; less rapidly than}
        \wrongchoice{lower than; more rapidly than}
    \end{choices}
\end{question}
}

\element{jpierce}{
\begin{question}{pt101tb4-Q14}
    Desert sand is very hot during the day and very cool during the night because
    \begin{choices}
        \wrongchoice{sand expands when it is hot and contracts when it is cool.}
      \correctchoice{sand has a relatively low specific heat capacity.}
        \wrongchoice{sand expands when it is cool and contracts when it is hot.}
        \wrongchoice{sand has a relatively high specific heat capacity.}
        \wrongchoice{sand has a relatively high melting point.}
    \end{choices}
\end{question}
}

\element{jpierce}{
\begin{question}{pt101tb4-Q15}
    An object with a relatively high specific heat capacity
    \begin{choices}
        \wrongchoice{tends to cool down rather quickly when removed from sources of heat.}
      \correctchoice{tends to be fairly resistant to changes in temperature.}
        \wrongchoice{is always very cold.}
        \wrongchoice{is always very hot.}
        \wrongchoice{tends to warm up rather quickly when exposed to sources of heat.}
    \end{choices}
\end{question}
}

\element{jpierce}{
\begin{question}{pt101tb4-Q16}
    An object with a relatively low specific heat capacity
    \begin{choices}
        \wrongchoice{tends to be fairly resistant to changes in temperature.}
        \wrongchoice{is always very cold.}
        \wrongchoice{is always very hot.}
        \wrongchoice{tends to warm up rather slowly when exposed to sources of heat.}
      \correctchoice{tends to cool down rather quickly when removed from sources of heat.}
    \end{choices}
\end{question}
}

\element{jpierce}{
\begin{question}{pt101tb4-Q17}
    An object with a relatively low specific heat capacity
    \begin{choices}
        \wrongchoice{is always very hot.}
        \wrongchoice{tends to be fairly resistant to changes in temperature.}
      \correctchoice{tends to warm up rather quickly when exposed to sources of heat.}
        \wrongchoice{is always very cold.}
        \wrongchoice{tends to cool down rather slowly when removed from sources of heat.}
    \end{choices}
\end{question}
}

\element{jpierce}{
\begin{question}{pt101tb4-Q18}
    An object with a relatively low specific heat capacity
    \begin{choices}
        \wrongchoice{tends to warm up quickly and cool down slowly.}
        \wrongchoice{tends to warm up slowly and cool down quickly.}
        \wrongchoice{tends to float in water.}
        \wrongchoice{tends to warm up slowly and cool down slowly.}
      \correctchoice{tends to warm up quickly and cool down quickly.}
    \end{choices}
\end{question}
}

\element{jpierce}{
\begin{question}{pt101tb4-Q19}
    An object with a relatively high specific heat capacity
    \begin{choices}
        \wrongchoice{tends to warm up quickly and cool down slowly.}
        \wrongchoice{tends to warm up slowly and cool down quickly.}
        \wrongchoice{tends to float in water.}
        \wrongchoice{tends to warm up quickly and cool down quickly.}
      \correctchoice{tends to warm up slowly and cool down slowly.}
    \end{choices}
\end{question}
}

\element{jpierce}{
\begin{question}{pt101tb4-Q20}
    The specific heat capacity of water is 1 calorie per gram per degree Celsius. 
    This means that 1 calorie will increase the temperature of 1 gram
        of water by \rule[-0.1pt]{4em}{0.1pt} degree(s).
    \begin{multicols}{2}
    \begin{choices}
        \wrongchoice{\num{0.01}}
        \wrongchoice{\num{0.1}}
      \correctchoice{\num{1}}
        \wrongchoice{\num{100}}
        \wrongchoice{\num{10}}
    \end{choices}
    \end{multicols}
\end{question}
}

\element{jpierce}{
\begin{question}{pt101tb4-Q21}
    Two identical beakers are placed on a hot plate. 
    Beaker $A$ contains \SI{100}{\gram} of water while beaker $B$
        contains \SI{500}{\gram} of water. 
    When the hot plate is turned on,
        in which beaker will the water temperature rise more rapidly?
    \begin{choices}
      \correctchoice{$A$, because it contains less water to heat.}
        \wrongchoice{$B$, because it contains more water to heat.}
        \wrongchoice{$B$, because it can absorb more heat.}
        \wrongchoice{Neither---the rates will be the same because the beakers are identical.}
        \wrongchoice{$A$, because it can absorb more heat.}
    \end{choices}
\end{question}
}

\element{jpierce}{
\begin{question}{pt101tb4-Q22}
    Two identical beakers are placed on a hot plate.
    Beaker $A$ contains \SI{500}{\gram} of water while beaker $B$
        contains \SI{100}{\gram} of water. 
    When the hot plate is turned on,
        in which beaker will the water temperature rise more rapidly?
    \begin{choices}
        \wrongchoice{$A$, because it contains more water to heat.}
        \wrongchoice{Neither---the rates will be the same because the beakers are identical.}
        \wrongchoice{$B$, because it can absorb more heat.}
      \correctchoice{$B$, because it contains less water to heat.}
        \wrongchoice{$A$, because it can absorb more heat.}
    \end{choices}
\end{question}
}

\element{jpierce}{
\begin{question}{pt101tb4-Q23}
    The specific heat capacity of water is 1 calorie per gram per degree Celsius. 
    This means that it will take \rule[-0.1pt]{4em}{0.1pt} calorie(s)
        to increase the temperature of 1 gram of water by 10 degrees.
    \begin{multicols}{3}
    \begin{choices}
      \correctchoice{\num{10}}
        \wrongchoice{\num{0.1}}
        \wrongchoice{\num{0.01}}
        \wrongchoice{\num{1}}
        \wrongchoice{\num{100}}
    \end{choices}
    \end{multicols}
\end{question}
}

\element{jpierce}{
\begin{question}{pt101tb4-Q24}
    The specific heat capacity of water is 1 calorie per gram per degree Celsius. 
    This means that it will take \rule[-0.1pt]{4em}{0.1pt} calorie(s)
        to increase the temperature of 10 grams of water by 1 degree.
    \begin{multicols}{3}
    \begin{choices}
        \wrongchoice{\num{0.1}}
        \wrongchoice{\num{0.01}}
        \wrongchoice{\num{100}}
        \wrongchoice{\num{1}}
      \correctchoice{\num{10}}
    \end{choices}
    \end{multicols}
\end{question}
}

\element{jpierce}{
\begin{question}{pt101tb4-Q25}
    The specific heat capacity of water is 1 calorie per gram per degree Celsius. 
    This means that 10 calories will increase the temperature of 1 gram of water
        by \rule[-0.1pt]{4em}{0.1pt}  degree(s).
    \begin{multicols}{3}
    \begin{choices}
        \wrongchoice{\num{0.1}}
        \wrongchoice{\num{100}}
        \wrongchoice{\num{1}}
        \wrongchoice{\num{0.01}}
      \correctchoice{\num{10}}
    \end{choices}
    \end{multicols}
\end{question}
}

\element{jpierce}{
\begin{question}{pt101tb4-Q26}
    The specific heat capacity of water is 1 calorie per gram per degree Celsius. 
    This means that 10 calories will increase the temperature of 10 grams of water
        by \rule[-0.1pt]{4em}{0.1pt}  degree(s).
    \begin{multicols}{3}
    \begin{choices}
        \wrongchoice{\num{10}}
      \correctchoice{\num{1}}
        \wrongchoice{\num{0.1}}
        \wrongchoice{\num{100}}
        \wrongchoice{\num{0.01}}
    \end{choices}
    \end{multicols}
\end{question}
}

\element{jpierce}{
\begin{question}{pt101tb4-Q27}
    The specific heat capacity of water is 1 calorie per gram per degree Celsius. 
    This means that 100 calories will increase the temperature of 10 grams of water
        by \rule[-0.1pt]{4em}{0.1pt}  degree(s).
    \begin{multicols}{3}
    \begin{choices}
        \wrongchoice{\num{1}}
        \wrongchoice{\num{0.01}}
        \wrongchoice{\num{100}}
      \correctchoice{\num{10}}
        \wrongchoice{\num{0.1}}
    \end{choices}
    \end{multicols}
\end{question}
}

\element{jpierce}{
\begin{question}{pt101tb4-Q28}
    The specific heat capacity of water is 1 calorie per gram per degree Celsius. 
    This means that 10 calories will increase the temperature of 100 grams of water
        by \rule[-0.1pt]{4em}{0.1pt} degree(s).
    \begin{multicols}{3}
    \begin{choices}
      \correctchoice{\num{0.1}}
        \wrongchoice{\num{1}}
        \wrongchoice{\num{100}}
        \wrongchoice{\num{0.01}}
        \wrongchoice{\num{10}}
    \end{choices}
    \end{multicols}
\end{question}
}

\element{jpierce}{
\begin{question}{pt101tb4-Q29}
    The specific heat capacity of water is 1 calorie per gram per degree Celsius. 
    This means that it will take \rule[-0.1pt]{4em}{0.1pt} calorie(s)
        to increase the temperature of 10 grams of water by 10 degrees.
    \begin{multicols}{3}
    \begin{choices}
        \wrongchoice{\num{0.1}}
        \wrongchoice{\num{1}}
        \wrongchoice{\num{20}}
        \wrongchoice{\num{10}}
      \correctchoice{\num{100}}
    \end{choices}
    \end{multicols}
\end{question}
}

%% Topic: Temperature
\element{jpierce}{
\begin{question}{pt101tb4-Q30}
    The temperature of a gas is a measure of
        the \rule[-0.1pt]{4em}{0.1pt} of a molecule in the gas.
    \begin{choices}
        \wrongchoice{mass}
        \wrongchoice{shape}
      \correctchoice{average kinetic energy}
        \wrongchoice{acceleration}
        \wrongchoice{gravitational potential energy}
    \end{choices}
\end{question}
}

\element{jpierce}{
\begin{question}{pt101tb4-Q31}
    A common thermometer measures temperature by means of
    \begin{choices}
        \wrongchoice{the rotation of a solid sphere.}
      \correctchoice{the expansion and contraction of a liquid.}
        \wrongchoice{the rising and falling of bubble of gas in water.}
        \wrongchoice{the changing color of a fluid.}
        \wrongchoice{the compression of a spring.}
    \end{choices}
\end{question}
}

\element{jpierce}{
\begin{question}{pt101tb4-Q32}
    Which of these temperature scales has its zero point located at absolute zero?
    \begin{choices}
        \wrongchoice{the Celsius scale}
      \correctchoice{the Kelvin scale}
        \wrongchoice{the Fahrenheit scale}
        \wrongchoice{the Bernoulli scale}
        \wrongchoice{the Archimedes scale}
    \end{choices}
\end{question}
}

%% Topic: Water/Ice
\element{jpierce}{
\begin{question}{pt101tb4-Q33}
    Water reaches its highest density at a temperature of:
    \begin{multicols}{2}
    \begin{choices}
        \wrongchoice{\SI{-4}{\degreeCelsius}}
        \wrongchoice{\SI{0}{\degreeCelsius}}
      \correctchoice{\SI{4}{\degreeCelsius}}
        \wrongchoice{\SI{-10}{\degreeCelsius}}
        \wrongchoice{\SI{10}{\degreeCelsius}}
    \end{choices}
    \end{multicols}
\end{question}
}

\element{jpierce}{
\begin{question}{pt101tb4-Q34}
    If the density of water were greatest at its freezing point,
        Minnesota lakes would
    \begin{choices}
        \wrongchoice{be warmest at the bottom.}
        \wrongchoice{not freeze.}
      \correctchoice{freeze from the bottom up.}
        \wrongchoice{freeze from the top down.}
        \wrongchoice{be most dense at the top.}
    \end{choices}
\end{question}
}

\element{jpierce}{
\begin{question}{pt101tb4-Q35}
    Ice is \rule[-0.1pt]{4em}{0.1pt} dense than water because
        of the \rule[-0.1pt]{4em}{0.1pt} of ice.
    \begin{choices}
        \wrongchoice{more; greater molecular motion}
        \wrongchoice{less; greater molecular motion}
        \wrongchoice{more; lower molecular motion}
      \correctchoice{less; crystal structure}
        \wrongchoice{more; crystal structure}
    \end{choices}
\end{question}
}

\element{jpierce}{
\begin{question}{pt101tb4-Q36}
    As the temperature of water is raised from \SI{1}{\degreeCelsius}
        to \SI{2}{\degreeCelsius}, what happens?
    \begin{choices}
        \wrongchoice{The water turns into a gas.}
      \correctchoice{The water contracts.}
        \wrongchoice{The water freezes into ice.}
        \wrongchoice{The density of the water remains constant.}
        \wrongchoice{The water expands.}
    \end{choices}
\end{question}
}

\element{jpierce}{
\begin{question}{pt101tb4-Q37}
    As the temperature of water is raised from \SI{6}{\degreeCelsius}
        to \SI{7}{\degreeCelsius}, what happens?
    \begin{choices}
        \wrongchoice{The density of the water remains constant.}
        \wrongchoice{The water freezes into ice.}
        \wrongchoice{The water contracts.}
      \correctchoice{The water expands.}
        \wrongchoice{The water turns into a gas.}
    \end{choices}
\end{question}
}

%% Topic: Conduction
\element{jpierce}{
\begin{question}{pt101tb4-Q38}
    Insulation keeps a house warm by:
    \begin{choices}
        \wrongchoice{decreasing the rate of molecular motion of the air inside.}
      \correctchoice{slowing the escape of heat to the outside.}
        \wrongchoice{preventing heat from escaping to the outside.}
        \wrongchoice{slowing the flow of cold to the inside.}
        \wrongchoice{preventing cold from getting inside.}
    \end{choices}
\end{question}
}

\element{jpierce}{
\begin{question}{pt101tb4-Q39}
    Bare feet standing on a bathroom rug feel warmer than the same
        feet standing on a cold linoleum floor:
    \begin{choices}
        \wrongchoice{because the rug slows the transfer of cold from the floor to the feet.}
        \wrongchoice{because the rug prevents cold from flowing from the floor to the feet.}
        \wrongchoice{because the rug is a good conductor of heat.}
        \wrongchoice{because the rug prevents heat from flowing from the feet to the floor.}
      \correctchoice{because the rug is a poor conductor of heat.}
    \end{choices}
\end{question}
}

\element{jpierce}{
\begin{question}{pt101tb4-Q40}
    Conduction is heat transfer by:
    \begin{choices}
        \wrongchoice{bulk fluid motions.}
      \correctchoice{atomic, molecular, and/or electronic collisions.}
        \wrongchoice{thermal expansion.}
        \wrongchoice{atmospheric currents.}
        \wrongchoice{electromagnetic waves.}
    \end{choices}
\end{question}
}

\element{jpierce}{
\begin{question}{pt101tb4-Q41}
    Which of these is an example of heat transfer by conduction?
    \begin{choices}
        \wrongchoice{You feel the heat from a bonfire even though you are several meters away from it.}
      \correctchoice{The handle of a metal spoon becomes hot when you use it to stir a pot of soup on the stove.}
        \wrongchoice{The air near the ceiling is normally warmer than air near the floor.}
        \wrongchoice{You can boil water in a microwave oven.}
        \wrongchoice{Smoke rises up a chimney.}
    \end{choices}
\end{question}
}

\element{jpierce}{
\begin{question}{pt101tb4-Q42}
    Which of these is an example of heat transfer by conduction?
    \begin{choices}
        \wrongchoice{You can boil water in a microwave oven.}
        \wrongchoice{The air near the ceiling is normally warmer than air near the floor.}
      \correctchoice{The bathroom floor feels cold when you stand on it in bare feet.}
        \wrongchoice{Smoke rises up a chimney.}
        \wrongchoice{You feel the heat from a bonfire even though you are several meters away from it.}
    \end{choices}
\end{question}
}

\element{jpierce}{
\begin{question}{pt101tb4-Q43}
    Heat transfer by conduction cannot occur:
    \begin{multicols}{2}
    \begin{choices}
        \wrongchoice{in a liquid.}
        \wrongchoice{at night.}
      \correctchoice{in a vacuum.}
        \wrongchoice{in a solid.}
        \wrongchoice{in a gas.}
    \end{choices}
    \end{multicols}
\end{question}
}

\element{jpierce}{
\begin{question}{pt101tb4-Q44}
    Convection is heat transfer by:
    \begin{choices}
        \wrongchoice{electromagnetic waves.}
      \correctchoice{bulk fluid motions.}
        \wrongchoice{molecular and electronic collisions.}
        \wrongchoice{direct contact.}
        \wrongchoice{molecular and electronic vibrations.}
    \end{choices}
\end{question}
}

\element{jpierce}{
\begin{question}{pt101tb4-Q45}
    Rising air tends to:
    \begin{choices}
        \wrongchoice{expand and become warmer.}
        \wrongchoice{become denser and warmer.}
        \wrongchoice{maintain a constant density and temperature.}
        \wrongchoice{become denser and cooler.}
      \correctchoice{expand and become cooler.}
    \end{choices}
\end{question}
}

\element{jpierce}{
\begin{question}{pt101tb4-Q46}
    Which of these is an example of heat transfer by convection?
    \begin{choices}
      \correctchoice{The air near the ceiling is normally warmer than air near the floor.}
        \wrongchoice{The bathroom floor feels cold when you stand on it in bare feet.}
        \wrongchoice{You feel the heat from a bonfire even though you are several meters away from it.}
        \wrongchoice{The handle of a metal spoon becomes hot when you use it to stir a pot of soup on the stove.}
        \wrongchoice{You can boil water in a microwave oven.}
    \end{choices}
\end{question}
}

\element{jpierce}{
\begin{question}{pt101tb4-Q47}
    Which of these is an example of heat transfer by convection?
    \begin{choices}
        \wrongchoice{You feel the heat from a bonfire even though you are several meters away from it.}
      \correctchoice{Smoke rises up a chimney.}
        \wrongchoice{The bathroom floor feels cold when you stand on it in bare feet.}
        \wrongchoice{The handle of a metal spoon becomes hot when you use it to stir a pot of soup on the stove.}
        \wrongchoice{You can boil water in a microwave oven.}
    \end{choices}
\end{question}
}

\element{jpierce}{
\begin{question}{pt101tb4-Q48}
    Heat transfer by convection cannot occur:
    \begin{choices}
        \wrongchoice{in a pot of soup.}
      \correctchoice{in the vacuum of space.}
        \wrongchoice{in a swimming pool.}
        \wrongchoice{in the air.}
        \wrongchoice{in the ocean.}
    \end{choices}
\end{question}
}

\element{jpierce}{
\begin{question}{pt101tb4-Q49}
    Convection does not work well in most solids because:
    \begin{choices}
        \wrongchoice{the atoms in most solids never collide with each other.}
        \wrongchoice{most solids are hollow inside.}
        \wrongchoice{most solids are opaque.}
        \wrongchoice{most solids melt when they get too hot.}
      \correctchoice{the atoms in most solids are not free to move around.}
    \end{choices}
\end{question}
}

%% Topic: Radiation
\element{jpierce}{
\begin{question}{pt101tb4-Q50}
    Radiation is heat transfer by:
    \begin{choices}
        \wrongchoice{atmospheric currents.}
        \wrongchoice{direct contact.}
        \wrongchoice{bulk fluid motions.}
        \wrongchoice{molecular and electronic collisions.}
      \correctchoice{electromagnetic waves.}
    \end{choices}
\end{question}
}

\element{jpierce}{
\begin{question}{pt101tb4-Q51}
    What causes dew to form on grass overnight?
    \begin{choices}
        \wrongchoice{Heat transfer by convection to the surrounding air cools the grass below the dew point.}
        \wrongchoice{Dew is just raindrops left over from an overnight shower.}
        \wrongchoice{Heat transfer by radiation to the cloudy night sky cools the grass below the dew point.}
        \wrongchoice{Heat transfer by conduction to the cold soil cools the grass below the dew point.}
      \correctchoice{Heat transfer by radiation to the clear night sky cools the grass below the dew point.}
    \end{choices}
\end{question}
}

\element{jpierce}{
\begin{question}{pt101tb4-Q52}
    Which of these is not an example of heat transfer by radiation?
    \begin{choices}
        \wrongchoice{Coffee in a black pot cools faster than coffee in a shiny pot.}
        \wrongchoice{You can boil water in a microwave oven.}
      \correctchoice{You feel the cold bathroom floor with your bare feet.}
        \wrongchoice{You can heat a room using a roaring fire in the fireplace.}
        \wrongchoice{You get sunburned while playing golf.}
    \end{choices}
\end{question}
}

\element{jpierce}{
\begin{question}{pt101tb4-Q53}
    Which of these is an example of heat transfer by radiation?
    \begin{choices}
        \wrongchoice{You feel the cold bathroom floor with your bare feet.}
        \wrongchoice{Smoke rises up a chimney.}
      \correctchoice{You feel the heat from a bonfire even though you are several meters away from it.}
        \wrongchoice{The air near the ceiling is normally warmer than air near the floor.}
        \wrongchoice{The handle of a metal spoon becomes hot when you use it to stir a pot of soup on the stove.}
    \end{choices}
\end{question}
}

\element{jpierce}{
\begin{question}{pt101tb4-Q54}
    Which of these is an example of heat transfer by radiation?
    \begin{choices}
        \wrongchoice{The handle of a metal spoon becomes hot when you use it to stir a pot of soup on the stove.}
        \wrongchoice{You feel the cold bathroom floor with your bare feet.}
      \correctchoice{The sun warms the earth.}
        \wrongchoice{Smoke rises up a chimney.}
        \wrongchoice{The air near the ceiling is normally warmer than air near the floor.}
    \end{choices}
\end{question}
}

\element{jpierce}{
\begin{question}{pt101tb4-Q55}
    Which of these is an example of heat transfer by radiation?
    \begin{choices}
        \wrongchoice{Smoke rises up a chimney.}
        \wrongchoice{The air near the ceiling is normally warmer than air near the floor.}
      \correctchoice{You can boil water in a microwave oven.}
        \wrongchoice{You feel the cold bathroom floor with your bare feet.}
        \wrongchoice{The handle of a metal spoon becomes hot when you use it to stir a pot of soup on the stove.}
    \end{choices}
\end{question}
}

\element{jpierce}{
\begin{question}{pt101tb4-Q56}
    An object will be a good radiator of heat if it is:
    \begin{multicols}{2}
    \begin{choices}
        \wrongchoice{hollow}
        \wrongchoice{dense}
        \wrongchoice{spherical}
        \wrongchoice{shiny}
      \correctchoice{black}
    \end{choices}
    \end{multicols}
\end{question}
}

\element{jpierce}{
\begin{question}{pt101tb4-Q57}
    An object will be a poor radiator of heat if it is:
    \begin{multicols}{2}
    \begin{choices}
        \wrongchoice{dense}
        \wrongchoice{hollow}
        \wrongchoice{spherical}
      \correctchoice{shiny}
        \wrongchoice{black}
    \end{choices}
    \end{multicols}
\end{question}
}

%% Topic: Bow/Shock
\element{jpierce}{
\begin{question}{pt101tb4-Q58}
    The pattern formed by overlapping waves in a bow wave is in the shape of the letter:
    \begin{multicols}{3}
    \begin{choices}
      \correctchoice{V}
        \wrongchoice{B}
        \wrongchoice{U}
        \wrongchoice{I}
        \wrongchoice{T}
    \end{choices}
    \end{multicols}
\end{question}
}

\element{jpierce}{
\begin{question}{pt101tb4-Q59}
    A bow wave is formed when a boat travels across the surface of the water
    \begin{choices}
        \wrongchoice{without getting wet.}
        \wrongchoice{at a speed greater than the speed of sound in water.}
      \correctchoice{at a speed greater than the speed of the water waves.}
        \wrongchoice{at a very low speed.}
        \wrongchoice{at a speed greater than the speed of sound in air.}
    \end{choices}
\end{question}
}

\element{jpierce}{
\begin{question}{pt101tb4-Q60}
    Sonic booms are caused by aircraft
    \begin{choices}
        \wrongchoice{crashing into the ground.}
        \wrongchoice{crashing into each other.}
      \correctchoice{flying faster than the speed of sound.}
        \wrongchoice{flying faster than the legal speed limit.}
        \wrongchoice{flying faster than the speed of light.}
    \end{choices}
\end{question}
}

\element{jpierce}{
\begin{question}{pt101tb4-Q61}
    Sonic booms are caused by \rule[-0.1pt]{4em}{0.1pt} waves.
    \begin{multicols}{2}
    \begin{choices}
        \wrongchoice{standing}
        \wrongchoice{electromagnetic}
      \correctchoice{shock}
        \wrongchoice{transverse}
        \wrongchoice{sine}
    \end{choices}
    \end{multicols}
\end{question}
}

%% Topic: Doppler
\element{jpierce}{
\begin{question}{pt101tb4-Q62}
    The Doppler effect is caused by:
    \begin{choices}
        \wrongchoice{interference of one wave with another wave of the same frequency.}
      \correctchoice{relative motion between the wave source and the observer.}
        \wrongchoice{interference of one wave with another wave of a slightly different frequency.}
        \wrongchoice{standing waves.}
        \wrongchoice{interference of a wave with itself.}
    \end{choices}
\end{question}
}

\element{jpierce}{
\begin{question}{pt101tb4-Q63}
    The change in the frequency of the horn on a train as it
        approaches and then passes the observer is:
    \begin{choices}
      \correctchoice{due to the Doppler effect.}
        \wrongchoice{due to resonance.}
        \wrongchoice{called a sonic boom.}
        \wrongchoice{only heard if the train is supersonic.}
        \wrongchoice{caused by interference.}
    \end{choices}
\end{question}
}

\element{jpierce}{
\begin{question}{pt101tb4-Q64}
    The Doppler effect causes the:
    \begin{choices}
        \wrongchoice{observed pitch of a sound to be higher if the source of sound is moving away from the observer.}
        \wrongchoice{observed pitch of a sound to be lower if the source of sound is approaching the observer.}
        \wrongchoice{speed of sound to increase if the source of sound is approaching the observer.}
      \correctchoice{observed pitch of a sound to be higher if the source of sound is approaching the observer.}
        \wrongchoice{speed of sound to decrease if the source of sound is approaching the observer.}
    \end{choices}
\end{question}
}

\element{jpierce}{
\begin{question}{pt101tb4-Q65}
    The Doppler effect causes the:
    \begin{choices}
        \wrongchoice{speed of sound to decrease if the source of sound is moving away from the observer.}
        \wrongchoice{observed pitch of a sound to be lower if the source of sound is approaching the observer.}
        \wrongchoice{speed of sound to increase if the source of sound is moving away from the observer.}
      \correctchoice{observed pitch of a sound to be lower if the source of sound is moving away from the observer.}
        \wrongchoice{observed pitch of a sound to be higher if the source of sound is moving away from the observer.}
    \end{choices}
\end{question}
}

%% Topic: Interference
\element{jpierce}{
\begin{question}{pt101tb4-Q66}
    Constructive interference is produced by:
    \begin{choices}
        \wrongchoice{water waves, but not by sound waves.}
        \wrongchoice{objects traveling faster than the speed of sound.}
        \wrongchoice{two waves arriving at the same point out of phase with each other.}
      \correctchoice{two waves arriving at the same point in phase with each other.}
        \wrongchoice{sound waves, but not by water waves.}
    \end{choices}
\end{question}
}

\element{jpierce}{
\begin{question}{pt101tb4-Q67}
    Destructive interference is produced by:
    \begin{choices}
        \wrongchoice{two waves arriving at the same point in phase with each other.}
        \wrongchoice{sound waves, but not by water waves.}
        \wrongchoice{water waves, but not by sound waves.}
        \wrongchoice{objects traveling faster than the speed of sound.}
      \correctchoice{two waves arriving at the same point out of phase with each other.}
    \end{choices}
\end{question}
}

%% Topic: Longitudinal/Transverse Waves
\element{jpierce}{
\begin{question}{pt101tb4-Q68}
    In a longitudinal wave, the medium vibrates in a direction that is:
    \begin{choices}
        \wrongchoice{at a \ang{60} angle to the direction the wave travels.}
        \wrongchoice{at a \ang{45} angle to the direction the wave travels.}
        \wrongchoice{perpendicular to the direction the wave travels.}
        \wrongchoice{randomly oriented with respect to the direction the wave travels.}
      \correctchoice{parallel to the direction the wave travels.}
    \end{choices}
\end{question}
}

\element{jpierce}{
\begin{question}{pt101tb4-Q69}
    In a transverse wave, the medium vibrates in a direction that is:
    \begin{choices}
        \wrongchoice{parallel to the direction the wave travels.}
        \wrongchoice{randomly oriented with respect to the direction the wave travels.}
        \wrongchoice{at a \ang{45} angle to the direction the wave travels.}
      \correctchoice{perpendicular to the direction the wave travels.}
        \wrongchoice{at a \ang{60} angle to the direction the wave travels.}
    \end{choices}
\end{question}
}

\element{jpierce}{
\begin{question}{pt101tb4-Q70}
    Examples of longitudinal waves are
    \begin{choices}
        \wrongchoice{light waves and $S$ waves.}
        \wrongchoice{light waves and sound waves.}
        \wrongchoice{sound waves and $S$ waves.}
        \wrongchoice{light waves and $P$ waves.}
      \correctchoice{sound waves and $P$ waves.}
    \end{choices}
\end{question}
}

\element{jpierce}{
\begin{question}{pt101tb4-Q71}
    Examples of transverse waves are
    \begin{choices}
        \wrongchoice{sound waves and $S$ waves.}
        \wrongchoice{light waves and $P$ waves.}
        \wrongchoice{light waves and sound waves.}
      \correctchoice{light waves and $S$ waves.}
        \wrongchoice{sound waves and $P$ waves.}
    \end{choices}
\end{question}
}

\element{jpierce}{
\begin{question}{pt101tb4-Q72}
    In a \rule[-0.1pt]{4em}{0.1pt} wave,
        the medium vibrates in a direction that is parallel to the direction the wave travels.\
    \begin{multicols}{2}
    \begin{choices}
        \wrongchoice{parallel}
        \wrongchoice{light}
      \correctchoice{longitudinal}
        \wrongchoice{water}
        \wrongchoice{transverse}
    \end{choices}
    \end{multicols}
\end{question}
}

\element{jpierce}{
\begin{question}{pt101tb4-Q73}
    In a \rule[-0.1pt]{4em}{0.1pt} wave,
        the medium vibrates in a direction that is perpendicular to the direction the wave travels.
    \begin{multicols}{2}
    \begin{choices}
        \wrongchoice{longitudinal}
        \wrongchoice{sound}
        \wrongchoice{perpendicular}
      \correctchoice{transverse}
        \wrongchoice{normal}
    \end{choices}
    \end{multicols}
\end{question}
}

%% Topic: Pendulum
\element{jpierce}{
\begin{question}{pt101tb4-Q74}
    The period of a pendulum depends on
    \begin{choices}
        \wrongchoice{the mass of the pendulum and the size of the arc it swings through.}
        \wrongchoice{the length of the pendulum and the size of the arc it swings through.}
        \wrongchoice{the mass of the pendulum and the acceleration of gravity.}
        \wrongchoice{the weight of the pendulum and the material it is made from.}
      \correctchoice{the length of the pendulum and the acceleration of gravity.}
    \end{choices}
\end{question}
}

\element{jpierce}{
\begin{question}{pt101tb4-Q75}
    Other things being equal,
        the pendulum that has the longest period will be:
    \begin{choices}
        \wrongchoice{the least massive one.}
        \wrongchoice{the shortest one.}
        \wrongchoice{the most massive one.}
      \correctchoice{the longest one.}
        \wrongchoice{the most spherical one.}
    \end{choices}
\end{question}
}

\element{jpierce}{
\begin{question}{pt101tb4-Q76}
    Other things being equal,
        the pendulum that has the shortest period will be:
    \begin{choices}
        \wrongchoice{the longest one.}
        \wrongchoice{the most massive one.}
      \correctchoice{the shortest one.}
        \wrongchoice{the most spherical one.}
        \wrongchoice{the least massive one.}
    \end{choices}
\end{question}
}

\element{jpierce}{
\begin{question}{pt101tb4-Q77}
    Two pendulums have strings of the same length, but bobs with different masses. 
    Pendulum $A$ has a mass of \SI{200}{\gram}
        while pendulum $B$ has a mass of \SI{400}{\gram}.
    How will their periods of oscillation compare?
    \begin{choices}
        \wrongchoice{The period of pendulum $A$ will be four times as long as the period of pendulum $B$.}
      \correctchoice{The period of pendulum $A$ will be the same as the period of pendulum $B$.}
        \wrongchoice{The period of pendulum $B$ will be four times as long as the period of pendulum $A$.}
        \wrongchoice{The period of pendulum $A$ will be twice as long as the period of pendulum $B$.}
        \wrongchoice{The period of pendulum $B$ will be twice as long as the period of pendulum $A$.}
    \end{choices}
\end{question}
}

\element{jpierce}{
\begin{question}{pt101tb4-Q78}
    Two pendulums have strings of the same length, but bobs with different masses. 
    Pendulum $A$ has a mass of \SI{200}{\gram}
        while pendulum $B$ has a mass of \SI{100}{\gram}. 
    How will their periods of oscillation compare?
    \begin{choices}
        \wrongchoice{The period of pendulum $A$ will be twice as long as the period of pendulum $B$.}
        \wrongchoice{The period of pendulum $B$ will be twice as long as the period of pendulum $A$.}
        \wrongchoice{The period of pendulum $A$ will be four times as long as the period of pendulum $B$.}
      \correctchoice{The period of pendulum $A$ will be the same as the period of pendulum $B$.}
        \wrongchoice{The period of pendulum $B$ will be four times as long as the period of pendulum $A$.}
    \end{choices}
\end{question}
}

\element{jpierce}{
\begin{question}{pt101tb4-Q79}
    When you swing standing up in a playground swing,
        the period of your oscillation is about 2 seconds. 
    If you then swing sitting down,
        the period of this new oscillation will be:
    \begin{choices}
        \wrongchoice{2 seconds, because the swing is the same length.}
      \correctchoice{more than 2 seconds because your sitting has effectively lengthened the pendulum.}
        \wrongchoice{less than 2 seconds because your sitting has effectively shortened the pendulum.}
        \wrongchoice{less than 2 seconds because your sitting has effectively lengthened the pendulum.}
        \wrongchoice{2 seconds, because your mass is still the same.}
    \end{choices}
\end{question}
}

\element{jpierce}{
\begin{question}{pt101tb4-Q80}
    When you swing sitting down in a playground swing,
        the period of your oscillation is about 2 seconds. 
    If you then swing standing up,
        the period of this new oscillation will be:
    \begin{choices}
        \wrongchoice{2 seconds, because the swing is the same length.}
      \correctchoice{less than 2 seconds because your standing has effectively shortened the pendulum.}
        \wrongchoice{more than 2 seconds because your standing has effectively shortened the pendulum.}
        \wrongchoice{more than 2 seconds because your standing has effectively lengthened the pendulum.}
        \wrongchoice{2 seconds, because your mass is still the same.}
    \end{choices}
\end{question}
}

%% topic: Standing Waves
\element{jpierce}{
\begin{question}{pt101tb4-Q81}
    Nodes in a standing wave on a plucked guitar string are:
    \begin{choices}
        \wrongchoice{points where the string was plucked.}
        \wrongchoice{points of compression.}
      \correctchoice{points where the displacement of the string is a minimum.}
        \wrongchoice{points of rarefaction.}
        \wrongchoice{points where the displacement of the string is a maximum.}
    \end{choices}
\end{question}
}

\element{jpierce}{
\begin{question}{pt101tb4-Q82}
    Standing waves are produced by:
    \begin{choices}
      \correctchoice{waves that reflect off a boundary and interfere with themselves.}
        \wrongchoice{the motion of the source away from the observer.}
        \wrongchoice{waves that vibrate in a vertical plane (standing upright).}
        \wrongchoice{the motion of the source toward the observer.}
        \wrongchoice{objects that travel faster than the speed of sound.}
    \end{choices}
\end{question}
}

\element{jpierce}{
\begin{question}{pt101tb4-Q83}
    \rule[-0.1pt]{4em}{0.1pt} are produced by waves that reflect off
        a boundary and interfere with themselves.
    \begin{choices}
        \wrongchoice{Transverse waves}
        \wrongchoice{Bow waves}
        \wrongchoice{Echoes}
        \wrongchoice{Shock waves}
      \correctchoice{Standing waves}
    \end{choices}
\end{question}
}

%% Topic: Waves
\element{jpierce}{
\begin{question}{pt101tb4-Q84}
    The hertz (\si{\hertz}) is a unit of:
    \begin{multicols}{2}
    \begin{choices}
        \wrongchoice{speed.}
        \wrongchoice{wavelength.}
        \wrongchoice{amplitude.}
      \correctchoice{frequency.}
        \wrongchoice{time.}
    \end{choices}
    \end{multicols}
\end{question}
}

\element{jpierce}{
\begin{question}{pt101tb4-Q85}
    The distance from the top of one wave crest to the next is called the
    \begin{multicols}{2}
    \begin{choices}
        \wrongchoice{speed.}
        \wrongchoice{amplitude.}
        \wrongchoice{period.}
        \wrongchoice{frequency.}
      \correctchoice{wavelength.}
    \end{choices}
    \end{multicols}
\end{question}
}

\element{jpierce}{
\begin{question}{pt101tb4-Q86}
    A wave that has a relatively long wavelength will also have a relatively
    \begin{choices}
        \wrongchoice{small amplitude.}
        \wrongchoice{large amplitude.}
        \wrongchoice{short period.}
      \correctchoice{low frequency.}
        \wrongchoice{high speed.}
    \end{choices}
\end{question}
}

\element{jpierce}{
\begin{question}{pt101tb4-Q87}
    A wave that has a relatively long wavelength will also have a relatively
    \begin{choices}
      \correctchoice{long period.}
        \wrongchoice{small amplitude.}
        \wrongchoice{high speed.}
        \wrongchoice{large amplitude.}
        \wrongchoice{high frequency.}
    \end{choices}
\end{question}
}

\element{jpierce}{
\begin{question}{pt101tb4-Q88}
    A wave that has a relatively short wavelength will also have a relatively
    \begin{choices}
        \wrongchoice{large amplitude.}
        \wrongchoice{long period.}
        \wrongchoice{small amplitude.}
      \correctchoice{high frequency.}
        \wrongchoice{low speed.}
    \end{choices}
\end{question}
}

\element{jpierce}{
\begin{question}{pt101tb4-Q89}
    A wave that has a relatively short wavelength will also have a relatively
    \begin{choices}
        \wrongchoice{low frequency.}
        \wrongchoice{large amplitude.}
      \correctchoice{short period.}
        \wrongchoice{small amplitude.}
        \wrongchoice{low speed.}
    \end{choices}
\end{question}
}

\element{jpierce}{
\begin{question}{pt101tb4-Q90}
    The speed of a wave is equal to the product of its
    \begin{choices}
        \wrongchoice{amplitude and frequency.}
        \wrongchoice{wavelength and period.}
      \correctchoice{frequency and wavelength.}
        \wrongchoice{period and amplitude.}
        \wrongchoice{period and frequency.}
    \end{choices}
\end{question}
}

\element{jpierce}{
\begin{question}{pt101tb4-Q91}
    The period of a wave is equal to 1 divided by the \rule[-0.1pt]{4em}{0.1pt} of the wave.
    \begin{multicols}{2}
    \begin{choices}
        \wrongchoice{speed}
        \wrongchoice{diameter}
        \wrongchoice{wavelength}
        \wrongchoice{amplitude}
      \correctchoice{frequency}
    \end{choices}
    \end{multicols}
\end{question}
}

\element{jpierce}{
\begin{question}{pt101tb4-Q92}
    The speed of a wave is equal to the \rule[-0.1pt]{4em}{0.1pt} divided
        by the \rule[-0.1pt]{4em}{0.1pt}.
    \begin{choices}
        \wrongchoice{frequency; wavelength}
        \wrongchoice{amplitude; frequency}
        \wrongchoice{wavelength; frequency}
        \wrongchoice{period; amplitude}
      \correctchoice{wavelength; period}
    \end{choices}
\end{question}
}

\element{jpierce}{
\begin{question}{pt101tb4-Q93}
    A train of freight cars, each \SI{10}{\meter} long,
        rolls by at the rate of 2 cars each second. 
    What is the speed of the train?
    \begin{multicols}{2}
    \begin{choices}
      \correctchoice{\SI{20}{\meter\per\second}}
        \wrongchoice{\SI{12}{\meter\per\second}}
        \wrongchoice{\SI{2}{\meter\per\second}}
        \wrongchoice{\SI{5}{\meter\per\second}}
        \wrongchoice{\SI{10}{\meter\per\second}}
    \end{choices}
    \end{multicols}
\end{question}
}

\element{jpierce}{
\begin{question}{pt101tb4-Q94}
    A train of freight cars, each \SI{10}{\meter} long,
        rolls by at the rate of 1 car each second. 
    What is the speed of the train?
    \begin{multicols}{2}
    \begin{choices}
        \wrongchoice{\SI{11}{\meter\per\second}}
        \wrongchoice{\SI{5}{\meter\per\second}}
      \correctchoice{\SI{10}{\meter\per\second}}
        \wrongchoice{\SI{1}{\meter\per\second}}
        \wrongchoice{\SI{20}{\meter\per\second}}
    \end{choices}
    \end{multicols}
\end{question}
}

\element{jpierce}{
\begin{question}{pt101tb4-Q95}
    A train of freight cars, each \SI{12}{\meter} long,
        rolls by at the rate of 3 cars each second. 
    What is the speed of the train?
    \begin{multicols}{2}
    \begin{choices}
      \correctchoice{\SI{36}{\meter\per\second}}
        \wrongchoice{\SI{15}{\meter\per\second}}
        \wrongchoice{\SI{4}{\meter\per\second}}
        \wrongchoice{\SI{12}{\meter\per\second}}
        \wrongchoice{\SI{3}{\meter\per\second}}
    \end{choices}
    \end{multicols}
\end{question}
}

\element{jpierce}{
\begin{question}{pt101tb4-Q96}
    A wave with a speed of \SI{6}{\meter\per\second} and a
        wavelength of \SI{3}{\meter} would have a frequency of:
    \begin{multicols}{2}
    \begin{choices}
      \correctchoice{\SI{2}{\hertz}}
        \wrongchoice{\SI{6}{\hertz}}
        \wrongchoice{\SI{3}{\hertz}}
        \wrongchoice{\SI{9}{\hertz}}
        \wrongchoice{\SI{18}{\hertz}}
    \end{choices}
    \end{multicols}
\end{question}
}

\element{jpierce}{
\begin{question}{pt101tb4-Q97}
    A wave with a speed of \SI{6}{\meter\per\second} and a
        frequency of \SI{3}{\hertz} would have a wavelength of:
    \begin{multicols}{2}
    \begin{choices}
        \wrongchoice{\SI{9}{\meter}}
        \wrongchoice{\SI{6}{\meter}}
      \correctchoice{\SI{2}{\meter}}
        \wrongchoice{\SI{18}{\meter}}
        \wrongchoice{\SI{3}{\meter}}
    \end{choices}
    \end{multicols}
\end{question}
}

\element{jpierce}{
\begin{question}{pt101tb4-Q98}
    A wave with a speed of \SI{6}{\meter\per\second} and a
        frequency of \SI{2}{\hertz} would have a wavelength of:
    \begin{multicols}{2}
    \begin{choices}
        \wrongchoice{\SI{2}{\meter}}
        \wrongchoice{\SI{6}{\meter}}
      \correctchoice{\SI{3}{\meter}}
        \wrongchoice{\SI{12}{\meter}}
        \wrongchoice{\SI{9}{\meter}}
    \end{choices}
    \end{multicols}
\end{question}
}

\element{jpierce}{
\begin{question}{pt101tb4-Q99}
    A wave with a speed of \SI{6}{\meter\per\second} and a
        wavelength of \SI{2}{\meter} would have a frequency of:
    \begin{multicols}{2}
    \begin{choices}
        \wrongchoice{\SI{12}{\hertz}}
        \wrongchoice{\SI{8}{\hertz}}
      \correctchoice{\SI{3}{\hertz}}
        \wrongchoice{\SI{6}{\hertz}}
        \wrongchoice{\SI{2}{\hertz}}
    \end{choices}
    \end{multicols}
\end{question}
}

\element{jpierce}{
\begin{question}{pt101tb4-Q100}
    A wave with a wavelength of \SI{6}{\meter} and a
        frequency of \SI{2}{\hertz} would have a speed of:
    \begin{multicols}{2}
    \begin{choices}
        \wrongchoice{\SI{18}{\meter\per\second}}
        \wrongchoice{\SI{3}{\meter\per\second}}
      \correctchoice{\SI{12}{\meter\per\second}}
        \wrongchoice{\SI{2}{\meter\per\second}}
        \wrongchoice{\SI{6}{\meter\per\second}}
    \end{choices}
    \end{multicols}
\end{question}
}

\element{jpierce}{
\begin{question}{pt101tb4-Q101}
    A wave with a wavelength of \SI{6}{\meter} and a
        frequency of \SI{3}{\hertz} would have a speed of:
    \begin{multicols}{2}
    \begin{choices}
        \wrongchoice{\SI{2}{\meter\per\second}}
        \wrongchoice{\SI{12}{\meter\per\second}}
      \correctchoice{\SI{18}{\meter\per\second}}
        \wrongchoice{\SI{6}{\meter\per\second}}
        \wrongchoice{\SI{3}{\meter\per\second}}
    \end{choices}
    \end{multicols}
\end{question}
}

%% Topic: Beats
\element{jpierce}{
\begin{question}{pt101tb4-Q102}
    Beats occur in sound waves:
    \begin{choices}
        \wrongchoice{when two sources vibrate at exactly the same frequency.}
      \correctchoice{when waves from two sources of slightly different frequencies are heard together.}
        \wrongchoice{when waves from two sources are exactly in phase.}
        \wrongchoice{when waves from two sources are exactly out of phase.}
        \wrongchoice{when an echo is heard at the same time as the original sound.}
    \end{choices}
\end{question}
}

\element{jpierce}{
\begin{question}{pt101tb4-Q103}
    The beat frequency is:
    \begin{choices}
        \wrongchoice{the sum of the frequencies of two different sound waves.}
        \wrongchoice{the quotient of the frequencies of two different sound waves.}
        \wrongchoice{the product of the frequencies of two different sound waves.}
      \correctchoice{the difference between the frequencies of two different sound waves.}
        \wrongchoice{the average of the frequencies of two different sound waves.}
    \end{choices}
\end{question}
}

\element{jpierce}{
\begin{question}{pt101tb4-Q104}
    When two vibrating objects are perfectly in tune with each other,
        the beat frequency should be:
    \begin{choices}
        \wrongchoice{\num{1} beats per second.}
      \correctchoice{\num{0} beats per second.}
        \wrongchoice{\num{2} beats per second.}
        \wrongchoice{equal to the frequency of vibration of either object.}
        \wrongchoice{\num{5} beats per second.}
    \end{choices}
\end{question}
}

\element{jpierce}{
\begin{question}{pt101tb4-Q105}
    Sound waves with frequencies of \SI{250}{\hertz} and \SI{256}{\hertz}
        would combine to produce a beat frequency of:
    \begin{multicols}{2}
    \begin{choices}
        \wrongchoice{\SI{253}{\hertz}}
        \wrongchoice{\SI{506}{\hertz}}
      \correctchoice{\SI{6}{\hertz}}
        \wrongchoice{\SI{64,000}{\hertz}}
        \wrongchoice{\SI{1024}{\hertz}}
    \end{choices}
    \end{multicols}
\end{question}
}

\element{jpierce}{
\begin{question}{pt101tb4-Q106}
    A vibrating string is being tuned to match a tuning fork
        with a frequency of \SI{256}{\hertz}.
    When \num{3} beats per second are heard,
        the vibration frequency of the string must be:
    \begin{choices}
        \wrongchoice{\SI{259}{\hertz}.}
        \wrongchoice{\SI{256}{\hertz}.}
        \wrongchoice{\SI{3}{\hertz}.}
        \wrongchoice{\SI{253}{\hertz}.}
      \correctchoice{either \SI{253}{\hertz} or \SI{259}{\hertz}.}
    \end{choices}
\end{question}
}

\element{jpierce}{
\begin{question}{pt101tb4-Q107}
    A vibrating string is being tuned to match a tuning
        fork with a frequency of \SI{384}{\hertz}.
    When \num{2} beats per second are heard,
        the vibration frequency of the string must be:
    \begin{choices}
      \correctchoice{either \SI{382}{\hertz} or \SI{386}{\hertz}.}
        \wrongchoice{\SI{2}{\hertz}.}
        \wrongchoice{\SI{382}{\hertz}.}
        \wrongchoice{\SI{384}{\hertz}.}
        \wrongchoice{\SI{386}{\hertz}.}
    \end{choices}
\end{question}
}

\element{jpierce}{
\begin{question}{pt101tb4-Q108}
    Sound waves with frequencies of \SI{500}{\hertz} and \SI{504}{\hertz}
        would combine to produce a beat frequency of:
    \begin{multicols}{2}
    \begin{choices}
      \correctchoice{\SI{4}{\hertz}}
        \wrongchoice{\SI{1.008}{\hertz}}
        \wrongchoice{\SI{502}{\hertz}}
        \wrongchoice{\SI{252 000}{\hertz}}
        \wrongchoice{\SI{1004}{\hertz}}
    \end{choices}
    \end{multicols}
\end{question}
}

\element{jpierce}{
\begin{question}{pt101tb4-Q109}
    An echo is caused by:
    \begin{choices}
      \correctchoice{reflection of sound waves.}
        \wrongchoice{resonance of sound waves.}
        \wrongchoice{refraction of sound waves.}
        \wrongchoice{changes in the speed of sound with temperature.}
        \wrongchoice{interference of sound waves with each other.}
    \end{choices}
\end{question}
}

\element{jpierce}{
\begin{question}{pt101tb4-Q110}
    Reflection of sound waves produces the phenomenon known as:
    \begin{choices}
      \correctchoice{an echo.}
        \wrongchoice{a sonic boom.}
        \wrongchoice{beats.}
        \wrongchoice{the Doppler effect.}
        \wrongchoice{a bow wave.}
    \end{choices}
\end{question}
}

%% Topic: Interference
\element{jpierce}{
\begin{question}{pt101tb4-Q111}
    Constructive interference of sound waves occurs:
    \begin{choices}
        \wrongchoice{when two waves arrive at the same point out of phase with each other.}
        \wrongchoice{whenever there is an echo.}
      \correctchoice{when two waves arrive at the same point in phase with each other.}
        \wrongchoice{whenever sound waves are reflected off distant buildings.}
        \wrongchoice{whenever sound waves are refracted by air layers of different temperatures.}
    \end{choices}
\end{question}
}

\element{jpierce}{
\begin{question}{pt101tb4-Q112}
    Destructive interference of sound waves occurs:
    \begin{choices}
      \correctchoice{when two waves arrive at the same point out of phase with each other.}
        \wrongchoice{whenever sound waves are refracted by air layers of different temperatures.}
        \wrongchoice{whenever sound waves are reflected off distant buildings.}
        \wrongchoice{when two waves arrive at the same point in phase with each other.}
        \wrongchoice{whenever there is an echo.}
    \end{choices}
\end{question}
}

\element{jpierce}{
\begin{question}{pt101tb4-Q113}
    Which of these is caused by interference?
    \begin{multicols}{2}
    \begin{choices}
        \wrongchoice{echo}
        \wrongchoice{shock wave}
        \wrongchoice{sonic boom}
      \correctchoice{standing wave}
        \wrongchoice{resonance}
    \end{choices}
    \end{multicols}
\end{question}
}

%% Topic: Limits of hearing
\element{jpierce}{
\begin{question}{pt101tb4-Q114}
    Sound waves with frequencies below \SI{20}{\hertz} are called:
    \begin{multicols}{2}
    \begin{choices}
      \correctchoice{infrasonic}
        \wrongchoice{hypersonic}
        \wrongchoice{ultrasonic}
        \wrongchoice{subsonic}
        \wrongchoice{supersonic}
    \end{choices}
    \end{multicols}
\end{question}
}

\element{jpierce}{
\begin{question}{pt101tb4-Q115}
    Sound waves with frequencies above \SI{20 000}{\hertz} are called:
    \begin{multicols}{2}
    \begin{choices}
        \wrongchoice{supersonic}
        \wrongchoice{hyposonic}
      \correctchoice{ultrasonic}
        \wrongchoice{subsonic}
        \wrongchoice{infrasonic}
    \end{choices}
    \end{multicols}
\end{question}
}

\element{jpierce}{
\begin{question}{pt101tb4-Q116}
    The ear of a young person is normally sensitive to pitches corresponding
        to the range of frequencies between about:
    \begin{choices}
      \correctchoice{\SI{20}{\hertz} and \SI{20 000}{\hertz}}
        \wrongchoice{\SI{2}{\hertz} and \SI{2 000}{\hertz}}
        \wrongchoice{\SI{2 000}{\hertz} and \SI{2 000 000}{\hertz}}
        \wrongchoice{\SI{1}{\hertz} and \SI{100}{\hertz}}
        \wrongchoice{\SI{200}{\hertz} and \SI{200,000}{\hertz}}
    \end{choices}
\end{question}
}

\element{jpierce}{
\begin{question}{pt101tb4-Q117}
    The ear of a young person is normally sensitive to pitches corresponding
        to the range of frequencies between about:
    \begin{choices}
        \wrongchoice{\SI{3 000}{\hertz} and \SI{3 000 000}{\hertz}}
      \correctchoice{\SI{20}{\hertz} and \SI{20 000}{\hertz}}
        \wrongchoice{\SI{600}{\hertz} and \SI{600 000}{\hertz}}
        \wrongchoice{\SI{0.1}{\hertz} and \SI{100}{\hertz}}
        \wrongchoice{\SI{5}{\hertz} and \SI{5 000}{\hertz}}
    \end{choices}
\end{question}
}

\element{jpierce}{
\begin{question}{pt101tb4-Q118}
    Which of the following is true concerning the range of sound waves in air?
    \begin{choices}
        \wrongchoice{The range of sound waves in air is limited only by the curvature of the Earth.}
        \wrongchoice{High-frequency sound waves travel farther than low-frequency sound waves.}
        \wrongchoice{Sound waves of different frequencies all have the same range in air.}
        \wrongchoice{The range of sound waves in air is unlimited.}
      \correctchoice{Low-frequency sound waves travel farther than high-frequency sound waves.}
    \end{choices}
\end{question}
}

\element{jpierce}{
\begin{question}{pt101tb4-Q119}
    Foghorns on ships have \rule[-0.1pt]{4em}{0.1pt} frequencies because \rule[-0.1pt]{4em}{0.1pt}.
    \begin{choices}
        \wrongchoice{low; these frequencies are easier to produce on a ship.}
        \wrongchoice{high; these frequencies are easier to produce on a ship.}
        \wrongchoice{low; ships' captains generally cannot hear high frequencies anymore.}
      \correctchoice{low; these frequencies travel farther in air.}
        \wrongchoice{high; these frequencies travel farther in air.}
    \end{choices}
\end{question}
}

%% Topic: Resonance
\element{jpierce}{
\begin{question}{pt101tb4-Q120}
    Pushing a person on a swing at the same rate as the natural
        frequency of the swing/pendulum is an example of
    \begin{choices}
        \wrongchoice{the Doppler effect.}
        \wrongchoice{constructive interference.}
      \correctchoice{resonance.}
        \wrongchoice{destructive interference.}
        \wrongchoice{refraction.}
    \end{choices}
\end{question}
}

\element{jpierce}{
\begin{question}{pt101tb4-Q121}
    Tuning your radio to make its electronics oscillate at the same
        frequency as the waves from your favorite radio station is an example of
    \begin{choices}
      \correctchoice{resonance.}
        \wrongchoice{refraction.}
        \wrongchoice{the Doppler effect.}
        \wrongchoice{constructive interference.}
        \wrongchoice{destructive interference.}
    \end{choices}
\end{question}
}

\element{jpierce}{
\begin{question}{pt101tb4-Q122}
    A trombone player vibrates his lips at a particular frequency,
        causing the air column inside the trombone to vibrate at the same frequency. 
    This is an example of
    \begin{choices}
        \wrongchoice{refraction.}
        \wrongchoice{destructive interference.}
      \correctchoice{resonance.}
        \wrongchoice{the Doppler effect.}
        \wrongchoice{constructive interference.}
    \end{choices}
\end{question}
}

\element{jpierce}{
\begin{question}{pt101tb4-Q123}
    Resonance occurs when
    \begin{choices}
      \correctchoice{the frequency of forced vibrations matches an object's natural frequency.}
        \wrongchoice{sound waves bounce off a distant surface and return to the source.}
        \wrongchoice{sound travels more rapidly through air layers with different temperatures.}
        \wrongchoice{an object moves through air faster than the speed of sound.}
        \wrongchoice{several waves arrive at the same point out of phase with each other.}
    \end{choices}
\end{question}
}

%% Topic: Speed of sound
\element{jpierce}{
\begin{question}{pt101tb4-Q124}
    In general, sound travels most rapidly in \rule[-0.1pt]{4em}{0.1pt},
        less rapidly in \rule[-0.1pt]{4em}{0.1pt},
        and even less rapidly in \rule[-0.1pt]{4em}{0.1pt}.
    \begin{choices}
        \wrongchoice{solids; gases; liquids}
      \correctchoice{solids; liquids; gases}
        \wrongchoice{gases; liquids; solids}
        \wrongchoice{liquids; gases; solids}
        \wrongchoice{liquids; solids; gases}
    \end{choices}
\end{question}
}

\element{jpierce}{
\begin{question}{pt101tb4-Q125}
    Compared to \rule[-0.1pt]{4em}{0.1pt},
        sound travels about 4 times faster in \rule[-0.1pt]{4em}{0.1pt},
        and about 15 times faster in \rule[-0.1pt]{4em}{0.1pt}.
    \begin{choices}
        \wrongchoice{water; steel; air}
        \wrongchoice{water; air; steel}
        \wrongchoice{air; steel; water}
        \wrongchoice{steel; water; air}
      \correctchoice{air; water; steel}
    \end{choices}
\end{question}
}

\element{jpierce}{
\begin{question}{pt101tb4-Q126}
    Sound travels
    \begin{choices}
        \wrongchoice{at the same speed in all materials.}
        \wrongchoice{faster in cold air than in warm air.}
        \wrongchoice{faster in in a vacuum than in air.}
      \correctchoice{faster in warm air than in cold air.}
        \wrongchoice{at the same speed in air of all temperatures.}
    \end{choices}
\end{question}
}

\element{jpierce}{
\begin{question}{pt101tb4-Q127}
    Sound travels faster in air at
    \begin{choices}
        \wrongchoice{lower temperatures because the molecules are closer together and collide more frequently.}
        \wrongchoice{lower temperatures because the molecules move faster and collide more frequently.}
        \wrongchoice{higher temperatures because the molecules are closer together and collide more frequently.}
        \wrongchoice{lower temperatures because the air is more solid then.}
      \correctchoice{higher temperatures because the molecules move faster and collide more frequently.}
    \end{choices}
\end{question}
}

\element{jpierce}{
\begin{question}{pt101tb4-Q128}
    The speed of sound in air at room temperature is about:
    \begin{choices}
        \wrongchoice{\SI{110}{\meter\per\second}}
        \wrongchoice{\SI{1100}{\meter\per\second}}
      \correctchoice{\SI{340}{\meter\per\second}}
        \wrongchoice{\SI{34 000}{\meter\per\second}}
        \wrongchoice{\SI{300 000}{\kilo\meter\per\second}}
    \end{choices}
\end{question}
}

\element{jpierce}{
\begin{question}{pt101tb4-Q129}
    In air, sound takes about \rule[-0.1pt]{4em}{0.1pt} to
        travel the length of a football stadium.
    \begin{multicols}{2}
    \begin{choices}
        \wrongchoice{\SI{30}{\second}}
        \wrongchoice{\SI{3}{\second}}
        \wrongchoice{\SI{3}{\minute}}
        \wrongchoice{\SI{1/30}{\second}}
      \correctchoice{\SI{1/3}{\second}}
    \end{choices}
    \end{multicols}
\end{question}
}

%% Topic: Waves
\element{jpierce}{
\begin{question}{pt101tb4-Q130}
    Compared to a \SI{200}{\hertz} sound, a \SI{400}{\hertz} sound would have
    \begin{choices}
      \correctchoice{a shorter wavelength and the same speed.}
        \wrongchoice{a longer wavelength and a lower speed.}
        \wrongchoice{a shorter wavelength and a lower speed.}
        \wrongchoice{a shorter wavelength and a higher speed.}
        \wrongchoice{a longer wavelength and a higher speed.}
    \end{choices}
\end{question}
}

\element{jpierce}{
\begin{question}{pt101tb4-Q131}
    Compared to a \SI{300}{\hertz} sound,
        a \SI{500}{\hertz} sound would have
    \begin{choices}
      \correctchoice{a shorter wavelength and the same speed.}
        \wrongchoice{a longer wavelength and a higher speed.}
        \wrongchoice{a shorter wavelength and a higher speed.}
        \wrongchoice{a shorter wavelength and a lower speed.}
        \wrongchoice{a longer wavelength and the same speed.}
    \end{choices}
\end{question}
}

\element{jpierce}{
\begin{question}{pt101tb4-Q132}
    Compared to a \SI{400}{\hertz} sound,
        a \SI{200}{\hertz} sound would have
    \begin{choices}
        \wrongchoice{a shorter wavelength and a lower speed.}
        \wrongchoice{a longer wavelength and a higher speed.}
        \wrongchoice{a shorter wavelength and the same speed.}
        \wrongchoice{a longer wavelength and a lower speed.}
      \correctchoice{a longer wavelength and the same speed.}
    \end{choices}
\end{question}
}

\element{jpierce}{
\begin{question}{pt101tb4-Q133}
    Compared to a \SI{500}{\hertz} sound,
        a \SI{300}{\hertz} sound would have
    \begin{choices}
        \wrongchoice{a shorter wavelength and a lower speed.}
        \wrongchoice{a longer wavelength and a lower speed.}
        \wrongchoice{a longer wavelength and a higher speed.}
      \correctchoice{a longer wavelength and the same speed.}
        \wrongchoice{a shorter wavelength and the same speed.}
    \end{choices}
\end{question}
}

%% Topic: Instruments
\element{jpierce}{
\begin{question}{pt101tb4-Q134}
    The musical sound produced by a brass instrument such as a trombone is caused by:
    \begin{choices}
        \wrongchoice{a vibrating air column in the player's throat, which is amplified by the instrument's metal tubing.}
      \correctchoice{standing sound waves in the air inside the tube, which are excited by the player's vibrating lips.}
        \wrongchoice{air rushing through the tube from the open end to the mouthpiece.}
        \wrongchoice{the metal tubing of the instrument vibrating at the resonant frequency of the player's lips.}
        \wrongchoice{air rushing through the tube from the mouthpiece to the open end.}
    \end{choices}
\end{question}
}

%% Topic: Intensity/Loud
\element{jpierce}{
\begin{question}{pt101tb4-Q135}
    The intensity of a sound wave depends on:
    \begin{choices}
        \wrongchoice{the number of waves that pass by every second.}
        \wrongchoice{the speed of the wave.}
      \correctchoice{the amplitude of the wave.}
        \wrongchoice{the wavelength of the wave}
        \wrongchoice{the frequency of the wave.}
    \end{choices}
\end{question}
}

\element{jpierce}{
\begin{question}{pt101tb4-Q136}
    Intensity of sound waves is measured in:
    \begin{choices}
      \correctchoice{decibels (\si{\decibel})}
        \wrongchoice{newtons (\si{\newton})}
        \wrongchoice{meters per second (\si{\meter\per\second})}
        \wrongchoice{meters (\si{\meter})}
        \wrongchoice{hertz (\si{\hertz})}
    \end{choices}
\end{question}
}

\element{jpierce}{
\begin{question}{pt101tb4-Q137}
    The physiological sensation directly related 
        to the \rule[-0.1pt]{4em}{0.1pt} of
        a sound is called \rule[-0.1pt]{4em}{0.1pt}.
    \begin{choices}
      \correctchoice{intensity; loudness}
        \wrongchoice{loudness; amplitude}
        \wrongchoice{speed; pitch}
        \wrongchoice{pitch; frequency}
        \wrongchoice{frequency; intensity}
    \end{choices}
\end{question}
}

\element{jpierce}{
\begin{question}{pt101tb4-Q138}
    The threshold of hearing is set at a sound level of:
    \begin{multicols}{2}
    \begin{choices}
        \wrongchoice{\SI{100}{\decibel}}
        \wrongchoice{\SI{2000}{\decibel}}
      \correctchoice{\SI{0}{\decibel}}
        \wrongchoice{\SI{10}{\decibel}}
        \wrongchoice{\SI{-10}{\decibel}}
    \end{choices}
    \end{multicols}
\end{question}
}

\element{jpierce}{
\begin{question}{pt101tb4-Q139}
    An intensity of 50 decibels is \rule[-0.1pt]{4em}{0.1pt} times
        as intense as an intensity of 30 decibels.
    \begin{multicols}{2}
    \begin{choices}
        \wrongchoice{\num{50}}
      \correctchoice{\num{100}}
        \wrongchoice{\num{1.67}}
        \wrongchoice{\num{30}}
        \wrongchoice{\num{20}}
    \end{choices}
    \end{multicols}
\end{question}
}

\element{jpierce}{
\begin{question}{pt101tb4-Q140}
    An intensity of 60 decibels is \rule[-0.1pt]{4em}{0.1pt} times
        as intense as an intensity of 30 decibels.
    \begin{multicols}{2}
    \begin{choices}
        \wrongchoice{\num{30}}
        \wrongchoice{\num{90}}
      \correctchoice{\num{1000}}
        \wrongchoice{\num{2}}
        \wrongchoice{\num{60}}
    \end{choices}
    \end{multicols}
\end{question}
}

\element{jpierce}{
\begin{question}{pt101tb4-Q141}
    An intensity of 60 decibels is \rule[-0.1pt]{4em}{0.1pt} times
        as intense as an intensity of 40 decibels.
    \begin{multicols}{2}
    \begin{choices}
      \correctchoice{\num{100}}
        \wrongchoice{\num{2400}}
        \wrongchoice{\num{20}}
        \wrongchoice{\num{60}}
        \wrongchoice{\num{1.5}}
    \end{choices}
    \end{multicols}
\end{question}
}

\element{jpierce}{
\begin{question}{pt101tb4-Q142}
    40 decibels represents sound intensity \rule[-0.1pt]{4em}{0.1pt} times
        greater than the threshold of hearing.
    \begin{multicols}{2}
    \begin{choices}
        \wrongchoice{\num{40}}
        \wrongchoice{\num{1600}}
        \wrongchoice{\num{10}}
        \wrongchoice{\num{1 000 000}}
      \correctchoice{\num{10 000}}
    \end{choices}
    \end{multicols}
\end{question}
}

%% Topic: Pitch
\element{jpierce}{
\begin{question}{pt101tb4-Q143}
    The pitch of a musical tone relates directly to the \rule[-0.1pt]{4em}{0.1pt} of the sound wave.
    \begin{multicols}{2}
    \begin{choices}
      \correctchoice{frequency}
        \wrongchoice{loudness}
        \wrongchoice{amplitude}
        \wrongchoice{intensity}
        \wrongchoice{speed}
    \end{choices}
    \end{multicols}
\end{question}
}

\element{jpierce}{
\begin{question}{pt101tb4-Q144}
    Sound waves with higher pitch will have:
    \begin{multicols}{2}
    \begin{choices}
        \wrongchoice{longer wavelengths.}
        \wrongchoice{higher speeds.}
        \wrongchoice{lower frequencies.}
        \wrongchoice{lower speeds.}
      \correctchoice{higher frequencies.}
    \end{choices}
    \end{multicols}
\end{question}
}

\element{jpierce}{
\begin{question}{pt101tb4-Q145}
    The ``highness'' or ``lowness'' of a musical tone is called the:
    \begin{multicols}{2}
    \begin{choices}
        \wrongchoice{loudness}
        \wrongchoice{rhythm}
        \wrongchoice{intensity}
        \wrongchoice{scale}
      \correctchoice{pitch}
    \end{choices}
    \end{multicols}
\end{question}
}

\element{jpierce}{
\begin{question}{pt101tb4-Q146}
    For a musical tone composed of several partial tones, the pitch refers to:
    \begin{choices}
      \correctchoice{the lowest frequency involved.}
        \wrongchoice{the frequency of the partial tone with the lowest intensity.}
        \wrongchoice{the average of the various frequencies involved.}
        \wrongchoice{the frequency of the partial tone with the highest intensity.}
        \wrongchoice{the highest frequency involved.}
    \end{choices}
\end{question}
}

\element{jpierce}{
\begin{question}{pt101tb4-Q147}
    Decreasing the length of a vibrating column of air or a
        vibrating string will generally result in:
    \begin{choices}
        \wrongchoice{a sound with a longer wavelength.}
        \wrongchoice{a sound with a lower pitch.}
        \wrongchoice{a sound wave traveling at a lower speed.}
        \wrongchoice{a sound wave traveling at a higher speed.}
      \correctchoice{a sound with a higher pitch.}
    \end{choices}
\end{question}
}

\element{jpierce}{
\begin{question}{pt101tb4-Q148}
    Increasing the length of a vibrating column of air or a vibrating string will generally result in:
    \begin{choices}
      \correctchoice{a sound with a lower pitch.}
        \wrongchoice{a sound with a higher pitch.}
        \wrongchoice{a sound with a shorter wavelength.}
        \wrongchoice{a sound wave traveling at a higher speed.}
        \wrongchoice{a sound wave traveling at a lower speed.}
    \end{choices}
\end{question}
}

\element{jpierce}{
\begin{question}{pt101tb4-Q149}
    A tuning fork has the number 320 stamped on it. 
    What does this indicate?
    \begin{choices}
        \wrongchoice{The speed of sound in the tuning fork is \SI{320}{\meter\per\second}.}
        \wrongchoice{The wavelength of sound produced by this tuning fork is \SI{320}{\meter}.}
        \wrongchoice{The wavelength of sound produced by this tuning fork is \SI{320}{\centi\meter}.}
      \correctchoice{This tuning fork vibrates at a frequency of \SI{320}{\hertz}.}
        \wrongchoice{The speed of sound in the tuning fork is \SI{320}{\centi\meter\per\second}.}
    \end{choices}
\end{question}
}

\element{jpierce}{
\begin{question}{pt101tb4-Q150}
    A tuning fork has the number 256 stamped on it. 
    What does this indicate?
    \begin{choices}
        \wrongchoice{The wavelength of sound produced by this tuning fork is \SI{256}{\centi\meter}.}
      \correctchoice{This tuning fork vibrates at a frequency of \SI{256}{\hertz}.}
        \wrongchoice{The wavelength of sound produced by this tuning fork is \SI{256}{\meter}.}
        \wrongchoice{The speed of sound in the tuning fork is \SI{256}{\meter\per\second}.}
        \wrongchoice{The speed of sound in the tuning fork is \SI{256}{\centi\meter\per\second}.}
    \end{choices}
\end{question}
}

\element{jpierce}{
\begin{question}{pt101tb4-Q151}
    A tuning fork has the number 512 stamped on it. 
    What does this indicate?
    \begin{choices}
        \wrongchoice{The wavelength of sound produced by this tuning fork is \SI{512}{\meter}.}
        \wrongchoice{The wavelength of sound produced by this tuning fork is \SI{512}{\centi\meter}.}
      \correctchoice{This tuning fork vibrates at a frequency of \SI{512}{\hertz}.}
        \wrongchoice{The speed of sound in the tuning fork is \SI{512}{\centi\meter\per\second}.}
        \wrongchoice{The speed of sound in the tuning fork is \SI{512}{\meter\per\second}.}
    \end{choices}
\end{question}
}

%% Topic: Quality
\element{jpierce}{
\begin{question}{pt101tb4-Q152}
    The quality of a musical tone depends on:
    \begin{choices}
        \wrongchoice{the relative intensities of the partial tones.}
        \wrongchoice{the harmonics present in the tone.}
        \wrongchoice{the frequencies of the partial tones.}
        \wrongchoice{the instrument that created the tone.}
      \correctchoice{all of the above.}
    \end{choices}
\end{question}
}

\element{jpierce}{
\begin{question}{pt101tb4-Q153}
    Partial tones whose frequencies are whole number multiples
        of the fundamental frequency are called:
    \begin{multicols}{2}
    \begin{choices}
        \wrongchoice{integers.}
        \wrongchoice{tonics.}
        \wrongchoice{noise.}
        \wrongchoice{radicals.}
      \correctchoice{harmonics.}
    \end{choices}
    \end{multicols}
\end{question}
}

\element{jpierce}{
\begin{question}{pt101tb4-Q154}
    The first harmonic is also called the:
    \begin{multicols}{2}
    \begin{choices}
        \wrongchoice{tonic.}
        \wrongchoice{basic.}
        \wrongchoice{full tone.}
        \wrongchoice{low note.}
      \correctchoice{fundamental.}
    \end{choices}
    \end{multicols}
\end{question}
}

\element{jpierce}{
\begin{question}{pt101tb4-Q155}
    The lowest possible frequency of vibration of an air pipe
        (such as a trombone or a penny whistle) is called:
    \begin{multicols}{2}
    \begin{choices}
        \wrongchoice{tonic.}
        \wrongchoice{full tone.}
        \wrongchoice{low note.}
      \correctchoice{fundamental.}
        \wrongchoice{basic.}
    \end{choices}
    \end{multicols}
\end{question}
}

\element{jpierce}{
\begin{question}{pt101tb4-Q156}
    When a string vibrates at its fundamental frequency,
        it will produce sounds with:
    \begin{choices}
        \wrongchoice{the most noise.}
      \correctchoice{the longest wavelength.}
        \wrongchoice{the lowest speed.}
        \wrongchoice{the highest pitch.}
        \wrongchoice{the most harmonics.}
    \end{choices}
\end{question}
}

\element{jpierce}{
\begin{question}{pt101tb4-Q157}
    When a string vibrates at its fundamental frequency,
        it will produce sounds with:
    \begin{choices}
        \wrongchoice{the shortest wavelength.}
        \wrongchoice{the most harmonics.}
        \wrongchoice{the lowest speed.}
        \wrongchoice{the most noise.}
      \correctchoice{the lowest pitch.}
    \end{choices}
\end{question}
}

\element{jpierce}{
\begin{question}{pt101tb4-Q158}
    A guitar string vibrates at a fundamental frequency of \SI{110}{\hertz}. 
    The frequency of the first harmonic for this string would be:
    \begin{multicols}{2}
    \begin{choices}
      \correctchoice{\SI{110}{\hertz}}
        \wrongchoice{\SI{440}{\hertz}}
        \wrongchoice{\SI{120}{\hertz}}
        \wrongchoice{\SI{111}{\hertz}}
        \wrongchoice{\SI{220}{\hertz}}
    \end{choices}
    \end{multicols}
\end{question}
}

\element{jpierce}{
\begin{question}{pt101tb4-Q159}
    When a guitar string vibrates at its fundamental frequency,
        it will have a node at each end and \rule[-0.1pt]{4em}{0.1pt} in between.
    \begin{multicols}{2}
    \begin{choices}
      \correctchoice{no nodes}
        \wrongchoice{four nodes}
        \wrongchoice{three nodes}
        \wrongchoice{one node}
        \wrongchoice{two nodes}
    \end{choices}
    \end{multicols}
\end{question}
}

\element{jpierce}{
\begin{question}{pt101tb4-Q160}
    When a guitar string vibrates at the frequency of its second harmonic,
        it will have a node at each end and \rule[-0.1pt]{4em}{0.1pt} in between.
    \begin{multicols}{2}
    \begin{choices}
        \wrongchoice{four nodes}
        \wrongchoice{no nodes}
        \wrongchoice{two nodes}
      \correctchoice{one node}
        \wrongchoice{three nodes}
    \end{choices}
    \end{multicols}
\end{question}
}

\element{jpierce}{
\begin{question}{pt101tb4-Q161}
    When a guitar string vibrates at the frequency of its third harmonic,
        it will have a node at each end and \rule[-0.1pt]{4em}{0.1pt} in between.
    \begin{multicols}{2}
    \begin{choices}
        \wrongchoice{four nodes}
        \wrongchoice{one node}
        \wrongchoice{no nodes}
        \wrongchoice{three nodes}
      \correctchoice{two nodes}
    \end{choices}
    \end{multicols}
\end{question}
}

\element{jpierce}{
\begin{question}{pt101tb4-Q162}
    A guitar string vibrates at a fundamental frequency of \SI{110}{\hertz}. 
    The frequency of the second harmonic for this string would be:
    \begin{multicols}{2}
    \begin{choices}
        \wrongchoice{\SI{330}{\hertz}}
        \wrongchoice{\SI{440}{\hertz}}
        \wrongchoice{\SI{120}{\hertz}}
      \correctchoice{\SI{220}{\hertz}}
        \wrongchoice{\SI{112}{\hertz}}
    \end{choices}
    \end{multicols}
\end{question}
}

\element{jpierce}{
\begin{question}{pt101tb4-Q163}
    A guitar string vibrates at a fundamental frequency of \SI{110}{\hertz}. 
    The frequency of the third harmonic for this string would be:
    \begin{multicols}{2}
    \begin{choices}
        \wrongchoice{\SI{130}{\hertz}}
        \wrongchoice{\SI{220}{\hertz}}
        \wrongchoice{\SI{113}{\hertz}}
        \wrongchoice{\SI{440}{\hertz}}
      \correctchoice{\SI{330}{\hertz}}
    \end{choices}
    \end{multicols}
\end{question}
}

\endinput

