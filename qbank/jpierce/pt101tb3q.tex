

%% Physics 101 Sample Test Questions by Dr. James Pierce
%%------------------------------------------------------------


%% JP's Physics 101 Test Bank 3
%%--------------------------------------------------


%% Topic: Atom Charge
\element{jpierce}{
\begin{question}{pt101tb3-Q01}
    The net charge of an atom is determined by the number of \rule[-0.1pt]{4em}{0.1pt} it has.
    \begin{choices}
        \wrongchoice{neutrons}
      \correctchoice{protons and electrons}
        \wrongchoice{protons}
        \wrongchoice{electrons and neutrons}
        \wrongchoice{neutrons and protons}
    \end{choices}
\end{question}
}

\element{jpierce}{
\begin{question}{pt101tb3-Q02}
    Protons have \rule[-0.1pt]{4em}{0.1pt} charge,
        neutrons have \rule[-0.1pt]{4em}{0.1pt} charge,
        and electrons have \rule[-0.1pt]{4em}{0.1pt} charge.
    \begin{choices}
      \correctchoice{positive; no; negative}
        \wrongchoice{no; negative; positive}
        \wrongchoice{negative; positive; no}
        \wrongchoice{negative; no; positive}
        \wrongchoice{positive; negative; no}
    \end{choices}
\end{question}
}

\element{jpierce}{
\begin{question}{pt101tb3-Q03}
    Neutrons have \rule[-0.1pt]{4em}{0.1pt} charge,
        electrons have \rule[-0.1pt]{4em}{0.1pt} charge,
        and protons have \rule[-0.1pt]{4em}{0.1pt} charge.
    \begin{choices}
      \correctchoice{no; negative; positive}
        \wrongchoice{negative; no; positive}
        \wrongchoice{negative; positive; no}
        \wrongchoice{positive; negative; no}
        \wrongchoice{positive; no; negative}
    \end{choices}
\end{question}
}

\element{jpierce}{
\begin{question}{pt101tb3-Q04}
    Electrons have \rule[-0.1pt]{4em}{0.1pt} charge,
        protons have \rule[-0.1pt]{4em}{0.1pt} charge,
        and neutrons have \rule[-0.1pt]{4em}{0.1pt} charge.
    \begin{choices}
        \wrongchoice{no; negative; positive}
      \correctchoice{negative; positive; no}
        \wrongchoice{positive; negative; no}
        \wrongchoice{negative; no; positive}
        \wrongchoice{positive; no; negative}
    \end{choices}
\end{question}
}

\element{jpierce}{
\begin{question}{pt101tb3-Q05}
    \rule[-0.1pt]{4em}{0.1pt} have negative charge,
        \rule[-0.1pt]{4em}{0.1pt} have positive charge,
        and \rule[-0.1pt]{4em}{0.1pt} have no charge.
    \begin{choices}
      \correctchoice{Electrons; protons; neutrons}
        \wrongchoice{Neutrons; electrons; protons}
        \wrongchoice{Protons; electrons; neutrons}
        \wrongchoice{Protons; neutrons; electrons}
        \wrongchoice{Electrons; neutrons; protons}
    \end{choices}
\end{question}
}

\element{jpierce}{
\begin{question}{pt101tb3-Q06}
    \rule[-0.1pt]{4em}{0.1pt} have positive charge,
        \rule[-0.1pt]{4em}{0.1pt} have no charge,
        and \rule[-0.1pt]{4em}{0.1pt} have negative charge.
    \begin{choices}
        \wrongchoice{Electrons; neutrons; protons}
        \wrongchoice{Protons; electrons; neutrons}
        \wrongchoice{Electrons; protons; neutrons}
        \wrongchoice{Neutrons; electrons; protons}
      \correctchoice{Protons; neutrons; electrons}
    \end{choices}
\end{question}
}

\element{jpierce}{
\begin{question}{pt101tb3-Q07}
    \rule[-0.1pt]{4em}{0.1pt} have no charge,
        \rule[-0.1pt]{4em}{0.1pt}  have negative charge,
        and \rule[-0.1pt]{4em}{0.1pt} have positive charge.
    \begin{choices}
        \wrongchoice{Electrons; neutrons; protons}
        \wrongchoice{Electrons; protons; neutrons}
        \wrongchoice{Protons; neutrons; electrons}
        \wrongchoice{Protons; electrons; neutrons}
      \correctchoice{Neutrons; electrons; protons}
    \end{choices}
\end{question}
}

\element{jpierce}{
\begin{question}{pt101tb3-Q08}
    In a neutral atom, the number of electrons is equal to
    \begin{choices}
        \wrongchoice{the number of protons minus the number of neutrons.}
        \wrongchoice{the number of neutrons.}
        \wrongchoice{the total number of neutrons and protons.}
        \wrongchoice{the number of neutrons minus the number of protons.}
      \correctchoice{the number of protons.}
    \end{choices}
\end{question}
}

\element{jpierce}{
\begin{question}{pt101tb3-Q09}
    If the number of electrons in an atom is equal to the number of protons,
        the atom is said to be \rule[-0.1pt]{4em}{0.1pt}.
    \begin{multicols}{2}
    \begin{choices}
      \correctchoice{neutral}
        \wrongchoice{a molecule}
        \wrongchoice{carbon}
        \wrongchoice{a compound}
        \wrongchoice{a neutron}
    \end{choices}
    \end{multicols}
\end{question}
}

%% Topic: Atom Element
\element{jpierce}{
\begin{question}{pt101tb3-Q10}
    Which of the following is true?
    \begin{choices}
        \wrongchoice{Some atoms do not belong to any particular element.}
        \wrongchoice{Some atoms belong to more than one element.}
        \wrongchoice{All atoms are identical.}
      \correctchoice{The number of protons in an atom determines which element it is.}
        \wrongchoice{The number of neutrons in an atom determines which element it is.}
    \end{choices}
\end{question}
}

\element{jpierce}{
\begin{question}{pt101tb3-Q11}
    The number of \rule[-0.1pt]{4em}{0.1pt} in an atom
        determines which element it is.
    \begin{choices}
        \wrongchoice{neutrons}
        \wrongchoice{shells}
        \wrongchoice{electrons}
        \wrongchoice{atomic mass units}
      \correctchoice{protons}
    \end{choices}
\end{question}
}

\element{jpierce}{
\begin{question}{pt101tb3-Q12}
    Any two atoms of gold
    \begin{choices}
        \wrongchoice{have the same nuclei.}
        \wrongchoice{can have different numbers of protons.}
        \wrongchoice{have the same number of neutrons.}
        \wrongchoice{have the same atomic mass number.}
      \correctchoice{have the same atomic number.}
    \end{choices}
\end{question}
}

\element{jpierce}{
\begin{question}{pt101tb3-Q13}
    The number of neutrons in an atom of a given element
    \begin{choices}
        \wrongchoice{is equal to the number of protons.}
        \wrongchoice{is equal to the number of electrons.}
        \wrongchoice{is equal to the atomic number.}
        \wrongchoice{is equal to the atomic mass number.}
      \correctchoice{may vary from atom to atom.}
    \end{choices}
\end{question}
}

%% Topic: Atom Force
\element{jpierce}{
\begin{question}{pt101tb3-Q14}
    Within the atom the electron cloud is bound to the nucleus by:
    \begin{choices}
        \wrongchoice{magnetic forces.}
        \wrongchoice{tiny springs.}
        \wrongchoice{gravitational forces.}
        \wrongchoice{very fine filments.}
      \correctchoice{electrostatic forces.}
    \end{choices}
\end{question}
}

\element{jpierce}{
\begin{question}{pt101tb3-Q15}
    Within the atom an electron is:
    \begin{choices}
      \correctchoice{attracted by the protons in the nucleus and repelled by other electrons.}
        \wrongchoice{repelled by the neutrons in the nucleus and attracted by other electrons.}
        \wrongchoice{attracted by protons and repelled by neutrons.}
        \wrongchoice{attracted by the neutrons in the nucleus and repelled by other electrons.}
        \wrongchoice{repelled by the protons in the nucleus and attracted by other electrons.}
    \end{choices}
\end{question}
}

\element{jpierce}{
\begin{question}{pt101tb3-Q16}
    The mass of one hydrogen atom is approximately:
    \begin{choices}
      \correctchoice{one atomic mass unit.}
        \wrongchoice{\num{1/2} atomic mass unit.}
        \wrongchoice{two atomic mass units.}
        \wrongchoice{\num{16} atomic mass units.}
        \wrongchoice{\num{12} atomic mass units.}
    \end{choices}
\end{question}
}

\element{jpierce}{
\begin{question}{pt101tb3-Q17}
    The mass of one oxygen atom is approximately:
    \begin{choices}
        \wrongchoice{two atomic mass units.}
        \wrongchoice{one atomic mass unit.}
        \wrongchoice{\num{1/2} atomic mass unit.}
      \correctchoice{\num{16} atomic mass units.}
        \wrongchoice{\num{12} atomic mass units.}
    \end{choices}
\end{question}
}

\element{jpierce}{
\begin{question}{pt101tb3-Q18}
    The mass of one carbon atom is approximately:
    \begin{choices}
        \wrongchoice{one atomic mass unit.}
        \wrongchoice{\num{1/2} atomic mass unit.}
        \wrongchoice{two atomic mass units.}
      \correctchoice{\num{12} atomic mass units.}
        \wrongchoice{\num{16} atomic mass units.}
    \end{choices}
\end{question}
}

\element{jpierce}{
\begin{question}{pt101tb3-Q19}
    The mass of one hydrogen molecule is approximately:
    \begin{choices}
        \wrongchoice{\num{1/2} atomic mass unit.}
      \correctchoice{two atomic mass units.}
        \wrongchoice{one atomic mass unit.}
        \wrongchoice{\num{12} atomic mass units.}
        \wrongchoice{\num{16} atomic mass units.}
    \end{choices}
\end{question}
}

\element{jpierce}{
\begin{question}{pt101tb3-Q20}
    Almost all of the atom's mass is located in the atom's:
    %\rule[-0.1pt]{4em}{0.1pt}.
    \begin{multicols}{2}
    \begin{choices}
        \wrongchoice{electron cloud.}
        \wrongchoice{protons.}
        \wrongchoice{electrons.}
      \correctchoice{nucleus.}
        \wrongchoice{neutrons.}
    \end{choices}
    \end{multicols}
\end{question}
}

%% Topic: Atom Size
\element{jpierce}{
\begin{question}{pt101tb3-Q21}
    The diameter of a typical atom is about \rule[-0.1pt]{4em}{0.1pt} times the diameter of its nucleus.
    \begin{multicols}{2}
    \begin{choices}
        \wrongchoice{\num{100}}
      \correctchoice{\num{100000}}
        \wrongchoice{\num{1000}}
        \wrongchoice{\num{10}}
        \wrongchoice{\num{2}}
    \end{choices}
    \end{multicols}
\end{question}
}

\element{jpierce}{
\begin{question}{pt101tb3-Q22}
    100 million atoms lined up next to each other would form a line extending for about
    \begin{multicols}{2}
    \begin{choices}
        \wrongchoice{1 million kilometers.}
        \wrongchoice{1 meter.}
        \wrongchoice{1000 kilometers.}
        \wrongchoice{1 kilometer.}
      \correctchoice{1 centimeter.}
    \end{choices}
    \end{multicols}
\end{question}
}

\element{jpierce}{
\begin{question}{pt101tb3-Q23}
    The diameter of a typical atom is about \rule[-0.1pt]{4em}{0.1pt} times
        the diameter of its nucleus.
    \begin{multicols}{2}
    \begin{choices}
        \wrongchoice{\num{1000}}
        \wrongchoice{\num{1/100000}}
        \wrongchoice{\num{1/1000}}
        \wrongchoice{\num{10}}
      \correctchoice{\num{100 000}}
    \end{choices}
    \end{multicols}
\end{question}
}

\element{jpierce}{
\begin{question}{pt101tb3-Q24}
    The diameter of a typical atomic nucleus is about \rule[-0.1pt]{4em}{0.1pt} times
        the diameter of the atom.
    \begin{multicols}{2}
    \begin{choices}
      \correctchoice{\num{1/100000}}
        \wrongchoice{\num{100 000}}
        \wrongchoice{\num{1/1000}}
        \wrongchoice{\num{10}}
        \wrongchoice{\num{1000}}
    \end{choices}
    \end{multicols}
\end{question}
}

%% Topic: Atom Structure
\element{jpierce}{
\begin{question}{pt101tb3-Q25}
    The atom consists of a nucleus of \rule[-0.1pt]{4em}{0.1pt} surrounded 
        by a cloud of \rule[-0.1pt]{4em}{0.1pt}.
    \begin{choices}
        \wrongchoice{neutrons and electrons; protons}
      \correctchoice{protons and neutrons; electrons}
        \wrongchoice{electrons and protons; neutrons}
        \wrongchoice{neutrons and electrons; positrons}
        \wrongchoice{positrons and neutrons; electrons}
    \end{choices}
\end{question}
}

\element{jpierce}{
\begin{question}{pt101tb3-Q26}
    The atom consists of a \rule[-0.1pt]{4em}{0.1pt} nucleus
        surrounded by a \rule[-0.1pt]{4em}{0.1pt} electron cloud.
    \begin{choices}
        \wrongchoice{positively charged; positively charged}
      \correctchoice{positively charged; negatively charged}
        \wrongchoice{negatively charged; negatively charged}
        \wrongchoice{negatively charged; positively charged}
        \wrongchoice{neutral; neutral}
    \end{choices}
\end{question}
}

%% Topic: Atom Number
\element{jpierce}{
\begin{question}{pt101tb3-Q27}
    Your body contains approximately \rule[-0.1pt]{4em}{0.1pt} atoms.
    \begin{multicols}{3}
    \begin{choices}
        \wrongchoice{\num{e12}}
        \wrongchoice{\num{e18}}
      \correctchoice{\num{e27}}
        \wrongchoice{\num{1000}}
        \wrongchoice{\num{e6}}
    \end{choices}
    \end{multicols}
\end{question}
}

\element{jpierce}{
\begin{question}{pt101tb3-Q28}
    The atomic number of an atom identifies which \rule[-0.1pt]{4em}{0.1pt} it is.
    \begin{multicols}{2}
    \begin{choices}
        \wrongchoice{mixture}
      \correctchoice{element}
        \wrongchoice{molecule}
        \wrongchoice{phase}
        \wrongchoice{compound}
    \end{choices}
    \end{multicols}
\end{question}
}

\element{jpierce}{
\begin{question}{pt101tb3-Q29}
    The atomic number is the number of \rule[-0.1pt]{4em}{0.1pt} in the nucleus.
    \begin{multicols}{2}
    \begin{choices}
        \wrongchoice{electrons}
        \wrongchoice{molecules}
      \correctchoice{protons}
        \wrongchoice{neutrons}
        \wrongchoice{elements}
    \end{choices}
    \end{multicols}
\end{question}
}

\element{jpierce}{
\begin{question}{pt101tb3-Q30}
    The number of protons in an atom is given by the atom's \rule[-0.1pt]{4em}{0.1pt}.
    \begin{choices}
        \wrongchoice{atomic mass number}
        \wrongchoice{atomic mass unit}
      \correctchoice{atomic number}
        \wrongchoice{atomic weight}
        \wrongchoice{isotopic number}
    \end{choices}
\end{question}
}

\element{jpierce}{
\begin{question}{pt101tb3-Q31}
    An element with an atomic number of 6 and an atomic mass number of 13 would have
    \begin{choices}
      \correctchoice{6 protons, 7 neutrons, and 6 electrons.}
        \wrongchoice{7 protons, 6 neutrons, and 6 electrons.}
        \wrongchoice{6 protons, 13 neutrons, and 7 electrons.}
        \wrongchoice{6 protons, 7 neutrons, and 13 electrons.}
        \wrongchoice{7 protons, 6 neutrons, and 7 electrons.}
    \end{choices}
\end{question}
}

\element{jpierce}{
\begin{question}{pt101tb3-Q32}
    An element with an atomic number of 3 and an atomic mass number of 7 would have
    \begin{choices}
      \correctchoice{3 protons, 4 neutrons, and 3 electrons.}
        \wrongchoice{4 protons, 3 neutrons, and 4 electrons.}
        \wrongchoice{3 protons, 4 neutrons, and 7 electrons.}
        \wrongchoice{4 protons, 3 neutrons, and 3 electrons.}
        \wrongchoice{3 protons, 7 neutrons, and 4 electrons.}
    \end{choices}
\end{question}
}

\element{jpierce}{
\begin{question}{pt101tb3-Q33}
    An element with an atomic number of 92 and an atomic mass number of 238 would have
    \begin{choices}
        \wrongchoice{146 protons, 92 neutrons, and 92 electrons.}
      \correctchoice{92 protons, 146 neutrons, and 92 electrons.}
        \wrongchoice{92 protons, 146 neutrons, and 238 electrons.}
        \wrongchoice{146 protons, 92 neutrons, and 146 electrons.}
        \wrongchoice{92 protons, 238 neutrons, and 146 electrons.}
    \end{choices}
\end{question}
}

\element{jpierce}{
\begin{question}{pt101tb3-Q34}
    An element with an atomic number of 7 and an atomic mass number of 13 would have
    \begin{choices}
        \wrongchoice{6 protons, 7 neutrons, and 7 electrons.}
        \wrongchoice{7 protons, 6 neutrons, and 13 electrons.}
        \wrongchoice{6 protons, 13 neutrons, and 7 electrons.}
      \correctchoice{7 protons, 6 neutrons, and 7 electrons.}
        \wrongchoice{6 protons, 7 neutrons, and 6 electrons.}
    \end{choices}
\end{question}
}

%% Topic: Brownian
\element{jpierce}{
\begin{question}{pt101tb3-Q35}
    Brownian motion is explained as being caused by the bombardment of visible particles by:
    \begin{choices}
        \wrongchoice{gusts of wind.}
      \correctchoice{atoms and molecules.}
        \wrongchoice{sound waves.}
        \wrongchoice{antimatter.}
        \wrongchoice{light waves.}
    \end{choices}
\end{question}
}

\element{jpierce}{
\begin{question}{pt101tb3-Q36}
    \rule[-0.1pt]{4em}{0.1pt} explained as being caused by the bombardment
        of visible particles by atoms and molecules.
    \begin{choices}
        \wrongchoice{Northern lights are}
        \wrongchoice{Electrons are}
        \wrongchoice{Lightning is}
      \correctchoice{Brownian motion is}
        \wrongchoice{Water waves are}
    \end{choices}
\end{question}
}

\element{jpierce}{
\begin{question}{pt101tb3-Q37}
    Brownian motion is the
    \begin{choices}
      \correctchoice{random motion of microscopic particles being bombarded by even smaller atoms and molecules.}
        \wrongchoice{vibration of atoms and molecules in a solid.}
        \wrongchoice{movement of electrons circulating within the atom.}
        \wrongchoice{very gradual flow of solid materials such as glass over long periods of time.}
        \wrongchoice{random motion of atoms and molecules being bombarded by larger microscopic particles.}
    \end{choices}
\end{question}
}

\element{jpierce}{
\begin{question}{pt101tb3-Q38}
    Brownian motion is explained as being caused by the
        bombardment of \rule[-0.1pt]{4em}{0.1pt} by atoms and molecules.
    \begin{choices}
        \wrongchoice{atomic nuclei}
        \wrongchoice{baseballs}
        \wrongchoice{brownies}
        \wrongchoice{individual electrons}
      \correctchoice{small, but visible, particles}
    \end{choices}
\end{question}
}

\element{jpierce}{
\begin{question}{pt101tb3-Q39}
    Brownian motion is evidence of the existence of \rule[-0.1pt]{4em}{0.1pt}.
    \begin{multicols}{2}
    \begin{choices}
        \wrongchoice{inertia}
        \wrongchoice{friction}
        \wrongchoice{kinetic energy}
      \correctchoice{atoms}
        \wrongchoice{gravity}
    \end{choices}
    \end{multicols}
\end{question}
}

%% Topic: Combinations
\element{jpierce}{
\begin{question}{pt101tb3-Q40}
    Chemical combinations of elements are called
    \begin{multicols}{2}
    \begin{choices}
        \wrongchoice{nuclei.}
        \wrongchoice{shells.}
        \wrongchoice{groups.}
        \wrongchoice{mixtures.}
      \correctchoice{compounds.}
    \end{choices}
    \end{multicols}
\end{question}
}

\element{jpierce}{
\begin{question}{pt101tb3-Q41}
    Atoms combine to form
    \begin{choices}
        \wrongchoice{atomic numbers.}
        \wrongchoice{electron clouds.}
      \correctchoice{molecules.}
        \wrongchoice{quarks.}
        \wrongchoice{antimatter.}
    \end{choices}
\end{question}
}

\element{jpierce}{
\begin{question}{pt101tb3-Q42}
    Sodium chloride (table salt) is an example of:
    \begin{choices}
        \wrongchoice{a molecule.}
        \wrongchoice{an element.}
        \wrongchoice{an atom.}
        \wrongchoice{a mixture.}
      \correctchoice{a compound.}
    \end{choices}
\end{question}
}

\element{jpierce}{
\begin{question}{pt101tb3-Q43}
    Air is an example of:
    \begin{multicols}{2}
    \begin{choices}
        \wrongchoice{an atom.}
        \wrongchoice{a compound.}
        \wrongchoice{an element.}
        \wrongchoice{a molecule.}
      \correctchoice{a mixture.}
    \end{choices}
    \end{multicols}
\end{question}
}

\element{jpierce}{
\begin{question}{pt101tb3-Q44}
    The mercury in a thermometer is an example of:
    \begin{multicols}{2}
    \begin{choices}
        \wrongchoice{antimatter.}
        \wrongchoice{a compound.}
      \correctchoice{an element.}
        \wrongchoice{a molecule.}
        \wrongchoice{a mixture.}
    \end{choices}
    \end{multicols}
\end{question}
}

\element{jpierce}{
\begin{question}{pt101tb3-Q45}
    \ce{H20} is an example of \rule[-0.1pt]{4em}{0.1pt}.
    \begin{multicols}{2}
    \begin{choices}
        \wrongchoice{a mixture.}
        \wrongchoice{an atom.}
      \correctchoice{a molecule.}
        \wrongchoice{an element.}
        \wrongchoice{an isotope.}
    \end{choices}
    \end{multicols}
\end{question}
}

\element{jpierce}{
\begin{question}{pt101tb3-Q46}
    The constituent elements of water are
    \begin{choices}
        \wrongchoice{ice and steam.}
      \correctchoice{hydrogen and oxygen.}
        \wrongchoice{helium and nitrogen.}
        \wrongchoice{hydrogen and helium.}
        \wrongchoice{nitrogen and oxygen.}
    \end{choices}
\end{question}
}

\element{jpierce}{
\begin{question}{pt101tb3-Q47}
    A chemical substance made of atoms of two or more different elements
        combined in a fixed proportion is called
    \begin{multicols}{2}
    \begin{choices}
        \wrongchoice{a mixture.}
        \wrongchoice{a nucleus.}
        \wrongchoice{an isotope.}
        \wrongchoice{a crystal.}
      \correctchoice{a compound.}
    \end{choices}
    \end{multicols}
\end{question}
}

\element{jpierce}{
\begin{question}{pt101tb3-Q48}
    Each molecule of air contains
    \begin{choices}
        \wrongchoice{two atoms of nitrogen and two atoms of oxygen.}
        \wrongchoice{two atoms of nitrogen and one atom of oxygen.}
        \wrongchoice{one atom of nitrogen and two atoms of oxygen.}
        \wrongchoice{one atom of nitrogen and one atom of oxygen.}
      \correctchoice{none of these---air is not a compound.}
    \end{choices}
\end{question}
}

\element{jpierce}{
\begin{question}{pt101tb3-Q49}
    The principal constituents of air are
    \begin{choices}
        \wrongchoice{oxygen molecules and water molecules.}
        \wrongchoice{hydrogen molecules and oxygen molecules.}
      \correctchoice{oxygen molecules and nitrogen molecules.}
        \wrongchoice{hydrogen molecules and water molecules.}
        \wrongchoice{hydrogen molecules and nitrogen molecules.}
    \end{choices}
\end{question}
}

\element{jpierce}{
\begin{question}{pt101tb3-Q50}
    Which of the following is a list of elements?
    \begin{choices}
      \correctchoice{hydrogen, oxygen, nitrogen}
        \wrongchoice{air, nitrogen, oxygen}
        \wrongchoice{water, nitrogen, oxygen}
        \wrongchoice{hydrogen, oxygen, water}
        \wrongchoice{hydrogen, nitrogen, air}
    \end{choices}
\end{question}
}

%% Topic: Elements Human
\element{jpierce}{
\begin{question}{pt101tb3-Q51}
    The human body is composed primarily of the elements
    \begin{choices}
        \wrongchoice{helium, oxygen, carbon, and nitrogen.}
      \correctchoice{hydrogen, oxygen, carbon, and nitrogen.}
        \wrongchoice{hydrogen, helium, oxygen, and carbon.}
        \wrongchoice{hydrogen, helium, nitrogen, and oxygen.}
        \wrongchoice{hydrogen, helium, carbon, and nitrogen.}
    \end{choices}
\end{question}
}

%% Topic: Periodic Table
\element{jpierce}{
\begin{question}{pt101tb3-Q52}
    The element just to the right of iron on the periodic table has
    \begin{choices}
        \wrongchoice{one fewer proton and one more electron than iron.}
      \correctchoice{one more proton and one more electron than iron.}
        \wrongchoice{the same numbers of protons and electrons as iron.}
        \wrongchoice{one more proton and one fewer electron than iron.}
        \wrongchoice{one fewer proton and one fewer electron than iron.}
    \end{choices}
\end{question}
}

\element{jpierce}{
\begin{question}{pt101tb3-Q53}
    The element just to the left of iron on the periodic table has
    \begin{choices}
        \wrongchoice{one fewer proton and one more electron than iron.}
        \wrongchoice{one more proton and one more electron than iron.}
        \wrongchoice{the same numbers of protons and electrons as iron.}
        \wrongchoice{one more proton and one fewer electron than iron.}
      \correctchoice{one fewer proton and one fewer electron than iron.}
    \end{choices}
\end{question}
}

\element{jpierce}{
\begin{question}{pt101tb3-Q54}
    Where on the periodic table would we find an element with one more
        proton and one more electron than silver?
    \begin{choices}
        \wrongchoice{Just below silver.}
        \wrongchoice{Just to the left of silver.}
      \correctchoice{Just to the right of silver.}
        \wrongchoice{Just above silver.}
        \wrongchoice{None of these---there is no such element.}
    \end{choices}
\end{question}
}

\element{jpierce}{
\begin{question}{pt101tb3-Q55}
    Where on the periodic table would we find an element with one fewer
        proton and one fewer electron than silver?
    \begin{choices}
        \wrongchoice{Just to the right of silver.}
        \wrongchoice{Just above silver.}
      \correctchoice{Just to the left of silver.}
        \wrongchoice{Just below silver.}
        \wrongchoice{None of these---there is no such element.}
    \end{choices}
\end{question}
}

\element{jpierce}{
\begin{question}{pt101tb3-Q56}
    Where on the periodic table would we find an element with one fewer
        proton and one fewer electron than hydrogen?
    \begin{choices}
        \wrongchoice{Just to the right of hydrogen.}
        \wrongchoice{Just below hydrogen.}
        \wrongchoice{Just to the left of hydrogen.}
        \wrongchoice{Just above hydrogen.}
      \correctchoice{None of these---there is no such element.}
    \end{choices}
\end{question}
}

\element{jpierce}{
\begin{question}{pt101tb3-Q57}
    The element just to the right of oxygen on the periodic table has
    \begin{choices}
      \correctchoice{one more proton and one more electron than oxygen.}
        \wrongchoice{one more proton and one fewer electron than oxygen.}
        \wrongchoice{one fewer proton and one fewer electron than oxygen.}
        \wrongchoice{the same numbers of protons and electrons as oxygen.}
        \wrongchoice{one fewer proton and one more electron than oxygen.}
    \end{choices}
\end{question}
}

\element{jpierce}{
\begin{question}{pt101tb3-Q58}
    The element just to the left of oxygen on the periodic table has
    \begin{choices}
        \wrongchoice{one more proton and one fewer electron than oxygen.}
      \correctchoice{one fewer proton and one fewer electron than oxygen.}
        \wrongchoice{one more proton and one more electron than oxygen.}
        \wrongchoice{the same numbers of protons and electrons as oxygen.}
        \wrongchoice{one fewer proton and one more electron than oxygen.}
    \end{choices}
\end{question}
}

\element{jpierce}{
\begin{question}{pt101tb3-Q59}
    Where does water appear on the periodic table of the elements?
    \begin{choices}
        \wrongchoice{Just to the right of hydrogen.}
        \wrongchoice{Just to the right of oxygen.}
        \wrongchoice{In the box marked `W'.}
      \correctchoice{Nowhere; water is not an element.}
        \wrongchoice{Between hydrogen and oxygen.}
    \end{choices}
\end{question}
}

\element{jpierce}{
\begin{question}{pt101tb3-Q60}
    Where does air appear on the periodic table of the elements?
    \begin{choices}
      \correctchoice{Nowhere---air is not an element.}
        \wrongchoice{Between nitrogen and oxygen.}
        \wrongchoice{In the box marked 'Ar'.}
        \wrongchoice{Just to the right of oxygen.}
        \wrongchoice{Just to the right of nitrogen.}
    \end{choices}
\end{question}
}

%% Topic: Bonding
\element{jpierce}{
\begin{question}{pt101tb3-Q61}
    Atoms bond to each other in solids through their:
    \begin{choices}
      \correctchoice{protons.}
        \wrongchoice{neutrons.}
        \wrongchoice{electrons.}
        \wrongchoice{nuclei.}
        \wrongchoice{none of these---the atoms in a solid are not bonded to each other.}
    \end{choices}
\end{question}
}

%% Topic: Density
\element{jpierce}{
\begin{question}{pt101tb3-Q62}
    Density is:
    \begin{choices}
        \wrongchoice{mass times volume.}
      \correctchoice{the same as atomic number.}
        \wrongchoice{mass divided by volume.}
        \wrongchoice{mass times velocity.}
        \wrongchoice{mass divided by velocity.}
    \end{choices}
\end{question}
}

\element{jpierce}{
\begin{question}{pt101tb3-Q63}
    Density is:
    \begin{choices}
        \wrongchoice{mass times volume.}
        \wrongchoice{mass minus volume.}
      \correctchoice{mass divided by volume.}
        \wrongchoice{mass plus volume.}
        \wrongchoice{volume divided by mass.}
    \end{choices}
\end{question}
}

\element{jpierce}{
\begin{question}{pt101tb3-Q64}
    The element with the highest density is:
    \begin{multicols}{2}
    \begin{choices}
      \correctchoice{osmium.}
        \wrongchoice{water.}
        \wrongchoice{aluminum.}
        \wrongchoice{gold.}
        \wrongchoice{mercury.}
    \end{choices}
    \end{multicols}
\end{question}
}

\element{jpierce}{
\begin{question}{pt101tb3-Q65}
    The density of water is approximately one
    \begin{choices}
        \wrongchoice{gram per liter (\si{\gram\per\liter}).}
      \correctchoice{gram per cubic centimeter (\si{\gram\per\centi\meter\cubed}).}
        \wrongchoice{kilogram per cubic centimeter (\si{\kilo\gram\per\centi\meter\cubed}).}
        \wrongchoice{gram per cubic meter (\si{\gram\per\centi\meter\cubed}).}
        \wrongchoice{kilogram per cubic meter (\si{\kilo\gram\per\meter\cubed}).}
    \end{choices}
\end{question}
}

\element{jpierce}{
\begin{question}{pt101tb3-Q66}
    The density of water is approximately 1000
    \begin{choices}
        \wrongchoice{grams per cubic meter (\si{\gram\per\meter\cubed}).}
      \correctchoice{kilograms per cubic meter (\si{\kilo\gram\per\meter\cubed}).}
        \wrongchoice{grams per cubic centimeter (\si{\gram\per\centi\meter\cubed}).}
        \wrongchoice{kilograms per cubic centimeter (\si{\kilo\gram\per\centi\meter\cubed}).}
        \wrongchoice{kilograms per liter (\si{\kilo\gram\per\liter}).}
    \end{choices}
\end{question}
}

\element{jpierce}{
\begin{question}{pt101tb3-Q67}
    The density of water is approximately \rule[-0.1pt]{4em}{0.1pt} per
        cubic centimeter.
    \begin{choices}
        \wrongchoice{1000 kilograms (\si{\kilo\gram})}
        \wrongchoice{100 grams (\si{\gram})}
        \wrongchoice{1 kilogram (\si{\kilo\gram})}
        \wrongchoice{10 grams (\si{\gram})}
      \correctchoice{1 gram (\si{\gram})}
    \end{choices}
\end{question}
}

\element{jpierce}{
\begin{question}{pt101tb3-Q68}
    The density of \rule[-0.1pt]{4em}{0.1pt} is
        approximately one gram per cubic centimeter.
    \begin{multicols}{2}
    \begin{choices}
      \correctchoice{water}
        \wrongchoice{air}
        \wrongchoice{aluminum}
        \wrongchoice{lead}
        \wrongchoice{mercury}
    \end{choices}
    \end{multicols}
\end{question}
}

\element{jpierce}{
\begin{question}{pt101tb3-Q69}
    One hundred cubic centimeters of water should have a mass of
        approximately \rule[-0.1pt]{4em}{0.1pt}.
    \begin{multicols}{2}
    \begin{choices}
        \wrongchoice{10 grams}
      \correctchoice{100 grams}
        \wrongchoice{1 gram}
        \wrongchoice{1 kilogram}
        \wrongchoice{100 kilograms}
    \end{choices}
    \end{multicols}
\end{question}
}

\element{jpierce}{
\begin{question}{pt101tb3-Q70}
    One thousand cubic centimeters of water should have a mass of
        approximately \rule[-0.1pt]{4em}{0.1pt}.
    \begin{multicols}{2}
    \begin{choices}
        \wrongchoice{10 grams}
        \wrongchoice{100 grams}
        \wrongchoice{1000 kilograms}
        \wrongchoice{1 gram}
      \correctchoice{1 kilogram}
    \end{choices}
    \end{multicols}
\end{question}
}

\element{jpierce}{
\begin{question}{pt101tb3-Q71}
    Ten cubic centimeters of water should have a mass of
        approximately \rule[-0.1pt]{4em}{0.1pt}.
    \begin{multicols}{2}
    \begin{choices}
        \wrongchoice{100 grams}
        \wrongchoice{1 kilogram}
      \correctchoice{10 grams}
        \wrongchoice{1000 kilograms}
        \wrongchoice{1 gram}
    \end{choices}
    \end{multicols}
\end{question}
}

\element{jpierce}{
\begin{question}{pt101tb3-Q72}
    One hundred grams of water should occupy a volume of
        about \rule[-0.1pt]{4em}{0.1pt}.
    %\begin{multicols}{2}
    \begin{choices}
        \wrongchoice{10 liters}
        \wrongchoice{100 cubic meters}
      \correctchoice{100 cubic centimeters}
        \wrongchoice{100 liters}
        \wrongchoice{one liter}
    \end{choices}
    %\end{multicols}
\end{question}
}

\element{jpierce}{
\begin{question}{pt101tb3-Q73}
    One thousand grams of water should occupy a volume of
        about \rule[-0.1pt]{4em}{0.1pt}.
    %\begin{multicols}{2}
    \begin{choices}
        \wrongchoice{100 cubic centimeters}
        \wrongchoice{1000 cubic meters}
        \wrongchoice{10 liters}
      \correctchoice{one liter}
        \wrongchoice{1000 liters}
    \end{choices}
    %\end{multicols}
\end{question}
}

%% Elastic
\element{jpierce}{
\begin{question}{pt101tb3-Q74}
    A material is said to be \rule[-0.1pt]{4em}{0.1pt} if it changes
        shape when a deforming force acts on it and returns to its
        original shape when the deforming force is removed.
    \begin{multicols}{2}
    \begin{choices}
        \wrongchoice{rigid}
      \correctchoice{elastic}
        \wrongchoice{stretchy}
        \wrongchoice{plastic}
        \wrongchoice{inelastic}
    \end{choices}
    \end{multicols}
\end{question}
}

\element{jpierce}{
\begin{question}{pt101tb3-Q75}
    A material is said to be \rule[-0.1pt]{4em}{0.1pt} if it changes shape
        when a deforming force acts on it and then does not return to its
        original shape when the deforming force is removed.
    \begin{multicols}{2}
    \begin{choices}
        \wrongchoice{elastic}
        \wrongchoice{rigid}
        \wrongchoice{stretchy}
      \correctchoice{inelastic}
        \wrongchoice{plastic}
    \end{choices}
    \end{multicols}
\end{question}
}

\element{jpierce}{
\begin{question}{pt101tb3-Q76}
    Of these, the most elastic material is:
    \begin{multicols}{2}
    \begin{choices}
        \wrongchoice{clay.}
      \correctchoice{rubber.}
        \wrongchoice{dough.}
        \wrongchoice{lead.}
        \wrongchoice{putty.}
    \end{choices}
    \end{multicols}
\end{question}
}

\element{jpierce}{
\begin{question}{pt101tb3-Q77}
    An example of an inelastic material is:
    \begin{multicols}{2}
    \begin{choices}
        \wrongchoice{a spring.}
        \wrongchoice{a golf ball.}
        \wrongchoice{rubber.}
      \correctchoice{clay.}
        \wrongchoice{a baseball.}
    \end{choices}
    \end{multicols}
\end{question}
}

\element{jpierce}{
\begin{question}{pt101tb3-Q78}
    If a spring is stretched beyond its elastic limit,
    \begin{choices}
        \wrongchoice{it will break.}
        \wrongchoice{it will snap back into its original shape.}
      \correctchoice{it will remain deformed.}
        \wrongchoice{its density will be forever changed.}
        \wrongchoice{it will still obey Hooke's Law.}
    \end{choices}
\end{question}
}

%% Topic: Hooke
\element{jpierce}{
\begin{question}{pt101tb3-Q79}
    Hooke's Law relates the
    \begin{choices}
        \wrongchoice{distance a spring stretches to the density of the spring.}
        \wrongchoice{distance a spring stretches to the mass of the spring.}
        \wrongchoice{density of a spring to the force applied to the spring.}
      \correctchoice{distance a spring stretches to the force applied to the spring.}
        \wrongchoice{density of a spring to the mass of the spring.}
    \end{choices}
\end{question}
}

\element{jpierce}{
\begin{question}{pt101tb3-Q80}
    The distance a spring stretches is related to the force
        applied to the spring by \rule[-0.1pt]{4em}{0.1pt} Law.
    \begin{multicols}{2}
    \begin{choices}
        \wrongchoice{Newton's}
      \correctchoice{Hooke's}
        \wrongchoice{Galileo's}
        \wrongchoice{Aristotle's}
        \wrongchoice{Pierce's}
    \end{choices}
    \end{multicols}
\end{question}
}

\element{jpierce}{
\begin{question}{pt101tb3-Q81}
    A mass of \SI{1}{\kilo\gram} is hung from a spring \SI{50}{\centi\meter} long,
        causing the spring to increase its length to \SI{60}{\centi\meter}.
    If the \SI{1}{\kilo\gram} mass is replaced by a \SI{2}{\kilo\gram} mass,
        how long will the spring then be (assuming the elastic limit is not reached)?
    \begin{multicols}{2}
    \begin{choices}
        \wrongchoice{\SI{120}{\centi\meter}}
      \correctchoice{\SI{70}{\centi\meter}}
        \wrongchoice{\SI{100}{\centi\meter}}
        \wrongchoice{\SI{85}{\centi\meter}}
        \wrongchoice{\SI{110}{\centi\meter}}
    \end{choices}
    \end{multicols}
\end{question}
}

\element{jpierce}{
\begin{question}{pt101tb3-Q82}
    A mass of \SI{1}{\kilo\gram} is hung from a spring \SI{50}{\centi\meter} long,
        causing the spring to increase its length to \SI{60}{\centi\meter}. 
    If the \SI{1}{\kilo\gram} mass is replaced by a \SI{3}{\kilo\gram} mass,
        how long will the spring then be (assuming the elastic limit is not reached)?
    \begin{multicols}{2}
    \begin{choices}
        \wrongchoice{\SI{180}{\centi\meter}}
      \correctchoice{\SI{80}{\centi\meter}}
        \wrongchoice{\SI{150}{\centi\meter}}
        \wrongchoice{\SI{70}{\centi\meter}}
        \wrongchoice{\SI{110}{\centi\meter}}
    \end{choices}
    \end{multicols}
\end{question}
}

\element{jpierce}{
\begin{question}{pt101tb3-Q83}
    A mass of \SI{1}{\kilo\gram} is hung from a spring \SI{50}{\centi\meter} long,
        causing the spring to increase its length to \SI{70}{\centi\meter}. 
    If the \SI{1}{\kilo\gram} mass is replaced by a \SI{2}{\kilo\gram} mass,
        how long will the spring then be (assuming the elastic limit is not reached)?
    \begin{multicols}{2}
    \begin{choices}
      \correctchoice{\SI{90}{\centi\meter}}
        \wrongchoice{\SI{80}{\centi\meter}}
        \wrongchoice{\SI{140}{\centi\meter}}
        \wrongchoice{\SI{100}{\centi\meter}}
        \wrongchoice{\SI{120}{\centi\meter}}
    \end{choices}
    \end{multicols}
\end{question}
}

\element{jpierce}{
\begin{question}{pt101tb3-Q84}
    A mass of \SI{2}{\kilo\gram} is hung from a spring \SI{50}{\centi\meter} long,
        causing the spring to increase its length to \SI{70}{\centi\meter}. 
    If the \SI{2}{\kilo\gram} mass is replaced by a \SI{1}{\kilo\gram} mass,
        how long will the spring then be (assuming the elastic limit is not reached)?
    \begin{multicols}{3}
    \begin{choices}
        \wrongchoice{\SI{120}{\centi\meter}}
        \wrongchoice{\SI{35}{\centi\meter}}
        \wrongchoice{\SI{90}{\centi\meter}}
      \correctchoice{\SI{60}{\centi\meter}}
        \wrongchoice{\SI{20}{\centi\meter}}
    \end{choices}
    \end{multicols}
\end{question}
}

\element{jpierce}{
\begin{question}{pt101tb3-Q85}
    A mass of \SI{2}{\kilo\gram} is hung from a spring \SI{50}{\centi\meter} long,
        causing the spring to increase its length to \SI{60}{\centi\meter}. 
    If the \SI{2}{\kilo\gram} mass is replaced by a \SI{1}{\kilo\gram} mass,
        how long will the spring then be (assuming the elastic limit is not reached)?
    \begin{multicols}{3}
    \begin{choices}
        \wrongchoice{\SI{110}{\centi\meter}}
        \wrongchoice{\SI{30}{\centi\meter}}
        \wrongchoice{\SI{25}{\centi\meter}}
        \wrongchoice{\SI{70}{\centi\meter}}
      \correctchoice{\SI{55}{\centi\meter}}
    \end{choices}
    \end{multicols}
\end{question}
}

%% Topic: Scaling
\element{jpierce}{
\begin{question}{pt101tb3-Q86}
    Which should cool most rapidly---a large bowl of porridge,
        a small bowl of porridge, or a medium-sized bowl of porridge?
        (Assume all three bowls were filled at the same time with equally hot porridge.)
    \begin{choices}
      \correctchoice{The small bowl, because it has the greatest surface-to-volume ratio.}
        \wrongchoice{The large bowl, because it has the greatest surface area.}
        \wrongchoice{They should all cool at the same rate because the porridge is the same in each one.}
        \wrongchoice{The large bowl, because it has the greatest volume.}
        \wrongchoice{The medium-sized bowl, because in the story it was the coldest.}
    \end{choices}
\end{question}
}

\element{jpierce}{
\begin{question}{pt101tb3-Q87}
    Elephants need large ears because
    \begin{choices}
        \wrongchoice{small ears would look silly on a body that large.}
        \wrongchoice{they provide shade for the rest of the elephant's body.}
        \wrongchoice{they use them as sails to help move their bodies around.}
      \correctchoice{they need more surface area to cool their large bodies.}
        \wrongchoice{otherwise their hearing would not be very good.}
    \end{choices}
\end{question}
}

\element{jpierce}{
\begin{question}{pt101tb3-Q88}
    When the length of each edge of a cube is doubled,
        the cube's surface area increases by a factor of:
    \begin{multicols}{3}
    \begin{choices}
        \wrongchoice{\num{6}}
        \wrongchoice{\num{8}}
        \wrongchoice{\num{2}}
      \correctchoice{\num{4}}
        \wrongchoice{\num{16}}
    \end{choices}
    \end{multicols}
\end{question}
}

\element{jpierce}{
\begin{question}{pt101tb3-Q89}
    When the length of each edge of a cube is doubled,
        the cube's volume increases by a factor of:
        %\rule[-0.1pt]{4em}{0.1pt}.
    \begin{multicols}{3}
    \begin{choices}
        \wrongchoice{\num{16}}
      \correctchoice{\num{8}}
        \wrongchoice{\num{2}}
        \wrongchoice{\num{6}}
        \wrongchoice{\num{4}}
    \end{choices}
    \end{multicols}
\end{question}
}

\element{jpierce}{
\begin{question}{pt101tb3-Q90}
    When the length of each edge of a cube is doubled,
        the cube's surface area:
    \begin{choices}
        \wrongchoice{increases by a factor of \num{2}.}
        \wrongchoice{decreases by a factor of \num{1/2}.}
        \wrongchoice{decreases by a factor of \num{1/4}.}
      \correctchoice{increases by a factor of \num{4}.}
        \wrongchoice{increases by a factor of \num{8}.}
    \end{choices}
\end{question}
}

\element{jpierce}{
\begin{question}{pt101tb3-Q91}
    When the length of each edge of a cube is doubled,
        the cube's volume:
    \begin{choices}
        \wrongchoice{increases by a factor of \num{4}.}
        \wrongchoice{decreases by a factor of \num{1/2}.}
      \correctchoice{increases by a factor of \num{8}.}
        \wrongchoice{decreases by a factor of \num{1/4}.}
        \wrongchoice{increases by a factor of \num{2}.}
    \end{choices}
\end{question}
}

\element{jpierce}{
\begin{question}{pt101tb3-Q92}
    When the length of each edge of a cube is tripled,
        the cube's surface area increases by a factor of:
    \begin{multicols}{3}
    \begin{choices}
        \wrongchoice{\num{27}}
        \wrongchoice{\num{3}}
        \wrongchoice{\num{6}}
        \wrongchoice{\num{18}}
      \correctchoice{\num{9}}
    \end{choices}
    \end{multicols}
\end{question}
}

\element{jpierce}{
\begin{question}{pt101tb3-Q93}
    When the length of each edge of a cube is tripled,
        the cube's volume increases by a factor of:
    \begin{multicols}{3}
    \begin{choices}
        \wrongchoice{\num{18}}
        \wrongchoice{\num{3}}
        \wrongchoice{\num{6}}
        \wrongchoice{\num{9}}
      \correctchoice{\num{27}}
    \end{choices}
    \end{multicols}
\end{question}
}

\element{jpierce}{
\begin{question}{pt101tb3-Q94}
    When the length of each edge of a cube is tripled,
        the cube's surface area
    \begin{choices}
        \wrongchoice{decreases by a factor of \num{1/9}.}
        \wrongchoice{decreases by a factor of \num{1/3}.}
        \wrongchoice{increases by a factor of \num{27}.}
        \wrongchoice{increases by a factor of \num{3}.}
      \correctchoice{increases by a factor of \num{9}.}
    \end{choices}
\end{question}
}

\element{jpierce}{
\begin{question}{pt101tb3-Q95}
    When the length of each edge of a cube is tripled,
        the cube's volume
    \begin{choices}
        \wrongchoice{increases by a factor of \num{9}.}
        \wrongchoice{decreases by a factor of \num{1/3}.}
        \wrongchoice{decreases by a factor of \num{1/9}.}
        \wrongchoice{increases by a factor of \num{3}.}
      \correctchoice{increases by a factor of \num{27}.}
    \end{choices}
\end{question}
}

\element{jpierce}{
\begin{question}{pt101tb3-Q96}
    When the length of each edge of a cube is doubled,
        the cube's surface-to-volume ratio
    \begin{choices}
        \wrongchoice{increases by a factor of \num{8}.}
        \wrongchoice{increases by a factor of \num{2}.}
      \correctchoice{decreases by a factor of \num{1/2}.}
        \wrongchoice{increases by a factor of \num{4}.}
        \wrongchoice{decreases by a factor of \num{1/4}.}
    \end{choices}
\end{question}
}

\element{jpierce}{
\begin{question}{pt101tb3-Q97}
    When the length of each edge of a cube is tripled,
        the cube's surface-to-volume ratio
    \begin{choices}
        \wrongchoice{decreases by a factor of \num{1/9}.}
        \wrongchoice{increases by a factor of \num{3}.}
      \correctchoice{decreases by a factor of \num{1/3}.}
        \wrongchoice{increases by a factor of \num{27}.}
        \wrongchoice{increases by a factor of \num{9}.}
    \end{choices}
\end{question}
}


%% Topic: Tension Compression
\element{jpierce}{
\begin{question}{pt101tb3-Q98}
    A horizontal steel beam is clamped at one end and a weight is placed
        on the other end, as shown. 
    Describe the stresses on the beam at the three points indicated ($a$, $b$, and $c$)
    %% NOTE: contains graphic
    \begin{choices}
        \wrongchoice{tension at $a$, compression at $c$, neutral layer at $b$}
        \wrongchoice{tension at $a$, compression at $b$, neutral layer at $c$}
        \wrongchoice{tension at $b$, compression at $a$ and $c$}
      \correctchoice{tension at $c$, compression at $a$, neutral layer at $b$}
        \wrongchoice{tension at $a$ and $c$, compression at $b$}
    \end{choices}
\end{question}
}

\element{jpierce}{
\begin{question}{pt101tb3-Q99}
    A horizontal steel beam is supported at each end as shown,
        and a weight is placed in the middle. 
    Describe the stresses at the three points indicated in the beam ($a$, $b$, and $c$).
    %% NOTE: contains graphic
    \begin{choices}
        \wrongchoice{compression at $a$, $b$, and $c$}
      \correctchoice{compression at $a$, tension at $c$, neutral layer at $b$}
        \wrongchoice{compression at $b$, tension at $a$ and $c$}
        \wrongchoice{compression at $c$, tension at $a$, neutral layer at $b$}
        \wrongchoice{compression at $a$ and $b$, tension at $c$}
    \end{choices}
\end{question}
}

\element{jpierce}{
\begin{question}{pt101tb3-Q100}
    An I-beam is relatively thin in the middle of its cross-section because:
    \begin{choices}
        \wrongchoice{this is where most of the tension forces are concentrated.}
      \correctchoice{relatively few forces are applied to this part of the beam.}
        \wrongchoice{this is where most of the compression forces are concentrated.}
        \wrongchoice{this is the part of the beam that needs to flex the most.}
        \wrongchoice{this is the part of the beam that needs to be the most rigid.}
    \end{choices}
\end{question}
}

\element{jpierce}{
\begin{question}{pt101tb3-Q101}
    The curve that gives maximum strength to an arch that supports only its own weight is called:
    \begin{multicols}{2}
    \begin{choices}
        \wrongchoice{a hyperbola.}
      \correctchoice{a catenary.}
        \wrongchoice{a semicircle.}
        \wrongchoice{a parabola.}
        \wrongchoice{an ellipse.}
    \end{choices}
    \end{multicols}
\end{question}
}

\element{jpierce}{
\begin{question}{pt101tb3-Q102}
    Stone doorways are often arched because:
    \begin{choices}
        \wrongchoice{stones with the shapes used in arches are most easily found in nature.}
        \wrongchoice{stone masons do not know how to build any other kind.}
        \wrongchoice{stone breaks more easily under compression than tension.}
      \correctchoice{stone breaks more easily under tension than compression.}
        \wrongchoice{stones with the shapes used in arches are easier to fabricate.}
    \end{choices}
\end{question}
}

\element{jpierce}{
\begin{question}{pt101tb3-Q103}
    Stone doorways are often arched because:
    \begin{choices}
        \wrongchoice{stones with the shapes used in arches are easier to fabricate.}
      \correctchoice{stone can withstand compression forces better than tension.}
        \wrongchoice{stone can withstand tension forces better than compression.}
        \wrongchoice{stone masons do not know how to build any other kind.}
        \wrongchoice{stones with the shapes used in arches are most easily found in nature.}
    \end{choices}
\end{question}
}

\element{jpierce}{
\begin{question}{pt101tb3-Q104}
    In a catenary curve such as is found in the St. Louis arch,
    \begin{choices}
        \wrongchoice{the compression forces produced by the weight of the material used act vertically.}
        \wrongchoice{the tension forces produced by the weight of the material used act horizontally.}
      \correctchoice{the compression forces produced by the weight of the material used act parallel to the curve.}
        \wrongchoice{the tension forces produced by the weight of the material used act parallel to the curve.}
        \wrongchoice{the compression forces produced by the weight of the material used act horizontally.}
    \end{choices}
\end{question}
}

\element{jpierce}{
\begin{question}{pt101tb3-Q105}
    The weight of a dome produces:
    \begin{choices}
      \correctchoice{compression forces parallel to the curve of the dome.}
        \wrongchoice{tension forces acting vertically.}
        \wrongchoice{tension forces parallel to the curve of the dome.}
        \wrongchoice{compression forces perpendicular to the curve of the dome.}
        \wrongchoice{tension forces acting horizontally.}
    \end{choices}
\end{question}
}

%% Topic: Buoyancy
\element{jpierce}{
\begin{question}{pt101tb3-Q106}
    A completely submerged object in a container of liquid always:
    \begin{choices}
        \wrongchoice{sinks to the bottom of the container.}
        \wrongchoice{remains at the same level in the container.}
        \wrongchoice{displaces a mass of liquid equal to its own mass.}
        \wrongchoice{floats to the top of the container.}
      \correctchoice{displaces a volume of liquid equal to its own volume.}
    \end{choices}
\end{question}
}

\element{jpierce}{
\begin{question}{pt101tb3-Q107}
    If the weight of a submerged object is less than the buoyant force on the object,
    \begin{choices}
        \wrongchoice{the object will be crushed by the liquid.}
      \correctchoice{the object will rise to the surface and float.}
        \wrongchoice{the object will remain at its present level.}
        \wrongchoice{the object will sink.}
        \wrongchoice{the object will expand.}
    \end{choices}
\end{question}
}

\element{jpierce}{
\begin{question}{pt101tb3-Q108}
    If the weight of a submerged object is greater than the buoyant force on the object,
    \begin{choices}
      \correctchoice{the object will sink.}
        \wrongchoice{the object will be crushed by the liquid.}
        \wrongchoice{the object will remain at its present level.}
        \wrongchoice{the object will expand.}
        \wrongchoice{the object will rise to the surface and float.}
    \end{choices}
\end{question}
}

\element{jpierce}{
\begin{question}{pt101tb3-Q109}
    The buoyant force:
    \begin{choices}
        \wrongchoice{is the net downward force of a submerged object acting on the surrounding liquid.}
        \wrongchoice{depends on the density of the submerged object.}
        \wrongchoice{is the difference between a submerged object's weight and the weight of an equal mass of water.}
      \correctchoice{is the net upward force of the surrounding liquid acting on a submerged object.}
        \wrongchoice{is the force of gravity acting on a submerged object.}
    \end{choices}
\end{question}
}

\element{jpierce}{
\begin{question}{pt101tb3-Q110}
    Archimedes' Principle states that an immersed object is buoyed up by a force equal to the:
    \begin{choices}
      \correctchoice{weight of the fluid it displaces.}
        \wrongchoice{total pressure on the object.}
        \wrongchoice{difference between the weight of the object and the weight of the fluid it displaces.}
        \wrongchoice{weight of the object.}
        \wrongchoice{centrifugal force acting on the object.}
    \end{choices}
\end{question}
}

\element{jpierce}{
\begin{question}{pt101tb3-Q111}
    If an object is less dense than the fluid in which it is immersed,
    \begin{choices}
        \wrongchoice{the object will sink.}
        \wrongchoice{the object will remain at its present level.}
      \correctchoice{the object will rise to the surface and float.}
        \wrongchoice{the object will expand.}
        \wrongchoice{the object will be crushed by the liquid.}
    \end{choices}
\end{question}
}

\element{jpierce}{
\begin{question}{pt101tb3-Q112}
    If an object is more dense than the fluid in which it is immersed,
    \begin{choices}
        \wrongchoice{the object will remain at its present level.}
      \correctchoice{the object will sink.}
        \wrongchoice{the object will rise to the surface and float.}
        \wrongchoice{the object will be crushed by the liquid.}
        \wrongchoice{the object will expand.}
    \end{choices}
\end{question}
}

\element{jpierce}{
\begin{question}{pt101tb3-Q113}
    A floating object:
    \begin{choices}
      \correctchoice{displaces a weight of fluid equal to its own weight.}
        \wrongchoice{displaces a volume of fluid equal to its own volume.}
        \wrongchoice{has a buoyant force less than its own weight.}
        \wrongchoice{has no weight.}
        \wrongchoice{has a buoyant force greater than its own weight.}
    \end{choices}
\end{question}
}

\element{jpierce}{
\begin{question}{pt101tb3-Q114}
    A ship floats higher in salt water than it does in fresh water because:
    \begin{choices}
        \wrongchoice{salt makes the water more rigid, and the ship does not sink in as far.}
        \wrongchoice{salt water is denser, and more displacement is needed to achieve the same buoyant force.}
        \wrongchoice{salt water is less dense, and more displacement is needed to achieve the same buoyant force.}
        \wrongchoice{salt water is less dense, and less displacement is needed to achieve the same buoyant force.}
      \correctchoice{salt water is denser, and less displacement is needed to achieve the same buoyant force.}
    \end{choices}
\end{question}
}

\element{jpierce}{
\begin{question}{pt101tb3-Q115}
    A ship does not float as high in fresh water as it does in salt water because:
    \begin{choices}
        \wrongchoice{salt water is denser, and more displacement is needed to achieve the same buoyant force.}
        \wrongchoice{salt water is less dense, and less displacement is needed to achieve the same buoyant force.}
      \correctchoice{salt water is less dense, and more displacement is needed to achieve the same buoyant force.}
        \wrongchoice{salt makes the water more rigid, and the ship does not sink in as far.}
        \wrongchoice{salt water is denser, and less displacement is needed to achieve the same buoyant force.}
    \end{choices}
\end{question}
}

\element{jpierce}{
\begin{question}{pt101tb3-Q116}
    A fish can swim horizontally in water if:
    \begin{choices}
        \wrongchoice{its buoyant force is greater than its weight.}
        \wrongchoice{it displaces a weight of water less than its own weight.}
        \wrongchoice{its density is less than that of water.}
      \correctchoice{its buoyant force is equal to its weight.}
        \wrongchoice{its buoyant force is less than its weight.}
    \end{choices}
\end{question}
}

\element{jpierce}{
\begin{question}{pt101tb3-Q117}
    If the weight of a submerged object is equal to the buoyant force on the object,
    \begin{choices}
        \wrongchoice{the object will rise to the surface and float.}
        \wrongchoice{the object will sink.}
        \wrongchoice{the object will be crushed by the liquid.}
        \wrongchoice{the object will expand.}
      \correctchoice{the object will remain at its present level.}
    \end{choices}
\end{question}
}

\element{jpierce}{
\begin{question}{pt101tb3-Q118}
    The buoyant force on a block of wood floating in water:
    \begin{choices}
        \wrongchoice{is equal to the weight of a volume of water with the same volume as the wood.}
      \correctchoice{is equal to the weight of the wood.}
        \wrongchoice{is greater than the weight of the wood.}
        \wrongchoice{cannot be calculated because the block is not completely submerged.}
        \wrongchoice{is less than the weight of the wood.}
    \end{choices}
\end{question}
}

\element{jpierce}{
\begin{question}{pt101tb3-Q119}
    An object with a mass of \SI{1}{\kilo\gram} displaces \SI{0.6}{\kilo\gram} of water. 
    Which of the following is true?
    \begin{choices}
        \wrongchoice{The density of this object is less than that of water.}
        \wrongchoice{The buoyant force on this object is \SI{10}{\newton}.}
        \wrongchoice{This object will not sink in water.}
      \correctchoice{The buoyant force on this object is \SI{6}{\newton}.}
        \wrongchoice{The buoyant force on this object is \SI{4}{\newton}.}
    \end{choices}
\end{question}
}

\element{jpierce}{
\begin{question}{pt101tb3-Q120}
    An object with a mass of \SI{1}{\kilo\gram} displaces \SI{0.6}{\kilo\gram} of water. 
    Which of the following is true?
    \begin{choices}
      \correctchoice{This object will sink in water.}
        \wrongchoice{The buoyant force on this object is \SI{4}{\newton}.}
        \wrongchoice{The density of this object is less than that of water.}
        \wrongchoice{The buoyant force on this object is \SI{10}{\newton}.}
        \wrongchoice{The weight of this object is \SI{6}{\newton}.}
    \end{choices}
\end{question}
}

\element{jpierce}{
\begin{question}{pt101tb3-Q121}
    An object with a mass of \SI{1}{\kilo\gram} displaces \SI{0.6}{\kilo\gram} of water. 
    Which of the following is true?
    \begin{choices}
      \correctchoice{The density of this object is greater than that of water.}
        \wrongchoice{The buoyant force on this object is \SI{4}{\newton}.}
        \wrongchoice{The weight of this object is \SI{6}{\newton}.}
        \wrongchoice{This object will not sink in water.}
        \wrongchoice{The buoyant force on this object is \SI{10}{\newton}.}
    \end{choices}
\end{question}
}

\element{jpierce}{
\begin{question}{pt101tb3-Q122}
    An object with a mass of \SI{1}{\kilo\gram} displaces \SI{0.6}{\kilo\gram} of water. 
    Which of the following is true?
    \begin{choices}
        \wrongchoice{The weight of this object is \SI{6}{\newton}.}
      \correctchoice{The weight of this object is \SI{10}{\newton}.}
        \wrongchoice{The weight of this object is \SI{4}{\newton}.}
        \wrongchoice{The buoyant force on this object is \SI{10}{\newton}.}
        \wrongchoice{The buoyant force on this object is \SI{4}{\newton}.}
    \end{choices}
\end{question}
}

\element{jpierce}{
\begin{question}{pt101tb3-Q123}
    An object with a mass of \SI{1}{\kilo\gram} displaces \SI{0.7}{\kilo\gram} of water. 
    Which of the following is true?
    \begin{choices}
        \wrongchoice{The buoyant force on this object is \SI{10}{\newton}.}
        \wrongchoice{The density of this object is less than that of water.}
      \correctchoice{The buoyant force on this object is \SI{7}{\newton}.}
        \wrongchoice{The buoyant force on this object is \SI{3}{\newton}.}
        \wrongchoice{This object will not sink in water.}
    \end{choices}
\end{question}
}

\element{jpierce}{
\begin{question}{pt101tb3-Q124}
    An object with a mass of \SI{1}{\kilo\gram} displaces \SI{0.7}{\kilo\gram} of water. 
    Which of the following is true?
    \begin{choices}
        \wrongchoice{The weight of this object is \SI{7}{\newton}.}
        \wrongchoice{The density of this object is less than that of water.}
      \correctchoice{This object will sink in water.}
        \wrongchoice{The buoyant force on this object is \SI{3}{\newton}.}
        \wrongchoice{The buoyant force on this object is \SI{10}{\newton}.}
    \end{choices}
\end{question}
}

\element{jpierce}{
\begin{question}{pt101tb3-Q125}
    An object with a mass of \SI{1}{\kilo\gram} displaces \SI{0.7}{\kilo\gram} of water. 
    Which of the following is true?
    \begin{choices}
        \wrongchoice{The buoyant force on this object is \SI{10}{\newton}.}
      \correctchoice{The density of this object is greater than that of water.}
        \wrongchoice{The buoyant force on this object is \SI{3}{\newton}.}
        \wrongchoice{This object will not sink in water.}
        \wrongchoice{The weight of this object is \SI{7}{\newton}.}
    \end{choices}
\end{question}
}

\element{jpierce}{
\begin{question}{pt101tb3-Q126}
    An object with a mass of \SI{1}{\kilo\gram} displaces \SI{0.7}{\kilo\gram} of water. 
    Which of the following is true?
    \begin{choices}
        \wrongchoice{The buoyant force on this object is \SI{10}{\newton}.}
        \wrongchoice{The weight of this object is \SI{3}{\newton}.}
        \wrongchoice{The buoyant force on this object is \SI{3}{\newton}.}
      \correctchoice{The weight of this object is \SI{10}{\newton}.}
        \wrongchoice{The weight of this object is \SI{7}{\newton}.}
    \end{choices}
\end{question}
}

\element{jpierce}{
\begin{question}{pt101tb3-Q127}
    An object with a mass of \SI{1}{\kilo\gram} displaces \SI{600}{\milli\liter} of water. 
    Which of the following is true?
    \begin{choices}
        \wrongchoice{This object will not sink in water.}
        \wrongchoice{The buoyant force on this object is \SI{10}{\newton}.}
      \correctchoice{The buoyant force on this object is \SI{6}{\newton}.}
        \wrongchoice{The buoyant force on this object is \SI{4}{\newton}.}
        \wrongchoice{The density of this object is less than that of water.}
    \end{choices}
\end{question}
}

\element{jpierce}{
\begin{question}{pt101tb3-Q128}
    An object with a mass of \SI{1}{\kilo\gram} displaces \SI{600}{\milli\liter} of water. 
    Which of the following is true?
    \begin{choices}
        \wrongchoice{The buoyant force on this object is \SI{10}{\newton}.}
        \wrongchoice{The weight of this object is \SI{6}{\newton}.}
        \wrongchoice{The density of this object is less than that of water.}
        \wrongchoice{The buoyant force on this object is \SI{4}{\newton}.}
      \correctchoice{This object will sink in water.}
    \end{choices}
\end{question}
}

\element{jpierce}{
\begin{question}{pt101tb3-Q129}
    An object with a mass of \SI{1}{\kilo\gram} displaces \SI{600}{\milli\liter} of water. 
    Which of the following is true?
    \begin{choices}
        \wrongchoice{The buoyant force on this object is \SI{4}{\newton}.}
        \wrongchoice{The weight of this object is \SI{6}{\newton}.}
        \wrongchoice{This object will not sink in water.}
      \correctchoice{The density of this object is greater than that of water.}
        \wrongchoice{The buoyant force on this object is \SI{10}{\newton}.}
    \end{choices}
\end{question}
}

\element{jpierce}{
\begin{question}{pt101tb3-Q130}
    An object with a mass of \SI{1}{\kilo\gram} displaces \SI{600}{\milli\liter} of water. 
    Which of the following is true?
    \begin{choices}
        \wrongchoice{The weight of this object is \SI{4}{\newton}.}
        \wrongchoice{The buoyant force on this object is \SI{4}{\newton}.}
        \wrongchoice{The weight of this object is \SI{6}{\newton}.}
        \wrongchoice{The buoyant force on this object is \SI{10}{\newton}.}
      \correctchoice{The weight of this object is \SI{10}{\newton}.}
    \end{choices}
\end{question}
}

\element{jpierce}{
\begin{question}{pt101tb3-Q131}
    An object with a mass of \SI{1}{\kilo\gram} displaces \SI{700}{\milli\liter} of water. 
    Which of the following is true?
    \begin{choices}
        \wrongchoice{This object will not sink in water.}
      \correctchoice{The buoyant force on this object is \SI{7}{\newton}.}
        \wrongchoice{The buoyant force on this object is \SI{3}{\newton}.}
        \wrongchoice{The density of this object is less than that of water.}
        \wrongchoice{The buoyant force on this object is \SI{10}{\newton}.}
    \end{choices}
\end{question}
}

\element{jpierce}{
\begin{question}{pt101tb3-Q132}
    An object with a mass of \SI{1}{\kilo\gram} displaces \SI{700}{\milli\liter} of water. 
    Which of the following is true?
    \begin{choices}
        \wrongchoice{The buoyant force on this object is \SI{10}{\newton}.}
      \correctchoice{This object will sink in water.}
        \wrongchoice{The weight of this object is \SI{7}{\newton}.}
        \wrongchoice{The density of this object is less than that of water.}
        \wrongchoice{The buoyant force on this object is \SI{3}{\newton}.}
    \end{choices}
\end{question}
}

\element{jpierce}{
\begin{question}{pt101tb3-Q133}
    An object with a mass of \SI{1}{\kilo\gram} displaces \SI{700}{\milli\liter} of water. 
    Which of the following is true?
    \begin{choices}
        \wrongchoice{The buoyant force on this object is \SI{10}{\newton}.}
      \correctchoice{The density of this object is greater than that of water.}
        \wrongchoice{The weight of this object is \SI{7}{\newton}.}
        \wrongchoice{This object will not sink in water.}
        \wrongchoice{The buoyant force on this object is \SI{3}{\newton}.}
    \end{choices}
\end{question}
}

\element{jpierce}{
\begin{question}{pt101tb3-Q134}
    An object with a mass of \SI{1}{\kilo\gram} displaces \SI{700}{\milli\liter} of water. 
    Which of the following is true?
    \begin{choices}
        \wrongchoice{The weight of this object is \SI{3}{\newton}.}
        \wrongchoice{The buoyant force on this object is \SI{3}{\newton}.}
      \correctchoice{The weight of this object is \SI{10}{\newton}.}
        \wrongchoice{The weight of this object is \SI{7}{\newton}.}
        \wrongchoice{The buoyant force on this object is \SI{10}{\newton}.}
    \end{choices}
\end{question}
}

%% Topic: Pressure
\element{jpierce}{
\begin{question}{pt101tb3-Q135}
    Pressure is defined as the \rule[-0.1pt]{4em}{0.1pt} per unit \rule[-0.1pt]{4em}{0.1pt}.
    \begin{multicols}{2}
    \begin{choices}
        \wrongchoice{mass; length}
        \wrongchoice{mass; volume}
      \correctchoice{force; area}
        \wrongchoice{force; mass}
        \wrongchoice{force; volume}
    \end{choices}
    \end{multicols}
\end{question}
}

\element{jpierce}{
\begin{question}{pt101tb3-Q136}
    The water pressure in a lake behind a dam depends on
    \begin{choices}
        \wrongchoice{the number of fish in the lake.}
        \wrongchoice{the distance from the dam at which the pressure is measured.}
        \wrongchoice{the surface area of the lake.}
        \wrongchoice{the volume of lake water behind the dam.}
      \correctchoice{the depth below the surface at which the pressure is measured.}
    \end{choices}
\end{question}
}

\element{jpierce}{
\begin{question}{pt101tb3-Q137}
    Pressure in a liquid is equal to
    \begin{choices}
        \wrongchoice{the mass density of the liquid times the depth.}
      \correctchoice{the weight density of the liquid times the depth.}
        \wrongchoice{the mass density of the liquid times the volume.}
        \wrongchoice{the weight density of the liquid times the volume.}
        \wrongchoice{the weight density of the liquid times the surface area.}
    \end{choices}
\end{question}
}

\element{jpierce}{
\begin{question}{pt101tb3-Q138}
    At any given point in the middle of a glass of water, the water pressure will be
    \begin{choices}
        \wrongchoice{greatest in the downward direction.}
        \wrongchoice{equal to the pressure at all other points in the glass of water.}
        \wrongchoice{greatest in the sideways direction.}
        \wrongchoice{greatest in the upward direction.}
      \correctchoice{equal in all directions.}
    \end{choices}
\end{question}
}

\element{jpierce}{
\begin{question}{pt101tb3-Q139}
    Water pressure acts \rule[-0.1pt]{4em}{0.1pt} the sides of a
        container and \rule[-0.1pt]{4em}{0.1pt} with increasing depth.
    \begin{choices}
      \correctchoice{perpendicular to; increases}
        \wrongchoice{perpendicular to; decreases}
        \wrongchoice{parallel to; increases}
        \wrongchoice{parallel to; decreases}
        \wrongchoice{perpendicular to; remains constant}
    \end{choices}
\end{question}
}

\element{jpierce}{
\begin{question}{pt101tb3-Q140}
    Pressure is defined as the \rule[-0.1pt]{4em}{0.1pt} per unit area.
    \begin{multicols}{2}
    \begin{choices}
        \wrongchoice{mass}
        \wrongchoice{work}
        \wrongchoice{volume}
      \correctchoice{force}
        \wrongchoice{density}
    \end{choices}
    \end{multicols}
\end{question}
}

\element{jpierce}{
\begin{question}{pt101tb3-Q141}
    Four different containers are filled to the same depth with water, as shown. 
    %% NOTE: graphics
    At the bottom of which container will the pressure be the greatest?
    \begin{multicols}{2}
    \begin{choices}
        \wrongchoice{K}
        \wrongchoice{M}
        \wrongchoice{Q}
        \wrongchoice{S}
      \correctchoice{The pressure will be \emph{the same} at the bottom of each container.}
    \end{choices}
    \end{multicols}
\end{question}
}

\element{jpierce}{
\begin{question}{pt101tb3-Q142}
    How does the pressure at point $A$ compare to the pressure at point $B$
        in this water-filled container?
    %% NOTE: graphics
    \begin{choices}
        \wrongchoice{It is greater at $A$ because $A$ is in the middle and feels pressure from all directions.}
        \wrongchoice{It is greater at $A$ because friction with the wall reduces the pressure at $B$.}
      \correctchoice{It is the same at $A$ and $B$ because they are at the same depth.}
        \wrongchoice{It is greater at $B$ because the wall of the container exerts an extra force on the water there.}
        \wrongchoice{It is the same at $A$ and $B$ because all points in the water have the same pressure.}
    \end{choices}
\end{question}
}

\element{jpierce}{
\begin{question}{pt101tb3-Q143}
    How does the pressure at point $A$ compare to the pressure at point $B$
        in this water-filled container?
    %% NOTE: graphics
    \begin{choices}
      \correctchoice{It is greater at $B$ because $B$ is at a greater depth.}
        \wrongchoice{It is greater at $A$ because $A$ is in the middle and feels pressure from all directions.}
        \wrongchoice{It is the same at $A$ and $B$ because they are equidistant from the walls of the container.}
        \wrongchoice{It is greater at $B$ because the bottom of the container exerts an extra force on the water there.}
        \wrongchoice{It is the same at $A$ and $B$ because all points in the water have the same pressure.}
    \end{choices}
\end{question}
}

\element{jpierce}{
\begin{question}{pt101tb3-Q144}
    How does the pressure at point $A$ compare to the pressure at point $B$
        in this water-filled container?
    %% NOTE: graphics
    \begin{choices}
        \wrongchoice{It is the same at $A$ and $B$ because the side and the bottom both push equally on the water.}
      \correctchoice{It is greater at $B$ because $B$ is at a greater depth.}
        \wrongchoice{It is greater at $A$ because the side wall of the container exerts an extra force on the water there.}
        \wrongchoice{It is the same at $A$ and $B$ because all points in the water have the same pressure.}
        \wrongchoice{It is greater at $B$ because water does not push sideways – only down.}
    \end{choices}
\end{question}
}

\element{jpierce}{
\begin{question}{pt101tb3-Q145}
    Alcohol has a density \SI{79}{\percent} that of water;
        the pressure at the bottom of a glass full of alcohol will be
    \begin{choices}
        \wrongchoice{\SI{79}{\percent} less than at the bottom of a similar glass filled with water.}
        \wrongchoice{the same as at the bottom of a similar glass filled with water.}
      \correctchoice{\SI{21}{\percent} less than at the bottom of a similar glass filled with water.}
        \wrongchoice{\SI{21}{\percent} greater than at the bottom of a similar glass filled with water.}
        \wrongchoice{\SI{79}{\percent} greater than at the bottom of a similar glass filled with water.}
    \end{choices}
\end{question}
}

\element{jpierce}{
\begin{question}{pt101tb3-Q146}
    Alcohol has a density \SI{79}{\percent} that of water;
        the pressure at the top of a glass full of alcohol will be
    \begin{choices}
        \wrongchoice{\SI{21}{\percent} greater than at the bottom of a similar glass filled with water.}
        \wrongchoice{\SI{79}{\percent} greater than at the bottom of a similar glass filled with water.}
      \correctchoice{the same as at the bottom of a similar glass filled with water.}
        \wrongchoice{\SI{79}{\percent} less than at the bottom of a similar glass filled with water.}
        \wrongchoice{\SI{21}{\percent} less than at the bottom of a similar glass filled with water.}
    \end{choices}
\end{question}
}

%% Topic: Air Pressure
\element{jpierce}{
\begin{question}{pt101tb3-Q147}
    Atmospheric pressure
    \begin{choices}
        \wrongchoice{acts every direction except upwards.}
        \wrongchoice{acts only sideways.}
        \wrongchoice{acts only downward.}
      \correctchoice{acts in all directions.}
        \wrongchoice{acts only upwards.}
    \end{choices}
\end{question}
}

\element{jpierce}{
\begin{question}{pt101tb3-Q148}
    Water rises in a drinking straw when you suck on it because
    \begin{choices}
        \wrongchoice{the air pressure inside the straw is greater than the air pressure on the water surface.}
        \wrongchoice{the air pressure inside the straw is equal to the air pressure on the water surface.}
        \wrongchoice{a gas always attempts to fill a vacuum.}
        \wrongchoice{a liquid always attempts to fill a vacuum.}
      \correctchoice{the air pressure inside the straw is less than the air pressure on the water surface.}
    \end{choices}
\end{question}
}

\element{jpierce}{
\begin{question}{pt101tb3-Q149}
    When air is removed from a metal can by a vacuum pump,
        the can buckles inwards and is crushed. 
    This occurs because
    \begin{choices}
      \correctchoice{the air pressure on the outside of the can is greater than the air pressure on the inside of the can.}
        \wrongchoice{the loss of air molecules from inside the can weakens the metal.}
        \wrongchoice{the air pressure on the inside of the can is greater than the air pressure on the outside of the can.}
        \wrongchoice{of Bernoulli's principle.}
        \wrongchoice{the opposite sides of the empty can strongly attract each other.}
    \end{choices}
\end{question}
}

\element{jpierce}{
\begin{question}{pt101tb3-Q150}
    A barometer made with mercury will be about \SI{30}{\inch} high while a
        barometer made with water will be about \SI{34}{\foot} high.
    This is because mercury and water have different
    \begin{choices}
        \wrongchoice{accelerations.}
      \correctchoice{densities.}
        \wrongchoice{volumes.}
        \wrongchoice{colors.}
        \wrongchoice{potential energies.}
    \end{choices}
\end{question}
}

\element{jpierce}{
\begin{question}{pt101tb3-Q151}
    The air pressure at the top of a mountain is \rule[-0.1pt]{4em}{0.1pt} the
        air pressure at sea level because \rule[-0.1pt]{4em}{0.1pt}.
    \begin{choices}
        \wrongchoice{greater than; the air on the mountain top can press from all sides, rather than just from above.}
        \wrongchoice{equal to; the air is in contact with the earth in both locations}
        \wrongchoice{greater than; the air has more potential energy at the top of the mountain}
        \wrongchoice{less than; gravity is not as strong at the top of the mountain}
      \correctchoice{less than; there is less air above the mountain top}
    \end{choices}
\end{question}
}

%% Topic: Bernoulli
\element{jpierce}{
\begin{question}{pt101tb3-Q152}
    Bernoulli's principle says that when the speed of a fluid increases,
    \begin{choices}
      \correctchoice{pressure in the fluid decreases.}
        \wrongchoice{the fluid does more work.}
        \wrongchoice{gravitational potential energy of the fluid increases.}
        \wrongchoice{pressure in the fluid increases.}
        \wrongchoice{kinetic energy of the fluid decreases.}
    \end{choices}
\end{question}
}

\element{jpierce}{
\begin{question}{pt101tb3-Q153}
    Bernoulli's principle explains why
    \begin{choices}
        \wrongchoice{a hot air balloon rises.}
        \wrongchoice{dead fish float.}
        \wrongchoice{liquid rises in a drinking straw.}
        \wrongchoice{submarines can remain submerged.}
      \correctchoice{airplanes fly.}
    \end{choices}
\end{question}
}

\element{jpierce}{
\begin{question}{pt101tb3-Q154}
    An airplane wing is shaped such that
    \begin{choices}
        \wrongchoice{air flows more rapidly across the bottom than over the top of the wing.}
      \correctchoice{air flows more rapidly over the top than across the bottom of the wing.}
        \wrongchoice{air does not flow over the top of the wing.}
        \wrongchoice{air flows at the same rate across the bottom and over the top of the wing.}
        \wrongchoice{air does not flow across the bottom of the wing.}
    \end{choices}
\end{question}
}

\element{jpierce}{
\begin{question}{pt101tb3-Q155}
    Bernoulli's principle says that when the speed of a fluid decreases,
    \begin{choices}
        \wrongchoice{kinetic energy of the fluid increases.}
      \correctchoice{pressure in the fluid increases.}
        \wrongchoice{gravitational potential energy of the fluid decreases.}
        \wrongchoice{pressure in the fluid decreases.}
        \wrongchoice{the fluid does less work.}
    \end{choices}
\end{question}
}

%% Topic: Boyle
\element{jpierce}{
\begin{question}{pt101tb3-Q156}
    Boyle's Law says that if the temperature of a given mass of gas does
        not change, the \rule[-0.1pt]{4em}{0.1pt} will be constant.
    \begin{choices}
        \wrongchoice{density}
        \wrongchoice{sum of the volume and the pressure}
      \correctchoice{product of the pressure and the volume}
        \wrongchoice{ratio of the pressure to the volume}
        \wrongchoice{ratio of the volume to the pressure}
    \end{choices}
\end{question}
}

\element{jpierce}{
\begin{question}{pt101tb3-Q157}
    In order to increase the pressure in an automobile tire, one normally
    \begin{choices}
        \wrongchoice{increases the temperature of the tire.}
        \wrongchoice{decreases the volume of the tire.}
        \wrongchoice{decreases the surface area of the tire.}
        \wrongchoice{decreases the number of air molecules in the tire.}
      \correctchoice{increases the density of air in the tire.}
    \end{choices}
\end{question}
}

\element{jpierce}{
\begin{question}{pt101tb3-Q158}
    In order to decrease the pressure in an automobile tire, one normally
    \begin{choices}
      \correctchoice{decreases the number of air molecules in the tire.}
        \wrongchoice{increases the density of air in the tire.}
        \wrongchoice{increases the volume of the tire.}
        \wrongchoice{decreases the temperature of the tire.}
        \wrongchoice{decreases the surface area of the tire.}
    \end{choices}
\end{question}
}

\element{jpierce}{
\begin{question}{pt101tb3-Q159}
    Two identical weights rest on a movable piston inside a cylinder,
        supported by the pressure of the trapped air below.
    If a third identical weight is placed on top of the original weights,
        what will happen?
    \begin{choices}
        \wrongchoice{nothing}
        \wrongchoice{The volume of trapped air will increase.}
        \wrongchoice{The number of trapped air molecules will increase.}
      \correctchoice{The volume of trapped air will decrease.}
        \wrongchoice{The number of trapped air molecules will decrease.}
    \end{choices}
\end{question}
}

\element{jpierce}{
\begin{question}{pt101tb3-Q160}
    Two identical weights rest on a movable piston inside a cylinder,
        supported by the pressure of the trapped air below. 
    If a one of the weights is removed,
        what will happen?
    \begin{choices}
        \wrongchoice{nothing}
        \wrongchoice{The volume of trapped air will decrease.}
        \wrongchoice{The number of trapped air molecules will decrease.}
      \correctchoice{The volume of trapped air will increase.}
        \wrongchoice{The number of trapped air molecules will increase.}
    \end{choices}
\end{question}
}

\element{jpierce}{
\begin{question}{pt101tb3-Q161}
    A Super Soaker squirt gun is filled with water and pumped with air such
        that pulling the trigger causes a stream of water to be ``shot'' from the gun.
    However, if the trigger is pulled continuously,
        the water stream gradually weakens and finally stops;
        this is best explained by:
    \begin{choices}
        \wrongchoice{Newton's Third Law}
        \wrongchoice{Pascal's Principle}
      \correctchoice{Boyle's Law}
        \wrongchoice{Bernoulli's Principle}
        \wrongchoice{Murphy's Law}
    \end{choices}
\end{question}
}

\element{jpierce}{
\begin{question}{pt101tb3-Q162}
    According to Boyle's Law, if the volume occupied by a certain gas is doubled,
    \begin{choices}
        \wrongchoice{the number of atoms in the gas will be halved.}
        \wrongchoice{the pressure of the gas will remain constant.}
        \wrongchoice{the pressure of the gas will be doubled.}
        \wrongchoice{the pressure of the gas will be quadrupled.}
      \correctchoice{the pressure of the gas will be halved.}
    \end{choices}
\end{question}
}

\element{jpierce}{
\begin{question}{pt101tb3-Q163}
    According to Boyle's Law,
        if the volume occupied by a certain gas is halved,
    \begin{choices}
        \wrongchoice{the pressure of the gas will be halved.}
        \wrongchoice{the number of atoms in the gas will be doubled.}
        \wrongchoice{the pressure of the gas will be quadrupled.}
      \correctchoice{the pressure of the gas will be doubled.}
        \wrongchoice{the pressure of the gas will remain constant.}
    \end{choices}
\end{question}
}

\element{jpierce}{
\begin{question}{pt101tb3-Q164}
    When a fixed amount of air is compressed,
        at constant temperature, to half its original volume,
    \begin{choices}
      \correctchoice{the pressure of the air will be twice as much as before.}
        \wrongchoice{the density of the air will be one half as much as before.}
        \wrongchoice{the pressure of the air will be four times as much as before.}
        \wrongchoice{the pressure of the air will be one half as much as before.}
        \wrongchoice{the density of the air will be one fourth as much as before.}
    \end{choices}
\end{question}
}

\element{jpierce}{
\begin{question}{pt101tb3-Q165}
    When a fixed amount of air is compressed,
        at constant temperature, to one fourth its original volume,
    \begin{choices}
        \wrongchoice{the pressure of the air will be twice as much as before.}
        \wrongchoice{the pressure of the air will be one fourth as much as before.}
        \wrongchoice{the density of the air will be one half as much as before.}
        \wrongchoice{the density of the air will be one fourth as much as before.}
      \correctchoice{the pressure of the air will be four times as much as before.}
    \end{choices}
\end{question}
}

\element{jpierce}{
\begin{question}{pt101tb3-Q166}
    When a fixed amount of air is compressed,
        at constant temperature, to one third its original volume,
    \begin{choices}
      \correctchoice{the pressure of the air will be three times as much as before.}
        \wrongchoice{the pressure of the air will be nine times as much as before.}
        \wrongchoice{the density of the air will be one third as much as before.}
        \wrongchoice{the pressure of the air will be one third as much as before.}
        \wrongchoice{the density of the air will be nine times as much as before.}
    \end{choices}
\end{question}
}

%% Topic: Buoyancy
\element{jpierce}{
\begin{question}{pt101tb3-Q167}
    Archimedes' Principle states that an object surrounded by
        air is buoyed up by a force equal to the:
    \begin{choices}
      \correctchoice{weight of the air it displaces.}
        \wrongchoice{weight of Archimedes.}
        \wrongchoice{total pressure on the object.}
        \wrongchoice{difference between the weight of the object and the weight of the air it displaces.}
        \wrongchoice{weight of the object.}
    \end{choices}
\end{question}
}

\element{jpierce}{
\begin{question}{pt101tb3-Q168}
    A balloon will cease rising in air only when:
    \begin{choices}
      \correctchoice{the buoyant force on the balloon equals the weight of the balloon.}
        \wrongchoice{the buoyant force on the balloon is zero.}
        \wrongchoice{the weight of the balloon is zero.}
        \wrongchoice{the air pressure is zero.}
        \wrongchoice{the balloon reaches the very top of the atmosphere.}
    \end{choices}
\end{question}
}

\element{jpierce}{
\begin{question}{pt101tb3-Q169}
    Two helium-filled balloons have the same mass but one is larger than the other. 
    Which will rise more rapidly in air?
    \begin{choices}
      \correctchoice{The larger one, because it has a greater buoyant force.}
        \wrongchoice{They will rise at the same rate because they both contain helium.}
        \wrongchoice{The smaller one, because it has a greater buoyant force.}
        \wrongchoice{The smaller one, because it has a higher density.}
        \wrongchoice{The larger one, because it has a higher density.}
    \end{choices}
\end{question}
}

\element{jpierce}{
\begin{question}{pt101tb3-Q170}
    Two lighter-than-air helium-filled containers have the same
        fixed volume but one holds twice as many helium atoms as the other. 
    Which will rise more rapidly in air?
    \begin{choices}
        \wrongchoice{The one with more helium, because it has a higher density.}
        \wrongchoice{The one with less helium, because it has a greater buoyant force.}
        \wrongchoice{The one with more helium, because it has a greater buoyant force.}
      \correctchoice{The one with less helium, because it has a lower weight.}
        \wrongchoice{They will rise at the same rate because they both contain helium.}
    \end{choices}
\end{question}
}

\element{jpierce}{
\begin{question}{pt101tb3-Q171}
    Humans generally do not rise into the air like helium-filled balloons because
    \begin{choices}
        \wrongchoice{our bodies contain bones.}
        \wrongchoice{our bodies contain no helium.}
      \correctchoice{our bodies are more dense than air.}
        \wrongchoice{there is no buoyant force acting on our bodies.}
        \wrongchoice{air pressure pushes us down onto the ground.}
    \end{choices}
\end{question}
}

\element{jpierce}{
\begin{question}{pt101tb3-Q172}
    On which of these would air produce the greatest buoyant force?
    \begin{choices}
        \wrongchoice{a flying robin}
      \correctchoice{an elephant}
        \wrongchoice{a cat}
        \wrongchoice{a perching robin}
        \wrongchoice{a flying mosquito}
    \end{choices}
\end{question}
}

\element{jpierce}{
\begin{question}{pt101tb3-Q173}
    A helium-filled balloon released at the Earth's surface rises into the air.
    If an identical helium-filled balloon were released at the surface of the Moon,
        where there is less gravity and no atmosphere, what would happen to the balloon?
    \begin{choices}
        \wrongchoice{The balloon would rise from the Moon's surface, but at a slower rate than it did on Earth.}
        \wrongchoice{The balloon would rise from the Moon's surface, but at a faster rate than it did on Earth.}
        \wrongchoice{The balloon would rise from the Moon's surface at the same rate as it did on Earth.}
      \correctchoice{The balloon would fall to the Moon's surface.}
        \wrongchoice{The balloon would hover above the Moon's surface.}
    \end{choices}
\end{question}
}

\endinput

