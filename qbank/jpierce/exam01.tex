


%% Physics 101 Sample Test Questions by Dr. James Pierce
%%------------------------------------------------------------


%% Exam #1 ---- (motion, inertia, force, acceleration, etc.)
%%------------------------------------------------------------
\element{cpo-mc}{
\begin{question}{exam01-Q01}
    A book is at rest on top of a table.
    Which of the following is correct?
    \begin{choices}
        \wrongchoice{There is no force acting on the book.}
        \wrongchoice{The book has no inertia.}
        \wrongchoice{There is no force acting on the table.}
      \correctchoice{The book is in equilibrium.}
        \wrongchoice{The inertia of the book is equal to the inertia of the table.}
    \end{choices}
\end{question}
}

\element{cpo-mc}{
\begin{question}{exam01-Q02}
    The property of a moving object to continue moving is what Galileo called
    \begin{multicols}{2}
    \begin{choices}
        \wrongchoice{velocity.}
        \wrongchoice{speed.}
        \wrongchoice{acceleration.}
      \correctchoice{inertia.}
        \wrongchoice{direction.}
    \end{choices}
    \end{multicols}
\end{question}
}

\element{cpo-mc}{
\begin{question}{exam01-Q03}
    According to Newton's First Law of Motion,
    \begin{choices}
        \wrongchoice{an object in motion eventually comes to a halt.}
        \wrongchoice{an object at rest eventually begins to move.}
        \wrongchoice{an object in motion moves in a parabolic trajectory unless acted upon by a net force.}
        \wrongchoice{an object at rest always remains at rest.}
      \correctchoice{an object at rest remains at rest unless acted upon by a net force.}
    \end{choices}
\end{question}
}

\element{cpo-mc}{
\begin{question}{exam01-Q04}
    If an object is moving,
        then the magnitude of its \rule[-0.1pt]{4em}{0.1pt} cannot be zero.
    \begin{choices}
        \wrongchoice{speed}
        \wrongchoice{velocity}
        \wrongchoice{acceleration}
      \correctchoice{speed and velocity}
        \wrongchoice{speed, velocity and acceleration}
    \end{choices}
\end{question}
}

\element{cpo-mc}{
\begin{question}{exam01-Q05}
    A car initially at rest accelerates in a straight line at \SI{3}{\meter\per\second\squared}.
    What will be its speed after \SI{2}{\second}?
    \begin{multicols}{2}
    \begin{choices}
        \wrongchoice{\SI{0}{\meter\per\second}} 
        \wrongchoice{\SI{5}{\meter\per\second}} 
        \wrongchoice{\SI{3}{\meter\per\second}} 
      \correctchoice{\SI{6}{\meter\per\second}} 
        \wrongchoice{\SI{2}{\meter\per\second}} 
    \end{choices}
    \end{multicols}
\end{question}
}

\element{cpo-mc}{
\begin{question}{exam01-Q06}
    A body in free fall in a vacuum
    \begin{choices}
        \wrongchoice{will drop the same distance during each second of its fall.}
        \wrongchoice{will have the same average speed during each second of its fall.}
        \wrongchoice{will have a constant velocity during each second of its fall.}
        \wrongchoice{will not be accelerated during its fall.}
      \correctchoice{will have the same acceleration during each second of its fall.}
    \end{choices}
\end{question}
}

\element{cpo-mc}{
\begin{question}{exam01-Q07}
    A bowling ball at a height of \SI{36}{\meter} above the ground is
        falling vertically at a rate of \SI{12}{\meter\per\second}.
    Which of these best describes its fate?
    \begin{choices}
        \wrongchoice{It will hit the ground in exactly three seconds at a speed of \SI{12}{\meter\per\second}.}
      \correctchoice{It will hit the ground in less than three seconds at a speed greater than \SI{12}{\meter\per\second}.}
        \wrongchoice{It will hit the ground in more than three seconds at a speed less than \SI{12}{\meter\per\second}.}
        \wrongchoice{It will hit the ground in less than three seconds at a speed less than \SI{12}{\meter\per\second}.}
        \wrongchoice{It will hit the ground in more than three seconds at a speed greater than \SI{12}{\meter\per\second}.}
    \end{choices}
\end{question}
}

\element{cpo-mc}{
\begin{question}{exam01-Q08}
    The speedometer in your car tells you the \rule[-0.1pt]{4em}{0.1pt} of your car.
    \begin{multicols}{2}
    \begin{choices}
        \wrongchoice{acceleration}
        \wrongchoice{average speed}
      \correctchoice{instantaneous speed}
        \wrongchoice{velocity}
        \wrongchoice{inertia}
    \end{choices}
    \end{multicols}
\end{question}
}

\element{cpo-mc}{
\begin{question}{exam01-Q09}
    To report the \rule[-0.1pt]{4em}{0.1pt} of an object,
        we must specify both its speed and its direction .
    \begin{multicols}{2}
    \begin{choices}
        \wrongchoice{acceleration}
        \wrongchoice{mass}
      \correctchoice{velocity}
        \wrongchoice{length}
        \wrongchoice{position}
    \end{choices}
    \end{multicols}
\end{question}
}

\element{cpo-mc}{
\begin{question}{exam01-Q10}
    Projectile $A$ is fired at an angle of \ang{50} above the horizontal;
        projectile $B$ is fired with the same speed at an angle of \ang{40} above the horizontal.
    Assuming level ground and negligible air resistance,
        which of the following is true?
    \begin{choices}
        \wrongchoice{$A$ will reach a greater height and have a greater range than $B$.}
      \correctchoice{$A$ will reach a greater height and have the same range as $B$.}
        \wrongchoice{$A$ will reach a greater height and have a shorter range than $B$.}
        \wrongchoice{$A$ will reach the same height and have the same range as $B$.}
        \wrongchoice{$A$ will reach the same height and have a shorter range than $B$.}
    \end{choices}
\end{question}
}

\element{cpo-mc}{
\begin{question}{exam01-Q11}
    In the absence of air resistance,
        the magnitude of the vertical component of a projectile's acceleration
    \begin{choices}
      \correctchoice{is constant until the projectile hits the ground.}
        \wrongchoice{always decreases with time until the projectile hits the ground.}
        \wrongchoice{is equal to the magnitude of the horizontal component of the projectile's acceleration.}
        \wrongchoice{increases and/or decreases with time, depending on the projectile's velocity.}
        \wrongchoice{always increases with time until the projectile hits the ground.}
    \end{choices}
\end{question}
}

\element{cpo-mc}{
\begin{question}{exam01-Q12}
    A ball is thrown horizontally with a speed of \SI{25}{\meter\per\second}
        from the top of a tower \SI{20}{\meter} high.
    Assuming level ground below and negligible air resistance,
        what will be the magnitude of the vertical velocity component when the ball hits the ground?
    \begin{multicols}{2}
    \begin{choices}
        \wrongchoice{\SI{25}{\meter\per\second}}
        \wrongchoice{\SI{15}{\meter\per\second}}
      \correctchoice{\SI{20}{\meter\per\second}}
        \wrongchoice{\SI{50}{\meter\per\second}}
        \wrongchoice{\SI{10}{\meter\per\second}}
    \end{choices}
    \end{multicols}
\end{question}
}

\element{cpo-mc}{
\begin{question}{exam01-Q13}
    Which of these is the best description of the trajectory of a projectile shot from the top of a high cliff at an angle of 60° below the horizontal (neglecting air resistance)?
    \begin{choices}
        \wrongchoice{The projectile will move downwards at a \ang{60} angle in a straight line at a constant speed until it stops and then falls straight down.}
        \wrongchoice{The projectile will move downwards at a \ang{60} angle in a straight line at a gradually diminishing speed until it stops and then falls straight down.}
        \wrongchoice{The projectile will move downwards at a \ang{60} angle in a straight line at a gradually increasing speed until it stops and then falls straight down.}
        \wrongchoice{The projectile will gradually arc downward, following the curve of a circle.}
      \correctchoice{The projectile will gradually arc downward, following the curve of a parabola.}
    \end{choices}
\end{question}
}

\element{cpo-mc}{
\begin{question}{exam01-Q14}
    A firefighter with a mass of \SI{70}{\kilo\gram} slides down a vertical pole,
        accelerating at \SI{2}{\meter\per\second\squared}.
    The force of friction that acts on the firefighter is
    \begin{multicols}{2}
    \begin{choices}
        \wrongchoice{\SI{70}{\newton}}
      \correctchoice{\SI{560}{\newton}}
        \wrongchoice{\SI{140}{\newton}}
        \wrongchoice{\SI{700}{\newton}}
        \wrongchoice{\SI{0}{\newton}}
    \end{choices}
    \end{multicols}
\end{question}
}

\element{cpo-mc}{
\begin{question}{exam01-Q15}
    The \rule[-0.1pt]{4em}{0.1pt} of an object on the Earth's surface
        are directly proportional to each other.
    \begin{multicols}{2}
    \begin{choices}
        \wrongchoice{acceleration and mass}
      \correctchoice{mass and weight}
        \wrongchoice{force and velocity}
        \wrongchoice{weight and acceleration}
        \wrongchoice{speed and velocity}
    \end{choices}
    \end{multicols}
\end{question}
}

\element{cpo-mc}{
\begin{question}{exam01-Q16}
    The Moon's gravity is 1/6 of the Earth's gravity.
    The weight of a bowling ball on the Earth would
        be \rule[-0.1pt]{4em}{0.1pt} its weight on the Moon.
    \begin{multicols}{2}
    \begin{choices}
        \wrongchoice{equal to}
        \wrongchoice{\num{1/6} of}
      \correctchoice{\num{6} times}
        \wrongchoice{\num{36} times}
        \wrongchoice{\num{1/36} of}
    \end{choices}
    \end{multicols}
\end{question}
}

\element{cpo-mc}{
\begin{question}{exam01-Q17}
    When a certain net force is applied to one brick on a frictionless surface,
        it accelerates at \SI{6}{\meter\per\second\squared}.
    When the same net force is applied to three bricks that are cemented together,
    \begin{choices}
        \wrongchoice{they accelerate at \SI{3}{\meter\per\second\squared}.}
        \wrongchoice{they accelerate at \SI{6}{\meter\per\second\squared}.}
        \wrongchoice{they accelerate at \SI{18}{\meter\per\second\squared}.}
      \correctchoice{they accelerate at \SI{2}{\meter\per\second\squared}.}
        \wrongchoice{they do not accelerate at all.}
    \end{choices}
\end{question}
}

\element{cpo-mc}{
\begin{question}{exam01-Q18}
    To accelerate a \SI{6}{\kilo\gram} mass at \SI{2}{\meter\per\second\squared}
        requires a net force of:
    \begin{multicols}{2}
    \begin{choices}
        \wrongchoice{\SI{3}{\newton}}
        \wrongchoice{\SI{8}{\newton}}
      \correctchoice{\SI{12}{\newton}}
        \wrongchoice{\SI{6}{\newton}}
        \wrongchoice{\SI{2}{\newton}}
    \end{choices}
    \end{multicols}
\end{question}
}

\element{cpo-mc}{
\begin{question}{exam01-Q19}
    A falling object is said to reach terminal speed
    \begin{multicols}{2}
    \begin{choices}
        \wrongchoice{when it lands on the ground.}
      \correctchoice{when its air resistance equals the force of gravity on it.}
        \wrongchoice{when there is no air resistance acting on it.}
        \wrongchoice{when there is no gravitational force acting on it.}
        \wrongchoice{when it stops falling.}
    \end{choices}
    \end{multicols}
\end{question}
}

\element{cpo-mc}{
\begin{question}{exam01-Q20}
    For every action there is an equal and opposite reaction.
    This is a statement of
    \begin{choices}
        \wrongchoice{Newton's First Law of Motion.}
        \wrongchoice{Newton's Second Law of Motion.}
      \correctchoice{Newton's Third Law of Motion.}
        \wrongchoice{Newton's Fourth Law of Motion.}
        \wrongchoice{Newton's Law of Action.}
    \end{choices}
\end{question}
}

\element{cpo-mc}{
\begin{question}{exam01-Q21}
    An airplane flying east at an airspeed of \SI{200}{\kilo\meter\per\hour} has a headwind blowing from the east at \SI{50}{\kilo\meter\per\hour}.
    How far will the plane fly relative to the ground in two hours?
    \begin{multicols}{2}
    \begin{choices}
        \wrongchoice{\SI{500}{\kilo\meter}}
        \wrongchoice{\SI{250}{\kilo\meter}}
      \correctchoice{\SI{300}{\kilo\meter}}
        \wrongchoice{\SI{400}{\kilo\meter}}
        \wrongchoice{\SI{200}{\kilo\meter}}
    \end{choices}
    \end{multicols}
\end{question}
}

\element{cpo-mc}{
\begin{question}{exam01-Q22}
    An airplane heading west at an airspeed of \SI{100}{\kilo\meter\per\hour} has a crosswind blowing from the south at \SI{100}{\kilo\meter\per\hour}.
    What will be the airplane's speed relative to the ground?
    \begin{multicols}{2}
    \begin{choices}
        \wrongchoice{\SI{0}{\kilo\meter\per\hour}}
        \wrongchoice{\SI{71}{\kilo\meter\per\hour}}
        \wrongchoice{\SI{100}{\kilo\meter\per\hour}}
      \correctchoice{\SI{141}{\kilo\meter\per\hour}}
        \wrongchoice{\SI{200}{\kilo\meter\per\hour}}
    \end{choices}
    \end{multicols}
\end{question}
}

\element{cpo-mc}{
\begin{question}{exam01-Q23}
    \rule[-0.1pt]{4em}{0.1pt} are examples of vector quantities.
    \begin{multicols}{2}
    \begin{choices}
        \wrongchoice{Acceleration and time}
      \correctchoice{Velocity and acceleration}
        \wrongchoice{Volume and velocity}
        \wrongchoice{Mass and volume}
        \wrongchoice{Time and mass}
    \end{choices}
    \end{multicols}
\end{question}
}

\endinput


