
%% http://wps.aw.com/aw_wilson_physics_7/109/28121/7199048.cw/index.html

%%--------------------------------------------------
%% Pearson: Physics 7th Edition
%%--------------------------------------------------


%% Chapter 1: Measurement and Problem Solving
%%--------------------------------------------------


%% Pearson Practice Questions
%%--------------------------------------------------
\element{pearson}{
\begin{question}{ch01-Q01}
    The SI unit of mass is the:
    \begin{multicols}{2}
    \begin{choices}
        \wrongchoice{pound (lb).}
        \wrongchoice{gram (g).}
        \wrongchoice{ounce (oz).}
      \correctchoice{kilogram (kg).}
    \end{choices}
    \end{multicols}
\end{question}
}

\element{pearson}{
\begin{question}{ch01-Q02}
    A kilogram (\si{\kilo\gram}) is equal to:
    \begin{multicols}{2}
    \begin{choices}
        \wrongchoice{\SI{e-3}{\gram}.}
      \correctchoice{\SI{e3}{\gram}.}
        \wrongchoice{\SI{e6}{\gram}.}
        \wrongchoice{\SI{e1}{\gram}.}
    \end{choices}
    \end{multicols}
\end{question}
}

\element{pearson}{
\begin{question}{ch01-Q03}
    A centimeter (\si{\centi\meter}) is equal to:
    \begin{multicols}{2}
    \begin{choices}
        \wrongchoice{\SI{100}{\meter}.}
        \wrongchoice{\SI{10}{\meter}.}
        \wrongchoice{\SI{0.1}{\meter}.}
      \correctchoice{\SI{0.01}{\meter}.}
    \end{choices}
    \end{multicols}
\end{question}
}

\element{pearson}{
\begin{question}{ch01-Q04}
    A megawatt (\si{\mega\watt}) is equal to:
    \begin{multicols}{2}
    \begin{choices}
      \correctchoice{\SI{e6}{\watt}.}
        \wrongchoice{\SI{e3}{\watt}.}
        \wrongchoice{\SI{e-3}{\watt}.}
        \wrongchoice{\SI{e-6}{\watt}.}
    \end{choices}
    \end{multicols}
\end{question}
}

\element{pearson}{
\begin{question}{ch01-Q05}
    A nanosecond (\si{\nano\second}) is equal to:
    \begin{multicols}{2}
    \begin{choices}
        \wrongchoice{\SI{e-3}{\second}.}
        \wrongchoice{\SI{e-6}{\second}.}
      \correctchoice{\SI{e-9}{\second}.}
        \wrongchoice{\SI{e-12}{\second}.}
    \end{choices}
    \end{multicols}
\end{question}
}

\element{pearson}{
\begin{question}{ch01-Q06}
    A kilometer (\si{\kilo\meter}) is:
    \begin{choices}
        \wrongchoice{greater than a mile.}
      \correctchoice{less than a mile.}
        \wrongchoice{about the size of a football field.}
        \wrongchoice{less than the size of a football field.}
    \end{choices}
\end{question}
}

%% NOTE:
\element{pearson}{
\begin{question}{ch01-Q07}
    A variable that remains unchanged throughout an experiment is called the:
    \begin{choices}
      \correctchoice{control variable.}
        \wrongchoice{experimental variable.}
        \wrongchoice{independent variable.}
        \wrongchoice{dependent variable.}
    \end{choices}
\end{question}
}

\element{pearson}{
\begin{question}{ch01-Q08}
    Compared to a laboratory cart at the top of a ramp,
        a cart at the bottom of the ramp has:
    \begin{choices}
        \wrongchoice{more energy and less stability.}
      \correctchoice{less energy and more stability.}
        \wrongchoice{more energy and more stability.}
        \wrongchoice{less energy and less stability.}
    \end{choices}
\end{question}
}

\element{pearson}{
\begin{question}{ch01-Q09}
    Of the following objects, the one which has
        the most mass is:
    \begin{choices}
      \correctchoice{the Goodyear blimp.}
        \wrongchoice{a silver dollar.}
        \wrongchoice{a piece of notebook paper.}
        \wrongchoice{a physics textbook.}
    \end{choices}
\end{question}
}

\element{pearson}{
\begin{question}{ch01-Q10}
    Robin measures the force needed to pull a wagon up an incline
        as more weight is added.
    In this investigation,
        weight is the \rule[-0.1pt]{4em}{0.1pt} variable.
    \begin{multicols}{2}
    \begin{choices}
        \wrongchoice{control}
        \wrongchoice{dependent}
      \correctchoice{independent}
        \wrongchoice{natural}
    \end{choices}
    \end{multicols}
\end{question}
}

\element{pearson}{
\begin{question}{ch01-Q11}
    The unit of time used most commonly by physicists and other
        scientists is the:
    \begin{multicols}{2}
    \begin{choices}
      \correctchoice{second}
        \wrongchoice{minute}
        \wrongchoice{hour}
        \wrongchoice{light year}
    \end{choices}
    \end{multicols}
\end{question}
}

\element{pearson}{
\begin{questionmult}{ch01-Q12}
    The independent variable on a graph can be described as the variable:
    \begin{choices}
      \correctchoice{represented on the $x$-axis.}
      \correctchoice{causing the change in the experimental system.}
      \correctchoice{over which a scientist has direct control when designing the experiment.}
    \end{choices}
\end{questionmult}
}

\element{pearson}{
\begin{question}{ch01-Q13}
    The conversion factor for changing one unit of length to another
        in the metric system is a multiple of:
    \begin{multicols}{2}
    \begin{choices}
        \wrongchoice{\num{3}}
      \correctchoice{\num{10}}
        \wrongchoice{\num{12}}
        \wrongchoice{\num{5280}}
    \end{choices}
    \end{multicols}
\end{question}
}

\element{pearson}{
\begin{question}{ch01-Q14}
    Because it is based on factors of 10 and is easy to work with,
        scientists prefer to use the \rule[-0.1pt]{4em}{0.1pt} system.
    \begin{multicols}{2}
    \begin{choices}
      \correctchoice{metric}
        \wrongchoice{English}
        \wrongchoice{scientific}
        \wrongchoice{control}
    \end{choices}
    \end{multicols}
\end{question}
}

\element{pearson}{
\begin{question}{ch01-Q15}
    A graph may be described as all of the following \emph{except}:
    \begin{choices}
      \correctchoice{a tool to be interpreted \emph{only} by trained scientists and mathematicians.}
        \wrongchoice{used to describe the data collected from an experiment.}
        \wrongchoice{a picture showing how two variables are related.}
        \wrongchoice{easier to read than a table of numbers.}
    \end{choices}
\end{question}
}

\element{pearson}{
\begin{question}{ch01-Q16}
    The length of a new pencil is closest to:
    \begin{choices}
        \wrongchoice{5 millimeters}
      \correctchoice{20 centimeters}
        \wrongchoice{1.5 meters}
        \wrongchoice{2 kilometers}
    \end{choices}
    %\begin{multicols}{2}
    %\begin{choices}
    %    \wrongchoice{\SI{5}{\milli\meter}}
    %    \correctchoice{\SI{20}{\milli\meter}}
    %    \wrongchoice{\SI{1.5}{\meter}}
    %    \wrongchoice{\SI{2}{\kilo\meter}}
    %\end{choices}
    %\end{multicols}
\end{question}
}

\element{pearson}{
\begin{question}{ch01-Q17}
    The number of seconds in one week is:
    \begin{multicols}{2}
    \begin{choices}
        \wrongchoice{\num{86 400}}
      \correctchoice{\num{604 800}}
        \wrongchoice{\num{31 557 600}}
        \wrongchoice{\num{3 155 760 000}}
    \end{choices}
    \end{multicols}
\end{question}
}

\element{pearson}{
\begin{question}{ch01-Q18}
    Which of the lists show units arranged in order from smallest to largest?
    \begin{choices}
        \wrongchoice{millimeter, centimeter, kilometer, meter}
        \wrongchoice{centimeter, meter, kilometer, millimeter}
      \correctchoice{millimeter, centimeter, meter, kilometer}
        \wrongchoice{meter, kilometer, millimeter, centimeter}
    \end{choices}
\end{question}
}

\element{pearson}{
\begin{question}{ch01-Q19}
    Which of the following lists of mass units are arranged in order from
        smallest to largest?
    \begin{choices}
        \wrongchoice{gigagram, microgram, kilogram, megagram}
      \correctchoice{microgram, centigram, kilogram, gigagram}
        \wrongchoice{milligram, microgram, centigram, kilogram}
        \wrongchoice{megagram, kilogram, centigram, milligram}
    \end{choices}
\end{question}
}

\element{pearson}{
\begin{question}{ch01-Q20}
    How many seconds are in a stopwatch showing 1 hour, 3 minutes, and 5 seconds?
    \begin{multicols}{2}
    \begin{choices}
        \wrongchoice{\num{68} seconds}
        \wrongchoice{\num{245} seconds}
      \correctchoice{\num{7 385} seconds}
        \wrongchoice{\num{10 805} seconds}
    \end{choices}
    \end{multicols}
    %\begin{multicols}{2}
    %\begin{choices}
    %    \wrongchoice{\SI{68}{\second}}
    %    \wrongchoice{\SI{245}{\second}}
    %    \correctchoice{\SI{7 385}{\second}}
    %    \wrongchoice{\SI{10 805}{\second}}
    %\end{choices}
    %\end{multicols}
\end{question}
}

\element{pearson}{
\begin{question}{ch01-Q21}
    A rectangular solid has dimensions of
        $\SI{27}{\milli\meter}\times\SI{6.8}{\centi\meter}\times\SI{0.00025}{\kilo\meter}$.
    The volume of the solid is \rule[-0.1pt]{4em}{0.1pt} cubic centimeters.
    \begin{multicols}{2}
    \begin{choices}
        \wrongchoice{\num{0.045 9}}
        \wrongchoice{\num{0.345}}
        \wrongchoice{\num{33.8}}
      \correctchoice{\num{459}}
    \end{choices}
    \end{multicols}
\end{question}
}

\element{pearson}{
\begin{question}{ch01-Q22}
    Orlando measures the brightness of a flashlight bulb as he adds
        more batteries to the circuit.
    If he prepares a graph of the data:
    \begin{choices}
      \correctchoice{the number of batteries should be represented on the $x$-axis.}
        \wrongchoice{the brightness of the flashlight bulb should be represented on the $x$-axis.}
        \wrongchoice{it doesn't matter which variable he places on the $x$-axis.}
        \wrongchoice{he will need more information before deciding where to place the variables.}
    \end{choices}
\end{question}
}

\element{pearson}{
\begin{question}{ch01-Q23}
    On his way to a concert, John stops at the mall to buy some camera film.
    If you divide the distance he travels to the concert by the amount of time it
        took to get him home to his concert seat, you are calculating:
    \begin{multicols}{2}
    \begin{choices}
        \wrongchoice{speed.}
        \wrongchoice{distance.}
      \correctchoice{time interval.}
        \wrongchoice{mixed units.}
    \end{choices}
    \end{multicols}
\end{question}
}

\element{pearson}{
\begin{question}{ch01-Q24}
    If you know the distance traveled and the amount of time it took,
        speed may be calculated by:
    \begin{choices}
        \wrongchoice{dividing time by distance.}
        \wrongchoice{multiplying time by distance.}
      \correctchoice{dividing distance by time.}
        \wrongchoice{multiplying distance squared by time.}
    \end{choices}
\end{question}
}

\element{pearson}{
\begin{question}{ch01-Q25}
    Of the following, which equation does \emph{not} correctly represent
        a relationship between distance, time and speed?
    \begin{choices}
        \wrongchoice{Distance equals speed multiplied by time.}
      \correctchoice{Speed equals time multiplied by distance.}
        \wrongchoice{Time equals distance divided by speed.}
        \wrongchoice{Speed equals distance divided by time.}
    \end{choices}
\end{question}
}

\element{pearson}{
\begin{question}{ch01-Q26}
    The speed of a cheetah running 300 yards in 10 seconds is:
    \begin{multicols}{2}
    \begin{choices}
      \correctchoice{30 yards per second.}
        \wrongchoice{\num{3 000} yards per second.}
        \wrongchoice{\num{30 000} yards per second.}
        %% NOTE: I changed the wording on this
        \wrongchoice{Not defined.}
        %\correctchoice{\SI{30}{yds\per\second}.}
        %\wrongchoice{\SI{3 000}{yds\per\second}.}
        %\wrongchoice{\SI{30 000}{yds\per\second}.}
        %\wrongchoice{Not defined.}
    \end{choices}
    \end{multicols}
\end{question}
}

\element{pearson}{
\begin{question}{ch01-Q27}
    Doug rides a motorcycle at an average speed of 42 miles per hour
        for 3.6 hours.
    The distance he travels is about \rule[-0.1pt]{4em}{0.1pt} miles.
    \begin{multicols}{2}
    \begin{choices}
        \wrongchoice{\num{11}}
        \wrongchoice{\num{38}}
        \wrongchoice{\num{47}}
      \correctchoice{\num{150}}
    \end{choices}
    \end{multicols}
\end{question}
}

\element{pearson}{
\begin{question}{ch01-Q28}
    Gwen rides her bicycle 2.4 kilometers up a steep hill in
        8 minutes.
    Her speed is \rule[-0.1pt]{4em}{0.1pt} kilometers per minute.
    \begin{multicols}{2}
    \begin{choices}
      \correctchoice{\num{0.3}}
        \wrongchoice{\num{0.6}}
        \wrongchoice{\num{3.3}}
        \wrongchoice{\num{19}}
    \end{choices}
    \end{multicols}
\end{question}
}

\element{pearson}{
\begin{question}{ch01-Q29}
    A professional LPGA golfer walks at an average rate of
        3.20 feet per second on the golf course.
    The amount of time required for her to walk from the
        tee to a green \num{612} feet away is:
    \begin{multicols}{2}
    \begin{choices}
        \wrongchoice{\num{0.544} minutes.}
        \wrongchoice{\num{1.91} minutes.}
        \wrongchoice{\num{1 958} seconds.}
      \correctchoice{\num{191} seconds.}
    \end{choices}
    \end{multicols}
\end{question}
}

\element{pearson}{
\begin{question}{ch01-Q30}
    A professional football quarterback throws a ball 32 yards
        down field to a receiver at a speed of 60 miles per hour.
    A mile equals \num{1 760} yards.
    Once the quarterback releases the ball, the football gets
        to the receiver in about \rule[-0.1pt]{4em}{0.1pt} seconds.
    \begin{multicols}{2}
    \begin{choices}
      \correctchoice{\num{1.1}}
        \wrongchoice{\num{0.53}}
        \wrongchoice{\num{0.92}}
        \wrongchoice{\num{1.9}}
    \end{choices}
    \end{multicols}
\end{question}
}

\element{pearson}{
\begin{question}{ch01-Q31}
    Of the following, the largest unit of speed is:
    \begin{multicols}{2}
    \begin{choices}
      \correctchoice{meters per second.}
        \wrongchoice{kilometers per hour.}
        \wrongchoice{miles per hour.}
        \wrongchoice{inches per second.}
    \end{choices}
    \end{multicols}
\end{question}
}

\element{pearson}{
\begin{question}{ch01-Q32}
    The speed of a car traveling 200 meters in 10 seconds is equivalent to:
    \begin{multicols}{2}
    \begin{choices}
        \wrongchoice{20 yards per second.}
        \wrongchoice{2000 meters per second.}
      \correctchoice{72 kilometers per hour.}
        \wrongchoice{115 miles per hour.}
    \end{choices}
    \end{multicols}
    %\begin{multicols}{2}
    %\begin{choices}
    %    \wrongchoice{\SI{20}{yds\per\second}.}
    %    \wrongchoice{\SI{2000}{\meter\per\second}.}
    %    \correctchoice{\SI{72}{\kilo\meter\per\hour}.}
    %    \wrongchoice{\SI{115}{mile\per\hour}.}
    %\end{choices}
    %\end{multicols}
\end{question}
}

