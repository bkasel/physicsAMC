

%% This defines the lua function mycommand()
\directlua{require('./qbank/lua/common.lua')}


%% Objectives
%%--------------------------------------------------

%% 1. Name and define all base SI units
%% 2. Differentiate between defined and derives SI units


%% SI Base Units
%% length       meter       m
%% mass         kilogram    kg
%% time         second      s
%% current      ampere      A
%% temperature  kelvin      K
%% amount       mole        mol
%% luminosity   candela     cd


%% The international System of Units
%%--------------------------------------------------


%% Defined Units
%%--------------------
\element{SI}{
\begin{questionmult}{SI-Q01}
\luaexec{
    %% Question
    local Q = [[
        Which of the following are \string\emph{defined}
            units in the International System of Units?
    ]]
    %% Random Permutations
    local tab1 = {}
    for i=1,22 do
        tab1[i] = i
    end
    local tab2 = {}
    for i=1,7 do
        tab2[i] = i
    end
    tab1 = permute(tab1,22,22)
    tab2 = permute(tab2,7,7)
    %% Random correct vs wrong
    local n1 = math.random(1,3)
    local n2 = 4 - n1
    %% Print Question
    tex.print( Q )
    %% Print MC Options
    tex.print( BeginMulticolsTwo )
        tex.print( BeginChoices )
            for i=1,n2 do
                tex.print( string.format(CorrectChoice,defined[tab2[i]]) )
            end
            for i=1,n1 do
                tex.print( string.format(WrongChoice,derived[tab1[i]]) )
            end
        tex.print( EndChoices )
    tex.print( EndMulticolsTwo )
}
\end{questionmult}
}


%% Derived Units
%%--------------------
\element{SI}{
\begin{questionmult}{SI-Q02}
\luaexec{
    %% Question
    local Q = [[
        Which of the following are \string\emph{derived}
            units in the International System of Units?
    ]]
    %% Random Permutations
    local tab1 = {}
    for i=1,22 do
        tab1[i] = i
    end
    local tab2 = {}
    for i=1,7 do
        tab2[i] = i
    end
    tab1 = permute(tab1,22,22)
    tab2 = permute(tab2,7,7)
    %% Random correct vs wrong
    local n1 = math.random(1,3)
    local n2 = 4 - n1
    %% Print Question
    tex.print( Q )
    %% Print MC Options
    tex.print( BeginMulticolsTwo )
        tex.print( BeginChoices )
            for i=1,n1 do
                tex.print( string.format(CorrectChoice,derived[tab1[i]]) )
            end
            for i=1,n2 do
                tex.print( string.format(WrongChoice,defined[tab2[i]]) )
            end
        tex.print( EndChoices )
    tex.print( EndMulticolsTwo )
}
\end{questionmult}
}

