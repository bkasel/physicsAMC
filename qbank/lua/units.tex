

%% This defines the lua function mycommand()
%\directlua{require('./qbank/lua/common.lua')}

%% Objectives
%%----------------------------------------

%% 1. Convert between various units


%% Measurement Topic: Unit Conversion
%%----------------------------------------

%% Break topics into
%% one dimensional conversion (time, length)
%% two dimensional conversion (area, density)
%% three dimensional conversion (volume)

%% Types
%% prefix changes (1d,2d,3d)
%% metric to customary (1d,2d,3d)


%% Customary to Metric
%%------------------------------
\element{units}{
\begin{question}{UnitsQ01}
\luaexec{
    %% Question
    local Q = [[
        A rectangular building lot measures
            \string\SI{\%d}{\string\foot} by
            \string\SI{\%d}{\string\foot}.
        What is the area of the lot?
    ]]
    %% Random Values
    local length = math.random(90,110)
    local width = math.random(60,90)
    %% Random Permutations
    local tab = {}
    for i=1,8 do
        tab[i] = i
    end
    tab = permute(tab,8,8)
    %% Convert Units
    local foot = 0.3048
    local right = length * width * foot * foot
    local wrong = {
        right * foot, 
        right / foot, 
        right * foot * foot, 
        right / foot / foot, 
        right * length,
        right / length,
        right * width,
        right / width,
    }
    %% Formatting
    local SI = [[\string\SI[retain-zero-exponent]{\%.2e}{\string\meter\string\squared}]]
    %% Print Question
    tex.print( string.format(Q,length,width) )
    %% Print MC Options
    %tex.print( string.format( BeginMulticolsN, ncol ) )
    %tex.print( BeginMulticolsThree )
    tex.print( BeginMulticolsTwo )
        tex.print( BeginChoices )
            tex.print( string.format( CorrectChoice,string.format(SI,right) ) )
            for i=1,nwrong do
                tex.print( string.format( WrongChoice,string.format(SI,wrong[tab[i]]) ) )
            end
        tex.print( EndChoices )
    tex.print( EndMulticols )
}
\end{question}
}


\element{units}{
\begin{question}{UnitsQ02}
\luaexec{
    %% Question
    local Q = [[
        A fish tank has internal dimension of
            \string\SI{\%d}{\string\inch} long,
            \string\SI{\%d}{\string\inch} wide,
            and \string\SI{\%d}{\string\inch} high.
        What is the maximum amount of water
            that the fish tank can hold?
    ]]
    %% Random Values
    local long = math.random(24,32)
    local wide = math.random(12,18)
    local high = math.random(18,24)
    %% Random Permutations
    local tab = {}
    for i=1,10 do
        tab[i] = i
    end
    tab = permute(tab,10,10)
    %% Convert Units
    local foot = 0.3048
    local inch = 0.0254
    local right = long * wide * high * inch * inch * inch
    local wrong = {
        right * inch,
        right / inch,
        right * inch * inch,
        right / inch / inch,
        right * inch * inch * inch,
        right / inch / inch / inch,
        right * foot,
        right / foot,
        right * foot * foot,
        right / foot / foot,
    }
    %% Formatting
    local SI = [[\string\SI[retain-zero-exponent]{\%.2e}{\string\meter\string\cubed}]]
    %% Print Question
    tex.print( string.format(Q,long,wide,high) )
    %% Print MC Options
    %tex.print( string.format( BeginMulticolsN, ncol ) )
    %tex.print( BeginMulticolsThree )
    tex.print( BeginMulticolsTwo )
        tex.print( BeginChoices )
            tex.print( string.format( CorrectChoice,string.format(SI,right) ) )
            for i=1,nwrong do
                tex.print( string.format( WrongChoice,string.format(SI,wrong[tab[i]]) ) )
            end
        tex.print( EndChoices )
    tex.print( EndMulticols )
}
\end{question}
}


\element{units}{
\begin{question}{UnitsQ03}
\luaexec{
    %% Question
    local Q = [[
        You observe a speedlimit sign for
            \string\SI{\%d}{\string\mile\string\per\string\hour}.
        What is the maximum speed that you
            can lawfully drive at?
    ]]
    %% Random Values
    local speed = 35 + 5 * math.random(0,6)
    %% Random Permutations
    local tab = {}
    for i=1,17 do
        tab[i] = i
    end
    tab = permute(tab,17,17)
    %% Convert Units
    local mile = 5280
    local hour = 3600
    local foot = 0.3048
    local right = speed * mile * foot / hour
    local wrong = {
        right - 10.0,
        right - 5.0,
        right + 5.0,
        right + 10.0,
        right * hour,
        right / hour,
        right * mile,
        right / mile,
        right * foot,
        right / foot,
        right * 1e3,
        right / 1e3,
        right * mile * foot,
        right / mile / foot,
        right * 2.54,
        right / 2.54,
        speed,
    }
    %% Formatting
    local SI = [[\string\SI[retain-zero-exponent]{\%.2e}{\string\meter\string\per\string\second}]]
    %% Print Question
    tex.print( string.format(Q,speed) )
    %% Print MC Options
    %tex.print( string.format( BeginMulticolsN, ncol ) )
    %tex.print( BeginMulticolsThree )
    tex.print( BeginMulticolsTwo )
        tex.print( BeginChoices )
            tex.print( string.format( CorrectChoice,string.format(SI,right) ) )
            for i=1,nwrong do
                tex.print( string.format( WrongChoice,string.format(SI,wrong[tab[i]]) ) )
            end
        tex.print( EndChoices )
    tex.print( EndMulticols )
}
\end{question}
}


%% NOTE: consider adding numbers to correct so they are close, forcing student to calculate
\element{units}{
\begin{question}{UnitsQ04}
\luaexec{
    %% Question
    local Q = [[
        You drive a Canadian made car in
            the United States and observe
            a speedlimit sign for
            \string\SI{\%d}{\string\mile\string\per\string\hour}.
        What is that equivalent to on your
            Canadian made speedometer?
    ]]
    %% Random Values
    local speed = 35 + 5 * math.random(0,6)
    %% Random Permutations
    local tab = {}
    for i=1,15 do
        tab[i] = i
    end
    tab = permute(tab,15,15)
    %% Convert Units
    local foot = 0.3048
    local mile = 5280
    local right = speed * (mile * foot) / (1e3)
    local wrong = {
        right - 10.0,
        right - 5.0,
        right + 5.0,
        right + 10.0,
        right * mile,
        right / mile,
        right * foot,
        right / foot,
        right * 1e3,
        right / 1e3,
        right * 3600,
        right / 3600,
        right * 2.54,
        right / 2.54,
        speed,
    }
    %% Formatting
    local SI = [[ \string\SI{\%d}{\string\kilo\string\meter\string\per\string\hour} ]]
    %% Print Question
    tex.print( string.format(Q,speed) )
    %% Print MC Options
    %tex.print( string.format( BeginMulticolsN, ncol ) )
    %tex.print( BeginMulticolsThree )
    tex.print( BeginMulticolsTwo )
        tex.print( BeginChoices )
            tex.print( string.format( CorrectChoice,string.format(SI,right) ) )
            for i=1,nwrong do
                tex.print( string.format( WrongChoice,string.format(SI,wrong[tab[i]]) ) )
            end
        tex.print( EndChoices )
    tex.print( EndMulticols )
}
\end{question}
}


\element{units}{
\begin{question}{UnitsQ05}
\luaexec{
    %% Question
    local Q = [[
        You drive an American made car in Canada
        and observe a speedlimit sign for
            \string\SI{\%d}{\string\kilo\string\meter\string\per\string\hour}.
        What is that equivalent to on your
            American made speedometer?
    ]]
    %% Random Values
    local speed = 30 + 10 * math.random(0,9)
    %% Random Permutations
    local tab = {}
    for i=1,15 do
        tab[i] = i
    end
    tab = permute(tab,15,15)
    %% Convert Units
    local foot = 0.3048
    local inch = 0.0254
    local mile = 5280
    local right = speed * 1e3 / (mile * foot)
    local wrong = {
        right - 10.0,
        right - 5.0,
        right + 5.0,
        right + 10.0,
        right * mile,
        right / mile,
        right * foot,
        right / foot,
        right * inch,
        right / inch,
        right * 1e3,
        right / 1e3,
        right * 2.54,
        right / 2.54,
        speed,
    }
    %% Formatting
    local SI = [[ \string\SI{\%d}{\string\mile\string\per\string\hour} ]]
    %% Print Question
    tex.print( string.format(Q,speed) )
    %% Print MC Options
    %tex.print( string.format( BeginMulticolsN, ncol ) )
    %tex.print( BeginMulticolsThree )
    tex.print( BeginMulticolsTwo )
        tex.print( BeginChoices )
            tex.print( string.format( CorrectChoice,string.format(SI,right) ) )
            for i=1,nwrong do
                tex.print( string.format( WrongChoice,string.format(SI,wrong[tab[i]]) ) )
            end
        tex.print( EndChoices )
    tex.print( EndMulticols )
}
\end{question}
}


\element{units}{
\begin{question}{UnitsQ06}
\luaexec{
    %% Question
    local Q = [[
        A fish tank with internal dimension of
            \string\SI{\%d}{\string\centi\string\meter} long,
            \string\SI{\%d}{\string\centi\string\meter} wide, and
            \string\SI{\%d}{\string\centi\string\meter} high
            is filled with water to a depth of
            \string\SI{\%d}{\string\centi\string\meter}.
        How much water is in the fish tank?
    ]]
    %% Random Values
    local long = math.random(50,80)
    local wide = math.random(20,30)
    local high = math.random(30,40)
    local depth = math.random(20,30)
    %% Random Permutations
    local tab = {}
    for i=1,12 do
        tab[i] = i
    end
    tab = permute(tab,12,12)
    %% Convert Units
    %% 1000 cc = 1 L
    local right = long * wide * depth / 1e3
    local wrong = {
        right * wide,
        right / wide,
        right * long,
        right / long,
        right * depth,
        right / depth,
        right * 1e2,
        right / 1e2,
        right * 1e3,
        right / 1e3,
        right * 1e1,
        right / 1e1,
    }
    %% Formatting
    local SI = [[ \string\SI{\%.1f}{\string\liter} ]]
    %% Print Question
    tex.print( string.format(Q,long,wide,high,depth) )
    %% Print MC Options
    %tex.print( string.format( BeginMulticolsN, ncol ) )
    %tex.print( BeginMulticolsThree )
    tex.print( BeginMulticolsTwo )
        tex.print( BeginChoices )
            tex.print( string.format( CorrectChoice,string.format(SI,right) ) )
            for i=1,nwrong do
                tex.print( string.format( WrongChoice,string.format(SI,wrong[tab[i]]) ) )
            end
        tex.print( EndChoices )
    tex.print( EndMulticols )
}
\end{question}
}


\element{units}{
\begin{question}{UnitsQ07}
\luaexec{
    %% Question
    local Q = [[
        A rectangular plate has a length of
            \string\SI{\%d}{\string\inch} and a width of
            \string\SI{\%d}{\string\inch}.
        What is the area of the plate?
    ]]
    %% Random Values
    local length = math.random(10,50)
    local width = math.random(5,35)
    %% Random Permutations
    local tab = {}
    for i=1,8 do
        tab[i] = i
    end
    tab = permute(tab,8,8)
    %% Convert Units
    local inch = 0.0254
    local right = length * width * inch * inch
    local wrong = {
        right * width,
        right / width,
        right * length,
        right / length,
        right * inch,
        right / inch,
        right * inch * inch,
        right / inch / inch,
    }
    %% Formatting
    local SI = [[\string\SI[retain-zero-exponent]{\%.2e}{\string\meter\string\squared}]]
    %% Print Question
    tex.print( string.format(Q,length,width) )
    %% Print MC Options
    %tex.print( string.format( BeginMulticolsN, ncol ) )
    %tex.print( BeginMulticolsThree )
    tex.print( BeginMulticolsTwo )
        tex.print( BeginChoices )
            tex.print( string.format( CorrectChoice,string.format(SI,right) ) )
            for i=1,nwrong do
                tex.print( string.format( WrongChoice,string.format(SI,wrong[tab[i]]) ) )
            end
        tex.print( EndChoices )
    tex.print( EndMulticols )
}
\end{question}
}


\element{units}{
\begin{question}{UnitsQ08}
\luaexec{
    %% Question
    local Q = [[
        A farmer measures the perimeter of a rectangular field.
        The length of each long side of the rectangle is found to be 
            \string\SI{\%d}{\string\yard},
            and the length of each short side is found to be
            \string\SI{\%d}{\string\yard}.
        What is the perimeter of the field?
    ]]
    %% Random Values
    local long  = math.random(90,110)
    local short = math.random(60,90)
    %% Random Permutations
    local tab = {}
    for i=1,10 do
        tab[i] = i
    end
    tab = permute(tab,10,10)
    %% Convert Units
    local foot = 0.3048
    local right = 2 * (long+short) * 3 * foot
    local wrong = {
        right * foot,
        right / foot,
        right * 3,
        right / 3,
        right * 3 * foot,
        right / 3 / foot,
        right * 2,
        right / 2,
        right * 2 * foot,
        right / 2 / foot,
    }
    %% Formatting
    local SI = [[\string\SI[retain-zero-exponent]{\%.2e}{\string\meter}]]
    %% Print Question
    tex.print( string.format(Q,long,short) )
    %% Print MC Options
    %tex.print( string.format( BeginMulticolsN, ncol ) )
    %tex.print( BeginMulticolsThree )
    tex.print( BeginMulticolsTwo )
        tex.print( BeginChoices )
            tex.print( string.format( CorrectChoice,string.format(SI,right) ) )
            for i=1,nwrong do
                tex.print( string.format( WrongChoice,string.format(SI,wrong[tab[i]]) ) )
            end
        tex.print( EndChoices )
    tex.print( EndMulticols )
}
\end{question}
}


\element{units}{
\begin{question}{UnitsQ09}
\luaexec{
    %% Question
    local Q = [[
        A gas station in Canada is selling
            gasoline for \string\num{\%.2f}
            Canadian dollars per liter.
        How much is that in US dollars (USD)
            per gallon?
        [On 21 August 2014, \string\num{1.00} U.S. dollar = \string\num{1.09} Canadian dollar.
            Source: Bank of Canada.]
    ]]
    %% Random Values
    local price  = 1 + round(math.random(),2)
    %% Random Permutations
    local tab = {}
    for i=1,8 do
        tab[i] = i
    end
    tab = permute(tab,8,8)
    %% Convert Units
    local gallon = 3.785411784
    local right = price * gallon / 1.09
    local wrong = {
        right * gallon,
        right / gallon,
        right * gallon * gallon,
        right / gallon / gallon,
        right * 1.09,
        right / 1.09,
        right * 1.09 * 1.09,
        right / 1.09 / 1.09,
    }
    %% Formatting
    local SI = [[ \string\SI{\%.2f}{USD\string\per\string\gallon} ]]
    %% Print Question
    tex.print( string.format(Q,price) )
    %% Print MC Options
    %tex.print( BeginMulticolsThree )
    tex.print( BeginMulticolsTwo )
        tex.print( BeginChoices )
            tex.print( string.format( CorrectChoice,string.format(SI,right) ) )
            for i=1,nwrong do
                tex.print( string.format( WrongChoice,string.format(SI,wrong[tab[i]]) ) )
            end
        tex.print( EndChoices )
    tex.print( EndMulticols )
}
\end{question}
}


\element{units}{
\begin{question}{UnitsQ10}
\luaexec{
    %% Question
    local Q = [[
        A swimming pool holds
            \string\SI{\%d}{gallons}
            of water.
        How much water is that equivalent to
            in SI units?
    ]]
    %% Random Values
    local size  = (1e4 * math.random(1,9)) + (1e5 * math.random(0,9) )
    %% Random Permutations
    local tab = {}
    for i=1,10 do
        tab[i] = i
    end
    tab = permute(tab,10,10)
    %% 1 gallon = 0.134 cubic feet
    %% 1 gallon = 3.785411784 liter
    %% Convert Units
    local gallon = 3.785411784
    local right = size * gallon * 1e-3
    local wrong = {
        right * gallon,
        right / gallon,
        right * gallon * gallon,
        right / gallon / gallon,
        right * 1e4,
        right / 1e4,
        right * 1e3,
        right / 1e3,
        right * 1e2,
        right / 1e2,
    }
    %% Formatting
    local SI = [[ \string\SI[retain-zero-exponent]{\%.2e}{\string\meter\string\cubed}]]
    %% Print Question
    tex.print( string.format(Q,size) )
    %% Print MC Options
    %tex.print( string.format( BeginMulticolsN, ncol ) )
    %tex.print( BeginMulticolsThree )
    tex.print( BeginMulticolsTwo )
        tex.print( BeginChoices )
            tex.print( string.format( CorrectChoice,string.format(SI,right) ) )
            for i=1,nwrong do
                tex.print( string.format( WrongChoice,string.format(SI,wrong[tab[i]]) ) )
            end
        tex.print( EndChoices )
    tex.print( EndMulticols )
}
\end{question}
}


%% Odd Customary Units
%%------------------------------
\element{units}{
\begin{question}{UnitsQ51}
\luaexec{
    %% Question
    local Q = [[
        There is \string\SI{3}{teaspoon} (tsp)
            in a tablespoon (tbsp),
            \string\SI{16}{tbsp} in a cup,
            and \string\SI{16}{cup} in a gallon.
        How much is \string\SI{\%d}{tsp}
            in SI units?
    ]]
    %% Random Values
    local tsp  = math.random(2,128)
    %% Random Permutations
    local tab = {}
    for i=1,12 do
        tab[i] = i
    end
    tab = permute(tab,12,12)
    %% 1 gallon = 3.785411784 liter
    %% 3*16*16 = 768
    %% Convert Units
    local liter = 3.785411784
    local right = tsp * 1e-3 * liter / (768) 
    local wrong = {
        right * liter,
        right / liter,
        right * 768,
        right / 768,
        right * 1e3,
        right / 1e3,
        right * 3,
        right / 3,
        right * 16,
        right / 16,
        right * tsp,
        right / tsp,
    }
    %% Formatting
    local SI = [[ \string\SI[retain-zero-exponent]{\%.2e}{\string\meter\string\cubed}]]
    %% Print Question
    tex.print( string.format(Q,tsp) )
    %% Print MC Options
    %tex.print( string.format( BeginMulticolsN, ncol ) )
    %tex.print( BeginMulticolsThree )
    tex.print( BeginMulticolsTwo )
        tex.print( BeginChoices )
            tex.print( string.format( CorrectChoice,string.format(SI,right) ) )
            for i=1,nwrong do
                tex.print( string.format( WrongChoice,string.format(SI,wrong[tab[i]]) ) )
            end
        tex.print( EndChoices )
    tex.print( EndMulticols )
}
\end{question}
}

\element{units}{
\begin{question}{UnitsQ52}
\luaexec{
    %% Question
    local Q = [[
        There is \string\SI{875/32}{grain} in a dram,
            \string\SI{16}{dram} in an ounce,
            and \string\SI{16}{ounce} in a pound.
        How much is \string\SI{\%d}{grain} in SI units?
    ]]
    %% Random Values
    local grain  = math.random(2,32)
    %% Random Permutations
    local tab = {}
    for i=1,12 do
        tab[i] = i
    end
    tab = permute(tab,12,12)
    %% 7000 grain = 1 pound
    %% Convert Units
    local pound = 0.45359237
    local right = grain * pound / 7000
    local wrong = {
        right * grain,
        right / grain,
        right * pound,
        right / pound,
        right / pound / pound,
        right * 7000,
        right * 7000 * 7000,
        right / 7000,
        right * 32,
        right / 32,
        right * 16,
        right / 16,
    }
    %% Formatting
    local SI = [[ \string\SI[retain-zero-exponent]{\%.2e}{\string\kilo\string\gram}]]
    %% Print Question
    tex.print( string.format(Q,grain) )
    %% Print MC Options
    %tex.print( string.format( BeginMulticolsN, ncol ) )
    %tex.print( BeginMulticolsThree )
    tex.print( BeginMulticolsTwo )
        tex.print( BeginChoices )
            tex.print( string.format( CorrectChoice,string.format(SI,right) ) )
            for i=1,nwrong do
                tex.print( string.format( WrongChoice,string.format(SI,wrong[tab[i]]) ) )
            end
        tex.print( EndChoices )
    tex.print( EndMulticols )
}
\end{question}
}


\element{units}{
\begin{question}{UnitsQ53}
\luaexec{
    %% Question
    local Q = [[
        There is \string\SI{25}{link} in a rod,
            \string\SI{4}{rod} in a chain,
            \string\SI{10}{chain} in a furlong,
            and \string\SI{8}{furlong} in a mile?
        How much is \string\SI{\%d}{link} in SI units?
    ]]
    %% Random Values
    local link  = math.random(2,32)
    %% Random Permutations
    local tab = {}
    for i=1,14 do
        tab[i] = i
    end
    tab = permute(tab,14,14)
    %% 8000 links = 1 mile
    %% Convert Units
    local right = link * 0.3048 * 5280 / 8000
    local wrong = {
        %% NOTE: add link to wrongs 
        right * 0.3048,
        right / 0.3048,
        right * 5280,
        right / 5280,
        right * 8000,
        right / 8000,
        right * 25,
        right / 25,
        right * 10,
        right / 10,
        right * 8,
        right / 8,
        right * 4,
        right / 4,
    }
    %% Formatting
    local SI = [[ \string\SI[retain-zero-exponent]{\%.2e}{\string\meter}]]
    %% Print Question
    tex.print( string.format(Q,link) )
    %% Print MC Options
    %tex.print( string.format( BeginMulticolsN, ncol ) )
    %tex.print( BeginMulticolsThree )
    tex.print( BeginMulticolsTwo )
        tex.print( BeginChoices )
            tex.print( string.format( CorrectChoice,string.format(SI,right) ) )
            for i=1,nwrong do
                tex.print( string.format( WrongChoice,string.format(SI,wrong[tab[i]]) ) )
            end
        tex.print( EndChoices )
    tex.print( EndMulticols )
}
\end{question}
}


\element{units}{
\begin{question}{UnitsQ54}
\luaexec{
    %% Question
    local Q = [[
        There is \string\SI{12}{point} in a pica,
            and \string\SI{6}{pica} in an inch,
            and \string\SI{12}{inch} in a foot.
        How much is \string\SI{\%d}{point} in SI units?
    ]]
    %% Random Values
    local point  = math.random(1,144)
    %% Random Permutations
    local tab = {}
    for i=1,10 do
        tab[i] = i
    end
    tab = permute(tab,10,10)
    %% Convert Units
    %% 864 point = 1 foot
    local right = point * 0.3048 / 864
    local wrong = {
        %% NOTE: add point to wrongs 
        right * 0.3048,
        right / 0.3048,
        right * 864,
        right / 864,
        right * 72,
        right / 72,
        right * 12,
        right / 12,
        right * 6,
        right / 6,
    }
    %% Formatting
    local SI = [[ \string\SI[retain-zero-exponent]{\%.2e}{\string\meter}]]
    %% Print Question
    tex.print( string.format(Q,point) )
    %% Print MC Options
    %tex.print( string.format( BeginMulticolsN, ncol ) )
    %tex.print( BeginMulticolsThree )
    tex.print( BeginMulticolsTwo )
        tex.print( BeginChoices )
            tex.print( string.format( CorrectChoice,string.format(SI,right) ) )
            for i=1,nwrong do
                tex.print( string.format( WrongChoice,string.format(SI,wrong[tab[i]]) ) )
            end
        tex.print( EndChoices )
    tex.print( EndMulticols )
}
\end{question}
}


%% Fixed correct, random wrong
%%------------------------------
\element{units}{
\begin{question}{UnitsQ71}
\luaexec{
    %% Question
    local Q = [[
        The speed of light is \string\SI{3.00e8}{\string\meter\string\per\string\second}.
        Convert this to furlongs per fortnight.
        A furlong is \string\SI{220}{\string\yard}, and a fortnight
            is \string\SI{14}{\string\day}.
    ]]
    %% Random Permutations
    local tab = {}
    for i=1,14 do
        tab[i] = i
    end
    tab = permute(tab,14,14)
    %% 1 furlong = 220 yards
    %% 0.9144 m = 1 yard
    %% 60s 60min 24hr 14day = 1209600
    %% Convert Units
    local fortnight = 1209600
    local yard = 0.9144
    local furlong = 220
    local right = 3.00e8 * fortnight / yard / furlong
    local wrong = {
        right * furlong,
        right / furlong,
        right * fortnight,
        right / fortnight,
        right * yard,
        right / yard,
        right * 24,
        right / 24,
        right * 60,
        right / 60,
        right * 7,
        right / 7,
        right * 14,
        right / 14,
    }
    %% Formatting
    local SI = [[ \string\SI[retain-zero-exponent]{\%.2e}{furlong per fortnight}]]
    %% Print Question
    tex.print( Q )
    %% Print MC Options
    %tex.print( BeginMulticolsTwo )
        tex.print( BeginChoices )
            tex.print( string.format( CorrectChoice,string.format(SI,right) ) )
            for i=1,nwrong do
                tex.print( string.format( WrongChoice,string.format(SI,wrong[tab[i]]) ) )
            end
        tex.print( EndChoices )
    %tex.print( EndMulticols )
}
\end{question}
}

%% Name             kcal    kJ
%% Protein          4       16.7
%% Fat              9       37.7
%% Carbohydrate     4       16.7
\element{units}{
\begin{question}{UnitsQ72}
\luaexec{
    %% Question
    local Q = [[
        If fat has an energy density of
            \string\SI{9}{\string\kilo\string\calorie\string\per\string\gram}.
        How many kilocalories must you burn to lose
            \string\SI{\%d}{\string\pound} of body fat?
    ]]
    %% Random Value
    local lbs = math.random(2,30)
    %% Random Permutations
    local tab = {}
    for i=1,10 do
        tab[i] = i
    end
    tab = permute(tab,10,10)
    %% Convert Units
    local pound = 453.59237
    local right = lbs * 9 * pound
    local wrong = {
        right / pound,
        right * pound,
        right / 9,
        right * 9,
        right / lbs,
        right * lbs,
        right / 1e3,
        right * 1e3,
        right / 1e2,
        right * 1e2,
    }
    %% Formatting
    local SI = [[ \string\SI[retain-zero-exponent]{\%.2e}{\string\kilo\string\calorie}]]
    %% Print Question
    tex.print( string.format(Q,lbs) )
    %% Print MC Options
    %tex.print( string.format( BeginMulticolsN, ncol ) )
    %tex.print( BeginMulticolsThree )
    tex.print( BeginMulticolsTwo )
        tex.print( BeginChoices )
            tex.print( string.format( CorrectChoice,string.format(SI,right) ) )
            for i=1,nwrong do
                tex.print( string.format( WrongChoice,string.format(SI,wrong[tab[i]]) ) )
            end
        tex.print( EndChoices )
    tex.print( EndMulticols )
}
\end{question}
}


%% NOTE: Moved Dimensional Analysis to Motion in 1D and 3D

