
%% Cracking the AP Physics B and C
%%----------------------------------------

%% Chapter 06: Rotational Motion
%%----------------------------------------
\element{AP}{
\begin{question}{rotation-Q01}
    A compact disc has a radius of \SI{6}{\centi\meter}.
    If the disc rotates about its center axis at an angular speed of \SI{5}{rev\per\second},
        what is the linear speed of a point on the rim of the disc?
    \begin{multicols}{2}
    \begin{choices}
        \wrongchoice{\SI{0.3}{\meter\per\second}}
        \wrongchoice{\SI{1.9}{\meter\per\second}}
        \wrongchoice{\SI{7.4}{\meter\per\second}}
        \wrongchoice{\SI{52}{\meter\per\second}}
        \wrongchoice{\SI{83}{\meter\per\second}}
    \end{choices}
    \end{multicols}
\end{question}
}

\element{AP}{
\begin{question}{rotation-Q02}
    A compact disc has a radius of \SI{6}{\centi\meter}.
    If the disc rotates about its center axis at an angular speed of \SI{5}{rev\per\second},
        what is the total distance traveled by a point on the rim of the disc in \SI{40}{\minute}?
    \begin{multicols}{2}
    \begin{choices}
        \wrongchoice{\SI{180}{\meter}}
        \wrongchoice{\SI{360}{\meter}}
        \wrongchoice{\SI{540}{\meter}}
        \wrongchoice{\SI{720}{\meter}}
        \wrongchoice{\SI{4.5}{\kilo\meter}}
    \end{choices}
    \end{multicols}
\end{question}
}

\element{AP}{
\begin{question}{rotation-Q03}
    An object of mass \SI{0.5}{\kilo\gram},  moving in a circular path of radius \SI{0.25}{\meter},
        experiences a centripetal acceleration of constant magnitude \SI{9}{\meter\per\second\squared}.
    What is the object's angular speed?
    \begin{multicols}{2}
    \begin{choices}
        \wrongchoice{\SI{2.3}{\radian\per\second}}}
        \wrongchoice{\SI{4.5}{\radian\per\second}}}
        \wrongchoice{\SI{6}{\radian\per\second}}}
        \wrongchoice{\SI{12}{\radian\per\second}}}
        \wrongchoice{Cannot be determined from the information given}
    \end{choices}
    \end{multicols}
\end{question}
}

\element{AP}{
\begin{question}{rotation-Q04}
    An object, originally at rest, begins spinning under uniform angular acceleration.
    In \SI{10}{\second}, it completes an angular displacement of \SI{60}{\radian}.
    What is the numerical value of the angular acceleration?
    \begin{multicols}{2}
    \begin{choices}
        \wrongchoice{\SI{0.3}{\radian\per\second}}}
        \wrongchoice{\SI{0.6}{\radian\per\second}}}
        \wrongchoice{\SI{1.2}{\radian\per\second}}}
        \wrongchoice{\SI{2.4}{\radian\per\second}}}
        \wrongchoice{\SI{3.6}{\radian\per\second}}}
    \end{choices}
    \end{multicols}
\end{question}
}

\element{AP}{
\begin{question}{rotation-Q05}
    In an effort to tighten a bolt, a force $\mathbf{F}$ is applied as shown in the figure below.
    \begin{center}
    \begin{tikzpicture}
        %% NOTE:
    \end{tikzpicture}
    \end{center}
    If the distance from the end of the wrench to the center of the bolt is \SI{20}{\centi\meter} and $\mathbf{F}=\SI{20}{\newton}$,
        what is hte magnitude of the torque produced by $\mathbf{F}$?
    \begin{multicols}{2}
    \begin{choices}
        \wrongchoice{\SI{0}{\newton\meter}}}
        \wrongchoice{\SI{1}{\newton\meter}}}
        \wrongchoice{\SI{2}{\newton\meter}}}
        \wrongchoice{\SI{4}{\newton\meter}}}
        \wrongchoice{\SI{10}{\newton\meter}}}
    \end{choices}
    \end{multicols}
\end{question}
}

\element{AP}{
\begin{question}{rotation-Q06}
    In the figure below,
    \begin{center}
    \begin{tikzpicture}
        %% NOTE:
    \end{tikzpicture}
    \end{center}
        what is the torque about the pendulum's suspension point produced by the weight of the bob,
        given that the length of the pendulum, $L$ is \SI{80}{\centi\meter} and $m=\SI{0.50}{\kilo\gram}$?
    \begin{multicols}{2}
    \begin{choices}
        \wrongchoice{\SI{0.5}{\newton\meter}}}
        \wrongchoice{\SI{1.0}{\newton\meter}}}
        \wrongchoice{\SI{1.7}{\newton\meter}}}
        \wrongchoice{\SI{2.0}{\newton\meter}}}
        \wrongchoice{\SI{3.4}{\newton\meter}}}
    \end{choices}
    \end{multicols}
\end{question}
}

\element{AP}{
\begin{question}{rotation-Q07}
    A uniform meter stick of mass \SI{1}{\kilo\gram} is hanging from a thread attached at the stick's midpoint.
    \begin{center}
    \begin{tikzpicture}
        %% NOTE:
    \end{tikzpicture}
    \end{center}
    One block of mass $m=\SI{3}{\kilo\gram}$ hangs from teh left end of the stick,
        and another block, of unknown mass $M$, hangs below the \SI{80}{\centi\meter} mark on the meter stick.
    If the stick remains at rest in the horizontal position shown, what is $M$?
    \begin{multicols}{2}
    \begin{choices}
        \wrongchoice{\SI{4}{\kilo\gram}}}
        \wrongchoice{\SI{5}{\kilo\gram}}}
        \wrongchoice{\SI{6}{\kilo\gram}}}
        \wrongchoice{\SI{8}{\kilo\gram}}}
        \wrongchoice{\SI{9}{\kilo\gram}}}
    \end{choices}
    \end{multicols}
\end{question}
}

\element{AP}{
\begin{question}{rotation-Q08}
    What is the rotatoinal inertia of the follwoing body about the indicated rotation axis?
    (The masses of the connecting rods are neglible.)
    \begin{center}
    \begin{tikzpicture}
        %% NOTE:
    \end{tikzpicture}
    \end{center}
    \begin{multicols}{2}
    \begin{choices}
        \wrongchoice{$4mL^2$}
        \wrongchoice{$\dfrac{32 mL^2}{3}$}
        \wrongchoice{$\dfrac{64 mL^2}{9}$}
        \wrongchoice{$\dfrac{128 mL^2}{9}$}
        \wrongchoice{$\dfrac{256 mL^2}{9}$}
    \end{choices}
    \end{multicols}
\end{question}
}

\element{AP}{
\begin{question}{rotation-Q09}
    The amount of inertia of a solid uniform sphere of mass $M$ and radius $R$ is given by the equation $I=\frac{2}{5}MR^2$.
    Such a sphere is released from rest at the top of an inclined plane of height, $h$, length $L$, and inclined angle $\theta$.
    If the sphere rolls without slipping, find its speed at the bottom of the incline?
    \begin{multicols}{2}
    \begin{choices}
        \wrongchoice{$\sqrt{\dfrac{10}{7}gh}$}
        \wrongchoice{$\sqrt{\dfrac{5}{2}gh}$}
        \wrongchoice{$\sqrt{\dfrac{7}{2}gh}$}
        \wrongchoice{$\sqrt{\dfrac{2}{7}gL\sin\theta}$}
        \wrongchoice{$\sqrt{\dfrac{7}{10}gL\sin\theta}$}
    \end{choices}
    \end{multicols}
\end{question}
}

\element{AP}{
\begin{question}{rotation-Q10}
    An object spins with angular velocity $\omega$.
    If the object's moment of inertia increases by a factor of 2 without the application of an external torque,
        what will be the object's new angular velocity?
    \begin{multicols}{2}
    \begin{choices}
        \wrongchoice{$\dfrac{\omega}{4}$}
        \wrongchoice{$\dfrac{\omega}{2}$}
        \wrongchoice{$\dfrac{\omega}{\sqrt{2}}$}
        \wrongchoice{$\sqrt{2}\omega$}
        \wrongchoice{$2\omega$}
    \end{choices}
    \end{multicols}
\end{question}
}


\endinput


