

%% this section contains XX problems
%%----------------------------------------


%% Jacobs 5 steps to a 5
%%------------------------------
\element{AP}{
\begin{question}{Jacobs-Q26}
    \begin{center}
        \begin{tikzpicture}
        \end{tikzpicture}
    \end{center}
    According to Bernoulli's principle, for a horizontall
        flowing fluid in the thin, closed pipe shown
        above, which of the following statements is valid?
    \begin{multicols}{2}
    \begin{choices}
        \wrongchoice{The fluid pressure is greatest at
                the left end of the pipe.}
        \wrongchoice{The fluid pressure is greatest at
                the right end of the pipe.}
      \correctchoice{The fluid pressure is the same at
                the left end of the pipe as it is at
                the right end of the pipe.}
        \wrongchoice{At the piont in the pipe where
                the fluid's speed is lowest, the fluid
                pressure is greatest.}
        \wrongchoice{At the piont in the pipe where
                the fluid's speed is greatest, the fluid
                pressure is lowest.}
    \end{choices}
    \end{multicols}
\end{question}
}

\element{AP}{
\begin{question}{Jacobs-Q27}
    \begin{center}
        \begin{tikzpicture}
        \end{tikzpicture}
    \end{center}
    Three containers shown above contain the same
        depth of oil.
    Which container experiences the greatest pressure
        at the bottom?
    \begin{multicols}{2}
    \begin{choices}
      \correctchoice{All experience the same pressure at the bottom.}
        \wrongchoice{The answer depends on which has the greater maximum diameter.}
        \wrongchoice{container A}
        \wrongchoice{container B}
        \wrongchoice{container C}
    \end{choices}
    \end{multicols}
\end{question}
}
