

%% Kaplan: AP Physics B and C
%%----------------------------------------


%% AP Physics C: Mechanics
%%--------------------------------------------------
\element{AP}{
\begin{question}{APC-exam02-Q01}
    The graph shows the velocity as a function of time $t$ for an object moving in a straight line.
    \begin{center}
    \begin{tikzpicture}
        %% NOTE:
    \end{tikzpicture}
    \end{center}
    Which of the following graphs shows the corresponding acceleration $a$ as a function of time $t$ for the moving object in the same time interval?
    \begin{multicols}{2}
    \begin{choices}
        \wrongchoice{
            \begin{tikzpicture}
            \end{tikzpicture}
        }
    \end{choices}
    \end{multicols}
\end{question}
}

\element{AP}{
\begin{question}{APC-exam02-Q02}
    A \SI{3}{\kilo\gram} object has a velocity at one instant of \SI{2i}{\meter\per\second}.
    Five seconds later, the object's velocity changes to 

    Find the components of the force acting on the object?
    \begin{multicols}{2}
    \begin{choices}
        \wrongchoice{$ $}
    \end{choices}
    \end{multicols}
\end{question}
}

\element{AP}{
\begin{question}{APC-exam02-Q03}
    A \SI{3}{\kilo\gram} object has a velocity at one instant of \SI{2i}{\meter\per\second}.
    Five seconds later, the object's velocity changes to 

    What is the magnitude of this force?
    \begin{multicols}{2}
    \begin{choices}
        \wrongchoice{\SI{4.24}{\newton}}
        \wrongchoice{\SI{1.90}{\newton}}
        \wrongchoice{\SI{2.54}{\newton}}
        \wrongchoice{\SI{12.73}{\newton}}
        \wrongchoice{\SI{8.48}{\newton}}
    \end{choices}
    \end{multicols}
\end{question}
}

\element{AP}{
\begin{question}{APC-exam02-Q04}
    A particle is moving with a velocity $v=\SI{40}{\meter\per\second}$ at $t=0$.
    Between $t=0$ and $t=\SI{10}{\second}$,
        the velocity decreases uniformly to zero.
    What is the particle's average acceleration during this time?
    \begin{multicols}{2}
    \begin{choices}
        \wrongchoice{\SI{-4}{\meter\per\second\squared}}
        \wrongchoice{\SI{0.25}{\meter\per\second\squared}}
        \wrongchoice{\SI{4}{\meter\per\second\squared}}
        \wrongchoice{\SI{-0.25}{\meter\per\second\squared}}
        \wrongchoice{\SI{400}{\meter\per\second\squared}}
    \end{choices}
    \end{multicols}
\end{question}
}

\element{AP}{
\begin{question}{APC-exam02-Q05}
    A particle moves along the $x$-axis with a displacement represented by the following:
    \begin{displaymath}
        x = \num{5} t + \num{9} t^2.
    \end{displaymath}
    What is the instantaneous velocity of this particle at some time $t$?
    \begin{choices}
        \wrongchoice{$\num{5} t + \num{9} t^2$}
        \wrongchoice{$\num{5} t + \num{27} t^2$}
        \wrongchoice{$\num{15} t + \num{27} t^2$}
        \wrongchoice{$\num{5} + \num{27} t^2$}
        \wrongchoice{$\num{5} t + \num{27} t^3$}
    \end{choices}
\end{question}
}

\element{AP}{
\begin{question}{APC-exam02-Q06}
    A particle moves along the $x$-axis with a displacement represented by the following:
    \begin{displaymath}
        x = \num{5} t + \num{9} t^2.
    \end{displaymath}
    What would be the instantaneous acceleration at $t=\SI{2}{\second}$?
    \begin{multicols}{2}
    \begin{choices}
        \wrongchoice{\SI{27}{\meter\per\second\squared}}
        \wrongchoice{\SI{108}{\meter\per\second\squared}}
        \wrongchoice{\SI{54}{\meter\per\second\squared}}
        \wrongchoice{\SI{5}{\meter\per\second\squared}}
        \wrongchoice{\SI{10}{\meter\per\second\squared}}
    \end{choices}
    \end{multicols}
\end{question}
}

\element{AP}{
\begin{question}{APC-exam02-Q07}
    A person jumps from a bridge spanning a rushing river,
        falling a distance $h$ before hitting the water.
    Neglecting air resistance,
        what would be the person's speed at the moment of impact?
    \begin{multicols}{2}
    \begin{choices}
        \wrongchoice{$v=\sqrt{2gh}$}
        \wrongchoice{$v=\sqrt{mgh}$}
        \wrongchoice{$v=\sqrt{\dfrac{gh}{2}}$}
        \wrongchoice{$v=\sqrt{\dfrac{mgh}{2}}$}
        \wrongchoice{$v=\sqrt{2mgh}$}
    \end{choices}
    \end{multicols}
\end{question}
}

\element{AP}{
\begin{question}{APC-exam02-Q08}
    The position of a ball thrown vertically up is described by the equation
    \begin{displaymath}
        y=\num{6}t - \num{2.8} t^2,
    \end{displaymath}
    where $y$ is in meters and $t$ is in seconds.
    What is the initial speed of the ball at $t=0$?
    \begin{multicols}{2}
    \begin{choices}
        \wrongchoice{\SI{-5.6}{\meter\per\second}}
        \wrongchoice{\SI{-2.8}{\meter\per\second}}
        \wrongchoice{\SI{-6}{\meter\per\second}}
        \wrongchoice{\SI{6}{\meter\per\second}}
        \wrongchoice{\SI{2.8}{\meter\per\second}}
    \end{choices}
    \end{multicols}
\end{question}
}

\element{AP}{
\begin{question}{APC-exam02-Q09}
    The position of a ball thrown vertically up is described by the equation
    \begin{displaymath}
        y=\num{6}t - \num{2.8} t^2,
    \end{displaymath}
    where $y$ is in meters and $t$ is in seconds.
    What is the speed of theball at time $t$?
    \begin{multicols}{2}
    \begin{choices}
        \wrongchoice{$\num{6} t - \num{5.6} t^2$}
        \wrongchoice{$\num{6} t - \num{2.8} t$}
        \wrongchoice{$\num{6} - \num{5.6} t$}
        \wrongchoice{$\num{6} - \num{2.8} t^2$}
        \wrongchoice{$\num{6} t^2 - \num{2.8} t$}
    \end{choices}
    \end{multicols}
\end{question}
}

\element{AP}{
\begin{question}{APC-exam02-Q10}
    The equation of motion of a simple harmonic oscillator is
    \begin{displaymath}
        \frac{\mathrm{d}^2x}{\mathrm{d}t^2} = \num{-25} x,
    \end{displaymath}
    where $x$ is the displacement and $t$ is time.
    The frequency of this oscillator will be:
    \begin{multicols}{2}
    \begin{choices}
        \wrongchoice{$\frac{5}{\pi}$}
        \wrongchoice{$\frac{5}{2\pi}$}
        \wrongchoice{$10\pi$}
        \wrongchoice{$\frac{2\pi}{5}$}
        \wrongchoice{$5\pi$}
    \end{choices}
    \end{multicols}
\end{question}
}

\element{AP}{
\begin{question}{APC-exam02-Q11}
    On planet $X$, an object weighs \SI{5}{\newton}.
    On planet $Y$, where the acceleration due to gravity is $\num{1.5}g$,
        the object weighs \SI{150}{\newton}.

    Using the information provided,
        determine the object's mass.
    \begin{multicols}{2}
    \begin{choices}
        \wrongchoice{\SI{5.00}{\kilo\gram}}
        \wrongchoice{\SI{1.50}{\kilo\gram}}
        \wrongchoice{\SI{15.0}{\kilo\gram}}
        \wrongchoice{\SI{10.0}{\kilo\gram}}
        \wrongchoice{\SI{0.500}{\kilo\gram}}
    \end{choices}
    \end{multicols}
\end{question}
}

\element{AP}{
\begin{question}{APC-exam02-Q12}
    On planet $X$, an object weighs \SI{5}{\newton}.
    On planet $Y$, where the acceleration due to gravity is $\num{1.5}g$,
        the object weighs \SI{150}{\newton}.

    What is the acceleration due to gravity on the surface of planet $X$?
    \begin{multicols}{2}
    \begin{choices}
        \wrongchoice{\SI{5.00}{\meter\per\second\squared}}
        \wrongchoice{\SI{0.500}{\meter\per\second\squared}}
        \wrongchoice{\SI{10.0}{\meter\per\second\squared}}
        \wrongchoice{\SI{15.0}{\meter\per\second\squared}}
        \wrongchoice{\SI{150}{\meter\per\second\squared}}
    \end{choices}
    \end{multicols}
\end{question}
}

\element{AP}{
\begin{question}{APC-exam02-Q13}
    An object of mass $m$ orbits a planet with mass $M$ at a distance $R$,
        measured from the center of the planet.
    \begin{center}
    \begin{tikzpicture}
        %% NOTE:
    \end{tikzpicture}
    \end{center}
    Assuming that the object's orbit is stable, calculate its orbital speed.
    \begin{multicols}{2}
    \begin{choices}
        \wrongchoice{$v = \sqrt{\dfrac{GMm}{R}}$}
        \wrongchoice{$v = \sqrt{\dfrac{GM}{R}}$}
        \wrongchoice{$v = \sqrt{\dfrac{Gm}{R}}$}
        \wrongchoice{$v = \sqrt{\dfrac{GMm}{R^2}}$}
        \wrongchoice{$v = \sqrt{GMR}$}
    \end{choices}
    \end{multicols}
\end{question}
}

\element{AP}{
\begin{question}{APC-exam02-Q14}
    A car of mass $m$ is moving down a flat road drives around a curve with a radius $r$.
    \begin{center}
    \begin{tikzpicture}
        %% NOTE:
    \end{tikzpicture}
    \end{center}
    If $\mu$ is the coefficient of static friction between the tires and the dry road,
        determine the maximum speed at which the car can travel such that it can make the turn and stay on the road.
    \begin{multicols}{2}
    \begin{choices}
        \wrongchoice{$\sqrt{\mu g r}$}
        \wrongchoice{$\sqrt{\dfrac{\mu g}{r}}$}
        \wrongchoice{$\sqrt{\dfrac{m g}{\mu r}}$}
        \wrongchoice{$\sqrt{\dfrac{m g r}{\mu}}$}
        \wrongchoice{$\sqrt{\dfrac{m \mu}{g r}}$}
    \end{choices}
    \end{multicols}
\end{question}
}

\element{AP}{
\begin{question}{APC-exam02-Q15}
    A \SI{10}{\kilo\gram} particle moves along the $x$-axis.
    Its position varies with time according to $x=2t + t^3$ where $x$ is in meters
        and $t$ is in seconds.

    What is the kinetic energy of this particle at time $t$?
    \begin{multicols}{2}
    \begin{choices}
        \wrongchoice{$\left(20 t + 10 t^3 \right)\,\mathrm{J}$}
        \wrongchoice{$\left(20 t + 30 t^2 \right)\,\mathrm{J}$}
        \wrongchoice{$\left(4 + 12 t^2 + 9 t^4 \right)\,\mathrm{J}$}
        \wrongchoice{$\left(20 t + 60 t^2 + 45 t^4 \right)\,\mathrm{J}$}
        \wrongchoice{$\left(2 t + t^3 \right)\,\mathrm{J}$}
    \end{choices}
    \end{multicols}
\end{question}
}

\element{AP}{
\begin{question}{APC-exam02-Q16}
    A \SI{10}{\kilo\gram} particle moves along the $x$-axis.
    Its position varies with time according to $x=2t + t^3$ where $x$ is in meters
        and $t$ is in seconds.

    What power is required to move this particle at time $t$?
    \begin{multicols}{2}
    \begin{choices}
        \wrongchoice{$\left(120 t + 180 t^3 \right)\,\mathrm{W}$}
        \wrongchoice{$\left(24 t + 36 t^3 \right)\,\mathrm{W}$}
        \wrongchoice{$\left(20 + 120 t + 180 t^4 \right)\,\mathrm{W}$}
        \wrongchoice{$\left(60 t \right)\,\mathrm{W}$}
        \wrongchoice{$\left(t + 3 t^2 \right)\,\mathrm{W}$}
    \end{choices}
    \end{multicols}
\end{question}
}

\element{AP}{
\begin{question}{APC-exam02-Q17}
    A force $F=\left(2\mathbf{i} - \mathbf{j}\right)\,\mathrm{N}$
        acts on a particle that experiences a displacement that can be  
        represented by $s=\left(\mathbf{i} - 3\mathbf{j}\right)\,\mathrm{m}$.

    Find the magnitude of the work that was done by the force $F$.
    \begin{multicols}{2}
    \begin{choices}
        \wrongchoice{\SI{\sqrt{5}}{\joule}}
        \wrongchoice{\SI{2}{\joule}}
        \wrongchoice{\SI{\sqrt{3}}{\joule}}
        \wrongchoice{\SI{2\sqrt{5}}{\joule}}
        \wrongchoice{\SI{5}{\joule}}
    \end{choices}
    \end{multicols}
\end{question}
}

\element{AP}{
\begin{question}{APC-exam02-Q18}
    A force $F=\left(2\mathbf{i} - \mathbf{j}\right)\,\mathrm{N}$
        acts on a particle that experiences a displacement that can be  
        represented by $s=\left(\mathbf{i} - 3\mathbf{j}\right)\,\mathrm{m}$.

    What is the angle between the two vectors?
    \begin{multicols}{2}
    \begin{choices}
        \wrongchoice{\ang{90}}
        \wrongchoice{\ang{180}}
        \wrongchoice{\ang{45}}
        \wrongchoice{\ang{30}}
        \wrongchoice{\ang{60}}
    \end{choices}
    \end{multicols}
\end{question}
}

\element{AP}{
\begin{question}{APC-exam02-Q19}
    Two skaters are facing each other,
        initially at rest.
    The skaters push on each other and move away in opposite directions at constant velocities.
    What can be said about the total momentum of the skaters after they push away from each other?
    \begin{choices}
        \wrongchoice{The sum of the final momentum is less than the sum of the initial momentum}
        \wrongchoice{The sum of the final momentum is greater than the sum of the initial momentum}
        \wrongchoice{The sum of the final momentum is exactly half the sum of the initial momentum.}
        \wrongchoice{The final total momentum is zero.}
        \wrongchoice{Each individual skater's momentum equals the sum of the total initial momentum.}
    \end{choices}
\end{question}
}

\element{AP}{
\begin{question}{APC-exam02-Q20}
    A disk with a mass of $3m$ is moving horizontally to the right with a speed $V$ on a frictionless surface.
    It collides head on with another disk with a mass $4m$ traveling to the left at a speed $2v$.
    Upon impact, the two disks stick together.
    Find the final speed of the combined masses following the collision.
    \begin{multicols}{2}
    \begin{choices}
        \wrongchoice{$-v$}
        \wrongchoice{$\frac{11}{7} v$}
        \wrongchoice{$\frac{-3}{7} v$}
        \wrongchoice{$\frac{-5}{12} v$}
        \wrongchoice{$\frac{-5}{7} v$}
    \end{choices}
    \end{multicols}
\end{question}
}




\endinput

