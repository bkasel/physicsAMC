

%% this section contains XX problems

%% AP Physics B practice workbook
%%--------------------------------------------------

%% Section B: Fluid Mechanics
%%--------------------------------------------------

\element{AP}{
\begin{question}{fluids-Q01}
    A car of mass $m$, traveling at speed $v$,
        stops in time $t$ when maximum braking force is applied. 
    Assuming the braking force is independent of mass,
        what time would be required to stop a car of mass $2m$ traveling at speed $v$?
    \begin{multicols}{2}
    \begin{choices}
        \wrongchoice{$\frac{1}{2} t$}
        \wrongchoice{$t$}
        \wrongchoice{$\sqrt{2} t$}
      \correctchoice{$2 t$}
        \wrongchoice{$4 t$}
    \end{choices}
    \end{multicols}
\end{question}
}



1. A cork has weight mg and density 25% of water density. A string is tied around the
cork and attached to the bottom of a water-filled container. The cork is totally
immersed. Express in terms of the cork weight mg, the tension in the string
A) 0
B) mg
C) 2mg
D) 3mg
E) 4mg
2. Which of the following is the best statement of Pascals Law?
A) pressure on a confined liquid is transmitted equally in all directions
B) a numerical arrangement where each number is the sum of the two numbers above
C) two electrons cannot occupy the same quantum state at the same time
D) the volume of a gas is directly related to its temperature
E) the farther away a galaxy is the faster it is receding
3. When submerged under water, the apparent mass of one cubic meter of pure gold is 18300 kg. What would be
its mass in air?
A) 16300 kg
B) 17300 kg C) 18300 kg
D) 19300 kg
E) 20300 kg
4. An ideal fluid flows through a long horizontal circular pipe. In one region of the pipe, it has radius R. The pipe
then widens to radius 2R. What is the ratio of the fluids speed in the region of radius R to the speed of the fluid
in region with radius 2R
A) 1⁄4
B) 1⁄2
C) 1
D) 2
E) 4
5. A fluid is forced through a pipe of changing cross section as shown. In which section would the pressure of the
fluid be a minimum
A) I
B) II
C) III
D) IV
E) all section have the same pressure.
6. Three fishing bobbers all float on top of water. They have the following relationships:
-A,B: same mass, same density, different shapes
-B,C: same size, same shape,
mass & density C < mass & density B
Three identical weights are tied to each bob, and each is pulled completely beneath the water. Which bob will
displace the greatest amount of water
A) A
B) B
C) C
D) A and B
E) All displace the same amount of water.
7. A hydraulic press allows large masses to be lifted with small forces as a result of which principle?
A) Pascal’s
B) Bernoulli’s
C) Archimedes’
D) Huygens’
E) Newton’s



8. A 500 N weight sits on the small piston of a hydraulic machine. The small piston has an area of 2 cm 2 . If the
large piston has an area of 40 cm 2 , how much weight can the large piston support?
A) 25 N
B) 500 N
C) 10000 N
D) 40000 N
9. As a rock sinks deeper and deeper into water of constant density, what happens to the buoyant force on it?
A) It increases.
B) It remains constant.
C) It decreases.
D) It may increase or decrease, depending on the shape of the rock.
10. 50 cm3 of wood is floating on water, and 50 cm3 of iron is totally submerged. Which has the greater buoyant
force on it?
A) The wood.
B) The iron.
C) Both have the same buoyant force.
D) Cannot be determined without knowing their densities.
11. Salt water is more dense than fresh water. A ship floats in both fresh water and salt water. Compared to the
fresh water, the amount of water displaced in the salt water is
A) more.
B) less.
C) the same.
D) Cannot be determined from the information given.
12. A liquid has a specific gravity of 0.357. What is its density?
C) 1000 kg/m 3
A) 357 kg/m3
B) 643 kg/m 3
D)
3570 kg/m 3
13. Water flows through a pipe. The diameter of the pipe at point B is larger than at point A. Where is the water
pressure greater?
A) Point A
B) Point B
C) Same at both A and B
D) Cannot be determined from the information given.
14. Liquid flows through a 4 cm diameter pipe at 1.0 m/s. There is a 2 cm diameter restriction in the line. What is
the velocity in this restriction?
A) 0.25 m/s
B) 0.50 m/s
C) 2 m/s
D) 4 m/s
15. A copper block is connected to a string and submerged in a container of water.
Position 1: The copper is completely submerged, but just under the surface of the water.
Position 2: The copper is completely submerged, mid-way between the water surface and the bottom of the
container.
Position 3: The copper is completely submerged, but just above the bottom surface of the container.
Assume that the water is incompressible. What is the ranking of the buoyant forces (B) acting on the copper
blocks for these positions, from least to greater?
(A) B 1 < B 2 < B 3
(B) B 3 < B 2 < B 1
(C) B 1 = B 2 = B 3
(D) B 1 < B 2 = B 3
(E) B 3 < B 1 = B 2




16. Two objects labeled K and L have equal mass but densities 0.95D o and D o , respectively. Each of these objects
floats after being thrown into a deep swimming pool. Which is true about the buoyant forces acting on these
objects?
(A) The buoyant force is greater on Object K since it has a lower density and displaces more water.
(B) The buoyant force is greater on Object K since it has lower density and lower density objects always float
“higher” in the fluid.
(C) The buoyant force is greater on Object L since it is denser than K and therefore “heavier.”
(D) The buoyant forces are equal on the objects since they have equal mass.
(E) Without knowing the specific gravity of the objects, nothing can be determined.
17. A driveway is 22.0 m long and 5.0 m wide. If the atmospheric pressure is 1.0 x 10 5 Pa, How much force does
the atmosphere exert on the driveway?
(A) 9.09 x 10 –8 N
(B) 1.1 x 10 –3 N
(C) 909 N
(D) 4545 N
(E) 1.1 x 10 7 N
18. Which of the following could be a correct unit for pressure?
( A )
kg
m 2
( B )
kg
m ⋅ s
( C )
kg
s 2
( D )
kg
m ⋅ s 2
( E )
m ⋅ s
kg
19. A block is connected to a light string attached to the bottom of a large
container of water. The tension in the string is 3.0 N. The gravitational
force from the earth on the block is 5.0 N. What is the block’s volume?
(A) 2.0×10 −4 m 3
(B) 3.0×10 −4 m 3
(C) 5.0×10 −4 m 3
(D) 8.0×10 −4 m 3
(E) 1.0×10 −3 m 3
20. A cube of unknown material and uniform density floats in a container of water with 60% of its volume
submerged. If this same cube were placed in a container of oil with density 800 kg/m 3 , what portion of the
cube’s volume would be submerged while floating?
(A) 33% (B) 50% (C) 58% (D) 67% (E) 75%
21. A piece of an ideal fluid is marked as it moves
along a horizontal streamline through a pipe, as
shown in the figure. In Region I, the speed of the
fluid on the streamline is V . The cylindrical,
horizontal pipe narrows so that the radius of the
pipe in Region II is half of what it was in Region
I. What is the speed of the marked fluid when it
is in Region II?
(A) 4V
(B) 2V
(C) V
(D) V/2
(E) V/4



22. A fluid flows steadily from left to right in the pipe
shown. The diameter of the pipe is less at point 2 then at
point 1, and the fluid density is constant throughout the
pipe. How do the velocity of flow and the pressure at
points 1 and 2 compare?
Velocity
(A) v1 < v2
(B) v1 < v2
(C) v1 = v2
(D) v1 > v2
(E) v1 > v2
Pressure
p1 = p2
p1 > p2
p1 < p2
p1 = p2
p1 > p2
23. The figure shows an object of mass 0.4 kg that is suspended from a scale and submerged in
a liquid. If the reading on the scale is 3 N, then the buoyant force that the fluid exerts on
the object is most nearly
(A) 1.3 N
(B) 1.0N
(C) 0.75 N
(D) 0.33 N
(E) 0.25 N
24. Each of the beakers shown is filled to the same depth h with liquid
of density ρ. The area A of the flat bottom is the same for each
beaker. Which of the following ranks the beakers according to the
net downward force exerted by the liquid on the flat bottom, from
greatest to least force?
(A) I, III, II, IV
(B) I, IV, III, II
(C) II, III, IV, I
(D) IV, III, I, II
(E) None of the above, the force on each is the same.
25. A T-shaped tube with a constriction is inserted in a vessel containing a
liquid, as shown. What happens if air is blown through the tube from the
left, as shown by the arrow in the diagram?
(A) The liquid level in the tube rises to a level above the surface of the
liquid in the surrounding tube
(B) The liquid level in the tube falls below the level of the surrounding
liquid
(C) The liquid level in the tube remains where it is
(D) The air bubbles out at the bottom of the tube.
(E) Any of the above depending on how hard the air flows.
26. A spring scale calibrated in kilograms is used to determine the density of a rock specimen. The reading on the
spring scale is 0.45 kg when the specimen is suspended in air and 0.36 kg when the specimen is fully
submerged in water. If the density of water is 1000 kg/m 3 , the density of the rock specimen is
(A) 2.0 x 10 2 kg/m 3 (B) 8.0 x 10 2 kg/m 3 (C) 1.25 x 10 3 kg/m 3 (D) 4.0 x 10 3 kg/m 3 (E) 5.0 x 10 3 kg/m 3



Questions 27-28: Refer to the diagram below and use 10 m/s 2 for g and 100,000 N/m 2 for 1 atm.
15 m
15 m
15 m
27. The pressure at A is 9.5 atm and the water velocity is 10 m/s. What is the water velocity at point C?
(a) 2.5 m/s
(b) 5 m/s
(c) 10 m/s
(d) 20 m/s
(e) 40 m/s
28. The pressure at C is
(a) 0 N/m 2
(b) 100,000 N/m 2
(c) 150,000 N/m 2
(d) 800,000 N/m 2
(e) 1,100,000 N/m 2
29. One cubic centimeter of iron (density ~7.8 g/cm 3 ) and 1 cubic centimeter of aluminum (density ~2.7 g/cm 3 ) are
dropped into a pool. Which has the largest buoyant force on it?
(a) iron
(b) aluminum
(c) both are the same.
(d) neither has a buoyant force on it.
30. One kilogram of iron (density ~7.8 g/cm 3 ) and 1 kilogram of aluminum (density ~2.7 g/cm 3 ) are dropped into a
pool. Which has the largest buoyant force on it?
(a) iron
(b) aluminum
(c) both are the same.
(d) neither has a buoyant force on it.
31. Find the approximate minimum mass needed for a spherical ball with a 40 cm radius to sink in a liquid of
density 1.4x10 3 kg/m 3
(a) 37.5 kg (b) 375 kg
(c) 3750 kg
(d) 37500 kg
(e) 375000 kg
32. What vertical percentage of a 0.25 m deep sheet of ice, whose density is 0.95x10 3 kg/m 3 , will be visible in an
ocean whose density is 1.1x10 3 kg/m 3
(a) 14%
(b) 34%
(c) 58%
(d) 71%
(e) 87%
33. The idea that the velocity of a fluid is high when pressure is low and that the velocity of a fluid is low when the
pressure is high embodies a principle attributed to
(a) Torricelli
(b) Pascal
(c) Galileo
(d) Archimedes
(e) Bernoulli
34. The mass of a 1.3 m 3 object with a specific gravity of 0.82 is
(a) 630 kg (b) 730 kg
(c) 820 kg
(d) 1100 kg
(e) 1600 kg
35. The apparent weight of a 600 kg object of volume 0.375 m 3 submerged in a liquid of density 1.25x10 3 kg/m 3 is
(a) 180 N
(b) 250 N
(c) 480 N
(d) 1300 N
(e) 4700 N
36. A conduit of radius 7R carries a uniformly dense liquid to a spigot of radius R at the same height, where it has a
velocity of V. What is its initial velocity
(a) 0.02V
(b) 0.11V
(c) V
(d) 7V
(e) 49V



37. The pressure in a pipe carrying a liquid with a density of ρ and an initial velocity v at the inlet is P, which is y
meters lower than its outlet, which has a velocity of 2v. In these terms, what is the final pressure?
P
ρ ( 3 v 2 + 2 gy )
2
1
( B ) P − ρ ( 3 v 2 + 2 gy )
2
1
( C ) P + ρ ( − 3 v 2 + ρ gy )
2
1
ρ ( v 2 − 4 v 2 ) − ρ gy
2
( D )
P
 1

( E ) P  ρ ( v 2 − 4 v 2 ) − ρ gy 
 2

( A )
38. The units of specific gravity are
(b) g/m 3
(c) m/s 2
(a) kg/m 3
(d) N/m
(e) none of the above
39. The buoyant force on an object is equal to the weight of the water displaced by a submerged object. This is a
principle attributed to
(a) Torricelli
(b) Pascal
(c) Galileo
(d) Archimedes
(e) Bernoulli
40. If the gauge pressure of a device reads 2.026x10 5 N/m 2 , the absolute pressure it is measuring is
(a) 1.013 x 10 5 N/m 2
(b) 2.052 x 10 5 N/m 2
(c) 2.026 x10 5 N/m 2
(d) 3.039 x 10 5 N/m 2
(e) 6.078 x 10 5 N/m 2
41. A block of mass m, density ρ B , and volume V is completely submerged in a liquid of density ρ L . The density of
the block is greater than the density of the liquid. The block
(a) floats, because ρ B > ρ L
(b) experiences a buoyant force equal to ρ B gV.
(c) experiences a buoyant force equal to ρ L gV.
(d) experiences a buoyant force equal to m B g
(e) does not experience any buoyant force, because ρ B > ρ L .
42. A river gradually deepens, from a depth of 4 m to a depth of 8 m as shown. The width, W, of the river does not
change. At the depth of 4 m, the river’s speed is 12 m/sec. Its velocity at the 8 m depth is
(a) 12 m/sec
332
(b) 24 m/sec
(c) 6 m/sec
(d) 8 m/sec
(e) 16 m/sec



43. In the open manometer shown, water occupies a part of the left arm, from a height of y 1 to a height of y 2 . The
remainder of the left arm, the bottom of the tube, and the right arm to a height of y 3 are filled with mercury.
Which of the following is correct?
(a) the pressure at a height y 3 is the same in both arms.
(b) the pressure at a height y 2 is the same in both arms.
(c) the pressure at the bottom of the right arm is greater than at the bottom of the left arm.
(d) the pressure at a height y 3 is less in the left arm than in the right arm.
(e) the pressure at a height y 1 is greater in the left arm than in the right arm.
44. Water flows in a pipe of uniform cross-sectional area A.
The pipe changes height from y 1 = 2 meters to y 2 = 3 meters. Since the areas are the same, we can say v 1 = v 2 .
Which of the following is true?
(a) P 1 = P 2 + ρg(y 2 – y 1 )
(b) P 1 = P 2
(c) P 1 = 0
(d) P 2 = 0
(e) ρ 1 > ρ 2
45. A vertical force of 30 N is applied uniformly to a flat button with a radius of 1 cm that is lying on a table. Which
of the following is the best order of magnitude estimate for the pressure applied to the button?
(A) 10 Pa
(B) 10 2 Pa
(C) 10 3 Pa
(D) 10 4 Pa
(E) 10 5 Pa




A ball that can float on water has mass 5.00 kg and volume 2.50 x 10 -2 m 3 . What is the magnitude of the
downward force that must be applied to the ball to hold it motionless and completely submerged in freshwater
of density 1.00 x 10 3 kg/m 3 ?
(A) 20.0 N
(B) 25.0 N
(C) 30.0 N
(D) 200 N
(E) 250 N
47. Water flows through the pipe shown. At the larger end, the pipe has diameter D and the speed of the water is v 1 .
What is the speed of the water at the smaller end, where the pipe has diameter d ?
( A ) v 1
334
( B )
d
v 1
D
( C )
D
v 1
d
( D )
d 2
v 1
D 2
( E )
D 2
v 1
d 2





\endinput


