

%% this section contains XX problems

%% AP Physics B practice workbook
%%--------------------------------------------------

%% Section B: Momentum
%%--------------------------------------------------

%% Page 217
\element{AP}{
\begin{question}{momentum-Q01}
    A car of mass $m$, traveling at speed $v$,
        stops in time $t$ when maximum braking force is applied. 
    Assuming the braking force is independent of mass,
        what time would be required to stop a car of mass $2m$ traveling at speed $v$?
    \begin{multicols}{2}
    \begin{choices}
        \wrongchoice{$\frac{1}{2} t$}
        \wrongchoice{$t$}
        \wrongchoice{$\sqrt{2} t$}
      \correctchoice{$2 t$}
        \wrongchoice{$4 t$}
    \end{choices}
    \end{multicols}
\end{question}
}

\element{AP}{
\begin{question}{momentum-Q02}
    A block of mass $M$ is initially at rest on a frictionless floor. 
    The block, attached to a massless spring with spring constant $k$,
        is initially at its equilibrium position. 
    An arrow with mass $m$ and velocity $v$ is shot into the block.
    The arrow sticks in the block. 
    What is the maximum compression of the spring?
    \begin{multicols}{2}
    \begin{choices}
        \wrongchoice{$v \sqrt{\dfrac{m}{k}}$}
        \wrongchoice{$v \sqrt{\dfrac{k}{m}}$}
        \wrongchoice{$v \sqrt{\dfrac{m+M}{k}}$}
        \wrongchoice{$\dfrac{v(m+M)}{k\sqrt{m}}$}
      \correctchoice{$\dfrac{mv}{k\sqrt{m+M}}$}
    \end{choices}
    \end{multicols}
\end{question}
}

\element{AP}{
\begin{question}{momentum-Q03}
    How long must a \SI{100}{\newton} net force act to produce a change in momentum of \SI{200}{\kilo\gram\meter\per\second}?
    \begin{multicols}{2}
    \begin{choices}
        \wrongchoice{\SI{0.25}{\second}}
        \wrongchoice{\SI{0.50}{\second}}
        \wrongchoice{\SI{1.0}{\second}}
      \correctchoice{\SI{2.0}{\second}}
        \wrongchoice{\SI{4.0}{\second}}
    \end{choices}
    \end{multicols}
\end{question}
}

\element{AP}{
\begin{question}{momentum-Q04}
    Which is a vector quantity:
    \begin{multicols}{2}
    \begin{choices}
        \wrongchoice{energy}
        \wrongchoice{mass}
      \correctchoice{impulse}
        \wrongchoice{power}
        \wrongchoice{work}
    \end{choices}
    \end{multicols}
\end{question}
}

\element{AP}{
\begin{question}{momentum-Q05}
    When the velocity of a moving object is doubled,
        its \rule[-0.1pt]{4em}{0.1pt} is also doubled.
    \begin{choices}
        \wrongchoice{acceleration}
        \wrongchoice{kinetic energy}
        \wrongchoice{mass}
      \correctchoice{momentum}
        \wrongchoice{potential energy}
    \end{choices}
\end{question}
}

\element{AP}{
\begin{question}{momentum-Q06}
    Two objects, $P$ and $Q$,
        have the same momentum. 
    $Q$ can have more kinetic energy than $P$ if it has:
    \begin{choices}
        \wrongchoice{More mass than $P$}
        \wrongchoice{The same mass as $P$}
      \correctchoice{More speed than $P$}
        \wrongchoice{The same speed at $P$ }
        \wrongchoice{$Q$ can not have more kinetic energy than $P$}
    \end{choices}
\end{question}
}

\element{AP}{
\begin{question}{momentum-Q07}
    A spring is compressed between two objects with unequal masses, $m$ and $M$,
        and held together. 
    The objects are initially at rest on a horizontal frictionless surface. 
    When released, which of the following is true?
    \begin{choices}
        \wrongchoice{Kinetic energy is the same as before begin released.}
        \wrongchoice{The total final kinetic energy is zero.}
        \wrongchoice{The two objects have equal kinetic energy.}
        \wrongchoice{The speed of one object is equal to the speed of the other.}
      \correctchoice{The total final momentum of the two objects is zero.}
    \end{choices}
\end{question}
}

\element{AP}{
\begin{question}{momentum-Q08}
    Net Impulse is best related to:
    \begin{choices}
        \wrongchoice{momentum}
      \correctchoice{change in momentum}
        \wrongchoice{kinetic energy}
        \wrongchoice{change in kinetic energy}
        \wrongchoice{none of the provided}
    \end{choices}
\end{question}
}

\element{AP}{
\begin{question}{momentum-Q09}
    Two football players with mass \SI{75}{\kilo\gram} and \SI{100}{\kilo\gram} run directly toward each other with speeds of \SI{6}{\meter\per\second} and \SI{8}{\meter\per\second} respectively. 
    If they grab each other as they collide,
        the combined speed of the two players just after the collision would be:
    \begin{multicols}{3}
    \begin{choices}
      \correctchoice{\SI{2}{\meter\per\second}}
        \wrongchoice{\SI{3.4}{\meter\per\second}}
        \wrongchoice{\SI{4.6}{\meter\per\second}}
        \wrongchoice{\SI{7.1}{\meter\per\second}}
        \wrongchoice{\SI{8}{\meter\per\second}}
    \end{choices}
    \end{multicols}
\end{question}
}

\element{AP}{
\begin{question}{momentum-Q10}
    A \SI{30}{\kilo\gram} child who is running at \SI{4}{\meter\per\second} jumps onto a stationary \SI{10}{\kilo\gram} skateboard. 
    The speed of the child and the skateboard is approximately:
    \begin{multicols}{3}
    \begin{choices}
      \correctchoice{\SI{3}{\meter\per\second}}
        \wrongchoice{\SI{4}{\meter\per\second}}
        \wrongchoice{\SI{5}{\meter\per\second}}
        \wrongchoice{\SI{6}{\meter\per\second}}
        \wrongchoice{\SI{7}{\meter\per\second}}
    \end{choices}
    \end{multicols}
\end{question}
}

\element{AP}{
\begin{question}{momentum-Q11}
    A \SI{5000}{\kilo\gram} freight car moving at \SI{4}{\kilo\meter\per\hour} collides and couples with an \SI{8000}{\kilo\gram} freight car which is initially at rest. 
    The approximate common final speed of these two cars is:
    \begin{multicols}{3}
    \begin{choices}
        \wrongchoice{\SI{1}{\kilo\meter\per\hour}}
        \wrongchoice{\SI{1.3}{\kilo\meter\per\hour}}
      \correctchoice{\SI{1.5}{\kilo\meter\per\hour}}
        \wrongchoice{\SI{2.5}{\kilo\meter\per\hour}}
        \wrongchoice{\SI{4}{\kilo\meter\per\hour}}
    \end{choices}
    \end{multicols}
\end{question}
}

\element{AP}{
\begin{question}{momentum-Q12}
    A rubber ball is held motionless a height $h_0$ above a hard floor and released. 
    Assuming that the collision with the floor is elastic,
        which one of the following graphs best shows the relationship between the total energy $E$ of the ball and its height $h$ above the surface.
    \begin{multicols}{2}
    \begin{choices}
        %%  NOTE: ans is E
        \wrongchoice{
            \begin{tikzpicture}
            \end{tikzpicture}
        }
    \end{choices}
    \end{multicols}
\end{question}
}

\element{AP}{
\begin{question}{momentum-Q13}
    Two carts are held together. 
    Cart 1 is more massive than Cart 2. 
    As they are forced apart by a compressed spring between them,
        which of the following will have the same magnitude for both carts.
    \begin{choices}
        \wrongchoice{acceleration}
        \wrongchoice{change of velocity}
      \correctchoice{force}
        \wrongchoice{speed}
        \wrongchoice{velocity}
    \end{choices}
\end{question}
}

\element{AP}{
\begin{question}{momentum-Q14}
    If the unit for force is $F$,
        the unit for velocity is $v$ and the unit for time $t$,
        then the unit for momentum is:
    \begin{multicols}{2}
    \begin{choices}
      \correctchoice{$Ft$}
        \wrongchoice{$Ftv$}
        \wrongchoice{$Ft^2 v$}
        \wrongchoice{$\dfrac{Ft}{v}$}
        \wrongchoice{$\dfrac{Fv}{t}$}
    \end{choices}
    \end{multicols}
\end{question}
}

\element{AP}{
\begin{question}{momentum-Q15}
    A ball with a mass of \SI{0.50}{\kilo\gram} and a speed of \SI{6}{\meter\per\second} collides perpendicularly with a wall and bounces off with a speed of \SI{4}{\meter\per\second} in the opposite direction. 
    What is the magnitude of the impulse acting on the ball?
    \begin{multicols}{2}
    \begin{choices}
        \wrongchoice{\SI{13}{\joule}} 
        \wrongchoice{\SI{1}{\newton\second}}
      \correctchoice{\SI{5}{\newton\second}}
        \wrongchoice{\SI{2}{\meter\per\second}}
        \wrongchoice{\SI{10}{\meter\per\second}}
    \end{choices}
    \end{multicols}
\end{question}
}

\element{AP}{
\begin{question}{momentum-Q16}
    A cart with mass $2m$ has a velocity $v$ before it strikes another cart of mass $3m$ at rest. 
    The two carts couple and move off together with a velocity of
    \begin{multicols}{2}
    \begin{choices}
        \wrongchoice{$\dfrac{v}{5}$}
      \correctchoice{$\dfrac{2v}{5}$}
        \wrongchoice{$\dfrac{3v}{5}$}
        \wrongchoice{$\dfrac{2v}{3}$}
        \wrongchoice{$\sqrt{\dfrac{2}{5}}v$}
    \end{choices}
    \end{multicols}
\end{question}
}

\element{AP}{
\begin{questionmult}{momentum-Q17}
    Consider two laboratory carts of different masses but identical kinetic energies and the three following statements.
    Which of the below statements would be correct?
    \begin{choices}
        %% (A) I only (B) II only (C) III only (D) I and II only E) I and III only
      \correctchoice{The one with the greatest mass has the greatest momentum}
        \wrongchoice{The same impulse was required to accelerate each cart from rest}
      \correctchoice{Both can do the same amount of work as they come to a stop}
    \end{choices}
\end{questionmult}
}

\element{AP}{
\begin{question}{momentum-Q18}
    A mass $m$ has speed $v$. 
    It then collides with a stationary object of mass $2m$. 
    If both objects stick together in a perfectly inelastic collision,
        what is the final speed of the newly formed object?
    \begin{multicols}{2}
    \begin{choices}
      \correctchoice{$\frac{1}{3} v$}
        \wrongchoice{$\frac{1}{2} v$}
        \wrongchoice{$\frac{2}{3} v$}
        \wrongchoice{$ v$}
        \wrongchoice{$\frac{3}{2} v$}
    \end{choices}
    \end{multicols}
\end{question}
}

\element{AP}{
\begin{question}{momentum-Q19}
    A Freight car is moving freely along a railroad track at \SI{7}{\meter\per\second} and collides with a tanker car that is at rest. 
    After the collision, the two cars stick together and continue to move down the track. 
    What is the magnitude of the final velocity of the cars if the freight car has a mass of \SI{1200}{\kilo\gram} and the tanker car has a mass of \SI{1600}{\kilo\gram}?
    \begin{multicols}{2}
    \begin{choices}
        \wrongchoice{\SI{0}{\meter\per\second}}
        \wrongchoice{\SI{1}{\meter\per\second}}
      \correctchoice{\SI{3}{\meter\per\second}}
        \wrongchoice{\SI{4}{\meter\per\second}}
        \wrongchoice{\SI{6}{\meter\per\second}}
    \end{choices}
    \end{multicols}
\end{question}
}

\element{AP}{
\begin{question}{momentum-Q20}
    A \SI{50}{\kilo\gram} skater at rest on a frictionless rink throws a \SI{2}{\kilo\gram} ball,
        giving the ball a velocity of \SI{10}{\meter\per\second}. 
    Which statement describes the skater's subsequent motion?
    \begin{choices}
        \wrongchoice{\SI{0.4}{\meter\per\second} in the same direction as the ball.}
      \correctchoice{\SI{0.4}{\meter\per\second} in the opposite direction of the ball}
        \wrongchoice{\SI{2}{\meter\per\second} in the same direction as the ball}
        \wrongchoice{\SI{4}{\meter\per\second} in the same direction as the ball}
        \wrongchoice{\SI{4}{\meter\per\second} in the opposite direction of the ball}
    \end{choices}
\end{question}
}

\element{AP}{
\begin{question}{momentum-Q21}
    A certain particle undergoes erratic motion. 
    At every point in its motion,
        the direction of the particles momentum is \emph{always}:
    \begin{choices}
      \correctchoice{the same as the direction of its velocity}
        \wrongchoice{the same as the direction of its acceleration}
        \wrongchoice{the same as the direction of the net force}
        \wrongchoice{the same as the direction of the kinetic energy vector}
        \wrongchoice{none of the provided}
    \end{choices}
\end{question}
}

\element{AP}{
\begin{question}{momentum-Q22}
    A student initially at rest on a frictionless frozen pond throws a \SI{1}{\kilo\gram} hammer in one direction. 
    After the throw,
        the hammer moves off in one direction while the student moves off in the other direction. 
    Which of the following correctly describes the above situation?
    \begin{choices}
        \wrongchoice{The hammer will have the momentum with the greater magnitude}
        \wrongchoice{The student will have the momentum with the greater magnitude}
      \correctchoice{The hammer will have the greater kinetic energy}
        \wrongchoice{The student will have the greater kinetic energy}
        \wrongchoice{The student and the hammer will have equal but opposite amounts of kinetic energy}
    \end{choices}
\end{question}
}

\element{AP}{
\begin{question}{momentum-Q23}
    The net force on a rocket with a weight of \SI{1.5e4}{\newton} is \SI{2.4e4}{\newton}.
    How much time is needed to increase the rockets speed from \SI{12}{\meter\per\second} to \SI{36}{\meter\per\second} near the surface of the Earth at takeoff?
    \begin{multicols}{2}
    \begin{choices}
        \wrongchoice{\SI{0.62}{\second}}
        \wrongchoice{\SI{0.78}{\second}}
      \correctchoice{\SI{1.5}{\second}}
        \wrongchoice{\SI{3.8}{\second}}
        \wrongchoice{\SI{15}{\second}}
    \end{choices}
    \end{multicols}
\end{question}
}

\element{AP}{
\begin{question}{momentum-Q24}
    A \SI{50}{\kilo\gram} gymnast falls freely from a height of 4 meters on to a trampoline. 
    The trampoline then bounces her back upward with a speed equal to the speed at which she first struck the trampoline. 
    What is the average force the trampoline applies to the gymnast.
    \begin{multicols}{2}
    \begin{choices}
        \wrongchoice{\SI{50}{\newton}}
        \wrongchoice{\SI{200}{\newton}}
        \wrongchoice{\SI{500}{\newton}}
        \wrongchoice{\SI{2000}{\newton}}
      \correctchoice{More information is required}
    \end{choices}
    \end{multicols}
\end{question}
}

\element{AP}{
\begin{question}{momentum-Q25}
    Two toy cars with different masses originally at rest are pushed apart by a spring between them. 
    Which of the following statements would \emph{not} be true?
    \begin{choices}
        %% NOTE: only four options
        \wrongchoice{both toy cars will acquire equal but opposite momenta}
      \correctchoice{both toy cars will acquire equal kinetic energies}
        \wrongchoice{the more massive toy car will acquire the least speed}
        \wrongchoice{the smaller toy car will experience an acceleration of the greatest magnitude}
    \end{choices}
\end{question}
}

\element{AP}{
\begin{question}{momentum-Q26}
    A bat striking a \SI{0.125}{\kilo\gram} baseball is in contact with the ball for a time of \SI{0.03}{\second}. 
    The ball travels in a straight line as it approaches and then leaves the bat. 
    If the ball arrives at the bat with a speed of \SI{4.5}{\meter\per\second} and leaves with a speed of \SI{6.5}{\meter\per\second} in the opposite direction, what is the magnitude of the average force acting on the ball?
    \begin{multicols}{2}
    \begin{choices}
        \wrongchoice{\SI{8.33}{\newton}}
        \wrongchoice{\SI{18.75}{\newton}}
        \wrongchoice{\SI{27.08}{\newton}}
      \correctchoice{\SI{45.83}{\newton}}
        \wrongchoice{\SI{458}{\newton}}
    \end{choices}
    \end{multicols}
\end{question}
}

\element{AP}{
\begin{question}{momentum-Q27}
    An arrow is shot through an apple. 
    If the \SI{0.1}{\kilo\gram} arrow changes speed by \SI{10}{\meter\per\second} during the collision (from \SI{30}{\meter\per\second} to \SI{20}{\meter\per\second}) and the apple goes from rest to a speed of 2 m/s during the collisions,
        then the mass of the apple must be:
    \begin{multicols}{2}
    \begin{choices}
        \wrongchoice{\SI{0.2}{\kilo\gram}}
      \correctchoice{\SI{0.5}{\kilo\gram}}
        \wrongchoice{\SI{0.8}{\kilo\gram}}
        \wrongchoice{\SI{1}{\kilo\gram}}
        \wrongchoice{\SI{2}{\kilo\gram}}
    \end{choices}
    \end{multicols}
\end{question}
}

\element{AP}{
\begin{question}{momentum-28}
    A railroad flatcar of mass \SI{2 000}{\kilo\gram} rolls to the right at \SI{10}{\meter\per\second} and collides with a flatcar of mass \SI{3 000}{\kilo\gram} that is rolling to the left at \SI{5}{\meter\per\second}. 
    The flatcars couple together. 
    Their speed after the collision is:
    \begin{multicols}{2}
    \begin{choices}
      \correctchoice{\SI{1}{\meter\per\second}}
        \wrongchoice{\SI{2.5}{\meter\per\second}}
        \wrongchoice{\SI{5}{\meter\per\second}}
        \wrongchoice{\SI{7}{\meter\per\second}}
        \wrongchoice{\SI{7.5}{\meter\per\second}}
    \end{choices}
    \end{multicols}
\end{question}
}

\element{AP}{
\begin{question}{momentum-Q29}
    Which of the following quantities is a scalar that is always positive or zero?
    \begin{choices}
        \wrongchoice{Power}
        \wrongchoice{Work}
      \correctchoice{Kinetic energy}
        \wrongchoice{Linear momentum}
        \wrongchoice{Angular momentum}
    \end{choices}
\end{question}
}

\element{AP}{
\begin{question}{momentum-Q30}
    A tennis ball of mass $m$ rebounds from a racquet with the same speed $v$ as it had initially as shown. 
    \begin{center}
    \begin{tikzpicture}
        %% NOTE:
    \end{tikzpicture}
    \end{center}
    The magnitude of the momentum change of the ball is:
    \begin{multicols}{2}
    \begin{choices}
        \wrongchoice{zero}
        \wrongchoice{$mv$}
        \wrongchoice{$2mv$}
        \wrongchoice{$2mv\sin\theta$}
      \correctchoice{$2mv\cos\theta$}
    \end{choices}
    \end{multicols}
\end{question}
}

\element{AP}{
\begin{question}{momentum-Q31}
    Two bodies of masses \SI{5}{\kilo\gram} and \SI{7}{\kilo\gram} are initially at rest on a horizontal frictionless surface. 
    A light spring is compressed between the bodies,
        which are held together by a thin thread. 
    After the spring is released by burning through the thread,
        the \SI{5}{\kilo\gram} body has a speed of \SI{0.2}{\meter\per\second}.
    The speed of the \SI{7}{\kilo\gram} body is:
    \begin{multicols}{2}
    \begin{choices}
        \wrongchoice{\SI{\dfrac{1}{12}}{\meter\per\second}}
      \correctchoice{\SI{\dfrac{1}{7}}{\meter\per\second}}
        \wrongchoice{\SI{\dfrac{1}{\sqrt{35}}}{\meter\per\second}}
        \wrongchoice{\SI{\dfrac{1}{5}}{\meter\per\second}}
        \wrongchoice{\SI{\dfrac{7}{25}}{\meter\per\second}}
    \end{choices}
    \end{multicols}
\end{question}
}

\element{AP}{
\begin{questionmult}{momentum-Q32}
    A satellite of mass M moves in a circular orbit of radius $R$ at a constant speed $v$. 
    Which of the following must be true?
    \begin{choices}
        %%(A) I only (B) III only (C) I and II only (D) II and III only (E) I, II, and III
        \wrongchoice{The net force on the satellite is equal to MR and is directed toward the center of the orbit.}
      \correctchoice{The net work done on the satellite by gravity in one revolution is zero.}
      \correctchoice{The angular momentum of the satellite is a constant.}
    \end{choices}
\end{questionmult}
}

\element{AP}{
\begin{question}{momentum-Q33}
    Two pucks are firmly attached by a stretched spring and are initially held at rest on a frictionless surface, as shown above. 
    The pucks are then released simultaneously. 
    \begin{center}
    \begin{tikzpicture}
        %% NOTE:
    \end{tikzpicture}
    \end{center}
    If puck I has three times the mass of puck II,
        which of the following quantities is the same for both pucks as the spring pulls the two pucks toward each other?
    \begin{choices}
        \wronghchoice{Speed}
        \wronghchoice{Velocity}
        \wronghchoice{Acceleration}
        \wronghchoice{Kinetic energy}
      \correcthchoice{Magnitude of momentum}
    \end{choices}
\end{question}
}

\element{AP}{
\begin{question}{momentum-Q34}
    Which of the following is true when an object of mass $m$ moving on a horizontal frictionless surface hits and sticks to an object of mass $M>m$,
        which is initially at rest on the surface?
    \begin{choices}
        \wrongchoice{The collision is elastic.}
        \wrongchoice{All of the initial kinetic energy of the less massive object is lost.}
        \wrongchoice{The momentum of the objects that are stuck together has a smaller magnitude than the initial momentum of the less-massive object.}
      \correctchoice{The speed of the objects that are stuck together will be less than the initial speed of the less massive object.}
        \wrongchoice{The direction of motion of the objects that are stuck together depends on whether the hit is a head-on collision.}
    \end{choices}
\end{question}
}

\element{AP}{
\begin{question}{momentum-Q35}
    Two objects having the same mass travel toward each other on a flat surface each with a speed of \SI{1.0}{\meter\per\second} relative to the surface. 
    The objects collide head-on and are reported to rebound after the collision,
        each with a speed of \SI{2.0}{\meter\per\second} relative to the surface. 
    Which of the following assessments of this report is most accurate?
    \begin{choices}
        \wrongchoice{Momentum was not conserved therefore the report is false.}
      \correctchoice{If potential energy was released to the objects during the collision the report could be true.}
        \wrongchoice{If the objects had different masses the report could be true.}
        \wrongchoice{If the surface was inclined the report could be true.}
        \wrongchoice{If there was no friction between the objects and the surface the report could be true.}
    \end{choices}
\end{question}
}

\element{AP}{
\begin{question}{momentum-Q36}
    A solid metal ball and a hollow plastic ball of the same external radius are released from rest in a large vacuum chamber. 
    When each has fallen \SI{1}{\meter}, they both have the same
    \begin{choices}
        \wrongchoice{inertia}
      \correctchoice{speed}
        \wrongchoice{momentum}
        \wrongchoice{kinetic energy}
        \wrongchoice{change in potential energy}
    \end{choices}
\end{question}
}

\element{AP}{
\begin{question}{momentum-Q37}
    A railroad car of mass $m$ is moving at speed $v$ when it collides with a second railroad car of mass $M$ which is at rest. 
    The two cars lock together instantaneously and move along the track. 
    What is the speed of the cars immediately after the collision?
    \begin{multicols}{2}
    \begin{choices}
        \wrongchoice{$\dfrac{v}{2}$}
        \wrongchoice{$\dfrac{mv}{M}$}
        \wrongchoice{$\dfrac{Mv}{m}$}
        \wrongchoice{$\dfrac{v(m+M)}{m}$}
      \correctchoice{$\dfrac{mv}{(m+M}$}
    \end{choices}
    \end{multicols}
\end{question}
}

\element{AP}{
\begin{question}{momentum-Q38}
    An open cart on a level surface is rolling without frictional loss through a vertical downpour of rain, as shown below.
    \begin{center}
    \begin{tikzpicture}
    \end{tikzpicture}
    \end{center}
    As the cart rolls, an appreciable amount of rainwater accumulates in the cart. 
    The speed of the cart will:
    \begin{choices}
        \wrongchoice{increase because of conservation of momentum}
        \wrongchoice{increase because of conservation of mechanical energy}
      \correctchoice{decrease because of conservation of momentum}
        \wrongchoice{decrease because of conservation of mechanical energy}
        \wrongchoice{remain the same because the raindrops are falling perpendicular to the direction of the cart's motion}
    \end{choices}
\end{question}
}

\element{AP}{
\begin{question}{momentum-Q39}
    A \SI{2}{\kilo\gram} object moves in a circle of radius \SI{4}{\meter} at a constant speed of \SI{3}{\meter\per\second}.
    A net force of \SI{4.5}{\newton} acts on the object. 
    What is the angular momentum of the object with respect to an axis perpendicular to the circle and through its center?
    \begin{multicols}{2}
    \begin{choices}
        \wrongchoice{\SI{9}{\newton\meter\per\kilo\gram}}
        \wrongchoice{\SI{12}{\meter\squared\per\second}}
        \wrongchoice{\SI{13.5}{\kilo\gram\meter\squared\per\second\squared}}
        \wrongchoice{\SI{18}{\newton\meter\per\kilo\gram}}
      \correctchoice{\SI{24}{\kilo\gram\meter\squared\per\second}}
    \end{choices}
    \end{multicols}
\end{question}
}

\element{AP}{
\begin{question}{momentum-Q40}
    Two objects of mass \SI{0.2}{\kilo\gram} and \SI{0.1}{\kilo\gram},
        respectively, move parallel to the x-axis, as shown below.
    \begin{center}
    \begin{tikzpicture}
        %% NOTE:
    \end{tikzpicture}
    \end{center}
    The \SI{0.2}{\kilo\gram} object overtakes and collides with the \SI{0.1}{\kilo\gram} object. 
    Immediately after the collision,
        the $y$-component of the velocity of the \SI{0.2}{\kilo\gram} object is \SI{1}{\meter\per\second} upward. 
    What is the $y$-component of the velocity of the \SI{0.1}{\kilo\gram} object immediately after the collision?
    \begin{multicols}{2}
    \begin{choices}
      \correctchoice{\SI{2}{\meter\per\second} downward}
        \wrongchoice{\SI{0.5}{\meter\per\second} upward}
        \wrongchoice{\SI{0.5}{\meter\per\second} downward}
        \wrongchoice{\SI{2}{\meter\per\second} upward}
        \wrongchoice{\SI{0}{\meter\per\second}}
    \end{choices}
    \end{multicols}
\end{question}
}

\newcommand{\momentumQFortyOneA}{
\begin{tikzpicture}
    %% NOTE:
\end{tikzpicture}
}

\newcommand{\momentumQFortyOneB}{
\begin{tikzpicture}
    %% NOTE:
\end{tikzpicture}
}

\newcommand{\momentumQFortyOneC}{
\begin{tikzpicture}
    %% NOTE:
\end{tikzpicture}
}

\element{AP}{
\begin{questionmult}{momentum-Q41}
    %% Questions 41-42
    Three objects can only move along a straight, level path. 
    The graphs below show the position $d$ of each of the objects plotted as a function of time $t$.
    The magnitude of the momentum of the object is increasing in which of the cases?
    \begin{multicols}{2}
    \begin{choices}
        %% (A) II only (B) III only (C) I and II only (D) I and III only (E) I, II, and III
        \wrongchoice{\momentumQFortyOneA}
        \wrongchoice{\momentumQFortyOneB}
      \correctchoice{\momentumQFortyOneC}
    \end{choices}
    \end{multicols}
\end{questionmult}
}

\element{AP}{
\begin{questionmult}{momentum-Q42}
    %% Questions 41-42
    Three objects can only move along a straight, level path. 
    The graphs below show the position $d$ of each of the objects plotted as a function of time $t$.
    The sum of the forces on the object is zero in which of the cases?
    \begin{multicols}{2}
    \begin{choices}
        %% (A) II only (B) III only (C) I and II only (D) I and III only (E) I, II, and III
      \correctchoice{\momentumQFortyOneA}
      \correctchoice{\momentumQFortyOneB}
        \wrongchoice{\momentumQFortyOneC}
    \end{choices}
    \end{multicols}
\end{questionmult}
}

\element{AP}{
\begin{question}{momentum-Q43}
    A ball of mass \SI{0.4}{\kilo\gram} is initially at rest on the ground. 
    It is kicked and leaves the kicker's foot with a speed of \SI{5.0}{\meter\per\second} in a direction \ang{60} above the horizontal. 
    The magnitude of the impulse imparted by the ball to the foot is most nearly:
    \begin{multicols}{2}
    \begin{choices}
        \wrongchoice{\SI{1}{\newton\second}}
        \wrongchoice{\SI{\sqrt{3}}{\newton\second}}
      \correctchoice{\SI{2}{\newton\second}}
        \wrongchoice{\SI{\dfrac{2}{\sqrt{3}}}{\newton\second}}
        \wrongchoice{\SI{4}{\newton\second}}
    \end{choices}
    \end{multicols}
\end{question}
}

\element{AP}{
\begin{question}{momentum-Q44}
    Two people of unequal mass are initially standing still on ice with negligible friction. 
    They then simultaneously push each other horizontally. 
    Afterward, which of the following is true?
    \begin{choices}
        \wrongchoice{The kinetic energies of the two people are equal.}
        \wrongchoice{The speeds of the two people are equal.}
      \correctchoice{The momenta of the two people are of equal magnitude.}
        \wrongchoice{The center of mass of the two-person system moves in the direction of the less massive person.}
        \wrongchoice{The less massive person has a smaller initial acceleration than the more massive person.}
    \end{choices}
\end{question}
}

\element{AP}{
\begin{question}{momentum-Q45}
    A stationary object explodes, breaking into three pieces of masses $m$, $m$, and $3m$. 
    The two pieces of mass $m$ move off at right angles to each other with the same magnitude of momentum $mV$,
        as shown in the diagram above. 
    What are the magnitude and direction of the velocity of the piece having mass $3m$?
    \begin{choices}
        Magnitude Direction
        %% NOTE: tabular options
        \wrongchoice{V 3 V}
        \wrongchoice{3}
        \wrongchoice{2 V 3}
      \correctchoice{2 V 3}
        \wrongchoice{2V}
    \end{choices}
\end{question}
}

\element{AP}{
\begin{question}{momentum-Q46}
    A ball is thrown straight up in the air. 
    When the ball reaches its highest point,
        which of the following is true?
    \begin{choices}
        \wrongchoice{It is in equilibrium.}
        \wrongchoice{It has zero acceleration.}
        \wrongchoice{It has maximum momentum.}
        \wrongchoice{It has maximum kinetic energy.}
      \correctchoice{None of the provided.}
    \end{choices}
\end{question}
}

\element{AP}{
\begin{question}{momentum-Q47}
    An empty sled of mass $M$ moves without friction across a frozen pond at speed $v_0$. 
    Two objects are dropped vertically into the sled one at a time:
        first an object of mass $m$ and then an object of mass $2m$. 
    Afterward the sled moves with speed $v_f$. 
    What would be the final speed of the sled if the objects were dropped into it in reverse order?
    \begin{multicols}{2}
    \begin{choices}
        \wrongchoice{$\frac{1}{3} v_f$}
        \wrongchoice{$\frac{1}{2} v_f$}
      \correctchoice{$v_f$}
        \wrongchoice{$2 v_f$}
        \wrongchoice{$3 v_f$}
    \end{choices}
    \end{multicols}
\end{question}
}
    
\newcommand{\momentumQFortyEight}{
\begin{tikzpicture}
\end{tikzpicture}
}

\element{AP}{
\begin{question}{momentum-Q48}
    A student obtains data on the magnitude of force applied to an object as a function of time and displays the data on the graph above.
    \begin{center}
        \momentumQFortyEight
    \end{center}
    The slope of the ``best fit'' straight line is most nearly:
    \begin{multicols}{2}
    \begin{choices}
      \correctchoice{\SI{5}{\newton\per\second}}
        \wrongchoice{\SI{6}{\newton\per\second}}
        \wrongchoice{\SI{7}{\newton\per\second}}
        \wrongchoice{\SI{8}{\newton\per\second}}
        \wrongchoice{\SI{10}{\newton\per\second}}
    \end{choices}
    \end{multicols}
\end{question}
}

\element{AP}{
\begin{question}{momentum-Q49}
    A student obtains data on the magnitude of force applied to an object as a function of time and displays the data on the graph above.
    \begin{center}
        \momentumQFortyEight
    \end{center}
    The increase in the momentum of the object between $t=\SI{0}{\second}$ and $t=\SI{4}{\second}$ is most nearly:
    \begin{multicols}{2}
    \begin{choices}
        \wrongchoice{\SI{40}{\newton\second}}
        \wrongchoice{\SI{50}{\newton\second}}
      \correctchoice{\SI{60}{\newton\second}}
        \wrongchoice{\SI{80}{\newton\second}}
        \wrongchoice{\SI{100}{\newton\second}}
    \end{choices}
    \end{multicols}
\end{question}
}

\element{AP}{
\begin{question}{momentum-Q50}
    How does an air mattress protect a stunt person landing on the ground after a stunt?
    \begin{choices}
        \wrongchoice{It reduces the kinetic energy loss of the stunt person.}
        \wrongchoice{It reduces the momentum change of the stunt person.}
        \wrongchoice{It increases the momentum change of the stunt person.}
        \wrongchoice{It shortens the stopping time of the stunt person and increases the force applied during the landing.}
      \correctchoice{It lengthens the stopping time of the stunt person and reduces the force applied during the landing.}
    \end{choices}
\end{question}
}

\element{AP}{
\begin{question}{momentum-Q51}
    Two objects, $A$ and $B$, initially at rest,
        are ``exploded'' apart by the release of a coiled spring that was compressed between them. 
    As they move apart,
        the velocity of object $A$ is \SI{5}{\meter\per\second} and the velocity of object $B$ is \SI{-2}{\meter\per\second}.
    The ratio of the mass of object $A$ to the mass object B, $m_a/m_b$ is:
    \begin{multicols}{2}
    \begin{choices}
        \wrongchoice{$4/25$}
      \correctchoice{$2/5$}
        \wrongchoice{$1/1$}
        \wrongchoice{$5/2$}
        \wrongchoice{$25/4$}
    \end{choices}
    \end{multicols}
\end{question}
}

\element{AP}{
\begin{question}{momentum-Q52}
    The two blocks of masses $M$ and $2M$ shown above initially travel at the same speed $v$ but in opposite directions.
    They collide and stick together. 
    \begin{center}
    \begin{tikzpicture}
        %% NOTE:
    \end{tikzpicture}
    \end{center}
    How much mechanical energy is lost to other forms of energy during the collision?
    \begin{multicols}{2}
    \begin{choices}
        \wrongchoice{Zero}
        \wrongchoice{$\frac{1}{2} M v^2$}
        \wrongchoice{$\frac{3}{4} M v^2$}
      \correctchoice{$\frac{4}{3} M v^2$}
        \wrongchoice{$\frac{3}{2} M v^2$}
    \end{choices}
    \end{multicols}
\end{question}
}

\element{AP}{
\begin{question}{momentum-Q53}
    Two particles of equal mass $m_0$,
        moving with equal speeds $v_0$ along paths inclined at \ang{60} to the $x$-axis as shown,
        collide and stick together.
    \begin{center}
    \begin{tikzpicture}
        %% NOTE:
    \end{tikzpicture}
    \end{center}
    Their velocity after the collision has magnitude:
    \begin{multicols}{2}
    \begin{choices}
        \wrongchoice{$\frac{1}{4} v_0$}
      \correctchoice{$\frac{1}{2} v_0$}
        \wrongchoice{$\frac{\sqrt{2}}{2} v_0$}
        \wrongchoice{$\frac{\sqrt{3}}{2} v_0$}
        \wrongchoice{$v_0$}
    \end{choices}
    \end{multicols}
\end{question}
}

\element{AP}{
\begin{question}{momentum-Q54}
    A particle of mass $m$ moves with a constant speed $v$ along the dashed line $y = a$. 
    \begin{center}
    \begin{tikzpicture}
        %% NOTE:
    \end{tikzpicture}
    \end{center}
    When the $x$-coordinate of the particle is $x_0$,
        the magnitude of the angular momentum of the particle with respect to the origin of the system is
    \begin{multicols}{2}
    \begin{choices}
        \wrongchoice{zero}
      \correctchoice{$mva$}
        \wrongchoice{$mvx_0$}
        \wrongchoice{$mv\sqrt{x^2 + a^2}$}
        \wrongchoice{$\dfrac{mva}{\sqrt{x^2 + a^2}}$}
    \end{choices}
    \end{multicols}
\end{question}
}

\element{AP}{
\begin{question}{momentum-Q55}
    %% Questions 55 and 56
    A \SI{4}{\kilo\gram} mass has a speed of \SI{6}{\meter\per\second} on a horizontal frictionless surface,
        as shown below. 
    \begin{center}
    \begin{tikzpicture}
        %% NOTE:
    \end{tikzpicture}
    \end{center}
    The mass collides head-on with an identical \SI{4}{\kilo\gram} mass initially at rest and sticks. 
    The combined masses then collide head-on and stick to a third \SI{4}{\kilo\gram} mass initially at rest.
    The final speed of the first \SI{4}{\kilo\gram} mass is:
    \begin{multicols}{2}
    \begin{choices}
        \wrongchoice{\SI{0}{\meter\per\second}}
        \wrongchoice{\SI{2}{\meter\per\second}}
      \correctchoice{\SI{3}{\meter\per\second}}
        \wrongchoice{\SI{4}{\meter\per\second}}
        \wrongchoice{\SI{6}{\meter\per\second}}
    \end{choices}
    \end{multicols}
\end{question}
}

\element{AP}{
\begin{question}{momentum-Q56}
    %% Questions 55 and 56
    A \SI{4}{\kilo\gram} mass has a speed of \SI{6}{\meter\per\second} on a horizontal frictionless surface,
        as shown below. 
    \begin{center}
    \begin{tikzpicture}
        %% NOTE:
    \end{tikzpicture}
    \end{center}
    The mass collides head-on with an identical \SI{4}{\kilo\gram} mass initially at rest and sticks. 
    The combined masses then collide head-on and stick to a third \SI{4}{\kilo\gram} mass initially at rest.
    The final speed of the two 4-kilogram masses that stick together is:
    \begin{multicols}{2}
    \begin{choices}
        \wrongchoice{\SI{0}{\meter\per\second}}
      \correctchoice{\SI{2}{\meter\per\second}}
        \wrongchoice{\SI{3}{\meter\per\second}}
        \wrongchoice{\SI{4}{\meter\per\second}}
        \wrongchoice{\SI{6}{\meter\per\second}}
    \end{choices}
    \end{multicols}
\end{question}
}

\element{AP}{
\begin{question}{momentum-Q57}
    A projectile of mass $M_1$ is fired horizontally from a spring gun that is initially at rest on a frictionless surface.
    The combined mass of the gun and projectile is $M_2$.
    If the kinetic energy of the projectile after firing is $K$,
        the gun will recoil with a kinetic energy equal to:
    \begin{multicols}{2}
    \begin{choices}
        \wrongchoice{$K$}
        \wrongchoice{$K\dfrac{M_2}{M_1}$}
        \wrongchoice{$K\dfrac{M_1^2}{M_2^2}$}
      \correctchoice{$K\dfrac{M_1}{M_2-M_1}$}
        \wrongchoice{$K\sqrt{\dfrac{M_1}{M_2-M_1}}$}
    \end{choices}
    \end{multicols}
\end{question}
}

\element{AP}{
\begin{question}{momentum-Q58}
    Two balls are on a frictionless horizontal tabletop. 
    Ball $X$ initially moves at \SI{10}{\meter\per\second}, as shown in Figure I. 
    It then collides elastically with identical ball $Y$ which is initially at rest.
    After the collision,
        ball $X$ moves at \SI{6}{\meter\per\second} along a path at \ang{53} to its original direction,
        as shown in Figure II. 
    \begin{center}
    \begin{tikzpicture}
        %% NOTE:
    \end{tikzpicture}
    \end{center}
    Which of the following diagrams best represents the motion of ball $Y$ after the collision?
    \begin{multicols}{2}
    \begin{choices}
        %% NOTE; ans is D
        \wrongchoice{
            \begin{tikzpicture}
            \end{tikzpicture}
        }
    \end{choices}
    \end{multicols}
\end{question}
}

\element{AP}{
\begin{question}{momentum-Q59}
    If one knows only the constant resultant force acting on an object and the time during which this force acts,
        one can determine the:
    \begin{choices}
      \correctchoice{change in momentum of the object}
        \wrongchoice{change in velocity of the object}
        \wrongchoice{change in kinetic energy of the object}
        \wrongchoice{mass of the object}
        \wrongchoice{acceleration of the object}
    \end{choices}
\end{question}
}

\element{AP}{
\begin{question}{momentum-Q60}
    An object of mass $m$ is moving with speed $v_0$ to the right on a horizontal frictionless surface,
        as shown below, when it explodes into two pieces. 
    \begin{center}
    \begin{tikzpicture}
        %% NOTE:
    \end{tikzpicture}
    \end{center}
    Subsequently,
        one piece of mass $2/5 m$ moves with a speed v o / 2 to the left.
    The speed of the other piece of the object is:
    \begin{multicols}{2}
    \begin{choices}
        \wrongchoice{$\frac{1}{2} v_0$}
        \wrongchoice{$\frac{1}{3} v_0$}
        \wrongchoice{$\frac{7}{5} v_0$}
        \wrongchoice{$\frac{3}{2} v_0$}
      \correctchoice{$2 v_0$}
    \end{choices}
    \end{multicols}
\end{question}
}

\element{AP}{
\begin{question}{momentum-Q61}
    The graph shows the force on an object of mass $M$ as a function of time. 
    \begin{center}
    \begin{tikzpicture}
        %% NOTE:
    \end{tikzpicture}
    \end{center}
    For the time interval \SI{0}{\second} to \SI{4}{\second},
        the total change in the momentum of the object is:
    \begin{choices}
        \wrongchoice{\SI{40}{\kilo\gram\meter\per\second}}
        \wrongchoice{\SI{20}{\kilo\gram\meter\per\second}}
      \correctchoice{\SI{0}{\kilo\gram\meter\per\second}}
        \wrongchoice{\SI{-20}{\kilo\gram\meter\per\second}}
        \wrongchoice{indeterminable unless the mass $M$ of the object is known}
    \end{choices}
\end{question}
}

\element{AP}{
\begin{question}{momentum-Q62}
    The graph shows the force on an object of mass $M$ as a function of time. 
    As shown in the top view,
        a disc of mass $m$ is moving horizontally to the right with speed $v$ on a table with negligible friction when it collides with a second disc of mass $2m$.
    The second disc is moving horizontally to the right with speed $v/2$ at the moment of impact.
    The two discs stick together upon impact. 
    \begin{center}
    \begin{tikzpicture}
        %% NOTE:
    \end{tikzpicture}
    \end{center}
    The speed of the composite body immediately after the collision is:
    \begin{multicols}{2}
    \begin{choices}
        \wrongchoice{$\frac{1}{3} v$}
        \wrongchoice{$\frac{1}{2} v$}
      \correctchoice{$\frac{2}{3} v$}
        \wrongchoice{$\frac{3}{2} v$}
        \wrongchoice{$2 v$}
    \end{choices}
    \end{multicols}
\end{question}
}

\element{AP}{
\begin{question}{momentum-Q63}
    An object having an initial momentum that may be represented by the vector above strikes an object that is initially at rest. 
    \begin{center}
    \begin{tikzpicture}
        %% NOTE:
    \end{tikzpicture}
    \end{center}
    Which of the following sets of vectors may represent the momenta of the two objects after the collision?
    \begin{multicols}{2}
    \begin{choices}
        %% NOTE; ans is E
        \wrongchoice{
            \begin{tikzpicture}
            \end{tikzpicture}
        }
    \end{choices}
    \end{multicols}
\end{question}
}

\element{AP}{
\begin{question}{momentum-Q64}
    A \SI{2}{\kilo\gram} ball collides with the floor at an angle $\theta$ and rebounds at the same angle and speed as shown below.
    \begin{center}
    \begin{tikzpicture}
        %% NOTE:
    \end{tikzpicture}
    \end{center}
    Which of the following vectors represents the impulse exerted on the ball by the floor?
    \begin{multicols}{2}
    \begin{choices}
        %% NOTE: ans is E
        \wrongchoice{
            \begin{tikzpicture}
            \end{tikzpicture}
        }
    \end{choices}
    \end{multicols}
\end{question}
}

\element{AP}{
\begin{question}{momentum-Q65}
    %%Questions 65-66
    Two pucks moving on a frictionless air table are about to collide,
        as shown.
    The \SI{1.5}{\kilo\gram} puck is moving directly east at \SI{2.0}{\meter\per\second}.
    The \SI{4.0}{\kilo\gram} puck is moving directly north at \SI{1.0}{\meter\per\second}.
    \begin{center}
    \begin{tikzpicture}
        %% NOTE:
    \end{tikzpicture}
    \end{center}
    What is the total kinetic energy of the two-puck system before the collision?
    \begin{multicols}{2}
    \begin{choices}
        \wrongchoice{\SI{\sqrt{13}}{\joule}} 
      \correctchoice{\SI{5.0}{\joule}} 
        \wrongchoice{\SI{7.0}{\joule}} 
        \wrongchoice{\SI{10}{\joule}} 
        \wrongchoice{\SI{11}{\joule}}
    \end{choices}
    \end{multicols}
\end{question}
}

\element{AP}{
\begin{question}{momentum-Q66}
    %%Questions 65-66
    Two pucks moving on a frictionless air table are about to collide,
        as shown.
    The \SI{1.5}{\kilo\gram} puck is moving directly east at \SI{2.0}{\meter\per\second}.
    The \SI{4.0}{\kilo\gram} puck is moving directly north at \SI{1.0}{\meter\per\second}.
    \begin{center}
    \begin{tikzpicture}
        %% NOTE:
    \end{tikzpicture}
    \end{center}
    What is the magnitude of the total momentum of the two-puck system after the collision?
    \begin{multicols}{2}
    \begin{choices}
        \wrongchoice{\SI{1.0}{\kilo\gram\meter\per\second}} 
        \wrongchoice{\SI{3.5}{\kilo\gram\meter\per\second}} 
      \correctchoice{\SI{5.0}{\kilo\gram\meter\per\second}} 
        \wrongchoice{\SI{7.0}{\kilo\gram\meter\per\second}} 
        \wrongchoice{\SI{\dfrac{11\sqrt{5}}{2}}{\kilo\gram\meter\per\second}}
    \end{choices}
    \end{multicols}
\end{question}
}

\element{AP}{
\begin{question}{momentum-Q67}
    An object $m$, on the end of a string,
        moves in a circle on a horizontal frictionless table as shown. 
    \begin{center}
    \begin{tikzpicture}
        %% NOTE:
    \end{tikzpicture}
    \end{center}
    As the string is pulled very slowly through a small hole in the table,
        which of the following is correct for an observer measuring from the hole in the table?
    \begin{choices}
      \correctchoice{The angular momentum of $m$ remains constant.}
        \wrongchoice{The angular momentum of $m$ decreases.}
        \wrongchoice{The kinetic energy of $m$ remains constant}
        \wrongchoice{The kinetic energy of $m$ decreases}
        \wrongchoice{None of the above occurs.}
    \end{choices}
\end{question}
}

\element{AP}{
\begin{question}{momentum-Q68}
    A car of mass \SI{900}{\kilo\gram} is traveling at \SI{20}{\meter\per\second} when the brakes are applied. 
    The car then comes to a complete stop in \SI{5}{\second}. 
    What is the average power that the brakes produce in stopping the car?
    \begin{multicols}{2}
    \begin{choices}
        \wrongchoice{\SI{1800}{\watt}} 
        \wrongchoice{\SI{3600}{\watt}} 
        \wrongchoice{\SI{7200}{\watt}} 
      \correctchoice{\SI{36 000}{\watt}} 
        \wrongchoice{\SI{72 000}{\watt}}
    \end{choices}
    \end{multicols}
\end{question}
}

\element{AP}{
\begin{question}{momentum-Q69}
    A boy of mass $m$ and a girl of mass $2m$ are initially at rest at the center of a frozen pond. 
    They push each other so that she slides to the left at speed $v$ across the frictionless ice surface and he slides to the right. 
    What is the total work done by the children?
    \begin{multicols}{2}
    \begin{choices}
        \wrongchoice{Zero}
        \wrongchoice{$mv$}
        \wrongchoice{$mv^2 $}
        \wrongchoice{$2mv^2$}
      \correctchoice{$3mv^2$}
    \end{choices}
    \end{multicols}
\end{question}
}

\element{AP}{
\begin{question}{momentum-Q70}
    An object of mass $M$ travels along a horizontal air track at a constant speed $v$ and collides elastically with an object of identical mass that is initially at rest on the track. Which of the following statements is true for the two objects after the impact?
    \begin{choices}
      \correctchoice{The total momentum is $Mv$ and the total kinetic energy is $\frac{1}{2} Mv^2$}
        \wrongchoice{The total momentum is $Mv$ and the total kinetic energy is less than $\frac{1}{2} Mv^2$}
        \wrongchoice{The total momentum is less than $Mv$ and the total kinetic energy is $\frac{1}{2} Mv^2$}
        \wrongchoice{The momentum of each object is $\frac{1}{2} Mv$}
        \wrongchoice{The kinetic energy of each object is $\frac{1}{4} Mv^2$}
    \end{choices}
\end{question}
}

\element{AP}{
\begin{question}{momentum-Q71}
    A \SI{2}{\kilo\gram} object initially moving with a constant velocity is subjected to a force of magnitude $F$ in the direction of motion. 
    A graph of $F$ as a function of time $t$ is shown. 
    \begin{center}
    \begin{tikzpicture}
    \end{tikzpicture}
    \end{center}
    What is the increase, if any,
        in the velocity of the object during the time the force is applied?
    \begin{multicols}{2}
    \begin{choices}
        \wrongchoice{\SI{0}{\meter\per\second}} 
        \wrongchoice{\SI{2.0}{\meter\per\second}} 
      \correctchoice{\SI{3.0}{\meter\per\second}} 
        \wrongchoice{\SI{4.0}{\meter\per\second}} 
        \wrongchoice{\SI{6.0}{\meter\per\second}}
    \end{choices}
    \end{multicols}
\end{question}
}

\element{AP}{
\begin{question}{momentum-Q72}
    A disk slides to the right on a horizontal,
        frictionless air table and collides with another disk that was initially stationary. 
    The figures below show a top view of the initial path $I$ of the sliding disk and a hypothetical path $H$ for each disk after the collision.
    Which figure shows an impossible situation?
    \begin{multicols}{2}
    \begin{choices}
        %% NOTE: ans is B
        \wrongchoice{
            \begin{tikzpicture}
            \end{tikzpicture}
        }
    \end{choices}
    \end{multicols}
\end{question}
}

\element{AP}{
\begin{question}{momentum-Q73}
    A ball of mass $m$ with speed $v$ strikes a wall at an angle $\theta$ with the normal,
        as shown. 
    It then rebounds with the same speed and at the same angle. 
    The impulse delivered by the ball to the wall is:
    \begin{multicols}{2}
    \begin{choices}
        \wrongchoice{zero}
        \wrongchoice{$mv\sin\theta$}
        \wrongchoice{$mv\cos\theta$}
        \wrongchoice{$2mv\sin\theta$}
      \correctchoice{$2mv\cos\theta$}
    \end{choices}
    \end{multicols}
\end{question}
}









\endinput


