

%% this section contains XX problems

%% AP Physics B practice workbook
%%--------------------------------------------------

%% Section B: Gravitation
%%--------------------------------------------------

%% Page 269
\element{AP}{
\begin{question}{gravity-Q01}
    A car of mass $m$, traveling at speed $v$,
        stops in time $t$ when maximum braking force is applied. 
    Assuming the braking force is independent of mass,
        what time would be required to stop a car of mass $2m$ traveling at speed $v$?
    \begin{multicols}{2}
    \begin{choices}
        \wrongchoice{$\frac{1}{2} t$}
        \wrongchoice{$t$}
        \wrongchoice{$\sqrt{2} t$}
      \correctchoice{$2 t$}
        \wrongchoice{$4 t$}
    \end{choices}
    \end{multicols}
\end{question}
}



1.
Each of five satellites makes a circular orbit about an object that is much more massive than any of the
satellites. The mass and orbital radius of each satellite are given below. Which satellite has the greatest speed?
Mass Radius
(A) 1⁄2m
R
(B)
m
1⁄2R
(C)
m
R
(D)
m
2R
(E)
2m
R
*2. An asteroid moves in an elliptic orbit with the Sun at one focus as shown above. Which of the following
quantities increases as the asteroid moves from point P in its orbit to point Q?
(A) Speed (B) Angular momentum (C) Total energy (D) Kinetic energy (E) Potential energy
3. Two planets have the same size, but different masses, and no atmospheres. Which of the following would be the
same for objects with equal mass on the surfaces of the two planets?
I. The rate at which each would fall freely
II. The amount of mass each would balance on an equal-arm balance
III. The amount of momentum each would acquire when given a certain impulse
(A) I only (B) III only (C) I and II only (D) II and III only (E) I, II, and III
4. A person weighing 800 newtons on Earth travels to another planet with twice the mass and twice the radius of
Earth. The person's weight on this other planet is most nearly
800
(A) 400 N
(B) N (C) 800 N (D) 800√2
(E) 1,600 N
5.
√2
Mars has a mass 1/10 that of Earth and a diameter 1/2 that of Earth. The acceleration of a falling body near the
surface of Mars is most nearly
(B) 0.5 m/s 2
(C) 2 m/s 2
(D) 4 m/s 2
(E) 25 m/s 2
(A) 0.25 m/s 2
6. A satellite of mass M moves in a circular orbit of radius R at a constant speed v. Which of the following must
be true?
I. The net force on the satellite is equal to Mv 2 /R and is directed toward the center of the orbit.
II. The net work done on the satellite by gravity in one revolution is zero.
III. The angular momentum of the satellite is a constant.
(A) I only (B) III only (C) I and II only (D) II and III only (E) I, II, and III
7. If Spacecraft X has twice the mass of Spacecraft Y, then true statements about X and Y include which of the
following?
I. On Earth. X experiences twice the gravitational force that Y experiences.
II. On the Moon, X has twice the weight of Y.
III. When both are in the same circular orbit, X has twice the centripetal acceleration of Y.
(A) I only (B) III only (C) I and II only (D) II and III only (E) I, II, and III



8. The two spheres pictured above have equal densities and are subject only to their mutual gravitational
attraction. Which of the following quantities must have the same magnitude for both spheres?
(A) Acceleration
(B) Velocity
(C) Kinetic energy
(D) Displacement from the center of mass
(E) Gravitational force
9. The planet Mars has mass 6.4 × 10 23 kilograms and radius 3.4 × 10 6 meters. The acceleration of an object in
free-fall near the surface of Mars is most nearly
(C) 1.9 m/s 2
(D) 3.7 m/s 2
(E) 9.8 m/s 2
(A) zero
(B) 1.0 m/s 2
10. An object has a weight W when it is on the surface of a planet of radius R. What will be the gravitational force
on the object after it has been moved to a distance of 4R from the center of the planet?
(A) 16W
(B) 4W
(C) W
(D) 4
(E) 1/16 W
11. A new planet is discovered that has twice the Earth's mass and twice the Earth's radius. On the surface of this
new planet, a person who weighs 500 N on Earth would experience a gravitational force of
(A) 125 N
(B) 250 N
(C) 500 N
(D) 1000 N
(E) 2000 N
*12. A simple pendulum and a mass hanging on a spring both have a period of 1 s when set into small oscillatory
motion on Earth. They are taken to Planet X, which has the same diameter as Earth but twice the mass. Which
of the following statements is true about the periods of the two objects on Planet X compared to their periods on
Earth?
(A) Both are shorter.
(B) Both are the same.
(C) Both are longer.
(D) The period of the mass on the spring is shorter, that of the pendulum is the same.
(E) The period of the pendulum is shorter; that of the mass on the spring is the same.
13. A satellite of mass m and speed v moves in a stable, circular orbit around a planet of mass M. What is the radius
of the satellite's orbit?
GM
Gv
GmM
GM
GmM
(B)
(C)
(D)
(E)
(A)
2
mv
mM
v
v
v 2
14. The mass of Planet X is one-tenth that of the Earth, and its diameter is one-half that of the Earth. The
acceleration due to gravity at the surface of Planet X is most nearly
(B) 4 m/s 2
(C) 5 m/s 2
(D) 7 m/s 2
(E) 10 m/s 2
(A) 2 m/s 2
*15. A satellite travels around the Sun in an elliptical orbit as shown above. As the satellite travels from point X to
point Y. which of the following is true about its speed and angular momentum?
Speed
Angular Momentum
(A) Remains constant Remains constant
(B) Increases
Increases
(C) Decreases
Decreases
(D) Increases
Remains constant
(E) Decreases
Remains constant
16. A newly discovered planet, "Cosmo," has a mass that is 4 times the mass of the Earth. The radius of the Earth is
R e . The gravitational field strength at the surface of Cosmo is equal to that at the surface of the Earth if the
radius of Cosmo is equal to
(A) 1⁄2R e
(B) R e
(C) 2R e
(D) �RR ee
(E) R e 2



17. The radius of the Earth is approximately 6,000 kilometers. The acceleration of an astronaut in a perfectly
circular orbit 300 kilometers above the Earth would be most nearly
(B) 0.05 m/s 2
(C) 5 m/s 2
(D) 9 m/s 2
(E) 11 m/s 2
(A) 0 m/s 2
18. Two artificial satellites, 1 and 2, orbit the Earth in circular orbits having radii R 1 and R 2 , respectively, as shown
above. If R 2 = 2R 1 , the accelerations a 2 and a 1 of the two satellites are related by which of the following?
(B) a 2 = 2a 1
(C) a 2 = a 1
(D) a 2 = a 1 /2
(E) a 2 = a 1 /4
(A) a 2 = 4a 1
19. A satellite moves in a stable circular orbit with speed v o at a distance R from the center of a planet. For this
satellite to move in a stable circular orbit a distance 2R from the center of the planet, the speed of the satellite
must be
v
(A) v 0
(B) 0
(C) v o
(D) 2v 0
(E) 2v o
2
2
20. If F 1 is the magnitude of the force exerted by the Earth on a satellite in orbit about the Earth and F 2 is the
magnitude of the force exerted by the satellite on the Earth, then which of the following is true?
(A) F 1 is much greater than F 2 .
(B) F 1 is slightly greater than F 2 .
(C) F 1 is equal to F 2 .
(D) F 2 is slightly greater than F 1 (E) F 2 is much greater than F 1
21. A newly discovered planet has twice the mass of the Earth, but the acceleration due to gravity on the new
planet's surface is exactly the same as the acceleration due to gravity on the Earth's surface. The radius of the
new planet in terms of the radius R of Earth is
(A) 1⁄2R
(B)
2
2
R
(C)
2R
(D) 2R
(E) 4R
*22. A satellite S is in an elliptical orbit around a planet P, as shown above, with r 1 and r 2 being its closest and
farthest distances, respectively, from the center of the planet. If the satellite has a speed v 1 at its closest
distance, what is its speed at its farthest distance?
(A)
r 1
v
r 2 1
(B)
r 2
v
r 1 1
(C)
( r
2
− r 2 ) v 1
(D)
r 1 + r 2
v 1
2
(E)
r 2 − r 1
v
r 1 + r 2


Questions 23 – 24
A ball is tossed straight up from the surface of a small, spherical asteroid with no atmosphere. The ball rises to a
height equal to the asteroid's radius and then falls straight down toward the surface of the asteroid.
23. What forces, if any, act on the ball while it is on the way up?
(A) Only a decreasing gravitational force that acts downward
(B) Only an increasing gravitational force that acts downward
(C) Only a constant gravitational force that acts downward
(D) Both a constant gravitational force that acts downward and a decreasing force that acts upward
(E) No forces act on the ball.
24. The acceleration of the ball at the top of its path is
(A) at its maximum value for the ball's flight
(B) equal to the acceleration at the surface of the asteroid
(C) equal to one-half the acceleration at the surface of the asteroid
(D) equal to one-fourth the acceleration at the surface of the asteroid
(E) zero
25. A satellite of mass M moves in a circular orbit of radius R with constant speed v. True statements about this
satellite include which of the following?
I. Its angular speed is v/R.
II. Its tangential acceleration is zero.
III. The magnitude of its centripetal acceleration is constant.
(A) I only
(B) II only
(C) I and III only
(D) II and III only
(E) I, II, and III
26. Two identical stars, a fixed distance D apart, revolve in a circle about their mutual center of mass, as shown
above. Each star has mass M and speed v. G is the universal gravitational constant. Which of the following is
a correct relationship among these quantities?
(B) v 2 = GM/2D
(C) v 2 = GM/D 2
(D) v 2 = MGD
(E) v 2 = 2GM 2 /D
(A) v 2 = GM/D
*27. A spacecraft orbits Earth in a circular orbit of radius R, as shown above. When the spacecraft is at position P
shown, a short burst of the ship's engines results in a small increase in its speed. The new orbit is best shown by
the solid curve in which of the following diagrams?





*28. The escape speed for a rocket at Earth's surface is v e . What would be the rocket's escape speed from the surface
of a planet with twice Earth's mass and the same radius as Earth?
(A) 2v e
(B)
2 v e
(C) v e
(D)
v e
2
(E) 1⁄2 v e
29. A hypothetical planet orbits a star with mass one-half the mass of our sun. The planet’s orbital radius is the
same as the Earth’s. Approximately how many Earth years does it take for the planet to complete one orbit?
1
(A) 1⁄2 (B)
(C) 1 (D) √2 (E) 2
√2
30. A hypothetical planet has seven times the mass of Earth and twice the radius of Earth. The magnitude of the
gravitational acceleration at the surface of this planet is most nearly
(A) 2.9 m/s 2 (B) 5.7 m/s 2 (C) 17.5 m/s 2 (D) 35 m/s 2 (E) 122 m/s 2
31. Two artificial satellites, 1 and 2, are put into circular orbit at the same altitude above Earth’s surface. The mass
of satellite 2 is twice the mass of satellite 1. If the period of satellite 1 is T, what is the period of satellite 2?
(A) T/4 (B) T/2 (C) T (D) 2T (E) 4T
32. A planet has a radius one-half that of Earth and a mass one-fifth the Earth’s mass. The gravitational acceleration
at the surface of the planet is most nearly
(A) 4.0 m/s 2 (B) 8.0 m/s 2 (C) 12.5 m/s 2 (D) 25 m/s 2 (E) 62.5 m/s 2
33. In the following problem, the word “weight” refers to the force a scale registers. If the Earth were to stop
rotating, but not change shape,
(A) the weight of an object at the equator would increase.
(B) the weight of an object at the equator would decrease.
(C) the weight of an object at the north pole would increase.
(D) the weight of an object at the north pole would decrease.
(E) all objects on Earth would become weightless.
*34. Assume that the Earth attracts John Glenn with a gravitational force F at the surface of the Earth. When he
made his famous second flight in orbit, the gravitational force on John Glenn while he was in orbit was closest
to which of the following?
(A) 0.95F (B) 0.50F (C) 0.25F (D) 0.10F (E) zero
35. What happens to the force of gravitational attraction between two small objects if the mass of each object is
doubled and the distance between their centers is doubled?
(A) It is doubled (B) It is quadrupled (C) It is halved (D) It is reduced fourfold (E) It remains the same
36. One object at the surface of the Moon experiences the same gravitational force as a second object at the surface
of the Earth. Which of the following would be a reasonable conclusion?
(A) both objects would fall at the same acceleration
(B) the object on the Moon has the greater mass
(C) the object on the Earth has the greater mass
(D) both objects have identical masses
(E) the object on Earth has a greater mass but the Earth has a greater rate of rotation.
37. Astronauts in an orbiting space shuttle are “weightless” because
(A) of their extreme distance from the earth
(B) the net gravitational force on them is zero
(C) there is no atmosphere in space
D) the space shuttle does not rotate
E) they are in a state of free fall



38. Consider an object that has a mass, m, and a weight, W, at the surface of the moon. If we assume the moon has a
nearly uniform density, which of the following would be closest to the object’s mass and weight at a distance
halfway between Moon’s center and its surface?
(A) 1⁄2 m & 1⁄2 W (B) 1⁄4 m & 1⁄4 W (C) 1 m & 1 W (D) 1 m & 1⁄2 W (E) 1 m & 1⁄4 W
39. The mass of a planet can be calculated if it is orbited by a small satellite by setting the gravitational force on the
satellite equal to the centripetal force on the satellite. Which of the following would NOT be required in this
calculation?
(A) the mass of the satellite
(B) the radius of the satellite's orbit
(C) the period of the satellite's orbit
(D) Newton's universal gravitational constant
(E) all of the above are required for this calculation
40. As a rocket blasts away from the earth with a cargo for the international space station, which of the following
graphs would best represent the gravitational force on the cargo versus distance from the surface of the Earth?
41. In chronological order (earliest to latest), place the following events:
(1) Henry Cavendish's experiment
(2) Newton's work leading towards the Law of Universal Gravitation
(3) Tycho Brahe takes astronomical data
(4) Nicolaus Copernicus proposes the heliocentric theory
(5) Johannes Kepler's work on the orbit of Mars.
(A) 43521 (B) 42135 (C) 43251 (D) 35421 (E) 42351
42. A 20 kg boulder rests on the surface of the Earth. Assume the Earth has mass 5.98 × 10 24 kg and g = 10 m/s 2 .
What is the magnitude of the gravitational force that the boulder exerts on the Earth?
(A) 5.98 × 10 25 N
(B) 5.98 × 10 24 N
(C) 20 N
(D) The boulder exerts no force on the Earth
(E) 200 N
43. An astronaut on the Moon simultaneously drops a feather and a hammer. The fact that they reach the surface at
the same instant shows that
(A) no gravity forces act on a body in a vacuum.
(B) the acceleration due to gravity on the Moon is less than the acceleration due to gravity on the Earth.
(C) the gravitational force from the Moon on heavier objects (the hammer) is equal to the gravitational force on
lighter objects (the feather).
(D) a hammer and feather have less mass on the Moon than on Earth.
(E) in the absence of air resistance all bodies at a given location fall with the same acceleration.
*44. A scientist in the International Space Station experiences "weightlessness" because
(A) there is no gravitational force from the Earth acting on her.
(B) the gravitational pull of the Moon has canceled the pull of the Earth on her.
(C) she is in free fall along with the Space Station and its contents.
(D) at an orbit of 200 miles above the Earth, the gravitational force of the Earth on her is 2% less than on its
surface.
(E) in space she has no mass.



45. A rocket is in a circular orbit with speed v and orbital radius R around a heavy stationary mass. An external
impulse is quickly applied to the rocket directly opposite to the velocity and the rocket’s speed is slowed to v/2,
putting the rocket into an elliptical orbit. In terms of R, the size of the semi-major axis a of this new elliptical
orbit is
1
(A) aa = RR
4
1
(B) aa = RR
(C) aa =
2
7
11
RR
(D) aa =
√8
3
RR
4
(E) aa = RR
7
*46. Kepler’s Second Law about “sweeping out equal areas in equal time” can be derived most directly from which
conservation law?
(A) energy (B) mechanical energy (C) angular momentum (D) mass (E) linear momentum
*47. Three equal mass satellites A, B, and C are in coplanar orbits around a planet as shown in the figure. The
magnitudes of the angular momenta of the satellites as measured about the planet are L A , L B , and L C . Which of
the following statements is correct?
(A) L A > L B > L C (B) L C > L B > L A (C) L B > L C > L A (D) L B > L A > L C
(E) The relationship between the magnitudes is different at various instants in time.
Questions 48 – 49
Two stars orbit their common center of mass as shown in the diagram below. The masses of the two stars are
3M and M. The distance between the stars is d.
*48. What is the value of the gravitational potential energy of the two star system?
(A) −
GGMM 2
(B)
dd
3GGMM 2
(C) −
dd
GGMM 2
(D) −
dd 2
*49. Determine the period of orbit for the star of mass 3M.
dd 3
(A) ππ� GGGG
(B)
3ππ
4
�
dd 3
GGGG
dd 3
(C) ππ� 3GGGG
3GGMM 2
dd
(E) −
dd 3
(D) 2ππ� GGGG
GGMM 2
dd 2
ππ
dd 3
(E) 4 � GGGG
*50. Two satellites are launched at a distance R from a planet of negligible radius. Both satellites are launched in the
tangential direction. The first satellite launches correctly at a speed v 0 and enters a circular orbit. The second
satellite, however, is launched at a speed 1⁄2v 0 . What is the minimum distance between the second satellite and
the planet over the course of its orbit?
1
1
1
1
1
(A) RR (B) RR (C) RR (D) RR (E) RR
√2
2
3
4
7
*51. When two stars are very far apart, their gravitational potential energy is zero; when they are separated by a
distance d, the gravitational potential energy of the system is U. What is the gravitational potential energy of the
system if the stars are separated by a distance 2d?
(A) U/4
(B) U/2 (C) U (D) 2U (E) 4U



52. The gravitational force on a textbook at the top of Pikes Peak (elevation 14,100 ft) is 40 newtons. What would
be the approximate gravitational force on the same textbook if it were taken to twice the elevation?
(A) 5 N (B) 10 N (C) 20 N (D) 40 N (E) 80 N
53. Assume the International Space Station has a mass m and is in a circular orbit of radius r about the center of the
Earth. If the Earth has a mass of M, what would be the speed of the Space Station around the Earth?
(A) �
GGGG
rr
GGGG
(B) �
rr 2
(C) �
GGGGGG
rr 2
GGGGGG
(D) �
rr
(E) √GGGGGGGG
54. What would be the gravitational force of attraction between the proton in the nucleus and the electron in an
orbit of radius 5.3 × 10 –11 m in a simple hydrogen atom?
(A) 2.0 × 10 –57 N (B) 3.7 × 10 –47 N (C) 6.1 × 10 –28 N (D) 8.2 × 10 –8 N
(E) the only force of attraction would be electrical
55. Two iron spheres separated by some distance have a minute gravitational attraction, F. If the spheres are moved
to one half their original separation and allowed to rust so that the mass of each sphere increases 41%, what
would be the resulting gravitational force?
(A) 2F (B) 4F (C) 6F (D) 8F (E) 10F
56. A ball which is thrown upward near the surface of the Earth with a velocity of 50 m/s will come to rest about 5
seconds later. If the ball were thrown up with the same velocity on Planet X, after 5 seconds it would be still
moving upwards at nearly 31 m/s. The magnitude of the gravitational field near the surface of Planet X is what
fraction of the gravitational field near the surface of the Earth?
(A) 0.16 (B) 0.39 (C) 0.53 (D) 0.63 (E) 1.59
57. An object placed on an equal arm balance requires 12 kg to balance it. When placed on a spring scale, the scale
reads 120 N. Everything (balance, scale, set of masses, and the object) is now transported to the moon where the
gravitational force is one-sixth that on Earth. What are the new readings of the balance and the spring scale
(respectively)?
(A) 12 kg, 20 N (B) 1 kg, 120 N (C) 12 kg, 720 N (D) 2 kg, 20 N (E) 2 kg, 120 N
58. Two artificial satellites I and II have circular orbits of radii R and 2R, respectively, about the same planet. The
orbital velocity of satellite I is v. What is the orbital velocity of satellite II?
vv
vv
(A)
(B)
(C) v (D) √2vv (E) 2v
2
√2
59. The gravitational acceleration on the surface of the moon is 1.6 m/s 2 . The radius of the moon is 1.7 × 10 6 m.
What is the period of a satellite placed in a low circular orbit about the moon?
(A) 1.0 × 10 3 s (B) 6.5 × 10 3 s (C) 1.1 × 10 6 s (D) 5.0 × 10 6 s (E) 7.1 × 10 12 s



\endinput


