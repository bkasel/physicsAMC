
%%--------------------------------------------------
%% College Board: Calculus BC
%%--------------------------------------------------

%% 310 Questions

%% 1969 AP Calculus BC: Section I (pp. 16)
%%--------------------------------------------------
\element{calculusBC}{
\begin{question}{1969-BC-q01}
    The asymptotes of the graph of the parametric equations $x = \dfrac{1}{t}$,
        $y=\dfrac{t}{t+1}$ are:
    \begin{multicols}{2}
    \begin{choices}
        \wrongchoice{$x=0$, $y=0$}
        \wrongchoice{$x=0$ only}
      \correctchoice{$x=-1$, $y=0$}
        \wrongchoice{$x=-1$ only}
        \wrongchoice{$x=0$, $y=1$}
    \end{choices}
    \end{multicols}
\end{question}
}

\element{calculusBC}{
\begin{question}{1969-BC-q02}
    What are the coordinates of the inflection point on the graph of $y=(x+1)\mathrm{arctan} x$?
    \begin{multicols}{3}
    \begin{choices}
        \wrongchoice{$\left(-1,0\right)$}
        \wrongchoice{$\left(0,0\right)$}
        \wrongchoice{$\left(0,1\right)$}
        \wrongchoice{$\left(1,\dfrac{\pi}{4}\right)$}
      \correctchoice{$\left(1,\dfrac{\pi}{2}\right)$}
    \end{choices}
    \end{multicols}
\end{question}
}

\element{calculusBC}{
\begin{question}{1969-BC-q03}
    The Mean Value Theorem guarantees the existence of a special point on the graph of $y=\sqrt{x}$
        between $(0,0)$ and $(4,2)$.
    What are the coordinates of this point?
    \begin{multicols}{2}
    \begin{choices}[o]
        \wrongchoice{$\left(2,1\right)$}
      \correctchoice{$\left(1,1\right)$}
        \wrongchoice{$\left(2,\sqrt{2}\right)$}
        \wrongchoice{$\left(\dfrac{1}{2},\dfrac{1}{\sqrt{2}}\right)$}
        \wrongchoice{None of the above}
    \end{choices}
    \end{multicols}
\end{question}
}

\element{calculusBC}{
\begin{question}{1969-BC-q04}
    $\displaystyle \int^{\;\;8}_{0} \dfrac{\mathrm{d}x}{\sqrt{1+x}} = $
    \begin{multicols}{3}
    \begin{choices}
        \wrongchoice{$1$}
        \wrongchoice{$\dfrac{3}{2}$}
        \wrongchoice{$2$}
      \correctchoice{$4$}
        \wrongchoice{$6$}
    \end{choices}
    \end{multicols}
\end{question}
}

\element{calculusBC}{
\begin{question}{1969-BC-q05}
    If $3x^2 + 2xy + y^2 = 2$, then the value of $\dfrac{\mathrm{d}y}{\mathrm{d}x}$ at $x=1$ is:
    \begin{multicols}{2}
    \begin{choices}
        \wrongchoice{$-2$}
        \wrongchoice{$0$}
        \wrongchoice{$2$}
        \wrongchoice{$4$}
      \correctchoice{not defined}
    \end{choices}
    \end{multicols}
\end{question}
}

\element{calculusBC}{
\begin{question}{1969-BC-q06}
    What is $\displaystyle \lim_{h \to 0} \frac{1}{h} \left[ 8\left(\frac{1}{2}+h\right)^8-8\left(\frac{1}{2}\right)^8 \right]$?
    \begin{choices}
        \wrongchoice{$0$}
      \correctchoice{$\dfrac{1}{2}$}
        \wrongchoice{$1$}
        \wrongchoice{The limit does not exist}
        \wrongchoice{It cannot be determined from the information given}
    \end{choices}
\end{question}
}

\element{calculusBC}{
\begin{question}{1969-BC-q07}
    For what value of $k$ will $x+\frac{k}{x}$ have a relative maximum at $x=-2$?
    \begin{multicols}{2}
    \begin{choices}
        \wrongchoice{$-4$}
        \wrongchoice{$-2$}
        \wrongchoice{$2$}
      \correctchoice{$4$}
        \wrongchoice{None of these}
    \end{choices}
    \end{multicols}
\end{question}
}

\element{calculusBC}{
\begin{question}{1969-BC-q08}
    If $h(x) = f^2(x) - g^2(x)$, $f\prime=-g(x)$, and $g\prime(x)=f(x)$, then $h\prime(x)=$
    \begin{multicols}{2}
    \begin{choices}
        \wrongchoice{$0$}
        \wrongchoice{$1$}
      \correctchoice{$-4f(x) g(x)$}
        \wrongchoice{$\left(-g(x)\right)^2 - \left(f(x)\right)^2$}
        \wrongchoice{$-2\left(-g(x) + f(x)\right)$}
    \end{choices}
    \end{multicols}
\end{question}
}

\element{calculusBC}{
\begin{question}{1969-BC-q09}
    The area of the closed region bounded by the polar graph of
        $r=\sqrt{3+\cos\theta}$ is given by the integral:
    \begin{multicols}{2}
    \begin{choices}
        \wrongchoice{$\displaystyle \int^{\;\;2\pi}_0 \sqrt{3+\cos\theta}\,\mathrm{d}\theta$}
        \wrongchoice{$\displaystyle \int^{\;\;\pi}_0 \sqrt{3+\cos\theta}\,\mathrm{d}\theta$}
        \wrongchoice{$\displaystyle 2 \int^{\;\;\pi/2}_0 \left(3+\cos\theta\right)\,\mathrm{d}\theta$}
      \correctchoice{$\displaystyle \int^{\;\;\pi}_0 \left(3+\cos\theta\right)\,\mathrm{d}\theta$}
        \wrongchoice{$\displaystyle 2\int^{\;\;\pi/2}_0 \sqrt{3+\cos\theta}\,\mathrm{d}\theta$}
    \end{choices}
    \end{multicols}
\end{question}
}

\element{calculusBC}{
\begin{question}{1969-BC-q10}
    $\displaystyle \int^{\;\;1}_0 \dfrac{x^2}{x^2+1}\,\mathrm{d}x=$
    \begin{multicols}{3}
    \begin{choices}
      \correctchoice{$\dfrac{4-\pi}{4}$}
        \wrongchoice{$\ln 2$}
        \wrongchoice{$0$}
        \wrongchoice{$\dfrac{1}{2}\ln 2$}
        \wrongchoice{$\dfrac{4+\pi}{4}$}
    \end{choices}
    \end{multicols}
\end{question}
}

\element{calculusBC}{
\begin{question}{1969-BC-q11}
    The point on the curve $x^2+2y=0$ that is nearest the point $\left(0,-\frac{1}{2}\right)$ occurs where $y$ is:
    \begin{multicols}{2}
    \begin{choices}[o]
        \wrongchoice{$\dfrac{1}{2}$}
      \correctchoice{$0$}
        \wrongchoice{$-\dfrac{1}{2}$}
        \wrongchoice{$-1$}
        \wrongchoice{none of the above}
    \end{choices}
    \end{multicols}
\end{question}
}

\element{calculusBC}{
\begin{question}{1969-BC-q12}
    If $\displaystyle F(x) = \int^{\;\;x}_{0} \mathrm{e}^{-t^2}\,\mathrm{d}t$, then $F\prime(x)=$
    \begin{multicols}{2}
    \begin{choices}
        \wrongchoice{$2x \mathrm{e}^{-x^2}$}
        \wrongchoice{$-2x \mathrm{e}^{-x^2}$}
        \wrongchoice{$\dfrac{\mathrm{e}^{-x^2+1}}{-x^2+1} - \mathrm{e}$}
        \wrongchoice{$\mathrm{e}^{-x^2} -1$}
      \correctchoice{$\mathrm{e}^{-x^2}$}
    \end{choices}
    \end{multicols}
\end{question}
}

\element{calculusBC}{
\begin{question}{1969-BC-q13}
    The region bounded by the $x$-axis and the part of the graph of $y=\cos x$
        between $x=-\frac{\pi}{2}$ and $x=\frac{\pi}{2}$ is separated into two
        regions by the line $x=k$.
    If the area of the region for $-\frac{\pi}{2}\leq x\leq k$ is three times
        the area of the region for $k\leq x\leq \frac{\pi}{2}$, then $k=$
    \begin{multicols}{2}
    \begin{choices}
        \wrongchoice{$\mathrm{arcsin}\left(\dfrac{1}{2}\right)$}
        \wrongchoice{$\mathrm{arcsin}\left(\dfrac{1}{3}\right)$}
      \correctchoice{$\dfrac{\pi}{6}$}
        \wrongchoice{$\dfrac{\pi}{4}$}
        \wrongchoice{$\dfrac{\pi}{3}$}
    \end{choices}
    \end{multicols}
\end{question}
}

\element{calculusBC}{
\begin{question}{1969-BC-q14}
    If $y=x^2+2$ and $u=2x-1$, then $\dfrac{\mathrm{d}y}{\mathrm{d}u} =$
    \begin{multicols}{2}
    \begin{choices}
        \wrongchoice{$\dfrac{2x^2-2x+4}{\left(2x-1\right)^2}$}
        \wrongchoice{$6x^2-2x+4$}
        \wrongchoice{$x^2$}
      \correctchoice{$x$}
        \wrongchoice{$\dfrac{1}{x}$}
    \end{choices}
    \end{multicols}
\end{question}
}

\element{calculusBC}{
\begin{question}{1969-BC-q15}
    If $f\prime(x)$ and $g\prime(x)$ exist and $f\prime>g\prime(x)$ for all real $x$,
        then the graph of $y=f(x)$ and the graph of $y=g(x)$
    \begin{choices}
        \wrongchoice{intersects exactly once}
      \correctchoice{intersects no more than once}
        \wrongchoice{do not intersect}
        \wrongchoice{could intersect more than once}
        \wrongchoice{have a common tangent at each point of intersection}
    \end{choices}
\end{question}
}

\element{calculusBC}{
\begin{question}{1969-BC-q16}
    If $y$ is a function $x$ such that $y\prime>0$ for all $x$
        and $y^{\dprime}<0$ for all $x$,
        which of the following could be part of the graph of $y=f(x)$?
    \begin{multicols}{2}
    \begin{choices}
        %% NOTE: ANS is B
        \wrongchoice{
            \begin{tikzpicture}
            \end{tikzpicture}
        }
    \end{choices}
    \end{multicols}
\end{question}
}

\element{calculusBC}{
\begin{question}{1969-BC-q17}
    The graph of $y=5x^4-x^5$ has a point of inflection at:
    \begin{multicols}{2}
    \begin{choices}
        \wrongchoice{$(0,0)$ only}
      \correctchoice{$(3,162)$ only}
        \wrongchoice{$(4,256)$ only}
        \wrongchoice{$(0,0)$ and $(3,162)$}
        \wrongchoice{$(0,0)$ and $(4,256)$}
    \end{choices}
    \end{multicols}
\end{question}
}

\element{calculusBC}{
\begin{question}{1969-BC-q18}
    If $f(x)=2 + |x-3|$ for all $x$, then the value of the derivative
        $f\prime(x)$ at $x=3$ is:
    \begin{multicols}{2}
    \begin{choices}
        \wrongchoice{$-1$}
        \wrongchoice{$0$}
        \wrongchoice{$1$}
        \wrongchoice{$2$}
      \correctchoice{nonexistent}
    \end{choices}
    \end{multicols}
\end{question}
}

\element{calculusBC}{
\begin{question}{1969-BC-q19}
    A point moves on the $x$-axis in such a way that its velocity at time $t(t>0)$
        is given by $v=\dfrac{\ln t}{t}$.
    At what value of $t$ does $v$ attain its maximum?
    \begin{choices}
        \wrongchoice{$1$}
        \wrongchoice{$\mathrm{e}^{\dfrac{1}{2}}$}
      \correctchoice{$\mathrm{e}$}
        \wrongchoice{$\mathrm{e}^{\dfrac{3}{2}}$}
        \wrongchoice{There is no maximum value for $v$.}
    \end{choices}
\end{question}
}

\element{calculusBC}{
\begin{question}{1969-BC-q20}
    An equation for a tangent to the graph of $y=\mathrm{arcsin}\dfrac{x}{2}$
        at the origin is:
    \begin{multicols}{2}
    \begin{choices}
      \correctchoice{$x-2y=0$}
        \wrongchoice{$x-y=0$}
        \wrongchoice{$x=0$}
        \wrongchoice{$y=0$}
        \wrongchoice{$\pi x - 2y=0$}
    \end{choices}
    \end{multicols}
\end{question}
}

\element{calculusBC}{
\begin{question}{1969-BC-q21}
    At $x=0$, which of the following is true of the function $f$    
        defined by $f(x) = x^2 + \mathrm{e}^{-2x}$
    \begin{choices}
        \wrongchoice{$f$ is increasing}
      \correctchoice{$f$ is decreasing}
        \wrongchoice{$f$ is discontinuous}
        \wrongchoice{$f$ has a relative minimum}
        \wrongchoice{$f$ has a relative maximum}
    \end{choices}
\end{question}
}

\element{calculusBC}{
\begin{question}{1969-BC-q22}
    If $f(x) = \int^{\,\,2}_0 \dfrac{1}{\sqrt{t^3+2}}\,\mathrm{d}t$,
        which of the following is \emph{false}?
    \begin{multicols}{2}
    \begin{choices}
        \wrongchoice{$f(0) = 0$}
        \wrongchoice{$f$ is continuous at $x$ for all $x\geq 0$}
        \wrongchoice{$f(1) > 0$}
        \wrongchoice{$f\prime(1) > \dfrac{1}{\sqrt{3}}$}
      \correctchoice{$f(-1) > 0$}
    \end{choices}
    \end{multicols}
\end{question}
}

\element{calculusBC}{
\begin{question}{1969-BC-q23}
    If the graph of $y=f(x)$ contains the point $(0,2)$, $\dfrac{\mathrm{d}y}{\mathrm{d}x}=\dfrac{-x}{y\mathrm{e}^{x^2}}$
        and $f(x)>0$ for all $x$, then $f(x)=$
    \begin{multicols}{2}
    \begin{choices}
        \wrongchoice{$3 + \mathrm{e}^{-x^2}$}
        \wrongchoice{$\sqrt{3} + \mathrm{e}^{-x}$}
        \wrongchoice{$1 + \mathrm{e}^{-x}$}
      \correctchoice{$\sqrt{3 + \mathrm{e}^{-x^2}}$}
        \wrongchoice{$\sqrt{3 + \mathrm{e}^{x^2}}$}
    \end{choices}
    \end{multicols}
\end{question}
}

\element{calculusBC}{
\begin{question}{1969-BC-q24}
    If $\sin x=\mathrm{e}^y$, $0<x<\pi$, what is $\dfrac{\mathrm{d}y}{\mathrm{d}x}$ in terms of $x$?
    \begin{multicols}{2}
    \begin{choices}
        \wrongchoice{$-\tan x$}
        \wrongchoice{$-\cot x$}
      \correctchoice{$\cot x$}
        \wrongchoice{$\tan x$}
        \wrongchoice{$\csc x$}
    \end{choices}
    \end{multicols}
\end{question}
}

\element{calculusBC}{
\begin{question}{1969-BC-q25}
    A region in the plane is bounded by the graph of $y=\frac{1}{x}$, the $x$-axis,
        the line $x=m$, and the line $x=2m$, $m>0$.
    The area of this region
    \begin{choices}
      \correctchoice{is independent of $m$.}
        \wrongchoice{increases as $m$ increases.}
        \wrongchoice{decreases as $m$ increases.}
        \wrongchoice{decreases as $m$ increases when $m<\frac{1}{2}$; increases as $m$ increases when $m>\frac{1}{2}$.}
        \wrongchoice{increases as $m$ increases when $m<\frac{1}{2}$; decreases as $m$ increases when $m>\frac{1}{2}$.}
    \end{choices}
\end{question}
}

\element{calculusBC}{
\begin{question}{1969-BC-q26}
    $\displaystyle \int^{\;\;1}_0 \sqrt{x^2-2x+1}\,\mathrm{d}x$ is:
    \begin{multicols}{2}
    \begin{choices}[o]
        \wrongchoice{$-1$}
        \wrongchoice{$-\dfrac{1}{2}$}
      \correctchoice{$\dfrac{1}{2}$}
        \wrongchoice{$1$}
        \wrongchoice{none of the above}
    \end{choices}
    \end{multicols}
\end{question}
}

\element{calculusBC}{
\begin{question}{1969-BC-q27}
    If $\dfrac{\mathrm{d}y}{\mathrm{d}x} = \tan x$, then $y=$
    \begin{multicols}{2}
    \begin{choices}
        \wrongchoice{$\dfrac{1}{2}\tan^2 x + C$}
        \wrongchoice{$\sec^2 x + C$}
      \correctchoice{$\ln |\sec x| + C$}
        \wrongchoice{$\ln |\cos x| + C$}
        \wrongchoice{$\sec x \tan x + C$}
    \end{choices}
    \end{multicols}
\end{question}
}

\element{calculusBC}{
\begin{question}{1969-BC-q28}
    What is $\displaystyle \lim_{x\to 0} \frac{\mathrm{e}^{2x}-1}{\tan x}$?
    \begin{choices}
        \wrongchoice{$-1$}
        \wrongchoice{$0$}
        \wrongchoice{$1$}
      \correctchoice{$2$}
        \wrongchoice{The limit does not exist.}
    \end{choices}
\end{question}
}

\element{calculusBC}{
\begin{question}{1969-BC-q29}
    $\displaystyle\int^{\;\;1}_0\left(4-x^2\right)^{-\tfrac{3}{2}}\,\mathrm{d}x=$ 
    \begin{multicols}{3}
    \begin{choices}
        \wrongchoice{$\dfrac{2-\sqrt{3}}{3}$}
        \wrongchoice{$\dfrac{2\sqrt{3} - 3}{4}$}
      \correctchoice{$\dfrac{\sqrt{3}}{12}$}
        \wrongchoice{$\dfrac{\sqrt{3}}{3}$}
        \wrongchoice{$\dfrac{\sqrt{3}}{2}$}
    \end{choices}
    \end{multicols}
\end{question}
}

\element{calculusBC}{
\begin{question}{1969-BC-q30}
    $\displaystyle \sum_{n=0}^{\infty} \frac{\left(-1\right)^n x^n}{n!}$ is the Taylor series about zero for which of the following functions?
    \begin{multicols}{2}
    \begin{choices}
        \wrongchoice{$\sin x$}
        \wrongchoice{$\cos x$}
        \wrongchoice{$\mathrm{e}^x$}
      \correctchoice{$\mathrm{e}^{-x}$}
        \wrongchoice{$\ln\left(1+x\right)$}
    \end{choices}
    \end{multicols}
\end{question}
}

\element{calculusBC}{
\begin{question}{1969-BC-q31}
    If $f\prime(x) = -f(x)$ and $f(1) = 1$, then $f(x)=$
    \begin{multicols}{2}
    \begin{choices}
        \wrongchoice{$\dfrac{1}{2} \mathrm{e}^{-2x+2}$}
        \wrongchoice{$\mathrm{e}^{-x-1}$}
      \correctchoice{$\mathrm{e}^{1-x}$}
        \wrongchoice{$\mathrm{e}^{-x}$}
        \wrongchoice{$-\mathrm{e}^{x}$}
    \end{choices}
    \end{multicols}
\end{question}
}

\element{calculusBC}{
\begin{question}{1969-BC-q32}
    For what values of $x$ does the series $1 + 2^x + 3^x + r^x + \cdots + n^x + \cdots$ converge?
    \begin{multicols}{2}
    \begin{choices}
        \wrongchoice{no values of $x$}
      \correctchoice{$x<-1$}
        \wrongchoice{$x\geq -1$}
        \wrongchoice{$x> -1$}
        \wrongchoice{All values of $x$}
    \end{choices}
    \end{multicols}
\end{question}
}

\element{calculusBC}{
\begin{question}{1969-BC-q33}
    What is the average (mean) value of $et^3 - t^2$ over the interval $-1\leq t\leq2$?
    \begin{multicols}{2}
    \begin{choices}
      \correctchoice{$\dfrac{11}{4}$}
        \wrongchoice{$\dfrac{7}{2}$}
        \wrongchoice{$8$}
        \wrongchoice{$\dfrac{33}{4}$}
        \wrongchoice{$16$}
    \end{choices}
    \end{multicols}
\end{question}
}

\element{calculusBC}{
\begin{question}{1969-BC-q34}
    %% NOTE:
    Which of the following is an equation of a curve that intersects at right angles every curve
        of the family $y=\dfrac{1}{x} + k$ (where $k$ takes all real values)?
    \begin{multicols}{2}
    \begin{choices}
        \wrongchoice{$y=-x$}
        \wrongchoice{$y=-x^2$}
        \wrongchoice{$y=-\dfrac{1}{3}x^3$}
      \correctchoice{$y=\dfrac{1}{3}x^3$}
        \wrongchoice{$y=\ln x$}
    \end{choices}
    \end{multicols}
\end{question}
}

\element{calculusBC}{
\begin{question}{1969-BC-q35}
    At $t=0$ a particle starts at rest and moves along a line in such a way that at time $t$
        its acceleration is $24t^2$ feet per second per second.
    Through how many feet does the particle move during the first 2 seconds?
    \begin{multicols}{3}
    \begin{choices}
      \correctchoice{$32$}
        \wrongchoice{$48$}
        \wrongchoice{$64$}
        \wrongchoice{$96$}
        \wrongchoice{$192$}
    \end{choices}
    \end{multicols}
\end{question}
}

\element{calculusBC}{
\begin{question}{1969-BC-q36}
    The approximate value of $y=\sqrt{4+\sin x}$ at $x=0.12$,
        obtained from the tangent to the graph at $x=0$, is:
    \begin{multicols}{3}
    \begin{choices}
        \wrongchoice{$2.00$}
      \correctchoice{$2.03$}
        \wrongchoice{$2.06$}
        \wrongchoice{$2.16$}
        \wrongchoice{$2.24$}
    \end{choices}
    \end{multicols}
\end{question}
}

\element{calculusBC}{
\begin{question}{1969-BC-q37}
    Of the following choices of $\delta$,
        which is the largest that could be used successfully with an arbitrary $\epsilon$
        in an epsilon-delta proof of $\displaystyle \lim_{x\to 2} \left(1-3x\right) = -5$?
    \begin{multicols}{3}
    \begin{choices}
        \wrongchoice{$\delta = 3\epsilon$}
        \wrongchoice{$\delta = \epsilon$}
        \wrongchoice{$\delta = \dfrac{\epsilon}{2}$}
      \correctchoice{$\delta = \dfrac{\epsilon}{4}$}
        \wrongchoice{$\delta = \dfrac{\epsilon}{5}$}
    \end{choices}
    \end{multicols}
\end{question}
}

\element{calculusBC}{
\begin{question}{1969-BC-q38}
    If $f(x) = \left(x^2+1\right)^{2-3x}$,
        then $f\prime(1)=$
    \begin{multicols}{3}
    \begin{choices}
      \correctchoice{$-\dfrac{1}{2} \ln\left(8\mathrm{e}\right)$}
        \wrongchoice{$-\ln\left(8\mathrm{e}\right)$}
        \wrongchoice{$-\dfrac{3}{2} \ln\left(2\right)$}
        \wrongchoice{$-\dfrac{1}{2}$}
        \wrongchoice{$\dfrac{1}{8}$}
    \end{choices}
    \end{multicols}
\end{question}
}

\element{calculusBC}{
\begin{question}{1969-BC-q39}
    If $y=\tan u$, $u=v-\dfrac{1}{v}$, and $v=\ln x$,
        what is the value of $\dfrac{\mathrm{d}y}{\mathrm{d}x}$ at $x=\mathrm{e}$?
    \begin{multicols}{3}
    \begin{choices}
        \wrongchoice{$0$}
        \wrongchoice{$\dfrac{1}{\mathrm{e}}$}
        \wrongchoice{$1$}
      \correctchoice{$\dfrac{2}{\mathrm{e}}$}
        \wrongchoice{$\sec^2 \mathrm{e}$}
    \end{choices}
    \end{multicols}
\end{question}
}

\element{calculusBC}{
\begin{question}{1969-BC-q40}
    If $n$ is a non-negative integer, the $\int^{\,\,1}_0 x^n\,\mathrm{d}x = \int^{\,\,1}_0 \left(1-x\right)^n\,\mathrm{d}x$ for:
    \begin{multicols}{2}
    \begin{choices}
        \wrongchoice{no $n$}
        \wrongchoice{$n$ even, only}
        \wrongchoice{$n$ odd, only}
        \wrongchoice{nonzero $n$, only}
      \correctchoice{all $n$}
    \end{choices}
    \end{multicols}
\end{question}
}

\element{calculusBC}{
\begin{question}{1969-BC-q41}
    If \begin{math}
    \begin{cases}
        f(x) = 8-x^2 & \text{for } -2\leq x\leq 2 \\
        f(x) = x^2   & \text{elsewhere,} \\
    \end{cases}
    \end{math}
    Then $\displaystyle \int^{\;\;3}_{-1} f(x)\,\mathrm{d}x$ is a number between
    \begin{multicols}{2}
    \begin{choices}
        \wrongchoice{$0$ and $8$}
        \wrongchoice{$8$ and $16$}
        \wrongchoice{$16$ and $24$}
      \correctchoice{$24$ and $32$}
        \wrongchoice{$32$ and $40$}
    \end{choices}
    \end{multicols}
\end{question}
}

\element{calculusBC}{
\begin{question}{1969-BC-q42}
    If $\int\,x^2\cos x\,\mathrm{d}x = f(x) - \int\,2x\sin x\,\mathrm{d}x$,
        then $f(x) = $
    \begin{choices}
        \wrongchoice{$2\sin x + 2x\cos x + C$}
      \correctchoice{$x^2 \sin x + C$}
        \wrongchoice{$2x\cos x - x^2 \sin x + C$}
        \wrongchoice{$4\cos x - 2x\sin x + C$}
        \wrongchoice{$\left(2 x^2\right)\cos x - 4\sin x + C$}
    \end{choices}
\end{question}
}

\element{calculusBC}{
\begin{question}{1969-BC-q43}
    Which of the following integrals gives the length of the graph of $y=\tan x$ between $x=a$ and $x=b$, where $0<a<b<\frac{\pi}{2}$.
    \begin{multicols}{2}
    \begin{choices}
        \wrongchoice{$\displaystyle \int^{\;\;b}_{a} \sqrt{x^2 + \tan^2 x}\,\mathrm{d}x$}
        \wrongchoice{$\displaystyle \int^{\;\;b}_{a} \sqrt{x + \tan x}\,\mathrm{d}x$}
        \wrongchoice{$\displaystyle \int^{\;\;b}_{a} \sqrt{1 + \sec^2 x}\,\mathrm{d}x$}
        \wrongchoice{$\displaystyle \int^{\;\;b}_{a} \sqrt{1 + \tan^2 x}\,\mathrm{d}x$}
      \correctchoice{$\displaystyle \int^{\;\;b}_{a} \sqrt{1 + \sec^4 x}\,\mathrm{d}x$}
    \end{choices}
    \end{multicols}
\end{question}
}

\element{calculusBC}{
\begin{question}{1969-BC-q44}
    If $f\dprime{}(x) - f\prime{}(x) - 2f(x) = 0$, $f\prime{}(0) = -2$, and $f(0)=2$, then $f(1)=$
    \begin{multicols}{3}
    \begin{choices}
        \wrongchoice{$\mathrm{e}^2 + \mathrm{e}^{-1}$}
        \wrongchoice{$1$}
        \wrongchoice{$0$}
        \wrongchoice{$\mathrm{e}^2$}
      \correctchoice{$2\mathrm{e}^2$}
    \end{choices}
    \end{multicols}
\end{question}
}

\element{calculusBC}{
\begin{question}{1969-BC-q45}
    The complete interval of convergence of the series $\displaystyle \sum^{\infty}_{k=1} \frac{\left(x+1\right)^k}{k^2}$ is:
    \begin{multicols}{2}
    \begin{choices}
        \wrongchoice{$0 < x < 2$}
        \wrongchoice{$0 \leq x \leq 2$}
        \wrongchoice{$-2 < x \leq 0$}
        \wrongchoice{$-2 \leq x < 2$}
      \correctchoice{$-2 \leq x \leq 0$}
    \end{choices}
    \end{multicols}
\end{question}
}


%% 1973 AP Calculus BC: Section I (pp. 35)
%%--------------------------------------------------
\element{calculusBC}{
\begin{question}{1973-BC-q01}
    If $f(x) = \mathrm{e}^{\frac{1}{x}}$, then $f\prime{}(x) = $
    \begin{multicols}{2}
    \begin{choices}
      \correctchoice{$-\dfrac{\mathrm{e}^{\frac{1}{x}}}{x^2}$}
        \wrongchoice{$-\mathrm{e}^{\frac{1}{x}}$}
        \wrongchoice{$\dfrac{\mathrm{e}^{\frac{1}{x}}}{x}$}
        \wrongchoice{$\dfrac{\mathrm{e}^{\frac{1}{x}}}{x^2}$}
        \wrongchoice{$\dfrac{1}{x}\mathrm{e}^{\frac{1}{x}-1}$}
    \end{choices}
    \end{multicols}
\end{question}
}

\element{calculusBC}{
\begin{question}{1973-BC-q02}
    $\displaystyle \int^{\;\;3}_{0} \left(x+1\right)^{1/2}\,\mathrm{d}x = $
    \begin{multicols}{3}
    \begin{choices}
        \wrongchoice{$\dfrac{21}{2}$}
        \wrongchoice{$7$}
        \wrongchoice{$\dfrac{16}{3}$}
      \correctchoice{$\dfrac{14}{3}$}
        \wrongchoice{$-\dfrac{1}{4}$}
    \end{choices}
    \end{multicols}
\end{question}
}

\element{calculusBC}{
\begin{question}{1973-BC-q03}
    If $f(x) = x + \frac{1}{x}$, then the set of values for which $f$ increases is:
    \begin{multicols}{2}
    \begin{choices}
      \correctchoice{$\left(\infty,-1\right]\cup{}\left[1,\infty\right)$}
        \wrongchoice{$\left[-1,1\right]$}
        \wrongchoice{$\left(-\infty,\infty\right)$}
        \wrongchoice{$\left(0,\infty\right)$}
        \wrongchoice{$\left(-\infty,0\right)\cup{}\left(0,\infty\right)$}
    \end{choices}
    \end{multicols}
\end{question}
}

\element{calculusBC}{
\begin{question}{1973-BC-q04}
    For what non-negative value of $b$ is the line given by $y=\frac{1}{3}x + b$ normal to the curve $y=x^3$?
    \begin{multicols}{3}
    \begin{choices}
        \wrongchoice{zero}
        \wrongchoice{one}
      \correctchoice{$\dfrac{4}{3}$}
        \wrongchoice{$\dfrac{10}{3}$}
        \wrongchoice{$\dfrac{10\sqrt{3}}{3}$}
    \end{choices}
    \end{multicols}
\end{question}
}

\element{calculusBC}{
\begin{question}{1973-BC-q05}
    $\displaystyle \int^{\;\;2}_{-1} \frac{\left| x\right|}{x}\,\mathrm{d}x$ is:
    \begin{multicols}{2}
    \begin{choices}
        \wrongchoice{$-3$}
      \correctchoice{$1$}
        \wrongchoice{$2$}
        \wrongchoice{$3$}
        \wrongchoice{nonexistent}
    \end{choices}
    \end{multicols}
\end{question}
}

\element{calculusBC}{
\begin{question}{1973-BC-q06}
    If $f(x)=\dfrac{x-1}{x+1}$ for all $x\neq 1$, then $f\prime{}(1)=$
    \begin{multicols}{3}
    \begin{choices}
        \wrongchoice{$-1$}
        \wrongchoice{$-\dfrac{1}{2}$}
        \wrongchoice{zero}
      \correctchoice{$\dfrac{1}{2}$}
        \wrongchoice{$1$}
    \end{choices}
    \end{multicols}
\end{question}
}

\element{calculusBC}{
\begin{question}{1973-BC-q07}
    If $y=\ln\left(x^2+y^2\right)$, then the value of $\dfrac{\mathrm{d}y}{\mathrm{d}x}$ at the point $(1,0)$ is:
    \begin{multicols}{2}
    \begin{choices}
        \wrongchoice{zero}
        \wrongchoice{$\dfrac{1}{2}$}
        \wrongchoice{$1$}
      \correctchoice{$2$}
        \wrongchoice{undefined}
    \end{choices}
    \end{multicols}
\end{question}
}

\element{calculusBC}{
\begin{question}{1973-BC-q08}
    If $y=\sin x$ and $y^{(n)}$ means the  `` $n$\textsuperscript{th} derivative of $y$ with respect to $x$,'' then the smallest positive integer $n$ for which $y^{(n)} = y$ is:
    \begin{multicols}{3}
    \begin{choices}
        \wrongchoice{$2$}
      \correctchoice{$4$}
        \wrongchoice{$5$}
        \wrongchoice{$6$}
        \wrongchoice{$8$}
    \end{choices}
    \end{multicols}
\end{question}
}

\element{calculusBC}{
\begin{question}{1973-BC-q09}
    If $y=\cos^2 3x$, then $\dfrac{\mathrm{d}y}{\mathrm{d}x} = $
    \begin{multicols}{2}
    \begin{choices}
      \correctchoice{$-6\sin 3x \cos 3x$}
        \wrongchoice{$-2\sin 3x$}
        \wrongchoice{$2\cos 3x$}
        \wrongchoice{$6\cos 3x$}
        \wrongchoice{$2\sin 3x \cos 3x$}
    \end{choices}
    \end{multicols}
\end{question}
}

\element{calculusBC}{
\begin{question}{1973-BC-q10}
    The length of the curve $y=\ln\sec x$ from $x=0$ to $x=b$,
        where $0<b<\frac{\pi}{2}$, may be expressed by which of the following integrals?
    \begin{choices}
      \correctchoice{$\displaystyle \int^{\;\;b}_{0} \sec x\,\mathrm{d}x$}
        \wrongchoice{$\displaystyle \int^{\;\;b}_{0} \sec^2 x\,\mathrm{d}x$}
        \wrongchoice{$\displaystyle \int^{\;\;b}_{0} \left(\sec^2  \tan x\right) \,\mathrm{d}x$}
        \wrongchoice{$\displaystyle \int^{\;\;b}_{0} \sqrt{1+\left(\ln \sec x\right)^2}\,\mathrm{d}x$}
        \wrongchoice{$\displaystyle \int^{\;\;b}_{0} \sqrt{1+\left(\sec^2 x \tan^2 x\right)}\,\mathrm{d}x$}
    \end{choices}
\end{question}
}

\element{calculusBC}{
\begin{question}{1973-BC-q11}
    Let $y=x\sqrt{1+x^2}$.
    When $x=0$ and $dx=2$, then value of $dy$ is:
    \begin{multicols}{3}
    \begin{choices}
        \wrongchoice{$-2$}
        \wrongchoice{$-1$}
        \wrongchoice{$0$}
        \wrongchoice{$1$}
      \correctchoice{$2$}
    \end{choices}
    \end{multicols}
\end{question}
}

\element{calculusBC}{
\begin{question}{1973-BC-q12}
    If $n$ is a known positive integer,
        for what value of $k$ is $\int^{\,\,k}_{1} x^{n-1}\,\mathrm{d}x = \frac{1}{n}$?
    \begin{multicols}{2}
    \begin{choices}
        \wrongchoice{$0$}
        \wrongchoice{$\left(\dfrac{2}{n}\right)^{1/n}$}
        \wrongchoice{$\left(\dfrac{2n-1}{n}\right)^{1/n}$}
      \correctchoice{$2^{1/n}$}
        \wrongchoice{$2^{n}$}
    \end{choices}
    \end{multicols}
\end{question}
}

\element{calculusBC}{
\begin{question}{1973-BC-q13}
    The acceleration $\alpha$ of a body moving in a straight line is given in terms of time $t$ by $\alpha=8-6t$.
    If the velocity of the body is 25 at $t=1$ and if $s(t)$ is the distance of the body from the origin at time $t$, what is $s(4)-s(2)$?
    \begin{multicols}{3}
    \begin{choices}
        \wrongchoice{20}
        \wrongchoice{24}
        \wrongchoice{28}
      \correctchoice{32}
        \wrongchoice{42}
    \end{choices}
    \end{multicols}
\end{question}
}

\element{calculusBC}{
\begin{question}{1973-BC-q14}
    If $x=t^2 - 1$ and $y=2\mathrm{e}^t$, then $\dfrac{\mathrm{d}y}{\mathrm{d}x} = $
    \begin{multicols}{3}
    \begin{choices}
      \correctchoice{$\dfrac{\mathrm{e}^{t}}{t}$}
        \wrongchoice{$\dfrac{2\mathrm{e}^{t}}{t}$}
        \wrongchoice{$\dfrac{2\mathrm{e}^{\left|t\right|}}{t^2}$}
        \wrongchoice{$\dfrac{4\mathrm{e}^{t}}{2t-1}$}
        \wrongchoice{$\mathrm{e}^{t}$}
    \end{choices}
    \end{multicols}
\end{question}
}

\element{calculusBC}{
\begin{question}{1973-BC-q15}
    The area of the region bounded by the lines $x=0$, $x=2$ and $y=0$ and the curve $y=\mathrm{e}^{x/2}$ is:
    \begin{multicols}{3}
    \begin{choices}
        \wrongchoice{$\dfrac{\mathrm{e}-1}{2}$}
        \wrongchoice{$\mathrm{e}-1$}
      \correctchoice{$2\left(\mathrm{e}-1\right)$}
        \wrongchoice{$2\mathrm{e}-1$}
        \wrongchoice{$2\mathrm{e}$}
    \end{choices}
    \end{multicols}
\end{question}
}

\element{calculusBC}{
\begin{question}{1973-BC-q16}
    A series expansion of $\dfrac{\sin t}{t}$ is:
    \begin{choices}
      \correctchoice{$1 - \dfrac{t^2}{3!} + \dfrac{t^4}{5!} - \dfrac{t^6}{7!} + \cdots$}
        \wrongchoice{$\dfrac{1}{t} - \dfrac{t}{2!} + \dfrac{t^3}{4!} - \dfrac{t^5}{6!} + \cdots$}
        \wrongchoice{$1 + \dfrac{t^2}{3!} + \dfrac{t^4}{5!} + \dfrac{t^6}{7!} + \cdots$}
        \wrongchoice{$\dfrac{1}{t} + \dfrac{t}{2!} + \dfrac{t^3}{4!} + \dfrac{t^5}{6!} + \cdots$}
        \wrongchoice{$t - \dfrac{t^3}{3!} + \dfrac{t^5}{5!} - \dfrac{t^7}{7!} + \cdots$}
    \end{choices}
\end{question}
}

\element{calculusBC}{
\begin{question}{1973-BC-q17}
    The number of bacteria in a culture is growing at a rate of $\num{3000}\,\mathrm{e}^{2t/5}$ per unit of time $t$.
    At $t=0$, the number of bacteria present was \num{7500}.
    Find the number present at $t=5$.
    \begin{multicols}{2}
    \begin{choices}
        \wrongchoice{$\num{1200}\mathrm{e}^2$}
        \wrongchoice{$\num{3000}\mathrm{e}^2$}
      \correctchoice{$\num{7500}\mathrm{e}^2$}
        \wrongchoice{$\num{7500}\mathrm{e}^5$}
        \wrongchoice{$\dfrac{\num{15 000}}{7}\mathrm{e}^7$}
    \end{choices}
    \end{multicols}
\end{question}
}

\element{calculusBC}{
\begin{question}{1973-BC-q18}
    Let $g$ be a continuous function on the closed interval $\left[0,1\right]$.
    Let $g(0)=1$ and $g(1)=0$. 
    Which of the following is \emph{not} necessarily true?
    \begin{choices}
        \wrongchoice{There exists a number $h$ in $\left[0,1\right]$ such that $g(h)\geq g(x)$ for all $x$ in $\left[0,1\right]$}
        \wrongchoice{For all $a$ and $b$ in $\left[0,1\right]$, if $a=b$, then $g(a) = g(b)$}
        \wrongchoice{There exists a number $h$ in $\left[0,1\right]$ such that $g(h) = \dfrac{1}{2}$}
      \correctchoice{There exists a number $h$ in $\left[0,1\right]$ such that $g(h) = \dfrac{3}{2}$}
        \wrongchoice{For all $h$ in the open interval $\left(0,1\right)$, $\lim_{x\to h} g(x) = g(h)$}
    \end{choices}
\end{question}
}

\element{calculusBC}{
\begin{questionmult}{1973-BC-q19}
    Which of the following series converge?
    \begin{multicols}{3}
    \begin{choices}
      \correctchoice{$\displaystyle \sum_{n=1}^{\infty} \frac{1}{n^2}$}
        \wrongchoice{$\displaystyle \sum_{n=1}^{\infty} \frac{1}{n}$}
      \correctchoice{$\displaystyle \sum_{n=1}^{\infty} \frac{\left(-1\right)^{n}}{\sqrt{n}}$}
    \end{choices}
    \end{multicols}
\end{questionmult}
}

\element{calculusBC}{
\begin{question}{1973-BC-q20}
    $\displaystyle \int \, x \sqrt{4-x^2}\,\mathrm{d}x = $
    \begin{multicols}{2}
    \begin{choices}
        \wrongchoice{$\dfrac{\left(4-x^2\right)^{3/2}}{3} + C$}
        \wrongchoice{$-\left(4-x^2\right)^{3/2} + C$}
        \wrongchoice{$\dfrac{x^2\left(4-x^2\right)^{3/2}}{3} + C$}
        \wrongchoice{$-\dfrac{x^2\left(4-x^2\right)^{3/2}}{3} + C$}
      \correctchoice{$-\dfrac{\left(4-x^2\right)^{3/2}}{3} + C$}
    \end{choices}
    \end{multicols}
\end{question}
}

\element{calculusBC}{
\begin{question}{1973-BC-q21}
    $\displaystyle \int \,\left(x+1\right) \mathrm{e}^{x^2+2x}\,\mathrm{d}x = $
    \begin{multicols}{3}
    \begin{choices}
        \wrongchoice{$\dfrac{\mathrm{e}^3}{2}$}
      \correctchoice{$\dfrac{\mathrm{e}^3-1}{2}$}
        \wrongchoice{$\dfrac{\mathrm{e}^4-\mathrm{e}}{2}$}
        \wrongchoice{$\mathrm{e}^3-1$}
        \wrongchoice{$\mathrm{e}^4-\mathrm{e}$}
    \end{choices}
    \end{multicols}
\end{question}
}

\element{calculusBC}{
\begin{question}{1973-BC-q22}
    A particle moves on the curve $y=\ln x$ so that the $x$-component has velocity $x\prime(t) = t+1$ for $t\geq 0$.
    At time $t=0$, the particle is at the point $(1,0)$.
    At time $t=1$, the particle is at the point:
    \begin{multicols}{2}
    \begin{choices}
        \wrongchoice{$\left(2,\ln 2\right)$}
        \wrongchoice{$\left(\mathrm{e}^2,2\right)$}
      \correctchoice{$\left(\dfrac{5}{2},\ln\dfrac{5}{2}\right)$}
        \wrongchoice{$\left(3,\ln 3\right)$}
        \wrongchoice{$\left(\dfrac{3}{2},\ln\dfrac{3}{2}\right)$}
    \end{choices}
    \end{multicols}
\end{question}
}

\element{calculusBC}{
\begin{question}{1973-BC-q23}
    $\displaystyle \lim_{h\to 0} \frac{1}{h} \ln\left(\frac{2+h}{2}\right)$ is:
    \begin{multicols}{2}
    \begin{choices}
        \wrongchoice{$\mathrm{e}^2$}
        \wrongchoice{one}
      \correctchoice{$\dfrac{1}{2}$}
        \wrongchoice{zero}
        \wrongchoice{nonexistent}
    \end{choices}
    \end{multicols}
\end{question}
}

\element{calculusBC}{
\begin{question}{1973-BC-q24}
    Let $f(x)=3x+1$ for all real $x$ and let $\epsilon>0$.
    For which of the following choices of $\delta$ is $\left| f(x)-7 \right| < \epsilon$ whenever $\left|x-2\right| < \delta$\,?
    \begin{multicols}{3}
    \begin{choices}
      \correctchoice{$\dfrac{\epsilon}{4}$}
        \wrongchoice{$\dfrac{\epsilon}{2}$}
        \wrongchoice{$\dfrac{\epsilon}{\epsilon +1}$}
        \wrongchoice{$\dfrac{\epsilon+1}{\epsilon}$}
        \wrongchoice{$3\epsilon$}
    \end{choices}
    \end{multicols}
\end{question}
}

\element{calculusBC}{
\begin{question}{1973-BC-q25}
    $\displaystyle \int^{\;\;\pi/4}_{0} \, \tan^2 x \,\mathrm{d}x = $
    \begin{multicols}{3}
    \begin{choices}
        \wrongchoice{$\dfrac{\pi}{4} -1$}
      \correctchoice{$1 - \dfrac{\pi}{4}$}
        \wrongchoice{$\dfrac{1}{3}$}
        \wrongchoice{$\sqrt{2}-1$}
        \wrongchoice{$\dfrac{\pi}{4} +1$}
    \end{choices}
    \end{multicols}
\end{question}
}

\element{calculusBC}{
\begin{question}{1973-BC-q26}
    Which of the following is true about the graph of $y=\ln\left|x^2-1\right|$ in the interval $(-1,1)$?  $\displaystyle \int^{\;\;\pi/4}_{0} \tan^2 x \,\mathrm{d}x = $
    \begin{choices}
        \wrongchoice{It is increasing.}
        \wrongchoice{It attains a relative minimum at $(0,0)$}
        \wrongchoice{It has a range of all real numbers.}
      \correctchoice{It is concave down.}
        \wrongchoice{It has an asymptote of $x=0$.}
    \end{choices}
\end{question}
}

\element{calculusBC}{
\begin{question}{1973-BC-q27}
    If $f\left(x\right) = \frac{1}{3} x^3 - 4x^2 + 12 x - 5$ and the domain is the set of all $x$ such that $0\leq x\leq 9$,
        then the absolute maximum value of the function $f$ occurs when $x$ is:
    \begin{multicols}{3}
    \begin{choices}
        \wrongchoice{$0$}
        \wrongchoice{$2$}
        \wrongchoice{$4$}
        \wrongchoice{$6$}
      \correctchoice{$9$}
    \end{choices}
    \end{multicols}
\end{question}
}

\element{calculusBC}{
\begin{question}{1973-BC-q28}
    If the substitution $\sqrt{x}=\sin y$ is made in the integrand of $\int^{\,\,1/2}_{0} \frac{\sqrt{x}}{\sqrt{1-x}}\,\mathrm{d}x$,
        the resulting integral is:
    \begin{multicols}{2}
    \begin{choices}
        \wrongchoice{$\displaystyle \int^{\;\;1/2}_{0} \sin^2 y \mathrm{d}y$}
        \wrongchoice{$\displaystyle 2\int^{\;\;1/2}_{0} \dfrac{\sin^2 y}{\cos y} \mathrm{d}y$}
      \correctchoice{$\displaystyle 2\int^{\;\;\pi/4}_{0} \sin^2 y \mathrm{d}y$}
        \wrongchoice{$\displaystyle \int^{\;\;\pi/4}_{0} \sin^2 y \mathrm{d}y$}
        \wrongchoice{$\displaystyle 2\int^{\;\;\pi/6}_{0} \sin^2 y \mathrm{d}y$}
    \end{choices}
    \end{multicols}
\end{question}
}

\element{calculusBC}{
\begin{question}{1973-BC-q29}
    If $y^n = 2y\prime$ and if $y=y\prime=\mathrm{e}$ when $x=0$, then when $x=1$, $y=$
    \begin{multicols}{3}
    \begin{choices}
      \correctchoice{$\dfrac{\mathrm{e}}{2}\left(\mathrm{e}^2+1\right)$}
        \wrongchoice{$\mathrm{e}$}
        \wrongchoice{$\dfrac{\mathrm{e}^3}{2}$}
        \wrongchoice{$\dfrac{\mathrm{e}}{2}$}
        \wrongchoice{$\dfrac{\mathrm{e}^3-\mathrm{e}}{2}$}
    \end{choices}
    \end{multicols}
\end{question}
}

\element{calculusBC}{
\begin{question}{1973-BC-q30}
    $\displaystyle \int^{\;\;2}_{1} \frac{x-4}{x^2}\,\mathrm{d}x = $
    \begin{multicols}{3}
    \begin{choices}
        \wrongchoice{$-\dfrac{1}{2}$}
      \correctchoice{$\ln 2 - 2$}
        \wrongchoice{$\ln 2$}
        \wrongchoice{$2$}
        \wrongchoice{$\ln 2 + 2$}
    \end{choices}
    \end{multicols}
\end{question}
}

\element{calculusBC}{
\begin{question}{1973-BC-q31}
    If $f(x) = \ln\left(\ln x\right)$, then $f\prime(x) = $
    \begin{multicols}{3}
    \begin{choices}
        \wrongchoice{$\dfrac{1}{x}$}
        \wrongchoice{$\dfrac{1}{\ln x}$}
        \wrongchoice{$\dfrac{\ln x}{x}$}
        \wrongchoice{$x$}
      \correctchoice{$\dfrac{1}{x\ln x}$}
    \end{choices}
    \end{multicols}
\end{question}
}

\element{calculusBC}{
\begin{question}{1973-BC-q32}
    If $y=x^{\ln x}$, then $y\prime$ is:
    \begin{multicols}{2}
    \begin{choices}
        \wrongchoice{$\dfrac{x^{\ln x}\ln x}{x^2}$}
        \wrongchoice{$x^{1/x}\ln x$}
      \correctchoice{$\dfrac{2x^{\ln x}\ln x}{x}$}
        \wrongchoice{$\dfrac{x^{\ln x}\ln x}{x}$}
        \wrongchoice{none of the provided}
    \end{choices}
    \end{multicols}
\end{question}
}

\element{calculusBC}{
\begin{question}{1973-BC-q33}
    Suppose that $f$ is an odd function; i.e., $f(-x) = -f(x)$ for all $x$.
    Suppose that $f\prime(x_0)$ exists.
    Which of the following must necessarily be equal to $f\prime(-x_0)$?
    \begin{multicols}{2}
    \begin{choices}
      \correctchoice{$f\prime\left(x\right)$}
        \wrongchoice{$-f\prime\left(x_0\right)$}
        \wrongchoice{$\dfrac{1}{f\prime\left(x_0\right)}$}
        \wrongchoice{$-\dfrac{1}{f\prime\left(x_0\right)}$}
        \wrongchoice{none of the provided}
    \end{choices}
    \end{multicols}
\end{question}
}

\element{calculusBC}{
\begin{question}{1973-BC-q34}
    The average (mean) value of $\sqrt{x}$ over the interval $0\leq x\leq 2$ is:
    \begin{multicols}{3}
    \begin{choices}
        \wrongchoice{$\dfrac{1}{3} \sqrt{2}$}
        \wrongchoice{$\dfrac{1}{2} \sqrt{2}$}
      \correctchoice{$\dfrac{2}{3} \sqrt{2}$}
        \wrongchoice{$1$}
        \wrongchoice{$\dfrac{4}{3} \sqrt{2}$}
    \end{choices}
    \end{multicols}
\end{question}
}

\element{calculusBC}{
\begin{question}{1973-BC-q35}
    The region in the first quadrant bounded by the graph of $y=\sec x$, $x=\pi/4$,
        and the axes is rotate about the $x$-axis.
    what is the volume of the solid generated?
    \begin{multicols}{3}
    \begin{choices}
        \wrongchoice{$\dfrac{\pi^2}{4}$}
        \wrongchoice{$\pi -1$}
      \correctchoice{$\pi$}
        \wrongchoice{$2\pi$}
        \wrongchoice{$\dfrac{8\pi}{3}$}
    \end{choices}
    \end{multicols}
\end{question}
}

\element{calculusBC}{
\begin{question}{1973-BC-q36}
    $\displaystyle \int^{\;\;1}_{0} \frac{x+1}{x^2+2x-3}\,\mathrm{d}x$ is:
    \begin{multicols}{2}
    \begin{choices}
        \wrongchoice{$-\ln\sqrt{3}$}
        \wrongchoice{$-\dfrac{\ln\sqrt{3}}{2}$}
        \wrongchoice{$-\dfrac{1-\ln\sqrt{3}}{2}$}
        \wrongchoice{$\ln\sqrt{3}$}
      \correctchoice{divergent}
    \end{choices}
    \end{multicols}
\end{question}
}

\element{calculusBC}{
\begin{question}{1973-BC-q37}
    $\displaystyle \lim_{x\to 0} \frac{1-\cos^2\left(2x\right)}{x^2} = $
    \begin{multicols}{3}
    \begin{choices}
        \wrongchoice{$-2$}
        \wrongchoice{$0$}
        \wrongchoice{$1$}
        \wrongchoice{$2$}
      \correctchoice{$4$}
    \end{choices}
    \end{multicols}
\end{question}
}

\element{calculusBC}{
\begin{question}{1973-BC-q38}
    If $\int^{\,\,2}_{1}f\left(x-c\right)\,\mathrm{d}x = 5$ where $c$ is a constant,
        then $\int^{\,\, 2-c}_{1-c}f\left(x\right)\,\mathrm{d}x = $
    \begin{multicols}{3}
    \begin{choices}
        \wrongchoice{$5+c$}
      \correctchoice{$5$}
        \wrongchoice{$5-c$}
        \wrongchoice{$c-5$}
        \wrongchoice{$-5$}
    \end{choices}
    \end{multicols}
\end{question}
}

\element{calculusBC}{
\begin{question}{1973-BC-q39}
    Let $f$ and $g$ be differentiable functions such that:
    \begin{align*}
        f\left(1\right)&=2,     & f\prime\left(1\right)&=3,  & f\prime\left(2\right)&=-4, \\
        g\left(1\right)&=2,     & g\prime\left(1\right)&=-3, & g\prime\left(2\right)&=5, \\
    \end{align*}
    If $h(x) = f\left(g(x)\right)$, then $h\prime (1) = $
    \begin{multicols}{3}
    \begin{choices}
        \wrongchoice{$-9$}
        \wrongchoice{$-4$}
        \wrongchoice{$0$}
      \correctchoice{$12$}
        \wrongchoice{$15$}
    \end{choices}
    \end{multicols}
\end{question}
}

\element{calculusBC}{
\begin{question}{1973-BC-q40}
    The area of the region enclosed by the polar curve $r=1-\cos\theta$ is:
    \begin{multicols}{3}
    \begin{choices}
        \wrongchoice{$\dfrac{3\pi}{4}$}
        \wrongchoice{$\pi$}
      \correctchoice{$\dfrac{3\pi}{2}$}
        \wrongchoice{$2\pi$}
        \wrongchoice{$3\pi$}
    \end{choices}
    \end{multicols}
\end{question}
}

\element{calculusBC}{
\begin{question}{1973-BC-q41}
    Given
    \begin{math}
        f(x) =
        \begin{cases}
            x+1         & \text{for } x<0, \\
            \cos\pi x   & \text{for } x\geq 0,
        \end{cases}
        \displaystyle 
        \int^{\;\;1}_{-1} f(x)\,\mathrm{d}x = 
    \end{math}
    \begin{multicols}{3}
    \begin{choices}
        \wrongchoice{$\dfrac{1}{2} + \dfrac{1}{\pi}$}
        \wrongchoice{$-\dfrac{1}{2}$}
        \wrongchoice{$\dfrac{1}{2} - \dfrac{1}{\pi}$}
      \correctchoice{$\dfrac{1}{2}$}
        \wrongchoice{$-\dfrac{1}{2} + \dfrac{1}{\pi}$}
    \end{choices}
    \end{multicols}
\end{question}
}

\element{calculusBC}{
\begin{question}{1973-BC-q42}
    \begin{center}
    \begin{tikzpicture}
        %% NOTE: TOOD: pgfplots
    \end{tikzpicture}
    \end{center}
    Calculate the approximate area of the shaded region in the figure by the trapezoidal rule,
        using divisions at $x=4/3$ and $x=5/3$.
    \begin{multicols}{3}
    \begin{choices}
        \wrongchoice{$\dfrac{50}{72}$}
        \wrongchoice{$\dfrac{251}{108}$}
        \wrongchoice{$\dfrac{7}{3}$}
      \correctchoice{$\dfrac{127}{54}$}
        \wrongchoice{$\dfrac{77}{27}$}
    \end{choices}
    \end{multicols}
\end{question}
}

\element{calculusBC}{
\begin{question}{1973-BC-q43}
    $\displaystyle \int \, \arcsin x\mathrm{d}x =$
    \begin{choices}
        \wrongchoice{$\displaystyle \sin x = \int \dfrac{x}{\sqrt{1-x^2}} \mathrm{d} x$}
        \wrongchoice{$\displaystyle \frac{\left(\arcsin x\right)^2}{2} + C$}
        \wrongchoice{$\displaystyle \arcsin x + \int \dfrac{1}{\sqrt(1-x^2}\mathrm{d} x$}
        \wrongchoice{$\displaystyle x\arcsin x - \int \dfrac{x}{\sqrt(1-x^2}\mathrm{d} x$}
      \correctchoice{$\displaystyle x\arcsin x - \int \dfrac{x}{\sqrt(1-x^2}\mathrm{d} x$}
    \end{choices}
\end{question}
}

\element{calculusBC}{
\begin{question}{1973-BC-q44}
    If $f$ is the solution of $xf\prime(x) - f(x) = x$ such that $f(-1) = 1$,
        then  $f\left(\mathrm{e}^{-1}\right) = $
    \begin{multicols}{3}
    \begin{choices}
      \correctchoice{$-2\mathrm{e}^{-1}$}
        \wrongchoice{zero}
        \wrongchoice{$\mathrm{e}^{-1}$}
        \wrongchoice{$-\mathrm{e}^{-1}$}
        \wrongchoice{$2\mathrm{e}^{-2}$}
    \end{choices}
    \end{multicols}
\end{question}
}

\element{calculusBC}{
\begin{question}{1973-BC-q45}
    Suppose $g\prime\left(x\right)<0$ for all $x\geq 0$ and $F\left(x\right) = \int^{\,\,x}_{0} tg\prime(t)\,\mathrm{d}t$ for all $x\geq 0$.
    Which of the following statements is \emph{false}?
    \begin{choices}
        \wrongchoice{$F$ takes on negative values}
        \wrongchoice{$F$ is continuous for all $x>0$}
        \wrongchoice{$F\left(x\right) = x g\left(x\right) - \int^{\,\,x}_{0}g(t)\,\mathrm{d}t$}
        \wrongchoice{$F\prime\left(x\right)$ exists for all $x>0$}
      \correctchoice{$F$ is an increasing function}
    \end{choices}
\end{question}
}


%% 1985 AP Calculus BC: Section I (pp. 53)
%%--------------------------------------------------
\element{calculusBC}{
\begin{question}{1985-BC-q01}
    The area of the region between the graph of $y=4x^3 + 2$ and the $x$-axis from $x=1$ to $x=2$ is:
    \begin{multicols}{3}
    \begin{choices}
        \wrongchoice{$36$}
        \wrongchoice{$23$}
        \wrongchoice{$20$}
      \correctchoice{$17$}
        \wrongchoice{$9$}
    \end{choices}
    \end{multicols}
\end{question}
}

\element{calculusBC}{
\begin{question}{1985-BC-q02}
    At what values of $x$ does $f(x) = 3x^5 - 5x^3 + 15$ have a relative maximum?
    \begin{multicols}{2}
    \begin{choices}
      \correctchoice{$-1$ only}
        \wrongchoice{$0$ only}
        \wrongchoice{$1$ only}
        \wrongchoice{$-1$ and $1$ only}
        \wrongchoice{$-1$, $0$, $1$ only}
    \end{choices}
    \end{multicols}
\end{question}
}

\element{calculusBC}{
\begin{question}{1985-BC-q03}
    $\displaystyle \int^{\;\;2}_{1} \dfrac{x+1}{x^2+2x}\,\mathrm{d}x = $
    \begin{multicols}{2}
    \begin{choices}
        \wrongchoice{$\ln 8 - \ln 3$}
      \correctchoice{$\dfrac{\ln 8 - \ln 3}{2}$}
        \wrongchoice{$\ln 8$}
        \wrongchoice{$\dfrac{3 \ln 2}{2}$}
        \wrongchoice{$\dfrac{3 \ln 2 + 2}{2}$}
    \end{choices}
    \end{multicols}
\end{question}
}

\element{calculusBC}{
\begin{question}{1985-BC-q04}
    A particle moves in the $xy$-plane so that at any time $t$ its coordinates are $x=t^2-1$ and $y=t^4-2t^3$.
    At $t=1$, its acceleration vector is:
    \begin{multicols}{3}
    \begin{choices}
        \wrongchoice{$\left(0,1\right)$}
        \wrongchoice{$\left(0,12\right)$}
        \wrongchoice{$\left(2,-2\right)$}
      \correctchoice{$\left(2,0\right)$}
        \wrongchoice{$\left(2,8\right)$}
    \end{choices}
    \end{multicols}
\end{question}
}

\element{calculusBC}{
\begin{question}{1985-BC-q05}
    The curves $y=f(x)$ and $y=g(x)$ shown in the figure below intersect at the point $(a,b)$.
    \begin{center}
    \begin{tikzpicture}
        %% NOTE:
    \end{tikzpicture}
    \end{center}
    The area of the shaded region enclosed by these curves and the line $x=-1$ is given by:
    \begin{choices}
        \wrongchoice{$\displaystyle \int^{\;\;a}_{0} \left(f(x)-g(x)\right)\,\mathrm{d}x + \int^{\;\;0}_{-1} \left(f(x)+g(x)\right)\,\mathrm{d}x$}
        \wrongchoice{$\displaystyle \int^{\;\;h}_{-1} -g(x)\,\mathrm{d}x + \int^{\;\;c}_{h} f(x)\,\mathrm{d}x$}
        \wrongchoice{$\displaystyle \int^{\;\;c}_{-1} \left(f(x)-g(x)\right)\,\mathrm{d}x$}
      \correctchoice{$\displaystyle \int^{\;\;a}_{-1} \left(f(x)-g(x)\right)\,\mathrm{d}x$}
        \wrongchoice{$\displaystyle \int^{\;\;a}_{-1} \left(|f(x)| - |g(x)|\right)\,\mathrm{d}x$}
    \end{choices}
\end{question}
}

\element{calculusBC}{
\begin{question}{1985-BC-q06}
    If $f(x) = \dfrac{x}{\tan x}$, then $f\prime\left(\dfrac{\pi}{4}\right) = $
    \begin{multicols}{3}
    \begin{choices}
        \wrongchoice{$2$}
        \wrongchoice{$\dfrac{1}{2}$}
        \wrongchoice{$1+\dfrac{\pi}{2}$}
        \wrongchoice{$\dfrac{\pi}{2}-1$}
      \correctchoice{$1-\dfrac{\pi}{2}$}
    \end{choices}
    \end{multicols}
\end{question}
}

\element{calculusBC}{
\begin{question}{1985-BC-q07}
    Which of the following is equal to $\int \dfrac{1}{\sqrt{25-x^2}}\,\mathrm{d}x$?
    \begin{multicols}{2}
    \begin{choices}
      \correctchoice{$\mathrm{arcsin}\,\dfrac{x}{2} + C$}
        \wrongchoice{$\mathrm{arcsin}\,x + C$}
        \wrongchoice{$\dfrac{1}{5}\mathrm{arcsin}\,\dfrac{x}{5} + C$}
        \wrongchoice{$\sqrt{25 - x^2} + C$}
        \wrongchoice{$2\sqrt{25 - x^2} + C$}
    \end{choices}
    \end{multicols}
\end{question}
}

\element{calculusBC}{
\begin{question}{1985-BC-q08}
    If $f$ is a function such that $\displaystyle \lim_{x\to 2} \frac{f(x)-f(2)}{x-2}=0$,
        which of the following must be true?
    \begin{choices}
        \wrongchoice{The limit of $f(x)$ as $x$ approaches $2$ does not exist.}
        \wrongchoice{$f$ is not defined at $x=2$.}
      \correctchoice{The derivative of $f$ at $x=2$ is $0$.}
        \wrongchoice{$f$ is continuous at $x=0$.}
        \wrongchoice{$f(2) = 0$.}
    \end{choices}
\end{question}
}

\element{calculusBC}{
\begin{question}{1985-BC-q09}
    If $xy^2 + 2xy = 8$, then, at the point $(1,2)$, $y\prime$ is:
    \begin{multicols}{3}
    \begin{choices}
        \wrongchoice{$-\dfrac{5}{2}$}
      \correctchoice{$-\dfrac{4}{3}$}
        \wrongchoice{$-1$}
        \wrongchoice{$-\dfrac{1}{2}$}
        \wrongchoice{$0$}
    \end{choices}
    \end{multicols}
\end{question}
}

\element{calculusBC}{
\begin{question}{1985-BC-q10}
    For $-1<x<1$ if $\displaystyle f(x) = \sum^{\infty}_{n=1} \dfrac{\left(-1\right)^{n+1} x^{2n-1}}{2n-1}$, then $f\prime = $
    \begin{multicols}{2}
    \begin{choices}
      \correctchoice{$\displaystyle \sum^{\infty}_{n=1} \left(-1\right)^{n+1} x^{2n-2}$}
        \wrongchoice{$\displaystyle \sum^{\infty}_{n=1} \left(-1\right)^{n} x^{2n-2}$}
        \wrongchoice{$\displaystyle \sum^{\infty}_{n=1} \left(-1\right)^{2n} x^{2n}$}
        \wrongchoice{$\displaystyle \sum^{\infty}_{n=1} \left(-1\right)^{n} x^{2n}$}
        \wrongchoice{$\displaystyle \sum^{\infty}_{n=1} \left(-1\right)^{n+1} x^{2n}$}
    \end{choices}
    \end{multicols}
\end{question}
}

\element{calculusBC}{
\begin{question}{1985-BC-q11}
    $\displaystyle \frac{\mathrm{d}}{\mathrm{d}x} \ln\left(\frac{1}{1-x}\right) = $
    \begin{multicols}{2}
    \begin{choices}
      \correctchoice{$\dfrac{1}{1-x}$}
        \wrongchoice{$\dfrac{1}{x-1}$}
        \wrongchoice{$1-x$}
        \wrongchoice{$x-1$}
        \wrongchoice{$\left(1-x\right)^2$}
    \end{choices}
    \end{multicols}
\end{question}
}

\element{calculusBC}{
\begin{question}{1985-BC-q12}
    $\displaystyle \int\, \dfrac{\mathrm{d}x}{\left(x-1\right)\left(x+2\right)} = $
    \begin{choices}
      \correctchoice{$\dfrac{1}{3} \ln\left|\dfrac{x-1}{x+2}\right| + C$}
        \wrongchoice{$\dfrac{1}{3} \ln\left|\dfrac{x+2}{x-1}\right| + C$}
        \wrongchoice{$\dfrac{1}{3} \ln\left|\left(x-1\right)\left(x+2\right)\right| + C$}
        \wrongchoice{$\left(\ln\left|x-1\right|\right)\left(\ln\left|x+1\right|\right) + C$}
        \wrongchoice{$\ln\left|\left(x-1\right)\left(x+2\right)^2\right| + C$}
    \end{choices}
\end{question}
}

\element{calculusBC}{
\begin{question}{1985-BC-q13}
    Let $f$ be the function given by $f(x) = x^3 - 3x^2$.
    What are all the value of $c$ that satisfy the conclusion of the Mean Value Theorem of differential calculus on the closed interval $\left[0,3\right]$?
    \begin{multicols}{2}
    \begin{choices}
        \wrongchoice{$0$ only}
      \correctchoice{$2$ only}
        \wrongchoice{$3$ only}
        \wrongchoice{$0$ and $3$}
        \wrongchoice{$2$ and $3$}
    \end{choices}
    \end{multicols}
\end{question}
}

\element{calculusBC}{
\begin{questionmult}{1985-BC-q14}
    Which of the following series are convergent?
    \begin{choices}
        %% NOTE: ANS is C
        \wrongchoice{$1 + \dfrac{2}{2^2} + \dfrac{1}{3^2} + \cdots + \dfrac{1}{n^2} + \cdots$}
        \wrongchoice{$1 + \dfrac{1}{2} + \dfrac{1}{3} + \cdots + \dfrac{1}{n} + \cdots$}
        \wrongchoice{$1 - \dfrac{1}{3} + \dfrac{1}{3^2} - \cdots + \dfrac{(-1)^{n+1}}{3^{n-1}} + \cdots$}
    \end{choices}
\end{questionmult}
}

\element{calculusBC}{
\begin{question}{1985-BC-q15}
    If the velocity of a particle moving along the $x$-axis is $v(t) = 2t-4$ and if at $t=0$ its position is $4$,
        then at any time $t$ its position $x(t)$ is 
    \begin{multicols}{2}
    \begin{choices}
        \wrongchoice{$t^2 - 4t$}
        \wrongchoice{$t^2 - 4t -4$}
      \correctchoice{$t^2 - 4t +4$}
        \wrongchoice{$2t^2 - 4t$}
        \wrongchoice{$2t^2 - 4t +4$}
    \end{choices}
    \end{multicols}
\end{question}
}

\element{calculusBC}{
\begin{question}{1985-BC-q16}
    Which of the following functions shows that the statement ``If a function is continuous at $x=0$, then it is differentiable at $x=0$'' is false?
    \begin{multicols}{2}
    \begin{choices}
        \wrongchoice{$f(x) = x^{-\frac{4}{3}}$}
        \wrongchoice{$f(x) = x^{-\frac{1}{3}}$}
      \correctchoice{$f(x) = x^{\frac{1}{3}}$}
        \wrongchoice{$f(x) = x^{\frac{4}{3}}$}
        \wrongchoice{$f(x) = x^{3}$}
    \end{choices}
    \end{multicols}
\end{question}
}

\element{calculusBC}{
\begin{question}{1985-BC-q17}
    If $f(x) = x \ln\left(x^2\right)$, then $f\prime = $
    Which of the following functions shows that the statement ``If a function is continuous at $x=0$, then it is differentiable at $x=0$'' is false?
    \begin{multicols}{2}
    \begin{choices}
        \wrongchoice{$\ln\left(x^2\right) + 1$}
      \correctchoice{$\ln\left(x^2\right) + 2$}
        \wrongchoice{$\ln\left(x^2\right) + \dfrac{1}{x}$}
        \wrongchoice{$\dfrac{1}{x^2}$}
        \wrongchoice{$\dfrac{1}{x}$}
    \end{choices}
    \end{multicols}
\end{question}
}

\element{calculusBC}{
\begin{question}{1985-BC-q18}
    $\displaystyle\int\,\sin\left(2x+3\right)\,\mathrm{d}x = $
    \begin{multicols}{2}
    \begin{choices}
        \wrongchoice{$-2\cos\left(2x+3\right) + C$}
        \wrongchoice{$-\cos\left(2x+3\right) + C$}
      \correctchoice{$-\dfrac{1}{2}\cos\left(2x+3\right) + C$}
        \wrongchoice{$\dfrac{1}{2}\cos\left(2x+3\right) + C$}
        \wrongchoice{$\cos\left(2x+3\right) + C$}
    \end{choices}
    \end{multicols}
\end{question}
}

\element{calculusBC}{
\begin{question}{1985-BC-q19}
    If $f$ and $g$ are twice differentiable functions such that $g(x)=\mathrm{e}^{f(x)}$ and $g^{\dprime}(x) = h(x)\mathrm{e}^{f(x)}$, the $h(x) =$
    \begin{multicols}{2}
    \begin{choices}
        \wrongchoice{$f\prime(x) + f^{\dprime}(x)$}
        \wrongchoice{$f\prime(x) + \left(f^{\dprime}(x)\right)^2$}
        \wrongchoice{$\left(f\prime(x) + f^{\dprime}(x)\right)^2$}
      \correctchoice{$\left(f\prime(x)\right)^2 + f^{\dprime}(x)$}
        \wrongchoice{$2 f\prime(x) + f^{\dprime}(x)$}
    \end{choices}
    \end{multicols}
\end{question}
}

\element{calculusBC}{
\begin{question}{1985-BC-q20}
    The graph of $y=f(x)$ on the closed interval $\left[2,7\right]$ is shown below.
    \begin{center}
    \begin{tikzpicture}
        %% NOTE:
    \end{tikzpicture}
    \end{center}
    How many points of inflection does this graph have on this interval?
    \begin{multicols}{3}
    \begin{choices}
        \wrongchoice{One}
        \wrongchoice{Two}
      \correctchoice{Three}
        \wrongchoice{Four}
        \wrongchoice{Five}
    \end{choices}
    \end{multicols}
\end{question}
}

\element{calculusBC}{
\begin{question}{1985-BC-q21}
    If $\int\,f(x)\sin x\,\mathrm{d}x = -f(x)\cos x + \int\,3x^2\cos x\,\mathrm{d}x$,
        then $f(x)$ could be::
    \begin{multicols}{3}
    \begin{choices}
        \wrongchoice{$3x^2$}
      \correctchoice{$x^3$}
        \wrongchoice{$-x^3$}
        \wrongchoice{$\sin x$}
        \wrongchoice{$\cos x$}
    \end{choices}
    \end{multicols}
\end{question}
}

\element{calculusBC}{
\begin{question}{1985-BC-q22}
    The area of a circular region is increasing at a rate of $96\pi$ square meters per second.
    When the area of the region is $64\pi$ square meters, how fast, in meters per second,
        is the radius of the region increasing?
    \begin{multicols}{3}
    \begin{choices}
      \correctchoice{$6$}
        \wrongchoice{$8$}
        \wrongchoice{$16$}
        \wrongchoice{$4\sqrt{3}$}
        \wrongchoice{$12\sqrt{3}$}
    \end{choices}
    \end{multicols}
\end{question}
}

\element{calculusBC}{
\begin{question}{1985-BC-q23}
    $\displaystyle \lim_{h\to 0} \dfrac{1}{h} \int^{\;\;1+h}_{1} \sqrt{x^5 + 8}\,\mathrm{d}x $ is:
    \begin{multicols}{2}
    \begin{choices}
        \wrongchoice{$0$}
        \wrongchoice{$1$}
      \correctchoice{$3$}
        \wrongchoice{$2\sqrt{2}$}
        \wrongchoice{nonexistent}
    \end{choices}
    \end{multicols}
\end{question}
}

\element{calculusBC}{
\begin{question}{1985-BC-q24}
    The area of the region enclosed by the polar curve $r=\sin\left(2\theta\right)$ for $0<\theta\leq\dfrac{\pi}{2}$ is:
    \begin{multicols}{3}
    \begin{choices}
        \wrongchoice{$0$}
        \wrongchoice{$\dfrac{1}{2}$}
        \wrongchoice{$1$}
      \correctchoice{$\dfrac{\pi}{8}$}
        \wrongchoice{$\dfrac{\pi}{4}$}
    \end{choices}
    \end{multicols}
\end{question}
}

\element{calculusBC}{
\begin{question}{1985-BC-q25}
    A particle moves along the $x$-axis so that at any time $t$ its position is given by $x(t) = t\mathrm{e}^{-2t}$.
    For what values of $t$ is the particle at rest?
    \begin{multicols}{2}
    \begin{choices}
        \wrongchoice{No values}
        \wrongchoice{$0$ only}
      \correctchoice{$\dfrac{1}{2}$ only}
        \wrongchoice{$1$ only}
        \wrongchoice{$0$ and $\dfrac{1}{2}$}
    \end{choices}
    \end{multicols}
\end{question}
}

\element{calculusBC}{
\begin{question}{1985-BC-q26}
    For $0<x<\dfrac{\pi}{2}$, if $y=\left(\sin x\right)^x$, then $\dfrac{\mathrm{d}y}{\mathrm{d}x}$ is:
    \begin{choices}
        \wrongchoice{$x \ln\left(\sin x\right)$}
        \wrongchoice{$\left(\sin x\right)^x \cot x$}
        \wrongchoice{$x\left(\sin x\right)^{x-1} \left(\cos x\right)$}
        \wrongchoice{$\left(\sin x\right)^{x} \left(x\cos x + \sin x\right)$}
      \correctchoice{$\left(\sin x\right)^{x} \left(x\cot x + \ln\left(\sin x\right)\right)$}
    \end{choices}
\end{question}
}

\element{calculusBC}{
\begin{questionmult}{1985-BC-q27}
    If $f$ is the continuous, strictly increasing function on the interval $a\leq x\leq b$ as shown below,
    \begin{center}
    \begin{tikzpicture}
        %% NOTE:
    \end{tikzpicture}
    \end{center}
        which of the following must be true?
    \begin{choices}
        %% NOTE: ANS is E
        \wrongchoice{$\displaystyle \int^{\;\;b}_a f(x)\,\mathrm{d}x < f(b)\left(b-a\right)$}
        \wrongchoice{$\displaystyle \int^{\;\;b}_a f(x)\,\mathrm{d}x < f(a)\left(b-a\right)$}
        \wrongchoice{$\displaystyle \int^{\;\;b}_a f(x)\,\mathrm{d}x = f(c)\left(b-a\right)$ for some number $c$ such that $a<c<b$}
    \end{choices}
\end{questionmult}
}

\element{calculusBC}{
\begin{question}{1985-BC-q28}
    An antiderivative of $f(x) = \mathrm{e}^{x+\mathrm{e}^x}$ is:
    \begin{multicols}{2}
    \begin{choices}
        \wrongchoice{$\dfrac{\mathrm{e}^{x+\mathrm{e}^2}}{1+\mathrm{e}^x}$}
        \wrongchoice{$\left(1+\mathrm{e}^{x}\right) \mathrm{e}^{x+\mathrm{e}^x}$}
        \wrongchoice{$\mathrm{e}^{1+\mathrm{e}^x}$}
        \wrongchoice{$\mathrm{e}^{x+\mathrm{e}^x}$}
      \correctchoice{$\mathrm{e}^{\mathrm{e}^x}$}
    \end{choices}
    \end{multicols}
\end{question}
}

\element{calculusBC}{
\begin{question}{1985-BC-q29}
    $\displaystyle\lim_{x\to\frac{\pi}{4}} \frac{\sin\left(x-\frac{\pi}{4}\right)}{x-\frac{\pi}{4}}$ is:
    \begin{multicols}{2}
    \begin{choices}
        \wrongchoice{$0$}
        \wrongchoice{$\dfrac{1}{\sqrt{2}}$}
        \wrongchoice{$\dfrac{\pi}{4}$}
      \correctchoice{$1$}
        \wrongchoice{nonexistent}
    \end{choices}
    \end{multicols}
\end{question}
}

\element{calculusBC}{
\begin{question}{1985-BC-q30}
    If $x=t^3 - t$ and $y=\sqrt{3t+1}$, then $\dfrac{\mathrm{d}y}{\mathrm{d}x}$ at $t=1$ is:
    \begin{multicols}{3}
    \begin{choices}
        \wrongchoice{$\dfrac{1}{8}$}
      \correctchoice{$\dfrac{3}{8}$}
        \wrongchoice{$\dfrac{3}{4}$}
        \wrongchoice{$\dfrac{8}{3}$}
        \wrongchoice{$8$}
    \end{choices}
    \end{multicols}
\end{question}
}

\element{calculusBC}{
\begin{question}{1985-BC-q31}
    What are all values of $x$ for which the series $\displaystyle \sum^{\infty}_{n=1} \frac{\left(x-1\right)^n}{n}$ converges?
    \begin{multicols}{2}
    \begin{choices}
        \wrongchoice{$-1 \leq x < 1$}
        \wrongchoice{$-1 \leq x \leq 1$}
        \wrongchoice{$0 \leq x < 2$}
      \correctchoice{$0 \leq x < 2$}
        \wrongchoice{$0 \leq x \leq 2$}
    \end{choices}
    \end{multicols}
\end{question}
}

\element{calculusBC}{
\begin{question}{1985-BC-q32}
    An equation of the line normal to the graph of $y=x^3 + 3x^2 + 7x -1$ at the point where $x=-1$ is:
    \begin{multicols}{2}
    \begin{choices}
        \wrongchoice{$4x + y = 10$}
        \wrongchoice{$x - 4y = 23$}
        \wrongchoice{$4x - y = 2$}
        \wrongchoice{$x + 4y = 25$}
      \correctchoice{$x + 4y = -25$}
    \end{choices}
    \end{multicols}
\end{question}
}

\element{calculusBC}{
\begin{question}{1985-BC-q33}
    If $\dfrac{\mathrm{d}y}{\mathrm{d}t} = -2y$ and if $y=1$ when $t=0$,
        what is the value of $t$ for which $y=\frac{1}{2}$?
    \begin{multicols}{3}
    \begin{choices}
        \wrongchoice{$-\dfrac{\ln 2}{2}$}
        \wrongchoice{$-\dfrac{1}{4}$}
      \correctchoice{$\dfrac{\ln 2}{2}$}
        \wrongchoice{$\dfrac{\sqrt{2}}{2}$}
        \wrongchoice{$\ln 2$}
    \end{choices}
    \end{multicols}
\end{question}
}

\element{calculusBC}{
\begin{question}{1985-BC-q34}
    Which of the following gives the area of the surface generated by revolving about the $y$-axis the arc of $x=y^3$ from $y=0$ to $y=1$?
    \begin{choices}
      \correctchoice{$\displaystyle 2\pi \int^{\;\;1}_0 y^3\sqrt{1+9y^4}\,\mathrm{d}y$}
        \wrongchoice{$\displaystyle 2\pi \int^{\;\;1}_0 y^3\sqrt{1+y^6}\,\mathrm{d}y$}
        \wrongchoice{$\displaystyle 2\pi \int^{\;\;1}_0 y^3\sqrt{1+3y^2}\,\mathrm{d}y$}
        \wrongchoice{$\displaystyle 2\pi \int^{\;\;1}_0 y \sqrt{1+9y^4}\,\mathrm{d}y$}
        \wrongchoice{$\displaystyle 2\pi \int^{\;\;1}_0 y\sqrt{1+y^6}\,\mathrm{d}y$}
    \end{choices}
\end{question}
}

\element{calculusBC}{
\begin{question}{1985-BC-q35}
    The region in the first quadrant between the $x$-axis and the graph of $y=6x-x^2$ is rotated around the $y$-axis.
    The volume of the resulting solid of revolution is given by
    \begin{choices}
        \wrongchoice{$\displaystyle \int^{\;\;6}_0 \pi\left(6x-x^2\right)^2\,\mathrm{d}y$}
      \correctchoice{$\displaystyle \int^{\;\;6}_0 2\pi x\left(6x-x^2\right)^2\,\mathrm{d}y$}
        \wrongchoice{$\displaystyle \int^{\;\;6}_0 \pi x\left(6x-x^2\right)^2\,\mathrm{d}y$}
        \wrongchoice{$\displaystyle \int^{\;\;6}_0 \pi\left(3 + \sqrt{9-y}\right)^2\,\mathrm{d}y$}
        \wrongchoice{$\displaystyle \int^{\;\;9}_0 \pi\left(3 + \sqrt{9-y}\right)^2\,\mathrm{d}y$}
    \end{choices}
\end{question}
}

\element{calculusBC}{
\begin{question}{1985-BC-q36}
    $\displaystyle \int^{\;\;1}_{-1} \frac{3}{x^2}\,\mathrm{d}x$ is:
    \begin{multicols}{2}
    \begin{choices}
        \wrongchoice{$-6$}
        \wrongchoice{$-3$}
        \wrongchoice{$0$}
        \wrongchoice{$6$}
      \correctchoice{nonexistent}
    \end{choices}
    \end{multicols}
\end{question}
}

\element{calculusBC}{
\begin{question}{1985-BC-q37}
    The general solution for the equation $\dfrac{\mathrm{d}y}{\mathrm{d}x} + y = x \mathrm{e}^{-x}$ is:
    \begin{multicols}{2}
    \begin{choices}
      \correctchoice{$y = \dfrac{x^2}{2}\mathrm{e}^{-x} + C\mathrm{e}^{-x}$}
        \wrongchoice{$y = \dfrac{x^2}{2}\mathrm{e}^{-x} + \mathrm{e}^{-x} + C$}
        \wrongchoice{$y = \mathrm{e}^{-x} + \dfrac{C}{1+x}$}
        \wrongchoice{$y = x\mathrm{e}^{-x} + C\mathrm{e}^{-x}$}
        \wrongchoice{$y = C_1\mathrm{e}^{-x} + C_2\mathrm{e}^{-x}$}
    \end{choices}
    \end{multicols}
\end{question}
}

\element{calculusBC}{
\begin{question}{1985-BC-q38}
    $\displaystyle \lim_{x\to\infty} \left(1+5\mathrm{e}^x\right)^{\frac{1}{x}}$ is:
    \begin{multicols}{2}
    \begin{choices}
        \wrongchoice{$0$}
        \wrongchoice{$1$}
      \correctchoice{$\mathrm{e}$}
        \wrongchoice{$\mathrm{e}^5$}
        \wrongchoice{nonexistent}
    \end{choices}
    \end{multicols}
\end{question}
}

\element{calculusBC}{
\begin{question}{1985-BC-q39}
    The base of a solid is the region enclosed by the graph of $y=\mathrm{e}^{-x}$,
        the coordinate axes, and the line $x=3$.
    If all plane cross sections perpendicular to the $x$-axis are squares,  
        then its volume is:
    \begin{multicols}{3}
    \begin{choices}
      \correctchoice{$\dfrac{\left(1-\mathrm{e}^{-6}\right)}{2}$}
        \wrongchoice{$\dfrac{1}{2}\mathrm{e}^{-6}$}
        \wrongchoice{$\mathrm{e}^{-6}$}
        \wrongchoice{$\mathrm{e}^{-3}$}
        \wrongchoice{$1-\mathrm{e}^{-3}$}
    \end{choices}
    \end{multicols}
\end{question}
}

\element{calculusBC}{
\begin{question}{1985-BC-q40}
    If the substitution $u=\frac{x}{2}$ is made, the integral
    %\begin{equation*}
    $\displaystyle \int^{\;\;4}_{2} \frac{1-\left(\frac{x}{2}\right)^2}{x}\,\mathrm{d}x = $
    %\end{equation*}
    \begin{multicols}{2}
    \begin{choices}
      \correctchoice{$\displaystyle \int^{\;\;2}_1 \frac{1-u^2}{u}\,\mathrm{d}x$}
        \wrongchoice{$\displaystyle \int^{\;\;4}_2 \frac{1-u^2}{u}\,\mathrm{d}x$}
        \wrongchoice{$\displaystyle \int^{\;\;2}_1 \frac{1-u^2}{2u}\,\mathrm{d}x$}
        \wrongchoice{$\displaystyle \int^{\;\;2}_1 \frac{1-u^2}{4u}\,\mathrm{d}x$}
        \wrongchoice{$\displaystyle \int^{\;\;4}_2 \frac{1-u^2}{2u}\,\mathrm{d}x$}
    \end{choices}
    \end{multicols}
\end{question}
}

\element{calculusBC}{
\begin{question}{1985-BC-q41}
    What is the length of the arc of $y=\frac{2}{3}x^{\frac{3}{2}}$ from $x=0$ to $x=3$?
    \begin{multicols}{3}
    \begin{choices}
        \wrongchoice{$\dfrac{8}{3}$}
        \wrongchoice{$4$}
      \correctchoice{$\dfrac{14}{3}$}
        \wrongchoice{$\dfrac{16}{3}$}
        \wrongchoice{$7$}
    \end{choices}
    \end{multicols}
\end{question}
}

\element{calculusBC}{
\begin{question}{1985-BC-q42}
    The coefficient of $x^3$ in the Taylor series for $\mathrm{e}^{3x}$ about $x=0$ is:
    \begin{multicols}{3}
    \begin{choices}
        \wrongchoice{$\dfrac{1}{6}$}
        \wrongchoice{$\dfrac{1}{3}$}
        \wrongchoice{$\dfrac{1}{2}$}
        \wrongchoice{$\dfrac{3}{2}$}
      \correctchoice{$\dfrac{9}{2}$}
    \end{choices}
    \end{multicols}
\end{question}
}

\element{calculusBC}{
\begin{question}{1985-BC-q43}
    Let $f$ by a function that is continuous on the closed interval $[-2,3]$ such that $f\prime(0)$ does not exist,
        $f\prime{2}=0$, and $f^{\dprime}(x)<0$ for all $x$ except $x=0$.
    which of the following could be the graph of $f$?
    \begin{multicols}{3}
    \begin{choices}
        %% NOTE: ANS is E
        \wrongchoice{
            \begin{tikzpicture}
            \end{tikzpicture}
        }
    \end{choices}
    \end{multicols}
\end{question}
}

\element{calculusBC}{
\begin{question}{1985-BC-q44}
    At each point $(x,y)$ on a certain curve, the slope of the curve is $3x^3y$.
    If the curve contains the point $(0,8)$, then its equation is:
    \begin{multicols}{2}
    \begin{choices}
      \correctchoice{$y=8\mathrm{e}^{x^3}$}
        \wrongchoice{$y=x^3 + 8$}
        \wrongchoice{$y=8\mathrm{e}^{x^3} +  7$}
        \wrongchoice{$y=\ln\left(x+1\right) + 8$}
        \wrongchoice{$y^2=x^3 + 8$}
    \end{choices}
    \end{multicols}
\end{question}
}

\element{calculusBC}{
\begin{question}{1985-BC-q45}
    If $n$ is a positive integer, the
    \begin{equation*}
        \lim_{n\to\infty} \dfrac{1}{n}\left[\left(\frac{1}{n}\right)^2 + \left(\dfrac{2}{n}\right)^2 + \ldots \left(\dfrac{3n}{n}\right)^2\right]
    \end{equation*}
    can be expressed as:
    \begin{multicols}{2}
    \begin{choices}
        \wrongchoice{$\displaystyle \int^{\;\;1}_0 \dfrac{1}{x^2}\,\mathrm{d}x$}
        \wrongchoice{$\displaystyle 3\int^{\;\;1}_0 \dfrac{1}{x^2}\,\mathrm{d}x$}
        \wrongchoice{$\displaystyle \int^{\;\;3}_0 \dfrac{1}{x^2}\,\mathrm{d}x$}
      \correctchoice{$\displaystyle \int^{\;\;3}_0 x^2\,\mathrm{d}x$}
        \wrongchoice{$\displaystyle 3\int^{\;\;3}_0 x^2\,\mathrm{d}x$}
    \end{choices}
    \end{multicols}
\end{question}
}


%% 1988 AP Calculus BC: Section I
%%--------------------------------------------------
\element{calculusBC}{
\begin{question}{1988-BC-q01}
    The area of the region in the first quadrant enclosed by the graph of $y=x\left(1-x\right)$ and the $x$-axis is:
    \begin{multicols}{3}
    \begin{choices}
      \correctchoice{$\dfrac{1}{6}$}
        \wrongchoice{$\dfrac{1}{3}$}
        \wrongchoice{$\dfrac{2}{3}$}
        \wrongchoice{$\dfrac{5}{6}$}
        \wrongchoice{$1$}
    \end{choices}
    \end{multicols}
\end{question}
}

\element{calculusBC}{
\begin{question}{1988-BC-q02}
    $\displaystyle \int^{\;\;1}_0 x\left(x^2+2\right)^2\,\mathrm{d}x = $
    \begin{multicols}{3}
    \begin{choices}
        \wrongchoice{$\dfrac{19}{2}$}
        \wrongchoice{$\dfrac{19}{3}$}
        \wrongchoice{$\dfrac{9}{2}$}
      \correctchoice{$\dfrac{19}{6}$}
        \wrongchoice{$\dfrac{1}{6}$}
    \end{choices}
    \end{multicols}
\end{question}
}

\element{calculusBC}{
\begin{question}{1988-BC-q03}
    If $f(x)=\ln\left( \right)$, then $f^{\dprime}(x) = $
    \begin{multicols}{3}
    \begin{choices}
        \wrongchoice{$-\dfrac{2}{x^2}$}
      \correctchoice{$-\dfrac{2}{2x^2}$}
        \wrongchoice{$-\dfrac{2}{2x}$}
        \wrongchoice{$-\dfrac{2}{2x^{3/2}}$}
        \wrongchoice{$\dfrac{2}{x^2}$}
    \end{choices}
    \end{multicols}
\end{question}
}

\element{calculusBC}{
\begin{question}{1988-BC-q04}
    If $u$, $v$ and $w$ are nonzero differentiable functions,
        then the derivative of $\dfrac{uv}{w}$ is:
    \begin{multicols}{2}
    \begin{choices}
        \wrongchoice{$\dfrac{uv\prime{}+u\prime{}v}{w\prime{}}$}
        \wrongchoice{$\dfrac{u\prime{}v\prime{}w-uvw\prime{}}{w^2}$}
        \wrongchoice{$\dfrac{uvw\prime{}-uv\prime{}w-u\prime{}vw}{w^2}$}
        \wrongchoice{$\dfrac{u\prime{}vw+uv\prime{}w+uvw\prime{}}{w^2}$}
      \correctchoice{$\dfrac{uv\prime{}w+u\prime{}vw-uvw\prime{}}{w^2}$}
    \end{choices}
    \end{multicols}
\end{question}
}

\element{calculusBC}{
\begin{question}{1988-BC-q05}
    Let $f$ be the function defined by the following.
    \begin{equation*}
        f(x) =
        \begin{cases}
            \sin x, & x<0 \\
            x^2,    & 0\leq x<1 \\
            2-x,    & 1\leq x<2 \\
            x-3,    & x\geq 2 \\
        \end{cases}
    \end{equation*}
    For what values of $x$ is $f$ \emph{not} continuous?
    \begin{multicols}{2}
    \begin{choices}
        \wrongchoice{0 only}
        \wrongchoice{1 only}
      \correctchoice{2 only}
        \wrongchoice{0 and 2 only}
        \wrongchoice{0, 1 and 2}
    \end{choices}
    \end{multicols}
\end{question}
}

\element{calculusBC}{
\begin{question}{1988-BC-q06}
    If $y^2-2xy=16$, then $\dfrac{\mathrm{d}y}{\mathrm{d}x} = $
    \begin{multicols}{2}
    \begin{choices}
        \wrongchoice{$\dfrac{x}{y-x}$}
        \wrongchoice{$\dfrac{y}{x-y}$}
      \correctchoice{$\dfrac{y}{y-x}$}
        \wrongchoice{$\dfrac{y}{2y-x}$}
        \wrongchoice{$\dfrac{2y}{x-y}$}
    \end{choices}
    \end{multicols}
\end{question}
}

\element{calculusBC}{
\begin{question}{1988-BC-q07}
    $\displaystyle \int^{\;\;+\infty}_2 \dfrac{\mathrm{d}x}{x^2}$ is:
    \begin{multicols}{2}
    \begin{choices}
      \correctchoice{$\dfrac{1}{2}$}
        \wrongchoice{$\ln 2$}
        \wrongchoice{$1$}
        \wrongchoice{$2$}
        \wrongchoice{nonexistent}
    \end{choices}
    \end{multicols}
\end{question}
}

\element{calculusBC}{
\begin{question}{1988-BC-q08}
    If $f(x) = \mathrm{e}^2$, then $\ln\left(f\prime(2)\right) = $
    \begin{multicols}{3}
    \begin{choices}
      \correctchoice{$2$}
        \wrongchoice{$0$}
        \wrongchoice{$\dfrac{1}{\mathrm{e}^2}$}
        \wrongchoice{$2\mathrm{e}$}
        \wrongchoice{$\mathrm{e}^2$}
    \end{choices}
    \end{multicols}
\end{question}
}

\element{calculusBC}{
\begin{question}{1988-BC-q09}
    Which pair of the following pairs of graphs could represent the graph of a function and the graph of its derivative?
    \begin{center}
    \begin{tikzpicture}
        %% NOTE: 3x2 graphs
    \end{tikzpicture}
    \end{center}
    \begin{multicols}{2}
    \begin{choices}
        \wrongchoice{I only}
        \wrongchoice{II only}
        \wrongchoice{III only}
      \correctchoice{I and III}
        \wrongchoice{II and III}
    \end{choices}
    \end{multicols}
\end{question}
}

\element{calculusBC}{
\begin{question}{1988-BC-q10}
    $\displaystyle \lim_{h\to 0} \dfrac{\sin\left(x+h\right) - \sin x}{h}$ is:
    \begin{multicols}{2}
    \begin{choices}
        \wrongchoice{$0$}
        \wrongchoice{$1$}
        \wrongchoice{$\sin x$}
      \correctchoice{$\cos x$}
        \wrongchoice{nonexistent}
    \end{choices}
    \end{multicols}
\end{question}
}

\element{calculusBC}{
\begin{question}{1988-BC-q11}
    If $x+7y=29$ is an equation of the line normal to the graph of $f$ at the point $(1,4)$, then $f\prime(1)=$
    \begin{multicols}{3}
    \begin{choices}
      \correctchoice{$7$}
        \wrongchoice{$\dfrac{1}{7}$}
        \wrongchoice{$-\dfrac{1}{7}$}
        \wrongchoice{$-\dfrac{7}{29}$}
        \wrongchoice{$-7$}
    \end{choices}
    \end{multicols}
\end{question}
}

\element{calculusBC}{
\begin{question}{1988-BC-q12}
    A particle travels in a straight line with a constant acceleration of 3 meters per second per second.
    If the velocity of the particle is 10 meters per second at time 2 seconds,
        how far does the particle travel during the time interval when its velocity increases from 4 meters per second to 10 meters per second?
    \begin{multicols}{3}
    \begin{choices}
        \wrongchoice{\SI{20}{\meter}}
      \correctchoice{\SI{14}{\meter}}
        \wrongchoice{\SI{7}{\meter}}
        \wrongchoice{\SI{6}{\meter}}
        \wrongchoice{\SI{3}{\meter}}
    \end{choices}
    \end{multicols}
\end{question}
}

\element{calculusBC}{
\begin{question}{1988-BC-q13}
    $\sin\left(2x\right) = $
    \begin{choices}
        \wrongchoice{$x -\dfrac{x^3}{3!} + \dfrac{x^5}{5!} - \ldots + \dfrac{\left(-1\right)^{n-1}x^{2n-1}}{\left(2n-1\right)!} + \ldots $}
      \correctchoice{$2x -\dfrac{\left(2x\right)^3}{3!} + \dfrac{\left(2x\right)^5}{5!} - \ldots + \dfrac{\left(-1\right)^{n-1}\left(2x\right)^{2n-1}}{\left(2n-1\right)!} + \ldots $}
        \wrongchoice{$-\dfrac{\left(2x\right)^2}{2!} +\dfrac{\left(2x\right)^4}{4!} - \ldots + \dfrac{\left(-1\right)^{n}\left(2x\right)^{2n}}{\left(2n\right)!} + \ldots $}
        \wrongchoice{$\dfrac{x^2}{2!} +\dfrac{x^4}{4!} +\dfrac{x^6}{6!} + \ldots + \dfrac{x^{2n}}{\left(2n\right)!} + \ldots $}
        \wrongchoice{$2x + \dfrac{\left(2x\right)^3}{3!} +\dfrac{\left(2x\right)^5}{5!} + \ldots + \dfrac{\left(2x\right)^{2n-1}}{\left(2n-1\right)!} + \ldots $}
    \end{choices}
\end{question}
}

\element{calculusBC}{
\begin{question}{1988-BC-q14}
    If $\displaystyle F(x) = \int^{\;\;x^2}_1 \sqrt{1+t^3}\,\mathrm{d}t$, then $F\prime\left(x\right) = $
    \begin{multicols}{2}
    \begin{choices}
      \correctchoice{$2x \sqrt{1+x^6}$}
        \wrongchoice{$2x \sqrt{1+x^3}$}
        \wrongchoice{$\sqrt{1+x^6}$}
        \wrongchoice{$\sqrt{1+x^3}$}
        \wrongchoice{$\displaystyle \int^{\;\;x^2}_1 \frac{3t^2}{2\sqrt{1+t^3}} \,\mathrm{d} t$}
    \end{choices}
    \end{multicols}
\end{question}
}

\element{calculusBC}{
\begin{question}{1988-BC-q15}
    For any time $t\geq 0$, if the position of a particle in the $xy$-plane is given by $x=t^2-1$ and $y=\ln\left(2t+3\right)$, then the acceleration vector is:
    \begin{multicols}{2}
    \begin{choices}
        \wrongchoice{$\left( 2t, \dfrac{2}{2t+3}\right)$}
        \wrongchoice{$\left( 2t, \dfrac{-4}{\left(2t+3\right)^2}\right)$}
        \wrongchoice{$\left( t, \dfrac{4}{\left(2t+3\right)^2}\right)$}
        \wrongchoice{$\left( 2, \dfrac{2}{\left(2t+3\right)^2}\right)$}
      \correctchoice{$\left( 2, \dfrac{-4}{\left(2t+3\right)^2}\right)$}
    \end{choices}
    \end{multicols}
\end{question}
}

\element{calculusBC}{
\begin{question}{1988-BC-q16}
    $\displaystyle\int\,x\mathrm{e}^{2x}\,\mathrm{d}x = $
    \begin{multicols}{2}
    \begin{choices}
      \correctchoice{$\dfrac{x\mathrm{e}^{2x}}{2} - \dfrac{\mathrm{e}^{2x}}{4} + C$}
        \wrongchoice{$\dfrac{x\mathrm{e}^{2x}}{2} - \dfrac{\mathrm{e}^{2x}}{2} + C$}
        \wrongchoice{$\dfrac{x\mathrm{e}^{2x}}{2} + \dfrac{\mathrm{e}^{2x}}{4} + C$}
        \wrongchoice{$\dfrac{x\mathrm{e}^{2x}}{2} + \dfrac{\mathrm{e}^{2x}}{2} + C$}
        \wrongchoice{$\dfrac{x^2\mathrm{e}^{2x}}{4} + C$}
    \end{choices}
    \end{multicols}
\end{question}
}

\element{calculusBC}{
\begin{question}{1988-BC-q17}
    $\displaystyle \int^{\;\;3}_{2} \frac{3}{\left(x-1\right)\left(x+2\right)}\,\mathrm{d}x = $
    \begin{multicols}{3}
    \begin{choices}
        \wrongchoice{$-\dfrac{33}{20}$}
        \wrongchoice{$-\dfrac{9}{20}$}
        \wrongchoice{$\ln\left(\dfrac{5}{2}\right)$}
      \correctchoice{$\ln\left(\dfrac{8}{5}\right)$}
        \wrongchoice{$\ln\left(\dfrac{2}{5}\right)$}
    \end{choices}
    \end{multicols}
\end{question}
}

\element{calculusBC}{
\begin{question}{1988-BC-q18}
    If three equal subdivisions of $\left[-4,2\right]$ are used,
        what is the trapezoidal approximation of
        $\displaystyle \int^{\;\;2}_{-4} \frac{\mathrm{e}^{-x}}{2} \mathrm{d} x \, $?
    \begin{choices}
        \wrongchoice{$\mathrm{e}^2 + \mathrm{e}^0 + \mathrm{e}^{-2}$}
        \wrongchoice{$\mathrm{e}^4 + \mathrm{e}^2 + \mathrm{e}^{0}$}
        \wrongchoice{$\mathrm{e}^4 + 2\mathrm{e}^2 + 2\mathrm{e}^{0} + \mathrm{e}^{-2}$}
        \wrongchoice{$\dfrac{1}{2}\left(\mathrm{e}^4 + \mathrm{e}^2 + \mathrm{e}^{0} + \mathrm{e}^{-2}\right)$}
      \correctchoice{$\dfrac{1}{2}\left(\mathrm{e}^4 + 2\mathrm{e}^2 + 2\mathrm{e}^{0} + \mathrm{e}^{-2}\right)$}
    \end{choices}
\end{question}
}

\element{calculusBC}{
\begin{question}{1988-BC-q19}
    A polynomial $p(x)$ has a relative maximum at $\left(-2,4\right)$,
        a relative minimum at $\left(1,1\right)$,
        a relative maximum at $\left(5,7\right)$ and no other critical points.
    How many zero does $p(x)$ have?
    \begin{multicols}{3}
    \begin{choices}
        \wrongchoice{one}
      \correctchoice{two}
        \wrongchoice{three}
        \wrongchoice{four}
        \wrongchoice{five}
    \end{choices}
    \end{multicols}
\end{question}
}

\element{calculusBC}{
\begin{question}{1988-BC-q20}
    The statement ``$\lim_{x\to a} f(x) = L$'' means that for each $\epsilon>0$,
        there exists a $\delta>0$ such that:
    \begin{choices}
        \wrongchoice{if $0<\left|x-a\right|<\epsilon$,      then $\left|f(x)-L\right|<\delta$}
        \wrongchoice{if $0<\left|f(x)-L\right|<\epsilon$,   then $\left|x-a\right|<\delta$}
        \wrongchoice{if $\left|f(x)-L\right|<\delta$,       then $0<\left|x-a\right|<\epsilon$}
        \wrongchoice{   $0<\left|x-a\right|<\delta$         and $\left|f(x)-L\right|<\epsilon$}
      \correctchoice{if $0<\left|x-a\right|<\delta$,        then $\left|f(x)-L\right|<\epsilon$}
    \end{choices}
\end{question}
}

\element{calculusBC}{
\begin{question}{1988-BC-q21}
    The average value of $\dfrac{1}{x}$ on the closed interval $\left[1,3\right]$ is:
    \begin{multicols}{3}
    \begin{choices}
        \wrongchoice{$\dfrac{1}{2}$}
        \wrongchoice{$\dfrac{2}{3}$}
        \wrongchoice{$\dfrac{\ln 2}{2}$}
      \correctchoice{$\dfrac{\ln 3}{2}$}
        \wrongchoice{$\ln 3$}
    \end{choices}
    \end{multicols}
\end{question}
}

\element{calculusBC}{
\begin{question}{1988-BC-q22}
    If $f(x) = \left(x^2+1\right)^x$, then $f\prime (x) = $
    \begin{choices}
        \wrongchoice{$x \left(x^2+1\right)^{x-1}$}
        \wrongchoice{$2x^2 \left(x^2+1\right)^{x-1}$}
        \wrongchoice{$x \ln\left(x^2+1\right)$}
        \wrongchoice{$\ln\left(x^2+1\right) + \dfrac{2x^2}{x^2+1}$}
      \correctchoice{$\left(x^2+1\right)^x \left[\ln\left(x^2+1\right) + \dfrac{2x^2}{x^2+1}\right]$}
    \end{choices}
\end{question}
}

\element{calculusBC}{
\begin{question}{1988-BC-q23}
    Which of the following gives the area of the region enclosed by the loop of the graph of the polar curve $r=4\cos\left(3\theta\right)$ shown in the figure below?
    \begin{center}
    \begin{tikzpicture}
        %% NOTE: TODO: draw graph
    \end{tikzpicture}
    \end{center}
    \begin{multicols}{2}
    \begin{choices}
        \wrongchoice{$\displaystyle 16 \int^{\;\;\pi/3}_{\pi/3} \cos\left(3\theta\right) \,\mathrm{d}\theta$}
        \wrongchoice{$\displaystyle 8 \int^{\;\;\pi/6}_{\pi/6} \cos\left(3\theta\right) \,\mathrm{d}\theta$}
        \wrongchoice{$\displaystyle 8 \int^{\;\;\pi/3}_{\pi/3} \cos^2\left(3\theta\right) \,\mathrm{d}\theta$}
        \wrongchoice{$\displaystyle 16 \int^{\;\;\pi/6}_{\pi/6} \cos^2\left(3\theta\right) \,\mathrm{d}\theta$}
      \correctchoice{$\displaystyle 8 \int^{\;\;\pi/6}_{\pi/6} \cos^2\left(3\theta\right) \,\mathrm{d}\theta$}
    \end{choices}
    \end{multicols}
\end{question}
}

\element{calculusBC}{
\begin{question}{1988-BC-q24}
    If $c$ is the number that satisfies the conclusion of the Mean Value Theorem for $f(x) = x^3-2x^2$ on the interval $x\leq x\leq 2$, then $c=$
    \begin{multicols}{3}
    \begin{choices}
        \wrongchoice{zero}
        \wrongchoice{$\dfrac{1}{2}$}
        \wrongchoice{one}
      \correctchoice{$\dfrac{4}{3}$}
        \wrongchoice{two}
    \end{choices}
    \end{multicols}
\end{question}
}

\element{calculusBC}{
\begin{question}{1988-BC-q25}
    The base of a solid is the region in the first quadrant enclosed by the parabola $y=4x^2$,
        the line $x=1$, and the $x$-axis.
    Each plane section of the solid perpendicular to the $x$-axis is a squared.
    The volume of the solid is:
    \begin{multicols}{3}
    \begin{choices}
        \wrongchoice{$\dfrac{4\pi}{3}$}
        \wrongchoice{$\dfrac{16\pi}{3}$}
        \wrongchoice{$\dfrac{4}{3}$}
      \correctchoice{$\dfrac{16}{5}$}
        \wrongchoice{$\dfrac{64}{4}$}
    \end{choices}
    \end{multicols}
\end{question}
}

\element{calculusBC}{
\begin{question}{1988-BC-q26}
    If $f$ is a function such that $f\prime\left(x\right)$ exists for all $x$ and $f\left(x\right)>0$ for all $x$,
        which of the following is \emph{not} necessarily true?
    \begin{choices}
        \wrongchoice{$\displaystyle \int^{\;\;1}_{1} f(x) \,\mathrm{d}x > 0$}
        \wrongchoice{$\displaystyle \int^{\;\;1}_{-1} 2f(x) \,\mathrm{d}x = 2 \int^{\;\;2}_{-1} f(x) \,\mathrm{d}x$}
      \correctchoice{$\displaystyle \int^{\;\;1}_{-1} f(x) \,\mathrm{d}x = 2 \int^{\;\;1}_{0} f(x) \,\mathrm{d}x$}
        \wrongchoice{$\displaystyle \int^{\;\;1}_{-1} f(x) \,\mathrm{d}x = -\int^{\;\;1}_{-1} f(x) \,\mathrm{d}x$}
        \wrongchoice{$\displaystyle \int^{\;\;1}_{-1} f(x) \,\mathrm{d}x = \int^{\;\;0}_{-1} f(x) \,\mathrm{d}x + \int^{\;\;1}_{0} f(x) \,\mathrm{d}x $}
    \end{choices}
\end{question}
}

\element{calculusBC}{
\begin{question}{1988-BC-q27}
    If the graphs of $y=x^3 + ax^2 + bx - 4$ has a point of inflection at $(1,-6)$, what is the value of $b$?
    \begin{choices}
        \wrongchoice{$-3$}
      \correctchoice{$0$}
        \wrongchoice{$1$}
        \wrongchoice{$3$}
        \wrongchoice{It cannot be determined from the information given}
    \end{choices}
\end{question}
}

\element{calculusBC}{
\begin{question}{1988-BC-q28}
    $\displaystyle \dfrac{\mathrm{d}}{\mathrm{d}x} \ln\left|\cos\left(\dfrac{\pi}{x}\right)\right|$ is:
    \begin{multicols}{2}
    \begin{choices}
        \wrongchoice{$\dfrac{-\pi}{x^2\cos\left(\dfrac{\pi}{x}\right)}$}
        \wrongchoice{$-\tan\left(\dfrac{\pi}{x}\right)$}
        \wrongchoice{$\dfrac{1}{\cos\left(\dfrac{\pi}{x}\right)}$}
        \wrongchoice{$\dfrac{\pi}{x}\tan\left(\dfrac{\pi}{x}\right)$}
      \correctchoice{$\dfrac{\pi}{x^2}\tan\left(\dfrac{\pi}{x}\right)$}
    \end{choices}
    \end{multicols}
\end{question}
}

\element{calculusBC}{
\begin{question}{1988-BC-q29}
    The region $R$ in the first quadrant is enclosed by the lines $x=0$ and $y=5$ and the graph of $y=x^2+1$.
    The volume of the solid generated when $R$ is revolved about the $y$-axis is:
    \begin{multicols}{3}
    \begin{choices}
        \wrongchoice{$6\pi$}
      \correctchoice{$8\pi$}
        \wrongchoice{$\dfrac{34\pi}{3}$}
        \wrongchoice{$16\pi$}
        \wrongchoice{$\dfrac{544\pi}{15}$}
    \end{choices}
    \end{multicols}
\end{question}
}

\element{calculusBC}{
\begin{question}{1988-BC-q30}
    $\displaystyle \sum_{i=n}^{\infty} \left(\frac{1}{3}\right)^i = $
    \begin{multicols}{2}
    \begin{choices}
        \wrongchoice{$\dfrac{3}{2} - \left(\dfrac{1}{3}\right)^n$}
        \wrongchoice{$\dfrac{3}{2}\left[1-\left(\dfrac{1}{3}\right)^n\right]$}
      \correctchoice{$\dfrac{3}{2}\left(\dfrac{1}{3}\right)^n$}
        \wrongchoice{$\dfrac{2}{3}\left(\dfrac{1}{3}\right)^n$}
        \wrongchoice{$\dfrac{2}{3}\left(\dfrac{1}{3}\right)^{n+1}$}
    \end{choices}
    \end{multicols}
\end{question}
}

\element{calculusBC}{
\begin{question}{1988-BC-q31}
    $\displaystyle \int^{\;\;2}_{0} \sqrt{4-x^2}\,\mathrm{d}x = $
    \begin{multicols}{3}
    \begin{choices}
        \wrongchoice{$\dfrac{8}{3}$}
        \wrongchoice{$\dfrac{16}{3}$}
      \correctchoice{$\pi$}
        \wrongchoice{$2\pi$}
        \wrongchoice{$4\pi$}
    \end{choices}
    \end{multicols}
\end{question}
}

\element{calculusBC}{
\begin{question}{1988-BC-q32}
    The general solution of the differential equation $y\prime = y + x^2$ is $y=$
    \begin{multicols}{2}
    \begin{choices}
        \wrongchoice{$C\mathrm{e}^x$}
        \wrongchoice{$C\mathrm{e}^x + x^2$}
        \wrongchoice{$-x^2 - 2x - 2 + C$}
        \wrongchoice{$\mathrm{e}^x - x^2 - 2x - 2 + C$}
      \correctchoice{$C\mathrm{e}^x - x^2 - 2x - 2$}
    \end{choices}
    \end{multicols}
\end{question}
}

\element{calculusBC}{
\begin{question}{1988-BC-q33}
    The length of the curve $y=x^3$ from $x=0$ to $x=2$ is given by:
    \begin{multicols}{2}
    \begin{choices}
        \wrongchoice{$\displaystyle \int^{\;\;2}_{0} \sqrt{1+x^2} \mathrm{d} x$}
        \wrongchoice{$\displaystyle \int^{\;\;2}_{0} \sqrt{1+3x^2} \mathrm{d} x$}
        \wrongchoice{$\displaystyle \pi\int^{\;\;2}_{0} \sqrt{1+9x^4} \mathrm{d} x$}
        \wrongchoice{$\displaystyle 2\pi\int^{\;\;2}_{0} \sqrt{1+9x^4} \mathrm{d} x$}
      \correctchoice{$\displaystyle \int^{\;\;2}_{0} \sqrt{1+9x^4} \mathrm{d} x$}
    \end{choices}
    \end{multicols}
\end{question}
}

\element{calculusBC}{
\begin{question}{1988-BC-q34}
    A curve in the plane is defined parametrically by the equations $x=t^3 + t$ and $y=t^4 + 2t^2$.
    An equation of the line tangent to the curve at $t=1$ is:
    \begin{multicols}{2}
    \begin{choices}
        \wrongchoice{$y = 2x$}
        \wrongchoice{$y = 8x$}
      \correctchoice{$y = 2x - 1$}
        \wrongchoice{$y = 4x - 5$}
        \wrongchoice{$y = 8x + 13$}
    \end{choices}
    \end{multicols}
\end{question}
}

\element{calculusBC}{
\begin{question}{1988-BC-q35}
    If $k$ is a positive integer, then $\displaystyle \lim_{x\to \infty} \frac{x^k}{\mathrm{e}^x}$ is:
    \begin{multicols}{2}
    \begin{choices}
      \correctchoice{zero}
        \wrongchoice{one}
        \wrongchoice{e}
        \wrongchoice{$k!$}
        \wrongchoice{nonexistent}
    \end{choices}
    \end{multicols}
\end{question}
}

\element{calculusBC}{
\begin{question}{1988-BC-q36}
    Let $R$ be the region between the graphs of $y=1$ and $y=\sin x$ from $x=0$ to $x=\frac{\pi}{2}$.
    The volume of the solid obtained by revolving $R$ about the $x$-axis is given by:
    \begin{multicols}{2}
    \begin{choices}
        \wrongchoice{$\displaystyle 2\pi \int^{\;\;\pi/2}_{0} x \sin x \mathrm{d}x$}
        \wrongchoice{$\displaystyle 2\pi \int^{\;\;\pi/2}_{0} x \cos x \mathrm{d}x$}
        \wrongchoice{$\displaystyle \pi \int^{\;\;\pi/2}_{0} \left(1-\sin x\right)^2 \mathrm{d}x$}
        %% NOTE: TODO: questionmult???
        %% E or D ??
      \correctchoice{$\displaystyle \pi \int^{\;\;\pi/2}_{0} \sin^2 x \mathrm{d}x$}
      \correctchoice{$\displaystyle \pi \int^{\;\;\pi/2}_{0} \left(1-\sin^2 x\right) \mathrm{d}x$}
    \end{choices}
    \end{multicols}
\end{question}
}

\element{calculusBC}{
\begin{question}{1988-BC-q37}
    A person 2 meters tall walks directly away from a streetlight that is 8 meters above the ground.
    If the person is walking at a constant rate and the person's shadow is lengthening at the rate of $4/9$ meter per second,
        at what rate, in meters per second, is the person walking?
    \begin{multicols}{3}
    \begin{choices}
        \wrongchoice{$\dfrac{4}{27}$}
        \wrongchoice{$\dfrac{4}{9}$}
        \wrongchoice{$\dfrac{3}{4}$}
      \correctchoice{$\dfrac{4}{3}$}
        \wrongchoice{$\dfrac{16}{9}$}
    \end{choices}
    \end{multicols}
\end{question}
}

\element{calculusBC}{
\begin{question}{1988-BC-q38}
    What are all values of $x$ for which the series $\displaystyle \sum_{n=1}^{\infty} \frac{x^n}{n}$ converges?
    \begin{multicols}{2}
    \begin{choices}
        \wrongchoice{$-1 \leq x \leq 1$}
        \wrongchoice{$-1 < x \leq 1$}
      \correctchoice{$-1 \leq x < 1$}
        \wrongchoice{$-1 < x < 1$}
        \wrongchoice{All real $x$}
    \end{choices}
    \end{multicols}
\end{question}
}

\element{calculusBC}{
\begin{question}{1988-BC-q39}
    If $\dfrac{\mathrm{d}y}{\mathrm{d}x} = y \sec^2 x$ and $y=5$ when $x=0$, then $y=$
    \begin{multicols}{2}
    \begin{choices}
        \wrongchoice{$\mathrm{e}^{\tan x} + 4$}
        \wrongchoice{$\mathrm{e}^{\tan x} + 5$}
      \correctchoice{$5\mathrm{e}^{\tan x}$}
        \wrongchoice{$\tan x + 5$}
        \wrongchoice{$\tan x + 5\mathrm{e}^x$}
    \end{choices}
    \end{multicols}
\end{question}
}

\element{calculusBC}{
\begin{question}{1988-BC-q40}
    Let $f$ and $g$ be functions that are differentiable everywhere.
    If $g$ is the inverse function of $f$ and if $g(-2) = 5$ and $f\prime\left(5\right) = -\dfrac{1}{2}$,
        then $g\prime\left(-2\right) = $
    \begin{multicols}{3}
    \begin{choices}
        \wrongchoice{$2$}
        \wrongchoice{$\dfrac{1}{2}$}
        \wrongchoice{$\dfrac{1}{5}$}
        \wrongchoice{$-\dfrac{1}{5}$}
      \correctchoice{$-2$}
    \end{choices}
    \end{multicols}
\end{question}
}

\element{calculusBC}{
\begin{question}{1988-BC-q41}
    $\displaystyle \lim_{n\to\infty} \frac{1}{n} \left[\sqrt{\dfrac{1}{n}} + \sqrt{\dfrac{2}{n}} + \ldots + \sqrt{\dfrac{n}{n}} \right] = $
    \begin{multicols}{2}
    \begin{choices}
        \wrongchoice{$\displaystyle \frac{1}{2}\int^{\;\;1}_{0} \frac{1}{\sqrt{x}} \,\mathrm{d}x$}
      \correctchoice{$\displaystyle \int^{\;\;1}_{0} \sqrt{x}\,\mathrm{d}x$}
        \wrongchoice{$\displaystyle \int^{\;\;1}_{0} x \,\mathrm{d}x$}
        \wrongchoice{$\displaystyle \int^{\;\;2}_{1} x \,\mathrm{d}x$}
        \wrongchoice{$\displaystyle 2 \int^{\;\;2}_{1} x\sqrt{x} \,\mathrm{d}x$}
    \end{choices}
    \end{multicols}
\end{question}
}

\element{calculusBC}{
\begin{question}{1988-BC-q42}
    If $\displaystyle \int^{\;\;4}_{1}f\left(x\right)\,\mathrm{d}x = 6$, what is the value of $\displaystyle {\int^{\;\;4}_{1} f\left(x-5\right)\,\mathrm{d}x}$?
    \begin{multicols}{3}
    \begin{choices}
      \correctchoice{$6$}
        \wrongchoice{$3$}
        \wrongchoice{$0$}
        \wrongchoice{$-1$}
        \wrongchoice{$-6$}
    \end{choices}
    \end{multicols}
\end{question}
}

\element{calculusBC}{
\begin{question}{1988-BC-q43}
    Bacteria in a certain culture increase at a rate proportional to the number present.
    If the number of bacteria doubles in three hours,
        in how many hours will the number of bacteria triple?
    \begin{multicols}{3}
    \begin{choices}
      \correctchoice{$\dfrac{3\ln 3}{\ln 2}$}
        \wrongchoice{$\dfrac{2\ln 3}{\ln 2}$}
        \wrongchoice{$\dfrac{\ln 3}{\ln 2}$}
        \wrongchoice{$\ln\left(\dfrac{27}{3}\right)$}
        \wrongchoice{$\ln\left(\dfrac{9}{2}\right)$}
    \end{choices}
    \end{multicols}
\end{question}
}

\element{calculusBC}{
\begin{questionmult}{1988-BC-q44}
    Which of the following series converge?
    \begin{choices}
      \correctchoice{$\displaystyle \sum_{n=1}^{\infty} \left(-1\right)^{n+1} \frac{1}{2n +1}$}
        \wrongchoice{$\displaystyle \sum_{n=1}^{\infty} \frac{1}{n} \left(\frac{3}{2}\right)$}
        \wrongchoice{$\displaystyle \sum_{n=2}^{\infty} \frac{1}{n \ln n}$}
    \end{choices}
\end{questionmult}
}

\element{calculusBC}{
\begin{question}{1988-BC-q45}
    What is the area of the largest rectangle that can be inscribed in the ellipse $4x^2 + 9y^2 = 36$?
    \begin{multicols}{3}
    \begin{choices}
        \wrongchoice{$6\sqrt{2}$}
      \correctchoice{$12$}
        \wrongchoice{$24$}
        \wrongchoice{$24\sqrt{2}$}
        \wrongchoice{$36$}
    \end{choices}
    \end{multicols}
\end{question}
}


%% 1993 AP Calculus BC: Section I (pp. 95)
%%--------------------------------------------------
\element{calculusBC}{
\begin{question}{1993-BC-q01}
    The area of the region enclosed by the graphs of $y=x^2$ and $y=x$ is:
    \begin{multicols}{3}
    \begin{choices}
      \correctchoice{$\dfrac{1}{6}$}
        \wrongchoice{$\dfrac{1}{3}$}
        \wrongchoice{$\dfrac{1}{2}$}
        \wrongchoice{$\dfrac{5}{6}$}
        \wrongchoice{$1$}
    \end{choices}
    \end{multicols}
\end{question}
}

\element{calculusBC}{
\begin{question}{1993-BC-q02}
    If $f(x) = 2x^2 + 1$, then $\displaystyle \lim_{x\to 0} \frac{f(x)-f(0)}{x^2}$ is:
    \begin{multicols}{2}
    \begin{choices}
        \wrongchoice{zero}
        \wrongchoice{$1$}
      \correctchoice{$2$}
        \wrongchoice{$4$}
        \wrongchoice{nonexistent}
    \end{choices}
    \end{multicols}
\end{question}
}

\element{calculusBC}{
\begin{question}{1993-BC-q03}
    If $p$ is a polynomial of degree $n$, $n>0$,
        what is the degree of the polynomial $\displaystyle Q(x) = \int^{\;\;x}_{0} p(t) \mathrm{d}t$?
    \begin{multicols}{3}
    \begin{choices}
        \wrongchoice{zero}
        \wrongchoice{$1$}
        \wrongchoice{$n-1$}
        \wrongchoice{$n$}
      \correctchoice{$n+1$}
    \end{choices}
    \end{multicols}
\end{question}
}

\element{calculusBC}{
\begin{question}{1993-BC-q04}
    A particle moves along the curve $xy=10$.
    If $x=2$ and $\dfrac{\mathrm{d}y}{\mathrm{d}t} = 3$,
        what is the value of $\dfrac{\mathrm{d}x}{\mathrm{d}t}$?
    \begin{multicols}{3}
    \begin{choices}
        \wrongchoice{$-\dfrac{5}{2}$}
      \correctchoice{$-\dfrac{6}{5}$}
        \wrongchoice{zero}
        \wrongchoice{$\dfrac{4}{5}$}
        \wrongchoice{$\dfrac{6}{5}$}
    \end{choices}
    \end{multicols}
\end{question}
}

\element{calculusBC}{
\begin{question}{1993-BC-q05}
    Which of the following represents the graph of the polar curve $r=2\sec\theta$?
    \begin{multicols}{2}
    \begin{choices}
        %% NOTE: ANS is D
        \AMCboxDimensions{down=-2.5em}
        \wrongchoice{
            \begin{tikzpicture}
                %% NOTE: TODO; pgfplots
            \end{tikzpicture}
        }
    \end{choices}
    \end{multicols}
\end{question}
}

\element{calculusBC}{
\begin{question}{1993-BC-q06}
    If $x=t^2 + 1$ and $y=t^3$, then $\dfrac{\mathrm{d}^2y}{\mathrm{d}x^2} = $
    \begin{multicols}{3}
    \begin{choices}
      \correctchoice{$\dfrac{3}{4t}$}
        \wrongchoice{$\dfrac{3}{2t}$}
        \wrongchoice{$3t$}
        \wrongchoice{$6t$}
        \wrongchoice{$\dfrac{3}{2}$}
    \end{choices}
    \end{multicols}
\end{question}
}

\element{calculusBC}{
\begin{question}{1993-BC-q07}
    $\displaystyle \int^{\;\;1}_{0} x^e \mathrm{e}^{x^4} \mathrm{d}x =$
    \begin{multicols}{2}
    \begin{choices}
      \correctchoice{$\dfrac{1}{4}\left(\mathrm{e}-1\right)$}
        \wrongchoice{$\dfrac{1}{4}\mathrm{e}$}
        \wrongchoice{$\mathrm{e}-1$}
        \wrongchoice{$\mathrm{e}$}
        \wrongchoice{$4\left(\mathrm{e}-1\right)$}
    \end{choices}
    \end{multicols}
\end{question}
}

\element{calculusBC}{
\begin{question}{1993-BC-q08}
    If $f(x) = \ln\left(\mathrm{e}^{2x}\right)$, then $f\prime\left(x\right) = $
    \begin{multicols}{3}
    \begin{choices}
        \wrongchoice{$1$}
      \correctchoice{$2$}
        \wrongchoice{$2x$}
        \wrongchoice{$\mathrm{e}^{-2x}$}
        \wrongchoice{$2\mathrm{e}^{-2x}$}
    \end{choices}
    \end{multicols}
\end{question}
}

\element{calculusBC}{
\begin{question}{1993-BC-q09}
    If $f(x) = 1 + x^{2/3}$, which of the following is \emph{not} true?
    \begin{choices}
        \wrongchoice{$f$ is continuous for all real numbers.}
        \wrongchoice{$f$ has a minimum at $x=0$}
        \wrongchoice{$f$ is increasing for $x>0$.}
      \correctchoice{$f\prime\left(x\right)$ exists for all $x$.}
        \wrongchoice{$f\dprime\left(x\right)$ is negative for $x>0$.}
    \end{choices}
\end{question}
}

\element{calculusBC}{
\begin{questionmult}{1993-BC-q10}
    Which of the following functions are continuous at $x=1$?
    \begin{multicols}{3}
    \begin{choices}
      \correctchoice{$\ln x$}
      \correctchoice{$\mathrm{e}^x$}
      \correctchoice{$\ln\left(\mathrm{e}^x-1\right)$}
    \end{choices}
    \end{multicols}
\end{questionmult}
}

\element{calculusBC}{
\begin{question}{1993-BC-q11}
    $\displaystyle \int^{\;\;\infty}_{4} \frac{-2x}{\sqrt[3]{9-x^2}} \mathrm{d}x$ is:
    \begin{multicols}{2}
    \begin{choices}
        \wrongchoice{$7^{2/3}$}
        \wrongchoice{$\dfrac{3}{2} 7^{2/3}$}
        \wrongchoice{$9^{2/3} + 7^{2/3}$}
        \wrongchoice{$\dfrac{3}{2} \left( 9^{2/3} + 7^{2/3} \right)$}
      \correctchoice{nonexistent}
    \end{choices}
    \end{multicols}
\end{question}
}

\element{calculusBC}{
\begin{question}{1993-BC-q12}
    The position of a particle moving along the $x$-axis is $x(t)=\sin\left(2t\right) - \cos\left(3t\right)$ for time $t\geq 0$.
    When $t=\pi$, the acceleration of the particle is:
    \begin{multicols}{3}
    \begin{choices}
        \wrongchoice{$9$}
        \wrongchoice{$\dfrac{1}{9}$}
        \wrongchoice{zero}
        \wrongchoice{$-\dfrac{1}{9}$}
      \correctchoice{$-9$}
    \end{choices}
    \end{multicols}
\end{question}
}

\element{calculusBC}{
\begin{question}{1993-BC-q13}
    If $\dfrac{\mathrm{d}y}{\mathrm{d}x} = x^2y$, then $y$ could be:
    \begin{multicols}{2}
    \begin{choices}
        \wrongchoice{$3\ln\left(\dfrac{x}{3}\right)$}
        \wrongchoice{$\mathrm{e}^{x^3/3} + 7$}
      \correctchoice{$2\mathrm{e}^{x^3/3}$}
        \wrongchoice{$3\mathrm{e}^{2x}$}
        \wrongchoice{$\dfrac{x^3}{3}+ 1$}
    \end{choices}
    \end{multicols}
\end{question}
}

\element{calculusBC}{
\begin{question}{1993-BC-q14}
    The \emph{derivative} of $f$ is $x^4 (x-2) (x+3)$.
    At how many points will the graph of $f$ have a relative maximum?
    \begin{multicols}{2}
    \begin{choices}
        \wrongchoice{none}
      \correctchoice{one}
        \wrongchoice{two}
        \wrongchoice{three}
        \wrongchoice{four}
    \end{choices}
    \end{multicols}
\end{question}
}

\element{calculusBC}{
\begin{question}{1993-BC-q15}
    If $f(x) = \mathrm{e}^{\tan^2 x}$, then $f\prime\left(x\right) = $
    \begin{multicols}{2}
    \begin{choices}
        \wrongchoice{$\mathrm{e}^{\tan^2 x}$}
        \wrongchoice{$\sec^2 x \mathrm{e}^{\tan^2 x}$}
        \wrongchoice{$\tan^2 x \mathrm{e}^{\tan^2 x-1}$}
      \correctchoice{$2\tan x \sec^2 x \mathrm{e}^{\tan^2 x}$}
        \wrongchoice{$2\tan x \mathrm{e}^{\tan^2 x}$}
    \end{choices}
    \end{multicols}
\end{question}
}

\element{calculusBC}{
\begin{question}{1993-BC-q16}
    Which of the following series diverge?
    \begin{multicols}{3}
    \begin{choices}
        %% ANS: none
        \wrongchoice{$\displaystyle \sum_{k=3}^{\infty} \frac{2}{k^2+1}$}
        \wrongchoice{$\displaystyle \sum_{k=3}^{\infty} \left(\frac{6}{7}\right)^k$}
        \wrongchoice{$\displaystyle \sum_{k=3}^{\infty} \frac{\left(-1\right)^k}{k}$}
    \end{choices}
    \end{multicols}
\end{question}
}

\element{calculusBC}{
\begin{question}{1993-BC-q17}
    The slope of the line tangent to the graph of $\ln\left(xy\right) = x$ at the point where $x=1$ is:
    \begin{multicols}{3}
    \begin{choices}
      \correctchoice{0}
        \wrongchoice{1}
        \wrongchoice{$\mathrm{e}$}
        \wrongchoice{$\mathrm{e}^2$}
        \wrongchoice{$1-\mathrm{e}$}
    \end{choices}
    \end{multicols}
\end{question}
}

\element{calculusBC}{
\begin{question}{1993-BC-q18}
    If $\mathrm{e}^{f(x)} = 1 + x^2$, then $f\prime\left(x\right) = $
    \begin{multicols}{2}
    \begin{choices}
        \wrongchoice{$\dfrac{1}{1+x^2}$}
      \correctchoice{$\dfrac{2x}{1+x^2}$}
        \wrongchoice{$2x \left(1+x^2\right)$}
        \wrongchoice{$2x \left(\mathrm{e}^{1+x^2}\right)$}
        \wrongchoice{$2x \ln\left(1+x^2\right)$}
    \end{choices}
    \end{multicols}
\end{question}
}

\element{calculusBC}{
\begin{question}{1993-BC-q19}
    The shaded region $R$, shown in the figure below,
        is rotated about the $y$-axis to form a solid whose volume is 10 cubic units.
    \begin{center}
    \begin{tikzpicture}
        %% NOTE:
    \end{tikzpicture}
    \end{center}
    Of the following, which best approximates $k$?
    \begin{multicols}{3}
    \begin{choices}
        \wrongchoice{$1.51$}
      \correctchoice{$2.09$}
        \wrongchoice{$2.49$}
        \wrongchoice{$4.18$}
        \wrongchoice{$4.77$}
    \end{choices}
    \end{multicols}
\end{question}
}

\element{calculusBC}{
\begin{question}{1993-BC-q20}
    A particle moves along the $x$-axis so that at any time $t\leq 0$ the acceleration of the particle is $a\left(t\right) = \mathrm{e}^{-2t}$.
    If at $t=0$ the velocity of the particle is $\frac{5}{2}$ and its position is $\frac{17}{4}$,
        then its position at any time $t>0$ is $x(t) = $
    \begin{multicols}{2}
    \begin{choices}
        \wrongchoice{$-\dfrac{\mathrm{e}^{-2t}}{2} + 3$}
        \wrongchoice{$\dfrac{\mathrm{e}^{-2t}}{4} + 4$}
        \wrongchoice{$4\mathrm{e}^{-2t} + \dfrac{9}{2}t + \dfrac{1}{4}$}
        \wrongchoice{$\dfrac{\mathrm{e}^{-2t}}{2} + 3t + \dfrac{15}{4}$}
      \correctchoice{$\dfrac{\mathrm{e}^{-2t}}{4} + 3t + 4$}
    \end{choices}
    \end{multicols}
\end{question}
}

\element{calculusBC}{
\begin{question}{1993-BC-q21}
    The value of the derivative of $y=\dfrac{\sqrt[3]{x^2+8}}{\sqrt[4]{2x+1}}$ at $x=0$ is:
    \begin{multicols}{3}
    \begin{choices}
      \correctchoice{$-1$}
        \wrongchoice{$-\dfrac{1}{2}$}
        \wrongchoice{$0$}
        \wrongchoice{$\dfrac{1}{2}$}
        \wrongchoice{$1$}
    \end{choices}
    \end{multicols}
\end{question}
}

\element{calculusBC}{
\begin{question}{1993-BC-q22}
    If $f\left(x\right) = x^2\mathrm{e}^x$, then the graph of $f$ is decreasing for all $x$ such that:
    \begin{multicols}{2}
    \begin{choices}
        \wrongchoice{$x < -2$}
      \correctchoice{$-2 < x < 0$}
        \wrongchoice{$x > -2$}
        \wrongchoice{$x < 0$}
        \wrongchoice{$x > 0$}
    \end{choices}
    \end{multicols}
\end{question}
}

\element{calculusBC}{
\begin{question}{1993-BC-q23}
    The length of the curve determined by the equations $x=t^2$ and $y=t$ from $t=0$ to $t=4$ is:
    \begin{multicols}{2}
    \begin{choices}
        \wrongchoice{$\displaystyle \int^{\;\;4}_{0} \sqrt{4x+1}\,\mathrm{d}t$}
        \wrongchoice{$\displaystyle 2\int^{\;\;4}_{0} \sqrt{t^2+1}\,\mathrm{d}t$}
        \wrongchoice{$\displaystyle \int^{\;\;4}_{0} \sqrt{2t^2+1}\,\mathrm{d}t$}
        \wrongchoice{$\displaystyle \int^{\;\;4}_{0} \sqrt{4t^2+1}\,\mathrm{d}t$}
      \correctchoice{$\displaystyle 2\pi\int^{\;\;4}_{0} \sqrt{4t^2+1}\,\mathrm{d}t$}
    \end{choices}
    \end{multicols}
\end{question}
}

\element{calculusBC}{
\begin{question}{1993-BC-q24}
    Let $f$ and $g$ be functions that are differentiable for all real numbers,
        with $g\left(x\right) \neq 0$ for $x \neq 0$.
    If $\displaystyle \lim_{x\to 0} f(x) = \lim_{x\to 0} g(x) = 0$ and
        $\displaystyle \lim_{x\to 0} \dfrac{f\prime\left(x\right)}{g\prime\left(x\right)}$ exists,
        then $\displaystyle \lim_{x\to 0} \dfrac{f\left(x\right)}{g\left(x\right)}$ is
    \begin{choices}
        \wrongchoice{$0$}
        \wrongchoice{$\dfrac{f\prime\left(x\right)}{g\prime\left(x\right)}$}
      \correctchoice{$\displaystyle \lim_{x\to 0} \dfrac{f\prime\left(x\right)}{g\prime\left(x\right)}$}
        \wrongchoice{$\displaystyle \frac{f\prime\left(x\right) g\left(x\right) - f\left(x\right)g\prime\left(x\right)}{\left(f\left(x\right)\right)^2}$}
        \wrongchoice{nonexistent}
    \end{choices}
\end{question}
}

\element{calculusBC}{
\begin{question}{1993-BC-q25}
    Consider the curve in the $xy$-plane represented by $x=\mathrm{e}^t$ and $y=t\mathrm{e}^{-t}$ for $t\geq 0$.
    The slope of the line tangent to the curve at the point where $x=3$ is:
    \begin{multicols}{2}
    \begin{choices}
        \wrongchoice{$20.086$}
        \wrongchoice{$0.342$}
        \wrongchoice{$-0.005$}
      \correctchoice{$-0.011$}
        \wrongchoice{$-0.033$}
    \end{choices}
    \end{multicols}
\end{question}
}

\element{calculusBC}{
\begin{question}{1993-BC-q26}
    If $y=\arctan\left(\mathrm{e}^{2x}\right)$, then $\dfrac{\mathrm{d}y}{\mathrm{d}x} = $
    \begin{multicols}{2}
    \begin{choices}
        \wrongchoice{$\dfrac{2\mathrm{e}^{2x}}{\sqrt{1-\mathrm{e}^{4x}}}$}
      \correctchoice{$\dfrac{2\mathrm{e}^{2x}}{1+\mathrm{e}^{4x}}$}
        \wrongchoice{$\dfrac{\mathrm{e}^{2x}}{1-\mathrm{e}^{4x}}$}
        \wrongchoice{$\dfrac{1}{\sqrt{1-\mathrm{e}^{4x}}}$}
        \wrongchoice{$\dfrac{1}{1+\mathrm{e}^{4x}}$}
    \end{choices}
    \end{multicols}
\end{question}
}

\element{calculusBC}{
\begin{question}{1993-BC-q27}
    The interval of convergences of $\displaystyle \sum_{n=0}^{\infty} \frac{\left(x-1\right)^n}{e^n}$ is
    \begin{multicols}{2}
    \begin{choices}
        \wrongchoice{$-3 < x \leq 3$}
        \wrongchoice{$-3 \leq x \leq 3$}
      \correctchoice{$-2 < x < 4$}
        \wrongchoice{$-2 \leq x < 4$}
        \wrongchoice{$0 \leq x \leq 2$}
    \end{choices}
    \end{multicols}
\end{question}
}

\element{calculusBC}{
\begin{question}{1993-BC-q28}
    If a particle moes in the $xy$-plane so that at time $>0$ its position vector vector is $\left(\ln\left(t^2+2t\right),2t^2\right)$, then at time $t=2$, its velocity is:
    \begin{multicols}{2}
    \begin{choices}
      \correctchoice{$\left(\dfrac{3}{4},8\right)$}
        \wrongchoice{$\left(\dfrac{3}{4},4\right)$}
        \wrongchoice{$\left(\dfrac{1}{8},8\right)$}
        \wrongchoice{$\left(\dfrac{1}{8},4\right)$}
        \wrongchoice{$\left(-\dfrac{5}{16},4\right)$}
    \end{choices}
    \end{multicols}
\end{question}
}

\element{calculusBC}{
\begin{question}{1993-BC-q29}
    $\displaystyle \int\,x\sec^2 x\,\mathrm{d} x =$
    \begin{choices}
        \wrongchoice{$x \tan x + C$}
        \wrongchoice{$\dfrac{x^2}{2} \tan x + C$}
        \wrongchoice{$\sec^2 x + 2\sec^2 x \tan x + C$}
        \wrongchoice{$x \tan x - \ln\left|\cos x\right| + C$}
      \correctchoice{$x \tan x + \ln\left|\cos x\right| + C$}
    \end{choices}
\end{question}
}

\element{calculusBC}{
\begin{question}{1993-BC-q30}
    What is the volume of the solid generated by rotating about the $x$-axis the region enclosed by the curve $y=\sec x$ and the lines $x=0$, $y=0$ and $x=\dfrac{\pi}{3}$?
    \begin{multicols}{2}
    \begin{choices}
        \wrongchoice{$\dfrac{\pi}{\sqrt{3}}$}
        \wrongchoice{$\pi$}
      \correctchoice{$\pi\sqrt{3}$}
        \wrongchoice{$\dfrac{8\pi}{3}$}
        \wrongchoice{$\pi\ln\left(\dfrac{1}{2}+\sqrt{3}\right)$}
    \end{choices}
    \end{multicols}
\end{question}
}

\element{calculusBC}{
\begin{question}{1993-BC-q31}
    If $s_n = \left(\dfrac{\left(5+n\right)^{100}}{5^{n+1}}\right) \left(\dfrac{5^n}{\left(4+n\right)^{100}}\right)$, to what number does the sequence $\left\{s_x\right\}$ converge?
    \begin{choices}
      \correctchoice{$\dfrac{1}{5}$}
        \wrongchoice{$1$}
        \wrongchoice{$\dfrac{5}{4}$}
        \wrongchoice{$\left(\dfrac{5}{4}\right)^{100}$}
        \wrongchoice{The sequence does not converge.}
    \end{choices}
\end{question}
}

\element{calculusBC}{
\begin{questionmult}{1993-BC-q32}
    If $\displaystyle \int^{\;\;b}_{a} f(x)\,\mathrm{d}x = 5$ and $\displaystyle \int^{\;\;b}_{a} g(x)\,\mathrm{d}x = -1$, which of the following must be true?
    \begin{choices}
        \wrongchoice{$f(x) > g(x)$ for $a \leq x \leq b$}
      \correctchoice{$\displaystyle \int^{\;\;b}_{a} \left(f(x) + g(x)\right)\,\mathrm{d}x = 4$}
        \wrongchoice{$\displaystyle \int^{\;\;b}_{a} \left(f(x) g(x)\right)\,\mathrm{d}x = -5$}
    \end{choices}
\end{questionmult}
}

\element{calculusBC}{
\begin{question}{1993-BC-q33}
    Which of the following is equal to $\displaystyle \int^{\;\;\pi}_{0} \sin x\,\mathrm{d}x$?
    \begin{multicols}{2}
    \begin{choices}
      \correctchoice{$\displaystyle \int^{\;\;\pi/2}_{-\pi/2} \cos x\,\mathrm{d}x$}
        \wrongchoice{$\displaystyle \int^{\;\;\pi}_{0} \cos x\,\mathrm{d}x$}
        \wrongchoice{$\displaystyle \int^{\;\;0}_{-\pi} \sin x\,\mathrm{d}x$}
        \wrongchoice{$\displaystyle \int^{\;\;\pi/2}_{-\pi/2} \sin x\,\mathrm{d}x$}
        \wrongchoice{$\displaystyle \int^{\;\;2\pi}_{\pi} \sin x\,\mathrm{d}x$}
    \end{choices}
    \end{multicols}
\end{question}
}

\element{calculusBC}{
\begin{question}{1993-BC-q34}
    In the figure above, $PQ$ represents a 40 foot ladder with end $P$ against a vertical wall and end $Q$ on level ground.
    If the ladder is slipping down the wall,
        what is the distance $RQ$ at the instalt when $Q$ is moving along the ground $\dfrac{3}{4}$ as fast as $P$ is moving down the wall?
    \begin{multicols}{3}
    \begin{choices}
        \wrongchoice{$\dfrac{6}{5}\sqrt{10}$}
        \wrongchoice{$\dfrac{8}{5}\sqrt{10}$}
        \wrongchoice{$\dfrac{80}{\sqrt{7}}$}
        \wrongchoice{$24$}
      \correctchoice{$32$}
    \end{choices}
    \end{multicols}
\end{question}
}

\element{calculusBC}{
\begin{question}{1993-BC-q35}
    If $F$ and $f$ are differentiable functions such that $F(x) = \int^{\,\,x}_{0} f(t)\,\mathrm{d}t$, and if $F(a) = -2$ and $F(b)=-2$ where $a<b$, which of the following must be true?
    \begin{choices}
      \correctchoice{$f(x) = 0$ for some $x$ such that $a<x<b$}
        \wrongchoice{$f(x) > 0$ for all $x$ such that $a<x<b$}
        \wrongchoice{$f(x) < 0$ for all $x$ such that $a<x<b$}
        \wrongchoice{$F(x)\leq 0$ for all $x$ such that $a<x<b$}
        \wrongchoice{$F(x)=0$ for some $x$ such that $a<x<b$}
    \end{choices}
\end{question}
}

\element{calculusBC}{
\begin{question}{1993-BC-q36}
    Consider all right circular cylinders for which the sum of the height and circumference is 30 centimeters.
    What is the radius of the one with maximum volume?
    \begin{multicols}{3}
    \begin{choices}
        \wrongchoice{\SI{3}{\centi\meter}}
        \wrongchoice{\SI{10}{\centi\meter}}
        \wrongchoice{\SI{20}{\centi\meter}}
        \wrongchoice{\SI[parse-numbers=false]{\dfrac{30}{\pi^2}}{\centi\meter}}
      \correctchoice{\SI[parse-numbers=false]{\dfrac{10}{\pi}}{\centi\meter}}
    \end{choices}
    \end{multicols}
\end{question}
}

\element{calculusBC}{
\begin{question}{1993-BC-q37}
    If
    \begin{math}
        f(x) =
        \begin{cases}
            x               & \text{for }x\leq 1 \\
            \dfrac{1}{x}    & \text{for }x> 1 \\
        \end{cases}
    \end{math}
    then $\displaystyle \int^{\;\;\mathrm{e}}_{0} f(x)\,\mathrm{d}x = $
    \begin{multicols}{3}
    \begin{choices}
        \wrongchoice{$0$}
      \correctchoice{$\dfrac{3}{2}$}
        \wrongchoice{$2$}
        \wrongchoice{$\mathrm{e}$}
        \wrongchoice{$\mathrm{e}+\dfrac{1}{2}$}
    \end{choices}
    \end{multicols}
\end{question}
}

\element{calculusBC}{
\begin{question}{1993-BC-q38}
    During a certain epidemic, the number of people that are infected at any time increases at a rate proportional to the number of people that are infected at that time.
    If 1,000 people are infected when the epidemic is first discovered, and 1,200 are infected 7 days later,
        how many people are infected 12 days after the epidemic is first discovered?
    \begin{multicols}{3}
    \begin{choices}
        \wrongchoice{$343$}
        \wrongchoice{\num{1 343}}
      \correctchoice{\num{1 367}}
        \wrongchoice{\num{1 400}}
        \wrongchoice{\num{2 057}}
    \end{choices}
    \end{multicols}
\end{question}
}

\element{calculusBC}{
\begin{question}{1993-BC-q39}
    If $\dfrac{\mathrm{d}y}{\mathrm{d}x} = \dfrac{1}{x}$, then the average rate of change of $y$ with respect to $x$ on the closed interval $\left[1,4\right]$ is:
    \begin{multicols}{3}
    \begin{choices}
        \wrongchoice{$-\dfrac{1}{4}$}
        \wrongchoice{$\dfrac{1}{2}\ln 2$}
      \correctchoice{$\dfrac{2}{3}\ln 2$}
        \wrongchoice{$\dfrac{2}{5}$}
        \wrongchoice{$2$}
    \end{choices}
    \end{multicols}
\end{question}
}

\element{calculusBC}{
\begin{question}{1993-BC-q40}
    Let $R$ be the region in the first quadrant enclosed by the $x$-axis and the graph of $y=\ln\left(1+2x-x^2\right)$.
    If Simpson's Rule with 2 subintervals is used to approximate the area of $R$, the approximation is:
    \begin{multicols}{3}
    \begin{choices}
        \wrongchoice{$0.462$}
        \wrongchoice{$0.693$}
      \correctchoice{$0.924$}
        \wrongchoice{$0.986$}
        \wrongchoice{$1.850$}
    \end{choices}
    \end{multicols}
\end{question}
}

\element{calculusBC}{
\begin{question}{1993-BC-q41}
    Let $\displaystyle f(x) = \int^{\;\;x^2-3x}_{-2} \mathrm{e}^{t^2}\,\mathrm{d}x$.
    At what value of $x$ is $f(x)$ a minimum?
    \begin{multicols}{2}
    \begin{choices}
        \wrongchoice{For no value of $x$}
        \wrongchoice{$\dfrac{1}{2}$}
      \correctchoice{$\dfrac{3}{2}$}
        \wrongchoice{$2$}
        \wrongchoice{$3$}
    \end{choices}
    \end{multicols}
\end{question}
}

\element{calculusBC}{
\begin{question}{1993-BC-q42}
    $\displaystyle \lim_{x\to 0} \left(1 + 2x\right)^{\csc x} =$
    \begin{multicols}{3}
    \begin{choices}
        \wrongchoice{$0$}
        \wrongchoice{$1$}
        \wrongchoice{$2$}
        \wrongchoice{$\mathrm{e}$}
      \correctchoice{$\mathrm{e}^2$}
    \end{choices}
    \end{multicols}
\end{question}
}

\element{calculusBC}{
\begin{question}{1993-BC-q43}
    The coefficient of $x^6$ in the Taylor series expansion about $x=0$ for $f(x) = \sin\left(x^2\right)$ is:
    \begin{multicols}{3}
    \begin{choices}
      \correctchoice{$-\dfrac{1}{6}$}
        \wrongchoice{$0$}
        \wrongchoice{$\dfrac{1}{120}$}
        \wrongchoice{$\dfrac{1}{6}$}
        \wrongchoice{$1$}
    \end{choices}
    \end{multicols}
\end{question}
}

\element{calculusBC}{
\begin{question}{1993-BC-q44}
    If $f$ is continuous on the interval $\left[a,b\right]$, then there exists $c$ such that $a<c<b$ and $\int^{\,\,b}_{a} f(x)\,\mathrm{d}x =$
    \begin{multicols}{2}
    \begin{choices}
        \wrongchoice{$\dfrac{f(c)}{b-a}$}
        \wrongchoice{$\dfrac{f(b) - f(a)}{b-a}$}
        \wrongchoice{$f(b) - f(a)$}
        \wrongchoice{$f\prime\left(c\right)\left(b-a\right)$}
      \correctchoice{$f\left(c\right)\left(b-a\right)$}
    \end{choices}
    \end{multicols}
\end{question}
}

\element{calculusBC}{
\begin{question}{1993-BC-q45}
    If $\displaystyle f(x) = \sum_{k=1}^{\infty} \left(\sin^2 x\right)^k$, then $f(1)$ is:
    \begin{multicols}{3}
    \begin{choices}
        \wrongchoice{$0.369$}
        \wrongchoice{$0.585$}
        \wrongchoice{$2.400$}
      \correctchoice{$2.426$}
        \wrongchoice{$3.426$}
    \end{choices}
    \end{multicols}
\end{question}
}

%% 1997 AP Calculus BC: Section I (pp. 119)
%%--------------------------------------------------
\element{calculusBC}{
\begin{question}{1997-BC-q01}
    $\displaystyle \int^{\;\;1}_{0} \sqrt{x} \left(x+1\right)\,\mathrm{d}x =$
    \begin{multicols}{3}
    \begin{choices}
        \wrongchoice{$0$}
        \wrongchoice{$1$}
      \correctchoice{$\dfrac{16}{15}$}
        \wrongchoice{$\dfrac{7}{5}$}
        \wrongchoice{$2$}
    \end{choices}
    \end{multicols}
\end{question}
}

\element{calculusBC}{
\begin{question}{1997-BC-q02}
    If $x=\mathrm{e}^{2t}$ and $y=\sin\left(2t\right)$, then $\dfrac{\mathrm{d}y}{\mathrm{d}x} = $
    \begin{multicols}{2}
    \begin{choices}
        \wrongchoice{$4\mathrm{e}^{2t}\cos\left(2t\right)$}
        \wrongchoice{$\dfrac{\mathrm{e}^{2t}}{\cos\left(2t\right)}$}
        \wrongchoice{$\dfrac{\sin\left(2t\right)}{2\mathrm{e}^{2t}}$}
        \wrongchoice{$\dfrac{\cos\left(2t\right)}{2\mathrm{e}^{2t}}$}
      \correctchoice{$\dfrac{\cos\left(2t\right)}{\mathrm{e}^{2t}}$}
    \end{choices}
    \end{multicols}
\end{question}
}

\element{calculusBC}{
\begin{question}{1997-BC-q03}
    The function $f$ given by $f(x) = 3x^5 - 4x^3 - 3x$ has a relative maximum at $x=$
    \begin{multicols}{3}
    \begin{choices}
      \correctchoice{$-1$}
        \wrongchoice{$-\dfrac{\sqrt{5}}{5}$}
        \wrongchoice{$0$}
        \wrongchoice{$\dfrac{\sqrt{5}}{5}$}
        \wrongchoice{$1$}
    \end{choices}
    \end{multicols}
\end{question}
}

\element{calculusBC}{
\begin{question}{1997-BC-q04}
    $\dfrac{\mathrm{d}}{\mathrm{d}x} \left( x \mathrm{e}^{\ln x^2}\right) = $
    \begin{multicols}{2}
    \begin{choices}
        \wrongchoice{$1+2x$}
        \wrongchoice{$x+x^2$}
      \correctchoice{$3x^2$}
        \wrongchoice{$x^3$}
        \wrongchoice{$x^2 + x^3$}
    \end{choices}
    \end{multicols}
\end{question}
}

\element{calculusBC}{
\begin{question}{1997-BC-q05}
    If $f(x) = \left(x-1\right)^{3/2} + \dfrac{\mathrm{e}^{x-2}}{2}$, then $f\prime\left(2\right) = $
    \begin{multicols}{2}
    \begin{choices}
        \wrongchoice{$1$}
        \wrongchoice{$\dfrac{3}{2}$}
      \correctchoice{$2$}
        \wrongchoice{$\dfrac{7}{2}$}
        \wrongchoice{$\dfrac{3+\mathrm{e}}{2}$}
    \end{choices}
    \end{multicols}
\end{question}
}

\element{calculusBC}{
\begin{question}{1997-BC-q06}
    The line normal to the curve $y=\sqrt{16-x}$ at the point $(0,4)$ has slope:
    \begin{multicols}{3}
    \begin{choices}
      \correctchoice{$8$}
        \wrongchoice{$4$}
        \wrongchoice{$\dfrac{1}{8}$}
        \wrongchoice{$-\dfrac{1}{8}$}
        \wrongchoice{$-8$}
    \end{choices}
    \end{multicols}
\end{question}
}

\newcommand{\calculusBCNineteenNinetySevenQSix}{
\begin{tikzpicture}
    %% NOTE: TODO: pgfplots
\end{tikzpicture}
}

\element{calculusBC}{
\begin{question}{1997-BC-q07}
    The function $f$ is defined on the closed interval $\left[0,8\right]$.
    The graph of its derivative $f\prime$ is shown above.
    \begin{center}
        \calculusBCNineteenNinetySevenQSix
    \end{center}
    The point $(3,5)$ is on the graph of $y=f(x)$.
    An equatoin of the line tangent to the graph of $f$ at $(3,5)$ is:
    \begin{multicols}{2}
    \begin{choices}
        \wrongchoice{$y=2$}
        \wrongchoice{$y=5$}
      \correctchoice{$y-5=2(x-3)$}
        \wrongchoice{$y+5=2(x-3)$}
        \wrongchoice{$y+5=2(x+3)$}
    \end{choices}
    \end{multicols}
\end{question}
}

\element{calculusBC}{
\begin{question}{1997-BC-q08}
    The function $f$ is defined on the closed interval $\left[0,8\right]$.
    The graph of its derivative $f\prime$ is shown above.
    \begin{center}
        \calculusBCNineteenNinetySevenQSix
    \end{center}
    How many points of inflection does the graph of $f$ have?
    \begin{multicols}{3}
    \begin{choices}
        \wrongchoice{Two}
        \wrongchoice{Three}
        \wrongchoice{Four}
        \wrongchoice{Five}
      \correctchoice{Six}
    \end{choices}
    \end{multicols}
\end{question}
}

\element{calculusBC}{
\begin{question}{1997-BC-q09}
    The function $f$ is defined on the closed interval $\left[0,8\right]$.
    The graph of its derivative $f\prime$ is shown above.
    \begin{center}
        \calculusBCNineteenNinetySevenQSix
    \end{center}
    At what value of $x$ does the absolute minimum of $f$ occur?
    \begin{multicols}{3}
    \begin{choices}
      \correctchoice{0}
        \wrongchoice{2}
        \wrongchoice{4}
        \wrongchoice{6}
        \wrongchoice{8}
    \end{choices}
    \end{multicols}
\end{question}
}

\element{calculusBC}{
\begin{question}{1997-BC-q10}
    If $y = xy + x^2 + 1$, then when $x=-1$, $\dfrac{\mathrm{d}y}{\mathrm{d}x}$ is:
    \begin{multicols}{2}
    \begin{choices}
        \wrongchoice{$\dfrac{1}{2}$}
      \correctchoice{$-\dfrac{1}{2}$}
        \wrongchoice{$-1$}
        \wrongchoice{$-2$}
        \wrongchoice{nonexistent}
    \end{choices}
    \end{multicols}
\end{question}
}

\element{calculusBC}{
\begin{question}{1997-BC-q11}
    $\displaystyle \int^{\;\;\infty}_{1} \frac{x}{\left(1+x^2\right)^2}\mathrm{d}x$ is
    \begin{multicols}{2}
    \begin{choices}
        \wrongchoice{$-\dfrac{1}{2}$}
        \wrongchoice{$-\dfrac{1}{4}$}
      \correctchoice{$\dfrac{1}{4}$}
        \wrongchoice{$\dfrac{1}{2}$}
        \wrongchoice{divergent}
    \end{choices}
    \end{multicols}
\end{question}
}

\element{calculusBC}{
\begin{question}{1997-BC-q12}
    The graph of $f\prime$, the derivative of $f$, is shown in the figure above.
    \begin{center}
    \begin{tikzpicture}
        %% NOTE: TODO: pgfplots
    \end{tikzpicture}
    \end{center}
    Which of the following describes all relative extrema of $f$ on the open interval $\left(a,b\right)$?
    \begin{choices}
      \correctchoice{One relative maximum and two relative minima}
        \wrongchoice{Two relative maximum and one relative minima}
        \wrongchoice{Three relative maximum and one relative minima}
        \wrongchoice{One relative maximum and three relative minima}
        \wrongchoice{Three relative maximum and two relative minima}
    \end{choices}
\end{question}
}

\element{calculusBC}{
\begin{question}{1997-BC-q13}
    A particle moves along the $x$-axis so that its acceleratoin at any time $t$ is $a(t) = 2t - 7$.
    If the initial velocity of the particle is 6,
        at what time $t$ during the interval $0\leq t\leq 4$ is the particle farthest to the right?
    \begin{multicols}{3}
    \begin{choices}
        \wrongchoice{$0$}
      \correctchoice{$1$}
        \wrongchoice{$2$}
        \wrongchoice{$3$}
        \wrongchoice{$4$}
    \end{choices}
    \end{multicols}
\end{question}
}

\element{calculusBC}{
\begin{question}{1997-BC-q14}
    The sum of the infinite geometric series ${\dfrac{3}{2} + \dfrac{9}{16} + \dfrac{27}{128} + \dfrac{81}{1024} + \ldots}$ is:
    \begin{multicols}{3}
    \begin{choices}
        \wrongchoice{$1.60$}
        \wrongchoice{$2.35$}
      \correctchoice{$2.40$}
        \wrongchoice{$2.45$}
        \wrongchoice{$2.50$}
    \end{choices}
    \end{multicols}
\end{question}
}

\element{calculusBC}{
\begin{question}{1997-BC-q15}
    The length of the path described by the parametric equations $x=\cos^3 t$ and $y=\sin^3 t$,
        for $0\leq t \leq \frac{\pi}{2}$, is given by:
    \begin{choices}
        \wrongchoice{$\displaystyle \int^{\;\;\pi/2}_{0} \sqrt{3\cos^2 t + 3\sin^2 t}\,\mathrm{d}x$}
        \wrongchoice{$\displaystyle \int^{\;\;\pi/2}_{0} \sqrt{-3\cos^2 t \sin t + 3\sin^2 t \cos t}\,\mathrm{d}x$}
        \wrongchoice{$\displaystyle \int^{\;\;\pi/2}_{0} \sqrt{9\cos^4 t + 9\sin^4 t}\,\mathrm{d}x$}
      \correctchoice{$\displaystyle \int^{\;\;\pi/2}_{0} \sqrt{9\cos^4 t \sin^2 t + 9\sin^4 t \cos^2 t}\,\mathrm{d}x$}
        \wrongchoice{$\displaystyle \int^{\;\;\pi/2}_{0} \sqrt{9\cos^6 t + \sin^6 t}\,\mathrm{d}x$}
    \end{choices}
\end{question}
}

\element{calculusBC}{
\begin{question}{1997-BC-q16}
    $\displaystyle \lim_{h\to 0} \frac{\mathrm{e}^h - 1}{2h}$ is
    \begin{multicols}{2}
    \begin{choices}
        \wrongchoice{$0$}
      \correctchoice{$\dfrac{1}{2}$}
        \wrongchoice{$1$}
        \wrongchoice{$\mathrm{e}$}
        \wrongchoice{nonexistent}
    \end{choices}
    \end{multicols}
\end{question}
}

\element{calculusBC}{
\begin{question}{1997-BC-q17}
    Let $f$ be the function given by $f(x)=\ln\left(3-x\right)$.
    The third-degree Taylor polynomial for $f$ about $x=2$:
    \begin{choices}
        \wrongchoice{$-\left(x-2\right) + \dfrac{\left(x-2\right)^2}{2} - \dfrac{\left(x-2\right)^3}{3}$}
      \correctchoice{$-\left(x-2\right) - \dfrac{\left(x-2\right)^2}{2} - \dfrac{\left(x-2\right)^3}{3}$}
        \wrongchoice{$\left(x-2\right) + \left(x-2\right)^2 + \left(x-2\right)^3$}
        \wrongchoice{$\left(x-2\right) + \dfrac{\left(x-2\right)^2}{2} + \dfrac{\left(x-2\right)^3}{3}$}
        \wrongchoice{$\left(x-2\right) - \dfrac{\left(x-2\right)^2}{2} + \dfrac{\left(x-2\right)^3}{3}$}
    \end{choices}
\end{question}
}

\element{calculusBC}{
\begin{question}{1997-BC-q18}
    For what values of $t$ does the curve given by the parametric equations $x=t^3 - t^2 - 1$ and $y=t^4 + 2t^2 - 8t$ have a vertical tangent?
    \begin{multicols}{2}
    \begin{choices}
        \wrongchoice{$0$ only}
        \wrongchoice{$1$ only}
      \correctchoice{$0$ and $\dfrac{2}{3}$ only}
        \wrongchoice{$0$, $\dfrac{2}{3}$, and 1}
        \wrongchoice{No value}
    \end{choices}
    \end{multicols}
\end{question}
}

\element{calculusBC}{
\begin{question}{1997-BC-q19}
    The graph of $y=f(x)$ is shown in the figure below.
    \begin{center}
    \begin{tikzpicture}
        %% NOTE: TODO: pgfplots
    \end{tikzpicture}
    \end{center}
    If $A_1$ and $A_2$ are positive numbers that represent the areas of the shaded regions,
        then in terms of $A_1$ and $A_2$,
    \begin{equation*}
        \int^{\;\;4}_{-4} f(x)\,\mathrm{d}x - 2\int^{\;\;4}_{-1} f(x) \mathrm{d}x =
    \end{equation*}
    \begin{multicols}{2}
    \begin{choices}
        \wrongchoice{$A_1$}
        \wrongchoice{$A_1 - A_2$}
        \wrongchoice{$2A_1 - A_2$}
      \correctchoice{$A_1 + A_2$}
        \wrongchoice{$A_1 + 2A_2$}
    \end{choices}
    \end{multicols}
\end{question}
}

\element{calculusBC}{
\begin{question}{1997-BC-q20}
    What are all the values of $x$ for which the series $\displaystyle \sum_{n=1}^{\infty} \frac{\left(x-2\right)^n}{n3^n}$ converges?
    \begin{multicols}{2}
    \begin{choices}
        \wrongchoice{$-3 \leq x \leq 3$}
        \wrongchoice{$-3 < x < 3$}
        \wrongchoice{$-1 < x \leq 5$}
        \wrongchoice{$-1 \leq x \leq 5$}
      \correctchoice{$-1 \leq x < 5$}
    \end{choices}
    \end{multicols}
\end{question}
}

\element{calculusBC}{
\begin{question}{1997-BC-q21}
    Which of the following is equal to the area of the region inside the polar curve $r=2\cos\theta$ and outside the polar curve $r=\cos\theta$?
    \begin{multicols}{2}
    \begin{choices}
      \correctchoice{$\displaystyle 3\int^{\;\;\pi/2}_{0} \cos^2\theta\,\mathrm{d}\theta$}
        \wrongchoice{$\displaystyle 3\int^{\;\;\pi}_{0} \cos^2\theta\,\mathrm{d}\theta$}
        \wrongchoice{$\displaystyle \frac{3}{2}\int^{\;\;\pi/2}_{0} \cos^2\theta\,\mathrm{d}\theta$}
        \wrongchoice{$\displaystyle 3\int^{\;\;\pi/2}_{0} \cos\theta\,\mathrm{d}\theta$}
        \wrongchoice{$\displaystyle 3\int^{\;\;\pi}_{0} \cos\theta\,\mathrm{d}\theta$}
    \end{choices}
    \end{multicols}
\end{question}
}

\element{calculusBC}{
\begin{question}{1997-BC-q22}
    The graph of $f$ is shown in the figure below.
    \begin{center}
    \begin{tikzpicture}
        %% NOTE: TODO: 
    \end{tikzpicture}
    \end{center}
    If $g(x)=\int^{\,\,x}_{a} f(t)\,\mathrm{d}t$, for what value of $x$ does $g(x)$ have a maximum?
    \begin{multicols}{2}
    \begin{choices}
        \wrongchoice{$a$}
        \wrongchoice{$b$}
      \correctchoice{$c$}
        \wrongchoice{$d$}
        \wrongchoice{It cannot be determined from the information given.}
    \end{choices}
    \end{multicols}
\end{question}
}

\element{calculusBC}{
\begin{question}{1997-BC-q23}
    In the triangle shown below,
    \begin{center}
    \begin{tikzpicture}
        %% NOTE: TODO: tikz
    \end{tikzpicture}
    \end{center}
        if $\theta$ increases at a constant rate of 3 radians per minute,
        at what rate is $x$ increasing in units per minute when $x$ equals 3 units?
    \begin{multicols}{3}
    \begin{choices}
        \wrongchoice{$3$}
        \wrongchoice{$\dfrac{15}{4}$}
        \wrongchoice{$4$}
        \wrongchoice{$9$}
      \correctchoice{$12$}
    \end{choices}
    \end{multicols}
\end{question}
}

\element{calculusBC}{
\begin{question}{1997-BC-q24}
    The Taylor series for $\sin x$ about $x=0$ is $x - \dfrac{x^3}{3!} + \dfrac{x^5}{5!} - \ldots$
    If $f$ is a function such that $f\prime\left(x\right) = \sin\left(x^2\right)$,
        then the coefficient of $x^7$ in the Taylor series for $f(x)$ about $x=0$ is:
    \begin{multicols}{3}
    \begin{choices}
        \wrongchoice{$\dfrac{1}{7!}$}
        \wrongchoice{$\dfrac{1}{7}$}
        \wrongchoice{$0$}
      \correctchoice{$-\dfrac{1}{42}$}
        \wrongchoice{$-\dfrac{1}{7!}$}
    \end{choices}
    \end{multicols}
\end{question}
}

\element{calculusBC}{
\begin{question}{1997-BC-q25}
    The closed interval $\left[a,b\right]$ is partitioned into $n$ equal subintervals,
        each of width $\Delta x$, by the numbers $x_0,x_1,\ldots ,x_n$
        where $a=x_0<x_1<x_2<\cdots <x_{n-1}<x_n=b$.
    What is $\displaystyle \lim_{n\to\infty}\sum_{i=1}^{n} \sqrt{x_i}\Delta x$? 
    \begin{multicols}{2}
    \begin{choices}
      \correctchoice{$\dfrac{2}{3}\left(b^{3/2}-a^{3/2}\right)$}
        \wrongchoice{$b^{3/2}-a^{3/2}$}
        \wrongchoice{$\dfrac{3}{2}\left(b^{3/2}-a^{3/2}\right)$}
        \wrongchoice{$b^{1/2}-a^{1/2}$}
        \wrongchoice{$2\left(b^{1/2}-a^{1/2}\right)$}
    \end{choices}
    \end{multicols}
\end{question}
}

\element{calculusBC}{
\begin{questionmult}{1997-BC-q76}
    Which of the following sequences converge?
    \begin{multicols}{2}
    \begin{choices}
        %% TODO: double check this
        %% NOTE: ANS is D
        \wrongchoice{$\left\{\dfrac{5n}{2n-1}\right\}$}
        \wrongchoice{$\left\{\dfrac{\mathrm{e}^n}{n}\right\}$}
        \wrongchoice{$\left\{\dfrac{\mathrm{e}^n}{1+\mathrm{e}^n}\right\}$}
    \end{choices}
    \end{multicols}
\end{questionmult}
}

\element{calculusBC}{
\begin{question}{1997-BC-q77}
    When the region enclosed by the graphs of $y=x$ and $y=4x-x^2$ is revolved about the $y$-axis,
        the volume of the solid generated is given by:
    \begin{choices}
        \wrongchoice{$\displaystyle \pi \int^{\;\;3}_{0} \left(x^3-3x^2\right)\,\mathrm{d}x$}
        \wrongchoice{$\displaystyle \pi \int^{\;\;3}_{0} \left(x^2-\left(4x -x^2\right)^2\right)\,\mathrm{d}x$}
        \wrongchoice{$\displaystyle \pi \int^{\;\;3}_{0} \left(3x-x^2\right)^2\,\mathrm{d}x$}
        \wrongchoice{$\displaystyle 2\pi \int^{\;\;3}_{0} \left(x^3-3x^2\right)\,\mathrm{d}x$}
      \correctchoice{$\displaystyle 2\pi \int^{\;\;3}_{0} \left(3x^2-x^3\right)\,\mathrm{d}x$}
    \end{choices}
\end{question}
}

\element{calculusBC}{
\begin{question}{1997-BC-q78}
    $\displaystyle \lim_{h\to 0} \frac{\ln\left(\mathrm{e}+h\right)-1}{h}$ is
    \begin{choices}
      \correctchoice{$f\prime\left(\mathrm{e}\right)$, where $f(x)=\ln x$}
        \wrongchoice{$f\prime\left(\mathrm{e}\right)$, where $f(x)=\dfrac{\ln x}{x}$}
        \wrongchoice{$f\prime\left(1\right)$, where $f(x)=\ln x$}
        \wrongchoice{$f\prime\left(1\right)$, where $f(x)=\ln\left(x+\mathrm{e}\right)$}
        \wrongchoice{$f\prime\left(0\right)$, where $f(x)=\ln x$}
    \end{choices}
\end{question}
}

\element{calculusBC}{
\begin{question}{1997-BC-q79}
    The position of an object attached to aspring is given by $y\prime\left(t\right) = \frac{1}{6}\cos\left(5t\right) -\frac{1}{4}\sin\left(5t\right)$,
        where $t$ is time in seconds.
    In the first 4 seconds, how many time is the velocity of the object equal to 0?
    \begin{multicols}{2}
    \begin{choices}
        \wrongchoice{Zero}
        \wrongchoice{Three}
        \wrongchoice{Five}
      \correctchoice{Six}
        \wrongchoice{Seven}
    \end{choices}
    \end{multicols}
\end{question}
}

\element{calculusBC}{
\begin{question}{1997-BC-q80}
    Let $f$ be the function given by $f(x) = \cos\left(2x\right) +\ln\left(3x\right)$.
    What is the least value of $x$ at which the graph of $f$ changes concavity?
    \begin{multicols}{3}
    \begin{choices}
        \wrongchoice{0.56}
      \correctchoice{0.93}
        \wrongchoice{1.18}
        \wrongchoice{2.38}
        \wrongchoice{2.44}
    \end{choices}
    \end{multicols}
\end{question}
}

\element{calculusBC}{
\begin{question}{1997-BC-q81}
    Let $f$ be a continuous function on the closed interval $\left[-3,6\right]$.
    If $f(-3) = -1$ and $f(6) = 3$, then the Intermediate Value Theorem guarantees that:
    \begin{choices}
        \wrongchoice{$f(0) = 0$}
        \wrongchoice{$f\prime\left(c\right) = \dfrac{4}{9}$ for at least one $c$ between $-3$ and $6$}
        \wrongchoice{$-1\leq f(x) \leq 3$ for all $x$ between $-3$ and $6$}
      \correctchoice{$f\prime\left(c\right) = 1$ for at least one $c$ between $-3$ and $6$}
        \wrongchoice{$f\prime\left(c\right) = 0$ for at least one $c$ between $-1$ and $6$}
    \end{choices}
\end{question}
}

\element{calculusBC}{
\begin{question}{1997-BC-q82}
    If $0\leq x\leq 4$, of the following, which is the greatest value of $x$ such that
        $\displaystyle \int^{\;\;x}_{0} \left(t^2-2t\right)\,\mathrm{d}t \geq \int^{\;\;x}_{2}t\,\mathrm{d}t$?
    \begin{multicols}{3}
    \begin{choices}
        \wrongchoice{$1.35$}
      \correctchoice{$1.38$}
        \wrongchoice{$1.41$}
        \wrongchoice{$1.48$}
        \wrongchoice{$1.59$}
    \end{choices}
    \end{multicols}
\end{question}
}

\element{calculusBC}{
\begin{question}{1997-BC-q83}
    If $\dfrac{\mathrm{d}y}{\mathrm{dx}} = \left(1+\ln x\right)y$ and if $y=1$ when $x=1$, then $y=$
    \begin{multicols}{2}
    \begin{choices}
        %\wrongchoice{$\mathrm{exp}\left(\dfrac{x^2-1}{x^2}\right)$}
        \wrongchoice{$\mathrm{e}^{\tfrac{x^2-1}{x^2}}$}
        \wrongchoice{$1 + \ln x$}
        \wrongchoice{$\ln x$}
        %% NOTE: TODO: double check this option!?
        \wrongchoice{$\mathrm{e}^{2x + x\ln x - 2}$}
      \correctchoice{$\mathrm{e}^{x\ln x}$}
    \end{choices}
    \end{multicols}
\end{question}
}

\element{calculusBC}{
\begin{question}{1997-BC-q84}
    $\displaystyle \int\, x^2\sin x\,\mathrm{d}x =$
    \begin{choices}
        \wrongchoice{$-x^2 \cos x - 2x\sin x - 2\cos x + C$}
        \wrongchoice{$-x^2 \cos x + 2x\sin x - 2\cos x + C$}
      \correctchoice{$-x^2 \cos x + 2x\sin x + 2\cos x + C$}
        \wrongchoice{$-\dfrac{x^3}{3}\cos x + C$}
        \wrongchoice{$2x\cos x + C$}
    \end{choices}
\end{question}
}

\element{calculusBC}{
\begin{questionmult}{1997-BC-q85}
    Let $f$ be a twice differentiable function such that $f(1) = 2$ and $f(3)=7$.
    Which fo the following must be true for the function $f$ on the interval $1\leq x\leq 3$?
    \begin{choices}
      \correctchoice{The average rate of change of $f$ is $\dfrac{5}{2}$.}
        \wrongchoice{The average value of $f$ is $\dfrac{9}{2}$.}
      \correctchoice{The average value of $f\prime$ is $\dfrac{5}{2}$.}
    \end{choices}
\end{questionmult}
}

\element{calculusBC}{
\begin{question}{1997-BC-q86}
    $\displaystyle \int \frac{\mathrm{d}x}{\left(x-1\right)\left(x+3\right)}$
    \begin{choices}
      \correctchoice{$\dfrac{1}{4} \ln\left|\dfrac{x-1}{x+3}\right| + C$}
        \wrongchoice{$\dfrac{1}{4} \ln\left|\dfrac{x+3}{x-1}\right| + C$}
        \wrongchoice{$\dfrac{1}{2} \ln\left|\left(x-1\right)\left(x+3\right)\right| + C$}
        \wrongchoice{$\dfrac{1}{2} \ln\left|\dfrac{2x+2}{\left(x-1\right)\left(x+3\right)}\right| + C$}
        \wrongchoice{$\ln\left|\left(x-1\right)\left(x+3\right)\right| + C$}
    \end{choices}
\end{question}
}

\element{calculusBC}{
\begin{question}{1997-BC-q87}
    The base of a solid is the region in the first quadrant enclosed by the graph of $y=2-x^2$ and the coordinate axes.
    If every cross section of the solid perpendicular to the $y$-axis is a squared,
        the volume of the solid is given by:
    \begin{choices}
        \wrongchoice{$\displaystyle \pi \int^{\;\;2}_{0} \left(2-y\right)^2\,\mathrm{d}y$}
      \correctchoice{$\displaystyle \int^{\;\;2}_{0} \left(2-y\right)\,\mathrm{d}y$}
        \wrongchoice{$\displaystyle \pi \int^{\;\;\sqrt{2}}_{0} \left(2-x^2\right)^2\,\mathrm{d}x$}
        \wrongchoice{$\displaystyle \int^{\;\;\sqrt{2}}_{0} \left(2-x^2\right)^2\,\mathrm{d}x$}
        \wrongchoice{$\displaystyle \int^{\;\;\sqrt{2}}_{0} \left(2-x^2\right)\,\mathrm{d}x$}
    \end{choices}
\end{question}
}

\element{calculusBC}{
\begin{question}{1997-BC-q88}
    Let $f(x) = \int^{\,\,x^2}_{0} \sin t\,\mathrm{d} t$.
    At how many points in the closed interval $\left[0,\sqrt{\pi}\right]$ does the instantaneous rate of change of $f$ equal the average rate of change of $f$ on that interval?
    \begin{multicols}{3}
    \begin{choices}
        \wrongchoice{Zero}
        \wrongchoice{One}
      \correctchoice{Two}
        \wrongchoice{Three}
        \wrongchoice{Four}
    \end{choices}
    \end{multicols}
\end{question}
}

\element{calculusBC}{
\begin{question}{1997-BC-q89}
    If $f$ is the antiderivative of $\dfrac{x^2}{1+x^5}$ such that $f(1)=0$, then $f(4) =$
    \begin{multicols}{3}
    \begin{choices}
        \wrongchoice{$-0.012$}
        \wrongchoice{$0$}
        \wrongchoice{$0.016$}
      \correctchoice{$0.376$}
        \wrongchoice{$0.629$}
    \end{choices}
    \end{multicols}
\end{question}
}

\element{calculusBC}{
\begin{question}{1997-BC-q90}
    A force of 10 pounds is required to stretch a spring 4 inches beyond its natural length.
    Assuming Hook'es law applies,
        how much work is done in stretching the spring from its natural length to 6 inches beyond it natural length?
    \begin{multicols}{2}
    \begin{choices}
        \wrongchoice{60.0 inch-pounds}
      \correctchoice{65.0 inch-pounds}
        \wrongchoice{40.0 inch-pounds}
        \wrongchoice{15.0 inch-pounds}
        \wrongchoice{7.2 inch-pounds}
    \end{choices}
    \end{multicols}
\end{question}
}

%% 1998 AP Calculus BC: Section I (pp. 114)
%%--------------------------------------------------
\element{calculusBC}{
\begin{question}{1998-BC-q01}
    What are all values of $x$ for which the function $f$ defined by $f(x) = x^3 + 3x^2 - 9x + 7$ is increasing?
    \begin{multicols}{2}
    \begin{choices}
        \wrongchoice{$-3 < x < 1$}
        \wrongchoice{$-1 < x < 1$}
      \correctchoice{$x<-3$ or $x>1$}
        \wrongchoice{$x<-1$ or $x>3$}
        \wrongchoice{All real numbers}
    \end{choices}
    \end{multicols}
\end{question}
}

\element{calculusBC}{
\begin{question}{1998-BC-q02}
    In the $xy$-plane, the graph of the parametric equations $x=5t+2$ and $y=3t$, for $-3\leq t\leq 3$,
        is a line segment with slope:
    \begin{multicols}{3}
    \begin{choices}
      \correctchoice{$\dfrac{3}{5}$}
        \wrongchoice{$\dfrac{5}{3}$}
        \wrongchoice{$3$}
        \wrongchoice{$5$}
        \wrongchoice{$13$}
    \end{choices}
    \end{multicols}
\end{question}
}

\element{calculusBC}{
\begin{question}{1998-BC-q03}
    The slope of the line tangent to the curve $y^2 + \left(xy +1\right)^3=0$ at $(2,-1)$ is:
    \begin{multicols}{3}
    \begin{choices}
        \wrongchoice{$-\dfrac{3}{2}$}
        \wrongchoice{$-\dfrac{3}{4}$}
        \wrongchoice{$0$}
      \correctchoice{$\dfrac{3}{4}$}
        \wrongchoice{$\dfrac{3}{2}$}
    \end{choices}
    \end{multicols}
\end{question}
}

\element{calculusBC}{
\begin{question}{1998-BC-q04}
    $\displaystyle\int\,\frac{1}{x^2-6x+8}\,\mathrm{d}x = $
    \begin{choices}
      \correctchoice{$\dfrac{1}{2} \ln\left|\dfrac{x-4}{x-2}\right| + C$}
        \wrongchoice{$\dfrac{1}{2} \ln\left|\dfrac{x-2}{x-4}\right| + C$}
        \wrongchoice{$\dfrac{1}{2} \ln\left|\left(x-2\right)\left(x-4\right)\right| + C$}
        \wrongchoice{$\dfrac{1}{2} \ln\left|\left(x-4\right)\left(x+2\right)\right| + C$}
        \wrongchoice{$\ln\left|\left(x-2\right)\left(x-4\right)\right| + C$}
    \end{choices}
\end{question}
}

\element{calculusBC}{
\begin{question}{1998-BC-q05}
    If $f$ and $g$ are twice differentiable and if $h(x) = f\left(g(x)\right)$,
        the $h^n(x) = $
    \begin{choices}
      \correctchoice{$f\dprime\left(g(x)\right)\left[g\prime(x)\right]^2 + f\prime\left(g(x)\right)g\dprime(x)$}
        \wrongchoice{$f\dprime\left(g(x)\right)g\prime(x) + f\prime\left(g(x)\right)g\dprime(x)$}
        \wrongchoice{$f\dprime\left(g(x)\right)\left[g\prime(x)\right]^2$}
        \wrongchoice{$f\dprime\left(g(x)\right)g\dprime(x)$}
        \wrongchoice{$f\dprime\left(g(x)\right)$}
    \end{choices}
\end{question}
}

\element{calculusBC}{
\begin{question}{1998-BC-q06}
    The graph of $y=h(x)$ is shown below.
    \begin{center}
    \begin{tikzpicture}
        %% NOTE: TODO: pgfplots
    \end{tikzpicture}
    \end{center}
    Which of the following could be the graph of $y=h\prime(x)$?
    \begin{multicols}{2}
    \begin{choices}
        %% NOTE: ANS is E
        \wrongchoice{
            \begin{tikzpicture} 
            \end{tikzpicture} 
        }
    \end{choices}
    \end{multicols}
\end{question}
}

\element{calculusBC}{
\begin{question}{1998-BC-q07}
    $\displaystyle \int^{\;\;\mathrm{e}}_{1} \left(\frac{x^2-1}{x}\right)\,\mathrm{d}x $
    \begin{multicols}{2}
    \begin{choices}
        \wrongchoice{$\mathrm{e} -\dfrac{1}{\mathrm{e}}$}
        \wrongchoice{$\mathrm{e}^2 -\mathrm{e}$}
        \wrongchoice{$\dfrac{\mathrm{e}^2}{2} -\mathrm{e} + \dfrac{1}{2}$}
        \wrongchoice{$\mathrm{e}^2 - 2$}
      \correctchoice{$\dfrac{\mathrm{e}^2}{2} -\dfrac{3}{2}$}
    \end{choices}
    \end{multicols}
\end{question}
}

\element{calculusBC}{
\begin{question}{1998-BC-q08}
    If $\dfrac{\mathrm{d}y}{\mathrm{d}x} = \sin x \cos^2 x$ and if $y=0$ when $x=\dfrac{\pi}{2}$,
        what is the value of $y$ when $x=0$?
    \begin{multicols}{3}
    \begin{choices}
        \wrongchoice{$-1$}
      \correctchoice{$-\dfrac{1}{3}$}
        \wrongchoice{$0$}
        \wrongchoice{$\dfrac{1}{3}$}
        \wrongchoice{$1$}
    \end{choices}
    \end{multicols}
\end{question}
}

\element{calculusBC}{
\begin{question}{1998-BC-q09}
    The flow of oil, in barrels per hour, through a pipeline on July 9 is given by the graph shown below.
    \begin{center}
    \begin{tikzpicture}
        %% NOTE: TODO: tikz
    \end{tikzpicture}
    \end{center}
    Of the following, which best approximates the total number of barrels of oil that passed through the pipeline that day?
    \begin{multicols}{3}
    \begin{choices}
        \wrongchoice{500}
        \wrongchoice{600}
        \wrongchoice{\num{2400}}
      \correctchoice{\num{3000}}
        \wrongchoice{\num{4800}}
    \end{choices}
    \end{multicols}
\end{question}
}

\element{calculusBC}{
\begin{question}{1998-BC-q10}
    A particle moves on a plane curve so that at any time $t<0$ its $x$-coordinate is $t^3-t$ and its $y$-coordinate is $\left(2t-1\right)^3$.
    The acceleration vector of the particle at $t=1$ is:
    \begin{multicols}{3}
    \begin{choices}
        \wrongchoice{$\left(0,1\right)$}
        \wrongchoice{$\left(2,3\right)$}
        \wrongchoice{$\left(2,6\right)$}
        \wrongchoice{$\left(6,12\right)$}
      \correctchoice{$\left(6,24\right)$}
    \end{choices}
    \end{multicols}
\end{question}
}

\element{calculusBC}{
\begin{question}{1998-BC-q11}
    If $f$ is a linear function and $0<a<b$, then $\int^{\,\,b}_{a} f\dprime\left(x\right)\,\mathrm{d}x = $
    \begin{multicols}{3}
    \begin{choices}
      \correctchoice{$0$}
        \wrongchoice{$1$}
        \wrongchoice{$\dfrac{ab}{2}$}
        \wrongchoice{$b-a$}
        \wrongchoice{$\dfrac{b^2-a^2}{2}$}
    \end{choices}
    \end{multicols}
\end{question}
}

\element{calculusBC}{
\begin{question}{1998-BC-q12}
    If 
    \begin{math}
        F(x) = 
        \begin{cases}
            \ln x & \text{for } 0<x\leq 2 \\
            x^2 \ln x & \text{for } 2<x\leq 4, \\
        \end{cases}
    \end{math}
    then $\displaystyle \lim_{x\to 2} f(x)$ is
    \begin{multicols}{2}
    \begin{choices}
        \wrongchoice{$\ln 2$}
        \wrongchoice{$\ln 8$}
        \wrongchoice{$\ln 16$}
        \wrongchoice{$4$}
      \correctchoice{nonexistent}
    \end{choices}
    \end{multicols}
\end{question}
}

\element{calculusBC}{
\begin{question}{1998-BC-q13}
    The graph of the function $f$ shown in the figure below has a vertical tangent at the point $(2,0)$ and horizontal tangents at the points $(1,-2)$ and $(3,1)$.
    \begin{center}
    \begin{tikzpicture}
    \end{tikzpicture}
    \end{center}
    For what values of $x$, $-1<x<4$, if $f$ not differentiable?
    \begin{multicols}{2}
    \begin{choices}
        \wrongchoice{$0$ only}
      \correctchoice{$0$ and $2$ only}
        \wrongchoice{$1$ and $3$ only}
        \wrongchoice{$0$, $1$, and $3$ only}
        \wrongchoice{$0$, $1$, $2$, and $3$}
    \end{choices}
    \end{multicols}
\end{question}
}

\element{calculusBC}{
\begin{question}{1998-BC-q14}
    What is the approximation of the value of $\sin 1$ obtained by using the fifth-degree Taylor polynomial about $x=1$ for $\sin x$?
    \begin{multicols}{2}
    \begin{choices}
        \wrongchoice{$1 - \dfrac{1}{2} + \dfrac{1}{24}$}
        \wrongchoice{$1 - \dfrac{1}{2} + \dfrac{1}{4}$}
        \wrongchoice{$1 - \dfrac{1}{3} + \dfrac{1}{5}$}
        \wrongchoice{$1 - \dfrac{1}{4} + \dfrac{1}{8}$}
      \correctchoice{$1 - \dfrac{1}{6} + \dfrac{1}{120}$}
    \end{choices}
    \end{multicols}
\end{question}
}

\element{calculusBC}{
\begin{question}{1998-BC-q15}
    $\displaystyle \int x\cos x\,\mathrm{d} x =$
    \begin{choices}
        \wrongchoice{$x \sin x - \cos x + C$}
      \correctchoice{$x \sin x + \cos x + C$}
        \wrongchoice{$-x \sin x + \cos x + C$}
        \wrongchoice{$x \sin x + C$}
        \wrongchoice{$\dfrac{1}{2}x^2 \sin x + C$}
    \end{choices}
\end{question}
}

\element{calculusBC}{
\begin{question}{1998-BC-q16}
    If $f$ is the function defined by $f(x) = 3x^5 - 5x^4$,
        what are all the $x$-coordinates of points of inflection for the graph of $f$?
    \begin{multicols}{2}
    \begin{choices}
        \wrongchoice{$-1$}
        \wrongchoice{$0$}
      \correctchoice{$1$}
        \wrongchoice{$0$ and $1$}
        \wrongchoice{$-1$, $0$ and $1$}
    \end{choices}
    \end{multicols}
\end{question}
}

\element{calculusBC}{
\begin{question}{1998-BC-q17}
    The graph of a twice-differentiable function $f$ is shown in the figure below.
    \begin{center}
    \begin{tikzpicture}
    \end{tikzpicture}
    \end{center}
    Which of the following is true?
    \begin{choices}
        \wrongchoice{$f(1) < f\prime(1) < f\dprime(1)$}
        \wrongchoice{$f(1) < f\dprime(1) < f\prime(1)$}
        \wrongchoice{$f\prime(1) < f(1) < f\dprime(1)$}
      \correctchoice{$f\dprime(1) < f(1) < f\prime(1)$}
        \wrongchoice{$f\dprime(1) < f\prime(1) < f(1)$}
    \end{choices}
\end{question}
}

\element{calculusBC}{
\begin{questionmult}{1998-BC-q18}
    Which of the following series converge?
    \begin{multicols}{2}
    \begin{choices}
        \wrongchoice{$\displaystyle \sum_{n=1}^{\infty} \frac{n}{n+2}$}
      \correctchoice{$\displaystyle \sum_{n=1}^{\infty} \frac{\cos\left(n\pi\right)}{n}$}
        \wrongchoice{$\displaystyle \sum_{n=1}^{\infty} \frac{1}{n}$}
    \end{choices}
    \end{multicols}
\end{questionmult}
}

\element{calculusBC}{
\begin{question}{1998-BC-q19}
    The area of the region inside the polar curve $r=4\sin\theta$ and outside the polar curve $r=2$ is given by:
    \begin{choices}
        \wrongchoice{$\displaystyle \dfrac{1}{2} \int^{\;\;\pi}_{0} \left(4\sin\theta - 2\right)^2\,\mathrm{d}\theta$}
        \wrongchoice{$\displaystyle \dfrac{1}{2} \int^{\;\;3\pi/4}_{\pi/4} \left(4\sin\theta - 2\right)^2\,\mathrm{d}\theta$}
        \wrongchoice{$\displaystyle \dfrac{1}{2} \int^{\;\;5\pi/6}_{\pi/6} \left(4\sin\theta - 2\right)^2\,\mathrm{d}\theta$}
      \correctchoice{$\displaystyle \dfrac{1}{2} \int^{\;\;5\pi/6}_{\pi/6} \left(16\sin\theta - 4\right)^2\,\mathrm{d}\theta$}
        \wrongchoice{$\displaystyle \dfrac{1}{2} \int^{\;\;\pi}_{0} \left(16\sin\theta - 4\right)^2\,\mathrm{d}\theta$}
    \end{choices}
\end{question}
}

\element{calculusBC}{
\begin{question}{1998-BC-q20}
    When $x=8$, the rate at which $\sqrt[3]{x}$ is increasing is $\tfrac{1}{k}$ times the rate at which $x$ is increasing.
    What is the value of $k$?
    \begin{multicols}{3}
    \begin{choices}
        \wrongchoice{$3$}
        \wrongchoice{$4$}
        \wrongchoice{$6$}
        \wrongchoice{$8$}
      \correctchoice{$12$}
    \end{choices}
    \end{multicols}
\end{question}
}

\element{calculusBC}{
\begin{question}{1998-BC-q21}
    The length of the path described by the parametric equations $x=\dfrac{1}{3}t^3$ and $y=\dfrac{1}{2}t^2$, where $0\leq t\leq 1$, is given by:
    \begin{multicols}{2}
    \begin{choices}
        \wrongchoice{$\displaystyle \int^{\;\;1}_{0} \sqrt{t^2+1}\,\mathrm{d}t$}
        \wrongchoice{$\displaystyle \int^{\;\;1}_{0} \sqrt{t^2+t}\,\mathrm{d}t$}
      \correctchoice{$\displaystyle \int^{\;\;1}_{0} \sqrt{t^4+t^2}\,\mathrm{d}t$}
        \wrongchoice{$\displaystyle \dfrac{1}{2} \int^{\;\;1}_{0} \sqrt{4+t^4}\,\mathrm{d}t$}
        \wrongchoice{$\displaystyle \dfrac{1}{6} \int^{\;\;1}_{0} t^2 \sqrt{4t^2+9}\,\mathrm{d}t$}
    \end{choices}
    \end{multicols}
\end{question}
}

\element{calculusBC}{
\begin{question}{1998-BC-q22}
    If $\displaystyle \lim_{b\to\infty} \int^{\;\;b}_{1} \dfrac{\mathrm{d}x}{x^p}$ is finite,
        then which of the following be true?
    \begin{multicols}{2}
    \begin{choices}
      \correctchoice{$\displaystyle \sum_{n=1}^{\infty} \frac{1}{n^p}$ converges}
        \wrongchoice{$\displaystyle \sum_{n=1}^{\infty} \frac{1}{n^p}$ diverges}
        \wrongchoice{$\displaystyle \sum_{n=1}^{\infty} \frac{1}{n^{p-2}}$ converges}
        \wrongchoice{$\displaystyle \sum_{n=1}^{\infty} \frac{1}{n^{p-1}}$ converges}
        \wrongchoice{$\displaystyle \sum_{n=1}^{\infty} \frac{1}{n^{p+1}}$ diverges}
    \end{choices}
    \end{multicols}
\end{question}
}

\element{calculusBC}{
\begin{question}{1998-BC-q23}
    Let $f$ be a function defined and continuous on the closed interval $\left[a,b\right]$.
    If $f$ has a relative maximum at $c$ and $a<c<b$,
        which of the following statements must be true?
    \begin{choices}
        \wrongchoice{$f\prime\left(c\right)$ exists.}
      \correctchoice{If $f\prime\left(c\right)$ exists, then $f\dprime\left(c\right)\leq0$}
      \correctchoice{If $f\dprime\left(c\right)$ exists, then $f\dprime\left(c\right)\leq0$}
    \end{choices}
\end{question}
}

\element{calculusBC}{
\begin{question}{1998-BC-q24}
    Shown above is a slope field for which of the following differential equations?
    \begin{center}
    \begin{tikzpicture}
        %% NOTE: TODO: tikz
    \end{tikzpicture}
    \end{center}
    \begin{multicols}{2}
    \begin{choices}
        \wrongchoice{$\dfrac{\mathrm{d}y}{\mathrm{d}x} = 1+x$}
        \wrongchoice{$\dfrac{\mathrm{d}y}{\mathrm{d}x} = x^2$}
      \correctchoice{$\dfrac{\mathrm{d}y}{\mathrm{d}x} = x+y$}
        \wrongchoice{$\dfrac{\mathrm{d}y}{\mathrm{d}x} = \dfrac{x}{y}$}
        \wrongchoice{$\dfrac{\mathrm{d}y}{\mathrm{d}x} = \ln y$}
    \end{choices}
    \end{multicols}
\end{question}
}

\element{calculusBC}{
\begin{question}{1998-BC-q25}
    $\displaystyle \int^{\;\;\infty}_{0} x^2\mathrm{e}^{-x^2}\,\mathrm{d}x$ is
    \begin{multicols}{2}
    \begin{choices}
        \wrongchoice{$-\dfrac{1}{3}$}
        \wrongchoice{$0$}
      \correctchoice{$\dfrac{1}{3}$}
        \wrongchoice{$1$}
        \wrongchoice{divergent}
    \end{choices}
    \end{multicols}
\end{question}
}

\element{calculusBC}{
\begin{question}{1998-BC-q26}
    The population $P(t)$ of a species satisfies the logistic differential equation
        $\dfrac{\mathrm{d}P}{\mathrm{d}t} = P\left(2-\dfrac{P}{5000}\right)$,
        where the initial population $P(0) = \num{3000}$ and $t$ is the time in years.
    What is $\displaystyle \lim_{t\to\infty} P(t)$?
    \begin{multicols}{3}
    \begin{choices}
        \wrongchoice{\num{2500}}
        \wrongchoice{\num{3000}}
        \wrongchoice{\num{4200}}
        \wrongchoice{\num{5000}}
      \correctchoice{\num{10000}}
    \end{choices}
    \end{multicols}
\end{question}
}

\element{calculusBC}{
\begin{question}{1998-BC-q27}
    If $\displaystyle \sum_{n=0}^{\infty} a_n x^n$ is a Taylor series that converges to $f(x)$ for all real $x$, then $f\prime(1) = $
    \begin{multicols}{3}
    \begin{choices}
        \wrongchoice{$0$}
        \wrongchoice{$a_1$}
        \wrongchoice{$\displaystyle \sum_{n=0}^{\infty} a_n$}
      \correctchoice{$\displaystyle \sum_{n=0}^{\infty} n a_n$}
        \wrongchoice{$\displaystyle \sum_{n=0}^{\infty} n a_n^{n-1}$}
    \end{choices}
    \end{multicols}
\end{question}
}

\element{calculusBC}{
\begin{question}{1998-BC-q28}
    $\displaystyle \lim_{x\to 1} \int^{\;\;x}_{1} \frac{\mathrm{e}^{t^2}}{x^2-1}\,\mathrm{d}t$ is
    \begin{multicols}{2}
    \begin{choices}
        \wrongchoice{$0$}
        \wrongchoice{$1$}
      \correctchoice{$\dfrac{\mathrm{e}}{2}$}
        \wrongchoice{$\mathrm{e}$}
        \wrongchoice{nonexistent}
    \end{choices}
    \end{multicols}
\end{question}
}

\element{calculusBC}{
\begin{question}{1998-BC-q76}
    For what integer $k$, $k>1$, will both $\displaystyle \sum_{n=1}^{\infty} \frac{\left(-1\right)^{kn}}{n}$ and $\displaystyle \sum_{n=1}^{\infty} \left(\frac{k}{4}\right)^n$ converge?
    \begin{multicols}{3}
    \begin{choices}
        \wrongchoice{$6$}
        \wrongchoice{$5$}
        \wrongchoice{$4$}
      \correctchoice{$3$}
        \wrongchoice{$2$}
    \end{choices}
    \end{multicols}
\end{question}
}

\element{calculusBC}{
\begin{question}{1998-BC-q77}
    If $f$ is a vector-valued function defined by $f(t) = \left(\mathrm{e}^{-t},\cos t\right)$, then $f\dprime\left(t\right) = $
    \begin{multicols}{2}
    \begin{choices}
        \wrongchoice{$-\mathrm{e}^{-t} + \sin t$}
        \wrongchoice{$-\mathrm{e}^{-t} - \cos t$}
        \wrongchoice{$\left(-\mathrm{e}^{-t},-\sin t\right)$}
        \wrongchoice{$\left(-\mathrm{e}^{-t},\cos t\right)$}
      \correctchoice{$\left(\mathrm{e}^{-t},-\cos t\right)$}
    \end{choices}
    \end{multicols}
\end{question}
}

\element{calculusBC}{
\begin{question}{1998-BC-q78}
    The radius of a circle is decreasing at a constant rate of 0.1 centimeter per second.
    In terms of the circumference $C$, what is the rate of change of the area of the circle,
        in squared centimeters per second?
    \begin{multicols}{2}
    \begin{choices}
        \wrongchoice{$-\left(0.2\right)\pi C$}
      \correctchoice{$-\left(0.2\right) C$}
        \wrongchoice{$-\dfrac{\left(0.1\right) C}{2\pi}$}
        \wrongchoice{$\left(0.1\right)^2 C$}
        \wrongchoice{$\left(0.1\right)^2 \pi C$}
    \end{choices}
    \end{multicols}
\end{question}
}

\element{calculusBC}{
\begin{question}{1998-BC-q79}
    Let $f$ be the function given by $f(x) = \dfrac{\left(x-1\right)\left(x^2-4\right)}{x^2-a}$.
    For what positive values of $a$ is $f$ continuous for all real numbers $x$?
    \begin{multicols}{2}
    \begin{choices}
      \correctchoice{None}
        \wrongchoice{1 only}
        \wrongchoice{2 only}
        \wrongchoice{4 only}
        \wrongchoice{1 and 4 only}
    \end{choices}
    \end{multicols}
\end{question}
}

\element{calculusBC}{
\begin{question}{1998-BC-q80}
    Let $R$ be the region enclosed by the graph of $y=1 + \ln\left(\cos^4 x\right)$,
        the $x$-axis, and the lines $x=-\dfrac{2}{3}$ and $x=\dfrac{2}{3}$.
    The closest integer approximation of the area of $R$ is:
    \begin{multicols}{3}
    \begin{choices}
        \wrongchoice{0}
      \correctchoice{1}
        \wrongchoice{2}
        \wrongchoice{3}
        \wrongchoice{4}
    \end{choices}
    \end{multicols}
\end{question}
}

\element{calculusBC}{
\begin{question}{1998-BC-q81}
    If $\dfrac{\mathrm{d}y}{\mathrm{d}x} = \sqrt{1-y^2}$, then $\dfrac{\mathrm{d}^2y}{\mathrm{d}x^2} =$
    \begin{multicols}{3}
    \begin{choices}
        \wrongchoice{$-2y$}
      \correctchoice{$-y$}
        \wrongchoice{$\dfrac{-y}{\sqrt{1-y^2}}$}
        \wrongchoice{$y$}
        \wrongchoice{$\dfrac{1}{2}$}
    \end{choices}
    \end{multicols}
\end{question}
}

\element{calculusBC}{
\begin{question}{1998-BC-q82}
    If $f(x) = g(x) + 7$ for $3\leq x\leq 5$, then $\displaystyle \int^{\;\;5}_{3} \left[f(x)+g(x)\right]\,\mathrm{d}x = $
    \begin{multicols}{2}
    \begin{choices}
        \wrongchoice{$\displaystyle 2 \int^{\;\;5}_{3} g(x) \mathrm{d}x + 7$}
      \correctchoice{$\displaystyle 2 \int^{\;\;5}_{3} g(x) \mathrm{d}x + 14y$}
        \wrongchoice{$\displaystyle 2 \int^{\;\;5}_{3} g(x) \mathrm{d}x + 28$}
        \wrongchoice{$\displaystyle \int^{\;\;5}_{3} g(x) \mathrm{d}x + 7$}
        \wrongchoice{$\displaystyle \int^{\;\;5}_{3} g(x) \mathrm{d}x + 14$}
    \end{choices}
    \end{multicols}
\end{question}
}

\element{calculusBC}{
\begin{question}{1998-BC-q83}
    The Taylor series for $\ln x$, centered at $x=1$, is $\displaystyle \sum_{n=1}^{\infty} \left(-1\right)^{n+1} \frac{\left(x-1\right)^n}{n}$.
    Let $f$ be the function given by the sum of the first three nonzero terms of this series.
    The maximum value of $\left| \ln x -f(x) \right|$ for $0.3 \leq x \leq 1.7$ is
    \begin{multicols}{3}
    \begin{choices}
        \wrongchoice{$0.030$}
        \wrongchoice{$0.039$}
      \correctchoice{$0.145$}
        \wrongchoice{$0.153$}
        \wrongchoice{$0.529$}
    \end{choices}
    \end{multicols}
\end{question}
}

\element{calculusBC}{
\begin{question}{1998-BC-q84}
    What are the values of $x$ for which the series $\displaystyle \sum_{n=1}^{\infty} \frac{\left(x+2\right)^n}{\sqrt{n}}$ converges?
    \begin{multicols}{2}
    \begin{choices}
        \wrongchoice{$-3 < x < -1$}
      \correctchoice{$-3 \leq x < -1$}
        \wrongchoice{$-3 \leq x \leq -1$}
        \wrongchoice{$-1 \leq x < -1$}
        \wrongchoice{$-1 \leq x \leq -1$}
    \end{choices}
    \end{multicols}
\end{question}
}

\element{calculusBC}{
\begin{question}{1998-BC-q85}
    The function $f$ is continuous on the closed interval $\left[2,8\right]$ and has values that are given in the table below.
    \begin{center}
    \begin{tabular}{ccccc}
        $x$    & 2  & 5  & 7  & 8 \\
        $f(x)$ & 10 & 30 & 40 & 20 \\
    \end{tabular}
    \end{center}
    Using the subintervals $\left[2,5\right]$, $\left[5,7\right]$, and $\left[7,8\right]$, what is the trapezoidal approximation of $\int^{\,\,8}_{2}f(x)\,\mathrm{d}x$?
    \begin{multicols}{3}
    \begin{choices}
        \wrongchoice{$110$}
        \wrongchoice{$130$}
      \correctchoice{$160$}
        \wrongchoice{$190$}
        \wrongchoice{$210$}
    \end{choices}
    \end{multicols}
\end{question}
}

\begin{comment}
\element{calculusBC}{
\begin{question}{1998-BC-q86}
    The base of a solid is a region in the first quadrant bounded by the $x$-axis,
        the $y$-axis, and the line $x+2y=8$, as shown below.
    \begin{center}
    \begin{tikzpicture}
        %% NOTE: TODSOL
    \end{tikzpicture}
    \end{center}
    If the cross sections of the solid perpendicular to the $x$-axis are semicircules, what is the volume of the solid?
    \begin{multicols}{3}
    \begin{choices}
        \wrongchoice{$12.566$}
        \wrongchoice{$14.661$}
      \correctchoice{$16.755$}
        \wrongchoice{$67.021$}
        \wrongchoice{$134.041$}
    \end{choices}
    \end{multicols}
\end{question}
}
\end{comment}

\element{calculusBC}{
\begin{question}{1998-BC-q87}
    Which of the following is an equation of the line tangent to the graph of $f(x) = x^4 + 2x^2$ at the point where $f\prime\left(x\right) = 1$?
    \begin{multicols}{2}
    \begin{choices}
        \wrongchoice{$y = 8x - 5$}
        \wrongchoice{$y = x + 7$}
        \wrongchoice{$y = x + 0.763$}
      \correctchoice{$y = x - 0.122$}
        \wrongchoice{$y = x - 2.146$}
    \end{choices}
    \end{multicols}
\end{question}
}

\begin{comment}
\element{calculusBC}{
\begin{question}{1998-BC-q88}
    Let $g(x) = \int^{\,\,x}_{a} f(t) \mathrm{d}t$, where $a\leq x\leq b$.
    The figure below shows the graph of $g$ on $\left[a,b\right]$.
    \begin{center}
    \begin{tikzpicture}
        %% NOTE: TODSOL
    \end{tikzpicture}
    \end{center}
    Which of the following could be the graph of $f$ on $\left[a,b\right]$?
    \begin{multicols}{2}
    \begin{choices}
        %% NOTE: ANS is C
        \wrongchoice{
            \begin{tikzpicture}
                %% NOTE: TODSOL
            \end{tikzpicture}
        }
    \end{choices}
    \end{multicols}
\end{question}
}
\end{comment}

\element{calculusBC}{
\begin{question}{1998-BC-q89}
    The graph of the function represented by the Maclaurin series
    \begin{equation*}
        1 - x + \frac{x^2}{2!} - \frac{x^3}{3!} + \ldots + \frac{\left(-1\right)^n x^n}{n!} + \ldots
    \end{equation*}
    intersects the graph of $y=x^3$ at $x=$
    \begin{multicols}{3}
    \begin{choices}
      \correctchoice{$0.773$}
        \wrongchoice{$0.865$}
        \wrongchoice{$0.929$}
        \wrongchoice{$1.000$}
        \wrongchoice{$1.857$}
    \end{choices}
    \end{multicols}
\end{question}
}

\begin{comment}
\element{calculusBC}{
\begin{question}{1998-BC-q90}
    A particle starts from rest at the point $\left(2,0\right)$ and moves along the $x$-axis with a constant positive acceleration for time $t\geq 0$.
    Which of the following could be the graph of the distance $s(t)$ of the particle from the origin as a function of time $t$?
    \begin{multicols}{2}
    \begin{choices}
        %% NOTE: ANS is A
        \wrongchoice{
            \begin{tikzpicture}
            \end{tikzpicture}
        }
    \end{choices}
    \end{multicols}
\end{question}
}
\end{comment}

\element{calculusBC}{
\begin{question}{1998-BC-q91}
    The data for the accleration $a(t)$ of a car from 0 to 6 seconds are given in the table above.
    If the velocity at $t=0$ is 11 feet per second,
        the approximate value of the velocity at $t=6$, computed using a left-hand Riemann sum with three subintervals of equal length, is:
    \begin{multicols}{3}
    \begin{choices}
        \wrongchoice{26 ft/sec}
        \wrongchoice{30 ft/sec}
        \wrongchoice{37 ft/sec}
        \wrongchoice{39 ft/sec}
      \correctchoice{41 ft/sec}
    \end{choices}
    \end{multicols}
\end{question}
}

\element{calculusBC}{
\begin{question}{1998-BC-q92}
    Let $f$ be the function given by $f(x) = x^2 - 2x + 3$.
    The tangent line to the graph of $f$ at $x=2$ is used to approximate values of $f(x)$.
    Which of the following is the greatest value of $x$ for which the error resulting from this tangent line approximation is less than $0.5$?
    \begin{multicols}{3}
    \begin{choices}
        \wrongchoice{$2.4$}
        \wrongchoice{$2.5$}
        \wrongchoice{$2.6$}
      \correctchoice{$2.7$}
        \wrongchoice{$2.8$}
    \end{choices}
    \end{multicols}
\end{question}
}


\endinput


