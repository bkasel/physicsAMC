

%% AAPT Physics Bowl Exams Questions
%%----------------------------------------


%% This section has 58 problems


%% PhysicsBowl 2015
%%----------------------------------------
\element{aapt}{ %% Bowl-A1
\begin{question}{bowl-2015-q02}
    A box uniformly slides \SI{7.50}{\meter} to rest across
        a flat surface in a time of \SI{12.0}{\second}.
    What was the initial speed of the box when it started its slide?
    \begin{multicols}{2}
    \begin{choices}
        \wrongchoice{\SI{0.313}{\meter\per\second}}
        \wrongchoice{\SI{0.625}{\meter\per\second}}
      \correctchoice{\SI{1.25}{\meter\per\second}}
        \wrongchoice{\SI{2.50}{\meter\per\second}}
        \wrongchoice{\SI{5.00}{\meter\per\second}}
    \end{choices}
    \end{multicols}
\end{question}
}

\element{aapt}{ %% Bowl-A1
\begin{question}{bowl-2015-q06}
    A car travels at \SI{20.0}{\mile\per\hour}.
    Which one of the following choices best represents the speed
        of the car in SI units of meter per second (\si{\meter\per\second})?
    \begin{multicols}{3}
    \begin{choices}
        \wrongchoice{\SI{533}{\meter\per\second}}
        \wrongchoice{\SI{45.0}{\meter\per\second}}
        \wrongchoice{\SI{20.0}{\meter\per\second}}
      \correctchoice{\SI{8.9}{\meter\per\second}}
        \wrongchoice{\SI{0.75}{\meter\per\second}}
    \end{choices}
    \end{multicols}
\end{question}
}

\element{aapt}{ %% Bowl-A1
\begin{question}{bowl-2015-q09}
    Two cars are moving to the right on a horizontal track,
        each with constant acceleration.
    At an instant of time, the information about the cars is shown:
    \begin{description}
        \item[Car \#1:] position = \SI{125.0}{\meter};
            velocity = \SI{13.0}{\meter\per\second};
            constant acceleration = \SI{1.5}{\meter\per\second\squared}
        \item[Car \#2:] position = \SI{80.0}{\meter};
            velocity = \SI{9.30}{\meter\per\second};
            constant acceleration = \SI{5.5}{\meter\per\second\squared}
    \end{description}
    During the next \SI{1.0}{\second} of motion,
        which one of the following choices best represents
        what happens to the distance between the cars?
    \begin{choices}
        \wrongchoice{It decreases during the entire \SI{1.0}{\second} of motion.}
        \wrongchoice{It increases during the entire \SI{1.0}{\second} of motion.}
      \correctchoice{It initially increases and then decreases resulting in a greater distance between the cars after \SI{1.0}{\second}.}
        \wrongchoice{It initially increases and then decreases resulting in a smaller distance between the cars after \SI{1.0}{\second}.}
        \wrongchoice{It initially increases and then decreases resulting in the same distance between the cars after \SI{1.0}{\second}.}
    \end{choices}
\end{question}
}

\element{aapt}{ %% Bowl-A1
\begin{question}{bowl-2015-q17}
    An object starts at the origin and its velocity along a line vs. time is graphed.
    \begin{center}
    \begin{tikzpicture}
        \begin{axis}[
            axis y line=left,
            axis x line=middle,
            axis line style={->},
            xlabel={time},
            x unit=\si{\second},
            xtick={0,2,4,6,8,10},
            minor x tick num=1,
            ylabel={velocity},
            y unit=\si{\meter\per\second},
            ytick={-5,0,5},
            minor y tick num=4,
            grid=major,
            xmin=0,xmax=10.2,
            ymin=-5.5,ymax=5.5,
            width=0.95\columnwidth,
            height=0.50\columnwidth,
        ]
        \addplot[line width=1pt,mark=\empty] plot coordinates { (0,0) (3,4) (5,4) (9,-4) (10,-4) };
        \end{axis}
    \end{tikzpicture}
    \end{center}
    Which one of the following choices best gives the proper interval(s) of time for which the object is moving away from the origin?
    \begin{choices}
        \wrongchoice{Only for times $\SI{0}{\second}<t<\SI{3}{\second}$}
        \wrongchoice{Only for times $\SI{0}{\second}<t<\SI{5}{\second}$}
        \wrongchoice{Only for times $\SI{3}{\second}<t<\SI{5}{\second}$}
      \correctchoice{Only for times $\SI{0}{\second}<t<\SI{7}{\second}$}
        \wrongchoice{For times $\SI{0}{\second}<t<\SI{3}{\second}$ and $\SI{5}{\second}<t<\SI{9}{\second}$}
    \end{choices}
\end{question}
}

\element{aapt}{ %% Bowl-A1
\begin{question}{bowl-2015-q32}
    An object moving along a line completes a \SI{20.0}{\second}
        trip with an average speed of \SI{10.0}{\meter\per\second} in two stages.
    During stage 1, the object moves with a constant velocity of
        \SI{6.0}{\meter\per\second} to the right for \SI{12.0}{\second}.
    What constant magnitude acceleration directed to the left
        must the object have during the \SI{8.0}{\second} of stage 2?
    \begin{multicols}{2}
    \begin{choices}
        \wrongchoice{\SI{2.5}{\meter\per\second\squared}}
        \wrongchoice{\SI{2.7}{\meter\per\second\squared}}
        \wrongchoice{\SI{4.0}{\meter\per\second\squared}}
      \correctchoice{\SI{5.3}{\meter\per\second\squared}}
        \wrongchoice{\SI{6.3}{\meter\per\second\squared}}
    \end{choices}
    \end{multicols}
\end{question}
}

\newcommand{\BowlTwentyFifteenQFortyOne}{
\begin{tikzpicture}
    \begin{axis}[
        axis y line=left,
        axis x line=bottom,
        axis line style={->},
        xlabel={time},
        xtick={0,5,10},
        xticklabels={$0$,$T$,$2T$},
        ylabel={position},
        ytick=\empty,
        grid=major,
        xmin=0,xmax=11,
        ymin=0,ymax=20,
        width=0.95\columnwidth,
        height=0.50\columnwidth,
        very thin,
        legend style={
            at={(0.90,0.1)},
            anchor=south,
        },
        colormap={mymap}{rgb=(0,0,0); rgb=(0,0,0)}
    ]
    \addplot[dashed,line width=1pt,domain=0:10]{4+(8*x/5)};
    \addplot[line width=1pt,black,draw=black,color=black,patch,patch type=quadratic spline] coordinates {
        (0,4) (5,12) (2,2.0) (5,12) (10,20) (7,18) };
    %\legend{Truck,Car}; node??
    \end{axis}
\end{tikzpicture}
}

\element{aapt}{ %% Bowl-A1
\begin{question}{bowl-2015-q41}
    %Questions 41--42 deal with the following information:
    A car (solid line) and a truck (dashed line) are moving on a horizontal track.
    The position vs. time graph for the two vehicles is shown.
    \begin{center}
        \BowlTwentyFifteenQFortyOne
    \end{center}
    For the entire time shown in the graph,
        which one of the following choices correctly describes the relationship
        between the average speed of the truck to that of the car?
    \begin{choices}
      \correctchoice{The truck's average speed is less than the average speed of the car.}
        \wrongchoice{The truck's average speed is the same as the average speed of the car.}
        \wrongchoice{The truck's average speed is greater than the average speed of the car.}
        \wrongchoice{The truck's average speed is positive while the car's average speed is negative but of the same magnitude.}
        \wrongchoice{A relationship cannot be determined without more information.}
    \end{choices}
\end{question}
}

\element{aapt}{ %% Bowl-A1
\begin{question}{bowl-2015-q42}
    %Questions 41--42 deal with the following information:
    A car (solid line) and a truck (dashed line) are moving on a horizontal track.
    The position vs. time graph for the two vehicles is shown.
    \begin{center}
        \BowlTwentyFifteenQFortyOne
    \end{center}
    Which one of the following choices best describes the instants of time, $t$,
        at which the car and truck travel with the same speed?
    \begin{choices}
        \wrongchoice{Only at times $t=0$, $t=T$ and $t=2T$.}
        \wrongchoice{At one instant during the interval $0 < t < T$ and at one instant during the interval $T < t < 2T$.}
      \correctchoice{At two instants during the interval $0 < t < T$ and at one instant during the interval $T < t < 2T$.}
        \wrongchoice{At one instant during the interval $0 < t < T$ and at two instants during the interval $T < t < 2T$.}
        \wrongchoice{At two instants during the interval $0 < t < T$ and at two instants during the interval $T < t < 2T$.}
    \end{choices}
\end{question}
}


%% PhysicsBowl 2014
%%----------------------------------------
\element{aapt}{ %% Bowl-A1
\begin{question}{bowl-2014-q07}
    Starting from rest,
        a cart uniformly accelerates to a speed of \SI{7.60}{\meter\per\second} in a time of \SI{3.00}{\second}.
    Through what distance does the cart move in this time?
    \begin{multicols}{3}
    \begin{choices}
        \wrongchoice{\SI{5.7}{\meter}}
        \wrongchoice{\SI{8.1}{\meter}}
      \correctchoice{\SI{11.4}{\meter}}
        \wrongchoice{\SI{16.1}{\meter}}
        \wrongchoice{\SI{22.8}{\meter}}
    \end{choices}
    \end{multicols}
\end{question}
}

\element{aapt}{ %% Bowl-A1
\begin{question}{bowl-2014-q16}
    A toy car initially moves to the right at \SI{60.0}{\centi\meter\per\second}.
    Five seconds later, the car is moving at \SI{40.0}{\centi\meter\per\second} to the left.
    The total displacement of the car during this time is \SI{10.0}{\centi\meter} to the left of where it started.
    %% begin question
    Which one of the following choices best represents the magnitude of the average velocity of the car during the five second motion?
    \begin{multicols}{2}
    \begin{choices}
        \wrongchoice{\SI{50.0}{\centi\meter\per\second}}
        \wrongchoice{\SI{10.0}{\centi\meter\per\second}}
        \wrongchoice{\SI{4.0}{\centi\meter\per\second}}
      \correctchoice{\SI{2.0}{\centi\meter\per\second}}
        \wrongchoice{\SI{0.40}{\centi\meter\per\second}}
    \end{choices}
    \end{multicols}
\end{question}
}

\element{aapt}{ %% Bowl-A1
\begin{question}{bowl-2014-q17}
    A toy car initially moves to the right at \SI{60.0}{\centi\meter\per\second}.
    Five seconds later, the car is moving at \SI{40.0}{\centi\meter\per\second} to the left.
    The total displacement of the car during this time is \SI{10.0}{\centi\meter} to the left of where it started.
    %% begin question
    Which one of the following choices best represents the magnitude of the average acceleration of the car during the five second motion?
    \begin{multicols}{2}
    \begin{choices}
      \correctchoice{\SI{20.0}{\centi\meter\per\second\squared}}
        \wrongchoice{\SI{4.0}{\centi\meter\per\second\squared}}
        \wrongchoice{\SI{2.0}{\centi\meter\per\second\squared}}
        \wrongchoice{\SI{0.80}{\centi\meter\per\second\squared}}
        \wrongchoice{\SI{0.40}{\centi\meter\per\second\squared}}
    \end{choices}
    \end{multicols}
\end{question}
}


\element{aapt}{ %% Bowl-A1
\begin{question}{bowl-2014-q33}
    An object moves with constant acceleration starting with velocity $v_0=\SI{5.00}{\meter\per\second}$ and ending with a velocity of $v=-\SI{1.00}{\meter\per\second}$ in a time of \SI{3.00}{\second}.
    For this motion, what is the average speed associated with the object.
    \begin{multicols}{3}
    \begin{choices}
        \wrongchoice{\SI{2.00}{\meter\per\second}}
      \correctchoice{\SI{2.17}{\meter\per\second}}
        \wrongchoice{\SI{2.50}{\meter\per\second}}
        \wrongchoice{\SI{2.83}{\meter\per\second}}
        \wrongchoice{\SI{3.00}{\meter\per\second}}
    \end{choices}
    \end{multicols}
\end{question}
}


%% PhysicsBowl 2013
%%----------------------------------------
\element{aapt}{ %% Bowl-A1
\begin{question}{bowl-2013-q17}
    A position vs. time graph of a particle moving along a horizontal axis is shown.
    \begin{center}
    \begin{tikzpicture}
        \begin{axis}[
            axis y line=left,
            axis x line=middle,
            axis line style={->},
            xlabel={time},
            x unit=\si{\second},
            xtick={0,2,4,6,8,10},
            minor xtick={1,3,5,7,9},
            ylabel={position},
            y unit=\si{\meter},
            ytick={-10,0,10},
            minor ytick={-8,-6,-4,-2,2,4,6,8},
            grid=both,
            xmin=0,xmax=10.5,
            ymin=-10.5,ymax=10.5,
            width=0.95\columnwidth,
            height=0.618\columnwidth,
        ]
        \addplot[line width=1pt,mark=\empty] plot coordinates { (0,8) (3,8) (5,0) (6,0) (8,-8) (9,-8) (10,10) };
        \end{axis}
    \end{tikzpicture}
    \end{center}
    What is the total distance traveled by the particle from $t=\SI{0}{\second}$ to $t=\SI{10}{\second}$?
    \begin{multicols}{3}
    \begin{choices}
        \wrongchoice{\SI{2}{\meter}}
        \wrongchoice{\SI{18}{\meter}}
        \wrongchoice{\SI{26}{\meter}}
      \correctchoice{\SI{34}{\meter}}
        \wrongchoice{\SI{42}{\meter}}
    \end{choices}
    \end{multicols}
\end{question}
}

\element{aapt}{ %% Bowl-A1
\begin{question}{bowl-2013-q26}
    An object moving only to the right completes a \SI{20.0}{\second} trip in two stages, I and II.
    The average speed of the entire \SI{20.0}{\second} trip is \SI{10.0}{\meter\per\second}.
    For state I, the object moves with a constant velocity of \SI{6.0}{\meter\per\second} for \SI{12.0}{\second}.
    What constant acceleration must the object have during the \SI{8.0}{\second} of stage II?
    \begin{multicols}{2}
    \begin{choices}
        \wrongchoice{\SI{2.25}{\meter\per\second\squared}}
      \correctchoice{\SI{2.50}{\meter\per\second\squared}}
        \wrongchoice{\SI{4.00}{\meter\per\second\squared}}
        \wrongchoice{\SI{6.25}{\meter\per\second\squared}}
        \wrongchoice{\SI{8.50}{\meter\per\second\squared}}
    \end{choices}
    \end{multicols}
\end{question}
}

\element{aapt}{ %% Bowl-A1
\begin{question}{bowl-2013-q32}
    An acceleration vs. time graph for an object moving along a line is shown.
    \begin{center}
    \begin{tikzpicture}
        \begin{axis}[
            axis y line=left,
            axis x line=middle,
            axis line style={->},
            xlabel={time},
            x unit=\si{\second},
            xtick={0,2,4,6,8,10},
            minor xtick={1,3,5,7,9},
            ylabel={acceleration},
            y unit=\si{\meter\per\second\squared},
            ytick={-10,0,10},
            minor ytick={-8,-6,-4,-2,2,4,6,8},
            grid=both,
            xmin=0,xmax=10.25,
            ymin=-10.5,ymax=10.5,
            width=0.95\columnwidth,
            height=0.618\columnwidth,
        ]
        \addplot[line width=1pt,domain=0:10]{8*sin(1.26*deg(x))};
        \end{axis}
    \end{tikzpicture}
    \end{center}
    The object starts from rest at time $t=\SI{0}{\second}$.
    At what time(s) does the object attain a maximum displacement from its starting position?
    \begin{choices}
        \wrongchoice{At times $t=\SI{2.5}{\second}$ and $t=\SI{7.5}{\second}$ only}
        \wrongchoice{At times $t=\SI{5.0}{\second}$ and $t=\SI{10}{\second}$ only}
        \wrongchoice{At times $t=\SI{1.25}{\second}$, $t=\SI{3.75}{\second}$,
            $t=\SI{6.25}{\second}$, and $t=\SI{8.75}{\second}$ only}
        \wrongchoice{At times $t=\SI{2.5}{\second}$, $t=\SI{5.0}{\second}$,
            $t=\SI{7.5}{\second}$, and $t=\SI{10}{\second}$ only}
      \correctchoice{At time $t=\SI{10}{\second}$ only}
    \end{choices}
\end{question}
}


%% PhysicsBowl 2012
%%----------------------------------------
\element{aapt}{ %% Bowl-A1
\begin{question}{bowl-2012-q08}
    An object of mass \SI{5.00}{\kilo\gram} moves only to the right along the $+x$-axis.
    During some time interval, the object's speed increased from \SI{4.00}{\meter\per\second} to \SI{8.00}{\meter\per\second} with a constant acceleration of \SI{2.00}{\meter\per\second\squared}.
    Through what distance does the object move during the time interval of the acceleration?
    \begin{multicols}{3}
    \begin{choices}
        \wrongchoice{\SI{2.00}{\meter}}
        \wrongchoice{\SI{4.00}{\meter}}
        \wrongchoice{\SI{8.00}{\meter}}
      \correctchoice{\SI{12.0}{\meter}}
        \wrongchoice{\SI{24.0}{\meter}}
    \end{choices}
    \end{multicols}
\end{question}
}

\element{aapt}{ %% Bowl-A1
\begin{question}{bowl-2012-q17}
    A mass moves according to the graph of position as a function of time shown below.
    \begin{center}
    \begin{tikzpicture}
        \begin{axis}[
            axis y line=left,
            axis x line=middle,
            axis line style={->},
            xlabel={time},
            x unit=\si{\second},
            xtick={0,2,4,6,8,10},
            minor xtick={1,3,5,7,9},
            ylabel={position},
            y unit=\si{\meter},
            ytick={-10,0,10},
            minor ytick={-8,-6,-4,-2,2,4,6,8},
            grid=both,
            xmin=0,xmax=10.25,
            ymin=-10.5,ymax=10.5,
            width=0.95\columnwidth,
            height=0.618\columnwidth,
        ]
        \addplot[line width=1pt,domain=0:10]{8*sin(6.28*deg(x)/10)};
        \end{axis}
    \end{tikzpicture}
    \end{center}
    Which one of the following choices correctly represents the time or time interval for which the instantaneous velocity of the mass is considered always to be negative?
    Let $t$ represent time.
    \begin{choices}
        \wrongchoice{$t=\SI{0.0}{\second}$, $t=\SI{5.0}{\second}$ and $t=\SI{10.0}{\second}$}
        \wrongchoice{$\SI{0.0}{\second}<t<\SI{2.5}{\second}$}
      \correctchoice{$\SI{2.5}{\second}<t<\SI{7.5}{\second}$}
        \wrongchoice{$\SI{5.0}{\second}<t<\SI{10.0}{\second}$}
        \wrongchoice{$\SI{2.5}{\second}<t<\SI{10.0}{\second}$}
    \end{choices}
\end{question}
}

\element{aapt}{ %% Bowl-A1
\begin{question}{bowl-2012-q29}
    A vehicle completes one lap around a circular track
        at an average speed of \SI{50}{\meter\per\second}
        and then completes a second lap at an average speed of $V$.
    The average speed of the vehicle for the completion
        of both laps was \SI{80}{\meter\per\second}.
    What was the average speed $V$ of the second lap?
    \begin{multicols}{3}
    \begin{choices}
        \wrongchoice{\SI{100}{\meter\per\second}}
        \wrongchoice{\SI{110}{\meter\per\second}}
        \wrongchoice{\SI{125}{\meter\per\second}}
        \wrongchoice{\SI{150}{\meter\per\second}}
      \correctchoice{\SI{200}{\meter\per\second}}
    \end{choices}
    \end{multicols}
\end{question}
}


%% PhysicsBowl 2011
%%----------------------------------------
\element{aapt}{ %% Bowl-A1
\begin{question}{bowl-2011-q07}
    A ball of mass $m=\SI{0.100}{\kilo\gram}$ is launched
        straight upward so that it rises to a maximum height
        of \SI{12.0}{\meter} above the launch point.
    Ignore air resistance.
    %% Start question
    Approximately how much time does it take the ball to reach
        the maximum height from its launch?
    \begin{multicols}{3}
    \begin{choices}
        \wrongchoice{\SI{0.65}{\second}}
        \wrongchoice{\SI{1.00}{\second}}
        \wrongchoice{\SI{1.20}{\second}}
      \correctchoice{\SI{1.55}{\second}}
        \wrongchoice{\SI{2.40}{\second}}
    \end{choices}
    \end{multicols}
\end{question}
}

\element{aapt}{ %% Bowl-A1
\begin{question}{bowl-2011-q08}
    An object moves along a horizontal line with increasing speed.
    Which one of the following choices could represent the signs of the velocity and of the acceleration for the object to achieve this motion?
    \begin{center}
    \begin{tabu}{cX[c]X[c]}
        \toprule
        \makebox[1.5em][c]{\textnumero}
            & Velocity & Acceleration \\
        \bottomrule
    \end{tabu}
    \end{center}
    \begin{choices}
        \wrongchoice{\begin{tabu}{X[c]X[c]} Zero & Zero \\ \end{tabu}}
        \wrongchoice{\begin{tabu}{X[c]X[c]} Positive & Zero \\ \end{tabu}}
        \wrongchoice{\begin{tabu}{X[c]X[c]} Positive & Negative \\ \end{tabu}}
        \wrongchoice{\begin{tabu}{X[c]X[c]} Negative & Positive \\ \end{tabu}}
      \correctchoice{\begin{tabu}{X[c]X[c]} Negative & Negative \\ \end{tabu}}
    \end{choices}
\end{question}
}

\element{aapt}{ %% Bowl-A1
\begin{question}{bowl-2011-q09}
    Which one of the following choices best represents the speed of \SI{60.0}{\mile\per\hour} rewritten in units of \si{\kilo\meter\per\day}?
    \begin{multicols}{2}
    \begin{choices}
      \correctchoice{\SI{2300}{\kilo\meter\per\day}}
        \wrongchoice{\SI{1440}{\kilo\meter\per\day}}
        \wrongchoice{\SI{900}{\kilo\meter\per\day}}
        \wrongchoice{\SI{4.00}{\kilo\meter\per\day}}
        \wrongchoice{\SI{1.56}{\kilo\meter\per\day}}
    \end{choices}
    \end{multicols}
\end{question}
}

\element{aapt}{ %% Bowl-A1
\begin{question}{bowl-2011-q14}
    A car makes a trip in two parts.
    \begin{description}[itemsep=0pt]
        \item[Part 1:] It travels a distance of \SI{800}{\meter} at a constant speed of \SI{4.0}{\meter\per\second}.
        \item[Part 2:] It travels a distance of \SI{1200}{\meter} at a constant speed of \SI{20.0}{\meter\per\second}.
    \end{description}
    What is the average speed of the two-part trip?
    \begin{multicols}{3}
    \begin{choices}
      \correctchoice{\SI{7.7}{\meter\per\second}}
        \wrongchoice{\SI{10.4}{\meter\per\second}}
        \wrongchoice{\SI{12.0}{\meter\per\second}}
        \wrongchoice{\SI{13.6}{\meter\per\second}}
        \wrongchoice{\SI{17.3}{\meter\per\second}}
    \end{choices}
    \end{multicols}
\end{question}
}

\newcommand{\BowlTwentyElevenQEighteen}{
\begin{tikzpicture}
    \begin{axis}[
        axis y line=left,
        axis x line=middle,
        axis line style={->},
        xlabel={time},
        x unit=\si{\second},
        xtick={0,2,4,6,8,10},
        minor x tick num=1,
        ylabel={velocity},
        y unit=\si{\meter\per\second},
        ytick={-10,0,10},
        minor y tick num=4,
        grid=both,
        xmin=0,xmax=10,
        ymin=-10,ymax=10,
        width=0.8\columnwidth,
        height=0.5\columnwidth,
    ]
    \addplot[line width=1pt,domain=0:10] { 8*sin(36*x) };
    \end{axis}
\end{tikzpicture}
}

\element{aapt}{ %% Bowl-A1
\begin{question}{bowl-2011-q18}
    The motion of an object moving along a straight line is given by the velocity vs. time graph shown.
    \begin{center}
        \BowlTwentyElevenQEighteen
    \end{center}
    Which one of the following choices best represents the average acceleration of the object during the time interval from $t=\SI{4.0}{\second}$ to $t=\SI{9.0}{\second}$?
    \begin{multicols}{2}
    \begin{choices}
        \wrongchoice{\SI{0.80}{\meter\per\second\squared}}
        \wrongchoice{\SI{0}{\meter\per\second\squared}}
        \wrongchoice{\SI{-0.80}{\meter\per\second\squared}}
      \correctchoice{\SI{-1.6}{\meter\per\second\squared}}
        \wrongchoice{\SI{-3.2}{\meter\per\second\squared}}
    \end{choices}
    \end{multicols}
\end{question}
}

\element{aapt}{ %% Bowl-A1
\begin{question}{bowl-2011-q19}
    The motion of an object moving along a straight line is given by the velocity vs. time graph shown.
    \begin{center}
        \BowlTwentyElevenQEighteen
    \end{center}
    Which one of the following choices best represents the instantaneous acceleration of the object at the time $t=\SI{4.0}{\second}$
    \begin{multicols}{2}
    \begin{choices}
        \wrongchoice{\SI{0}{\meter\per\second\squared}}
        \wrongchoice{\SI{-1.6}{\meter\per\second\squared}}
        \wrongchoice{\SI{-2.0}{\meter\per\second\squared}}
        \wrongchoice{\SI{-3.2}{\meter\per\second\squared}}
      \correctchoice{\SI{-4.0}{\meter\per\second\squared}}
    \end{choices}
    \end{multicols}
\end{question}
}

%% NOTE: bowl-2011-q21?

\element{aapt}{ %% Bowl-A1
\begin{question}{bowl-2011-q30}
    Two cars travel to the right, each starting from rest, along a straight road.
    Car $A$ has twice the acceleration of car $B$.
    After traveling a distance $d$, Car $A$ has speed $v$.
    When Car $B$ has traveled the same distance $d$, what is its speed in terms of $v$?
    \begin{multicols}{3}
    \begin{choices}
        \wrongchoice{$\dfrac{1}{4}v$}
        \wrongchoice{$\dfrac{1}{2}v$}
        \wrongchoice{$\dfrac{\sqrt{3}}{2}v$}
      \correctchoice{$\dfrac{\sqrt{2}}{2}v$}
        \wrongchoice{$v$}
    \end{choices}
    \end{multicols}
\end{question}
}


%% PhysicsBowl 2010
%%----------------------------------------
\element{aapt}{ %% Bowl-A1
\begin{question}{bowl-2010-q04}
    Which of the following relationships correctly ranks
        the three given speeds from least to greatest?
    The speeds are given as $v_1=\SI{1.25e-4}{\centi\meter\per\micro\second}$,
        $v_2=\SI{0.076}{\mega\meter\per\week}$,
        $v_3=\SI{9.50}{\kilo\meter\per\day}$.
    \begin{multicols}{2}
    \begin{choices}
        \wrongchoice{$v_1<v_2<v_3$}
      \correctchoice{$v_3<v_2<v_1$}
        \wrongchoice{$v_2<v_3<v_1$}
        \wrongchoice{$v_1<v_3<v_2$}
        \wrongchoice{$v_3<v_2=v_1$}
    \end{choices}
    \end{multicols}
\end{question}
}

\element{aapt}{ %% Bowl-A1
\begin{question}{bowl-2010-q07}
    A small object is thrown straight downward on Earth with
        an initial speed of \SI{12.0}{\meter\per\second}
        from a position \SI{10.0}{\meter} above the ground.
    Ignoring air resistance, the speed of the object when it reaches the ground is:
    \begin{multicols}{3}
    \begin{choices}
      \correctchoice{\SI{18.4}{\meter\per\second}}
        \wrongchoice{\SI{14.6}{\meter\per\second}}
        \wrongchoice{\SI{14.0}{\meter\per\second}}
        \wrongchoice{\SI{12.8}{\meter\per\second}}
        \wrongchoice{\SI{12.0}{\meter\per\second}}
    \end{choices}
    \end{multicols}
\end{question}
}

\element{aapt}{ %% Bowl-A1
\begin{question}{bowl-2010-q10}
    A particle travels at a constant speed around
        a circular path of radius $R$.
    If the particle makes one complete trip around
        the entire circle, what is the magnitude
        of the displacement for this trip?
    \begin{multicols}{3}
    \begin{choices}
        \wrongchoice{$\pi R$}
        \wrongchoice{$2 R$}
        \wrongchoice{$2\pi R$}
        \wrongchoice{$\pi R^2$}
      \correctchoice{zero}
        %% NOTE: added for symmetry
        %\wrongchoice{$4\pi R$}
    \end{choices}
    \end{multicols}
\end{question}
}

\element{aapt}{ % Bowl-A1
\begin{question}{bowl-2010-q11}
    Consider the motion of an object given by the velocity
        vs. time graph shown below.
    \begin{center}
    \begin{tikzpicture}
        \begin{axis}[
            axis y line=left,
            axis x line=middle,
            axis line style={->},
            xlabel={time},
            x unit=\si{\second},
            xtick={0,2,4,6,8,10},
            minor x tick num=1,
            ylabel={velocity},
            y unit=\si{\meter\per\second},
            ytick={-10,0,10},
            minor y tick num=4,
            grid=both,
            xmin=0,xmax=10,
            ymin=-10,ymax=10,
            width=0.95\columnwidth,
            height=0.50\columnwidth,
        ]
        \addplot[mark=\empty,smooth] plot coordinates { (0,-6) (5,8) (10,-8) };
        \end{axis}
    \end{tikzpicture}
    \end{center}
    For which time(s) is the acceleration of the object
        equal to \SI{0}{\meter\per\second\squared}?
    \begin{choices}
        \wrongchoice{Only at time $t=\SI{2.0}{\second}$}
      \correctchoice{Only at time $t=\SI{5.0}{\second}$}
        \wrongchoice{Only at time $t=\SI{8.0}{\second}$}
        \wrongchoice{At times $t=\SI{2.0}{\second}$ and $t=\SI{5.0}{\second}$}
        \wrongchoice{At times $t=\SI{2.0}{\second}$, $t=\SI{5.0}{\second}$, and $t=\SI{8.0}{\second}$}
    \end{choices}
\end{question}
}

\element{aapt}{ %% Bowl-A1
\begin{question}{bowl-2010-q24}
    By computing the area under the acceleration vs time
        graph for a fixed time interval of an object's
        motion, what quantity has been determined for that object?
    \begin{choices}
        \wrongchoice{The average velocity during the time interval.}
        \wrongchoice{The velocity at the end of the time interval.}
        \wrongchoice{The average speed during the time interval.}
      \correctchoice{The change in velocity during the time interval.}
        \wrongchoice{The velocity at the time midway through the time interval.}
    \end{choices}
\end{question}
}

\element{aapt}{ %% Bowl-A1
\begin{question}{bowl-2010-q29}
    A ball initially at rest falls without air resistance from
        a height $h$ above the ground.
    If the ball falls the first distance $\frac{h}{2}$ in a time
        $t$, how much time is required to fall the remaining
        distance of $\frac{h}{2}$?
    \begin{multicols}{3}
    \begin{choices}
        \wrongchoice{$\num{0.25}t$}
      \correctchoice{$\num{0.41}t$}
        \wrongchoice{$\num{0.50}t$}
        \wrongchoice{$\num{0.71}t$}
        \wrongchoice{$\num{1.00}t$}
    \end{choices}
    \end{multicols}
\end{question}
}

\element{aapt}{ %% Bowl-A1
\begin{question}{bowl-2010-q38}
    Two objects both move uniformly accelerate to the right.
    At time $t=\SI{0}{\second}$, the objects are at the same
        initial position but
    \begin{itemize}
        \item Object 1 has initial speed twice that of Object 2
        \item Object 1 has one-half the acceleration of Object 2
    \end{itemize}
    After some time $T$, the velocity of the two objects is the same.
    What is the ratio of the distance traveled in this time $T$
        by Object 2 to that traveled by Object 1?
    \begin{multicols}{3}
    \begin{choices}
        \wrongchoice{$5:6$}
      \correctchoice{$4:5$}
        \wrongchoice{$3:4$}
        \wrongchoice{$2:3$}
        \wrongchoice{$1:2$}
    \end{choices}
    \end{multicols}
\end{question}
}


%% PhysicsBowl 2009
%%----------------------------------------

%% NOTE: bowl-2009-q08

\element{aapt}{ %% Bowl-A1
\begin{question}{bowl-2009-q17}
    A toy car moves \SI{3.0}{\meter} to the North in \SI{1.0}{\second}.
    The car then moves at \SI{9.0}{\meter\per\second} due South
        for \SI{2.0}{\second}.
    What is the average speed of the car for this \SI{3.0}{\second} trip?
    \begin{multicols}{3}
    \begin{choices}
        \wrongchoice{\SI{4.0}{\meter\per\second}}
        \wrongchoice{\SI{5.0}{\meter\per\second}}
        \wrongchoice{\SI{6.0}{\meter\per\second}}
      \correctchoice{\SI{7.0}{\meter\per\second}}
        \wrongchoice{\SI{12.0}{\meter\per\second}}
    \end{choices}
    \end{multicols}
\end{question}
}

\element{aapt}{ %% Bowl-A1
\begin{question}{bowl-2009-q31}
    A car moves to the right along a one-dimensional track
        for a total time $T$ in two parts.
    \begin{description}
        \item[Part One:] The car maintains constant non-zero speed $V$
            for the first \num{3/4} of the total time.
        \item[Part Two:] The car accelerates uniformly to rest during
                the last \num{1/4} of the total time.
    \end{description}
    What is the ratio of the distance traveled during Part One of
        the trip to the distance traveled during Part Two of the trip?
    \begin{multicols}{2}
    \begin{choices}
      \correctchoice{$6:1$}
        \wrongchoice{$3:2$}
        \wrongchoice{$4:3$}
        \wrongchoice{$8:3$}
        \wrongchoice{The values of $V$ and $T$ are required to answer the question.}
    \end{choices}
    \end{multicols}
\end{question}
}


%% PhysicsBowl 2008
%%----------------------------------------
\element{aapt}{ %% Bowl-A1
\begin{question}{bowl-2008-q03}
    A dog starts from rest and runs in a straight line
        with constant acceleration of \SI{2.5}{\meter\per\second\squared}.
    How much time does it take for the dog to run
        a distance of \SI{10.0}{\meter}?
    \begin{multicols}{3}
    \begin{choices}
        \wrongchoice{\SI{8.0}{\second}}
        \wrongchoice{\SI{4.0}{\second}}
      \correctchoice{\SI{2.8}{\second}}
        \wrongchoice{\SI{2.0}{\second}}
        \wrongchoice{\SI{1.4}{\second}}
    \end{choices}
    \end{multicols}
\end{question}
}

\newcommand{\BowlTwentyZeroEightQTwentyThree}{
\begin{tikzpicture}
    \begin{axis}[
        axis y line=left,
        axis x line=middle,
        axis line style={->},
        xlabel={time},
        x unit=\si{\second},
        xtick={0,2,4,6,8,10},
        minor x tick num=1,
        ylabel={velocity},
        y unit=\si{\meter\per\second},
        ytick={-10,-5,0,5,10},
        minor y tick num=4,
        grid=major,
        xmin=0,xmax=10,
        ymin=-10,ymax=10,
        width=0.95\columnwidth,
        height=0.618\columnwidth,
    ]
    \addplot[line width=1pt,domain=0:3]{8};
    \addplot[line width=1pt,domain=3:6]{20 - 4*x};
    \addplot[line width=1pt,domain=6:10]{-4};
    \end{axis}
\end{tikzpicture}
}

\element{aapt}{ %% Bowl-A1
\begin{question}{bowl-2008-q23}
    The velocity vs. time graph for the motion of a car on a straight track is shown in the diagram.
    The thick line represents the velocity.
    Assume that the car starts at the origin $x=0$.
    \begin{center}
        \BowlTwentyZeroEightQTwentyThree
    \end{center}
    At which time is the car the greatest distance from the origin?
    \begin{multicols}{3}
    \begin{choices}
        \wrongchoice{$t=\SI{10}{\second}$}
        \wrongchoice{$t=\SI{6}{\second}$}
      \correctchoice{$t=\SI{5}{\second}$}
        \wrongchoice{$t=\SI{3}{\second}$}
        \wrongchoice{$t=\SI{0}{\second}$}
    \end{choices}
    \end{multicols}
\end{question}
}

\element{aapt}{ %% Bowl-A1
\begin{question}{bowl-2008-q24}
    The velocity vs. time graph for the motion of a car on a straight track is shown in the diagram.
    The thick line represents the velocity.
    Assume that the car starts at the origin $x=0$.
    \begin{center}
        \BowlTwentyZeroEightQTwentyThree
    \end{center}
    What is the average speed of the car for the \SI{10}{\second} interval?
    \begin{multicols}{3}
    \begin{choices}
        \wrongchoice{\SI{1.20}{\meter\per\second}}
        \wrongchoice{\SI{1.40}{\meter\per\second}}
        \wrongchoice{\SI{3.30}{\meter\per\second}}
      \correctchoice{\SI{5.00}{\meter\per\second}}
        \wrongchoice{\SI{5.40}{\meter\per\second}}
    \end{choices}
    \end{multicols}
\end{question}
}


%% PhysicsBowl 2007
%%----------------------------------------
\element{aapt}{ %% Bowl-A1
\begin{question}{bowl-2007-q03}
    Two automobiles are \SI{150}{\kilo\meter} apart and traveling toward each other.
    One automobile is moving at \SI{60}{\kilo\meter\per\hour} and the other is moving \SI{40}{\kilo\meter\per\hour}.
    In how many hours will they meet?
    \begin{multicols}{3}
    \begin{choices}
      \correctchoice{\SI{1.5}{\hour}}
        \wrongchoice{\SI{1.75}{\hour}}
        \wrongchoice{\SI{2.0}{\hour}}
        \wrongchoice{\SI{2.5}{\hour}}
        \wrongchoice{\SI{3.0}{\hour}}
    \end{choices}
    \end{multicols}
\end{question}
}

\element{aapt}{ %% Bowl-A1
\begin{questionmult}{bowl-2007-q04}
    A particle moves on the $x$-axis.
    When the particle's acceleration is positive and increasing
    \begin{choices}
        \wrongchoice{its velocity must be positive.}
        \wrongchoice{its velocity must be negative.}
        \wrongchoice{it must be slowing down.}
        \wrongchoice{it must be speeding up.}
        %\correctchoice{none of the other options are true.}
    \end{choices}
\end{questionmult}
}

\element{aapt}{ %% Bowl-A1
\begin{question}{bowl-2007-q05}
    The position-time, $y$ vs. $t$, graph for the motion
        of an object is shown.
    \begin{center}
    \begin{tikzpicture}
        \begin{axis}[
            axis y line=left,
            axis x line=bottom,
            axis line style={->},
            xlabel={time},
            x unit=\si{\second},
            xtick={0,1,2,3,4,5},
            minor x tick num=1,
            ylabel={position},
            y unit=\si{\meter},
            ytick={0,10,20,30},
            minor y tick num=4,
            grid=major,
            xmin=0,xmax=5.2,
            ymin=0,ymax=32,
            width=0.95\columnwidth,
            height=0.50\columnwidth,
        ]
        \addplot[line width=1pt,domain=0:5]{25 - 4*(x-2.5)*(x-2.5)};
        \end{axis}
    \end{tikzpicture}
    \end{center}
    What would be a reasonable equation for the acceleration $a$ that would account for this motion?
    \begin{choices}
        \wrongchoice{$a=0$}
        \wrongchoice{$a=$ positive constant}
      \correctchoice{$a=$ negative constant}
        \wrongchoice{$a=$ positive constant times $t$}
        \wrongchoice{$a=$ negative constant times $t$}
    \end{choices}
\end{question}
}

\element{aapt}{ %% Bowl-A1
\begin{question}{bowl-2007-q08}
    What does one obtain by dividing the distance of \SI{12}{\mega\meter} by the time of \SI{4}{\tera\second}?
    \begin{multicols}{3}
    \begin{choices}
        \wrongchoice{\SI{3}{\nano\meter\per\second}}
      \correctchoice{\SI{3}{\micro\meter\per\second}}
        \wrongchoice{\SI{3}{\milli\meter\per\second}}
        \wrongchoice{\SI{3}{\kilo\meter\per\second}}
        \wrongchoice{\SI{3}{\giga\meter\per\second}}
    \end{choices}
    \end{multicols}
\end{question}
}

\element{aapt}{ %% Bowl-A1
\begin{question}{bowl-2007-q11}
    A cart is initially moving at \SI{0.5}{\meter\per\second} along a track.
    The cart comes to rest after traveling \SI{1}{\meter}.
    The experiment is repeated on the same track,
        but now the cart is initially moving at \SI{1}{\meter\per\second}.
    How far does the cart travel before coming to rest?
    \begin{multicols}{3}
    \begin{choices}
        \wrongchoice{\SI{1}{\meter}}
        \wrongchoice{\SI{2}{\meter}}
        \wrongchoice{\SI{3}{\meter}}
      \correctchoice{\SI{4}{\meter}}
        \wrongchoice{\SI{5}{\meter}}
    \end{choices}
    \end{multicols}
\end{question}
}

\element{aapt}{ %% Bowl-A1
\begin{question}{bowl-2007-q12}
    The definition of average velocity is:
    \begin{choices}
        \wrongchoice{the average acceleration multiplied by the time.}
        \wrongchoice{distance traveled divided by the time.}
        \wrongchoice{$\frac{1}{2}\left(v_f+v_i\right)$}
        \wrongchoice{radius multiplied by angular velocity.}
      \correctchoice{displacement divided by the time.}
    \end{choices}
\end{question}
}


%% PhysicsBowl 2006
%%----------------------------------------
\element{aapt}{ %% Bowl-A1
\begin{question}{bowl-2006-q03}
    Three students were arguing about the height of a parking garage.
    One student suggested that to determine the height of the garage,
        they simply had to drop tennis balls from the top and time
        the fall of the tennis balls.
    If the time for the ball to fall was \SI{1.4}{\second},
        approximately how tall is the parking garage?
    \begin{multicols}{3}
    \begin{choices}
        \wrongchoice{\SI{4.9}{\meter}}
        \wrongchoice{\SI{7.0}{\meter}}
      \correctchoice{\SI{9.8}{\meter}}
        \wrongchoice{\SI{13.8}{\meter}}
        \wrongchoice{\SI{19.6}{\meter}}
    \end{choices}
    \end{multicols}
\end{question}
}

\element{aapt}{ %% Bowl-A1
\begin{question}{bowl-2006-q07}
    A car has the velocity versus time curve shown.
    %% NOTE: bowl-2007-q17
    \begin{center}
    \begin{tikzpicture}
        \begin{axis}[
            axis y line=left,
            axis x line=middle,
            axis line style={->},
            clip=false,
            xlabel={time},
            x unit=\si{\second},
            xtick={0,1,2,3,4,5,6},
            ylabel={velocity},
            y unit=\si{\meter\per\second},
            ytick={-10,0,10},
            minor y tick num=1,
            grid=major,
            xmin=0,xmax=6.5,
            ymin=-15,ymax=15,
            width=0.8\columnwidth,
            height=0.5\columnwidth,
        ]
        \addplot[very thick,smooth,tension=0.7,mark=\empty] plot coordinates { (0,0) (1,6) (2,15) (3,7) (3.3,0) (4,-10) (5.1,0) (5,-1) (6,6) };
        \end{axis}
    \end{tikzpicture}
    \end{center}
    Which of the following statements regarding its motion is \emph{incorrect}?
    \begin{choices}
        \wrongchoice{The car is moving fastest at \SI{2.0}{\second}.}
        \wrongchoice{The car is at rest at approximately \SI{5.2}{\second}.}
        \wrongchoice{The car is speeding up from $t=\SI{0}{\second}$ to $t=\SI{2.0}{\second}$.}
      \correctchoice{The car has negative acceleration at $t=\SI{4.5}{\second}$.}
        \wrongchoice{The car has no acceleration at the instant $t=\SI{2.0}{\second}$.}
    \end{choices}
\end{question}
}

\element{aapt}{ %% Bowl-A1
\begin{question}{bowl-2006-q18}
    A cart is initially moving at \SI{0.5}{\meter\per\second} along a track.
    The cart comes to rest after traveling \SI{1}{\meter}.
    The experiment is repeated on the same track,
        but now the cart is initially moving at \SI{1}{\meter\per\second}.
    How far does the cart travel before coming to rest?
    \begin{multicols}{3}
    \begin{choices}
        \wrongchoice{\SI{1}{\meter}}
        \wrongchoice{\SI{2}{\meter}}
        \wrongchoice{\SI{3}{\meter}}
      \correctchoice{\SI{4}{\meter}}
        \wrongchoice{\SI{8}{\meter}}
    \end{choices}
    \end{multicols}
\end{question}
}


%% PhysicsBowl 2005
%%----------------------------------------
\element{aapt}{ %% Bowl-A1
\begin{question}{bowl-2005-q07}
    A snail is moving along a straight line.
    Its initial position is $x_0 = \SI{-5}{\meter}$ and it is moving away from the origin and slowing down.
    In this coordinate system, the signs of the initial position $x_0$,
        initial velocity $v_0$ and acceleration $a$, respectively, are
    \begin{choices}
        \wrongchoice{$x_0=-$, $v_0=+$, $a=+$}
      \correctchoice{$x_0=-$, $v_0=-$, $a=+$}
        \wrongchoice{$x_0=-$, $v_0=-$, $a=-$}
        \wrongchoice{$x_0=-$, $v_0=+$, $a=-$}
        \wrongchoice{$x_0=+$, $v_0=+$, $a=+$}
    \end{choices}
\end{question}
}


%% PhysicsBowl 2000
%%----------------------------------------
\element{aapt}{ %% Bowl-A1
\begin{question}{bowl-2000-q09}
    Is it possible for an object's velocity to increase while its acceleration decreases?
    \begin{choices}
        \wrongchoice{No, this is impossible because of the way in which acceleration is defined.}
        \wrongchoice{No, because if acceleration is decreasing the object will be slowing down.}
        \wrongchoice{No, because velocity and acceleration must always be in the same direction.}
        \wrongchoice{Yes, an example would be a falling object near the surface of the moon.}
      \correctchoice{Yes, an example would be a falling object in the presence of air resistance}
    \end{choices}
\end{question}
}

\element{aapt}{ %% Bowl-A1
\begin{question}{bowl-2000-q40}
    Suppose two cars are racing on a circular track \SI{1}{\kilo\meter} in circumference.
    The first car can circle the track in \SI{15}{\second} at top speed
        while the second car can circle the track in \SI{12}{\second} at top speed.
    How much lead does the first car need starting the last lap of the race not to lose?
    \begin{multicols}{2}
    \begin{choices}
      \correctchoice{at least \SI{250}{\meter}}
        \wrongchoice{at least \SI{200}{\meter}}
        \wrongchoice{at least \SI{104}{\meter}}
        \wrongchoice{at least \SI{83}{\meter}}
        \wrongchoice{at least \SI{67}{\meter}}
    \end{choices}
    \end{multicols}
\end{question}
}


%% PhysicsBowl 1999
%%----------------------------------------
\element{aapt}{ %% Bowl-A1
\begin{question}{bowl-1999-q01}
    %% NOTE: NIST style convention is distance divied by time
    The change of distance per unit time without reference
        to a particular direction is called
    \begin{multicols}{2}
    \begin{choices}
        \wrongchoice{inertia}
      \correctchoice{speed}
        \wrongchoice{velocity}
        \wrongchoice{acceleration}
        \wrongchoice{position}
    \end{choices}
    \end{multicols}
\end{question}
}

\element{aapt}{ %% Bowl-A1
\begin{question}{bowl-1999-q20}
    What is the shape of the velocity time graph for an object with the position time graph shown below.
    \begin{center}
    \begin{tikzpicture}
        \begin{axis}[
            axis y line=left,
            axis x line=middle,
            axis line style={->},
            xlabel={time},
            xtick=\empty,
            ylabel={position},
            ytick=\empty,
            xmin=0,xmax=11,
            ymin=-6,ymax=6,
            width=0.8\linewidth,
            height=0.5\linewidth,
        ]
        \addplot[line width=1pt,domain=0:10]{5-x};
        \end{axis}
    \end{tikzpicture}
    \end{center}
    \begin{multicols}{2}
    \begin{choices}
        \AMCboxDimensions{down=-2em}
        \wrongchoice{
            \begin{tikzpicture}
                \begin{axis}[
                    axis y line=left,
                    axis x line=middle,
                    axis line style={->},
                    xlabel={time},
                    xtick=\empty,
                    x label style={anchor=north east},
                    ylabel={position},
                    ytick=\empty,
                    xmin=0,xmax=11,
                    ymin=-6,ymax=6,
                    width=1.0\columnwidth,
                ]
                \addplot[line width=1pt,domain=0:10]{4};
                \end{axis}
            \end{tikzpicture}
        }
        %% ANS is B
        \correctchoice{
            \begin{tikzpicture}
                \begin{axis}[
                    axis y line=left,
                    axis x line=middle,
                    axis line style={->},
                    xlabel={time},
                    xtick=\empty,
                    ylabel={position},
                    ytick=\empty,
                    xmin=0,xmax=11,
                    ymin=-6,ymax=6,
                    width=1.0\columnwidth,
                ]
                \addplot[line width=1pt,domain=0:10]{-4};
                \end{axis}
            \end{tikzpicture}
        }
        \wrongchoice{
            \begin{tikzpicture}
                \begin{axis}[
                    axis y line=left,
                    axis x line=middle,
                    axis line style={->},
                    xlabel={time},
                    xtick=\empty,
                    x label style={anchor=north east},
                    ylabel={position},
                    ytick=\empty,
                    xmin=0,xmax=11,
                    ymin=-6,ymax=6,
                    width=1.0\columnwidth,
                ]
                \addplot[line width=1pt,mark=\empty] plot coordinates { (0,5) (5,0) (10,5) };
                \end{axis}
            \end{tikzpicture}
        }
        \wrongchoice{
            \begin{tikzpicture}
                \begin{axis}[
                    axis y line=left,
                    axis x line=middle,
                    axis line style={->},
                    xlabel={time},
                    xtick=\empty,
                    ylabel={position},
                    ytick=\empty,
                    xmin=0,xmax=11,
                    ymin=-6,ymax=6,
                    width=1.0\columnwidth,
                ]
                \addplot[line width=1pt,domain=0:5] { -0.2*x*x };
                \addplot[line width=1pt,domain=5:10] { -0.2*(x-10)*(x-10) };
                \end{axis}
            \end{tikzpicture}
        }
        \wrongchoice{
            \begin{tikzpicture}
                \begin{axis}[
                    axis y line=left,
                    axis x line=middle,
                    axis line style={->},
                    xlabel={time},
                    xtick=\empty,
                    ylabel={position},
                    ytick=\empty,
                    xmin=0,xmax=11,
                    ymin=-6,ymax=6,
                    width=1.0\columnwidth,
                ]
                \addplot[line width=1pt,domain=0:5] { 5 - 0.2*x*x };
                \addplot[line width=1pt,domain=5:10] { -5 + 0.2*(x-10)*(x-10) };
                \end{axis}
            \end{tikzpicture}
        }
    \end{choices}
    \end{multicols}
\end{question}
}


%% PhysicsBowl 1998
%%----------------------------------------
\element{aapt}{ %% Bowl-A1
\begin{question}{bowl-1998-q14}
    How long must a \SI{2.5}{\meter\per\second\squared} acceleration
        act to change the velocity of a \SI{2.0}{\kilo\gram} object
        by \SI{3.0}{\meter\per\second}?
    \begin{multicols}{3}
    \begin{choices}
        \wrongchoice{\SI{0.83}{\second}}
      \correctchoice{\SI{1.2}{\second}}
        \wrongchoice{\SI{1.7}{\second}}
        \wrongchoice{\SI{2.5}{\second}}
        \wrongchoice{\SI{7.5}{\second}}
    \end{choices}
    \end{multicols}
\end{question}
}


\element{aapt}{ %% Bowl-A1
\begin{question}{bowl-1998-q15}
    A freely falling object is found to be moving downward at \SI{18}{\meter\per\second}.
    If it continues to fall,
        two seconds later the object would be moving with a speed of:
    \begin{multicols}{3}
    \begin{choices}
        \wrongchoice{\SI{8.0}{\meter\per\second}}
        \wrongchoice{\SI{10}{\meter\per\second}}
        \wrongchoice{\SI{18}{\meter\per\second}}
      \correctchoice{\SI{38}{\meter\per\second}}
        \wrongchoice{\SI{180}{\meter\per\second}}
    \end{choices}
    \end{multicols}
\end{question}
}

\element{aapt}{ %% Bowl-A1
\begin{question}{bowl-1998-q31}
    An object sliding down an inclined plane has a speed \SI{0.40}{\meter\per\second} two seconds after it begins to slide.
    Approximately how far does it travel in three seconds?
    \begin{multicols}{3}
    \begin{choices}
        \wrongchoice{\SI{0.15}{\meter}}
        \wrongchoice{\SI{0.30}{\meter}}
        \wrongchoice{\SI{0.45}{\meter}}
        \wrongchoice{\SI{0.60}{\meter}}
      \correctchoice{\SI{0.90}{\meter}}
    \end{choices}
    \end{multicols}
\end{question}
}


%% PhysicsBowl 1997
%%----------------------------------------
\element{aapt}{ %% Bowl-A1
\begin{question}{bowl-1997-q18}
    A motorist travels \SI{400}{\kilo\meter} at \SI{80}{\kilo\meter\per\hour} and \SI{400}{\kilo\meter} at \SI{100}{\kilo\meter\per\hour}.
    What is the average speed of the motorist on this trip?
    \begin{multicols}{2}
    \begin{choices}
        \wrongchoice{\SI{84}{\kilo\meter\per\hour}}
      \correctchoice{\SI{89}{\kilo\meter\per\hour}}
        \wrongchoice{\SI{90}{\kilo\meter\per\hour}}
        \wrongchoice{\SI{91}{\kilo\meter\per\hour}}
        \wrongchoice{\SI{95}{\kilo\meter\per\hour}}
    \end{choices}
    \end{multicols}
\end{question}
}

\newcommand{\BowlNineteenNinetySevenQTwenty}{
\begin{tikzpicture}
    \begin{axis}[
        axis y line=left,
        axis x line=middle,
        axis line style={->},
        xlabel={time},
        x unit=\si{\second},
        xtick={0,5,10,15,20,25,30,35},
        ylabel={velocity},
        y unit=\si{\meter\per\second},
        ytick={-2,-1,0,1,2,3},
        grid=major,
        xmin=0,xmax=37,
        ymin=-2.2,ymax=3.3,
        width=0.95\columnwidth,
        height=0.50\columnwidth,
    ]
    \addplot[line width=1pt,domain=0:10]{0.3*x};
    \addplot[line width=1pt,domain=10:20]{3};
    \addplot[line width=1pt,domain=20:25]{3 - 0.8*(x-20)};
    \addplot[line width=1pt,domain=25:30]{-1};
    \addplot[line width=1pt,domain=30:35]{-1 + 0.2*(x-30)};
    \end{axis}
\end{tikzpicture}
}

\element{aapt}{ %% Bowl-A1
\begin{question}{bowl-1997-q20}
    %Questions 20, 21, and 22 refer to the motion of a toy car traveling along the x-axis.
    The graph shown below is a plot of the car's
        velocity in the $x$ direction, $v_x$, versus time, $t$.
    \begin{center}
        \BowlNineteenNinetySevenQTwenty
    \end{center}
    During what time interval was the car moving towards
        its initial position at constant velocity?
    \begin{multicols}{2}
    \begin{choices}
        \wrongchoice{\SIrange{0}{10}{\second}}
        \wrongchoice{\SIrange{10}{20}{\second}}
        \wrongchoice{\SIrange{20}{25}{\second}}
      \correctchoice{\SIrange{25}{30}{\second}}
        \wrongchoice{\SIrange{30}{35}{\second}}
    \end{choices}
    \end{multicols}
\end{question}
}

\element{aapt}{ %% Bowl-A1
\begin{question}{bowl-1997-q21}
    %Questions 20, 21, and 22 refer to the motion of a toy car traveling along the x-axis.
    The graph shown below is a plot of the car's
        velocity in the $x$ direction, $v_x$, versus time, $t$.
    \begin{center}
        \BowlNineteenNinetySevenQTwenty
    \end{center}
    What was the acceleration at \SI{33}{\second}?
    \begin{multicols}{2}
    \begin{choices}
        \wrongchoice{\SI{0.40}{\meter\per\second\squared}}
      \correctchoice{\SI{0.20}{\meter\per\second\squared}}
        \wrongchoice{\SI{0.0}{\meter\per\second\squared}}
        \wrongchoice{\SI{-0.20}{\meter\per\second\squared}}
        \wrongchoice{\SI{-0.40}{\meter\per\second\squared}}
    \end{choices}
    \end{multicols}
\end{question}
}

\element{aapt}{ %% Bowl-A1
\begin{question}{bowl-1997-q22}
    %Questions 20, 21, and 22 refer to the motion of a toy car traveling along the x-axis.
    The graph shown below is a plot of the car's
        velocity in the $x$ direction, $v_x$, versus time, $t$.
    \begin{center}
        \BowlNineteenNinetySevenQTwenty
    \end{center}
    How far did the car travel during the first \SI{15}{\second}?
    \begin{multicols}{3}
    \begin{choices}
        \wrongchoice{\SI{0.0}{\meter}}
        \wrongchoice{\SI{3.0}{\meter}}
        \wrongchoice{\SI{15}{\meter}}
      \correctchoice{\SI{30}{\meter}}
        \wrongchoice{\SI{45}{\meter}}
    \end{choices}
    \end{multicols}
\end{question}
}



%% PhysicsBowl 1996
%%----------------------------------------
\element{aapt}{ %% Bowl-A1
\begin{question}{bowl-1996-q17}
    A ball which is dropped from the top of a building strikes the ground with a speed of \SI{30}{\meter\per\second}.
    Assume air resistance can be ignored.
    The height of the building is approximately:
    \begin{multicols}{3}
    \begin{choices}
        \wrongchoice{\SI{15}{\meter}}
        \wrongchoice{\SI{30}{\meter}}
      \correctchoice{\SI{45}{\meter}}
        \wrongchoice{\SI{75}{\meter}}
        \wrongchoice{\SI{90}{\meter}}
    \end{choices}
    \end{multicols}
\end{question}
}


%% PhysicsBowl 1995
%%----------------------------------------
\element{aapt}{ %% Bowl-A1
\begin{question}{bowl-1995-q03}
    A toy car moves \SI{0.80}{\meter} in \SI{1.0}{\second}
        at the constant velocity.
    If it continues, how far will it travel in \SI{3.0}{\second}?
    \begin{multicols}{3}
    \begin{choices}
      \correctchoice{\SI{2.4}{\meter}}
        \wrongchoice{\SI{3.6}{\meter}}
        \wrongchoice{\SI{4.8}{\meter}}
        \wrongchoice{\SI{7.2}{\meter}}
        \wrongchoice{\SI{14.4}{\meter}}
    \end{choices}
    \end{multicols}
\end{question}
}

\element{aapt}{ %% Bowl-A1
\begin{question}{bowl-1995-q13}
    A freely falling body is found to be moving downwards at \SI{27}{\meter\per\second} at one instant.
    If it continues to fall, one second later the object would be moving with a downward velocity closest to:
    \begin{multicols}{3}
    \begin{choices}
        \wrongchoice{\SI{270}{\meter\per\second}}
      \correctchoice{\SI{37}{\meter\per\second}}
        \wrongchoice{\SI{27}{\meter\per\second}}
        \wrongchoice{\SI{17}{\meter\per\second}}
        \wrongchoice{\SI{10}{\meter\per\second}}
    \end{choices}
    \end{multicols}
\end{question}
}

\element{aapt}{ %% Bowl-A1
\begin{question}{bowl-1995-q39}
    A car starts from rest and accelerates at \SI{0.80}{\meter\per\second\squared} for \SI{10}{\second}.
    It then continues at constant velocity.
    Twenty seconds (\SI{20}{\second}) after it began to move, the car has:
    \begin{choices}
        \wrongchoice{velocity \SI{8.0}{\meter\per\second} and has traveled \SI{40}{\meter}.}
        \wrongchoice{velocity \SI{8.0}{\meter\per\second} and has traveled \SI{80}{\meter}.}
      \correctchoice{velocity \SI{8.0}{\meter\per\second} and has traveled \SI{120}{\meter}.}
        \wrongchoice{velocity \SI{16.0}{\meter\per\second} and has traveled \SI{160}{\meter}.}
        \wrongchoice{velocity \SI{16.0}{\meter\per\second} and has traveled \SI{320}{\meter}.}
    \end{choices}
\end{question}
}


%% PhysicsBowl 1994
%%----------------------------------------
\element{aapt}{ %% Bowl-A1
\begin{question}{bowl-1994-q01}
    Starting from rest, object 1 falls freely for \SI{4.0}{\second},
        and object 2 falls freely for \SI{8.0}{\second}.
    Compared to object 1, object 2 falls:
    \begin{choices}
        \wrongchoice{half as far.}
        \wrongchoice{twice as far.}
        \wrongchoice{three times as far.}
      \correctchoice{four times as far.}
        \wrongchoice{sixteen times as far.}
    \end{choices}
\end{question}
}

\element{aapt}{ %% Bowl-A1
\begin{question}{bowl-1994-q02}
    A car starts from rest and uniformly accelerates to a final speed of \SI{20.0}{\meter\per\second} in a time of \SI{15}{\second}.
    How far does the car travel during this time?
    \begin{multicols}{3}
    \begin{choices}
      \correctchoice{\SI{150}{\meter}}
        \wrongchoice{\SI{300}{\meter}}
        \wrongchoice{\SI{450}{\meter}}
        \wrongchoice{\SI{600}{\meter}}
        \wrongchoice{\SI{800}{\meter}}
    \end{choices}
    \end{multicols}
\end{question}
}

\endinput


