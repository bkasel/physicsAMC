

%% AAPT Physics Bowl Exams Questions
%%----------------------------------------


%% This section has XX problems


%% PhysicsBowl 2015
%%----------------------------------------


%% PhysicsBowl 2011
%%----------------------------------------
\element{aapt}{ %% Bowl-C1
\begin{question}{bowl-2011-q31}
    Approximately how much energy is required to transform a \SI{20}{\gram} cube of ice at a temperature of \SI{-10.0}{\degreeCelsius} into liquid water at a temperature of \SI{10}{\degreeCelsius}?
    \begin{multicols}{3}
    \begin{choices}
        \wrongchoice{\SI{840}{\joule}}
        \wrongchoice{\SI{1 300}{\joule}}
        \wrongchoice{\SI{1 700}{\joule}}
      \correctchoice{\SI{7 900}{\joule}}
        \wrongchoice{\SI{46 000}{\joule}}
    \end{choices}
    \end{multicols}
\end{question}
}


%% PhysicsBowl 2010
%%----------------------------------------
\element{aapt}{ %% Bowl-C1
\begin{question}{bowl-2010-q14}
    What temperature change on the Kelvin scale is equivalent to
        a \SI{27}{\degreeCelsius} change?
    \begin{multicols}{3}
    \begin{choices}
        \wrongchoice{\SI{300}{\kelvin}}
        \wrongchoice{\SI{273}{\kelvin}}
        \wrongchoice{\SI{246}{\kelvin}}
      \correctchoice{\SI{27}{\kelvin}}
        \wrongchoice{\SI{9}{\kelvin}}
    \end{choices}
    \end{multicols}
\end{question}
}

\element{aapt}{ %% Bowl-C1
\begin{question}{bowl-2010-q21}
    Which terminology is best associated with the amount
        of energy required to change the phase of a material
        per unit mass?
    \begin{choices}
        \wrongchoice{Thermal conductivity}
        \wrongchoice{Specific heat}
        \wrongchoice{Work}
      \correctchoice{Latent heat}
        \wrongchoice{Entropy}
    \end{choices}
\end{question}
}


%% PhysicsBowl 2009
%%----------------------------------------
\element{aapt}{ %% Bowl-C1
\begin{question}{bowl-2009-q19}
    A scientist claims to be investigating
        ``The transfer of energy that results from the bulk motion of a fluid.''
    Which of the following terms best describes the energy
        transfer method that this scientist is studying?
    \begin{multicols}{2}
    \begin{choices}
        \wrongchoice{radiation}
      \correctchoice{convection}
        \wrongchoice{conduction}
        \wrongchoice{latent heat}
        \wrongchoice{specific heat}
    \end{choices}
    \end{multicols}
\end{question}
}


%% PhysicsBowl 2008
%%----------------------------------------
\element{aapt}{ %% Bowl-C1
\begin{question}{bowl-2008-q22}
    In a calorimeter, \SI{20}{\gram} of liquid water at \SI{100}{\degreeCelsius} is mixed with \SI{50}{\gram} of water vapor at \SI{100}{\degreeCelsius}.
    The system is allowed to come to equilibrium.
    Assuming that the calorimeter and the surroundings can be ignored,
        which of the following best describes the net energy exchange between the vapor and the liquid during the process of coming to equilibrium?
    \begin{choices}
      \correctchoice{There is no net energy exchange.}
        \wrongchoice{Energy is transferred from the vapor to the liquid, vaporizing some of the liquid.}
        \wrongchoice{Energy is transferred from the vapor to the liquid, increasing the liquid's temperature.}
        \wrongchoice{Energy is transferred from the vapor to the liquid until all of the liquid vaporizes.}
        \wrongchoice{Energy is transferred from the liquid to the vapor, condensing some vapor.}
    \end{choices}
\end{question}
}


%% PhysicsBowl 2007
%%----------------------------------------
\element{aapt}{ %% Bowl-C1
\begin{question}{bowl-2007-q38}
    Absolute zero is best described as that temperature at which:
    \begin{choices}
        \wrongchoice{water freezes at standard pressure.}
        \wrongchoice{water is at its triple point.}
        \wrongchoice{the molecules of a substance have a maximum kinetic energy.}
        \wrongchoice{the molecules of a substance have a maximum potential energy.}
      \correctchoice{the molecules of a substance have minimum kinetic energy.}
    \end{choices}
\end{question}
}

\element{aapt}{ %% Bowl-C1
\begin{question}{bowl-2007-q39}
    A mass of material exists in its solid form at its melting temperature \SI{10}{\degreeCelsius}.
    The following processes then occur to the material:
    \begin{description}[itemsep=1ex]
        \item[Process 1:] An amount of thermal energy $Q$ is added to the material and $\frac{3}{4}$ of the material melts.
        \item[Process 2:] An identical additional amount of thermal energy $Q$ is added to the material and the material is now a liquid at \SI{50}{\degreeCelsius}.
    \end{description}
    What is the ratio of the latent heat of fusion to the specific heat of the liquid for this material?
    \begin{multicols}{2}
    \begin{choices}
      \correctchoice{\SI{80}{\degreeCelsius}}
        \wrongchoice{\SI{60}{\degreeCelsius}}
        \wrongchoice{\SI{40}{\degreeCelsius}}
        \wrongchoice{\SI{20}{\degreeCelsius}}
        \wrongchoice{More information is needed to answer this question.}
    \end{choices}
    \end{multicols}
\end{question}
}


%% PhysicsBowl 2006
%%----------------------------------------
\element{aapt}{ %% Bowl-C1
\begin{question}{bowl-2006-q27}
    A frozen hamburger in plastic needs to be thawed quickly.
    Which of the methods described provides the most rapid thawing of the burger?
    \begin{choices}
      \correctchoice{Place the burger itself in a metal pan at room temperature.}
        \wrongchoice{Place the burger in its package on the kitchen counter at room temperature.}
        \wrongchoice{Place the burger in its package in a pot of non-boiling warm water.}
        \wrongchoice{Place the burger itself on a plastic plate in the refrigerator.}
        \wrongchoice{Place the burger itself on the ceramic kitchen counter at room temperature.}
    \end{choices}
\end{question}
}


%% PhysicsBowl 2005
%%----------------------------------------
\element{aapt}{ %% Bowl-C1
\begin{question}{bowl-2005-q46}
    A uniform square piece of metal has initial side length $L_0$.
    A square piece is cut out of the center of the metal.
    \begin{center}
    \begin{tikzpicture}
        %% uniform squared piece
        \draw[pattern=north east lines] (-2,-2) rectangle (2,2);
        \draw[fill=white] (-0.5,-0.5) rectangle (0.5,0.5);
        %% length 
        \draw[thick,<->] (-2.5,-2) -- (-2.5,2) node[pos=0.5,anchor=center,fill=white] {$L_0$};
        \draw[thick,<->] (-2,2.5) -- (+2,2.5) node[pos=0.5,anchor=center,fill=white] {$L_0$};
    \end{tikzpicture}
    \end{center}
    The temperature of the metal is now raised so that the side lengths are increased by \SI{4}{\percent}.
    What has happened to the area of the square piece cut out of the center of the metal?
    \begin{choices}
        \wrongchoice{It is increased by \SI{16}{\percent}}
      \correctchoice{It is increased by \SI{8}{\percent}}
        \wrongchoice{It is increased by \SI{4}{\percent}}
        \wrongchoice{It is decreased by \SI{4}{\percent}}
        \wrongchoice{It is decreased by \SI{8}{\percent}}
    \end{choices}
\end{question}
}


%% PhysicsBowl 2000
%%----------------------------------------
\element{aapt}{ %% Bowl-C1
\begin{question}{bowl-2000-q35}
    A \SI{100}{\gram} block of aluminum at \SI{90}{\degreeCelsius} is brought into contact with a \SI{100}{\gram} block of lead at \SI{10}{\degreeCelsius} inside a thermally isolated container.
    The final temperature of the system at equilibrium would be closest to:
    \begin{multicols}{2}
    \begin{choices}
        \wrongchoice{\SI{10}{\degreeCelsius}}
        \wrongchoice{\SI{20}{\degreeCelsius}}
        \wrongchoice{\SI{50}{\degreeCelsius}}
      \correctchoice{\SI{80}{\degreeCelsius}}
        \wrongchoice{\SI{90}{\degreeCelsius}}
    \end{choices}
    \end{multicols}
\end{question}
}


%% PhysicsBowl 1999
%%----------------------------------------
\element{aapt}{ %% Bowl-C1
\begin{question}{bowl-1998-q05}
    Which of the following temperatures would be most appropriate to keep milk at inside a refrigerator?
    \begin{multicols}{3}
    \begin{choices}
        \wrongchoice{\SI{-20}{\degreeCelsius}}
        \wrongchoice{\SI{5}{\kelvin}}
        \wrongchoice{\SI{40}{\degreeCelsius}}
      \correctchoice{\SI{278}{\kelvin}}
        \wrongchoice{\SI{350}{\kelvin}}
    \end{choices}
    \end{multicols}
\end{question}
}


%% PhysicsBowl 1996
%%----------------------------------------
\element{aapt}{ %% Bowl-C1
\begin{question}{bowl-1996-q05}
    When metal rod 1 is placed in contact with metal rod 2,
        thermal energy flows from 1 to 2.
    A possible explanation is that 1 has a higher \rule[-0.1pt]{4em}{0.1pt} than 2.
    \begin{multicols}{2}
    \begin{choices}
        \wrongchoice{heat}
        \wrongchoice{heat capacity}
        \wrongchoice{mass}
        \wrongchoice{specific heat}
      \correctchoice{temperature}
    \end{choices}
    \end{multicols}
\end{question}
}


%% PhysicsBowl 1995
%%----------------------------------------
\element{aapt}{ %% Bowl-C1
\begin{question}{bowl-1995-q05}
    What temperature change on the Kelvin scale is equivalent to a \ang{10} change on the Celsius scale?
    \begin{multicols}{3}
    \begin{choices}
        \wrongchoice{\SI{283}{\kelvin}}
        \wrongchoice{\SI{273}{\kelvin}}
        \wrongchoice{\SI{18}{\kelvin}}
      \correctchoice{\SI{10}{\kelvin}}
        \wrongchoice{\SI{0}{\kelvin}}
    \end{choices}
    \end{multicols}
\end{question}
}

\element{aapt}{ %% Bowl-C1
\begin{question}{bowl-1995-q14}
    Which would be the most comfortable temperature for your bath water?
    \begin{multicols}{3}
    \begin{choices}
        \wrongchoice{\SI{0}{\degreeCelsius}}
        \wrongchoice{\SI{40}{\kelvin}}
        \wrongchoice{\SI{110}{\degreeCelsius}}
      \correctchoice{\SI{310}{\kelvin}}
        \wrongchoice{\SI{560}{\kelvin}}
    \end{choices}
    \end{multicols}
\end{question}
}

\element{aapt}{ %% Bowl-C1
\begin{question}{bowl-1995-q20}
    One kilogram of water at \SI{85}{\degreeCelsius} is added to a one kilogram thermally isolated copper container initially at \SI{15}{\degreeCelsius}.
    Which of the following statements is true once the system has reached thermal equilibrium?
    \begin{choices}
        \wrongchoice{The thermal energy gained by the copper is greater than the thermal energy lost by the water.}
        \wrongchoice{The thermal energy gained by the copper is less than the thermal energy lost by the water.}
      \correctchoice{The temperature change of the copper is greater than the temperature change of the water.}
        \wrongchoice{The temperature change of the copper is the same as the temperature change of the water.}
        \wrongchoice{The temperature change of the copper is less than the temperature change of the water.}
    \end{choices}
\end{question}
}


%% PhysicsBowl 1994
%%----------------------------------------
\element{aapt}{ %% Bowl-C1
\begin{question}{bowl-1994-q12}
    Which of the following could \emph{not} be used to indicate a temperature change?
    A change in:
    \begin{choices}
        \wrongchoice{color of a metal rod.}
        \wrongchoice{length of a liquid column.}
        \wrongchoice{pressure of a gas at constant volume.}
        \wrongchoice{electrical resistance.}
      \correctchoice{mass of one mole of gas at constant pressure.}
    \end{choices}
\end{question}
}


\endinput


