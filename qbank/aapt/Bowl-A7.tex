

%% AAPT Physics Bowl Exams Questions
%%----------------------------------------


%% this section contains 19 problems


%% PhysicsBowl 2015
%%----------------------------------------
\element{aapt}{ %% Bowl-A7
\begin{question}{bowl-2015-q22}
    Satellite 1 makes a circular orbit around the Earth with a radius $r_1=R$.
    Satellite 2 makes a circular orbit around the Earth with a radius $r_2=2R$.
    We let $v$ represent the speed of a satellite and $a$
        represent the magnitude of a satellite's acceleration.
    Which one of the following choices gives the correct relation
        between the speeds and accelerations of the satellites?
    \begin{choices}
      \correctchoice{$v_2=\dfrac{1}{\sqrt{2}}v_1$; $a_2=\dfrac{1}{4}a_1$}
        \wrongchoice{$v_2=\dfrac{1}{2}v_1$; $a_2=\dfrac{1}{4}a_1$}
        \wrongchoice{$v_2=\dfrac{1}{\sqrt{2}}v_1$; $a_2=\dfrac{1}{2}a_1$}
        \wrongchoice{$v_2=\dfrac{1}{2}v_1$; $a_2=\dfrac{1}{2}a_1$}
        \wrongchoice{$v_2=v_1$; $a_2=\dfrac{1}{2}a_1$}
    \end{choices}
\end{question}
}


%% PhysicsBowl 2014
%%----------------------------------------
\element{aapt}{ %% Bowl-A7
\begin{question}{bowl-2014-q27}
    Two spherical, non-rotating planets, $X$ and $Y$, have the same density $\rho$.
    Planet $X$ has twice the radius of planet $Y$.
    Let $g_X$ and $g_Y$ represent the acceleration due to gravity at the surfaces of Planet $X$ and planet $Y$, respectively.
    What is the ratio of $g_X:g_Y$?
    \begin{multicols}{3}
    \begin{choices}
      \correctchoice{$2:1$}
        \wrongchoice{$1:2$}
        \wrongchoice{$1:1$}
        \wrongchoice{$4:1$}
        \wrongchoice{$1:4$}
    \end{choices}
    \end{multicols}
\end{question}
}

\element{aapt}{ %% Bowl-A7
\begin{question}{bowl-2014-q30}
    Considering only the Moon-Earth system (ignore any influence of the Sun),
        which one of the following best represents the magnitude of the Moon's acceleration about the Earth?
    \begin{multicols}{2}
    \begin{choices}
        \wrongchoice{\SI{3e-1}{\meter\per\second\squared}}
        \wrongchoice{\SI{3e-2}{\meter\per\second\squared}}
      \correctchoice{\SI{3e-3}{\meter\per\second\squared}}
        \wrongchoice{\SI{3e-4}{\meter\per\second\squared}}
        \wrongchoice{\SI{3e-5}{\meter\per\second\squared}}
    \end{choices}
    \end{multicols}
\end{question}
}


%% PhysicsBowl 2013
%%----------------------------------------
\element{aapt}{ %% Bowl-A7
\begin{question}{bowl-2013-q12}
    It takes the Earth one day to rotate about its axis.
    Which one of the following choices best represents the time that it takes the Moon to make one rotation about its axis?
    \begin{choices}
        \wrongchoice{One day.}
        \wrongchoice{One week.}
      \correctchoice{One month.}
        \wrongchoice{One year.}
        \wrongchoice{It does not rotate at all.}
    \end{choices}
\end{question}
}


%% PhysicsBowl 2010
%%----------------------------------------
\element{aapt}{ %% Bowl-A7
\begin{question}{bowl-2010-q42}
    A comet moves in an elliptical orbit around the sun.
    As the comet moves from aphelion (the point on the orbit farthest from the sun)
        to perihelion (the point on the orbit closest to the sun),
        which of the following results is true?
    \begin{center}
    \begin{tabu}{cX[c]X[c]X[c]}
        \toprule
        \makebox[1.5em][c]{\textnumero}
            & Speed of the comet
            & Angular momentum of the  comet/sun system
            & Gravitational potential energy of the comet/sun system \\
        \bottomrule
    \end{tabu}
    \end{center}
    \begin{choices}
        \wrongchoice{\begin{tabu}{X[c]X[c]X[c]} Increases & Increases & Decreases \\ \end{tabu}}
      \correctchoice{\begin{tabu}{X[c]X[c]X[c]} Increases & Constant  & Decreases \\ \end{tabu}}
        \wrongchoice{\begin{tabu}{X[c]X[c]X[c]} Decreases & Decreases & Increases \\ \end{tabu}}
        \wrongchoice{\begin{tabu}{X[c]X[c]X[c]} Increases & Increases & Constant  \\ \end{tabu}}
        \wrongchoice{\begin{tabu}{X[c]X[c]X[c]} Constant  & Constatn  & Constant  \\ \end{tabu}}
    \end{choices}
\end{question}
}


%% PhysicsBowl 2008
%%----------------------------------------
\element{aapt}{ %% Bowl-A7
\begin{question}{bowl-2008-q19}
    Kepler's Second Law about ``sweeping out equal areas in equal time''
        can be derived most directly from which conservation law?
    \begin{multicols}{2}
    \begin{choices}
        \wrongchoice{energy}
      \correctchoice{angular momentum}
        \wrongchoice{linear momentum}
        \wrongchoice{mechanical energy}
        \wrongchoice{mass}
    \end{choices}
    \end{multicols}
\end{question}
}

\element{aapt}{ %% Bowl-A7
\begin{question}{bowl-2008-q38}
    A \SI{1200}{\kilo\gram} satellite orbits Planet $X$ in circular orbit with constant speed of \SI{5.00e3}{\meter\per\second}.
    The radius of orbit is \SI{7.50e7}{\meter}.
    What is the magnitude of the gravitational force exerted on the satellite by Planet $X$?
    \begin{multicols}{2}
    \begin{choices}
      \correctchoice{\SI{400}{\newton}}
        \wrongchoice{\SI{200}{\newton}}
        \wrongchoice{\SI{0.080}{\newton}}
        \wrongchoice{\SI{0.0127}{\newton}}
        \wrongchoice{More information is required to answer this question.}
    \end{choices}
    \end{multicols}
\end{question}
}


%% PhysicsBowl 2006
%%----------------------------------------
\element{aapt}{ %% Bowl-A7
\begin{question}{bowl-2006-q50}
    A rocket is in a circular orbit with speed $v$ and orbital radius $R$ around a heavy stationary mass.
    An external impulse is quickly applied to the rocket directly opposite to the velocity and the rocket's speed is slowed to $\frac{v}{2}$,
        putting the rocket into an elliptical orbit.
    In terms of $R$,
        the size of the semi-major axis a of this new elliptical orbit is:
    \begin{multicols}{2}
    \begin{choices}
        \wrongchoice{$a = \dfrac{1}{4} R$}
        \wrongchoice{$a = \dfrac{1}{2} R$}
      \correctchoice{$a = \dfrac{4}{7} R$}
        \wrongchoice{$a = \dfrac{7}{11} R$}
        \wrongchoice{$a = \dfrac{\sqrt{8}}{3} R$}
    \end{choices}
    \end{multicols}
\end{question}
}


%% PhysicsBowl 2000
%%----------------------------------------
\element{aapt}{ %% Bowl-A7
\begin{question}{bowl-2000-q12}
    Astronauts in an orbiting space shuttle are ``weightless'' because:
    \begin{choices}
        \wrongchoice{of their extreme distance from the earth}
        \wrongchoice{the net gravitational force on them is zero}
        \wrongchoice{there is no atmosphere in space}
        \wrongchoice{the space shuttle does not rotate}
      \correctchoice{they are in a state of free fall}
    \end{choices}
\end{question}
}

\element{aapt}{ %% Bowl-A7
\begin{question}{bowl-2000-q19}
    An astronaut in an orbiting space craft attaches a mass $m$
        to a string and whirls it around in uniform circular motion.
    The radius of the circle is $r$, the speed of the mass is $v$,
        and the tension in the string is $F$.
    If the mass, radius, and speed were all to double the tension
        required to maintain uniform circular motion would be
    \begin{multicols}{3}
    \begin{choices}
        \wrongchoice{$\dfrac{F}{2}$}
        \wrongchoice{$F$}
        \wrongchoice{$2F$}
      \correctchoice{$4F$}
        \wrongchoice{$8F$}
    \end{choices}
    \end{multicols}
\end{question}
}

\element{aapt}{ %% Bowl-A7
\begin{question}{bowl-2000-q24}
    Consider an object that has a mass, $m$,
        and a weight, $W$, at the surface of the moon.
    If we assume the moon has a nearly uniform density,
        which of the following would be closest to the object's mass
        and weight at a distance halfway between Moon’s center and its surface?
    \begin{multicols}{2}
    \begin{choices}
        \wrongchoice{$\num{1/2} m$ and $\num{1/2} W$}
        \wrongchoice{$\num{1/4} m$ and $\num{1/4} W$}
        \wrongchoice{$\num{1} m$ and $\num{1} W$}
      \correctchoice{$\num{1} m$ and $\num{1/2} W$}
        \wrongchoice{$\num{1} m$ and $\num{1/4} W$}
    \end{choices}
    \end{multicols}
\end{question}
}


%% PhysicsBowl 1999
%%----------------------------------------
\element{aapt}{ %% Bowl-A7
\begin{question}{bowl-1999-q11}
    Assume that the Earth attracts John Glenn with a gravitational
        force $F$ at the surface of the Earth.
    When he made his famous second flight in orbit, the gravitational
        force on John Glenn while he was in orbit was closest to which
        of the following?
    \begin{multicols}{2}
    \begin{choices}
      \correctchoice{$\num{0.95}\,F$}
        \wrongchoice{$\num{0.50}\,F$}
        \wrongchoice{$\num{0.25}\,F$}
        \wrongchoice{$\num{0.10}\,F$}
        \wrongchoice{zero}
    \end{choices}
    \end{multicols}
\end{question}
}

\element{aapt}{ %% Bowl-A7
\begin{question}{bowl-1999-q21}
    What happens to the force of gravitational attraction
        between two small objects if the mass of each object
        is doubled and the distance between their centers is doubled?
    \begin{multicols}{2}
    \begin{choices}
        \wrongchoice{It is doubled}
        \wrongchoice{It is quadrupled}
        \wrongchoice{It is halved}
        \wrongchoice{It is reduced fourfold}
      \correctchoice{It remains the same}
    \end{choices}
    \end{multicols}
\end{question}
}

\element{aapt}{ %% Bowl-A7
\begin{question}{bowl-1999-q33}
    One object at the surface of the Moon experiences the same gravitational force as a second object at the surface of the Earth.
    Which of the following would be a reasonable conclusion?
    \begin{choices}
        \wrongchoice{both objects would fall at the same acceleration}
      \correctchoice{the object on the Moon has the greater mass}
        \wrongchoice{the object on the Earth has the greater mass}
        \wrongchoice{both objects have identical masses}
        \wrongchoice{the object on Earth has a greater mass but the Earth has a greater rate of motion.}
    \end{choices}
\end{question}
}


%% PhysicsBowl 1998
%%----------------------------------------
\element{aapt}{ %% Bowl-A7
\begin{question}{bowl-1998-q24}
    A planet has a radius one-half that of Earth and mass one-fifth the Earth's mass.
    The gravitational acceleration at the surface of the planet is most nearly:
    \begin{multicols}{3}
    \begin{choices}
        \wrongchoice{\SI{4.0}{\meter\per\second\squared}}
      \correctchoice{\SI{8.0}{\meter\per\second\squared}}
        \wrongchoice{\SI{12.5}{\meter\per\second\squared}}
        \wrongchoice{\SI{25}{\meter\per\second\squared}}
        \wrongchoice{\SI{62.5}{\meter\per\second\squared}}
    \end{choices}
    \end{multicols}
\end{question}
}

\element{aapt}{ %% Bowl-A7
\begin{question}{bowl-1998-q33}
    In the following problem,
        the word ``weight'' refers to the force a scale registers.
    If the Earth were to stop rotating, but not change shape,
    \begin{choices}
      \correctchoice{the weight of an object at the equator would increase.}
        \wrongchoice{the weight of an object at the equator would decrease.}
        \wrongchoice{the weight of an object at the north pole would increase.}
        \wrongchoice{the weight of an object at the north pole would decrease.}
        \wrongchoice{all objects on Earth would become weightless.}
    \end{choices}
\end{question}
}


%% PhysicsBowl 1997
%%----------------------------------------
\element{aapt}{ %% Bowl-A7
\begin{question}{bowl-1997-q12}
    Two artificial satellites, 1 and 2,
        are put into circular orbit at the same altitude above Earth's surface.
    The mass of satellite 2 is twice the mass of satellite 1.
    If the period of satellite 1 is $T$,
        what is the period of satellite 2?
    \begin{multicols}{3}
    \begin{choices}
        \wrongchoice{$\dfrac{T}{4}$}
        \wrongchoice{$\dfrac{T}{2}$}
      \correctchoice{$T$}
        \wrongchoice{$2T$}
        \wrongchoice{$4T$}
    \end{choices}
    \end{multicols}
\end{question}
}

\element{aapt}{ %% Bowl-A7
\begin{question}{bowl-1997-q13}
    A small sphere is moving in a vertical circle at constant speed.
    The magnitude of the net force on the sphere:
    \begin{choices}
        \wrongchoice{at the bottom of the loop is greater than the net force at the top.}
        \wrongchoice{at the top of the loop is greater than the net force at the bottom.}
        \wrongchoice{increases as the sphere moves from the bottom to the top.}
        \wrongchoice{decreases as the sphere moves from the bottom to the top.}
      \correctchoice{is the same at the top of the loop as it is at the bottom of the loop.}
    \end{choices}
\end{question}
}


%% PhysicsBowl 1996
%%----------------------------------------
\element{aapt}{ %% Bowl-A7
\begin{question}{bowl-1996-q23}
    A hypothetical planet has seven times the mass of Earth and twice the radius of Earth.
    The magnitude of the gravitational acceleration at the surface of this planet is most nearly:
    \begin{multicols}{2}
    \begin{choices}
        \wrongchoice{\SI{2.9}{\meter\per\second\squared}}
        \wrongchoice{\SI{5.7}{\meter\per\second\squared}}
      \correctchoice{\SI{17.5}{\meter\per\second\squared}}
        \wrongchoice{\SI{35}{\meter\per\second\squared}}
        \wrongchoice{\SI{122}{\meter\per\second\squared}}
    \end{choices}
    \end{multicols}
\end{question}
}


%% PhysicsBowl 1995
%%----------------------------------------
\element{aapt}{ %% Bowl-A7
\begin{question}{bowl-1995-q09}
    Two spherical bodies attract each other with a gravitational force of \SI{4.0}{\newton}.
    What will be the force if the distance between them is doubled?
    \begin{multicols}{3}
    \begin{choices}
      \correctchoice{\SI{1.0}{\newton}}
        \wrongchoice{\SI{2.0}{\newton}}
        \wrongchoice{\SI{4.0}{\newton}}
        \wrongchoice{\SI{8.0}{\newton}}
        \wrongchoice{\SI{16.0}{\newton}}
    \end{choices}
    \end{multicols}
\end{question}
}


%% PhysicsBowl 1994
%%----------------------------------------
\element{aapt}{ %% Bowl-A7
\begin{question}{bowl-1994-q34}
    A hypothetical planet orbits a star with mass one-half the mass of our sun.
    The planet's orbital radius is the same as the Earth's.
    Approximately how many Earth years does it take for the planet to complete one orbit?
    \begin{multicols}{3}
    \begin{choices}
        \wrongchoice{$\dfrac{1}{2}$}
        \wrongchoice{$\dfrac{1}{\sqrt{2}}$}
        \wrongchoice{$1$}
      \correctchoice{$\sqrt{2}$}
        \wrongchoice{$2$}
    \end{choices}
    \end{multicols}
\end{question}
}


\endinput


