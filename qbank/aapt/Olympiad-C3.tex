

%% AAPT Physics Olympiad F=ma Questions
%%----------------------------------------


%% PhysicsOlympiad 2015
%%----------------------------------------


%% PhysicsOlympiad 2000
%%----------------------------------------
\element{aapt}{ %% Olympiad-C3
\begin{question}{olympiad-2000-q09}
    If an ideal Carnot heat engine with an efficiency of \SI{30}{\percent} absorbs heat from a reservoir at \SI{727}{\degreeCelsius},
        what must be the exhaust temperature of the engine?
    \begin{multicols}{3}
    \begin{choices}
        \wrongchoice{\SI{509}{\degreeCelsius}}
      \correctchoice{\SI{427}{\degreeCelsius}}
        \wrongchoice{\SI{273}{\degreeCelsius}}
        \wrongchoice{\SI{218}{\degreeCelsius}}
        \wrongchoice{\SI{0}{\degreeCelsius}}
    \end{choices}
    \end{multicols}
\end{question}
}


%% PhysicsOlympiad 1999
%%----------------------------------------
\element{aapt}{ %% Olympiad-C3
\begin{question}{olympiad-1999-q17}
    An ideal heat engine takes in heat energy at a high temperature and exhausts energy at a lower temperature.
    If the amount of energy exhausted at the low temperature is 3 times the amount of work done by the heat engine,
        what is its efficiency?
    \begin{multicols}{3}
    \begin{choices}
      \correctchoice{\num{0.25}}
        \wrongchoice{\num{0.33}}
        \wrongchoice{\num{0.67}}
        \wrongchoice{\num{0.9}}
        \wrongchoice{\num{1.33}}
    \end{choices}
    \end{multicols}
\end{question}
}


%% PhysicsOlympiad 1997
%%----------------------------------------
\newcommand{\aaptOlympiadNinetySevenQSixteen}{
\begin{tikzpicture}
    \begin{semilogyaxis}[
        clip=false,
        axis y line=left,
        axis x line=bottom,
        axis line style={->},
        xlabel={$V$},
        xtick=\empty,
        ylabel={$p$},
        ytick=\empty,
        y label style={
            at={(current axis.above origin)},
            anchor=east,
            rotate=270,
        },
        xmin=0,xmax=10,
        ymin=0.5,ymax=16,
        width=0.8\columnwidth,
        height=0.5\columnwidth,
    ]
    \node[anchor=south] at (axis cs:2,8) {1};
    \node[anchor=west] at (axis cs:8,2) {2};
    \node[anchor=west] at (axis cs:8,0.8) {3};
    \addplot[line width=1pt,domain=2:8]{ 25/x^(1.66) };
    \draw[thick,->]  (axis cs:5.1,1.69) -- (axis cs:5,1.74);
    \addplot[line width=1pt,domain=2:8]{ 16/x };
    \draw[thick,->] (axis cs:5,3.2) -- (axis cs:5.1,3.17);
    \addplot[line width=1pt,domain=2:8] plot coordinates { (8,0.8) (8,2) };
    \draw[thick,->] (axis cs:8,1.1) -- (axis cs:8,0.9);
    \end{semilogyaxis}
\end{tikzpicture}
}

\element{aapt}{ %% Olympiad-C3
\begin{question}{olympiad-1997-q16}
    Three processes compose a thermodynamic cycle shown in the accompanying pV diagram of an ideal gas.
    \begin{center}
        \aaptOlympiadNinetySevenQSixteen
    \end{center}
    Process $1\to 2$ takes place at constant temperature (\SI{300}{\kelvin}).
    During this process \SI{60}{\joule} of heat enters the system.
    Process $2\to 3$ takes place at constant volume.
    During this process \SI{40}{\joule} of heat leaves the system.
    Process $3\to 1$ is adiabatic.
    $T_3$ is \SI{275}{\kelvin}.
    %% start question
    What is the change in internal energy of the system during process $3\to 1$?
    \begin{multicols}{3}
    \begin{choices}
        \wrongchoice{\SI{-40}{\joule}}
        \wrongchoice{\SI{-20}{\joule}}
        \wrongchoice{zero}
        \wrongchoice{\SI{+20}{\joule}}
      \correctchoice{\SI{+40}{\joule}}
    \end{choices}
    \end{multicols}
\end{question}
}

\element{aapt}{ %% Olympiad-C3
\begin{question}{olympiad-1997-q17}
    Three processes compose a thermodynamic cycle shown in the accompanying pV diagram of an ideal gas.
    \begin{center}
        \aaptOlympiadNinetySevenQSixteen
    \end{center}
    Process $1\to 2$ takes place at constant temperature (\SI{300}{\kelvin}).
    During this process \SI{60}{\joule} of heat enters the system.
    Process $2\to 3$ takes place at constant volume.
    During this process \SI{40}{\joule} of heat leaves the system.
    Process $3\to 1$ is adiabatic.
    $T_3$ is \SI{275}{\kelvin}.
    %% start question
    %What is the change in entropy of the system described in Question \#16 during the process $3\to 1$?
    What is the change in entropy of the system during the process $3\to 1$?
    \begin{multicols}{3}
    \begin{choices}
        \wrongchoice{\SI{+5.0}{\kelvin\per\joule}}
        \wrongchoice{\SI{+0.20}{\joule\per\kelvin}}
      \correctchoice{zero}
        \wrongchoice{\SI{-1.6}{\joule\per\kelvin}}
        \wrongchoice{\SI{-6.9}{\kelvin\per\joule}}
    \end{choices}
    \end{multicols}
\end{question}
}


%% PhysicsOlympiad 1996
%%----------------------------------------
\element{aapt}{ %% Olympiad-C3
\begin{question}{olympiad-1996-q17}
    Which of the accompanying $pV$ diagrams best represents an adiabatic process (process where no heat enters or leaves the system)?
    \begin{multicols}{2}
    \begin{choices}
        \AMCboxDimensions{down=-2.5em}
        \wrongchoice{
            \begin{tikzpicture}
                \begin{axis}[
                    axis y line=left,
                    axis x line=bottom,
                    axis line style={->},
                    xlabel={$V$},
                    xtick=\empty,
                    x label style={
                        at={(current axis.right of origin)},
                        anchor=west,
                    },
                    ylabel={$p$},
                    ytick=\empty,
                    y label style={
                        at={(current axis.above origin)},
                        anchor=east,
                        rotate=270,
                    },
                    xmin=0,xmax=10,
                    ymin=0,ymax=10,
                    width=0.95\columnwidth,
                ]
                \addplot[line width=1pt,domain=2:8]{6};
                \draw[thick,->] (axis cs:4,6) -- (axis cs:6,6);
                \end{axis}
            \end{tikzpicture}
        }
        \wrongchoice{
            \begin{tikzpicture}
                \begin{axis}[
                    axis y line=left,
                    axis x line=bottom,
                    axis line style={->},
                    xlabel={$V$},
                    xtick=\empty,
                    x label style={
                        at={(current axis.right of origin)},
                        anchor=west,
                    },
                    ylabel={$p$},
                    ytick=\empty,
                    y label style={
                        at={(current axis.above origin)},
                        anchor=east,
                        rotate=270,
                    },
                    xmin=0,xmax=10,
                    ymin=0,ymax=10,
                    width=0.95\columnwidth,
                ]
                \addplot[line width=1pt,no marks] plot coordinates { (1,1) (9,9) };
                \draw[thick,->] (axis cs:4,4) -- (axis cs:6,6);
                \end{axis}
            \end{tikzpicture}
        }
        %% ANS is C
        \correctchoice{
            \begin{tikzpicture}
                \begin{axis}[
                    axis y line=left,
                    axis x line=bottom,
                    axis line style={->},
                    xlabel={$V$},
                    xtick=\empty,
                    x label style={
                        at={(current axis.right of origin)},
                        anchor=west,
                    },
                    ylabel={$p$},
                    ytick=\empty,
                    y label style={
                        at={(current axis.above origin)},
                        anchor=east,
                        rotate=270,
                    },
                    xmin=0,xmax=10,
                    ymin=0,ymax=10,
                    width=0.95\columnwidth,
                ]
                \addplot[line width=1pt,domain=1:8]{ 22/x^(1.66) };
                \draw[thick,->] (axis cs:5,1.52) -- (axis cs:5.2,1.425);
                \end{axis}
            \end{tikzpicture}
        }
        \wrongchoice{
            \begin{tikzpicture}
                \begin{axis}[
                    axis y line=left,
                    axis x line=bottom,
                    axis line style={->},
                    xlabel={$V$},
                    xtick=\empty,
                    x label style={
                        at={(current axis.right of origin)},
                        anchor=west,
                    },
                    ylabel={$p$},
                    ytick=\empty,
                    y label style={
                        at={(current axis.above origin)},
                        anchor=east,
                        rotate=270,
                    },
                    xmin=0,xmax=10,
                    ymin=0,ymax=10,
                    width=0.95\columnwidth,
                ]
                \addplot[line width=1pt,domain=1:8]{ x^(1.66)/3.5 };
                \draw[thick,->] (axis cs:5,4.13) -- (axis cs:5.2,4.41);
                \end{axis}
            \end{tikzpicture}
        }
        \wrongchoice{
            \begin{tikzpicture}
                \begin{axis}[
                    axis y line=left,
                    axis x line=bottom,
                    axis line style={->},
                    xlabel={$V$},
                    xtick=\empty,
                    x label style={
                        at={(current axis.right of origin)},
                        anchor=west,
                    },
                    ylabel={$p$},
                    ytick=\empty,
                    y label style={
                        at={(current axis.above origin)},
                        anchor=east,
                        rotate=270,
                    },
                    xmin=0,xmax=10,
                    ymin=0,ymax=10,
                    width=0.95\columnwidth,
                ]
                \addplot[line width=1pt,no marks] plot coordinates { (5,1) (5,9) };
                \draw[thick,->] (axis cs:5,4) -- (axis cs:5,6);
                \end{axis}
            \end{tikzpicture}
        }
    \end{choices}
    \end{multicols}
\end{question}
}


%% PhysicsOlympiad 1995
%%----------------------------------------
\element{aapt}{ %% Olympiad-C3
\begin{question}{olympiad-1995-q17}
    If the heat is added at constant volume,
        \SI{6300}{\joule} of heat are required to raise the temperature of an ideal gas by \SI{150}{\kelvin}.
    If instead, the heat is added at constant pressure,
        \SI{8800}{\joule} are needed for the same temperature change.
    When the temperature of the gas changes by \SI{150}{\kelvin},
        the internal energy of the gas changes by:
    \begin{multicols}{3}
    \begin{choices}
        \wrongchoice{\SI{2500}{\joule}}
      \correctchoice{\SI{6300}{\joule}}
        \wrongchoice{\SI{8800}{\joule}}
        \wrongchoice{\SI{11 300}{\joule}}
        \wrongchoice{\SI{15 100}{\joule}}
    \end{choices}
    \end{multicols}
\end{question}
}


%% PhysicsOlympiad 1994
%%----------------------------------------
\element{aapt}{ %% Olympiad-C3
\begin{question}{olympiad-1994-q17}
    A Carnot cycle takes in \SI{1000}{\joule} of heat at a high temperature of \SI{400}{\kelvin}.
    How much heat is expelled at the cooler temperature of \SI{300}{\kelvin}?
    \begin{multicols}{3}
    \begin{choices}
        \wrongchoice{zero}
        \wrongchoice{\SI{250}{\joule}}
        \wrongchoice{\SI{500}{\joule}}
      \correctchoice{\SI{750}{\joule}}
        \wrongchoice{\SI{1000}{\joule}}
    \end{choices}
    \end{multicols}
\end{question}
}

\element{aapt}{ %% Olympiad-C3
\begin{question}{olympiad-1994-q18}
    An ideal gas is expanded at constant pressure from initial volume $V_i$ and temperature $T_i$ to final volume $V_f$ and temperature $T_f$.
    The gas has molar heat capacity $C_P$ at constant pressure.
    The amount of work done by $n$ moles of the gas during the process can be expressed:
    \begin{multicols}{2}
    \begin{choices}
        \wrongchoice{zero}
        \wrongchoice{$nRT\ln\left(\dfrac{V_f}{V_i}\right)$}
        \wrongchoice{$C_P\left(T_f-T_i\right)$}
        \wrongchoice{$nR\left(V_f-V_i\right)$}
      \correctchoice{$nR\left(T_f-T_i\right)$}
    \end{choices}
    \end{multicols}
\end{question}
}


\endinput


