

%% CAP High School Prize Examination
%%----------------------------------------


%% this section contains 40 problems


%% CAP Exam 2015
%%----------------------------------------
\element{cap}{ %% cap-A7
\begin{question}{CAP-A-2015-q11}
    A sentence from a book by a famous bestselling author Dan Brown:
    \begin{quote}
        The pilot nodded. 
        ``Altitude sickness. 
        We were at sixty thousand feet. 
        You're thirty percent lighter up there. 
        Lucky we only did a puddle jump. 
        If we'd gone to Tokyo I'd have taken her all the way up a hundred miles. 
        Now that'll get your insides rolling.''
    \end{quote}
    Is the statement correct?
    \begin{choices}[o]
        \wrongchoice{Yes}
      \correctchoice{No, the distance from the surface of the Earth have to be about \SI{1000}{\kilo\meter} for a person to feel \SI{30}{\percent} lighter.}
        \wrongchoice{No, the person will not feel lighter at higher altitude since the upward force of airplanes engine compensates for the lack of gravity.}
        \wrongchoice{No both arguments (b) and (c) are true.}
    \end{choices}
\end{question}
}

\element{cap}{ %% cap-A7
\begin{question}{CAP-A-2015-q24}
    An astronaut takes a pendulum up to the International Space Station (ISS). 
    The ISS orbits \SI{330}{\kilo\meter} above the surface of the Earth. 
    The radius of the Earth is \SI{6371}{\kilo\meter}. 
    Compared to when it is at ground level,
        the pendulum on the ISS swings with a period that is:
    \begin{choices}
        \wrongchoice{\num{49.3e-3} times as long.}
        \wrongchoice{0.95 times as long.}
        \wrongchoice{1.05 times longer.}
        \wrongchoice{20.3 times longer.}
      \correctchoice{The pendulum does not swing on the ISS.}
    \end{choices}
\end{question}
}


%% CAP Exam 2014
%%----------------------------------------
\element{cap}{ %% cap-A7
\begin{question}{CAP-A-2014-q03}
    How does the magnitude of the gravitational force with which the Moon attracts the Earth compare to the magnitude of the gravitational force with which the Earth attracts the Moon?
    \begin{choices}
      \correctchoice{They are equal.}
        \wrongchoice{The first is greater.}
        \wrongchoice{The first is smaller.}
    \end{choices}
\end{question}
}

\element{cap}{ %% cap-A7
\begin{question}{CAP-A-2014-q10}
    Two artificial satellites, named Argo and Navis,
        have circular orbits of radii $R$ and $2R$,
        respectively, about the same planet. 
    The orbital speed of Argo is $v$. 
    What is the orbital speed of Navis?
    \begin{multicols}{3}
    \begin{choices}
        \wrongchoice{$\dfrac{v}{2}$}
      \correctchoice{$\dfrac{v}{\sqrt{2}}$}
        \wrongchoice{$v$}
        \wrongchoice{$v\sqrt{2}$}
        \wrongchoice{$2v$}
    \end{choices}
    \end{multicols}
\end{question}
}

\element{cap}{ %% cap-A7
\begin{question}{CAP-A-2014-q19}
    In a binary star system consisting of two stars of equal mass,
        where is the gravitational potential equal to zero? 
    Assume that for a single star in empty space,
        the potential is zero at infinity.
    \begin{choices}
        \wrongchoice{Exactly halfway between the stars.}
        \wrongchoice{Along a line bisecting the line connecting the stars and perpendicular to the plane of the stars' orbit.}
      \correctchoice{Infinitely far from the stars.}
        \wrongchoice{At any point on a plane bisecting the line connecting the stars and perpendicular to the plane of the stars' orbit.}
    \end{choices}
\end{question}
}


%% CAP Exam 2013
%%----------------------------------------
\element{cap}{ %% cap-A7
\begin{question}{CAP-A-2013-q23}
    A hypothetical planet has density $\rho$, radius $R$,
        and surface gravitational acceleration $g$. 
    What would be the acceleration due to gravity at the surface of a planet with double the radius,
        and the same planetary density?
    \begin{multicols}{3}
    \begin{choices}
        \wrongchoice{$4g$}
      \correctchoice{$2g$}
        \wrongchoice{$g$}
        \wrongchoice{$\dfrac{g}{2}$}
        \wrongchoice{$\dfrac{g}{4}$}
    \end{choices}
    \end{multicols}
\end{question}
}


%% CAP Exam 2012
%%----------------------------------------
\element{cap}{ %% cap-A7
\begin{question}{CAP-A-2012-q03}
    An Earth satellite revolves in a circular orbit at a height $h$ from the surface of the Earth. 
    If $R$ is the Earth's radius and $g$ is the acceleration due to gravity at the surface of the Earth,
        then the speed of the satellite is given by:
    \begin{multicols}{2}
    \begin{choices}
        \wrongchoice{$\sqrt{gR}$}
        \wrongchoice{$\sqrt{g \left(R+h\right)}$}
      \correctchoice{$\sqrt{g \dfrac{R^2}{R+h}}$}
        \wrongchoice{$\sqrt{g \dfrac{\left(R+h\right)^2}{R}}$}
    \end{choices}
    \end{multicols}
\end{question}
}


%% CAP Exam 2009
%%----------------------------------------
\element{cap}{ %% cap-A7
\begin{question}{CAP-A-2009-q08}
    An astronaut lifts off from planet Zuton in a spaceship.
    The free-fall acceleration on Zuton is four times less than on the Earth.
    At the moment of liftoff the acceleration of the spaceship is \SI{2.45}{\meter\per\second\squared} (up).
    The weight of the astronaut at that instant is more than her weight on the surface of the earth by the factor of:
    \begin{multicols}{3}
    \begin{choices}
        \wrongchoice{$4$}
        \wrongchoice{$2$}
        \wrongchoice{$1$}
      \correctchoice{$0.5$}
        \wrongchoice{$0.25$}
    \end{choices}
    \end{multicols}
\end{question}
}

\element{cap}{ %% cap-A7
\begin{question}{CAP-A-2009-q13}
    A spacecraft of mass $m$ orbits a planet of mass $M$ in a circular orbit of radius $R$.
    What is the minimum energy required to send this spacecraft to a distant point in space where the gravitational force of the planet on the spacecraft is negligible?
    \begin{multicols}{2}
    \begin{choices}
        \wrongchoice{$\dfrac{GmM}{4R}$}
        \wrongchoice{$\dfrac{GmM}{R}$}
      \correctchoice{$\dfrac{GmM}{2R}$}
        \wrongchoice{$\dfrac{GmM}{3R}$}
        \wrongchoice{$\dfrac{2GmM}{5R}$}
    \end{choices}
    \end{multicols}
\end{question}
}


\endinput


