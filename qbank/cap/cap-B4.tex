

%% CAP High School Prize Examination
%%----------------------------------------


%% this section contains 40 problems


%% CAP Exam 2015
%%----------------------------------------
\element{cap}{ %% cap-B4
\begin{question}{CAP-A-2015-q22}
    Students are watching a science fiction movie where the crew of one spaceship watches an explosion on the other spaceship. 
    After a short time interval the crew hear the sound of the explosion. 
    The students think it is unphysical.
    \begin{choices}
        \wrongchoice{The students are right, because there should be no delay: the sound propagates in vacuum at the speed of light.}
        \wrongchoice{The students are wrong. The science adviser of the movie made it according to the laws of physics.}
        \wrongchoice{The students are right, because the steel body of the spaceship blocks the sound}
      \correctchoice{The students are right, because sound cannot propagate in the region between the spaceships}
    \end{choices}
\end{question}
}


%% CAP Exam 2014
%%----------------------------------------
\element{cap}{ %% cap-B4
\begin{question}{CAP-A-2014-q09}
    Two identical loudspeakers, placed close to each other,
        are supplied with the same sinusoidal voltage. 
    One can imagine a pattern around the loudspeakers with alternating areas of increased and decreased sound intensity. 
    Which of the actions below will not change the positions of these areas?
    \begin{choices}
        \wrongchoice{Moving one of the speakers.}
      \correctchoice{Changing the amplitude of the voltage.}
        \wrongchoice{Changing the frequency.}
        \wrongchoice{Replacing the air in the room with a gas of a different density.}
    \end{choices}
\end{question}
}


%% CAP Exam 2012
%%----------------------------------------
\element{cap}{ %% cap-B4
\begin{question}{CAP-A-2012-q20}
    You are on the side of the highway,
        listening to the siren of a fast-approaching ambulance. 
    When the ambulance is not moving the frequency of its siren is \SI{1000}{\hertz}.
    Which of the following graphs best describes the frequency that you hear as the ambulance approaches and then passes you?
    \begin{choices}
        %% NOTE: make graphs better
        \AMCboxDimensions{down=-4em}
        \wrongchoice{
            \begin{tikzpicture}
                \begin{axis}[
                    axis y line=left, 
                    axis x line=bottom, 
                    axis line style={->},
                    xlabel={time},
                    x unit=\si{\second},
                    xtick={0,1,2,3,4,5,6},
                    ylabel={frequency},
                    y unit=\si{\hertz},
                    ytick={920,960,1000,1040,1080},
                    xmin=0,xmax=6.5,
                    ymin=920,ymax=1090,
                    width=0.8\columnwidth,
                    height=0.5\columnwidth,
                ]
                \addplot[line width=1pt] plot coordinates { (1,1080) (5,920) };
                \end{axis}
            \end{tikzpicture}
        }
        \wrongchoice{
            \begin{tikzpicture}
                \begin{axis}[
                    axis y line=left, 
                    axis x line=bottom, 
                    axis line style={->},
                    xlabel={time},
                    x unit=\si{\second},
                    xtick={0,1,2,3,4,5,6},
                    ylabel={frequency},
                    y unit=\si{\hertz},
                    ytick={920,960,1000,1040,1080},
                    xmin=0,xmax=6.5,
                    ymin=920,ymax=1090,
                    width=0.8\columnwidth,
                    height=0.5\columnwidth,
                ]
                \addplot[line width=1pt] plot coordinates { (1,920) (5,1080) };
                \end{axis}
            \end{tikzpicture}
        }
        %% ANS is C
        \correctchoice{
            \begin{tikzpicture}
                \begin{axis}[
                    axis y line=left, 
                    axis x line=bottom, 
                    axis line style={->},
                    xlabel={time},
                    x unit=\si{\second},
                    xtick={0,1,2,3,4,5,6},
                    ylabel={frequency},
                    y unit=\si{\hertz},
                    ytick={920,960,1000,1040,1080},
                    xmin=0,xmax=6.5,
                    ymin=920,ymax=1090,
                    width=0.8\columnwidth,
                    height=0.5\columnwidth,
                ]
                \addplot[line width=1pt,domain=0:6] { 940 + 120 / (1 + exp(10*(x-3)) };
                \end{axis}
            \end{tikzpicture}
        }
        \wrongchoice{
            \begin{tikzpicture}
                \begin{axis}[
                    axis y line=left, 
                    axis x line=bottom, 
                    axis line style={->},
                    xlabel={time},
                    x unit=\si{\second},
                    xtick={0,1,2,3,4,5,6},
                    ylabel={frequency},
                    y unit=\si{\hertz},
                    ytick={920,960,1000,1040,1080},
                    xmin=0,xmax=6.5,
                    ymin=920,ymax=1090,
                    width=0.8\columnwidth,
                    height=0.5\columnwidth,
                ]
                \draw (axis cs:0,1060) to[out=0,in=180] (axis cs:6,940);
                \end{axis}
            \end{tikzpicture}
        }
    \end{choices}
\end{question}
}


%% CAP Exam 2009
%%----------------------------------------
\element{cap}{ %% cap-B4
\begin{question}{CAP-A-2009-q07}
    An object is moving at a constant speed $v_0$ towards a source that is at rest and that is emitting sound waves of frequency $f_0$.
    The frequency of the echo that returns to the source after being reflected from the object is given by:
    \begin{multicols}{2}
    \begin{choices}
        \wrongchoice{$f_{\text{echo}} = f_0 \dfrac{v}{v-v_0}$}
        \wrongchoice{$f_{\text{echo}} = f_0 \dfrac{v-v_0}{v+v_0}$}
      \correctchoice{$f_{\text{echo}} = f_0 \dfrac{v+v_0}{v-v_0}$}
        \wrongchoice{$f_{\text{echo}} = f_0 \dfrac{v+v_0}{v}$}
    \end{choices}
    \end{multicols}
\end{question}
}


\endinput


