

%% CAP High School Prize Examination
%%----------------------------------------


%% this section contains 40 problems


%% CAP Exam 2015
%%----------------------------------------
\element{cap}{ %% CAP-D2
\begin{question}{CAP-A-2015-q02}
    Consider a circuit made of a wire with uniform resistance in a shape of a circle as shown in the picture. 
    The circle is connected diagonally from point $A$ to point $B$ with the same type of wire. 
    \begin{center}
    \begin{circuitikz}[scale=0.95]
        %% left branch
        \draw[thick] (4,0) to [short,i=$i_0$] (2,0);
        %% center circle
        \draw[thick] (0,0) circle (2cm);
        %% wire AB
        \draw[thick] (240:2) -- (60:2);
        \node[anchor=north east] at  (240:2) {$A$};
        \node[anchor=south west] at  (60:2) {$B$};
        \draw[dashed] (-1,0) -- (1,0);
        \draw[<->] (0.66,0) arc (0:60:0.66) node[pos=0.5,anchor=west] {$\theta$};
        %% right branch
        \draw[thick] (-2,0) to [short,i=$i_0$] (-4,0);
    \end{circuitikz}
    \end{center}
    If the current passing through the circuit is $i_0$,
        what is the current passing through the wire $AB$ as a function of angle $\theta$?
    \begin{multicols}{2}
    \begin{choices}
        \wrongchoice{zero}
        \wrongchoice{$\dfrac{\theta}{\pi-\theta} i_0$}
        \wrongchoice{$\dfrac{\pi-2\theta}{\pi+2} i_0$}
      \correctchoice{$\dfrac{\pi-2\theta}{\pi+4} i_0$}
    \end{choices}
    \end{multicols}
\end{question}
}

\element{cap}{ %% cap-D2
\begin{questionmult}{CAP-A-2014-q01}
    Consider the electric power dissipation due to resistance in a circuit. 
    Which of the following changes leave the dissipated electric power unchanged?
    \begin{choices}
      \correctchoice{Doubling the voltage and reducing the current by a factor of two.}
      \correctchoice{Doubling the voltage and increasing the resistance by a factor of four.}
      \correctchoice{Doubling the current and reducing the resistance by a factor of four.}
        %\wrongchoice{None of the above.}
        %\wrongchoice{Both (b) and (c) are correct.}
        %\correctchoice{(a),(b) and (c) are correct.}
    \end{choices}
\end{questionmult}
}


%% CAP Exam 2014
%%----------------------------------------
\element{cap}{ %% cap-D2
\begin{question}{CAP-A-2014-q08}
    In the circuit shown below,
        the resistance $R_1$ is increased. 
    \begin{center}
    \ctikzset{bipoles/length=1.00cm}
    \begin{circuitikz}
        \draw (0,0) to [battery] (4,0) to (4,2) to [R,l=$R_2$] (2,2) to [R,l=$R_1$] (0,2) to (0,0);
    \end{circuitikz}
    \end{center}
    What happens to the magnitude of the potential difference across $R_1$?
    \begin{choices}
      \correctchoice{It increases.}
        \wrongchoice{It decreases.}
        \wrongchoice{It remains the same.}
    \end{choices}
\end{question}
}

\element{cap}{ %% cap-D2
\begin{question}{CAP-A-2014-q18}
    A circuit contains nothing but a battery of voltage $V$ wired to three resistors of resistance $R$. 
    Which of the following cannot be the power dissipated in the circuit
        (assuming negligible resistance for the wires)?
    \begin{multicols}{2}
    \begin{choices}
        \wrongchoice{$P = \dfrac{V^2}{3 R}$}
        \wrongchoice{$P = \dfrac{3 V^2}{R}$}
        \wrongchoice{$P = \dfrac{3 V^2}{2 R}$}
        \wrongchoice{$P = \dfrac{2 V^2}{3 R}$}
      \correctchoice{All of the provided are possible}
    \end{choices}
    \end{multicols}
\end{question}
}

\element{cap}{ %% cap-D2
\begin{question}{CAP-A-2014-q23}
    The three electric heaters in the following circuit all have the same resistance. 
    \begin{center}
    \ctikzset{bipoles/length=1.00cm}
    \begin{circuitikz}[]
        \draw (0,0) to [battery] (4,0) to (4,1) to [R,l=$C$] (2,1) to [R,l=$B$] (0,1) to (0,0);
        \draw (4,1) to (4,2) to [R,l=$A$] (0,2) to (0,1);
    \end{circuitikz}
    \end{center}
    Given that the total heat emitted by a heater is proportional to the power dissipated,
        the total heat produced by $B$ and $C$ together,
        compared with the heat produced in $A$, is:
    \begin{choices}
        \wrongchoice{A quarter as much.}
      \correctchoice{Half as much.}
        \wrongchoice{The same.}
        \wrongchoice{Twice as much.}
        \wrongchoice{Four times as much.}
    \end{choices}
\end{question}
}


%% CAP Exam 2012
%%----------------------------------------
\element{cap}{ %% cap-D2
\begin{question}{CAP-A-2012-q05}
    Three batteries are placed in series. 
    Each battery has an internal resistance $r$. 
    If one of the batteries is placed the wrong way around as shown in the picture,
    \begin{center}
    \ctikzset{bipoles/length=1.00cm}
    \begin{circuitikz}[scale=1.1]
        %% nodes A and B
        \draw[fill] (-3,0) circle (2pt);
        \draw[fill] (+3,0) circle (2pt);
        %% Circuit
        \draw (-1,0) to [battery,v=$ $] (-3,0);
        \draw (-1,0) to [battery,v=$ $] (1,0) to [battery,v=$ $] (3,0);
    \end{circuitikz}
    \end{center}
        what will be the total resistance of the three cells now?
    \begin{multicols}{3}
    \begin{choices}
        \wrongchoice{$\dfrac{r}{3}$}
        \wrongchoice{$\dfrac{r}{2}$}
        \wrongchoice{$r$}
        \wrongchoice{$2r$}
      \correctchoice{$3r$}
    \end{choices}
    \end{multicols}
\end{question}
}


%% CAP Exam 2011
%%----------------------------------------
\element{cap}{ %% cap-D2
\begin{question}{CAP-A-2011-q23}
    Rank in order, from brightest to dimmest,
        the identical bulbs $A$ to $D$.
    \begin{center}
    \ctikzset{bipoles/length=1.00cm}
    \begin{circuitikz}[yscale=0.8]
        \draw (4,0) to [battery] (0,0) to (0,2) to [R,l=$A$] (2,2) to [R,l=$B$] (4,2) to (4,4) to (4,0);
        \draw (0,2) to (0,4) to [R,l=$C$] (4,4) to (4,2);
        \draw (0,4) to (0,6) to [R,l=$D$] (4,6) to (4,4);
    \end{circuitikz}
    \end{center}
    \begin{choices}
        \wrongchoice{$A = B = C = D$}
        \wrongchoice{$A = B > C = D$}
        \wrongchoice{$A > C > B > D$}
        \wrongchoice{$A > C = D > B$}
      \correctchoice{$C = D > B = A$}
    \end{choices}
\end{question}
}


%% CAP Exam 2009
%%----------------------------------------
\element{cap}{ %% cap-D2
\begin{question}{CAP-A-2009-q04}
    In the two circuits shown below,
        the batteries are identical and maintain constant voltage.
    The light bulbs $A$, $B$, $C$ and $D$, are identical and have resistance $R$.
    \begin{center}
    \ctikzset{bipoles/length=0.75cm}
    \begin{circuitikz}
        %% Left
        \begin{scope}[xshift=-1.5cm,xscale=0.8]
            \draw (0,0) to [battery] (0,3) to (2,3) to [R,l=$A$] (2,1.5) to [R,l=$B$] (2,0) to (0,0);
        \end{scope}
        %% Right
        \begin{scope}[xshift=+1.5cm,xscale=0.8]
            \draw (0,0) to [battery] (0,3) to (2,3) to [R,l=$C$] (2,0) to (0,0);
            \draw (2,3) to (4,3) to [R,l=$D$] (4,0) to (2,0);
        \end{scope}
    \end{circuitikz}
    \end{center}
    Assume that the bulbs are brighter when there is more current flowing through them.
    Which of the following relationships correctly describe the brightness of the bulbs?
    \begin{choices}
        \wrongchoice{$A = B > C = D$}
      \correctchoice{$C = D > A = B$}
        \wrongchoice{$A = B = C = D$}
        \wrongchoice{$A = C > B = D$}
    \end{choices}
\end{question}
}


\endinput


