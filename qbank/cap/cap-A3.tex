

%% CAP High School Prize Examination
%%----------------------------------------


%% this section contains 40 problems


%% CAP Exam 2015
%%----------------------------------------
\element{cap}{ %% CAP-A3
\begin{question}{CAP-A-2015-q01}
    As shown in the picture, a ball is attached to a ceiling and a wall with massless ropes $A$ and $B$.
    Rope $A$ is at angle $\theta$ from the vertical direction, and rope $B$ is horizontal. 
    The system is static. 
    \begin{center}
    \begin{tikzpicture}
        %% ceiling, A
        \draw (1,0) -- (3,0);
        \node[anchor=south,fill,pattern=north east lines,minimum width=2cm, minimum height=0.05cm] at (2,0) {};
        %% wall, B
        \draw (-2,-4) -- (-2,-2);
        \node[anchor=east,fill,pattern=north east lines,minimum width=0.05cm, minimum height=2cm] at (-2,-3) {};
        %% String A and B
        \draw[thick] (0,-3) -- (2,0) node[pos=0.5,anchor=south east] {$A$};
        \draw[thick] (-2,-3) -- (0,-3) node[pos=0.5,anchor=south] {$B$};
        %% Pendulum
        \draw[fill=white!90!black] (0,-3) circle (10pt);
        %% angle
        \draw[dashed] (2,0) -- (2,-1.5);
        \draw[<->] (2,-1) arc (270:238:1) node[pos=0.5,anchor=north] {$\theta$};
    \end{tikzpicture}
    \end{center}
    If rope $B$ is cut,
        what is the ratio of the tension in rope $A$ immediately after it is cut to the tension in $A$ before it is cut?
    \begin{multicols}{2}
    \begin{choices}
        \wrongchoice{$1$}
        \wrongchoice{$\cos\theta$}
        \wrongchoice{$\dfrac{1}{\cos\theta}$}
      \correctchoice{$\cos 2\theta$}
        \wrongchoice{$\cos\theta \sin\theta$}
    \end{choices}
    \end{multicols}
\end{question}
}

\element{cap}{ %% CAP-A3
\begin{question}{CAP-A-2015-q03}
    The following graph represents the speed of a car as a function of time. 
    We know that as the car speeds up there is a friction force with air that can be approximately considered to be proportional to the speed of the car.
    \begin{center}
    \begin{tikzpicture}
        \begin{axis}[
            axis y line=left, 
            axis x line=bottom, 
            axis line style={->},
            xlabel={time},
            xtick={3},
            xticklabels={$t_0$},
            x label style={
                at={(current axis.right of origin)},
                anchor=north east,
            },
            ylabel={velocity},
            ytick=\empty,
            xmin=0,xmax=11,
            ymin=0,ymax=11,
            width=0.8\columnwidth,
            height=0.5\columnwidth,
        ]
        \addplot[line width=1pt] plot coordinates { (0,0) (3,9) (10,9) };
        \addplot[line width=1pt,dashed] plot coordinates { (3,9) (3,0) };
        \end{axis}
    \end{tikzpicture}
    \end{center}
    Which of the following graphs can be the force of the engine as a function of time?
    \begin{multicols}{2}
    \begin{choices}
        \AMCboxDimensions{down=-2.5em}
        \wrongchoice{
            \begin{tikzpicture}
                \begin{axis}[
                    axis y line=left, 
                    axis x line=bottom, 
                    axis line style={->},
                    xlabel={time},
                    xtick={3},
                    xticklabels={$t_0$},
                    x label style={
                        at={(current axis.right of origin)},
                        anchor=north east,
                    },
                    ylabel={velocity},
                    ytick=\empty,
                    xmin=0,xmax=11,
                    ymin=0,ymax=11,
                    width=1.0\columnwidth,
                ]
                \addplot[line width=1pt] plot coordinates { (0,9) (3,9) };
                \addplot[line width=1pt,dashed] plot coordinates { (3,9) (3,0) };
                \addplot[line width=1pt] plot coordinates { (3,5) (10,5) };
                \end{axis}
            \end{tikzpicture}
        }
        \wrongchoice{
            \begin{tikzpicture}
                \begin{axis}[
                    axis y line=left, 
                    axis x line=bottom, 
                    axis line style={->},
                    xlabel={time},
                    xtick={3},
                    xticklabels={$t_0$},
                    x label style={
                        at={(current axis.right of origin)},
                        anchor=north east,
                    },
                    ylabel={velocity},
                    ytick=\empty,
                    xmin=0,xmax=11,
                    ymin=0,ymax=11,
                    width=1.0\columnwidth,
                ]
                \addplot[line width=1pt] plot coordinates { (0,3) (3,9) };
                \addplot[line width=1pt,dashed] plot coordinates { (3,9) (3,0) };
                \addplot[line width=1pt] plot coordinates { (3,2) (10,2) };
                \end{axis}
            \end{tikzpicture}
        }
        %% ANS is C
        \correctchoice{
            \begin{tikzpicture}
                \begin{axis}[
                    axis y line=left, 
                    axis x line=bottom, 
                    axis line style={->},
                    xlabel={time},
                    xtick={3},
                    xticklabels={$t_0$},
                    x label style={
                        at={(current axis.right of origin)},
                        anchor=north east,
                    },
                    ylabel={velocity},
                    ytick=\empty,
                    xmin=0,xmax=11,
                    ymin=0,ymax=11,
                    width=1.0\columnwidth,
                ]
                \addplot[line width=1pt] plot coordinates { (0,3) (3,9) };
                \addplot[line width=1pt,dashed] plot coordinates { (3,9) (3,0) };
                \addplot[line width=1pt] plot coordinates { (3,6) (10,6) };
                \end{axis}
            \end{tikzpicture}
        }
        \wrongchoice{
            \begin{tikzpicture}
                \begin{axis}[
                    axis y line=left, 
                    axis x line=bottom, 
                    axis line style={->},
                    xlabel={time},
                    xtick={3},
                    xticklabels={$t_0$},
                    x label style={
                        at={(current axis.right of origin)},
                        anchor=north east,
                    },
                    ylabel={velocity},
                    ytick=\empty,
                    xmin=0,xmax=11,
                    ymin=0,ymax=11,
                    width=1.0\columnwidth,
                ]
                \addplot[line width=1pt,domain=0:3] { 3 + 0.66 * x * x };
                \addplot[line width=1pt,dashed] plot coordinates { (3,9) (3,0) };
                \addplot[line width=1pt] plot coordinates { (3,6) (10,6) };
                \end{axis}
            \end{tikzpicture}
        }
        \wrongchoice{
            \begin{tikzpicture}
                \begin{axis}[
                    axis y line=left, 
                    axis x line=bottom, 
                    axis line style={->},
                    xlabel={time},
                    xtick={3},
                    xticklabels={$t_0$},
                    x label style={
                        at={(current axis.right of origin)},
                        anchor=north east,
                    },
                    ylabel={velocity},
                    ytick=\empty,
                    xmin=0,xmax=11,
                    ymin=0,ymax=11,
                    width=1.0\columnwidth,
                ]
                \addplot[line width=1pt,domain=0:3] { 3 + 0.66 * x * x };
                \addplot[line width=1pt,dashed] plot coordinates { (3,9) (3,0) };
                \addplot[line width=1pt] plot coordinates { (3,3) (10,3) };
                \end{axis}
            \end{tikzpicture}
        }
    \end{choices}
    \end{multicols}
\end{question}
}


%% CAP Exam 2014
%%----------------------------------------
\element{cap}{ %% cap-A3
\begin{question}{CAP-A-2014-q24}
    A box sits on a horizontal surface that exerts a normal force $N$ on the box. 
    You apply a horizontal force to it and it does not move. 
    If you had applied a force twice as large,
        it still would not have moved. 
    Let $\mu_s$ be the coefficient of static friction of the surface. 
    While you are applying your initial force,
        which of the following is true of the force of friction acting on the box?
    \begin{multicols}{2}
    \begin{choices}
        \wrongchoice{$F_f = 0$}
      \correctchoice{$F_f <    \mu_s N$}
        \wrongchoice{$F_f \leq \mu_s N$}
        \wrongchoice{$F_f =    \mu_s N$}
        \wrongchoice{$F_f \geq \mu_s N$}
        \wrongchoice{$F_f >    \mu_s N$}
    \end{choices}
    \end{multicols}
\end{question}
}


%% CAP Exam 2013
%%----------------------------------------
\element{cap}{ %% cap-A3
\begin{question}{CAP-A-2013-q04}
    A car driven at a constant speed turns left. 
    Which force makes the car turn?
    \begin{choices}
      \correctchoice{The force of friction, directed towards the left.}
        \wrongchoice{The force of friction, directed towards the right.}
        \wrongchoice{The force with which the driver is turning the steering wheel, directed towards the left.}
        \wrongchoice{The force with which the driver is turning the steering wheel, directed towards the right.}
    \end{choices}
\end{question}
}

\element{cap}{ %% cap-A3
\begin{question}{CAP-A-2013-q05}
    A cyclist stops pedaling at a velocity $v=\SI{36}{\kilo\meter\per\hour}$ and notices that her bike keeps moving for $d=\SI{500}{\meter}$ before it stops. 
    The total mass of the bike, the biker and her camping equipment is $m=\SI{100}{\kilo\gram}$.
    What is the average combined force on the bicycle due to friction and drag?
    \begin{multicols}{2}
    \begin{choices}
      \correctchoice{\SI{10}{\newton}}
        \wrongchoice{\SI{20}{\newton}}
        \wrongchoice{\SI{130}{\newton}}
        \wrongchoice{\SI{260}{\newton}}
    \end{choices}
    \end{multicols}
\end{question}
}

\element{cap}{ %% cap-A3
\begin{question}{CAP-A-2013-q06}
    A child is swinging to and fro on a playground swing.
    At the instant the chains of the swing are vertical,
        what is the direction of the child's acceleration?
    \begin{choices}
        \wrongchoice{Downwards.}
      \correctchoice{Upwards.}
        \wrongchoice{In the direction of motion.}
        \wrongchoice{Opposite to the direction of motion.}
        \wrongchoice{At that moment, the acceleration is zero.}
    \end{choices}
\end{question}
}

\element{cap}{ %% cap-A3
\begin{question}{CAP-A-2013-q07}
    A bullet of mass \SI{5}{\gram} is accelerated in a rifle barrel with an approximately constant force of \SI{2500}{\newton}.
    The mass of the rifle is \SI{5}{\kilo\gram}.
    What is the force pushing the rifle back?
    \begin{multicols}{2}
    \begin{choices}
        \wrongchoice{\SI{2.5}{\newton}}
      \correctchoice{\SI{2 500}{\newton}}
        \wrongchoice{\SI{2 500 000}{\newton}}
        \wrongchoice{\SI{0}{\newton}}
    \end{choices}
    \end{multicols}
\end{question}
}

\element{cap}{ %% cap-A3
\begin{question}{CAP-A-2013-q09}
    A Ferris wheel is turning about a horizontal axis through its center. 
    The linear speed of a passenger sitting on the rim is constant. 
    Which one of the following sentences about the passenger's acceleration is correct?
    \begin{choices}
        \wrongchoice{Its magnitude at the highest point is higher than at the lowest point, and it is downwards at both points.}
        \wrongchoice{Both its magnitude and direction are the same at the highest and at the lowest points.}
      \correctchoice{Its magnitude is the same at the highest and at the lowest points, but the directions are opposite.}
        \wrongchoice{Its magnitude at the highest point is smaller than at the lowest point, and it is downwards at both points.}
    \end{choices}
\end{question}
}

\element{cap}{ %% cap-A3
\begin{question}{CAP-A-2013-q18}
    Two identical objects of mass $m$ are connected to a massless string which is hung over two frictionless pulleys as shown below. 
    %% NOTE: duplicate of olympiad-1995-q06
    \begin{center}
    \begin{tikzpicture}
        %% Ceiling
        \node[anchor=south,fill,pattern=north east lines,minimum width=8cm, minimum height=0.05cm] at (0,0) {};
        \draw (-4,0) -- (4,0);
        %% Mass Left
        \node[draw,anchor=north,minimum size=1cm,fill=white!90!black,rounded corners=0.5ex] (A) at (-2.5,-2.5) {$m$};
        %% Mass Right
        \node[draw,anchor=north,minimum size=1cm,fill=white!90!black,rounded corners=0.5ex] (B) at (+2.5,-2.5) {$m$};
        %% Rope
        \draw[very thick] (A.north) -- (-2.5,-1) arc(180:90:0.5) -- (+2,-0.5) arc(90:0:0.5) -- (B.north);
        %% Pulley Left
        \draw (-2,-1) circle (0.5cm);
        \draw[fill=white!90!black] (-2.2,0) -- (-2.1,-1.1) arc(190:350:0.1) -- (-1.8,0) -- cycle;
        \draw[fill] (-2,-1) circle (1pt);
        %% Pulley Right
        \draw (+2,-1) circle (0.5cm);
        \draw[fill=white!90!black] (+1.8,0) -- (+1.9,-1.1) arc(190:350:0.1) -- (+2.2,0) -- cycle;
        \draw[fill] (+2,-1) circle (1pt);
    \end{tikzpicture}
    \end{center}
    If everything is at rest,
        what is the tension in the cord?
    \begin{choices}
        \wrongchoice{Less than $mg$.}
      \correctchoice{Exactly $mg$.}
        \wrongchoice{More than $mg$, but less than $2mg$.}
        \wrongchoice{Exactly $2mg$.}
        \wrongchoice{More than $2mg$.}
    \end{choices}
\end{question}
}

\element{cap}{ %% cap-A3
\begin{question}{CAP-A-2013-q22}
    A ball of mass $m$ is thrown vertically upward. 
    Instead of neglecting air resistance,
        assume that the force of air resistance has a magnitude proportional to the ball's velocity,
        but pointing in the opposite direction. 
    What is the ball's acceleration at the highest point?
    \begin{choices}
        \wrongchoice{zero}
        \wrongchoice{Less than $g$.}
      \correctchoice{$g$}
        \wrongchoice{Greater than $g$.}
    \end{choices}
\end{question}
}


%% CAP Exam 2012
%%----------------------------------------
\element{cap}{ %% cap-A3
\begin{question}{CAP-A-2012-q02}
    You are holding a bottle of sparkling water inside a car moving forward. 
    When the driver applies the brakes:
    \begin{choices}
        \wrongchoice{Bubbles in the middle of the liquid will start to move forward with respect to the bottle.}
      \correctchoice{Bubbles will start to move backward with respect the bottle.}
        \wrongchoice{Bubbles will stay at the same horizontal location in the water.}
        \wrongchoice{Depending on the speed of the car, bubbles might move forward or backward.}
    \end{choices}
\end{question}
}


%% CAP Exam 2011
%%----------------------------------------
\element{cap}{ %% cap-A3
\begin{question}{CAP-A-2011-q14}
    A huge case, attached to a cable,
        is descending at a constant velocity. 
    The tension in the cable is (neglecting the air resistance):
    \begin{choices}
        \wrongchoice{greater than the weight of the case.}
        \wrongchoice{smaller than the weight of the case.}
      \correctchoice{equal to the weight of the case.}
        \wrongchoice{we cannot tell since we don't know the weight of the case.}
    \end{choices}
\end{question}
}

\element{cap}{ %% cap-A3
\begin{question}{CAP-A-2011-q16}
    When a car is starting, its driving wheels experience:
    \begin{choices}
        \wrongchoice{The force of kinetic friction directed backward.}
        \wrongchoice{The force of static friction directed backward.}
        \wrongchoice{The force of kinetic friction directed forward.}
      \correctchoice{The force of static friction directed forward.}
    \end{choices}
\end{question}
}


%% CAP Exam 2010
%%----------------------------------------
\element{cap}{ %% cap-A3
\begin{question}{CAP-A-2010-q02}
    A mass $m_1$ rests on a frictionless surface. 
    It is attached to mass $m_2$ by a light string which passes over a massless,
        frictionless pulley, as shown in the figure.
    \begin{center}
    \begin{tikzpicture}
        %% Floor and wall
        \draw (-4,2) -- (-4,0) -- (0,0) -- (0,-3);
        \node[anchor=north,fill,pattern=north east lines,minimum width=4cm, minimum height=0.05cm] at (-2,0) {};
        \node[anchor=east,fill,pattern=north east lines,minimum width=0.05cm, minimum height=2.2cm] at (-4,0.9) {};
        %% Mass
        \node[draw,fill=white!90!black,rectangle,rounded corners=1ex,minimum size=1.25cm,anchor=south] (A) at (-2,0) {$m_1$};
        \node[draw,fill=white!90!black,rectangle,rounded corners=1ex,minimum size=1.00cm,anchor=north] (B) at (1,-1) {$m_2$};
        %% Rope and Pully
        \draw[thick] (A.south east) ++(90:0.5) -- (0.75,0.5) arc(90:0:0.25) -- (B.north);
        \draw[thick,fill=white!90!black] (0.75,0.25) circle (0.25); 
        \draw[thick,fill=white] (0,0) -- (0.75,0.35) arc (90:-60:0.1) -- (0,-0.5) -- cycle;
        \draw[fill] (0.75,0.25) circle (1pt);
    \end{tikzpicture}
    \end{center}
    The system is released from rest.
    The initial acceleration a of $m_2$ is given by:
    \begin{multicols}{2}
    \begin{choices}
        \wrongchoice{$a = g$}
        \wrongchoice{$a = \left(m_2-m_1\right) g$}
        \wrongchoice{$a = \dfrac{m_2}{m_1+m_2} g$}
      \correctchoice{$a = \dfrac{m_1}{m_1+m_2} g$}
    \end{choices}
    \end{multicols}
\end{question}
}

\element{cap}{ %% cap-A3
\begin{question}{CAP-A-2010-q08}
    A block of mass $m$ is sandwiched between a rough vertical wall and a horizontally oriented spring with spring constant $k$. 
    \begin{center}
    \begin{tikzpicture}
        %% wall
        \draw (0,-2) -- (0,2);
        \node[anchor=west,fill,pattern=north east lines,minimum width=0.05cm, minimum height=4cm] at (0,0) {};
        %% block
        \node[draw,minimum height=2cm,minimum width=1.25cm,anchor=east] (A) at (0,0) {$m$};
        %% Spring
        \draw[thick,decoration={aspect=0.2,segment length=2mm,amplitude=2mm,coil},decorate]  (A.west) -- (-3,0) node[anchor=south,pos=0.5,yshift=2mm] {$k$};
        \draw[thick,->] (-5,0) -- (-3,0);
    \end{tikzpicture}
    \end{center}
    If the spring is compressed a distance $x$ beyond its uncompressed length,
        then the minimum coefficient of static friction for which the block does not fall is:
    \begin{multicols}{2}
    \begin{choices}
      \correctchoice{$\dfrac{mg}{kx}$}
        \wrongchoice{$\dfrac{kx}{mg}$}
        \wrongchoice{$\dfrac{kx}{mg + kx}$}
        \wrongchoice{$\dfrac{mg}{mg + kx}$}
    \end{choices}
    \end{multicols}
\end{question}
}

\element{cap}{ %% cap-A3
\begin{question}{CAP-A-2010-q17}
    At time $t=0$ a block of mass $m$ moves with momentum $p$ along a rough horizontal surface.
    If it comes to a complete stop in time $t$,
        then the coefficient of kinetic friction $\mu_k$ is given by:
    \begin{multicols}{2}
    \begin{choices}
      \correctchoice{$\dfrac{p}{mgt}$}
        \wrongchoice{$\dfrac{pg}{mt}$}
        \wrongchoice{$\dfrac{ptg}{m}$}
        \wrongchoice{$\dfrac{pt}{mg}$}
        \wrongchoice{none of the provided}
    \end{choices}
    \end{multicols}
\end{question}
}


%% CAP Exam 2009
%%----------------------------------------
\element{cap}{ %% cap-A3
\begin{question}{CAP-A-2009-q09}
    A \SI{0.50}{\kilo\gram} mass attached to the end of a string swings in a vertical circle with radius of \SI{2.0}{\meter}.
    When the string is horizontal,
        the speed of the mass is \SI{8.0}{\meter\per\second}.
    What is the magnitude of the force of the string on the mass at this position?
    \begin{multicols}{3}
    \begin{choices}
      \correctchoice{\SI{16}{\newton}}
        \wrongchoice{\SI{17}{\newton}}
        \wrongchoice{\SI{21}{\newton}}
        \wrongchoice{\SI{11}{\newton}}
        \wrongchoice{\SI{25}{\newton}}
    \end{choices}
    \end{multicols}
\end{question}
}


\endinput


