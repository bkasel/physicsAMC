
%%--------------------------------------------------
%% CPO: AMC Open Free Response Questions
%%--------------------------------------------------


%% Chapter 19: Harmonic Motion
%%--------------------------------------------------


%% CPO Short Answer Questions
%%--------------------------------------------------
\element{cpo-short}{
\begin{question}{ch19-short-q01}
    List three ways in which Earth might be considered to be part of an oscillating system.
    \AMCOpen{lines=3}{
        \wrongchoice[W]{w}\scoring{0}
        \wrongchoice[P]{p}\scoring{1}
        \correctchoice[C]{c}\scoring{2}
    }
\end{question}
}

\element{cpo-short}{
\begin{question}{ch19-short-q02}
    Why is harmonic motion useful for time keeping?
    \AMCOpen{lines=3}{
        \wrongchoice[W]{w}\scoring{0}
        \wrongchoice[P]{p}\scoring{1}
        \correctchoice[C]{c}\scoring{2}
    }
\end{question}
}

%% NOTE: this is really a mc question
\element{cpo-short}{
\begin{question}{ch19-short-q03}
    Point $A$ is shown on the harmonic motion graph of a vibrating string.
    Another point is located \num{1/2} of a phase away from point $A$.
    Which point on the graph shows the position of the second point?
    \begin{center}
        %% NOTE: add diagram
    \end{center}
    \AMCOpen{lines=3}{
        \wrongchoice[W]{w}\scoring{0}
        \wrongchoice[P]{p}\scoring{1}
        \correctchoice[C]{c}\scoring{2}
    }
\end{question}
}

%\element{cpo-short}{
%\begin{question}{ch19-short-q04}
%    The diagram below represents the graph produced from the motion
%        of two oscillators $A$ and $B$.
%    What is the phase difference measure in degrees between $A$ and $B$.
%    \begin{center}
%        %% NOTE: add diagram
%    \end{center}
%    \AMCOpen{lines=3}{
%        \wrongchoice[W]{w}\scoring{0}
%        \wrongchoice[P]{p}\scoring{1}
%        \correctchoice[C]{c}\scoring{2}
%    }
%    %% ANS: \ang{90}
%\end{question}
%}

%\element{cpo-short}{
%\begin{question}{ch19-short-q05}
%    The diagram below shows sound transmitted as the compression
%        and expansion and expansion of air molecules,
%        an example of harmonic motion.
%    Labeling the diagram with $X$'s, show the distance that represents
%        one cycle.
%    \begin{center}
%        %% NOTE: add diagram
%    \end{center}
%    \AMCOpen{lines=3}{
%        \wrongchoice[W]{w}\scoring{0}
%        \wrongchoice[P]{p}\scoring{1}
%        \correctchoice[C]{c}\scoring{2}
%    }
%\end{question}
%}

\element{cpo-short}{
\begin{question}{ch19-short-q06}
    What is the meaning of the term restoring force?
    In what direction does it always move?
    \AMCOpen{lines=3}{
        \wrongchoice[W]{w}\scoring{0}
        \wrongchoice[P]{p}\scoring{1}
        \correctchoice[C]{c}\scoring{2}
    }
\end{question}
}

\element{cpo-short}{
\begin{question}{ch19-short-q06}
    List three ways to change the natural frequency of a guitar's ``A'' string.
    \AMCOpen{lines=3}{
        \wrongchoice[W]{w}\scoring{0}
        \wrongchoice[P]{p}\scoring{1}
        \correctchoice[C]{c}\scoring{2}
    }
\end{question}
}


%% CPO Problem Questions
%%--------------------------------------------------
\element{cpo-problem}{
\begin{question}{ch19-problem-q01}
    If a bumble bee flaps its wings at a frequency of \num{130}
        beats per second, what is the period of vibration of
        the bee's wings?
    \AMCOpen{lines=3}{
        \wrongchoice[W]{w}\scoring{0}
        \wrongchoice[P]{p}\scoring{1}
        \correctchoice[C]{c}\scoring{2}
    }
    %% ANS: \SI{0.0077}{\second}
\end{question}
}

\element{cpo-problem}{
\begin{question}{ch19-problem-q02}
    If your heart beats at a rate of \num{65} beats per minute,
        what is the frequency of your heart measured in hertz?
    \AMCOpen{lines=3}{
        \wrongchoice[W]{w}\scoring{0}
        \wrongchoice[P]{p}\scoring{1}
        \correctchoice[C]{c}\scoring{2}
    }
    %% ANS: \SI{1.1}{\hertz}
\end{question}
}

\element{cpo-problem}{
\begin{question}{ch19-problem-q03}
    The distance between the highest and lowest position
        for an oscillator on a graph of harmonic motion
        is \SI{40}{\centi\meter}.
    What is the amplitude of the oscillator?
    \AMCOpen{lines=3}{
        \wrongchoice[W]{w}\scoring{0}
        \wrongchoice[P]{p}\scoring{1}
        \correctchoice[C]{c}\scoring{2}
    }
    %% ANS: \SI{20}{\centi\meter}
\end{question}
}

\element{cpo-problem}{
\begin{question}{ch19-problem-q04}
    An oscillator with a frequency of \SI{0.25}{\hertz} is set in motion.
    One second later, an identical oscillator is set in motion.
    What is the phase separation of these two oscillators?
    \AMCOpen{lines=3}{
        \wrongchoice[W]{w}\scoring{0}
        \wrongchoice[P]{p}\scoring{1}
        \correctchoice[C]{c}\scoring{2}
    }
    %% ANS: \ang{90}
\end{question}
}

%\element{cpo-problem}{
%\begin{question}{ch19-problem-q05}
%    Using the diagram below, sketch a graph that represents the
%        motion of a harmonic oscillator whose frequnecy is
%        \SI{2.0}{\hertz} and amplitude is \SI{0.2}{\meter}.
%    \AMCOpen{lines=3}{
%        \wrongchoice[W]{w}\scoring{0}
%        \wrongchoice[P]{p}\scoring{1}
%        \correctchoice[C]{c}\scoring{2}
%    }
%    \begin{center}
%        \begin{tikzpicture}
%        \end{tikzpicture}
%    \end{center}
%\end{question}
%}


%% CPO Essay Questions
%%--------------------------------------------------
\element{cpo-essay}{
\begin{question}{ch19-essay-q01}
    ``Circular motion represents harmonic motion.''
    Explain why that statement might be considered a correct statement.
    \AMCOpen{lines=3}{
        \wrongchoice[W]{w}\scoring{0}
        \wrongchoice[P]{p}\scoring{1}
        \correctchoice[C]{c}\scoring{2}
    }
\end{question}
}

\element{cpo-essay}{
\begin{question}{ch19-essay-q02}
    How could you change the natural frequency of a pendulum?
    \AMCOpen{lines=3}{
        \wrongchoice[W]{w}\scoring{0}
        \wrongchoice[P]{p}\scoring{1}
        \correctchoice[C]{c}\scoring{2}
    }
\end{question}
}

\endinput


