
%%--------------------------------------------------
%% CPO: Multiple Choice Questions
%%--------------------------------------------------


%% Chapter 16: Magnetism
%%--------------------------------------------------


%% Learning Objectives
%%--------------------------------------------------

%% Recognize that magnetic poles always exist in pairs. 
%% Decide whether two magnetic poles will attract or repel. 
%% Describe the magnetic field and forces around a permanent magnet. 
%% Learn how to build a simple electromagnet and change its strength. 
%% Use the right-hand rule to locate an electromagnets poles. 
%% Explain the source of magnetism in materials. 
%% Explain how a compass responds to a magnetic field.
%% Describe the cause of Earth’s magnetism. 
%% Recognize the difference between Earth’s magnetic and geographic poles. 
%% Explain how a compass is used to indicate direction


%% CPO Multiple Choice Questions
%%--------------------------------------------------
\element{cpo-mc}{
\begin{question}{cpo-ch16-q01}
    Placing the north pole of a magnet near the south pole of another magnet results in:
    \begin{choices}
      \correctchoice{an attractive force between the magnets.}
        \wrongchoice{a repulsive force between the magnets.}
        \wrongchoice{an electric force between the magnets.}
        \wrongchoice{no force between the magnets.}
    \end{choices}
\end{question}
}

%\element{cpo-mc}{
%\begin{question}{cpo-ch16-q02}
%    A student places two magnets with their north poles facing
%        each other about \SI{70.0}{\centi\meter} apart.
%    When she moves one magnet toward the other, the first
%        magnet repels the second at a distance of \SI{26}{\centi\meter}.
%    She repeats the procedure, but now places the magnets so the south pole
%        of one faces the north pole at the other.
%    \begin{center}
%    \begin{tikzpicture}
%        %% NOTE: TODO: draw tikz
%    \end{tikzpicture}
%    \end{center}
%    She is likely to observe that the first magnet will:
%    \begin{choices}
%        \wrongchoice{again repel the second at a distance of \SI{26}{\centi\meter}.}
%      \correctchoice{attract the second at a distance of \SI{26}{\centi\meter}.}
%        \wrongchoice{repel the second at a distance of \SI{52}{\centi\meter}.}
%        \wrongchoice{attract the second at a distance of \SI{52}{\centi\meter}.}
%    \end{choices}
%\end{question}
%}

\element{cpo-mc}{
\begin{question}{cpo-ch16-q03}
    If a \SI{12}{inch} long bar magnet is cut into two pieces,
        one inch from the north pole end of the magnet:
    \begin{choices}
        \wrongchoice{the short piece of the magnet has a north pole on one end and no pole on the other.}
        \wrongchoice{the long piece of the magnet has a south pole on both ends.}
        \wrongchoice{the short piece has a north pole on both ends.}
      \correctchoice{both pieces have a north and a south pole on each end.}
    \end{choices}
\end{question}
}

\tikzset{
    north/.style={rectangle, draw=white, fill=black, minimum width=1cm, minimum height=0.5cm},
    south/.style={rectangle, draw=black, fill=white, minimum width=1cm, minimum height=0.5cm},
}

%\element{cpo-mc}{
%\begin{question}{cpo-ch16-q04}
%    Which pair of objects experiences a repulsive force?
%    \begin{center}
%    \begin{tikzpicture}
%        %% NOTE: TODO: draw tikz
%    \end{tikzpicture}
%    \end{center}
%    \begin{multicols}{2}
%    \begin{choices}
%        \wrongchoice{
%            \begin{tikzpicture}[scale=0.5]
%                \foreach \i in {-1,1} {
%                    \draw[fill=black] (0,0) rectangle (2cm,1cm);
%                    \node[anchor=center,draw=white] at (1cm,0.5cm) {\bfseries N};
%                    \draw[fill=white] (0,0) rectangle (2cm,-1cm);
%                    \node[anchor=center,draw=black] at (1cm,-0.5cm) {\bfseries S};
%                }
%            \end{tikzpicture}
%        }
%        \wrongchoice{$A$ and $B$}
%      \correctchoice{$A$ and $C$}
%        \wrongchoice{$B$ and $D$}
%        \wrongchoice{$B$ and $C$}
%    \end{choices}
%    \end{multicols}
%\end{question}
%}

%\element{cpo-mc}{
%\begin{question}{cpo-ch16-q05}
%    Which pair of objects experiences an attractive force?
%    \begin{center}
%    \begin{tikzpicture}
%        %% NOTE: TODO: draw tikz
%    \end{tikzpicture}
%    \end{center}
%    \begin{multicols}{2}
%    \begin{choices}
%        \wrongchoice{$A$ and $B$}
%        \wrongchoice{$A$ and $C$}
%      \correctchoice{$B$ and $D$}
%        \wrongchoice{$B$ and $C$}
%    \end{choices}
%    \end{multicols}
%\end{question}
%}

%\element{cpo-mc}{
%\begin{question}{cpo-ch16-q06}
%    A magnet is shown surrounded by its magnetic field.
%    \begin{center}
%    \begin{tikzpicture}
%        %% NOTE: TODO: draw tikz
%    \end{tikzpicture}
%    \end{center}
%    In which area is the magnetic field strongest?
%    \begin{multicols}{4}
%    \begin{choices}[o]
%        \wrongchoice{$A$}
%        \wrongchoice{$B$}
%        \wrongchoice{$C$}
%      \correctchoice{$D$}
%    \end{choices}
%    \end{multicols}
%\end{question}
%}

%\element{cpo-mc}{
%\begin{question}{cpo-ch16-q07}
%    A magnet is shown surrounded by its magnetic field.
%    \begin{center}
%    \begin{tikzpicture}
%        %% NOTE: TODO: draw tikz
%    \end{tikzpicture}
%    \end{center}
%    Which direction should the magnetic field lines
%        be pointing?
%    \begin{multicols}{2}
%    \begin{choices}
%        \wrongchoice{From South to North.}
%      \correctchoice{From North to South.}
%        \wrongchoice{From East to West.}
%        \wrongchoice{From West to East.}
%    \end{choices}
%    \end{multicols}
%\end{question}
%}

\element{cpo-mc}{
\begin{question}{cpo-ch16-q08}
    The magnetic field strength inside a current-carrying coil will be larger if the coil is wound around a:
    \begin{multicols}{2}
    \begin{choices}
        \wrongchoice{vacuum}
        \wrongchoice{wooden rod}
        \wrongchoice{glass rod}
      \correctchoice{iron rod}
    \end{choices}
    \end{multicols}
\end{question}
}

\element{cpo-mc}{
\begin{question}{cpo-ch16-q09}
    The source of a material's magnetism is the:
    \begin{choices}
        \wrongchoice{charge of its protons.}
        \wrongchoice{mass of its neutrons.}
      \correctchoice{spin of its electrons.}
        \wrongchoice{density of its nucleus.}
    \end{choices}
\end{question}
}

\element{cpo-mc}{
\begin{question}{cpo-ch16-q10}
    Materials in which the magnetic fields of individual electrons in an atom cancel out so that each atom has zero net magnetic field are known as:
    \begin{multicols}{2}
    \begin{choices}
        \wrongchoice{ferromagnetic}
        \wrongchoice{paramagnetic}
      \correctchoice{diamagnetic}
        \wrongchoice{monomagnetic}
    \end{choices}
    \end{multicols}
\end{question}
}

\element{cpo-mc}{
\begin{question}{cpo-ch16-q11}
    Materials in which each atom has a tiny magnetic field,
        but the north and south poles of atoms within the material are randomly arranged so that the magnetic fields cancel out, are known as:
    \begin{multicols}{2}
    \begin{choices}
        \wrongchoice{ferromagnetic}
      \correctchoice{paramagnetic}
        \wrongchoice{diamagnetic}
        \wrongchoice{monomagnetic}
    \end{choices}
    \end{multicols}
\end{question}
}

\element{cpo-mc}{
\begin{question}{cpo-ch16-q12}
    If you reverse the direction of current flow in an electromagnet:
    \begin{choices}
      \correctchoice{the north and south poles are reversed}
        \wrongchoice{the magnet is neutralized}
        \wrongchoice{the strength of the magnetic field increases}
        \wrongchoice{a short circuit occurs}
    \end{choices}
\end{question}
}

%\element{cpo-mc}{
%\begin{questionmult}{ch16-Q13}
%    The diagram below represents an iron nail wrapped with
%        a current carrying wire.
%    \begin{center}
%    \begin{tikzpicture}
%        %% NOTE: TODO: draw tikz
%    \end{tikzpicture}
%    \end{center}
%    What type of device does it represent?
%    \begin{multicols}{2}
%    \begin{choices}
%        \wrongchoice{Permanent magnet}
%      \correctchoice{Electromagnet}
%        \wrongchoice{Compass}
%        \wrongchoice{Potentiometer}
%    \end{choices}
%    \end{multicols}
%\end{questionmult}
%}

%\element{cpo-mc}{
%\begin{questionmult}{ch16-Q14}
%    The diagram below represents an iron nail wrapped with
%        a current carrying wire.
%    \begin{center}
%    \begin{tikzpicture}
%        %% NOTE: TODO: draw tikz
%    \end{tikzpicture}
%    \end{center}
%    Using the right hand rule, the magnetic poles of the
%        devices are located:
%    \begin{choices}
%      \correctchoice{North at the ``head'' and South at the ``point''.}
%        \wrongchoice{North at the ``point'' and South at the ``head''.}
%        \wrongchoice{East at the ``point'' and West at the ``head''.}
%        \wrongchoice{East at the ``head'' and West at the ``point''.}
%    \end{choices}
%\end{questionmult}
%}

\element{cpo-mc}{
\begin{question}{cpo-ch16-q15}
    Dani and Gina are trying to make a permanent magnet out of an iron bar.
    Any of the following methods would permanently magnetize the iron bar \emph{except}:
    \begin{choices}
        \wrongchoice{stroking the iron bar with a powerful magnet.}
        \wrongchoice{placing the iron bar in a very strong magnetic field.}
        \wrongchoice{placing the iron bar near a very strong electromagnet.}
      \correctchoice{placing the iron bar near a diamagnetic material.}
    \end{choices}
\end{question}
}

\element{cpo-mc}{
\begin{question}{cpo-ch16-q16}
    An example of a ferromagnetic material is a:
    \begin{choices}
        \wrongchoice{ceramic mug}
      \correctchoice{nail attracted to a bar magnet}
        \wrongchoice{penny}
        \wrongchoice{Compact Disc (CD)}
    \end{choices}
\end{question}
}

%% NOTE
\element{cpo-mc}{
\begin{question}{cpo-ch16-q17}
    Which of the following creates a magnetic field?
    \begin{choices}
        \wrongchoice{A metal ball with \SI{2}{\coulomb} of static charge on it}
        \wrongchoice{A piece of aluminum}
      \correctchoice{A coil of wire carrying current}
        \wrongchoice{A diamagnetic material}
    \end{choices}
\end{question}
}

\element{cpo-mc}{
\begin{questionmult}{ch16-Q18}
    A permanent magnet can be demagnetized by:
    \begin{choices}
      \correctchoice{dropping it on a hard surface}
      \correctchoice{heating it to very high temperature}
      \correctchoice{forcing two north poles together}
    \end{choices}
\end{questionmult}
}

\element{cpo-mc}{
\begin{question}{cpo-ch16-q19}
    Three ways you can increase the strength of an electromagnet are \rule[-0.1pt]{4em}{0.1pt}, \rule[-0.1pt]{4em}{0.1pt} and add iron to the core.
    \begin{choices}
        \wrongchoice{decrease the number of coils; increase the current}
      \correctchoice{increase the number of coils; increase the current}
        \wrongchoice{increase the number of coils; decrease the current}
        \wrongchoice{decrease the number of coils; decrease the current}
    \end{choices}
\end{question}
}

\element{cpo-mc}{
\begin{question}{cpo-ch16-q20}
    Electromagnets are generally more useful than permanent magnets for all of the following reasons \emph{except}:
    \begin{choices}
      \correctchoice{they are always magnetized}
        \wrongchoice{their polarity can be changed}
        \wrongchoice{their strength can be altered}
        \wrongchoice{the magnetism can be turned off}
    \end{choices}
\end{question}
}

\element{cpo-mc}{
\begin{question}{cpo-ch16-q21}
    All atoms act like tiny magnets.
    Why do only a few materials show magnetic properties?
    \begin{choices}
        \wrongchoice{Magnetic materials have atoms that are much stronger magnets than the atoms of other materials.}
        \wrongchoice{Atomic magnets are magnified when combined with a rare substance.  Magnetic materials contain this rare substance.}
        \wrongchoice{We see magnetic properties only if atomic magnets line up with Earth's geographic south and north poles. In magnetic materials this arrangement can occur.}
      \correctchoice{We see magnetic properties only if atomic magnets line up in the same direction throughout a material. In magnetic materials, this arrangement can occur.}
    \end{choices}
\end{question}
}

\element{cpo-mc}{
\begin{question}{cpo-ch16-q22}
    If a pin is brought close to a magnet,
        it is attracted to the magnet and attracts other pins to it.
    If the pin is removed from the magnet, it does not attract other pins.
    The pin might be referred to as:
    \begin{choices}
        \wrongchoice{a hard magnet}
      \correctchoice{a soft magnet}
        \wrongchoice{magnetic mono-pole}
        \wrongchoice{diamagnetic material}
    \end{choices}
\end{question}
}

\element{cpo-mc}{
\begin{question}{cpo-ch16-q23}
    Earth's magnetic north pole is:
    \begin{choices}
        \wrongchoice{aligned with the north star.}
        \wrongchoice{aligned with Earth's geographic north pole.}
      \correctchoice{under Antarctica, near Earth's geographic south pole.}
        \wrongchoice{at the equator.}
    \end{choices}
\end{question}
}

\element{cpo-mc}{
\begin{question}{cpo-ch16-q24}
    The earliest records of magnetically aided navigation are found in the history of the:
    \begin{multicols}{2}
    \begin{choices}
        \wrongchoice{Japanese}
        \wrongchoice{Italians}
      \correctchoice{Greeks}
        \wrongchoice{Chinese}
    \end{choices}
    \end{multicols}
\end{question}
}

\element{cpo-mc}{
\begin{question}{cpo-ch16-q25}
    The difference between true geographic north and the ``north'' indicated by a compass is the difference measured in degrees and known as:
    \begin{multicols}{2}
    \begin{choices}
        \wrongchoice{inclination}
      \correctchoice{declination}
        \wrongchoice{azimuth}
        \wrongchoice{elevation}
    \end{choices}
    \end{multicols}
\end{question}
}

\element{cpo-mc}{
\begin{question}{cpo-ch16-q26}
    According to historical data and current scientific theory,
        the statement that is \emph{not} true concerning Earth's magnetic field is that the field:
    \begin{choices}
        \wrongchoice{reverses every \SI{5e5}{year}}
        \wrongchoice{is weakening by \SI{7}{\percent} every \SI{100}{year}}
        \wrongchoice{will reverse within the next \SI{2 000}{year}}
      \correctchoice{will complete disappear in the future}
    \end{choices}
\end{question}
}

\element{cpo-mc}{
\begin{question}{cpo-ch16-q27}
    %% NOTE: changed wording
    A unit used to measure the strength of a magnetic field is:
    \begin{multicols}{2}
    \begin{choices}
        \wrongchoice{ohm (\si{\ohm})}
      %\correctchoice{tesla (\si{\tesla})}
      \correctchoice{gauss (\si{G})}
        \wrongchoice{ampere (\si{\ampere})}
        \wrongchoice{coulomb (\si{\coulomb})}
    \end{choices}
    \end{multicols}
\end{question}
}

\element{cpo-mc}{
\begin{question}{cpo-ch16-q28}
    Earth behaves like a giant:
    \begin{multicols}{2}
    \begin{choices}
        \wrongchoice{electric circuit}
        \wrongchoice{permanent magnet}
        \wrongchoice{compass}
      \correctchoice{electromagnet}
    \end{choices}
    \end{multicols}
\end{question}
}

%\element{cpo-mc}{
%\begin{question}{cpo-ch16-q29}
%    Which compass in the diagram below is pointing in the wrong direction?
%    \begin{center}
%        %% NOTE: add diagram
%    \end{center}
%    \begin{multicols}{2}
%    \begin{choices}
%        %% NOTE: change ABCD to IJKL
%        \wrongchoice{$A$}
%        \wrongchoice{$B$}
%        \wrongchoice{$C$}
%      \correctchoice{$D$}
%    \end{choices}
%    \end{multicols}
%\end{question}
%}

\element{cpo-mc}{
\begin{question}{cpo-ch16-q30}
    Magnetic declination can best be described as the:
    \begin{choices}
        \wrongchoice{difference between the magnetic field and Earth's surface.}
        \wrongchoice{magnetic field strength of Earth at the equator.}
        \wrongchoice{tendency for the magnetic field of Earth to reverse itself.}
      \correctchoice{difference between directions to true north and magnetic north.}
    \end{choices}
\end{question}
}

\endinput

