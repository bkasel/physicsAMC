
%%--------------------------------------------------
%% CPO: AMC Open Free Response Questions
%%--------------------------------------------------


%% Chapter 11: Changes in Matter
%%--------------------------------------------------


%% CPO Short Answer Questions
%%--------------------------------------------------
\element{cpo-short}{
\begin{question}{ch11-short-q01}
    For the following antacid reaction, list the reactants and the products:
    \begin{equation*}
        \mathbf{2 HCl + CaCO_3 \rightarrow CaCl_2 + CO_2 + H_2O}
    \end{equation*}
    \AMCOpen{lines=3}{
        \wrongchoice[W]{w}\scoring{0}
        \wrongchoice[P]{p}\scoring{1}
        \correctchoice[C]{c}\scoring{2}
    }
    %% ANS: reactants: HCl and CaCO_3;
    %% ANS: products: CaCl_2 and H_2O
\end{question}
}

\element{cpo-short}{
\begin{question}{ch11-short-q02}
    A balanced combustion reaction is provided below.
    \begin{equation*}
        \mathbf{C_6H_{12})_6 + 6O_2 \rightarrow 6CO_2 + 6H_2O} + energy
    \end{equation*}
    Based on the information given in the chemical equation,
        what temperature change would you expect
        in the reaction?
    Why would you expect this?
    \AMCOpen{lines=3}{
        \wrongchoice[W]{w}\scoring{0}
        \wrongchoice[P]{p}\scoring{1}
        \correctchoice[C]{c}\scoring{2}
    }
\end{question}
}

\element{cpo-short}{
\begin{question}{ch11-short-q03}
    A balanced combustion reaction is provided below.
    \begin{equation*}
        \mathbf{C_6H_{12})_6 + 6O_2 \rightarrow 6CO_2 + 6H_2O} + energy
    \end{equation*}
    Is this an endothermic or exothermic reaction?
    \AMCOpen{lines=3}{
        \wrongchoice[W]{w}\scoring{0}
        \wrongchoice[P]{p}\scoring{1}
        \correctchoice[C]{c}\scoring{2}
    }
    %% ANS: exothermic
\end{question}
}

\element{cpo-short}{
\begin{question}{ch11-short-q04}
    Write the balanced equation for the following reaction:
    The fuel used in gas grills is called propane (C\textsubscript{3}H\textsubscript{8}).
    Propane reacts with oxygen gas (O\textsubscript{2}) found in the air
        to produce carbon dioxide gas (CO\textsubscript{2}) and water vapor
        (H\textsubscript{2}).
    \AMCOpen{lines=3}{
        \wrongchoice[W]{w}\scoring{0}
        \wrongchoice[P]{p}\scoring{1}
        \correctchoice[C]{c}\scoring{2}
    }
    %% ANS: C_3H_8 + 5O_2 \rightarrow 3CO_2 + 4H_2O
\end{question}
}

\element{cpo-short}{
\begin{question}{ch11-short-q05}
    What radioactive technique is used to determine the age of fossils?
    \AMCOpen{lines=3}{
        \wrongchoice[W]{w}\scoring{0}
        \wrongchoice[P]{p}\scoring{1}
        \correctchoice[C]{c}\scoring{2}
    }
\end{question}
}

\element{cpo-short}{
\begin{question}{ch11-short-q06}
    Give one example of harmful radiation and 1 example of helpful radiation.
    \AMCOpen{lines=3}{
        \wrongchoice[W]{w}\scoring{0}
        \wrongchoice[P]{p}\scoring{1}
        \correctchoice[C]{c}\scoring{2}
    }
\end{question}
}

\element{cpo-short}{
\begin{question}{ch11-short-q07}
    List at least three everyday uses for nuclear reactions.
    \AMCOpen{lines=3}{
        \wrongchoice[W]{w}\scoring{0}
        \wrongchoice[P]{p}\scoring{1}
        \correctchoice[C]{c}\scoring{2}
    }
\end{question}
}

\element{cpo-short}{
\begin{question}{ch11-short-q08}
    The decay of sodium-25 over time is show in the graph below.
    \begin{center}
        %% NOTE: add graph
    \end{center}
    What is the half-life of sodium-25?
    \AMCOpen{lines=3}{
        \wrongchoice[W]{w}\scoring{0}
        \wrongchoice[P]{p}\scoring{1}
        \correctchoice[C]{c}\scoring{2}
    }
    %% ANS: \SI{1}{\minute}
\end{question}
}

\element{cpo-short}{
\begin{question}{ch11-short-q09}
    The decay of sodium-25 over time is show in the graph below.
    \begin{center}
        %% NOTE: add graph
    \end{center}
    At time \num{0}, what percent of the radioactive nuclei are present?
    \AMCOpen{lines=3}{
        \wrongchoice[W]{w}\scoring{0}
        \wrongchoice[P]{p}\scoring{1}
        \correctchoice[C]{c}\scoring{2}
    }
    %% ANS: \SI{1}{\minute}
\end{question}
}

\element{cpo-short}{
\begin{question}{ch11-short-q10}
    The decay of sodium-25 over time is show in the graph below.
    \begin{center}
        %% NOTE: add graph
    \end{center}
    After \SI{2}{\minute}, what percent of the original nuclei are still radioactive?
    \AMCOpen{lines=3}{
        \wrongchoice[W]{w}\scoring{0}
        \wrongchoice[P]{p}\scoring{1}
        \correctchoice[C]{c}\scoring{2}
    }
    %% ANS: \SI{25}{\percent}
\end{question}
}

\element{cpo-short}{
\begin{question}{ch11-short-q11}
    Explain what is meant by the statement
        ``radioactive decay depends on chance.''
    \AMCOpen{lines=3}{
        \wrongchoice[W]{w}\scoring{0}
        \wrongchoice[P]{p}\scoring{1}
        \correctchoice[C]{c}\scoring{2}
    }
    %% ANS: \SI{25}{\percent}
\end{question}
}


%% CPO Problem Questions
%%--------------------------------------------------
\element{cpo-problem}{
\begin{question}{ch11-problem-q01}
    Carbon-14 (C\textsupersript{14}) decays into nitrogen-14 (N\textsuperscript{14})
        with a half life of \SI{5 700}{year}.
    How many years would it take a sample of C\textsuperscript{14}
        to decay so much that only \num{1/8} of the atoms were still
        C\textsuperscript{14}
    The figure shows a 4-sided dice, with the sides labels 1 through 4.
    If you were to roll this dice once, what is the probability that
        you would roll a 3?
    If you rolled the dice 100 times, how many times would you expect
        to roll a 3?
    \AMCOpen{lines=3}{
        \wrongchoice[W]{w}\scoring{0}
        \wrongchoice[P]{p}\scoring{1}
        \correctchoice[C]{c}\scoring{2}
    }
    %% ANS: \num{25}
\end{question}
}


%% CPO Essay Questions
%%--------------------------------------------------
\element{cpo-essay}{
\begin{question}{ch11-essay-q01}
    Describe how electrons are involved in the formation of
        chemical bonds.
    Include in your description the difference between
        ionic and covalent bonds.
    \AMCOpen{lines=3}{
        \wrongchoice[W]{w}\scoring{0}
        \wrongchoice[P]{p}\scoring{1}
        \correctchoice[C]{c}\scoring{2}
    }
\end{question}
}

\element{cpo-essay}{
\begin{question}{ch11-essay-q02}
    When you mix baking soda and vinegar, the mixture bubbles
        violently and the temperature drops.
    Is this a chemical change or a physical change?
    Provide evidence to support your answer.
    \AMCOpen{lines=3}{
        \wrongchoice[W]{w}\scoring{0}
        \wrongchoice[P]{p}\scoring{1}
        \correctchoice[C]{c}\scoring{2}
    }
\end{question}
}

\element{cpo-essay}{
\begin{question}{ch11-essay-q03}
    Describe how photosynthesis and respiration are almost reverse reactions.
    \AMCOpen{lines=3}{
        \wrongchoice[W]{w}\scoring{0}
        \wrongchoice[P]{p}\scoring{1}
        \correctchoice[C]{c}\scoring{2}
    }
\end{question}
}

\element{cpo-essay}{
\begin{question}{ch11-essay-q04}
    The mass of an iron bolt was \SI{5.4}{\gram} when it was manufactured.
    After being bolted to an outdoor structure for several months,
        the mass of the bolt was found to have increased by \SI{0.2}{\gram}.
    Given the following balanced equation for the reaction,
        does this example support the law of conservation of mass?
    Why or why not?
    \begin{equation*}
        4Fe + 3O_2 \rightarrow 2Fe_2O_3
    \end{equation*}
    \AMCOpen{lines=3}{
        \wrongchoice[W]{w}\scoring{0}
        \wrongchoice[P]{p}\scoring{1}
        \correctchoice[C]{c}\scoring{2}
    }
\end{question}
}

\endinput


