
%%--------------------------------------------------
%% CPO: Multiple Choice Questions
%%--------------------------------------------------


%% Chapter 24: The Physical Nature of Light
%%--------------------------------------------------


%% Learning Objectives
%%--------------------------------------------------

%% Identify the relationship among the frequency, energy, and wavelength of light. 
%% Explain how the speed of light changes in different materials. 
%% Identify the different kinds of electromagnetic waves.
%% Recognize the interference pattern and how it is created with a diffraction grating. 
%% Understand the difference between polarized light and unpolarized light. 
%% Learn about applications of polarizations. 
%% Define a photon. 
%% Explain how energy is related to the color of light. 
%% Know that a single atom absorbs a single photon at a time.


%% CPO Multiple Choice Questions
%%--------------------------------------------------
\element{cpo-mc}{
\begin{question}{cpo-mc-ch24-q01}
    The electromagnetic waves with the shortest wavelength are:
    \begin{multicols}{2}
    \begin{choices}
        \wrongchoice{microwaves}
      \correctchoice{gamma rays}
        \wrongchoice{radio waves}
        \wrongchoice{visible light waves}
    \end{choices}
    \end{multicols}
\end{question}
}

\element{cpo-mc}{
\begin{question}{cpo-mc-ch24-q02}
    The electromagnetic waves that are beneficial in small amounts but responsible
        for skin cancer, sunburn, and cataracts in larger amounts are:
    \begin{multicols}{2}
    \begin{choices}
        \wrongchoice{visible light}
        \wrongchoice{AM radio}
      \correctchoice{ultraviolet}
        \wrongchoice{infrared}
    \end{choices}
    \end{multicols}
\end{question}
}

\element{cpo-mc}{
\begin{question}{cpo-mc-ch24-q03}
    The energy of light is directly related to its: 
    \begin{multicols}{2}
    \begin{choices}
        \wrongchoice{amplitude}
        \wrongchoice{brightness}
        \wrongchoice{speed}
      \correctchoice{frequency}
    \end{choices}
    \end{multicols}
\end{question}
}

\element{cpo-mc}{
\begin{question}{cpo-mc-ch24-q04}
        The speed of light is special because it:
    \begin{choices}
        \wrongchoice{can be slowed down in a vacuum.}
      \correctchoice{is the greatest speed attainable in nature.}
        \wrongchoice{is similar to the speed of sound.}
        \wrongchoice{is easily measured using stopwatches.}
    \end{choices}
\end{question}
}

\element{cpo-mc}{
\begin{question}{cpo-mc-ch24-q05}
    Which statement about the frequency and wavelength of visible light is \emph{correct}?
    \begin{choices}
        \wrongchoice{The frequencies and wavelengths of light are measured using the same units.}
        \wrongchoice{The frequencies and wavelengths of light are similar to the frequencies and wavelength.}
      \correctchoice{The frequencies of light are incredibly high and the wavelengths of light are tiny.}
        \wrongchoice{The frequencies of light are low and the wavelengths of light are large.}
    \end{choices}
\end{question}
}

\element{cpo-mc}{
\begin{question}{cpo-mc-ch24-q06}
    To calculate the speed of light in a material:
    \begin{choices}
      \correctchoice{divide the speed of light in a vacuum by the index of refraction for the material.}
        \wrongchoice{multiply the speed of light in a vacuum by the index of refraction for the material.}
        \wrongchoice{divide the index of refraction for the material by the speed of light in a vacuum.}
        \wrongchoice{add the speed of light in a vacuum to the index of refraction for the material.}
    \end{choices}
\end{question}
}

\element{cpo-mc}{
\begin{question}{cpo-mc-ch24-q07}
    Compared to the speed of light in a vacuum,
        the speed of light traveling in a material like glass:
    \begin{choices}
      \correctchoice{is slower}
        \wrongchoice{is unchanged}
        \wrongchoice{is faster}
        \wrongchoice{may be faster or slower depending up on the material}
    \end{choices}
\end{question}
}

\element{cpo-mc}{
\begin{question}{cpo-mc-ch24-q08}
    All electromagnetic waves have the same speed in:
    \begin{multicols}{2}
    \begin{choices}
        \wrongchoice{water}
        \wrongchoice{glass}
      \correctchoice{a vacuum}
        \wrongchoice{plastic}
    \end{choices}
    \end{multicols}
\end{question}
}

\element{cpo-mc}{
\begin{question}{cpo-mc-ch24-q09}
    A gamma ray with frequency of \SI{6.0e20}{\hertz} has a wavelength of:
    \begin{multicols}{2}
    \begin{choices}
        \wrongchoice{\SI{1.7e-21}{\meter}}
      \correctchoice{\SI{5.0e-13}{\meter}}
        \wrongchoice{\SI{2.0e12}{\meter}}
        \wrongchoice{\SI{6.0e20}{\meter}}
    \end{choices}
    \end{multicols}
\end{question}
}

\element{cpo-mc}{
\begin{question}{cpo-mc-ch24-q10}
    All of the following waves travel at \SI{3e8}{\meter\per\second} \emph{except}:
    \begin{multicols}{2}
    \begin{choices}
        \wrongchoice{light rays}
      \correctchoice{sound waves}
        \wrongchoice{microwaves}
        \wrongchoice{x-rays}
    \end{choices}
    \end{multicols}
\end{question}
}

\element{cpo-mc}{
\begin{question}{cpo-mc-ch24-q11}
    Which of the following can \emph{never} happen when light travels from one material into another?
    \begin{choices}
        \wrongchoice{The wavelength decreases}
      \correctchoice{The frequency decreases}
        \wrongchoice{The speed increases}
        \wrongchoice{The speed decreases}
    \end{choices}
\end{question}
}

\element{cpo-mc}{
\begin{question}{cpo-mc-ch24-q12}
    The speed of light in a vacuum is \SI{3.0e8}{\meter\per\second}.
    The speed of light in a diamond whose index of refraction is \num{2.42} is:
    \begin{multicols}{2}
    \begin{choices}
        \wrongchoice{\SI{7.26}{\meter\per\second}}
        \wrongchoice{\SI{7.26e8}{\meter\per\second}}
        \wrongchoice{\SI{1.24}{\meter\per\second}}
      \correctchoice{\SI{1.24e8}{\meter\per\second}}
    \end{choices}
    \end{multicols}
\end{question}
}

\element{cpo-mc}{
\begin{question}{cpo-mc-ch24-q13}
    Light waves display all of the following characteristics \emph{except}:
    \begin{multicols}{2}
    \begin{choices}
        \wrongchoice{resonance}
        \wrongchoice{diffraction}
      \correctchoice{mass}
        \wrongchoice{wavelength}
    \end{choices}
    \end{multicols}
\end{question}
}

\element{cpo-mc}{
\begin{question}{cpo-mc-ch24-q14}
    The addition of waves that creates a pattern of alternating dark and light bands is called:
    \begin{multicols}{2}
    \begin{choices}
        \wrongchoice{diffusion}
        \wrongchoice{refraction}
      \correctchoice{interference}
        \wrongchoice{reflection}
    \end{choices}
    \end{multicols}
\end{question}
}

%% NOTE: NYSED June1996-Q50
%\element{cpo-mc}{
%\begin{question}{cpo-mc-ch24-q15}
%    The diagram below represents sunglasses being used to eliminate glare.
%    \begin{center}
%        %% NOTE: add diagram
%    \end{center}
%    The phenomenon represented in the diagram is:
%    \begin{multicols}{2}
%    \begin{choices}
%        \wrongchoice{dispersion.}
%      \correctchoice{polarization.}
%        \wrongchoice{refraction.}
%        \wrongchoice{reflection.}
%    \end{choices}
%    \end{multicols}
%\end{question}
%}

\element{cpo-mc}{
\begin{question}{cpo-mc-ch24-q16}
    A device that measures the wavelength of light is called a:
    \begin{multicols}{2}
    \begin{choices}
        \wrongchoice{polarizer}
      \correctchoice{spectrometer}
        \wrongchoice{liquid crystal diode}
        \wrongchoice{magnetron}
    \end{choices}
    \end{multicols}
\end{question}
}

%\element{cpo-mc}{
%\begin{question}{cpo-mc-ch24-q17}
%    The diagram below illustrates two sources of light energy
%        produced at a constant frequency such that the light
%        falling on the screen at points $A$ and $C$ is bright.
%    \begin{center}
%    \begin{tikzpicture}
%        %% NOTE: TODO: draw tikz
%    \end{tikzpicture}
%    \end{center}
%    There is no light at point $B$.
%    The wave phenomenon represented by this diagram is:
%    \begin{multicols}{2}
%    \begin{choices}
%      \correctchoice{interference.}
%        \wrongchoice{polarization.}
%        \wrongchoice{reflection.}
%        \wrongchoice{refraction.}
%    \end{choices}
%    \end{multicols}
%\end{question}
%}

\element{cpo-mc}{
\begin{question}{cpo-mc-ch24-q18}
    A device whose operation depends upon the ability of light waves to be polarized is the:
    \begin{multicols}{2}
    \begin{choices}
      \correctchoice{LCD}
        \wrongchoice{magnetron}
        \wrongchoice{magnet}
        \wrongchoice{spectrometer}
    \end{choices}
    \end{multicols}
\end{question}
}

\element{cpo-mc}{
\begin{question}{cpo-mc-ch24-q19}
    A material that selectively absorbs light depending on the orientation of its electromagnetic waves acts as a:
    \begin{multicols}{2}
    \begin{choices}
        \wrongchoice{magnetron}
      \correctchoice{polarizer}
        \wrongchoice{diffraction grating}
        \wrongchoice{spectrometer}
    \end{choices}
    \end{multicols}
\end{question}
}

%\element{cpo-mc}{
%\begin{question}{cpo-mc-ch24-q20}
%    Two wave sources operating in phase in the same material
%        produce circular wave patterns shown in the diagram below.
%    The solid lines represent wave crests and the dashed lines represent
%        wave troughs.
%    %% NOTE: TODO: draw tikz
%    \begin{center}
%        \begin{tikzpicture}
%            \draw (0,0) arc (0:180:1);
%            \draw (0,0) arc (0:180:2);
%            \draw (0,0) arc (0:180:3);
%            \draw (0,0) arc (0:180:4);
%
%            \draw (1,0) arc (0:180:1);
%            \draw (2,0.5) arc (0:180:1);
%            %% Source A
%            %\draw[dashed] (-1,0) circle [radius=0.5];
%            %\draw (-1,0) circle [radius=1.0];
%            %\draw[dashed] (-1,0) circle [radius=1.5];
%            %\draw (-1,0) circle [radius=2.0];
%
%            %% Source B
%            %\draw[dashed] (1,0) circle [radius=0.5];
%            %\draw (1,0) circle [radius=1.0];
%            %\draw[dashed] (1,0) circle [radius=1.5];
%            %\draw (1,0) circle [radius=2.0];
%        \end{tikzpicture}
%    \end{center}
%    The greatest destructive interference occurs at point:
%    \begin{multicols}{4}
%    \begin{choices}[o]
%        \wrongchoice{$A$}
%      \correctchoice{$B$}
%        \wrongchoice{$C$}
%        \wrongchoice{$D$}
%    \end{choices}
%    \end{multicols}
%\end{question}
%}

%\element{cpo-mc}{
%\begin{question}{cpo-mc-ch24-q21}
%    The diagram below represents waves of wavelength $\lambda$ passing
%        through two small openings, $A$ and $B$, in a barrier.
%    The solid lines represent wave crests and the dashed lines represent
%        wave troughs.
%    \begin{center}
%    %% NOTE: TODO: draw tikz
%    \begin{tikzpicture}
%        %% diffraction
%        \draw (0,0) arc (0:180:1);
%        \draw (0,0) arc (0:180:2);
%        \draw (0,0) arc (0:180:3);
%        \draw (0,0) arc (0:180:4);
%        \draw (1,0) arc (0:180:1);
%        \draw (2,0.5) arc (0:180:1);
%        %% source
%        %\draw[line thick=10pt]  (-2,0) -- (2,0);
%        \draw[black]  (-2,-2) -- (2,-2);
%        \draw[black]  (-2,-2) -- (2,-2);
%        \draw[dashed] (-2,-2) -- (2,-2);
%        \draw[black]  (-2,-2) -- (2,-2);
%        \draw (-1,0) circle [radius=1.0];
%        \draw[dashed] (-1,0) circle [radius=1.5];
%        \draw (-1,0) circle [radius=2.0];
%        %% Source B
%        %\draw[dashed] (1,0) circle [radius=0.5];
%        %\draw (1,0) circle [radius=1.0];
%        %\draw[dashed] (1,0) circle [radius=1.5];
%        %\draw (1,0) circle [radius=2.0];
%    \end{tikzpicture}
%    \end{center}
%    Compared to the distance from $B$ to $P$, the distance from
%        $A$ to $P$ is:
%    \begin{multicols}{2}
%    \begin{choices}
%        \wrongchoice{$1\lambda$ longer.}
%        \wrongchoice{$2\lambda$ longer.}
%      \correctchoice{$\frac{1}{2}\lambda$ longer.}
%        \wrongchoice{the same.}
%    \end{choices}
%    \end{multicols}
%\end{question}
%}

\element{cpo-mc}{
\begin{question}{cpo-mc-ch24-q22}
    Looking through a spectrometer, you see a line of light
        at \SI{410}{\nano\meter}.
    This means the wavelength of the light is \SI{410}{\nano\meter}.
    The color of this light is:
    \begin{multicols}{2}
    \begin{choices}
        \wrongchoice{red}
        \wrongchoice{orange}
        \wrongchoice{green}
      \correctchoice{violet}
    \end{choices}
    \end{multicols}
\end{question}
}

\element{cpo-mc}{
\begin{question}{cpo-mc-ch24-q23}
    Phosphorous atoms embedded in plastic that absorb light energy
        wills lowly release the energy in a glow-in-the-dark process known as:
    \begin{multicols}{2}
    \begin{choices}
        \wrongchoice{photosynthesis}
        \wrongchoice{phototropism}
      \correctchoice{photoluminescence}
        \wrongchoice{polarization}
    \end{choices}
    \end{multicols}
\end{question}
}

\element{cpo-mc}{
\begin{question}{cpo-mc-ch24-q24}
    Light is composed of tiny particles of energy known as:
    \begin{multicols}{2}
    \begin{choices}
        \wrongchoice{protons}
        \wrongchoice{neutrons}
        \wrongchoice{electrons}
      \correctchoice{photons}
    \end{choices}
    \end{multicols}
\end{question}
}

\element{cpo-mc}{
\begin{question}{cpo-mc-ch24-q25}
    As atoms gain energy due to a temperature increase,
        the color of the light emitted by the atoms may change from:
    \begin{multicols}{2}
    \begin{choices}
        \wrongchoice{blue to green}
      \correctchoice{red to yellow}
        \wrongchoice{green to orange}
        \wrongchoice{violet to red}
    \end{choices}
    \end{multicols}
\end{question}
}

\endinput


