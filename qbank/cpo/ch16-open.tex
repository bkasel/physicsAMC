
%%--------------------------------------------------
%% CPO: AMC Open Free Response Questions
%%--------------------------------------------------


%% Chapter 16: Magnetism
%%--------------------------------------------------


%% CPO Short Answer Questions
%%--------------------------------------------------
\element{cpo-short}{
\begin{question}{ch16-short-q01}
    Is it possible to have a magnet with only a north pole? Explain.
    \AMCOpen{lines=3}{
        \wrongchoice[W]{w}\scoring{0}
        \wrongchoice[P]{p}\scoring{1}
        \correctchoice[C]{c}\scoring{2}
    }
\end{question}
}

\element{cpo-short}{
\begin{question}{ch16-short-q02}
    The forces between two magnets depend up on the magnets' alignment.
    Explain how you might cause attractive and repulsive forces between
        two magnets.
    \AMCOpen{lines=3}{
        \wrongchoice[W]{w}\scoring{0}
        \wrongchoice[P]{p}\scoring{1}
        \correctchoice[C]{c}\scoring{2}
    }
\end{question}
}

\element{cpo-short}{
\begin{question}{ch16-short-q03}
    What two pieces of information can you get from a magnetic field diagram?
    \AMCOpen{lines=3}{
        \wrongchoice[W]{w}\scoring{0}
        \wrongchoice[P]{p}\scoring{1}
        \correctchoice[C]{c}\scoring{2}
    }
\end{question}
}

\element{cpo-short}{
\begin{question}{ch16-short-q04}
    A strong magnet and a weak magnet are placed north pole to south pole
        and are attracted to each other with a magnetic force.
    Which magnet has the stronger force?
    \AMCOpen{lines=3}{
        \wrongchoice[W]{w}\scoring{0}
        \wrongchoice[P]{p}\scoring{1}
        \correctchoice[C]{c}\scoring{2}
    }
\end{question}
}

\element{cpo-short}{
\begin{question}{ch16-short-q05}
    Name \num{3} ferromagnetic materials.
    \AMCOpen{lines=3}{
        \wrongchoice[W]{w}\scoring{0}
        \wrongchoice[P]{p}\scoring{1}
        \correctchoice[C]{c}\scoring{2}
    }
    %% ANS: iron, cobalt, nickel, etc\ldots
\end{question}
}

\element{cpo-short}{
\begin{question}{ch16-short-q06}
    Name \num{2} methods that might be used to demagnetize a material.
    \AMCOpen{lines=3}{
        \wrongchoice[W]{w}\scoring{0}
        \wrongchoice[P]{p}\scoring{1}
        \correctchoice[C]{c}\scoring{2}
    }
    %% ANS: an electric field
\end{question}
}

\element{cpo-short}{
\begin{question}{ch16-short-q07}
    What terms describe a material that is very weakly magnetized due
        to its randomly arranged magnetic domains?
    \AMCOpen{lines=3}{
        \wrongchoice[W]{w}\scoring{0}
        \wrongchoice[P]{p}\scoring{1}
        \correctchoice[C]{c}\scoring{2}
    }
    %% ANS: paramagnetic
\end{question}
}

\element{cpo-short}{
\begin{question}{ch16-short-q08}
    Describe how to make an electromagnet.
    \AMCOpen{lines=3}{
        \wrongchoice[W]{w}\scoring{0}
        \wrongchoice[P]{p}\scoring{1}
        \correctchoice[C]{c}\scoring{2}
    }
\end{question}
}

\element{cpo-short}{
\begin{question}{ch16-short-q09}
    Name two ways to increase the strength of an electromagnetic.
    \AMCOpen{lines=3}{
        \wrongchoice[W]{w}\scoring{0}
        \wrongchoice[P]{p}\scoring{1}
        \correctchoice[C]{c}\scoring{2}
    }
\end{question}
}

%\element{cpo-short}{
%\begin{question}{ch16-short-q10}
%    Several compasses are placed around a bar magnet.
%    Which end of the magnet is its north pole?
%    \begin{center}
%        %% NOTE: add diagram
%    \end{center}
%    \AMCOpen{lines=3}{
%        \wrongchoice[W]{w}\scoring{0}
%        \wrongchoice[P]{p}\scoring{1}
%        \correctchoice[C]{c}\scoring{2}
%    }
%\end{question}
%}

%\element{cpo-short}{
%\begin{question}{ch16-short-q11}
%    In this diagram of Earth's magnetic field, where is the magnetid field strongest?
%    \begin{center}
%        %% NOTE: add diagram
%    \end{center}
%    \AMCOpen{lines=3}{
%        \wrongchoice[W]{w}\scoring{0}
%        \wrongchoice[P]{p}\scoring{1}
%        \correctchoice[C]{c}\scoring{2}
%    }
%\end{question}
%}


%% CPO Essay Questions
%%--------------------------------------------------
\element{cpo-essay}{
\begin{question}{ch16-essay-q01}
    Explain why a magnet always attracts ferromagnetic materials
        even when they are not magnets?
    \AMCOpen{lines=3}{
        \wrongchoice[W]{w}\scoring{0}
        \wrongchoice[P]{p}\scoring{1}
        \correctchoice[C]{c}\scoring{2}
    }
\end{question}
}

\element{cpo-essay}{
\begin{question}{ch16-essay-q02}
    How can a permanent magnet be created?
    \AMCOpen{lines=3}{
        \wrongchoice[W]{w}\scoring{0}
        \wrongchoice[P]{p}\scoring{1}
        \correctchoice[C]{c}\scoring{2}
    }
\end{question}
}

\element{cpo-essay}{
\begin{question}{ch16-essay-q03}
    If a permanent magnet is used to pick up one pin,
        that pin will, in turn, attract a second pin that
        will attract a third pin and so on.
    Explain how the magnetism shown by the pins and the
        permanent magnet differ.
    \AMCOpen{lines=3}{
        \wrongchoice[W]{w}\scoring{0}
        \wrongchoice[P]{p}\scoring{1}
        \correctchoice[C]{c}\scoring{2}
    }
    %% ANS: differentiate between hard and soft magnets
\end{question}
}

\element{cpo-essay}{
\begin{question}{ch16-essay-q04}
    Describe the mechanism that provides the magnetic
        field that surrounds Earth.
    \AMCOpen{lines=3}{
        \wrongchoice[W]{w}\scoring{0}
        \wrongchoice[P]{p}\scoring{1}
        \correctchoice[C]{c}\scoring{2}
    }
    %% ANS: differentiate between hard and soft magnets
\end{question}
}

\element{cpo-essay}{
\begin{question}{ch16-essay-q05}
    Name an example of a naturally-occurring magnetic material,
        and describe how it its magnetic properties have been used.
    \AMCOpen{lines=3}{
        \wrongchoice[W]{w}\scoring{0}
        \wrongchoice[P]{p}\scoring{1}
        \correctchoice[C]{c}\scoring{2}
    }
    %% ANS: differentiate between hard and soft magnets
\end{question}
}

\endinput


