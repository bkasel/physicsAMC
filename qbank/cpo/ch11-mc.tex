
%%--------------------------------------------------
%% CPO: Multiple Choice Questions
%%--------------------------------------------------


%% Chapter 11: Changes in Matter
%%--------------------------------------------------


%% Learning Objectives
%%--------------------------------------------------

%% Explain why most atoms combine with other atoms to make compounds. 
%% Predict whether a chemical bond is ionic or covalent. 
%% Describe the differences between ionic and covalent bonds. 
%% Recognize the signs of chemical change. 
%% Explain how energy is used or released by chemical reactions. 
%% Write an equation for a chemical reaction. 
%% Compare and contrast nuclear reactions and chemical reactions. 
%% Explain the difference between fusion and fission reactions.
%% Describe what happens during radioactive decay.


%% CPO Multiple Choice Questions
%%--------------------------------------------------
\element{cpo-mc}{
\begin{question}{cpo-ch11-q01}
    A chemical bond forms when atoms transfer or share:
    \begin{multicols}{2}
    \begin{choices}
        \wrongchoice{neutrons}
        \wrongchoice{protons}
      \correctchoice{electrons}
        \wrongchoice{ions}
    \end{choices}
    \end{multicols}
\end{question}
}

\element{cpo-mc}{
\begin{question}{cpo-ch11-q02}
    Which statement best explains why atoms form chemical bonds with other atoms?
    \begin{choices}
        \wrongchoice{Most atoms have greater energy when they combine with other atoms.}
        \wrongchoice{When atoms collide with other atoms, they bond automatically.}
        \wrongchoice{Atoms are always attracted to other atoms.}
      \correctchoice{Most atoms have lower energy when they are combined with other atoms.}
    \end{choices}
\end{question}
}

\element{cpo-mc}{
\begin{question}{cpo-ch11-q03}
    Which of the following is \emph{true}?
    Covalent bonding occurs:
    \begin{choices}
        \wrongchoice{in ionic compounds like NaCl.}
      \correctchoice{when electrons are shared between two atoms.}
        \wrongchoice{only when electrons are shared between two identical atoms.}
        \wrongchoice{when electrons are transferred from one atom to another.}
    \end{choices}
\end{question}
}

\element{cpo-mc}{
\begin{question}{cpo-ch11-q04}
    When an atom gains or loses electrons,
        it has an electrical charge.
    It is known as a(n):
    \begin{choices}
      \correctchoice{ion}
        \wrongchoice{free radical}
        \wrongchoice{hydrate}
        \wrongchoice{monoatomic molecule}
    \end{choices}
\end{question}
}

%% NOTE: mass is not conserved!
\element{cpo-mc}{
\begin{questionmult}{ch11-Q05}
    When a chemical change occurs:
    \begin{choices}
      \correctchoice{atoms are rearranged.}
        \wrongchoice{the laws of conservation of mass is always obeyed.}
      \correctchoice{the chemical properties of new substances are different from the ones you started with.}
      %\correctchoice{all of the above.}
    \end{choices}
\end{questionmult}
}

\element{cpo-mc}{
\begin{question}{cpo-ch11-q06}
    The chemical reaction for the formation of rust is shown below.
    \begin{equation*}
        \ce{Fe + O2 -> FeO3}
        %\mathbf{Fe} + \mathbf{O}_2 \rightarrow \mathbf{Fe}_2\mathbf{0}_3
    \end{equation*}
    Identify the reactant(s):
    \begin{multicols}{2}
    \begin{choices}
        \wrongchoice{\ce{Fe} and \ce{Fe2O3}.}
      \correctchoice{\ce{Fe} and \ce{O2}.}
        \wrongchoice{\ce{Fe2O3}.}
        \wrongchoice{\ce{O2} and \ce{Fe2O3}.}
    \end{choices}
    \end{multicols}
\end{question}
}

\element{cpo-mc}{
\begin{question}{cpo-ch11-q07}
    The chemical reaction for the formation of rust is shown below.
    \begin{equation*}
        \ce{Fe + O2 -> FeO3}
    \end{equation*}
    Identify the product(s):
    \begin{multicols}{2}
    \begin{choices}
        \wrongchoice{\ce{Fe} and \ce{Fe2O3}.}
        \wrongchoice{\ce{Fe} and \ce{O2}.}
      \correctchoice{\ce{Fe2O3}.}
        \wrongchoice{\ce{O2} and \ce{Fe2O3}.}
    \end{choices}
    \end{multicols}
\end{question}
}

\element{cpo-mc}{
\begin{question}{cpo-ch11-q08}
    The chemical reaction for the formation of rust is shown below.
    \begin{equation*}
        \ce{Fe + O2 -> FeO3}
    \end{equation*}
    Is the chemical reaction for the formation of rust balanced?
    If not, select the correct equation from the ones listed below.
    Identify the product(s):
    \begin{choices}
        \wrongchoice{Yes, it is balanced.}
        \wrongchoice{No, this is balanced: \ce{2Fe + O2 -> Fe2O3}}
      \correctchoice{No, this is balanced: \ce{4Fe + 3O2 -> 2Fe2O3}}
        \wrongchoice{No, this is balanced: \ce{3Fe + 2O2 -> 3FeO3}}
        %% NOTE: make question where answer is Yes
    \end{choices}
\end{question}
}

\element{cpo-mc}{
\begin{question}{cpo-ch11-q09}
    Sodium forms an ionic bond with chlorine when sodium \rule[-0.2pt]{4em}{0.1pt} an electron and chlorine \rule[-0.2pt]{4em}{0.1pt} an electron.
    \begin{multicols}{2}
    \begin{choices}
        \wrongchoice{shares, shares}
      \correctchoice{loses, gains}
        \wrongchoice{gains, loses}
        \wrongchoice{loses, loses}
    \end{choices}
    \end{multicols}
\end{question}
}

\element{cpo-mc}{
\begin{question}{cpo-ch11-q10}
    What is the chemical formula for a compound that contains the aluminum ion (\ce{Al^{3+}}) and the hydroxide ion (OH-)?
    \begin{choices}
      \correctchoice{\ce{AlOH3}}
        \wrongchoice{\ce{AlO3H3}}
        \wrongchoice{\ce{AlOH3}}
        %% NOTE: cannot use ``none of the above''
        \wrongchoice{Cannot be determined.}
    \end{choices}
\end{question}
}

\element{cpo-mc}{
\begin{question}{cpo-ch11-q11}
    When you eat an apple, the process of digestion involves:
    \begin{choices}
        \wrongchoice{only physical change.}
        \wrongchoice{only chemical change.}
      \correctchoice{physical and chemical change.}
        \wrongchoice{phase change.}
    \end{choices}
\end{question}
}

\element{cpo-mc}{
\begin{question}{cpo-ch11-q12}
    In which of the following situations does water undergo a change in physical properties?
    \begin{choices}
      \correctchoice{The bathroom fills with steam when you take a hot shower.}
        \wrongchoice{You pour a glass of water from the faucet.}
        \wrongchoice{Water is broken down to yield H\textsubscript{2} and O\textsubscript{2}.}
        \wrongchoice{When hydrogen is used as a fuel for rockets, water is a product.}
    \end{choices}
\end{question}
}

\element{cpo-mc}{
\begin{question}{cpo-ch11-q13}
    Balance the following equation to demonstrate the conservation of atoms in a reaction.
    Choose the answer that provides the correct coefficients for each reactant and product.
    \begin{equation*}
        \ce{Al2O3 -> Al + O2} 
    \end{equation*}
    \begin{choices}
      \correctchoice{\ce{2Al2O3 -> 4Al + 3O2}.}
        \wrongchoice{\ce{4Al2O3 -> 3Al + 2O2}.}
        \wrongchoice{\ce{3Al2O3 -> 2Al + 4O2}.}
        \wrongchoice{\ce{2Al2O3 -> 3Al + 4O2}.}
    \end{choices}
\end{question}
}

\element{cpo-mc}{
\begin{question}{cpo-ch11-q14}
    If you designed your own experiment to prove the law of conservation of mass,
        what conditions would be required?
    \begin{choices}
        \wrongchoice{You must find the mass of the reactants before the reaction and find the mass of the products after the reaction.}
      \correctchoice{You must find the mass of the reactants before the reaction, find the mass of the products after the reaction, \emph{and} perform the reaction in a closed system.}
        \wrongchoice{You must choose a combustion reaction.}
        \wrongchoice{You must use mercury because the law applies only to mercury.}
    \end{choices}
\end{question}
}

\element{cpo-mc}{
\begin{question}{cpo-ch11-q15}
    What would be the most likely product(s) of the following reaction?
    %%% NOTE: Reword with X as an unknown quantity
    \begin{equation*}
        \ce{H2 + O2 ->}\rule[-0.2pt]{2em}{0.1pt}
    \end{equation*}
    \begin{choices}
        \wrongchoice{\ce{H3 + O3}.}
      \correctchoice{\ce{H2O}.}
        \wrongchoice{\ce{O3H3}.}
        \wrongchoice{No reaction would occur.}
    \end{choices}
\end{question}
}

\element{cpo-mc}{
\begin{question}{cpo-ch11-q16}
    When you activate an instant cold pack,
        water mixes with a chemical and the pack gets very cold.
    This is an example of a(n):
    \begin{choices}
      \correctchoice{endothermic reaction.}
        \wrongchoice{exothermic reaction.}
        \wrongchoice{combustion reaction.}
        \wrongchoice{a physical change.}
    \end{choices}
\end{question}
}

\element{cpo-mc}{
\begin{question}{cpo-ch11-q17}
    The compound \ce{CaCl2} contains which of the following ions?
    \begin{multicols}{2}
    \begin{choices}
        \wrongchoice{\ce{Ca^{+}} and \ce{Cl^{-}}.}
      \correctchoice{\ce{Ca^{2+}} and \ce{Cl^{-}}.}
        \wrongchoice{\ce{Ca^{4+}} and \ce{Cl^{2-}}.}
        \wrongchoice{\ce{Ca^{2+}} and \ce{Cl^{4-}}.}
    \end{choices}
    \end{multicols}
\end{question}
}

\element{cpo-mc}{
\begin{question}{cpo-ch11-q18}
    The atomic mass number for the radioisotope carbon-13 (\ce{C^{13}}) is:
    \begin{multicols}{4}
    \begin{choices}
        \wrongchoice{6}
        \wrongchoice{7}
        \wrongchoice{12}
      \correctchoice{13}
    \end{choices}
    \end{multicols}
\end{question}
}

\element{cpo-mc}{
\begin{question}{cpo-ch11-q19}
    when an unstable isotope undergoes alpha decay, it gives off:
    \begin{choices}
        \wrongchoice{an electron.}
      \correctchoice{two protons and two neutrons.}
        \wrongchoice{high energy electromagnetic radiation.}
        \wrongchoice{a hydrogen atom.}
    \end{choices}
\end{question}
}

\element{cpo-mc}{
\begin{questionmult}{ch11-Q20}
    Which of the following statements is true of gamma radiation?
    \begin{choices}
      \correctchoice{It can be harmful to living things.}
      \correctchoice{The nucleus lowers its energy.}
      \correctchoice{It requires heavy shielding.}
      %\correctchoice{All of the above.}
    \end{choices}
\end{questionmult}
}

\element{cpo-mc}{
\begin{question}{cpo-ch11-q21}
    Which type of radioactivity leaves the atomic number unchanged?
    \begin{multicols}{2}
    \begin{choices}
        \wrongchoice{Alpha decay}
        \wrongchoice{Beta decay}
      \correctchoice{Gamma decay}
        \wrongchoice{Fission}
    \end{choices}
    \end{multicols}
\end{question}
}

\element{cpo-mc}{
\begin{question}{cpo-ch11-q22}
    The half-life is best described as the time it takes for:
    \begin{choices}
        \wrongchoice{an atom to rotate halfway around.}
      \correctchoice{50 percent of a radioactive element to decay into something else.}
        \wrongchoice{each atom of a radioactive element to decay halfway.}
        \wrongchoice{a nuclear fission reaction to split a nucleus in half.}
    \end{choices}
\end{question}
}

\element{cpo-mc}{
\begin{question}{cpo-ch11-q23}
    Radiation is harmful to living things when it:
    \begin{choices}
        \wrongchoice{moves at the speed of light.}
      \correctchoice{has enough energy to break chemical bonds.}
        \wrongchoice{contains any infrared rays.}
        \wrongchoice{is created by nuclear processes instead of chemical processes.}
    \end{choices}
\end{question}
}

\element{cpo-mc}{
\begin{question}{cpo-ch11-q24}
    A fission reaction:
    \begin{choices}
      \correctchoice{breaks a nucleus up into smaller pieces.}
        \wrongchoice{combines light nuclei to make heavier nuclei.}
        \wrongchoice{is any reaction involving an isotope or uranium.}
        \wrongchoice{releases only alpha particles and gamma rays.}
    \end{choices}
\end{question}
}

\element{cpo-mc}{
\begin{question}{cpo-ch11-q25}
    Bromine-82 (\ce{Br^{82}}) is used as a tracer for organic materials in environmental studies.
    Its half-life is \SI{36}{\hour}.
    If you start out with \SI{10}{\gram} of \ce{Br^{82}}, how long will it take
        for there to be \SI{5}{\gram} of \ce{Br^{82}} remaining?
    \begin{multicols}{2}
    \begin{choices}
        \wrongchoice{\SI{10}{\hour}}
        \wrongchoice{\SI{18}{\hour}}
      \correctchoice{\SI{36}{\hour}}
        \wrongchoice{\SI{52}{\hour}}
    \end{choices}
    \end{multicols}
\end{question}
}

\element{cpo-mc}{
\begin{question}{cpo-ch11-q26}
    which kind of reaction do nuclear power plants currently use to generate electricity?
    \begin{multicols}{2}
    \begin{choices}
        \wrongchoice{Fusion}
      \correctchoice{Fission}
        \wrongchoice{Chemical}
        \wrongchoice{Hydro-powered}
    \end{choices}
    \end{multicols}
\end{question}
}

\element{cpo-mc}{
\begin{question}{cpo-ch11-q27}
    A nuclear chain reaction can occur when:
    \begin{choices}
        \wrongchoice{one atom of uranium hits another atom of uranium and causes it to split apart.}
        \wrongchoice{fusion reactions produce helium.}
        \wrongchoice{an isotope gives off alpha decay.}
      \correctchoice{A fission reaction releases enough neutrons to trigger more fission reactions.}
    \end{choices}
\end{question}
}

%\newcommand{\myChapterElevenGraph}{
%    \begin{tikzpicture}
%    \end{tikzpicture}
%}

%\element{cpo-mc}{
%\begin{question}{cpo-ch11-q28}
%    \begin{center}
%        \mychapterElevenGraph
%    \end{center}
%    Which of the following nuclear reactions is an example of a fusion reaction?
%    \begin{multicols}{2}
%    \begin{choices}
%      \correctchoice{helium-4 and carbon-12}
%        \wrongchoice{uranium-235 and strongium-135.}
%        \wrongchoice{radium-228 and actinium-228.}
%        \wrongchoice{radon-220 and polonium-216.}
%    \end{choices}
%    \end{multicols}
%\end{question}
%}

%\element{cpo-mc}{
%\begin{question}{cpo-ch11-q29}
%    \begin{center}
%        \mychapterElevenGraph
%    \end{center}
%    Which of the following is an example of a fission reaction?
%    \begin{multicols}{2}
%    \begin{choices}
%        \wrongchoice{helium-4 and carbon-12}
%        \wrongchoice{carbon-12 and carbon-135.}
%        \wrongchoice{radium-228 and actinium-228.}
%      \correctchoice{carbon-220 and magnesium-24.}
%    \end{choices}
%    \end{multicols}
%\end{question}
%}

\element{cpo-mc}{
\begin{question}{cpo-ch11-q30}
    An amount of sodium-25 decreases to one-fourth its original amount in \SI{2}{\minute}.
    What is the half-life of this radioisotope?
    \begin{multicols}{2}
    \begin{choices}
        \wrongchoice{\SI{4}{\minute}}
        \wrongchoice{\SI{2}{\minute}}
      \correctchoice{\SI{1}{\minute}}
        \wrongchoice{\SI{30}{\second}}
    \end{choices}
    \end{multicols}
\end{question}
}

\element{cpo-mc}{
\begin{question}{cpo-ch11-q31}
    Which of the following nuclear reactions is an example of alpha decay?
    \begin{choices}
        \wrongchoice{carbon-14 to nitrogen-14}
        \wrongchoice{carbon-12 to carbon-12}
      \correctchoice{uranium-234 to thorium-230}
        \wrongchoice{cesium-137 to barium-137}
    \end{choices}
\end{question}
}


\endinput


