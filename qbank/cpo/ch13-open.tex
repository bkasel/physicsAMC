
%%--------------------------------------------------
%% CPO: AMC Open Free Response Questions
%%--------------------------------------------------


%% Chapter 13: Electric Circuits
%%--------------------------------------------------


%% CPO Short Answer Questions
%%--------------------------------------------------
\element{cpo-short}{
\begin{question}{ch13-short-q01}
    why are circuit diagrams drawn?
    %\AMCOpen{lines=3}{
    \AMCOpen{}{
        \wrongchoice[W]{w}\scoring{0}
        \wrongchoice[P]{p}\scoring{1}
        \correctchoice[C]{c}\scoring{2}
    }
\end{question}
}

\element{cpo-short}{
\begin{question}{ch13-short-q02}
    To read the voltage, you accidentally connect the positive lead
        of a voltmeter to a negative terminal of a \SI{1.5}{\volt}
        battery and the negative lead to the positive terminal.
    What is the result?
    \AMCOpen{lines=3}{
        \wrongchoice[W]{w}\scoring{0}
        \wrongchoice[P]{p}\scoring{1}
        \correctchoice[C]{c}\scoring{2}
    }
    %% ANS: \SI{-1.5}{\volt}
\end{question}
}

\element{cpo-short}{
\begin{question}{ch13-short-q03}
    You install two batteries in a flashlight so that their positive
        ends are facing each other.
    Will the flashlight work?
    Why or why not?
    \AMCOpen{lines=3}{
        \wrongchoice[W]{w}\scoring{0}
        \wrongchoice[P]{p}\scoring{1}
        \correctchoice[C]{c}\scoring{2}
    }
    %% ANS: No!
\end{question}
}

\element{cpo-short}{
\begin{question}{ch13-short-q04}
    Explain how a potentiometer is different from a fixed resistor?
    \AMCOpen{lines=3}{
        \wrongchoice[W]{w}\scoring{0}
        \wrongchoice[P]{p}\scoring{1}
        \correctchoice[C]{c}\scoring{2}
    }
    %% ANS: a potentiometer can varry its resistance
\end{question}
}

%% CPO Problem Questions
%%--------------------------------------------------
%\element{cpo-problem}{
%\begin{question}{ch13-problem-q01}
%    which of the circuits pictures below is an open circiut?
%    \begin{center}
%        %% NOTE: add diagrams
%    \end{center}
%    \AMCOpen{lines=3}{
%        \wrongchoice[W]{w}\scoring{0}
%        \wrongchoice[P]{p}\scoring{1}
%        \correctchoice[C]{c}\scoring{2}
%    }
%\end{question}
%}

%\element{cpo-problem}{
%\begin{question}{ch13-problem-q02}
%    What should the voltmeter read (approximately)?
%    \begin{center}
%        %% NOTE: add diagrams
%    \end{center}
%    \AMCOpen{lines=3}{
%        \wrongchoice[W]{w}\scoring{0}
%        \wrongchoice[P]{p}\scoring{1}
%        \correctchoice[C]{c}\scoring{2}
%    }
%    %% ANS: \SI{3.0}{\volt}
%\end{question}
%}

\element{cpo-problem}{
\begin{question}{ch13-problem-q03}
    A miniature light bulb with resistance of \SI{3}{\ohm} is connected
        to a \SI{6}{\volt} source.
    How much current will flow through the bulb?
    \AMCOpen{lines=3}{
        \wrongchoice[W]{w}\scoring{0}
        \wrongchoice[P]{p}\scoring{1}
        \correctchoice[C]{c}\scoring{2}
    }
    %% ANS: \SI{2}{\ampere}
\end{question}
}

\element{cpo-problem}{
\begin{question}{ch13-problem-q04}
    If the current moving through the filament of a light bulb is \SI{0.5}{\ampere}
        When the voltage across the bulb is \SI{120}{\volt},
        what is the resistance of the bulb?
    \AMCOpen{lines=3}{
        \wrongchoice[W]{w}\scoring{0}
        \wrongchoice[P]{p}\scoring{1}
        \correctchoice[C]{c}\scoring{2}
    }
    %% ANS: \SI{240}{\ohm}
\end{question}
}

\element{cpo-problem}{
\begin{question}{ch13-problem-q05}
    Typically, household appliances operate at \SI{120}{\volt}.
    What is the current flowing in the circuit of a microwave when the
        resistance of the microwave oven is \SI{30}{\ohm}?
    \AMCOpen{lines=3}{
        \wrongchoice[W]{w}\scoring{0}
        \wrongchoice[P]{p}\scoring{1}
        \correctchoice[C]{c}\scoring{2}
    }
    %% ANS: \SI{4}{\ampere}
\end{question}
}

\newcommand{\myThirteenCircuit}{
    \ctikzset{bipoles/length=0.75cm}
    \begin{circuitikz}[scale=1.20]
        \draw (0,0) to [battery,l=\SI{9}{\volt}] (0,2)
                    to [ammeter,i=\SI{3}{\ampere}] (2,2)
                    to [R] (2,0)
                    to (0,0);
    \end{circuitikz}
}

\element{cpo-problem}{
\begin{question}{ch13-problem-q06}
    A light bulb is connected to a \SI{9}{\volt} battery that causes
        \SI{3}{\ampere} of current to flow through the bulb.
    \begin{center}
        \myThirteenCircuit
    \end{center}
    What is the resistance of the light bulb in the circuit diagram?
    \AMCOpen{lines=3}{
        \wrongchoice[W]{w}\scoring{0}
        \wrongchoice[P]{p}\scoring{1}
        \correctchoice[C]{c}\scoring{2}
    }
    %% ANS: \SI{3}{\ohm}
\end{question}
}

\element{cpo-problem}{
\begin{question}{ch13-problem-q07}
    A light bulb is connected to a \SI{9}{\volt} battery that causes
        \SI{3}{\ampere} of current to flow through the bulb.
    \begin{center}
        \myThirteenCircuit
    \end{center}
    If the lightbulb in the circiut is replaced with a lightbulb having
        a resistance of \SI{9}{\ohm}, what would be the new current in
        the circuit?
    \AMCOpen{lines=3}{
        \wrongchoice[W]{w}\scoring{0}
        \wrongchoice[P]{p}\scoring{1}
        \correctchoice[C]{c}\scoring{2}
    }
    %% ANS: \SI{1}{\ampere}
\end{question}
}

\element{cpo-problem}{
\begin{question}{ch13-problem-q08}
    A light bulb is connected to a \SI{9}{\volt} battery that causes
        \SI{3}{\ampere} of current to flow through the bulb.
    \begin{center}
        \myThirteenCircuit
    \end{center}
    If a second \SI{9}{\volt} batter is stacked with the battery shown,
        what would be the new current in the circuit?
    \AMCOpen{lines=3}{
        \wrongchoice[W]{w}\scoring{0}
        \wrongchoice[P]{p}\scoring{1}
        \correctchoice[C]{c}\scoring{2}
    }
    %% ANS: \SI{2}{\ampere}
\end{question}
}

%\element{cpo-problem}{
%\begin{question}{ch13-problem-q09}
%    This is a graph of the current that flows through a mini-bulb
%        as the voltage is changed.
%    Use the graph and  your knowledge of Ohm's Law to determine the
%        resistance of the bulb.
%    \begin{center}
%        %% NOTE add pgfpltos
%    \end{center}
%    \AMCOpen{lines=3}{
%        \wrongchoice[W]{w}\scoring{0}
%        \wrongchoice[P]{p}\scoring{1}
%        \correctchoice[C]{c}\scoring{2}
%    }
%    %% ANS: \SI{2}{\ohm}
%\end{question}
%}

\element{cpo-problem}{
\begin{question}{ch13-problem-q10}
    A \SI{120}{\volt} household circuit has a fuse that breaks
        the circuit if more than \SI{10}{\ampere} of current passes
        through it.
    What is the minimum amount of resistance in the circuit to keep
        the fuse from blowing?
    \AMCOpen{lines=3}{
        \wrongchoice[W]{w}\scoring{0}
        \wrongchoice[P]{p}\scoring{1}
        \correctchoice[C]{c}\scoring{2}
    }
    %% ANS: \SI{12}{\ohm}
\end{question}
}


%% CPO Essay Questions
%%--------------------------------------------------
\element{cpo-essay}{
\begin{question}{ch13-essay-q01}
    Describe a resistor and its function in a circuit?
    \AMCOpen{lines=3}{
        \wrongchoice[W]{w}\scoring{0}
        \wrongchoice[P]{p}\scoring{1}
        \correctchoice[C]{c}\scoring{2}
    }
\end{question}
}

\element{cpo-essay}{
\begin{question}{ch13-essay-q02}
    According to Ohm's law, how is current related to the resistance in a circuit?
    \AMCOpen{lines=3}{
        \wrongchoice[W]{w}\scoring{0}
        \wrongchoice[P]{p}\scoring{1}
        \correctchoice[C]{c}\scoring{2}
    }
\end{question}
}


\endinput


