
%%--------------------------------------------------
%% CPO: Multiple Choice Questions
%%--------------------------------------------------


%% Chapter 18: Fields and Forces
%%--------------------------------------------------


%% Learning Objectives
%%--------------------------------------------------

%% Propose a general explanation for how forces act over a distance.
%% Describe why the strength of a field decreases with distance from the source. 
%% Describe how fast a force moves between objects. 
%% Describe how gravity acts between objects through a field. 
%% Calculate the gravitational field on other planets. 
%% Describe how electric charges act in an electric field. 
%% Calculate electric field strength.


%% CPO Multiple Choice Questions
%%--------------------------------------------------
\element{cpo-mc}{
\begin{question}{cpo-ch18-q01}
    All of the following are types of fields \emph{except}:
    \begin{multicols}{2}
    \begin{choices}
        \wrongchoice{gravity}
        \wrongchoice{light}
        \wrongchoice{magnetism}
      \correctchoice{mass}
    \end{choices}
    \end{multicols}
\end{question}
}

\element{cpo-mc}{
\begin{question}{cpo-ch18-q02}
    The strength of a field:
    \begin{choices}
      \correctchoice{decreases the farther you get from the source.}
        \wrongchoice{increases the farther you get from the source.}
        \wrongchoice{stays the same throughout.}
        \wrongchoice{varies randomly throughout.}
    \end{choices}
\end{question}
}

\element{cpo-mc}{
\begin{question}{cpo-ch18-q03}
    The strength of a magnetic field compared to an electric or gravitational field:
    \begin{choices}
        \wrongchoice{increases more quickly as you get farther from the source.}
      \correctchoice{decreases more quickly as you get farther from the source.}
        \wrongchoice{is identical.}
        \wrongchoice{cannot be compared.}
    \end{choices}
\end{question}
}

\element{cpo-mc}{
\begin{question}{cpo-ch18-q04}
    The force of gravity you feel from Earth reaches you through:
    \begin{choices}
        \wrongchoice{Earth's magnetic field}
        \wrongchoice{Earth's core}
      \correctchoice{Earth's gravitational field}
        \wrongchoice{electromagnetic waves}
    \end{choices}
\end{question}
}

\element{cpo-mc}{
\begin{questionmult}{ch18-Q05}
    What type of field surrounds a moving charged particle?
    \begin{choices}
      \correctchoice{Electric Field}
      \correctchoice{Magnetic Field}
      \correctchoice{Gravitational Field}
    \end{choices}
\end{questionmult}
}

\element{cpo-mc}{
\begin{question}{cpo-ch18-q06}
    If an object with a charge of \SI{0.05}{\coulomb} experiences an electric force of \SI{5}{\newton},
        the electric field strength is:
    \begin{multicols}{2}
    \begin{choices}
        \wrongchoice{\SI{0.01}{\newton\per\coulomb}}
        \wrongchoice{\SI{0.25}{\newton\per\coulomb}}
      \correctchoice{\SI{100}{\newton\per\coulomb}}
        \wrongchoice{\SI{500}{\newton\per\coulomb}}
    \end{choices}
    \end{multicols}
\end{question}
}

\element{cpo-mc}{
\begin{question}{cpo-ch18-q07}
    Gravitational fields and electric fields are similar in all the following ways \emph{except}:
    \begin{choices}
        \wrongchoice{their intensities follow an inverse square law.}
        \wrongchoice{they are both vector fields.}
        \wrongchoice{they are both force fields.}
      \correctchoice{they both are created by mass.}
    \end{choices}
\end{question}
}

\element{cpo-mc}{
\begin{question}{cpo-ch18-q08}
    How does the intensity of light \SI{2}{\meter} from a light bulb compare to the intensity \SI{4}{\meter} away from the light bulb?
    \begin{choices}
        \wrongchoice{It is \num{2} times more intense.}
        \wrongchoice{It is \num{2} times less intense.}
      \correctchoice{It is \num{4} times more intense.}
        \wrongchoice{It is \num{4} times less intense.}
    \end{choices}
\end{question}
}

\element{cpo-mc}{
\begin{question}{cpo-ch18-q09}
    The fastest speed of a field can spread forces,
        energy, or information is:
    \begin{choices}
        \wrongchoice{\SI{100 000}{\meter\per\second}}
      \correctchoice{\SI{300 000 000}{\meter\per\second}}
        \wrongchoice{\SI{9.8}{\meter\per\second}}
        \wrongchoice{there is no limit to the speed}
    \end{choices}
\end{question}
}

\element{cpo-mc}{
\begin{question}{cpo-ch18-q10}
    The distance between the sun and Earth is \SI{1.5e11}{\meter}.
    The mass of the sun is \SI{1.99e30}{\kilo\gram}.
    The radius of the sun is \SI{6.9e8}{\meter}.
    How long does it take the light produced by the sun to travel to your eyes on Earth?
    \begin{multicols}{2}
    \begin{choices}
        \wrongchoice{\SI{0.002}{\second}}
        \wrongchoice{\SI{4.5e19}{\second}}
      \correctchoice{\SI{500}{\second}}
        \wrongchoice{instantly}
    \end{choices}
    \end{multicols}
\end{question}
}

\element{cpo-mc}{
\begin{question}{cpo-ch18-q11}
    The distance between the sun and Earth is \SI{1.5e11}{\meter}.
    The mass of the sun is \SI{1.99e30}{\kilo\gram}.
    The radius of the sun is \SI{6.9e8}{\meter}.
    If the sun were to explode, how long would it be before the explosion would be seen on Earth?
    \begin{multicols}{2}
    \begin{choices}
        \wrongchoice{\SI{0.002}{\second}}
        \wrongchoice{\SI{4.5e19}{\second}}
      \correctchoice{\SI{500}{\second}}
        \wrongchoice{instantly}
    \end{choices}
    \end{multicols}
\end{question}
}

\element{cpo-mc}{
\begin{question}{cpo-ch18-q12}
    The distance between the sun and Earth is \SI{1.5e11}{\meter}.
    The mass of the sun is \SI{1.99e30}{\kilo\gram}.
    The radius of the sun is \SI{6.9e8}{\meter}.
    If the sun were to explode and vanish,
        which of the following would happen to Earth?
    \begin{choices}
        \wrongchoice{Earth would immediately explode and vanish.}
      \correctchoice{Earth would fly out of its orbit after a \SI{500}{\second} delay.}
        \wrongchoice{Earth would fly out of it orbit immediately.}
        \wrongchoice{No change would happen.}
    \end{choices}
\end{question}
}

\element{cpo-mc}{
\begin{question}{cpo-ch18-q13}
    The distance between the sun and Earth is \SI{1.5e11}{\meter}.
    The mass of the sun is \SI{1.99e30}{\kilo\gram}.
    The radius of the sun is \SI{6.9e8}{\meter}.
    The gravitational field strength \emph{due to the sun} at the surface of Earth is:
    \begin{multicols}{2}
    \begin{choices}
        \wrongchoice{\SI{9.8}{\newton\per\kilo\gram}}
        \wrongchoice{\SI{8.9e8}{\newton\per\kilo\gram}}
        \wrongchoice{\SI{7.5e-19}{\newton\per\kilo\gram}}
      \correctchoice{\SI{0.006}{\newton\per\kilo\gram}}
    \end{choices}
    \end{multicols}
\end{question}
}

\element{cpo-mc}{
\begin{question}{cpo-ch18-q14}
    The distance between the sun and Earth is \SI{1.5e11}{\meter}.
    The mass of the sun is \SI{1.99e30}{\kilo\gram}.
    The radius of the sun is \SI{6.9e8}{\meter}.
    The gravitational field (value of $g$) on the surface of the sun is:
    \begin{multicols}{2}
    \begin{choices}
        \wrongchoice{\SI{1.9e11}{\newton\per\kilo\gram}}
        \wrongchoice{\SI{6.67e-11}{\newton\per\kilo\gram}}
      \correctchoice{\SI{279}{\newton\per\kilo\gram}}
        \wrongchoice{\SI{9.5e47}{\newton\per\kilo\gram}}
    \end{choices}
    \end{multicols}
\end{question}
}

\element{cpo-mc}{
\begin{question}{cpo-ch18-q15}
    The gravitational field (value of $g$) at the surface of
        a \SI{0.045}{\kilo\gram} ball with a radius of \SI{0.021}{\meter} is:
    \begin{multicols}{2}
    \begin{choices}
        \wrongchoice{\SI{6.67e-11}{\newton\per\kilo\gram}}
        \wrongchoice{\SI{1.4e-10}{\newton\per\kilo\gram}}
        \wrongchoice{\SI{9.8}{\newton\per\kilo\gram}}
      \correctchoice{\SI{6.8e-9}{\newton\per\kilo\gram}}
    \end{choices}
    \end{multicols}
\end{question}
}

\element{cpo-mc}{
\begin{question}{cpo-ch18-q16}
    What creates an electric field?
    \begin{choices}
        \wrongchoice{Drift speed}
      \correctchoice{The forces between charged particles}
        \wrongchoice{Magnetic attractions}
        \wrongchoice{The forces between masses}
    \end{choices}
\end{question}
}

\element{cpo-mc}{
\begin{question}{cpo-ch18-q17}
    What happens to an electric field as you get farther away from the charge that creates the field?
    \begin{choices}
        \wrongchoice{It changes to a magnetic field}
      \correctchoice{It decreases}
        \wrongchoice{It increases}
        \wrongchoice{It remains constant}
    \end{choices}
\end{question}
}

\element{cpo-mc}{
\begin{question}{cpo-ch18-q18}
    The electric field inside a conductor that is not carrying current is:
    \begin{multicols}{2}
    \begin{choices}
        \wrongchoice{increasing}
      \correctchoice{zero}
        \wrongchoice{positive}
        \wrongchoice{negative}
    \end{choices}
    \end{multicols}
\end{question}
}

\element{cpo-mc}{
\begin{question}{cpo-ch18-q19}
    Electric field lines always point?
    \begin{choices}
      \correctchoice{away from positive charge and toward negative charge}
        \wrongchoice{toward positive charge and away from negative charge}
        \wrongchoice{across each other}
        \wrongchoice{to the inside of a conductor}
    \end{choices}
\end{question}
}

\element{cpo-mc}{
\begin{question}{cpo-ch18-q20}
    Placing a conductor into an electric field creates a:
    \begin{choices}
      \correctchoice{shielding effect with no electric field inside the conductor}
        \wrongchoice{current inside the conductor}
        \wrongchoice{negative charge inside the conductor}
        \wrongchoice{positive charge inside the conductor}
    \end{choices}
\end{question}
}

\element{cpo-mc}{
\begin{question}{cpo-ch18-q21}
    The force of an electric field of strength \SI{2.0}{\newton\per\coulomb} on a positive charge of \SI{0.5}{\coulomb} is:
    \begin{multicols}{2}
    \begin{choices}
        \wrongchoice{\SI{0.25}{\newton\per\coulomb}}
        \wrongchoice{\SI{0.5}{\newton\per\coulomb}}
      \correctchoice{\SI{1.0}{\newton\per\coulomb}}
        \wrongchoice{\SI{4.0}{\newton\per\coulomb}}
    \end{choices}
    \end{multicols}
\end{question}
}

\element{cpo-mc}{
\begin{questionmult}{ch18-Q22}
    An object with charge \SI{5e-9}{\coulomb} experiences an upward force of \SI{20e-9}{\newton} when placed at a certain point in an electric field.
    The electric field strength at that point is:
    \begin{multicols}{2}
    \begin{choices}
        \wrongchoice{\SI{0.25}{\newton\per\coulomb}}
      \correctchoice{\SI{4.0}{\newton\per\coulomb}}
        \wrongchoice{\SI{100}{\newton\per\coulomb}}
        \wrongchoice{\SI{4e-9}{\newton\per\coulomb}}
    \end{choices}
    \end{multicols}
\end{questionmult}
}

%\element{cpo-mc}{
%\begin{question}{cpo-ch18-q23}
%    Which of the following diagrams is \emph{not} a possible
%        representation of an electric field?
%    \begin{multicols}{2}
%    \begin{choices}
%        \AMCboxDimensions{down=-2.5em}
%        %% NOTE: TODO: draw tikz
%    \end{choices}
%    \end{multicols}
%\end{question}
%}

\element{cpo-mc}{
\begin{question}{cpo-ch18-q24}
    A positive charge of \SI{0.002}{\coulomb} is in a \SI{2}{\volt\per\meter} electric field.
    The force on the charge is:
    \begin{multicols}{2}
    \begin{choices}
        \wrongchoice{\SI{9.8}{\newton}}
      \correctchoice{\SI{0.004}{\newton}}
        \wrongchoice{\SI{1 000}{\newton}}
        \wrongchoice{\SI{0.001}{\newton}}
    \end{choices}
    \end{multicols}
\end{question}
}

\element{cpo-mc}{
\begin{question}{cpo-ch18-q25}
    A negative charge of \SI{0.01}{\coulomb} is in a \SI{200}{\volt\per\meter} electric field.
    The force on the charge is:
    \begin{multicols}{2}
    \begin{choices}
        \wrongchoice{\SI{9.8}{\newton}}
      \correctchoice{\SI{2}{\newton}}
        \wrongchoice{\SI{20 000}{\newton}}
        \wrongchoice{\SI{5e-5}{\newton}}
    \end{choices}
    \end{multicols}
\end{question}
}

\element{cpo-mc}{
\begin{questionmult}{ch18-Q26}
    Which of the following are a unit used to measure the strength of an electric field?
    \begin{multicols}{2}
    \begin{choices}
      \correctchoice{\si{\volt\per\meter}}
      \correctchoice{\si{\newton\per\coulomb}}
      \correctchoice{\si{\kilogram\meter\per\second\squared\per\coulomb}}
        %% NOTE: modified
        \wrongchoice{\si{\newton\per\meter}}
    \end{choices}
    \end{multicols}
\end{questionmult}
}

\element{cpo-mc}{
\begin{question}{cpo-ch18-q27}
    The graph represents the relationship between electric force and the charge of an object.
    \begin{center}
    \begin{tikzpicture}
        \begin{axis}[
            axis y line=left,
            axis x line=bottom,
            axis line style={->},
            xlabel={charge},
            x unit=\si{\coulomb},
            xtick={0,2,4,6,8},
            ylabel={force},
            y unit=\si{\newton},
            ytick={0,2,4,6},
            xmin=0,xmax=8.1,
            ymin=0,ymax=6.1,
            grid=major,
            width=0.8\columnwidth,
            height=0.5\columnwidth,
            very thin,
        ]
        \addplot[line width=1.5pt,domain=0:8]{5*x/9};
        \end{axis}
    \end{tikzpicture}
    \end{center}
    The slope of the graph represents:
    \begin{choices}
      \correctchoice{the strength of the electric field}
        \wrongchoice{Coulomb's constant, $k=\SI{9e9}{\newton\meter\squared\per\coulomb}$}
        \wrongchoice{momentum}
        \wrongchoice{voltage}
    \end{choices}
\end{question}
}

%\element{cpo-mc}{
%\begin{question}{cpo-ch18-q28}
%    The electric field around two positive charges looks most like:
%    \begin{multicols}{2}
%    \begin{choices}
%        \AMCboxDimensions{down=-2.5em}
%        %% NOTE: TODO: draw tikz
%    \end{choices}
%    \end{multicols}
%\end{question}
%}

\endinput

