%%--------------------------------------------------
%% Customizations
%%--------------------------------------------------

%% http://www.texample.net/tikz/examples/periodic-table-of-chemical-elements/
\newcommand{\CommonElementTextFormat}[4]
{
  \begin{minipage}{2.2cm}
    \centering
      {\textbf{#1} \hfill #2}%
      \linebreak \linebreak
      {\textbf{#3}}%
      \linebreak \linebreak
      {{#4}}
  \end{minipage}
}


%%--------------------------------------------------
%% CPO: AMC Open Free Response Questions
%%--------------------------------------------------


%% Chapter 9: The Atom
%%--------------------------------------------------


%% CPO Short Answer Questions
%%--------------------------------------------------
\element{cpo-short}{
\begin{question}{ch09-short-q01}
    List the four forces found within atoms and rank them in order of
        decreasing strength.
    \AMCOpen{lines=3}{
        \wrongchoice[W]{w}\scoring{0}
        \wrongchoice[P]{p}\scoring{1}
        \correctchoice[C]{c}\scoring{2}
    }
\end{question}
}

\element{cpo-short}{
\begin{question}{ch09-short-q02}
    Describe the reason the atomic mass of magnesium is listed
        as \SI{24.31}{\amu} when magnesium has 3 stable isotopes:
        Mg\textsuperscript{24}, Mg\textsuperscript{25},
        Mg\textsuperscript{26}.
    Which isotope is the most commonly found on Earth?
    \AMCOpen{lines=3}{
        \wrongchoice[W]{w}\scoring{0}
        \wrongchoice[P]{p}\scoring{1}
        \correctchoice[C]{c}\scoring{2}
    }
\end{question}
}

\element{cpo-short}{
\begin{question}{ch09-short-q03}
    For the nucleus shown below, do the following:
    \begin{center}
        %% NOTE: add diagram
    \end{center}
    \begin{enumerate*}
        \item Name the element.
        \item Give the mass number.
        \item Give the number of electrons.
    \end{enumerate*}
    \AMCOpen{lines=3}{
        \wrongchoice[W]{w}\scoring{0}
        \wrongchoice[P]{p}\scoring{1}
        \correctchoice[C]{c}\scoring{2}
    }
    %% ANS: A) carbon; B) 13; C) 6
\end{question}
}

\element{cpo-short}{
\begin{question}{ch09-short-q04}
    How many energy levels would be completely filled by an atom
        of neon (Ne)?
    \begin{center}
        \NaturalElementTextFormat{10}{20.180}{Ne}{Neon}
    \end{center}
    How many electrons would be left over?
    \AMCOpen{lines=3}{
        \wrongchoice[W]{w}\scoring{0}
        \wrongchoice[P]{p}\scoring{1}
        \correctchoice[C]{c}\scoring{2}
    }
    %% ANS: two filled energy levels and no electrons left over
\end{question}
}

\element{cpo-short}{
\begin{question}{ch09-short-q05}
    What is a quanta?
    \AMCOpen{lines=3}{
        \wrongchoice[W]{w}\scoring{0}
        \wrongchoice[P]{p}\scoring{1}
        \correctchoice[C]{c}\scoring{2}
    }
\end{question}
}

\element{cpo-short}{
\begin{question}{ch09-short-q06}
    What is the evidence that electrons in atoms are only allowed to
        have specific amounts of energy?
    \AMCOpen{lines=3}{
        \wrongchoice[W]{w}\scoring{0}
        \wrongchoice[P]{p}\scoring{1}
        \correctchoice[C]{c}\scoring{2}
    }
\end{question}
}


%% CPO Problem Questions
%%--------------------------------------------------
\element{cpo-problem}{
\begin{question}{ch09-problem-q01}
    The figure shows a 4-sided dice, with the sides labels 1 through 4.
    If you were to roll this dice once, what is the probability that
        you would roll a 3?
    If you rolled the dice 100 times, how many times would you expect
        to roll a 3?
    \AMCOpen{lines=3}{
        \wrongchoice[W]{w}\scoring{0}
        \wrongchoice[P]{p}\scoring{1}
        \correctchoice[C]{c}\scoring{2}
    }
    %% ANS: \num{25}
\end{question}
}


%% CPO Essay Questions
%%--------------------------------------------------
\element{cpo-essay}{
\begin{question}{ch09-essay-q01}
    Describe Ernest Rutherford's golf foil experiments and the
        changes it led to in the model of the atom.
    \AMCOpen{lines=3}{
        \wrongchoice[W]{w}\scoring{0}
        \wrongchoice[P]{p}\scoring{1}
        \correctchoice[C]{c}\scoring{2}
    }
\end{question}
}

\element{cpo-essay}{
\begin{question}{ch09-essay-q02}
    Describe how electrons are involved in the formation of chemical bonds.
    \AMCOpen{lines=3}{
        \wrongchoice[W]{w}\scoring{0}
        \wrongchoice[P]{p}\scoring{1}
        \correctchoice[C]{c}\scoring{2}
    }
\end{question}
}

\element{cpo-essay}{
\begin{question}{ch09-essay-q03}
    Draw dot diagram for the following elements:
    \begin{enumerate*}
        \item Xenon (Xe) has 8 valence electron
        \item Potassium (K) has 1 valence electron
        \item Fluorine (F) has 7 valence electron
    \begin{enumerate*}
    \AMCOpen{lines=3}{
        \wrongchoice[W]{w}\scoring{0}
        \wrongchoice[P]{p}\scoring{1}
        \correctchoice[C]{c}\scoring{2}
    }
\end{question}
}

\endinput


