
%%--------------------------------------------------
%% CPO: Multiple Choice Questions
%%--------------------------------------------------


%% Chapter 13: Electric Circuits
%%--------------------------------------------------


%% Learning Objectives
%%--------------------------------------------------

%% Explain how electrical energy is supplied to devices in a circuit.
%% Use electrical symbols to draw a simple circuit. 
%% Distinguish between open and closed circuits. 
%% List the units used to measure current and voltage. • Describe how to measure current and voltage in a circuit. 
%% Explain the function of a battery in a circuit. 
%% Explain the relationships between current, voltage, and resistance. 
%% Use Ohm’s law to calculate current, resistance, or voltage. 
%% Distinguish between conductors and insulators


%% CPO Multiple Choice Questions
%%--------------------------------------------------
\element{cpo-mc}{
\begin{question}{cpo-ch13-q01}
    The motion of charges in wires, motors, light bulbs and other devices is best called electric:
    \begin{multicols}{2}
    \begin{choices}
        \wrongchoice{power}
        \wrongchoice{voltage}
      \correctchoice{current}
        \wrongchoice{conductance}
    \end{choices}
    \end{multicols}
\end{question}
}

\element{cpo-mc}{
\begin{questionmult}{ch13-Q02}
    A circuit diagram:
    \begin{choices}
      \correctchoice{uses symbols to represent each part of the circuit.}
      \correctchoice{can be interpreted by anyone familiar with electricity.}
      \correctchoice{makes the design of an electrical circuit easier.}
      %\correctchoice{is described, in part, by each of the above statements.}
    \end{choices}
\end{questionmult}
}

\element{cpo-mc}{
\begin{question}{cpo-ch13-q03}
    In many circuit diagrams, any device that uses electrical energy is represented by a:
    \begin{multicols}{2}
    \begin{choices}
        \wrongchoice{light bulb}
        \wrongchoice{heating element}
      \correctchoice{resistor}
        \wrongchoice{open circuit}
    \end{choices}
    \end{multicols}
\end{question}
}

\element{cpo-mc}{
\begin{question}{cpo-ch13-q04}
    A circuit with a switch turned to the \emph{off} position or a circuit with any break is called a(n):
    \begin{multicols}{2}
    \begin{choices}
        \wrongchoice{closed circuit}
      \correctchoice{short circuit}
        \wrongchoice{resistor circuit}
        \wrongchoice{open circuit}
    \end{choices}
    \end{multicols}
\end{question}
}

\element{cpo-mc}{
\begin{question}{cpo-ch13-q05}
    A closed circuit:
    \begin{choices}
        \wrongchoice{is off}
      \correctchoice{is on}
        \wrongchoice{has a break in it}
        \wrongchoice{required no voltage}
    \end{choices}
\end{question}
}

\element{cpo-mc}{
\begin{question}{cpo-ch13-q06}
    Suppose you are trying to help someone gain a better understanding of electric circuits.
    If you compare an electrical circuit to a system that carries water,
        what would the water pipes represent?
    \begin{multicols}{2}
    \begin{choices}
        \wrongchoice{Battery}
      \correctchoice{Wires}
        \wrongchoice{Electromagnet}
        \wrongchoice{Switch}
    \end{choices}
    \end{multicols}
\end{question}
}

\element{cpo-mc}{
\begin{question}{cpo-ch13-q07}
    All of the following devices are used to create an open circuit \emph{except} a:
    \begin{multicols}{2}
    \begin{choices}
        \wrongchoice{fuse}
      \correctchoice{battery}
        \wrongchoice{switch}
        \wrongchoice{circuit breaker}
    \end{choices}
    \end{multicols}
\end{question}
}

\element{cpo-mc}{
\begin{question}{cpo-ch13-q08}
    In an electrical circuit, the term \emph{current} refers to:
    \begin{choices}
        \wrongchoice{resistance}
        \wrongchoice{potential difference}
      \correctchoice{flowing charges}
        \wrongchoice{energy loss}
    \end{choices}
\end{question}
}

\element{cpo-mc}{
\begin{question}{cpo-ch13-q09}
    When you talk about a battery's \emph{voltage}, you are referring to:
    \begin{choices}
        \wrongchoice{how easily current moves through it.}
        \wrongchoice{the flow of electricity through the battery.}
        \wrongchoice{its ability to carry electric current.}
      \correctchoice{the difference of energy per unit charge.}
    \end{choices}
\end{question}
}

\element{cpo-mc}{
\begin{question}{cpo-ch13-q10}
    A device that uses chemical energy to push current in a circuit is called a:
    \begin{multicols}{2}
    \begin{choices}
      \correctchoice{battery}
        \wrongchoice{voltmeter}
        \wrongchoice{ammeter}
        \wrongchoice{potentiometer}
    \end{choices}
    \end{multicols}
\end{question}
}

\element{cpo-mc}{
\begin{question}{cpo-ch13-q11}
    Devices that can be used to measure current directly include the multimeter and the:
    \begin{multicols}{2}
    \begin{choices}
        \wrongchoice{potentiometer}
      \correctchoice{ammeter}
        \wrongchoice{voltmeter}
        \wrongchoice{ohmeter}
    \end{choices}
    \end{multicols}
\end{question}
}

\element{cpo-mc}{
\begin{question}{cpo-ch13-q12}
    A device that protects a circuit from too much current by creating a break in the circuit is a:
    \begin{multicols}{2}
    \begin{choices}
        \wrongchoice{switch}
        \wrongchoice{short}
      \correctchoice{fuse}
        \wrongchoice{resistor}
    \end{choices}
    \end{multicols}
\end{question}
}

\element{cpo-mc}{
\begin{question}{cpo-ch13-q13}
    The energy carried by each unit of moving charge in a circuit is called:
    \begin{multicols}{2}
    \begin{choices}
        \wrongchoice{amperage}
      \correctchoice{voltage}
        \wrongchoice{resistance}
        \wrongchoice{wattage}
    \end{choices}
    \end{multicols}
\end{question}
}

\element{cpo-mc}{
\begin{question}{cpo-ch13-q14}
    In an electrical circuit, a voltage difference:
    \begin{choices}
      \correctchoice{supplies energy to make a charge flow.}
        \wrongchoice{causes a short circuit.}
        \wrongchoice{is the term used for rate of charge flow.}
        \wrongchoice{wastes energy.}
    \end{choices}
\end{question}
}

\element{cpo-mc}{
\begin{question}{cpo-ch13-q15}
    Which of the following makes a good analogy for a battery?
    \begin{choices}
        \wrongchoice{Water pipes}
        \wrongchoice{Narrow areas in water pipes}
        \wrongchoice{Water faucet}
      \correctchoice{Water tower and pump}
    \end{choices}
\end{question}
}

\element{cpo-mc}{
\begin{question}{cpo-ch13-q16}
    What should the voltmeter read (approximately)?
    \begin{center}
    \ctikzset{bipoles/length=0.75cm}
    \begin{circuitikz}[scale=1.20]
        \draw (0,0) to [voltmeter] (3,0) to (3,1) to [battery,l=\SI{1.5}{\volt}] (2,1) to [battery,l=\SI{1.5}{\volt}] (1,1) to [battery,l=\SI{1.5}{\volt}] (0,1) to (0,0);
    \end{circuitikz}
    \end{center}
    \begin{multicols}{2}
    \begin{choices}
      \correctchoice{\SI{4.5}{\volt}}
        \wrongchoice{\SI{1.5}{\volt}}
        \wrongchoice{\SI{3.375}{\volt}}
        \wrongchoice{\SI{3}{\volt}}
    \end{choices}
    \end{multicols}
\end{question}
}

%% NOTE: circuit diagram used for Q17 to Q19
\newcommand{\cpoThirteenCircuitOne}{
    \ctikzset{bipoles/length=0.75cm}
    \begin{circuitikz}[scale=1.60]
        \draw (0,0) to [battery] (0,1) to (1,1)
                    to [R] (0,1) to (0,0);
        \draw (1,1) to (1,2) 
                    to [ammeter] (0,2) to (0,1);
    \end{circuitikz}
}

\newcommand{\cpoThirteenCircuitTwo}{
    \ctikzset{bipoles/length=0.75cm}
    \begin{circuitikz}[scale=1.60]
        \draw (0,0) to [battery] (0,1)
                    to [R] (0.5,1)
                    to [ammeter] (0,0);
    \end{circuitikz}
}

\newcommand{\cpoThirteenCircuitThree}{
    \ctikzset{bipoles/length=0.75cm}
    \begin{circuitikz}[scale=1.60]
        \draw (0,0) to [battery] (0,1)
                    to [R] (0.5,1)
                    to [ammeter] (0,0);
    \end{circuitikz}
}

%\element{cpo-mc}{
%\begin{question}{cpo-ch13-q17}
%    The diagram that shows a meter propertly connected for measuring
%        current through the light bulb is:
%    \begin{multicols}{4}
%    \begin{choices}[o]
%        \wrongchoice{\cpoThirteenCircuitOne}
%        \wrongchoice{\cpoThirteenCircuitTwo}
%        \wrongchoice{\cpoThirteenCircuitThree}
%        \wrongchoice{\cpoThirteenCircuitFour}
%    \end{choices}
%    \end{multicols}
%\end{question}
%}

%\element{cpo-mc}{
%\begin{question}{cpo-ch13-q18}
%    The diagram that shows a meter properly connected for measuring
%        voltage across the light bulb is:
%    \begin{multicols}{4}
%    \begin{choices}[o]
%        \wrongchoice{\cpoThirteenCircuitOne}
%        \wrongchoice{\cpoThirteenCircuitTwo}
%        \wrongchoice{\cpoThirteenCircuitThree}
%        \wrongchoice{\cpoThirteenCircuitFour}
%    \end{choices}
%    \end{multicols}
%\end{question}
%}

\element{cpo-mc}{
\begin{question}{cpo-ch13-q19}
    What voltage reading would you get if you connect both probes of a voltmeter to one end of a \SI{1.5}{\volt} battery?
    \begin{multicols}{2}
    \begin{choices}
        \wrongchoice{\SI{1.5}{\volt}}
        \wrongchoice{\SI{1.5}{\ampere}}
      \correctchoice{\SI{0}{\volt}}
        \wrongchoice{\SI{4.5}{\ampere}}
    \end{choices}
    \end{multicols}
\end{question}
}

\element{cpo-mc}{
\begin{question}{cpo-ch13-q20}
    All of the following are considered conductors \emph{except}:
    \begin{multicols}{2}
    \begin{choices}
        \wrongchoice{iron}
        \wrongchoice{gold}
      \correctchoice{silicon}
        \wrongchoice{copper}
    \end{choices}
    \end{multicols}
\end{question}
}

\element{cpo-mc}{
\begin{question}{cpo-ch13-q21}
    Materials through which current will not flow easily are called:
    \begin{multicols}{2}
    \begin{choices}
        \wrongchoice{conductors}
        \wrongchoice{semiconductors}
      \correctchoice{insulators}
        \wrongchoice{absorbers.
    \end{choices}
    \end{multicols}
\end{question}
}

\element{cpo-mc}{
\begin{question}{cpo-ch13-q22}
    Electric current will pass easily through a(n):
    \begin{multicols}{2}
    \begin{choices}
        \wrongchoice{absorber}
      \correctchoice{conductor}
        \wrongchoice{semiconductor}
        \wrongchoice{insulator}
    \end{choices}
    \end{multicols}
\end{question}
}

\element{cpo-mc}{
\begin{question}{cpo-ch13-q23}
    Which of the following could be a good conductor of electricity?
    \begin{choices}
      \correctchoice{Metal pot}
        \wrongchoice{Ceramic coffee cup}
        \wrongchoice{Piece of foam packing material}
        \wrongchoice{Plastic spoon}
    \end{choices}
\end{question}
}

\element{cpo-mc}{
\begin{question}{cpo-ch13-q24}
    If you look inside a stereo or telephone,
        you will find a circuit board,
        which has wires printed on it and is covered with little parts.
    Components called \rule[-0.1pt]{4em}{0.1pt} are used to control current in the circuits on the board.
    \begin{multicols}{2}
    \begin{choices}
      \correctchoice{resistors}
        \wrongchoice{wires}
        \wrongchoice{batteries}
        \wrongchoice{amperes}
    \end{choices}
    \end{multicols}
\end{question}
}

\element{cpo-mc}{
\begin{question}{cpo-ch13-q25}
    The ability of an object to resist current is called:
    \begin{choices}
        \wrongchoice{potential difference.}
        \wrongchoice{electrical inertia.}
        \wrongchoice{alternating current.}
      \correctchoice{electrical resistance.}
    \end{choices}
\end{question}
}

\element{cpo-mc}{
\begin{question}{cpo-ch13-q26}
    An ohm, \si{\ohm}, is the unit of measurement for:
    \begin{multicols}{2}
    \begin{choices}
        \wrongchoice{electrical power}
        \wrongchoice{voltage}
        \wrongchoice{current}
      \correctchoice{resistance}
    \end{choices}
    \end{multicols}
\end{question}
}

\element{cpo-mc}{
\begin{question}{cpo-ch13-q27}
    The mathematical relationship between current,
        voltage, and resistance is known as:
    \begin{multicols}{2}
    \begin{choices}
        \wrongchoice{Kirchoff's law}
        \wrongchoice{Faraday's law}
      \correctchoice{Ohm's law}
        \wrongchoice{Murphy's law}
    \end{choices}
    \end{multicols}
\end{question}
}

\element{cpo-mc}{
\begin{question}{cpo-ch13-q28}
    The ratio of voltage to current represents a quantity that can be expressed using units of:
    \begin{multicols}{2}
    \begin{choices}
        \wrongchoice{volts (\si{\volt})}
        \wrongchoice{watts (\si{\watt})}
        \wrongchoice{amperes (\si{\ampere})}
      \correctchoice{ohms (\si{\ohm})}
    \end{choices}
    \end{multicols}
\end{question}
}

\element{cpo-mc}{
\begin{question}{cpo-ch13-q29}
    In the circuit diagram below, \SI{3}{\ampere} of current passes through the light bulb.
    The resistance of the light bulb is \SI{3}{\ohm}.
    \begin{center}
    \ctikzset{bipoles/length=0.75cm}
    \begin{circuitikz}[scale=1.25]
        \draw (0,0) to [battery,v] (0,1) to [R,l=\SI{3}{\ohm}] (2,1) to (2,0) to [ammeter,i=$\SI{3}{\ampere}$] (0,0);
    \end{circuitikz}
    \end{center}
    What is the voltage of the battery?
    \begin{multicols}{2}
    \begin{choices}
        \wrongchoice{\SI{1}{\volt}}
        \wrongchoice{\SI{1}{\ohm}}
      \correctchoice{\SI{9}{\volt}}
        \wrongchoice{\SI{9}{\ampere}}
    \end{choices}
    \end{multicols}
\end{question}
}

\element{cpo-mc}{
\begin{question}{cpo-ch13-q30}
    A battery connected to a light bulb with resistance of \SI{5}{\ohm} causes a current of \SI{2}{\ampere} to flow through the bulb pictured in the diagram below:
    \begin{center}
    \ctikzset{bipoles/length=0.75cm}
    \begin{circuitikz}[scale=1.25]
        \draw (0,0) to [battery,v] (0,1) to [R,l=\SI{5}{\ohm}] (2,1) to (2,0) to [ammeter,i=$\SI{2}{\ampere}$] (0,0);
    \end{circuitikz}
    \end{center}
    The voltage across the light bulb is:
    \begin{multicols}{2}
    \begin{choices}
        \wrongchoice{\SI{0.4}{\volt}}
        \wrongchoice{\SI{1}{\volt}}
        \wrongchoice{\SI{2.5}{\volt}}
      \correctchoice{\SI{10}{\volt}}
    \end{choices}
    \end{multicols}
\end{question}
}

\element{cpo-mc}{
\begin{question}{cpo-ch13-q31}
    As Jing toasts her morning waffle in the family toaster,
        \SI{4.0}{\ampere} of current flows with a voltage of \SI{120}{\volt} across the toaster.
    The resistance of the toaster is:
    \begin{multicols}{2}
    \begin{choices}
        \wrongchoice{\SI{30}{\watt}}
        \wrongchoice{\SI{30}{\ohm}}
        \wrongchoice{\SI{480}{\watt}}
      \correctchoice{\SI{0.033}{\ohm}}
    \end{choices}
    \end{multicols}
\end{question}
}

\element{cpo-mc}{
\begin{question}{cpo-ch13-q32}
    A stereo receiver is plugged into a \SI{120}{\volt} outlet.
    If the receiver has a resistance of \SI{240}{\ohm}, 
        how much current does it use?
    \begin{multicols}{2}
    \begin{choices}
        \wrongchoice{\SI{2}{\ampere}}
        \wrongchoice{\SI{2}{\volt}}
        \wrongchoice{\SI{0.5}{\ampere}}
      \correctchoice{\SI{60}{\volt}}
    \end{choices}
    \end{multicols}
\end{question}
}

\element{cpo-mc}{
\begin{question}{cpo-ch13-q33}
    What happens to the current in a wire with fixed voltage if you increase the resistance of the circuit?
    \begin{choices}
        \wrongchoice{The current increases.}
      \correctchoice{The current decreases.}
        \wrongchoice{The current stays the same.}
        \wrongchoice{The voltage increases.}
    \end{choices}
\end{question}
}

\element{cpo-mc}{
\begin{question}{cpo-ch13-q34}
    This graph shows the current through a resistor as the voltage is changed:
    \begin{center}
    \begin{tikzpicture}
        \begin{axis}[
            axis y line=left,
            axis x line=bottom,
            axis line style={->},
            ylabel={current},
            y unit=\si{\ampere},
            ytick={0,1,2,3,4,5,6},
            xlabel={voltage},
            x unit=\si{\volt},
            xtick={0,1,2,3,4,5,6,7,8,9},
            xmin=0,xmax=9.1,
            ymin=0,ymax=6.1,
            grid=major,
            width=0.8\columnwidth,
            height=0.5\columnwidth,
            very thin,
        ]
        \addplot[line width=1pt,domain=0:9]{2*x/3};
        \end{axis}
    \end{tikzpicture}
    \end{center}
    The resistance of the bulb is:
    \begin{multicols}{2}
    \begin{choices}
        \wrongchoice{\SI{6}{\ohm}.}
        \wrongchoice{\SI{0.67}{\ohm}.}
        \wrongchoice{\SI{5}{\ohm}.}
      \correctchoice{\SI{1.5}{\ohm}.}
    \end{choices}
    \end{multicols}
\end{question}
}

\element{cpo-mc}{
\begin{question}{cpo-ch13-q35}
    A light bulb requires \SI{2}{\ampere} to produce light.
    The resistance of the bulb is \SI{3}{\ohm}.
    How many batteries do you need if each battery is \SI{1.5}{\volt}?
    \begin{multicols}{2}
    \begin{choices}
        \wrongchoice{6}
        \wrongchoice{4}
        \wrongchoice{1.5}
      \correctchoice{5}
    \end{choices}
    \end{multicols}
\end{question}
}

\element{cpo-mc}{
\begin{question}{cpo-ch13-q36}
    A \SI{120}{\volt} household circuit has a circuit breaker that opens the circuit if it draws more than \SI{12}{\ampere} of current.
    What is the minimum amount of resistance in the circuit required to keep the circuit breaker from activating?
    \begin{multicols}{2}
    \begin{choices}
        \wrongchoice{\SI{0.1}{\ohm}}
        \wrongchoice{\SI{132}{\ohm}}
        \wrongchoice{\SI{1 440}{\ohm}}
      \correctchoice{\SI{10}{\ohm}}
    \end{choices}
    \end{multicols}
\end{question}
}

\endinput


