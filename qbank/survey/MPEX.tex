
%% Maryland Physics Expectation (MPEX) Survey Questions
%%------------------------------------------------------------

\element{mpex}{
\begin{question}{mpex-Q01}
    \QuestionIndicative
    \scoring{auto=0,v=-1,e=-2}
    All I need to do to understand most of the basic ideas in this course
        is just read the text, work most of the problems, and/or pay close
        attention in class.
    \begin{choiceshoriz}[o]
        \correctchoice{0}
        \correctchoice{1}
        \correctchoice{2}
        \correctchoice{3}
        \correctchoice{4}
        \correctchoice{5}
    \end{choiceshoriz}
\end{question}
}

\element{mpex}{
\begin{question}{mpex-Q02}
    \QuestionIndicative
    \scoring{auto=0,v=-1,e=-2}
    All I learn from a derivation or proof of a formula is that the formula
        obtained is valid and that it is OK to use it in problem.
    \begin{choiceshoriz}[o]
        \correctchoice{0}
        \correctchoice{1}
        \correctchoice{2}
        \correctchoice{3}
        \correctchoice{4}
        \correctchoice{5}
    \end{choiceshoriz}
\end{question}
}

\element{mpex}{
\begin{question}{mpex-Q03}
    \QuestionIndicative
    \scoring{auto=0,v=-1,e=-2}
    I go over my class notes carefully to prepare for tests in this course.
    \begin{choiceshoriz}[o]
        \correctchoice{0}
        \correctchoice{1}
        \correctchoice{2}
        \correctchoice{3}
        \correctchoice{4}
        \correctchoice{5}
    \end{choiceshoriz}
\end{question}
}

\element{mpex}{
\begin{question}{mpex-Q04}
    \QuestionIndicative
    \scoring{auto=0,v=-1,e=-2}
    Problem solving in physics basically means matching problems with facts
        or equations and then substituting values to get a number.
    \begin{choiceshoriz}[o]
        \correctchoice{0}
        \correctchoice{1}
        \correctchoice{2}
        \correctchoice{3}
        \correctchoice{4}
        \correctchoice{5}
    \end{choiceshoriz}
\end{question}
}

\element{mpex}{
\begin{question}{mpex-Q05}
    \QuestionIndicative
    \scoring{auto=0,v=-1,e=-2}
    Learning physics made me change some of my ideas about how the physical
        world works.
    \begin{choiceshoriz}[o]
        \correctchoice{0}
        \correctchoice{1}
        \correctchoice{2}
        \correctchoice{3}
        \correctchoice{4}
        \correctchoice{5}
    \end{choiceshoriz}
\end{question}
}

\element{mpex}{
\begin{question}{mpex-Q06}
    \QuestionIndicative
    \scoring{auto=0,v=-1,e=-2}
    I spend a lot of time figuring out and understanding at least some of
        the derivations or proofs given either in class or in the text.
    \begin{choiceshoriz}[o]
        \correctchoice{0}
        \correctchoice{1}
        \correctchoice{2}
        \correctchoice{3}
        \correctchoice{4}
        \correctchoice{5}
    \end{choiceshoriz}
\end{question}
}

\element{mpex}{
\begin{question}{mpex-Q07}
    \QuestionIndicative
    \scoring{auto=0,v=-1,e=-2}
    I read the text in detail and work through many of the examples given there.
    \begin{choiceshoriz}[o]
        \correctchoice{0}
        \correctchoice{1}
        \correctchoice{2}
        \correctchoice{3}
        \correctchoice{4}
        \correctchoice{5}
    \end{choiceshoriz}
\end{question}
}

\element{mpex}{
\begin{question}{mpex-Q08}
    \QuestionIndicative
    \scoring{auto=0,v=-1,e=-2}
    In this course, I do not expect to understand equations in an intuitive sense;
        they just have to be taken as givens.
    \begin{choiceshoriz}[o]
        \correctchoice{0}
        \correctchoice{1}
        \correctchoice{2}
        \correctchoice{3}
        \correctchoice{4}
        \correctchoice{5}
    \end{choiceshoriz}
\end{question}
}

\element{mpex}{
\begin{question}{mpex-Q09}
    \QuestionIndicative
    \scoring{auto=0,v=-1,e=-2}
    The best way for me to learn physics is by solving many problems rather
        than by carefully analyzing a few in detail.
    \begin{choiceshoriz}[o]
        \correctchoice{0}
        \correctchoice{1}
        \correctchoice{2}
        \correctchoice{3}
        \correctchoice{4}
        \correctchoice{5}
    \end{choiceshoriz}
\end{question}
}

\element{mpex}{
\begin{question}{mpex-Q10}
    \QuestionIndicative
    \scoring{auto=0,v=-1,e=-2}
    Physical laws have little relation to what I experience in the real world.
    \begin{choiceshoriz}[o]
        \correctchoice{0}
        \correctchoice{1}
        \correctchoice{2}
        \correctchoice{3}
        \correctchoice{4}
        \correctchoice{5}
    \end{choiceshoriz}
\end{question}
}

\element{mpex}{
\begin{question}{mpex-Q11}
    \QuestionIndicative
    \scoring{auto=0,v=-1,e=-2}
    A good understanding of physics is necessary for me to achieve my career goals.
    A good grade in this course is not enough.
    \begin{choiceshoriz}[o]
        \correctchoice{0}
        \correctchoice{1}
        \correctchoice{2}
        \correctchoice{3}
        \correctchoice{4}
        \correctchoice{5}
    \end{choiceshoriz}
\end{question}
}

\element{mpex}{
\begin{question}{mpex-Q12}
    \QuestionIndicative
    \scoring{auto=0,v=-1,e=-2}
    Knowledge in physics consists of many pieces of information each of which
        applies primarily to a specific situation.
    \begin{choiceshoriz}[o]
        \correctchoice{0}
        \correctchoice{1}
        \correctchoice{2}
        \correctchoice{3}
        \correctchoice{4}
        \correctchoice{5}
    \end{choiceshoriz}
\end{question}
}

\element{mpex}{
\begin{question}{mpex-Q13}
    \QuestionIndicative
    \scoring{auto=0,v=-1,e=-2}
    My grade in this course is primarily determined by how familiar I am with
        the material.
    Insight or creativity has little to do with it.
    \begin{choiceshoriz}[o]
        \correctchoice{0}
        \correctchoice{1}
        \correctchoice{2}
        \correctchoice{3}
        \correctchoice{4}
        \correctchoice{5}
    \end{choiceshoriz}
\end{question}
}

\element{mpex}{
\begin{question}{mpex-Q14}
    \QuestionIndicative
    \scoring{auto=0,v=-1,e=-2}
    Learning physics is a matter of acquiring knowledge that is specifically
        located in the laws, principles, and equations given in class
        and/or in the textbook.
    \begin{choiceshoriz}[o]
        \correctchoice{0}
        \correctchoice{1}
        \correctchoice{2}
        \correctchoice{3}
        \correctchoice{4}
        \correctchoice{5}
    \end{choiceshoriz}
\end{question}
}

\element{mpex}{
\begin{question}{mpex-Q15}
    \QuestionIndicative
    \scoring{auto=0,v=-1,e=-2}
    In doing a physics problem, if my calculation gives a result that differs
        significantly from what I expect, I would have to trust the calculation.
    \begin{choiceshoriz}[o]
        \correctchoice{0}
        \correctchoice{1}
        \correctchoice{2}
        \correctchoice{3}
        \correctchoice{4}
        \correctchoice{5}
    \end{choiceshoriz}
\end{question}
}

\element{mpex}{
\begin{question}{mpex-Q16}
    \QuestionIndicative
    \scoring{auto=0,v=-1,e=-2}
    The derivations or proofs of equations in class or in the text have little
        to do with problem solving or with the skills I need to succeed in
        this course.
    \begin{choiceshoriz}[o]
        \correctchoice{0}
        \correctchoice{1}
        \correctchoice{2}
        \correctchoice{3}
        \correctchoice{4}
        \correctchoice{5}
    \end{choiceshoriz}
\end{question}
}

\element{mpex}{
\begin{question}{mpex-Q17}
    \QuestionIndicative
    \scoring{auto=0,v=-1,e=-2}
    Only very few specially qualified people are capable of really
        understanding physics.
    \begin{choiceshoriz}[o]
        \correctchoice{0}
        \correctchoice{1}
        \correctchoice{2}
        \correctchoice{3}
        \correctchoice{4}
        \correctchoice{5}
    \end{choiceshoriz}
\end{question}
}

\element{mpex}{
\begin{question}{mpex-Q18}
    \QuestionIndicative
    \scoring{auto=0,v=-1,e=-2}
    To understand physics, I sometimes think about my personal experiences
        and relate them to the topic being analyzed.
    \begin{choiceshoriz}[o]
        \correctchoice{0}
        \correctchoice{1}
        \correctchoice{2}
        \correctchoice{3}
        \correctchoice{4}
        \correctchoice{5}
    \end{choiceshoriz}
\end{question}
}

\element{mpex}{
\begin{question}{mpex-Q19}
    \QuestionIndicative
    \scoring{auto=0,v=-1,e=-2}
    The most crucial thing in solving a physics problem is finding the
        right equation.
    \begin{choiceshoriz}[o]
        \correctchoice{0}
        \correctchoice{1}
        \correctchoice{2}
        \correctchoice{3}
        \correctchoice{4}
        \correctchoice{5}
    \end{choiceshoriz}
\end{question}
}

\element{mpex}{
\begin{question}{mpex-Q20}
    \QuestionIndicative
    \scoring{auto=0,v=-1,e=-2}
    If I do not remember a particular equation needed for a problem in
        an exam there is nothing much I can do (legally!) to come up with it.
    \begin{choiceshoriz}[o]
        \correctchoice{0}
        \correctchoice{1}
        \correctchoice{2}
        \correctchoice{3}
        \correctchoice{4}
        \correctchoice{5}
    \end{choiceshoriz}
\end{question}
}

\element{mpex}{
\begin{question}{mpex-Q21}
    \QuestionIndicative
    \scoring{auto=0,v=-1,e=-2}
    If I came up with two different approaches to a problem and they gave
        different answers, I would not worry about it; I would just choose
        the answer that seemed most reasonable. 
    (Assume the answer is not in the back of the book.)
    \begin{choiceshoriz}[o]
        \correctchoice{0}
        \correctchoice{1}
        \correctchoice{2}
        \correctchoice{3}
        \correctchoice{4}
        \correctchoice{5}
    \end{choiceshoriz}
\end{question}
}

\element{mpex}{
\begin{question}{mpex-Q22}
    \QuestionIndicative
    \scoring{auto=0,v=-1,e=-2}
    Physics is related to the real world and it sometimes helps to think
        about the connections, but it is rarely essential for what I have
        to do in this course.
    \begin{choiceshoriz}[o]
        \correctchoice{0}
        \correctchoice{1}
        \correctchoice{2}
        \correctchoice{3}
        \correctchoice{4}
        \correctchoice{5}
    \end{choiceshoriz}
\end{question}
}

\element{mpex}{
\begin{question}{mpex-Q23}
    \QuestionIndicative
    \scoring{auto=0,v=-1,e=-2}
    The main skill I get out of this course is learning how to solve
        physics problems.
    \begin{choiceshoriz}[o]
        \correctchoice{0}
        \correctchoice{1}
        \correctchoice{2}
        \correctchoice{3}
        \correctchoice{4}
        \correctchoice{5}
    \end{choiceshoriz}
\end{question}
}

\element{mpex}{
\begin{question}{mpex-Q24}
    \QuestionIndicative
    \scoring{auto=0,v=-1,e=-2}
    The results of an exam don't give me any useful guidance to
        improve my understanding of the course material.
    All the learning associated with an exam is in the studying
        I do before it takes place.
    \begin{choiceshoriz}[o]
        \correctchoice{0}
        \correctchoice{1}
        \correctchoice{2}
        \correctchoice{3}
        \correctchoice{4}
        \correctchoice{5}
    \end{choiceshoriz}
\end{question}
}

\element{mpex}{
\begin{question}{mpex-Q25}
    \QuestionIndicative
    \scoring{auto=0,v=-1,e=-2}
    Learning physics helps me understand situations in my everyday life.
    \begin{choiceshoriz}[o]
        \correctchoice{0}
        \correctchoice{1}
        \correctchoice{2}
        \correctchoice{3}
        \correctchoice{4}
        \correctchoice{5}
    \end{choiceshoriz}
\end{question}
}

\element{mpex}{
\begin{question}{mpex-Q26}
    \QuestionIndicative
    \scoring{auto=0,v=-1,e=-2}
    When I solve most exam or homework problems, I explicitly think
        about the concepts that underlie the problem.
    \begin{choiceshoriz}[o]
        \correctchoice{0}
        \correctchoice{1}
        \correctchoice{2}
        \correctchoice{3}
        \correctchoice{4}
        \correctchoice{5}
    \end{choiceshoriz}
\end{question}
}

\element{mpex}{
\begin{question}{mpex-Q27}
    \QuestionIndicative
    \scoring{auto=0,v=-1,e=-2}
    Understanding physics basically means being able to recall
        something you have read or been shown.
    \begin{choiceshoriz}[o]
        \correctchoice{0}
        \correctchoice{1}
        \correctchoice{2}
        \correctchoice{3}
        \correctchoice{4}
        \correctchoice{5}
    \end{choiceshoriz}
\end{question}
}

\element{mpex}{
\begin{question}{mpex-Q28}
    \QuestionIndicative
    \scoring{auto=0,v=-1,e=-2}
    Spending a lot of time (half an hour or more) working on a
        problem is a waste of time.
    If I do not make progress quickly, I would be better off asking
        someone who knows more than I do.
    \begin{choiceshoriz}[o]
        \correctchoice{0}
        \correctchoice{1}
        \correctchoice{2}
        \correctchoice{3}
        \correctchoice{4}
        \correctchoice{5}
    \end{choiceshoriz}
\end{question}
}

\element{mpex}{
\begin{question}{mpex-Q29}
    \QuestionIndicative
    \scoring{auto=0,v=-1,e=-2}
    A significant problem in this course is being able to
        memorize all the information I need to know.
    \begin{choiceshoriz}[o]
        \correctchoice{0}
        \correctchoice{1}
        \correctchoice{2}
        \correctchoice{3}
        \correctchoice{4}
        \correctchoice{5}
    \end{choiceshoriz}
\end{question}
}

\element{mpex}{
\begin{question}{mpex-Q30}
    \QuestionIndicative
    \scoring{auto=0,v=-1,e=-2}
    The main skill I get out of this course is to learn how to reason
        logically about the physical world.
    \begin{choiceshoriz}[o]
        \correctchoice{0}
        \correctchoice{1}
        \correctchoice{2}
        \correctchoice{3}
        \correctchoice{4}
        \correctchoice{5}
    \end{choiceshoriz}
\end{question}
}

\element{mpex}{
\begin{question}{mpex-Q31}
    \QuestionIndicative
    \scoring{auto=0,v=-1,e=-2}
    I use the mistakes I make on homework and on exam problems as clues
        to what I need to do understand the material better.
    \begin{choiceshoriz}[o]
        \correctchoice{0}
        \correctchoice{1}
        \correctchoice{2}
        \correctchoice{3}
        \correctchoice{4}
        \correctchoice{5}
    \end{choiceshoriz}
\end{question}
}

\element{mpex}{
\begin{question}{mpex-Q32}
    \QuestionIndicative
    \scoring{auto=0,v=-1,e=-2}
    To be able to use an equation in a problem (particularly in a problem
        that I have not seen before), I need to know more than what each
        term in the equation represents.
    \begin{choiceshoriz}[o]
        \correctchoice{0}
        \correctchoice{1}
        \correctchoice{2}
        \correctchoice{3}
        \correctchoice{4}
        \correctchoice{5}
    \end{choiceshoriz}
\end{question}
}

\element{mpex}{
\begin{question}{mpex-Q33}
    \QuestionIndicative
    \scoring{auto=0,v=-1,e=-2}
    It is possible to pass this course (get a ``C'' or better) without
        understanding physics very well.
    \begin{choiceshoriz}[o]
        \correctchoice{0}
        \correctchoice{1}
        \correctchoice{2}
        \correctchoice{3}
        \correctchoice{4}
        \correctchoice{5}
    \end{choiceshoriz}
\end{question}
}

\element{mpex}{
\begin{question}{mpex-Q34}
    \QuestionIndicative
    \scoring{auto=0,v=-1,e=-2}
    Learning physics requires that I substantially rethink, restructure,
        and reorganize the information that I am given in class and/or
        in the text.
    \begin{choiceshoriz}[o]
        \correctchoice{0}
        \correctchoice{1}
        \correctchoice{2}
        \correctchoice{3}
        \correctchoice{4}
        \correctchoice{5}
    \end{choiceshoriz}
\end{question}
}

