
%% Open
%What do you like best about XYZ School?
%What do you like least about XYZ School?
%As of today, do you plan for your child to graduate from XYZ School? Please provide the reason(s) for your response:
%Bsed on your experience, how could the leadership improve XYZ School?


%% My Personal Parent Survey
%%------------------------------------------------------------

\begin{question}{Parent01}
    \QuestionIndicative
    \scoring{auto=0,v=-1,e=-2}
    I would recommend Mergenthaler Vocational-Technical High School
        to friends in my community.
    \begin{choiceshoriz}[o]
        \correctchoice{0}
        \correctchoice{1}
        \correctchoice{2}
        \correctchoice{3}
        \correctchoice{4}
        \correctchoice{5}
    \end{choiceshoriz}
\end{question}

\begin{question}{Parent02}
    \QuestionIndicative
    \scoring{auto=0,v=-1,e=-2}
    %Please indicate your overall level of satisfaction with your child(ren)’s educational experience at XYZ School:
    I am satisfied with my child's overall educational experience at
        Mergenthaler Vocational-Technical High School.
    \begin{choiceshoriz}[o]
        \correctchoice{0}
        \correctchoice{1}
        \correctchoice{2}
        \correctchoice{3}
        \correctchoice{4}
        \correctchoice{5}
    \end{choiceshoriz}
\end{question}

\begin{question}{Parent03}
    \QuestionIndicative
    \scoring{auto=0,v=-1,e=-2}
    I expect my child to graduate from 
        Mergenthaler Vocational-Technical High School.
    \begin{choiceshoriz}[o]
        \correctchoice{0}
        \correctchoice{1}
        \correctchoice{2}
        \correctchoice{3}
        \correctchoice{4}
        \correctchoice{5}
    \end{choiceshoriz}
\end{question}

\begin{question}{Parent04}
    \QuestionIndicative
    \scoring{auto=0,v=-1,e=-2}
    I am satisified with the Parent Teacher Conferences at
        Mergenthaler Vocational-Technical High School.
    \begin{choiceshoriz}[o]
        \correctchoice{0}
        \correctchoice{1}
        \correctchoice{2}
        \correctchoice{3}
        \correctchoice{4}
        \correctchoice{5}
    \end{choiceshoriz}
\end{question}

\begin{question}{Parent05}
    \QuestionIndicative
    \scoring{auto=0,v=-1,e=-2}
    I feel that my voice matters in decisions at Mergenthaler Vocational-Technical High School.
    \begin{choiceshoriz}[o]
        \correctchoice{0}
        \correctchoice{1}
        \correctchoice{2}
        \correctchoice{3}
        \correctchoice{4}
        \correctchoice{5}
    \end{choiceshoriz}
\end{question}

\begin{question}{Parent06}
    \QuestionIndicative
    \scoring{auto=0,v=-1,e=-2}
    I expect my child to be given physics homework every day.
    \begin{choiceshoriz}[o]
        \correctchoice{0}
        \correctchoice{1}
        \correctchoice{2}
        \correctchoice{3}
        \correctchoice{4}
        \correctchoice{5}
    \end{choiceshoriz}
\end{question}

\begin{question}{Parent07}
    \QuestionIndicative
    \scoring{auto=0,v=-1,e=-2}
    I expect my child to spend at least 30 minutes a day on physics homework.
    \begin{choiceshoriz}[o]
        \correctchoice{0}
        \correctchoice{1}
        \correctchoice{2}
        \correctchoice{3}
        \correctchoice{4}
        \correctchoice{5}
    \end{choiceshoriz}
\end{question}

\begin{question}{Parent08}
    \QuestionIndicative
    \scoring{auto=0,v=-1,e=-2}
    Mergenthaler Vocational-Technical High School meets my expectations of
        a good school.
    \begin{choiceshoriz}[o]
        \correctchoice{0}
        \correctchoice{1}
        \correctchoice{2}
        \correctchoice{3}
        \correctchoice{4}
        \correctchoice{5}
    \end{choiceshoriz}
\end{question}

\begin{question}{Parent09}
    \QuestionIndicative
    \scoring{auto=0,v=-1,e=-2}
    I think there are suficient opporutnities for parental involvmeent at
        Mergenthaler Vocational-Technical High School.
    \begin{choiceshoriz}[o]
        \correctchoice{0}
        \correctchoice{1}
        \correctchoice{2}
        \correctchoice{3}
        \correctchoice{4}
        \correctchoice{5}
    \end{choiceshoriz}
\end{question}

\begin{question}{Parent10}
    \QuestionIndicative
    \scoring{auto=0,v=-1,e=-2}
    I am satisified with the class sizes at 
        Mergenthaler Vocational-Technical High School.
    \begin{choiceshoriz}[o]
        \correctchoice{0}
        \correctchoice{1}
        \correctchoice{2}
        \correctchoice{3}
        \correctchoice{4}
        \correctchoice{5}
    \end{choiceshoriz}
\end{question}

\begin{question}{Parent11}
    \QuestionIndicative
    \scoring{auto=0,v=-1,e=-2}
    I think the homework that my child receives from
        Mergenthaler Vocational-Technical High School
        is relevant for my child to be successful
    \begin{choiceshoriz}[o]
        \correctchoice{0}
        \correctchoice{1}
        \correctchoice{2}
        \correctchoice{3}
        \correctchoice{4}
        \correctchoice{5}
    \end{choiceshoriz}
\end{question}

\begin{question}{Parent12}
    \QuestionIndicative
    \scoring{auto=0,v=-1,e=-2}
    Knowledge in physics consists of many pieces of information each of which
        applies primarily to a specific situation.
    \begin{choiceshoriz}[o]
        \correctchoice{0}
        \correctchoice{1}
        \correctchoice{2}
        \correctchoice{3}
        \correctchoice{4}
        \correctchoice{5}
    \end{choiceshoriz}
\end{question}

\begin{question}{Parent13}
    \QuestionIndicative
    \scoring{auto=0,v=-1,e=-2}
    My grade in this course is primarily determined by how familiar I am with
        the material.
    Insight or creativity has little to do with it.
    \begin{choiceshoriz}[o]
        \correctchoice{0}
        \correctchoice{1}
        \correctchoice{2}
        \correctchoice{3}
        \correctchoice{4}
        \correctchoice{5}
    \end{choiceshoriz}
\end{question}

\begin{question}{Parent14}
    \QuestionIndicative
    \scoring{auto=0,v=-1,e=-2}
    Learning physics is a matter of acquiring knowledge that is specifically
        located in the laws, principles, and equations given in class
        and/or in the textbook.
    \begin{choiceshoriz}[o]
        \correctchoice{0}
        \correctchoice{1}
        \correctchoice{2}
        \correctchoice{3}
        \correctchoice{4}
        \correctchoice{5}
    \end{choiceshoriz}
\end{question}

\begin{question}{Parent15}
    \QuestionIndicative
    \scoring{auto=0,v=-1,e=-2}
    In doing a physics problem, if my calculation gives a result that differs
        significantly from what I expect, I would have to trust the calculation.
    \begin{choiceshoriz}[o]
        \correctchoice{0}
        \correctchoice{1}
        \correctchoice{2}
        \correctchoice{3}
        \correctchoice{4}
        \correctchoice{5}
    \end{choiceshoriz}
\end{question}

\begin{question}{Parent16}
    \QuestionIndicative
    \scoring{auto=0,v=-1,e=-2}
    The derivations or proofs of equations in class or in the text have little
        to do with problem solving or with the skills I need to succeed in
        this course.
    \begin{choiceshoriz}[o]
        \correctchoice{0}
        \correctchoice{1}
        \correctchoice{2}
        \correctchoice{3}
        \correctchoice{4}
        \correctchoice{5}
    \end{choiceshoriz}
\end{question}

\begin{question}{Parent17}
    \QuestionIndicative
    \scoring{auto=0,v=-1,e=-2}
    Only very few specially qualified people are capable of really
        understanding physics.
    \begin{choiceshoriz}[o]
        \correctchoice{0}
        \correctchoice{1}
        \correctchoice{2}
        \correctchoice{3}
        \correctchoice{4}
        \correctchoice{5}
    \end{choiceshoriz}
\end{question}

\begin{question}{Parent18}
    \QuestionIndicative
    \scoring{auto=0,v=-1,e=-2}
    To understand physics, I sometimes think about my personal experiences
        and relate them to the topic being analyzed.
    \begin{choiceshoriz}[o]
        \correctchoice{0}
        \correctchoice{1}
        \correctchoice{2}
        \correctchoice{3}
        \correctchoice{4}
        \correctchoice{5}
    \end{choiceshoriz}
\end{question}

\begin{question}{Parent19}
    \QuestionIndicative
    \scoring{auto=0,v=-1,e=-2}
    The most crucial thing in solving a physics problem is finding the
        right equation.
    \begin{choiceshoriz}[o]
        \correctchoice{0}
        \correctchoice{1}
        \correctchoice{2}
        \correctchoice{3}
        \correctchoice{4}
        \correctchoice{5}
    \end{choiceshoriz}
\end{question}

\begin{question}{Parent20}
    \QuestionIndicative
    \scoring{auto=0,v=-1,e=-2}
    If I do not remember a particular equation needed for a problem in
        an exam there is nothing much I can do (legally!) to come up with it.
    \begin{choiceshoriz}[o]
        \correctchoice{0}
        \correctchoice{1}
        \correctchoice{2}
        \correctchoice{3}
        \correctchoice{4}
        \correctchoice{5}
    \end{choiceshoriz}
\end{question}

\begin{question}{Parent21}
    \QuestionIndicative
    \scoring{auto=0,v=-1,e=-2}
    If I came up with two different approaches to a problem and they gave
        different answers, I would not worry about it; I would just choose
        the answer that seemed most reasonable. 
    (Assume the answer is not in the back of the book.)
    \begin{choiceshoriz}[o]
        \correctchoice{0}
        \correctchoice{1}
        \correctchoice{2}
        \correctchoice{3}
        \correctchoice{4}
        \correctchoice{5}
    \end{choiceshoriz}
\end{question}

\begin{question}{Parent22}
    \QuestionIndicative
    \scoring{auto=0,v=-1,e=-2}
    Physics is related to the real world and it sometimes helps to think
        about the connections, but it is rarely essential for what I have
        to do in this course.
    \begin{choiceshoriz}[o]
        \correctchoice{0}
        \correctchoice{1}
        \correctchoice{2}
        \correctchoice{3}
        \correctchoice{4}
        \correctchoice{5}
    \end{choiceshoriz}
\end{question}

\begin{question}{Parent23}
    \QuestionIndicative
    \scoring{auto=0,v=-1,e=-2}
    The main skill I get out of this course is learning how to solve
        physics problems.
    \begin{choiceshoriz}[o]
        \correctchoice{0}
        \correctchoice{1}
        \correctchoice{2}
        \correctchoice{3}
        \correctchoice{4}
        \correctchoice{5}
    \end{choiceshoriz}
\end{question}

\begin{question}{Parent24}
    \QuestionIndicative
    \scoring{auto=0,v=-1,e=-2}
    The results of an exam don't give me any useful guidance to
        improve my understanding of the course material.
    All the learning associated with an exam is in the studying
        I do before it takes place.
    \begin{choiceshoriz}[o]
        \correctchoice{0}
        \correctchoice{1}
        \correctchoice{2}
        \correctchoice{3}
        \correctchoice{4}
        \correctchoice{5}
    \end{choiceshoriz}
\end{question}

\begin{question}{Parent25}
    \QuestionIndicative
    \scoring{auto=0,v=-1,e=-2}
    Learning physics helps me understand situations in my everyday life.
    \begin{choiceshoriz}[o]
        \correctchoice{0}
        \correctchoice{1}
        \correctchoice{2}
        \correctchoice{3}
        \correctchoice{4}
        \correctchoice{5}
    \end{choiceshoriz}
\end{question}

\begin{question}{Parent26}
    \QuestionIndicative
    \scoring{auto=0,v=-1,e=-2}
    When I solve most exam or homework problems, I explicitly think
        about the concepts that underlie the problem.
    \begin{choiceshoriz}[o]
        \correctchoice{0}
        \correctchoice{1}
        \correctchoice{2}
        \correctchoice{3}
        \correctchoice{4}
        \correctchoice{5}
    \end{choiceshoriz}
\end{question}

\begin{question}{Parent27}
    \QuestionIndicative
    \scoring{auto=0,v=-1,e=-2}
    Understanding physics basically means being able to recall
        something you have read or been shown.
    \begin{choiceshoriz}[o]
        \correctchoice{0}
        \correctchoice{1}
        \correctchoice{2}
        \correctchoice{3}
        \correctchoice{4}
        \correctchoice{5}
    \end{choiceshoriz}
\end{question}

\begin{question}{Parent28}
    \QuestionIndicative
    \scoring{auto=0,v=-1,e=-2}
    Spending a lot of time (half an hour or more) working on a
        problem is a waste of time.
    If I do not make progress quickly, I would be better off asking
        someone who knows more than I do.
    \begin{choiceshoriz}[o]
        \correctchoice{0}
        \correctchoice{1}
        \correctchoice{2}
        \correctchoice{3}
        \correctchoice{4}
        \correctchoice{5}
    \end{choiceshoriz}
\end{question}

\begin{question}{Parent29}
    \QuestionIndicative
    \scoring{auto=0,v=-1,e=-2}
    A significant problem in this course is being able to
        memorize all the information I need to know.
    \begin{choiceshoriz}[o]
        \correctchoice{0}
        \correctchoice{1}
        \correctchoice{2}
        \correctchoice{3}
        \correctchoice{4}
        \correctchoice{5}
    \end{choiceshoriz}
\end{question}

\begin{question}{Parent30}
    \QuestionIndicative
    \scoring{auto=0,v=-1,e=-2}
    The main skill I get out of this course is to learn how to reason
        logically about the physical world.
    \begin{choiceshoriz}[o]
        \correctchoice{0}
        \correctchoice{1}
        \correctchoice{2}
        \correctchoice{3}
        \correctchoice{4}
        \correctchoice{5}
    \end{choiceshoriz}
\end{question}

\begin{question}{Parent31}
    \QuestionIndicative
    \scoring{auto=0,v=-1,e=-2}
    I use the mistakes I make on homework and on exam problems as clues
        to what I need to do understand the material better.
    \begin{choiceshoriz}[o]
        \correctchoice{0}
        \correctchoice{1}
        \correctchoice{2}
        \correctchoice{3}
        \correctchoice{4}
        \correctchoice{5}
    \end{choiceshoriz}
\end{question}

\begin{question}{Parent32}
    \QuestionIndicative
    \scoring{auto=0,v=-1,e=-2}
    To be able to use an equation in a problem (particularly in a problem
        that I have not seen before), I need to know more than what each
        term in the equation represents.
    \begin{choiceshoriz}[o]
        \correctchoice{0}
        \correctchoice{1}
        \correctchoice{2}
        \correctchoice{3}
        \correctchoice{4}
        \correctchoice{5}
    \end{choiceshoriz}
\end{question}

\begin{question}{Parent33}
    \QuestionIndicative
    \scoring{auto=0,v=-1,e=-2}
    It is possible to pass this course (get a ``C'' or better) without
        understanding physics very well.
    \begin{choiceshoriz}[o]
        \correctchoice{0}
        \correctchoice{1}
        \correctchoice{2}
        \correctchoice{3}
        \correctchoice{4}
        \correctchoice{5}
    \end{choiceshoriz}
\end{question}

\begin{question}{Parent34}
    \QuestionIndicative
    \scoring{auto=0,v=-1,e=-2}
    Learning physics requires that I substantially rethink, restructure,
        and reorganize the information that I am given in class and/or
        in the text.
    \begin{choiceshoriz}[o]
        \correctchoice{0}
        \correctchoice{1}
        \correctchoice{2}
        \correctchoice{3}
        \correctchoice{4}
        \correctchoice{5}
    \end{choiceshoriz}
\end{question}

