
%% Q: Colorodo Learning Attitudes about Science Survey
%%------------------------------------------------------------

%Introduction:
%Here are a number of statements that may or may not describe
%    your beliefs about learning physics.
%You are asked to rate each statement by selecting a number
%    between 1 and 5 where the numbers mean the following:
%\begin{enumerate*}
%    \item Strongly Disagree
%    \item Disagree
%    \item Neutral
%    \item Agree
%    \item Strongly Agree
%\end{enumerate*}
%Choose one of the above five choices that best expresses
%    your feeling about the statement.
%If you don't understand a statement, leave it blank.
%If you have no strong opinion, choose 3.

\element{CLASS}{
\begin{question}{class-Q01}
    \QuestionIndicative
    \scoring{auto=1,v=-1,e=-2}
    A significant problem in learning physics is being able to memorize all
        the information I need to know.
    \begin{multicols}{5}
    \begin{choices}[o]
        \correctchoice{}
        \correctchoice{}
        \correctchoice{}
        \correctchoice{}
        \correctchoice{}
    \end{choices}
    \end{multicols}
\end{question}
}

\element{CLASS}{
\begin{question}{class-Q02}
    \QuestionIndicative
    \scoring{auto=1,v=-1,e=-2}
    When I am solving a physics problem, I try to decide what would be
        a reasonable value for the answer.
    \begin{multicols}{5}
    \begin{choices}[o]
        \correctchoice{}
        \correctchoice{}
        \correctchoice{}
        \correctchoice{}
        \correctchoice{}
    \end{choices}
    \end{multicols}
\end{question}
}

\element{CLASS}{
\begin{question}{class-Q03}
    \QuestionIndicative
    \scoring{auto=1,v=-1,e=-2}
    I think about the physics I experience in everyday life.
    \begin{multicols}{5}
    \begin{choices}[o]
        \correctchoice{}
        \correctchoice{}
        \correctchoice{}
        \correctchoice{}
        \correctchoice{}
    \end{choices}
    \end{multicols}
\end{question}
}

\element{CLASS}{
\begin{question}{class-Q04}
    \QuestionIndicative
    \scoring{auto=1,v=-1,e=-2}
    It is useful for me to do lots and lots of problems when learning physics.
    \begin{multicols}{5}
    \begin{choices}[o]
        \correctchoice{}
        \correctchoice{}
        \correctchoice{}
        \correctchoice{}
        \correctchoice{}
    \end{choices}
    \end{multicols}
\end{question}
}

\element{CLASS}{
\begin{question}{class-Q05}
    \QuestionIndicative
    \scoring{auto=1,v=-1,e=-2}
    After I study a topic in physics and feel that I understand it, I have
        difficulty solving problems on the same topic.
    \begin{multicols}{5}
    \begin{choices}[o]
        \correctchoice{}
        \correctchoice{}
        \correctchoice{}
        \correctchoice{}
        \correctchoice{}
    \end{choices}
    \end{multicols}
\end{question}
}

\element{CLASS}{
\begin{question}{class-Q06}
    \QuestionIndicative
    \scoring{auto=1,v=-1,e=-2}
    Knowledge in physics consists of many disconnected topics.
    \begin{multicols}{5}
    \begin{choices}[o]
        \correctchoice{}
        \correctchoice{}
        \correctchoice{}
        \correctchoice{}
        \correctchoice{}
    \end{choices}
    \end{multicols}
\end{question}
}

\element{CLASS}{
\begin{question}{class-Q07}
    \QuestionIndicative
    \scoring{auto=1,v=-1,e=-2}
    As physicists learn more, most physics ideas we use today are likely
        to be proven wrong.
    \begin{multicols}{5}
    \begin{choices}[o]
        \correctchoice{}
        \correctchoice{}
        \correctchoice{}
        \correctchoice{}
        \correctchoice{}
    \end{choices}
    \end{multicols}
\end{question}
}

\element{CLASS}{
\begin{question}{class-Q08}
    \QuestionIndicative
    \scoring{auto=1,v=-1,e=-2}
    When I solve a physics problem, I locate an equation that uses the
        variables given in the problem and plug in the values.
    \begin{multicols}{5}
    \begin{choices}[o]
        \correctchoice{}
        \correctchoice{}
        \correctchoice{}
        \correctchoice{}
        \correctchoice{}
    \end{choices}
    \end{multicols}
\end{question}
}

\element{CLASS}{
\begin{question}{class-Q09}
    \QuestionIndicative
    \scoring{auto=1,v=-1,e=-2}
    I find that reading the text in detail is a good way for me to learn physics.
    \begin{multicols}{5}
    \begin{choices}[o]
        \correctchoice{}
        \correctchoice{}
        \correctchoice{}
        \correctchoice{}
        \correctchoice{}
    \end{choices}
    \end{multicols}
\end{question}
}

\element{CLASS}{
\begin{question}{class-Q10}
    \QuestionIndicative
    \scoring{auto=1,v=-1,e=-2}
    There is usually only one correct approach to solving a physics problem.
    \begin{multicols}{5}
    \begin{choices}[o]
        \correctchoice{}
        \correctchoice{}
        \correctchoice{}
        \correctchoice{}
        \correctchoice{}
    \end{choices}
    \end{multicols}
\end{question}
}

\element{CLASS}{
\begin{question}{class-Q11}
    \QuestionIndicative
    \scoring{auto=1,v=-1,e=-2}
    I am not satisfied until I understand why something works the way it does.
    \begin{multicols}{5}
    \begin{choices}[o]
        \correctchoice{}
        \correctchoice{}
        \correctchoice{}
        \correctchoice{}
        \correctchoice{}
    \end{choices}
    \end{multicols}
\end{question}
}

\element{CLASS}{
\begin{question}{class-Q12}
    \QuestionIndicative
    \scoring{auto=1,v=-1,e=-2}
    I cannot learn physics if the teacher does not explain things well in class.
    \begin{multicols}{5}
    \begin{choices}[o]
        \correctchoice{}
        \correctchoice{}
        \correctchoice{}
        \correctchoice{}
        \correctchoice{}
    \end{choices}
    \end{multicols}
\end{question}
}

\element{CLASS}{
\begin{question}{class-Q13}
    \QuestionIndicative
    \scoring{auto=1,v=-1,e=-2}
    I do not expect physics equations to help my understanding of the ideas,
        they are just for doing calculations.
    \begin{multicols}{5}
    \begin{choices}[o]
        \correctchoice{}
        \correctchoice{}
        \correctchoice{}
        \correctchoice{}
        \correctchoice{}
    \end{choices}
    \end{multicols}
\end{question}
}

\element{CLASS}{
\begin{question}{class-Q14}
    \QuestionIndicative
    \scoring{auto=1,v=-1,e=-2}
    I study physics to learn knowledge that will be useful in my life
        outside of school.
    \begin{multicols}{5}
    \begin{choices}[o]
        \correctchoice{}
        \correctchoice{}
        \correctchoice{}
        \correctchoice{}
        \correctchoice{}
    \end{choices}
    \end{multicols}
\end{question}
}

\element{CLASS}{
\begin{question}{class-Q15}
    \QuestionIndicative
    \scoring{auto=1,v=-1,e=-2}
    If I get stuck on a physics problem my first try, I usually try to figure
        out a different way that works.
    \begin{multicols}{5}
    \begin{choices}[o]
        \correctchoice{}
        \correctchoice{}
        \correctchoice{}
        \correctchoice{}
        \correctchoice{}
    \end{choices}
    \end{multicols}
\end{question}
}

\element{CLASS}{
\begin{question}{class-Q16}
    \QuestionIndicative
    \scoring{auto=1,v=-1,e=-2}
    Nearly everyone is capable of understanding physics if they work at it.
    \begin{multicols}{5}
    \begin{choices}[o]
        \correctchoice{}
        \correctchoice{}
        \correctchoice{}
        \correctchoice{}
        \correctchoice{}
    \end{choices}
    \end{multicols}
\end{question}
}

\element{CLASS}{
\begin{question}{class-Q17}
    \QuestionIndicative
    \scoring{auto=1,v=-1,e=-2}
    Understanding physics basically means being able to recall something
        you've read or been shown.
    \begin{multicols}{5}
    \begin{choices}[o]
        \correctchoice{}
        \correctchoice{}
        \correctchoice{}
        \correctchoice{}
        \correctchoice{}
    \end{choices}
    \end{multicols}
\end{question}
}

\element{CLASS}{
\begin{question}{class-Q18}
    \QuestionIndicative
    \scoring{auto=1,v=-1,e=-2}
    There could be two different correct values to a physics problem
        if I use two different approaches.
    \begin{multicols}{5}
    \begin{choices}[o]
        \correctchoice{}
        \correctchoice{}
        \correctchoice{}
        \correctchoice{}
        \correctchoice{}
    \end{choices}
    \end{multicols}
\end{question}
}

\element{CLASS}{
\begin{question}{class-Q19}
    \QuestionIndicative
    \scoring{auto=1,v=-1,e=-2}
    To understand physics I discuss it with friends and other students.
    \begin{multicols}{5}
    \begin{choices}[o]
        \correctchoice{}
        \correctchoice{}
        \correctchoice{}
        \correctchoice{}
        \correctchoice{}
    \end{choices}
    \end{multicols}
\end{question}
}

\element{CLASS}{
\begin{question}{class-Q20}
    \QuestionIndicative
    \scoring{auto=1,v=-1,e=-2}
    I do not spend more than five minutes stuck on a physics problem before
        giving up or seeking help from someone else.
    \begin{multicols}{5}
    \begin{choices}[o]
        \correctchoice{}
        \correctchoice{}
        \correctchoice{}
        \correctchoice{}
        \correctchoice{}
    \end{choices}
    \end{multicols}
\end{question}
}

\element{CLASS}{
\begin{question}{class-Q21}
    \QuestionIndicative
    \scoring{auto=1,v=-1,e=-2}
    If I don't remember a particular equation needed to solve a problem
        on an exam, there's nothing much I can do (legally) to come up with it.
    \begin{multicols}{5}
    \begin{choices}[o]
        \correctchoice{}
        \correctchoice{}
        \correctchoice{}
        \correctchoice{}
        \correctchoice{}
    \end{choices}
    \end{multicols}
\end{question}
}

\element{CLASS}{
\begin{question}{class-Q22}
    \QuestionIndicative
    \scoring{auto=1,v=-1,e=-2}
    If I want to apply a method used for solving one physics problem to another
        problem, the problems must involve very similar situations.
    \begin{multicols}{5}
    \begin{choices}[o]
        \correctchoice{}
        \correctchoice{}
        \correctchoice{}
        \correctchoice{}
        \correctchoice{}
    \end{choices}
    \end{multicols}
\end{question}
}

\element{CLASS}{
\begin{question}{class-Q23}
    \QuestionIndicative
    \scoring{auto=1,v=-1,e=-2}
    In doing a physics problem, if my calculation gives a result very different
        from what I'd expect, I'd trust the calculation rather than going back
        through the problem.
    \begin{multicols}{5}
    \begin{choices}[o]
        \correctchoice{}
        \correctchoice{}
        \correctchoice{}
        \correctchoice{}
        \correctchoice{}
    \end{choices}
    \end{multicols}
\end{question}
}

\element{CLASS}{
\begin{question}{class-Q24}
    \QuestionIndicative
    \scoring{auto=1,v=-1,e=-2}
    In physics, it is important for me to make sense out of formulas before I
        can use them correctly.
    \begin{multicols}{5}
    \begin{choices}[o]
        \correctchoice{}
        \correctchoice{}
        \correctchoice{}
        \correctchoice{}
        \correctchoice{}
    \end{choices}
    \end{multicols}
\end{question}
}

\element{CLASS}{
\begin{question}{class-Q25}
    \QuestionIndicative
    \scoring{auto=1,v=-1,e=-2}
    I enjoy solving physics problems.
    \begin{multicols}{5}
    \begin{choices}[o]
        \correctchoice{}
        \correctchoice{}
        \correctchoice{}
        \correctchoice{}
        \correctchoice{}
    \end{choices}
    \end{multicols}
\end{question}
}

\element{CLASS}{
\begin{question}{class-Q26}
    \QuestionIndicative
    \scoring{auto=1,v=-1,e=-2}
    In physics, mathematical formulas express meaningful relationships
        among measurable quantities.
    \begin{multicols}{5}
    \begin{choices}[o]
        \correctchoice{}
        \correctchoice{}
        \correctchoice{}
        \correctchoice{}
        \correctchoice{}
    \end{choices}
    \end{multicols}
\end{question}
}

\element{CLASS}{
\begin{question}{class-Q27}
    \QuestionIndicative
    \scoring{auto=1,v=-1,e=-2}
    It is important for the government to approve new scientific ideas before
        they can be widely accepted.
    \begin{multicols}{5}
    \begin{choices}[o]
        \correctchoice{}
        \correctchoice{}
        \correctchoice{}
        \correctchoice{}
        \correctchoice{}
    \end{choices}
    \end{multicols}
\end{question}
}

\element{CLASS}{
\begin{question}{class-Q28}
    \QuestionIndicative
    \scoring{auto=1,v=-1,e=-2}
    Learning physics changes my ideas about how the world works.
    \begin{multicols}{5}
    \begin{choices}[o]
        \correctchoice{}
        \correctchoice{}
        \correctchoice{}
        \correctchoice{}
        \correctchoice{}
    \end{choices}
    \end{multicols}
\end{question}
}

\element{CLASS}{
\begin{question}{class-Q29}
    \QuestionIndicative
    \scoring{auto=1,v=-1,e=-2}
    To learn physics, I only need to memorize solutions to sample problems.
    \begin{multicols}{5}
    \begin{choices}[o]
        \correctchoice{}
        \correctchoice{}
        \correctchoice{}
        \correctchoice{}
        \correctchoice{}
    \end{choices}
    \end{multicols}
\end{question}
}

\element{CLASS}{
\begin{question}{class-Q30}
    \QuestionIndicative
    \scoring{auto=1,v=-1,e=-2}
    Reasoning skills used to understanding physics can be helpful to me in
        my everyday life.
    \begin{multicols}{5}
    \begin{choices}[o]
        \correctchoice{}
        \correctchoice{}
        \correctchoice{}
        \correctchoice{}
        \correctchoice{}
    \end{choices}
    \end{multicols}
\end{question}
}

\element{CLASS}{
\begin{question}{class-Q31}
    \QuestionIndicative
    \scoring{auto=1,v=-1,e=-2}
    We use this statement to discard the survey of people who are not
        reading the questions.
    Please select agree-option 4 (not strong agree) for this question
        to preserve your answers.
    \begin{multicols}{5}
    \begin{choices}[o]
        \correctchoice{}
        \correctchoice{}
        \correctchoice{}
        \correctchoice{}
        \correctchoice{}
    \end{choices}
    \end{multicols}
\end{question}
}

\element{CLASS}{
\begin{question}{class-Q32}
    \QuestionIndicative
    \scoring{auto=1,v=-1,e=-2}
    Spending a lot of time understanding where formulas come from is
        a waste of time.
    \begin{multicols}{5}
    \begin{choices}[o]
        \correctchoice{}
        \correctchoice{}
        \correctchoice{}
        \correctchoice{}
        \correctchoice{}
    \end{choices}
    \end{multicols}
\end{question}
}

\element{CLASS}{
\begin{question}{class-Q33}
    \QuestionIndicative
    \scoring{auto=1,v=-1,e=-2}
    I find carefully analyzing only a few problems in detail is a good
        way for me to learn physics.
    \begin{multicols}{5}
    \begin{choices}[o]
        \correctchoice{}
        \correctchoice{}
        \correctchoice{}
        \correctchoice{}
        \correctchoice{}
    \end{choices}
    \end{multicols}
\end{question}
}

\element{CLASS}{
\begin{question}{class-Q34}
    \QuestionIndicative
    \scoring{auto=1,v=-1,e=-2}
    I can usually figure out a way to solve physics problems.
    \begin{multicols}{5}
    \begin{choices}[o]
        \correctchoice{}
        \correctchoice{}
        \correctchoice{}
        \correctchoice{}
        \correctchoice{}
    \end{choices}
    \end{multicols}
\end{question}
}

\element{CLASS}{
\begin{question}{class-Q35}
    \QuestionIndicative
    \scoring{auto=1,v=-1,e=-2}
    The subject of physics has little relation to what I experience in
        the real world.
    \begin{multicols}{5}
    \begin{choices}[o]
        \correctchoice{}
        \correctchoice{}
        \correctchoice{}
        \correctchoice{}
        \correctchoice{}
    \end{choices}
    \end{multicols}
\end{question}
}

\element{CLASS}{
\begin{question}{class-Q36}
    \QuestionIndicative
    \scoring{auto=1,v=-1,e=-2}
    There are times I solve a physics problem more than one way to
        help my understanding.
    \begin{multicols}{5}
    \begin{choices}[o]
        \correctchoice{}
        \correctchoice{}
        \correctchoice{}
        \correctchoice{}
        \correctchoice{}
    \end{choices}
    \end{multicols}
\end{question}
}

\element{CLASS}{
\begin{question}{class-Q37}
    \QuestionIndicative
    \scoring{auto=1,v=-1,e=-2}
    To understand physics, I sometimes think about my personal experiences
        and relate them to the topic being analyzed.
    \begin{multicols}{5}
    \begin{choices}[o]
        \correctchoice{}
        \correctchoice{}
        \correctchoice{}
        \correctchoice{}
        \correctchoice{}
    \end{choices}
    \end{multicols}
\end{question}
}

\element{CLASS}{
\begin{question}{class-Q38}
    \QuestionIndicative
    \scoring{auto=1,v=-1,e=-2}
    It is possible to explain physics ideas without mathematical formulas.
    \begin{multicols}{5}
    \begin{choices}[o]
        \correctchoice{}
        \correctchoice{}
        \correctchoice{}
        \correctchoice{}
        \correctchoice{}
    \end{choices}
    \end{multicols}
\end{question}
}

\element{CLASS}{
\begin{question}{class-Q39}
    \QuestionIndicative
    \scoring{auto=1,v=-1,e=-2}
    When I solve a physics problem, I explicitly think about which physics
        ideas apply to the problem.
    \begin{multicols}{5}
    \begin{choices}[o]
        \correctchoice{}
        \correctchoice{}
        \correctchoice{}
        \correctchoice{}
        \correctchoice{}
    \end{choices}
    \end{multicols}
\end{question}
}

\element{CLASS}{
\begin{question}{class-Q40}
    \QuestionIndicative
    \scoring{auto=1,v=-1,e=-2}
    If I get stuck on a physics problem, there is no chance I'll figure it
        out on my own.
    \begin{multicols}{5}
    \begin{choices}[o]
        \correctchoice{}
        \correctchoice{}
        \correctchoice{}
        \correctchoice{}
        \correctchoice{}
    \end{choices}
    \end{multicols}
\end{question}
}

\element{CLASS}{
\begin{question}{class-Q41}
    \QuestionIndicative
    \scoring{auto=1,v=-1,e=-2}
    It is possible for physicists to carefully perform the same experiment
        and get two very different results that are both correct.
    \begin{multicols}{5}
    \begin{choices}[o]
        \correctchoice{}
        \correctchoice{}
        \correctchoice{}
        \correctchoice{}
        \correctchoice{}
    \end{choices}
    \end{multicols}
\end{question}
}

\element{CLASS}{
\begin{question}{class-Q42}
    \QuestionIndicative
    \scoring{auto=1,v=-1,e=-2}
    When studying physics, I relate important information to what
        I already know rather than just memorizing in the way it is presented.
    \begin{multicols}{5}
    \begin{choices}[o]
        \correctchoice{}
        \correctchoice{}
        \correctchoice{}
        \correctchoice{}
        \correctchoice{}
    \end{choices}
    \end{multicols}
\end{question}
}

\endinput


