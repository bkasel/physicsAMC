
%%--------------------------------------------------
%% Serway: Physics for Scientists and Engineers
%%--------------------------------------------------


%% Chapter 06: Circular Motion and other
%%     Applications of Newton's Laws
%%--------------------------------------------------


%% Serway Numeric Questions
%%--------------------------------------------------
\element{serway-num}{
\begin{questionmultx}{serway-ch06-q48}
    A sample of blood is placed into a centrifuge of radius \SI{15.0}{\centi\meter}. 
    The mass of a red corpuscle is \SI{3.0e-16}{\kilo\gram},
        and the centripetal force required to make it settle out of the plasma is \SI{4.0e-11}{\newton}. 
    At how many revolutions per second should the centrifuge be operated?
    %% ANSWER:  \SI{rev\per\second}
    \AMCnumericChoices{1.50}{
        vertical=false,
        digits=3,decimals=2,sign=true,Tsign=\hspace{1ex}\Large a,
        borderwidth=0pt,backgroundcol=white,approx=5
    }
    \AMCnumericChoices{2}{
        vertical=false,
        digits=2,decimals=0,sign=true,Tsign=\hspace{1ex}\Large b,
        borderwidth=0pt,backgroundcol=white,approx=5
    }
\end{questionmultx}
}

\element{serway-num}{
\begin{questionmultx}{serway-ch06-q49}
    A space station in the form of a large wheel, \SI{120}{\meter} in diameter,
        rotates to provide an ``artificial gravity'' of \SI{3.00}{\meter\per\second\squared} for persons located at the outer rim. 
    Find the rotational frequency of the wheel (in revolutions per minute) that will produce this effect.
    %% ANSWER:  \SI{2.14}{\rev\per\minute}
    \AMCnumericChoices{2.14}{
        vertical=false,
        digits=3,decimals=2,sign=true,Tsign=\hspace{1ex}\Large a,
        borderwidth=0pt,backgroundcol=white,approx=5
    }
    \AMCnumericChoices{0}{
        vertical=false,
        digits=2,decimals=0,sign=true,Tsign=\hspace{1ex}\Large b,
        borderwidth=0pt,backgroundcol=white,approx=5
    }
\end{questionmultx}
}

\element{serway-num}{
\begin{questionmultx}{serway-ch06-q50}
    An airplane pilot experiences weightlessness as she passes over the top of a loop-the-loop maneuver. 
    If her speed is \SI{200}{\meter\per\second} at the time,
        find the radius of the loop.
    %% ANSWER:  \num{4080}{\meter}
    \AMCnumericChoices{4.08}{
        vertical=false,
        digits=3,decimals=2,sign=true,Tsign=\hspace{1ex}\Large a,
        borderwidth=0pt,backgroundcol=white,approx=5
    }
    \AMCnumericChoices{3}{
        vertical=false,
        digits=2,decimals=0,sign=true,Tsign=\hspace{1ex}\Large b,
        borderwidth=0pt,backgroundcol=white,approx=5
    }
\end{questionmultx}
}

\element{serway-num}{
\begin{questionmultx}{serway-ch06-q51}
    A race car starts from rest on a circular track of radius \SI{400}{\meter}. 
    Its speed increases at the constant rate of \SI{0.500}{\meter\per\second\squared}.
    At the point where the magnitudes of the radial and tangential accelerations are equal,
        determine (a) the speed of the race car, and (b) the elapsed time.
    %% NOTE: split into A and B questions
    %% ANSWER:  \SI{14.1}{\meter\per\second}, \SI{28.2}{\second}
    \AMCnumericChoices{14.1}{
        vertical=false,
        digits=3,decimals=2,sign=true,Tsign=\hspace{1ex}\Large a,
        borderwidth=0pt,backgroundcol=white,approx=5
    }
    \AMCnumericChoices{1}{
        vertical=false,
        digits=2,decimals=0,sign=true,Tsign=\hspace{1ex}\Large b,
        borderwidth=0pt,backgroundcol=white,approx=5
    }
\end{questionmultx}
}



\endinput


