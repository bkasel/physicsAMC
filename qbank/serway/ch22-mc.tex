
%%--------------------------------------------------
%% Serway: Physics for Scientists and Engineers
%%--------------------------------------------------


%% Chapter 22: Heat Engines, Entropy and the Second Law of Thermodynamics
%%--------------------------------------------------


%% Table of Contents
%%--------------------------------------------------

%% 22.1 Heat Engines and the Second Law of Thermodynamics
%% 22.2 Heat Pumps and Refrigerators
%% 22.3 Reversible and Irreversible Processes
%% 22.4 The Carnot Engine
%% 22.5 Gasoline and Diesel Engines
%% 22.6 Entropy 
%% 22.7 Entropy Changes in Irreversible Processes
%% 22.8 Entropy on a Microscopic Scale


%% Serway Multiple Choice Questions
%%--------------------------------------------------
\element{serway-mc}{
\begin{question}{serway-ch22-q01}
    A gasoline engine absorbs \SI{2500}{\joule} of heat and performs \SI{1000}{\joule} of mechanical work in each cycle. 
    The efficiency of the engine is:
    \begin{multicols}{3}
    \begin{choices}
        \wrongchoice{\SI{80}{\percent}}
      \correctchoice{\SI{40}{\percent}}
        \wrongchoice{\SI{60}{\percent}}
        \wrongchoice{\SI{20}{\percent}}
        \wrongchoice{\SI{50}{\percent}}
    \end{choices}
    \end{multicols}
\end{question}
}

\element{serway-mc}{
\begin{question}{serway-ch22-q02}
    A gasoline engine absorbs \SI{2500}{\joule} of heat and performs \SI{1000}{\joule} of mechanical work in each cycle. 
    The amount of heat expelled in each cycle is:
    \begin{multicols}{3}
    \begin{choices}
        \wrongchoice{\SI{1000}{\joule}}
      \correctchoice{\SI{1500}{\joule}}
        \wrongchoice{\SI{2000}{\joule}}
        \wrongchoice{\SI{500}{\joule}}
        \wrongchoice{\SI{3000}{\joule}}
    \end{choices}
    \end{multicols}
\end{question}
}

\element{serway-mc}{
\begin{question}{serway-ch22-q03}
    A heat pump has a coefficient of performance of 4. 
    If the heat pump absorbs \SI{20}{\calorie} of heat from the cold outdoors in each cycle,
        the heat expelled to the warm indoors is:
    \begin{multicols}{3}
    \begin{choices}
        \wrongchoice{\SI{34}{\calorie}}
      \correctchoice{\SI{27}{\calorie}}
        \wrongchoice{\SI{36}{\calorie}}
        \wrongchoice{\SI{40}{\calorie}}
        \wrongchoice{\SI{80}{\calorie}}
    \end{choices}
    \end{multicols}
\end{question}
}

\element{serway-mc}{
\begin{question}{serway-ch22-q04}
    A refrigerator has a coefficient of performance of 4. 
    If the refrigerator absorbs \SI{30}{\calorie} of heat from the cold reservoir in each cycle,
        the heat expelled into the heat reservoir is:
    \begin{multicols}{3}
    \begin{choices}
        \wrongchoice{\SI{40.5}{\calorie}}
      \correctchoice{\SI{37.5}{\calorie}}
        \wrongchoice{\SI{36.5}{\calorie}}
        \wrongchoice{\SI{34.5}{\calorie}}
        \wrongchoice{\SI{22.5}{\calorie}}
    \end{choices}
    \end{multicols}
\end{question}
}

\element{serway-mc}{
\begin{question}{serway-ch22-q05}
    A lawn mower has a six horsepower engine ($\SI{1}{hp}=\SI{750}{\watt}$). 
    If the engine has an efficiency of \SI{20}{\percent},
        and the throttle is such that the engine cycles ten times a second,
        the heat that is expelled in one cycle is:
    \begin{multicols}{3}
    \begin{choices}
      \correctchoice{\SI{1800}{\joule}}
        \wrongchoice{\SI{2000}{\joule}}
        \wrongchoice{\SI{2200}{\joule}}
        \wrongchoice{\SI{2400}{\joule}}
        \wrongchoice{\SI{2250}{\joule}}
    \end{choices}
    \end{multicols}
\end{question}
}

\element{serway-mc}{
\begin{question}{serway-ch22-q06}
    A steam engine is operating at its theoretical maximum efficiency of \SI{60}{\percent}. 
    If the waste heat has a temperature of \SI{100}{\degree\Fahrenheit} (\SI{38}{\degreeCelsius}),
        what is the temperature of the boiler?
    \begin{multicols}{3}
    \begin{choices}
        \wrongchoice{\SI{350}{\degreeCelsius}}
        \wrongchoice{\SI{94}{\degreeCelsius}}
        \wrongchoice{\SI{225}{\degreeCelsius}}
      \correctchoice{\SI{504}{\degreeCelsius}}
        \wrongchoice{\SI{775}{\degreeCelsius}}
    \end{choices}
    \end{multicols}
\end{question}
}

\element{serway-mc}{
\begin{question}{serway-ch22-q07}
    A company that produces pulsed gas heaters claims their efficiency is approximately \SI{90}{\percent}. 
    If an engine operates between \SI{250}{\degreeCelsius} and \SI{25}{\degreeCelsius},
        what is its maximum thermodynamic efficiency?
    \begin{multicols}{3}
    \begin{choices}
        \wrongchoice{\SI{83}{\percent}}
        \wrongchoice{\SI{65}{\percent}}
      \correctchoice{\SI{43}{\percent}}
        \wrongchoice{\SI{90}{\percent}}
        \wrongchoice{\SI{56}{\percent}}
    \end{choices}
    \end{multicols}
\end{question}
}

\element{serway-mc}{
\begin{question}{serway-ch22-q08}
    A heat engine absorbs \SI{2500}{\joule} of heat from a hot reservoir and expels \SI{1000}{\joule} to a cold reservoir. 
    When it is run in reverse, with the same reservoirs,
        the engine pumps \SI{2500}{\joule} of heat to the hot reservoir,
        requiring \SI{1500}{\joule} of work to do so. 
    Find the ratio of the work done by the heat engine to the work done by the pump. 
    Is the heat engine reversible?
    \begin{multicols}{3}
    \begin{choices}
        \wrongchoice{1.0 (no)}
      \correctchoice{1.0 (yes)}
        \wrongchoice{1.5 (yes)}
        \wrongchoice{1.5 (no)}
        \wrongchoice{2.5 (no)}
    \end{choices}
    \end{multicols}
\end{question}
}

\element{serway-mc}{
\begin{question}{serway-ch22-q09}
    %% NOTE: define COP: coefficient of performance 
    On a cold day, a heat pump absorbs heat from the outside air at \SI{14}{\degree\Fahrenheit} (\SI{-10}{\degreeCelsius}) and transfers it into a home at a temperature of \SI{86}{\degree\Fahrenheit} (\SI{30}{\degreeCelsius}).
    %$ NOTE: Determine the maximum COP of the heat pump.
    Determine the maximum coefficient of performance of the heat pump.
    \begin{multicols}{3}
    \begin{choices}
        \wrongchoice{\num{0.2}}
        \wrongchoice{\num{4.4}}
        \wrongchoice{\num{0.5}}
      \correctchoice{\num{7.6}}
        \wrongchoice{\num{6.7}}
    \end{choices}
    \end{multicols}
\end{question}
}

\element{serway-mc}{
\begin{question}{serway-ch22-q10}
    A new electric power plant has an efficiency of \SI{42}{\percent}.
    For every 100 barrels of oil needed to run the turbine,
        how many are essentially lost as waste heat
        (in barrels of oil) to the environment?
    \begin{multicols}{3}
    \begin{choices}
        \wrongchoice{\num{21}}
        \wrongchoice{\num{42}}
      \correctchoice{\num{58}}
        \wrongchoice{\num{10}}
        \wrongchoice{\num{79}}
    \end{choices}
    \end{multicols}
\end{question}
}

\element{serway-mc}{
\begin{question}{serway-ch22-q11}
    An \SI{800}{\mega\watt} electric power plant has an efficiency of \SI{30}{\percent}.
    It loses its waste heat in large cooling towers. 
    Approximately how much waste heat is discharged to the atmosphere every second?
    \begin{multicols}{2}
    \begin{choices}
        \wrongchoice{\SI{1200}{\mega\joule}}
      \correctchoice{\SI{1900}{\mega\joule}}
        \wrongchoice{\SI{800}{\mega\joule}}
        \wrongchoice{\SI{560}{\mega\joule}}
        \wrongchoice{\SI{240}{\mega\joule}}
    \end{choices}
    \end{multicols}
\end{question}
}

\element{serway-mc}{
\begin{question}{serway-ch22-q12}
    A homeowner has a new oil furnace which has an efficiency of \SI{60}{\percent}.
    For every 100 barrels of oil needed to heat his house,
        how much (in barrels of oil) goes up the chimney as waste heat?
    \begin{multicols}{3}
    \begin{choices}
        \wrongchoice{\num{20}}
        \wrongchoice{\num{60}}
      \correctchoice{\num{40}}
        \wrongchoice{\num{80}}
        \wrongchoice{\num{10}}
    \end{choices}
    \end{multicols}
\end{question}
}

\element{serway-mc}{
\begin{question}{serway-ch22-q13}
    One kilogram of chilled water at \SI{32}{\degree\Fahrenheit} (\SI{0}{\degreeCelsius}) is placed in a freezer which is kept at \SI{0}{\degree\Fahrenheit} (\SI{-18}{\degreeCelsius}).
    Approximately how much electric energy is needed to operate the compressor to cool this water to 0°F if the room temperature is maintained at \SI{75}{\degree\Fahrenheit} (\SI{24}{\degreeCelsius})? 
    ($L_{\text{ice}}=\SI{3.33e5}{\joule}$; $c_{\text{ice}}=\SI{2.09e3}{\joule}$)
    \begin{multicols}{3}
    \begin{choices}
        \wrongchoice{\SI{13}{\kilo\calorie}}
        \wrongchoice{\SI{1.5}{\kilo\calorie}}
      \correctchoice{\SI{15}{\kilo\calorie}}
        \wrongchoice{\SI{16}{\kilo\calorie}}
        \wrongchoice{\SI{33}{\kilo\calorie}}
    \end{choices}
    \end{multicols}
\end{question}
}

\element{serway-mc}{
\begin{question}{serway-ch22-q14}
    One kilogram of chilled water (\SI{0}{\degreeCelsius}) is placed in a freezer which is kept at \SI{0}{\degree\Fahrenheit} (\SI{-18}{\degreeCelsius}).
    Approximately how much electric energy is needed just to freeze the water if the room temperature is maintained at \SI{75}{\degree\Fahrenheit} (\SI{24}{\degreeCelsius})?
    ($L_{\text{ice}}=\SI{333}{\joule\per\gram}$; $c_{\text{ice}}=\SI{209}{\joule\per\gram\per\degreeCelsius}$.)
    \begin{multicols}{3}
    \begin{choices}
        \wrongchoice{\SI{11}{\kilo\calorie}}
        \wrongchoice{\SI{15}{\kilo\calorie}}
        \wrongchoice{\SI{16}{\kilo\calorie}}
      \correctchoice{\SI{13}{\kilo\calorie}}
        \wrongchoice{\SI{33}{\kilo\calorie}}
    \end{choices}
    \end{multicols}
\end{question}
}

\element{serway-mc}{
\begin{question}{serway-ch22-q15}
    An automobile engine operates with an overall efficiency of \SI{12}{\percent}.
    How much energy is delivered as waste heat (in gallons of gasoline) for each ten gallons of fuel burned?
    \begin{multicols}{3}
    \begin{choices}
        \wrongchoice{\SI{1.2}{\gallon}}
      \correctchoice{\SI{8.8}{\gallon}}
        \wrongchoice{\SI{6.5}{\gallon}}
        \wrongchoice{\SI{4.7}{\gallon}}
        \wrongchoice{\SI{7.5}{\gallon}}
    \end{choices}
    \end{multicols}
\end{question}
}

\element{serway-mc}{
\begin{question}{serway-ch22-q16}
    An engine is designed to obtain energy from the temperature gradient of the ocean. 
    What is the thermodynamic efficiency of such an engine if the temperature of the surface of the water is \SI{59}{\degree\Fahrenheit} (\SI{15}{\degreeCelsius}) and the temperature well below the surface is \SI{41}{\degree\Fahrenheit} (\SI{5}{\degreeCelsius})?
    \begin{multicols}{3}
    \begin{choices}
      \correctchoice{\SI{3.5}{\percent}}
        \wrongchoice{\SI{67}{\percent}}
        \wrongchoice{\SI{31}{\percent}}
        \wrongchoice{\SI{17}{\percent}}
        \wrongchoice{\SI{96}{\percent}}
    \end{choices}
    \end{multicols}
\end{question}
}

\element{serway-mc}{
\begin{question}{serway-ch22-q17}
    A vessel containing \SI{5.0}{\kilo\gram} of water at \SI{10}{\degreeCelsius} is put into a refrigerator. 
    The \SI{1/7}{hp} motor ($\SI{1}{hp}=\SI{746}{\watt}$) runs for \SI{5.0}{\minute} to cool the liquid to the refrigerator's low temperature, \SI{0}{\degreeCelsius}. 
    %% NOTE: define COP: coefficient of performance
    %What is the COP of the refrigerator?
    What is the coefficient of performance of the refrigerator?
    \begin{multicols}{3}
    \begin{choices}
        \wrongchoice{\num{5.7}}
        \wrongchoice{\num{4.6}}
      \correctchoice{\num{6.5}}
        \wrongchoice{\num{7.2}}
        \wrongchoice{\num{3.6}}
    \end{choices}
    \end{multicols}
\end{question}
}

\element{serway-mc}{
\begin{question}{serway-ch22-q18}
    Exactly \SI{500}{\gram} of ice are melted at a temperature of \SI{32}{\degree\Fahrenheit}. 
    ($L_{\text{ice}}=\SI{333}{\joule\per\gram}$.)
    The change in entropy is:
    \begin{multicols}{2}
    \begin{choices}
        \wrongchoice{\SI{321}{\joule\per\kelvin}}
        \wrongchoice{\SI{146}{\joule\per\kelvin}}
        \wrongchoice{\SI{512}{\joule\per\kelvin}}
      \correctchoice{\SI{610}{\joule\per\kelvin}}
        \wrongchoice{\SI{5230}{\joule\per\kelvin}}
    \end{choices}
    \end{multicols}
\end{question}
}

\element{serway-mc}{
\begin{question}{serway-ch22-q19}
    When water of mass $m$ and specific heat $c$ is heated from absolute temperature $T_1$ to absolute temperature $T_2$,
        its change in entropy is:
    \begin{multicols}{2}
    \begin{choices}
      \correctchoice{$cm\ln\left(\dfrac{T_2}{T_1}\right)$}
        \wrongchoice{$cm\dfrac{T_2}{T_1}$}
        \wrongchoice{$cm\dfrac{T_2-T_1}{T_1}$}
        \wrongchoice{$cm\dfrac{T_2-T_1}{T_2}$}
        \wrongchoice{$cm\dfrac{T_2-T_1}{T_2+T_1}$}
    \end{choices}
    \end{multicols}
\end{question}
}

\element{serway-mc}{
\begin{question}{serway-ch22-q20}
    The change in entropy of \SI{1.00}{\kilo\gram} of water that is heated from \SI{50}{\degreeCelsius} to \SI{100}{\degreeCelsius} is:
    \begin{multicols}{2}
    \begin{choices}
        \wrongchoice{\SI{516}{\calorie\per\kelvin}}
        \wrongchoice{\SI{312}{\calorie\per\kelvin}}
      \correctchoice{\SI{144}{\calorie\per\kelvin}}
        \wrongchoice{\SI{946}{\calorie\per\kelvin}}
        \wrongchoice{\SI{391}{\calorie\per\kelvin}}
    \end{choices}
    \end{multicols}
\end{question}
}

\element{serway-mc}{
\begin{question}{serway-ch22-q21}
    The change in entropy of a mass $m$ of a solid substance which has a latent heat of fusion $L$ and melts at a temperature $T$ is:
    \begin{multicols}{3}
    \begin{choices}
        \wrongchoice{$\dfrac{LT}{m}$}
        \wrongchoice{$mL\,\ln\left(T\right)$}
        \wrongchoice{$mLT$}
      \correctchoice{$\dfrac{mL}{T}$}
        \wrongchoice{$\dfrac{L}{mT}$}
    \end{choices}
    \end{multicols}
\end{question}
}

\element{serway-mc}{
\begin{question}{serway-ch22-q22}
    Since $L_{\text{ice}}=\SI{333}{\joule\per\gram}$,
        the change in entropy when \SI{1.00}{\kilo\gram} of ice melts is:
    \begin{multicols}{2}
    \begin{choices}
        \wrongchoice{\SI{144}{\calorie\per\kelvin}}
      \correctchoice{\SI{291}{\calorie\per\kelvin}}
        \wrongchoice{\SI{312}{\calorie\per\kelvin}}
        \wrongchoice{\SI{516}{\calorie\per\kelvin}}
        \wrongchoice{\SI{80}{\calorie\per\kelvin}}
    \end{choices}
    \end{multicols}
\end{question}
}

\element{serway-mc}{
\begin{question}{serway-ch22-q23}
    If $n$ moles of an ideal gas are compressed isothermally from an initial volume $V_1$ to a final volume $V_2$,
        the change in entropy is:
    \begin{multicols}{2}
    \begin{choices}
      \correctchoice{$nR\ln\left(\dfrac{V_2}{V_1}\right)$}
        \wrongchoice{$nRT\ln\left(\dfrac{V_2}{V_1}\right)$}
        \wrongchoice{$nk_B\ln\left(\dfrac{V_2}{V_1}\right)$}
        \wrongchoice{$\displaystyle nC_v \int\;\dfrac{\mathrm{d}T}{T}$}
        \wrongchoice{$\dfrac{nC_v}{T}$}
    \end{choices}
    \end{multicols}
\end{question}
}

\element{serway-mc}{
\begin{question}{serway-ch22-q24}
    Determine the change in entropy when \SI{5.00}{\mole} of an ideal gas at \SI{0}{\degreeCelsius} are compressed isothermally from an initial volume of \SI{100}{\centi\meter\cubed} to a final volume of \SI{20}{\centi\meter\cubed}.
    \begin{multicols}{2}
    \begin{choices}
        \wrongchoice{\SI{-191}{\joule\per\kelvin}}
        \wrongchoice{\SI{-52}{\joule\per\kelvin}}
        \wrongchoice{\SI{-71}{\joule\per\kelvin}}
      \correctchoice{\SI{-67}{\joule\per\kelvin}}
        \wrongchoice{\SI{-208}{\joule\per\kelvin}}
    \end{choices}
    \end{multicols}
\end{question}
}

\element{serway-mc}{
\begin{question}{serway-ch22-q25}
    An ideal gas is allowed to undergo a free expansion. 
    If its initial volume is $V_1$ and its final volume is $V_2$,
        the change in entropy is:
    \begin{multicols}{2}
    \begin{choices}
      \correctchoice{$nR\ln\left(\dfrac{V_2}{V_1}\right)$}
        \wrongchoice{$nRT\ln\left(\dfrac{V_2}{V_1}\right)$}
        \wrongchoice{$nk_B\ln\left(\dfrac{V_2}{V_1}\right)$}
        \wrongchoice{zero}
        \wrongchoice{$\dfrac{nRV_2}{V_1}$}
    \end{choices}
    \end{multicols}
\end{question}
}

\element{serway-mc}{
\begin{question}{serway-ch22-q26}
    Find the change in entropy when \SI{5.00}{\mole} of an ideal gas undergo a free expansion from an initial volume of \SI{20}{\centi\meter\cubed} to a final volume of \SI{100}{\centi\meter\cubed}.
    \begin{multicols}{3}
    \begin{choices}
        \wrongchoice{\SI{71}{\joule\per\kelvin}}
        \wrongchoice{\SI{52}{\joule\per\kelvin}}
      \correctchoice{\SI{67}{\joule\per\kelvin}}
        \wrongchoice{\SI{191}{\joule\per\kelvin}}
        \wrongchoice{\SI{208}{\joule\per\kelvin}}
    \end{choices}
    \end{multicols}
\end{question}
}

\element{serway-mc}{
\begin{question}{serway-ch22-q27}
    An ideal gas is allowed to expand adiabatically. 
    Assume the process is reversible.
    The change in entropy is:
    \begin{multicols}{2}
    \begin{choices}
      \correctchoice{zero}
        \wrongchoice{$nR\ln\left(\dfrac{V_2}{V_1}\right)$}
        \wrongchoice{$nR\ln\left(\dfrac{T_2}{T_1}\right)$}
        \wrongchoice{$k_Bn\ln\left(\dfrac{V_2}{V_1}\right)$}
        \wrongchoice{$k_Bn\ln\left(\dfrac{T_2}{T_1}\right)$}
    \end{choices}
    \end{multicols}
\end{question}
}

\element{serway-mc}{
\begin{question}{serway-ch22-q28}
    Find the change in entropy when \SI{5.00}{\mole} of an ideal monatomic gas are allowed to expand isobarically from an initial volume of \SI{20}{\centi\meter\cubed} to a final volume of \SI{100}{\centi\meter\cubed}.
    \begin{multicols}{2}
    \begin{choices}
       \correctchoice{\SI{167}{\joule\per\kelvin}}
         \wrongchoice{\SI{100}{\joule\per\kelvin}}
         \wrongchoice{\SI{67}{\joule\per\kelvin}}
         \wrongchoice{\SI{52}{\joule\per\kelvin}}
         \wrongchoice{\SI{152}{\joule\per\kelvin}}
    \end{choices}
    \end{multicols}
\end{question}
}

\element{serway-mc}{
\begin{question}{serway-ch22-q29}
    Ten kilograms of water at \SI{0}{\degreeCelsius} is mixed with \SI{10}{\kilo\gram} of water at \SI{100}{\degreeCelsius}.
    The change in entropy of the system is:
    \begin{multicols}{2}
    \begin{choices}
        \wrongchoice{\SI{1000}{\calorie\per\kelvin}}
        \wrongchoice{\SI{480}{\calorie\per\kelvin}}
        \wrongchoice{\SI{-720}{\calorie\per\kelvin}}
      \correctchoice{\SI{240}{\calorie\per\kelvin}}
        \wrongchoice{\SI{-168}{\calorie\per\kelvin}}
    \end{choices}
    \end{multicols}
\end{question}
}

\element{serway-mc}{
\begin{question}{serway-ch22-q30}
    A vessel containing \SI{10}{\kilo\gram} of water is left out where it evaporates. 
    If the temperature remains constant at \SI{20}{\degreeCelsius},
        what is the change in entropy?
    (The latent heat of vaporization at \SI{20}{\degreeCelsius} is \SI{585}{\calorie\per\gram}.)
    \begin{multicols}{2}
    \begin{choices}
        \wrongchoice{\SI{30}{\calorie\per\gram}}
        \wrongchoice{\SI{10}{\calorie\per\gram}}
      \correctchoice{\SI{20}{\calorie\per\gram}}
        \wrongchoice{\SI{40}{\calorie\per\gram}}
        \wrongchoice{\SI{290}{\calorie\per\gram}}
    \end{choices}
    \end{multicols}
\end{question}
}

\element{serway-mc}{
\begin{question}{serway-ch22-q31}
    A gasoline engine absorbs \SI{2500}{\joule} of heat at \SI{250}{\degreeCelsius} and expels \SI{2000}{\joule} at a temperature of \SI{50}{\degreeCelsius}.
    The change in entropy for the system is:
    \begin{multicols}{2}
    \begin{choices}
        \wrongchoice{\SI{6.2}{\joule\per\kelvin}}
        \wrongchoice{\SI{4.7}{\joule\per\kelvin}}
      \correctchoice{\SI{1.4}{\joule\per\kelvin}}
        \wrongchoice{\SI{10.9}{\joule\per\kelvin}}
        \wrongchoice{\SI{3.2}{\joule\per\kelvin}}
    \end{choices}
    \end{multicols}
\end{question}
}

\element{serway-mc}{
\begin{question}{serway-ch22-q32}
    \SI{100}{\gram} of molten lead (\SI{600}{\degreeCelsius}) is used to make musket balls. 
    If the lead shot is allowed to cool to room temperature (\SI{21}{\degreeCelsius}),
        what is the change in entropy of the lead? 
    (For the specific heat of molten and solid lead use \SI{1.29}{\joule\per\gram\per\degreeCelsius};
        the latent heat of fusion and the melting point of lead are \SI{2.45e4}{\joule\per\kilo\gram} and \SI{327}{\degreeCelsius}.)
    \begin{multicols}{2}
    \begin{choices}
      \correctchoice{\SI{-140}{\joule\per\kelvin}}
        \wrongchoice{\SI{-252}{\joule\per\kelvin}}
        \wrongchoice{\SI{-302}{\joule\per\kelvin}}
        \wrongchoice{\SI{-429}{\joule\per\kelvin}}
        \wrongchoice{\SI{-100}{\joule\per\kelvin}}
    \end{choices}
    \end{multicols}
\end{question}
}

\element{serway-mc}{
\begin{question}{serway-ch22-q33}
    The reason that we can calculate the change in entropy of a system is that:
    \begin{choices}
        \wrongchoice{entropy always decreases.}
        \wrongchoice{entropy always increases.}
        \wrongchoice{the entropy of the universe always remains constant.}
      \correctchoice{it depends only on the properties of the initial and final equilibrium states.}
        \wrongchoice{systems always follow reversible paths.}
    \end{choices}
\end{question}
}

\element{serway-mc}{
\begin{question}{serway-ch22-q34}
    By operating a reversible heat engine with an ideal gas as the working substance in a Carnot cycle and measuring the ratio $\dfrac{Q_c}{Q_h}$,
        we can calculate
    \begin{choices}
        \wrongchoice{$n$, the number of moles of the ideal gas.}
        \wrongchoice{the ratio $\dfrac{V_c}{V_h}$ of the volumes of the ideal gas.}
        \wrongchoice{the ratio $\dfrac{P_c}{P_h}$ of the pressures of the ideal gas.}
      \correctchoice{the ratio $\dfrac{P_c V_c}{P_h V_h}$ of the products of volumes and pressures of the ideal gas.}
        \wrongchoice{the value of Avogadro's number.}
    \end{choices}
\end{question}
}

\element{serway-mc}{
\begin{question}{serway-ch22-q35}
    Which of the following is an almost reversible process?
    \begin{choices}
        \wrongchoice{The explosion of hydrogen and oxygen to form water.}
        \wrongchoice{Heat transfer through thick insulation.}
        \wrongchoice{The adiabatic free expansion of a gas.}
      \correctchoice{A slow isothermal compression of a gas.}
        \wrongchoice{A slow leakage of gas into an empty chamber through a small hole in a membrane.}
    \end{choices}
\end{question}
}

\element{serway-mc}{
\begin{question}{serway-ch22-q36}
    The change in entropy when \SI{1}{\kilo\gram} of ice melts at \SI{0}{\degreeCelsius} is:
    ($L_{\text{ice}}=\SI{333}{\joule\per\gram}$.)
    \begin{multicols}{2}
    \begin{choices}
        \wrongchoice{\SI{335}{\joule\per\kelvin}}
        \wrongchoice{\SI{603}{\joule\per\kelvin}}
      \correctchoice{\SI{1220}{\joule\per\kelvin}}
        \wrongchoice{\SI{1310}{\joule\per\kelvin}}
        \wrongchoice{\SI{2160}{\joule\per\kelvin}}
    \end{choices}
    \end{multicols}
\end{question}
}

\element{serway-mc}{
\begin{question}{serway-ch22-q37}
    An ideal heat engine can have an efficiency of one if the temperature of the low temperature reservoir is:
    \begin{multicols}{2}
    \begin{choices}
      \correctchoice{\SI{0}{\kelvin}}
        \wrongchoice{\SI{0}{\degreeCelsius}}
        \wrongchoice{\SI{0}{\degree\Fahrenheit}}
        \wrongchoice{\SI{0}{\degree\Rankine}}
        \wrongchoice{the same as the temperature of the heat source.}
    \end{choices}
    \end{multicols}
\end{question}
}

\element{serway-mc}{
\begin{questionmult}{serway-ch22-q38}
    An adiabatic free expansion of a gas in a thermally isolated container is not reversible because:
    \begin{choices}
      \correctchoice{work must be done on the gas to return it to its original volume.}
        \wrongchoice{heat must be exchanged with the surroundings to return the gas to its original temperature.}
        \wrongchoice{its internal energy has a greater value after the expanded gas is returned to its original volume and temperature.}
        %\wrongchoice{of all of the above.}
        %\wrongchoice{of (a) and (b) above only.}
    \end{choices}
\end{questionmult}
}

\element{serway-mc}{
\begin{question}{serway-ch22-q39}
    A Carnot cycle, operating as a heat engine, consists,
        in the order given, of
    \begin{choices}
        \wrongchoice{an isothermal expansion, an isothermal compression, an adiabatic expansion and an adiabatic compression.}
        \wrongchoice{an adiabatic expansion, an adiabatic compression, an isothermal expansion and an isothermal compression.}
        \wrongchoice{an isothermal expansion, an adiabatic compression, an isothermal compression and an adiabatic expansion.}
        \wrongchoice{an adiabatic compression, an isothermal compression, an isothermal expansion and an adiabatic expansion.}
      \correctchoice{an isothermal expansion, an adiabatic expansion, an isothermal compression and an adiabatic compression.}
    \end{choices}
\end{question}
}

\element{serway-mc}{
\begin{question}{serway-ch22-q40}
    All real engines are less efficient than the Carnot engine because:
    \begin{choices}
        \wrongchoice{they do not operate through the Otto cycle.}
      \correctchoice{they do not operate through a reversible cycle.}
        \wrongchoice{the working substance does not maintain a constant volume through the cycle.}
        \wrongchoice{the working substance does not maintain a constant pressure through the cycle.}
        \wrongchoice{the working substance does not maintain a constant temperature through the cycle.}
    \end{choices}
\end{question}
}

\element{serway-mc}{
\begin{question}{serway-ch22-q41}
    In an engine operating in the Otto cycle,
        the final volume of the fuel-air mixture is one sixth the initial volume. 
    Assume $\gamma=1.4$.
    The maximum theoretical efficiency of this cycle is:
    \begin{multicols}{3}
    \begin{choices}
        \wrongchoice{\SI{17}{\percent}}
        \wrongchoice{\SI{49}{\percent}}
      \correctchoice{\SI{51}{\percent}}
        \wrongchoice{\SI{56}{\percent}}
        \wrongchoice{\SI{83}{\percent}}
    \end{choices}
    \end{multicols}
\end{question}
}

\element{serway-mc}{
\begin{question}{serway-ch22-q42}
    For a gas of $N$ identical molecules of molecular volume $V_m$ in total volume $V$ at temperature $T$,
        the number of ways of locating the $N$ molecules in the volume is:
    \begin{multicols}{3}
    \begin{choices}
        \wrongchoice{$\dfrac{V_m}{V}$}
        \wrongchoice{$\dfrac{V}{V_m}$}
        \wrongchoice{$\left(\dfrac{V}{V_m}\right)^T$}
      \correctchoice{$\left(\dfrac{V_m}{V}\right)^N$}
        \wrongchoice{$\ln\left(\dfrac{V}{V_m}\right)$}
    \end{choices}
    \end{multicols}
\end{question}
}

\element{serway-mc}{
\begin{questionmult}{serway-ch22-q43}
    A violation of the second law of thermodynamics would occur if:
    \begin{choices}
      \correctchoice{a ball lying on the ground started to bounce.}
      \correctchoice{transfer of energy by heat moved energy from an object at low temperature to an object at a higher temperature.}
        \wrongchoice{a motion picture was run backwards through the projector.}
        %\wrongchoice{any of the above occurred.}
        %\correctchoice{(a) or (b) occurred, but not (c).}
    \end{choices}
\end{questionmult}
}

\element{serway-mc}{
\begin{questionmult}{serway-ch22-q44}
    A violation of the second law of thermodynamics would occur if:
    \begin{choices}
      \correctchoice{a ball lying on the ground started to bounce.}
      \correctchoice{transfer of energy by heat moved energy from an object at low temperature to an object at a higher temperature.}
        %% NOTE: C is different
        \wrongchoice{a refrigerator heated the air in the room in which the refrigerator is located.}
        %\wrongchoice{any of the above occurred.}
        %\correctchoice{(a) or (b) occurred, but not (c).}
    \end{choices}
\end{questionmult}
}

\element{serway-mc}{
\begin{questionmult}{serway-ch22-q45}
    The thermal efficiency of a heat engine is given by:
    \begin{choices}
      \correctchoice{$e = \dfrac{W_{eng}}{|Q_h|}$}
      \correctchoice{$e = \dfrac{|Q_h|-|Q_c|}{|Q_h|}$}
        \wrongchoice{$e = 1 - \dfrac{T_h}{T_c}$}
        %\wrongchoice{all of the formulas above.}
        %\correctchoice{only (a) or (b) above.}
    \end{choices}
\end{questionmult}
}

\element{serway-mc}{
\begin{question}{serway-ch22-q46}
    Selena states that $S_f-S_i=nR\ln\left(\dfrac{V_v}{V_i}\right)$ proves that entropy has a definite value at the beginning and end of an adiabatic free expansion. 
    Ron says $S_f-S_i = k_B\ln\left(\dfrac{W_f}{W_i}\right)$,
        where $W$ is the number of microstates of a given macrostate.  
    Which one, if either, is correct?
    \begin{choices}
        \wrongchoice{Only Selena, because entropy can depend only on macroscopic variables.  }
        \wrongchoice{Only Ron, because entropy can depend only on microscopic variables.}
        \wrongchoice{Only Selena, because $\left(\dfrac{V_f}{V_i}\right)=\left(\dfrac{T_f}{T_i}\right)$ in an adibatic free expansion.}
        \wrongchoice{Neither, because we cannot calculate changes in entropy in an adiabatic free expansion.}
      \correctchoice{Both, because entropy, which is macroscopic is a function of microscopic disorder.}
    \end{choices}
\end{question}
}

\element{serway-mc}{
\begin{question}{serway-ch22-q47}
    Which answer below is not a statement of the second law of thermodynamics?
    \begin{choices}
        \wrongchoice{Real processes proceed in a preferred direction.}
        \wrongchoice{Energy does not flow spontaneously by heat from a cold to a hot reservoir.}
        \wrongchoice{The entropy of the universe increases in all natural processes.}
      \correctchoice{In theory, heat engines working in a cycle employ reversible processes.}
        \wrongchoice{You cannot construct a heat engine, operating in a cycle that does nothing but take heat from a reservoir and perform an equal amount of work.}
    \end{choices}
\end{question}
}

\element{serway-mc}{
\begin{question}{serway-ch22-q48}
    The change in entropy, $\Delta S$, is largest in a(n):
    \begin{choices}
        \wrongchoice{constant volume process.}
      \correctchoice{constant pressure process.}
        \wrongchoice{adiabatic process.}
        \wrongchoice{process in which no heat is transferred.}
        \wrongchoice{process in which no work is performed.}
    \end{choices}
\end{question}
}


\endinput


