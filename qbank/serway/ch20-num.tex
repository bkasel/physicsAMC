
%%--------------------------------------------------
%% Serway: Physics for Scientists and Engineers
%%--------------------------------------------------


%% Chapter 20: Heat and the First Law of Thermodynamics
%%--------------------------------------------------


%% Serway Numeric Questions
%%--------------------------------------------------
\element{serway-num}{
\begin{questionmultx}{serway-ch20-q54}
    \SI{100}{\gram} of liquid nitrogen at \SI{77}{\kelvin} is stirred into a beaker containing \SI{200}{\gram} of \SI{5}{\degreeCelsius} water. 
    If the nitrogen leaves the solution as soon as it turns to gas,
        how much water freezes? 
    (The heat of evaporation of nitrogen is \SI{6.09}{\calorie\per\gram} and the heat of fusion of water is \SI{80}{\calorie\per\gram}.)
    %% ANSWER:  none
    \AMCnumericChoices{1.5}{
        vertical=false,
        digits=3,decimals=2,sign=true,Tsign=\hspace{1ex}\Large a,
        borderwidth=0pt,backgroundcol=white,approx=5
    }
    \AMCnumericChoices{2}{
        vertical=false,
        digits=2,decimals=0,sign=true,Tsign=\hspace{1ex}\Large b,
        borderwidth=0pt,backgroundcol=white,approx=5
    }
\end{questionmultx}
}

\element{serway-num}{
\begin{questionmultx}{serway-ch20-q55}
    How much water at \SI{20}{\degreeCelsius} is needed to melt \SI{1}{\kilo\gram} of solid mercury at \SI{-39}{\degreeCelsius}?
    (The heat of fusion of mercury is \SI{2.8}{\calorie\per\gram}).
    %% ANSWER:  \SI{23.4}{\gram}
    \AMCnumericChoices{1.5}{
        vertical=false,
        digits=3,decimals=2,sign=true,Tsign=\hspace{1ex}\Large a,
        borderwidth=0pt,backgroundcol=white,approx=5
    }
    \AMCnumericChoices{2}{
        vertical=false,
        digits=2,decimals=0,sign=true,Tsign=\hspace{1ex}\Large b,
        borderwidth=0pt,backgroundcol=white,approx=5
    }
\end{questionmultx}
}

\element{serway-num}{
\begin{questionmultx}{serway-ch20-q56}
    A styrofoam container used as a picnic cooler contains a block of ice at \SI{0}{\degreeCelsius}.
    If \SI{225}{\gram} of ice melts in \SI{1}{\hour},
        how much heat energy per second is passing through the walls of the container? 
    (The heat of fusion of ice is \SI{3.33e5}{\joule\per\kilo\gram}).
    %% ANSWER:  \SI{20.8}{\joule\per\second}
    \AMCnumericChoices{1.5}{
        vertical=false,
        digits=3,decimals=2,sign=true,Tsign=\hspace{1ex}\Large a,
        borderwidth=0pt,backgroundcol=white,approx=5
    }
    \AMCnumericChoices{2}{
        vertical=false,
        digits=2,decimals=0,sign=true,Tsign=\hspace{1ex}\Large b,
        borderwidth=0pt,backgroundcol=white,approx=5
    }
\end{questionmultx}
}

\element{serway-num}{
\begin{questionmultx}{serway-ch20-q57}
    In braking an automobile, the friction between the brake drums and brake shoes converts the car's kinetic energy into heat. 
    If a \SI{1500}{\kilo\gram} automobile traveling at \SI{30}{\meter\per\second} brakes to a halt, how much does the temperature rise in each of the four \SI{8}{\kilo\gram} brake drums? 
    (The specific heat of each iron brake drum is \SI{448}{\joule\per\kilo\gram\per\degreeCelsius}).  
    %% ANSWER:  \SI{47}{\degreeCelsius}
    \AMCnumericChoices{1.5}{
        vertical=false,
        digits=3,decimals=2,sign=true,Tsign=\hspace{1ex}\Large a,
        borderwidth=0pt,backgroundcol=white,approx=5
    }
    \AMCnumericChoices{2}{
        vertical=false,
        digits=2,decimals=0,sign=true,Tsign=\hspace{1ex}\Large b,
        borderwidth=0pt,backgroundcol=white,approx=5
    }
\end{questionmultx}
}


\endinput


