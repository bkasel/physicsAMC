
%%--------------------------------------------------
%% Serway: Physics for Scientists and Engineers
%%--------------------------------------------------


%% Chapter 19: Temperature
%%--------------------------------------------------


%% Serway Numeric Questions
%%--------------------------------------------------
\element{serway-num}{
\begin{questionmultx}{serway-ch19-q40}
    A gold ring has an inner diameter of \SI{2.168}{\centi\meter} at a temperature of \SI{15.0}{\degreeCelsius}.
    Determine its diameter at \SI{100}{\degreeCelsius}. ($\alpha_{gold}=\SI{1.42e-5}{\per\degreeCelsius}$)
    %% ANSWER:  \SI{2.171}{\centi\meter}
    \AMCnumericChoices{1.5}{
        vertical=false,
        digits=3,decimals=2,sign=true,Tsign=\hspace{1ex}\Large a,
        borderwidth=0pt,backgroundcol=white,approx=5
    }
    \AMCnumericChoices{2}{
        vertical=false,
        digits=2,decimals=0,sign=true,Tsign=\hspace{1ex}\Large b,
        borderwidth=0pt,backgroundcol=white,approx=5
    }
\end{questionmultx}
}

\element{serway-num}{
\begin{questionmultx}{serway-ch19-q41}
    Determine the change in length of a \SI{20}{\meter} railroad track made of steel if the temperature is changed from \SI{-15}{\degreeCelsius} to \SI{+35}{\degreeCelsius}.
    ($\alpha_{steel}=\SI{1.1e-5}{\per\degreeCelsius}$)
    %% ANSWER:  \SI{1.1}{\centi\meter}
    \AMCnumericChoices{1.5}{
        vertical=false,
        digits=3,decimals=2,sign=true,Tsign=\hspace{1ex}\Large a,
        borderwidth=0pt,backgroundcol=white,approx=5
    }
    \AMCnumericChoices{2}{
        vertical=false,
        digits=2,decimals=0,sign=true,Tsign=\hspace{1ex}\Large b,
        borderwidth=0pt,backgroundcol=white,approx=5
    }
\end{questionmultx}
}

\element{serway-num}{
\begin{questionmultx}{serway-ch19-q42}
    At what Fahrenheit temperature are the Kelvin and Fahrenheit temperatures numerically equal?
    %% ANSWER:  \SI{574}{\degree\Fahrenheit} = \SI{574}{\kelvin}
    \AMCnumericChoices{1.5}{
        vertical=false,
        digits=3,decimals=2,sign=true,Tsign=\hspace{1ex}\Large a,
        borderwidth=0pt,backgroundcol=white,approx=5
    }
    \AMCnumericChoices{2}{
        vertical=false,
        digits=2,decimals=0,sign=true,Tsign=\hspace{1ex}\Large b,
        borderwidth=0pt,backgroundcol=white,approx=5
    }
\end{questionmultx}
}

\element{serway-num}{
\begin{questionmultx}{serway-ch19-q43}
    Suppose the ends of a \SI{30}{\meter} long steel beam are rigidly clamped at \SI{0}{\degreeCelsius} to prevent expansion. 
    The beam has a cross-sectional area of 30 cm 2 . 
    What force against the clamps does the beam exert when it is heated to \SI{40}{\degreeCelsius}?
    [$\alpha_{Steel} = \SI{1.1e-5}{\per\degreeCelsius}$, $\gamma_{Steel} = \SI{20e10}{\newton\per\meter\squared}$].
    %% ANSWER:  \SI{2.6e5}{\newton}
    \AMCnumericChoices{1.5}{
        vertical=false,
        digits=3,decimals=2,sign=true,Tsign=\hspace{1ex}\Large a,
        borderwidth=0pt,backgroundcol=white,approx=5
    }
    \AMCnumericChoices{2}{
        vertical=false,
        digits=2,decimals=0,sign=true,Tsign=\hspace{1ex}\Large b,
        borderwidth=0pt,backgroundcol=white,approx=5
    }
\end{questionmultx}
}

\element{serway-num}{
\begin{questionmultx}{serway-ch19-q44}
    The pressure of a substance is directly proportional to its volume when the temperature is held constant and inversely proportional to its temperature when the volume is held constant. 
    Is this substance an ideal gas? 
    Explain why your answer is correct.
    %% ANSWER:  No, since P=kV and P=k'/T
    \AMCnumericChoices{1.5}{
        vertical=false,
        digits=3,decimals=2,sign=true,Tsign=\hspace{1ex}\Large a,
        borderwidth=0pt,backgroundcol=white,approx=5
    }
    \AMCnumericChoices{2}{
        vertical=false,
        digits=2,decimals=0,sign=true,Tsign=\hspace{1ex}\Large b,
        borderwidth=0pt,backgroundcol=white,approx=5
    }
\end{questionmultx}
}


\endinput


