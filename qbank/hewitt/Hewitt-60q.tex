
%% Classification
%%------------------------------

%% Kinematics
%% Q01--Q04

%% Mechanics
%% Q01--Q14

%% Gravity
%% Q15--Q19

%% Atomic
%% Q20

%% Archimedes
%% Q21--Q22

%% Geophysics
%% Q23

%% Thermodynamics
%% Q24--Q26

%% Electricity
%% Q27--Q32

%% Magnetism
%% Q33--Q35

%% Sound
%% Q38--Q43

%% Light
%% Q44--Q51

%% Modern
%% Q52--Q60


%% Hewitt 60 Questions
%%------------------------------
\element{60q}{
\begin{question}{60q-Q01}
    In the absence of air resistance,
        a ball of of mass $m$ is tossed upward to reach a height of \SI{20}{\meter}.
    At the \SI{10}{\meter} position, half way up,
        the net force on the ball is:
    \begin{multicols}{2}
    \begin{choices}
        \wrongchoice{$2mg$.}
      \correctchoice{$mg$.}
        \wrongchoice{$\frac{mg}{2}$.}
        \wrongchoice{$\frac{mg}{4}$.}
    \end{choices}
    \end{multicols}
\end{question}
}

\element{60q}{
\begin{question}{60q-Q02}
    When you drop a ball it accelerates downward at
        \SI{9.8}{\meter\per\second\squared}.
    If you instead throw it downward,
        then its acceleration immediately after leaving your hand,
        assuming no air resistance, is:
    \begin{choices}
      \correctchoice{\SI{9.8}{\meter\per\second\squared}.}
        \wrongchoice{more than \SI{9.8}{\meter\per\second\squared}.}
        \wrongchoice{less than \SI{9.8}{\meter\per\second\squared}.}
        \wrongchoice{cannot say, unless the speed of throw is given.}
    \end{choices}
\end{question}
}

\element{60q}{
\begin{question}{60q-Q03}
    A heavy rock and a light rock in free fall
        (zero air resistance) have the same  acceleration.
    The reason the heavy rock doesn't have a greater acceleration is that the:
    \begin{choices}
        \wrongchoice{force due to gravity is the same on each.}
        \wrongchoice{air resistance is always zero in free fall.}
        \wrongchoice{inertia of both rocks is the same.}
      \correctchoice{ratio of force to mass is the same.}
        \wrongchoice{None of these.}
    \end{choices}
\end{question}
}

\element{60q}{
\begin{question}{60q-Q04}
    A cannonball is fired horizontally at \SI{10}{\meter\per\second} from a cliff.
    Its speed one second after being fired is about:
    \begin{multicols}{2}
    \begin{choices}
        \wrongchoice{\SI{10}{\meter\per\second}.}
      \correctchoice{\SI{14}{\meter\per\second}.}
        \wrongchoice{\SI{16}{\meter\per\second}.}
        \wrongchoice{\SI{20}{\meter\per\second}.}
    \end{choices}
    \end{multicols}
\end{question}
}

\element{60q}{
\begin{question}{60q-Q05}
    Relative to the ground,
        an airplane gains speed when it encounters wind from behind,
        and loses speed when it encounters wind head on.
    When it encounters wind at a right angle to the direction it is pointing,
        its speed relative to the ground below:
    \begin{choices}
      \correctchoice{increases.}
        \wrongchoice{decreases.}
        \wrongchoice{is the same as if there were no wind.}
        \wrongchoice{need more information.}
    \end{choices}
\end{question}
}

\element{60q}{
\begin{question}{60q-Q06}
    A karate chop delivers a force of \SI{3000}{\newton}
        to a board that breaks.
    The force that the board exerts on the hand during this event is:
    \begin{choices}
        \wrongchoice{less than \SI{3000}{\newton}.}
      \correctchoice{\SI{3000}{\newton}.}
        \wrongchoice{greater than \SI{3000}{\newton}.}
        \wrongchoice{need more information.}
    \end{choices}
\end{question}
}

\element{60q}{
\begin{question}{60q-Q07}
    A math book and a physics book are tied together with a length of string.
    With the string taut,
        one book is pushed off the edge of a table.
    As it falls,
        the other book is dragged horizontally across the table surface.
    With no friction,
        acceleration of the books is:
    \begin{choices}
        \wrongchoice{zero.}
        \wrongchoice{$\frac{g}{2}$.}
        \wrongchoice{$g$.}
      \correctchoice{a value between zero and $g$.}
        \wrongchoice{a value that could be greater than $g$.}
    \end{choices}
\end{question}
}

\element{60q}{
\begin{question}{60q-Q08}
    When an increase in speed doubles the momentum of a moving body,
        its kinetic energy:
    \begin{choices}
        \wrongchoice{increases, but less than doubles.}
        \wrongchoice{doubles.}
      \correctchoice{more than doubles.}
        \wrongchoice{depends on factors not stated.}
    \end{choices}
\end{question}
}

\element{60q}{
\begin{question}{60q-Q09}
    When an increase in speed doubles the kinetic energy of a moving body,
        its momentum:
    \begin{choices}
      \correctchoice{increases, but less than doubles.}
        \wrongchoice{doubles.}
        \wrongchoice{more than doubles.}
        \wrongchoice{depends on factors not stated.}
    \end{choices}
\end{question}
}

\element{60q}{
\begin{question}{60q-Q10}
    Big brother and little sister can balance on a seesaw because of balanced:
    \begin{choices}
        \wrongchoice{forces, only.}
      \correctchoice{torques, only.}
        \wrongchoice{energies, only.}
        \wrongchoice{forces, torques and energies.}
    \end{choices}
\end{question}
}

\element{60q}{
\begin{question}{60q-Q11}
    When a spinning system contracts in the absence of an external torque,
        its rotational speed increases and its angular momentum:
    \begin{choices}
        \wrongchoice{decreases.}
        \wrongchoice{increases.}
      \correctchoice{remains unchanged.}
        \wrongchoice{may increase or decrease.}
    \end{choices}
\end{question}
}

\element{60q}{
\begin{question}{60q-Q12}
    Consider a ball rolling down an inclined plane.
    The normal force on the ball (the force perpendicular to the plane).
    \begin{choices}
        \wrongchoice{is $mg$.}
        \wrongchoice{is greater than $mg$, always.}
        \wrongchoice{may be greater or less than $mg$.}
      \correctchoice{is less than $mg$ always.}
    \end{choices}
\end{question}
}

\element{60q}{
\begin{question}{60q-Q13}
    Consider a ball rolling in a horizontal circular path on the inside surface of a cone.
    The normal force on the ball:
    \begin{choices}
        \wrongchoice{is $mg$.}
      \correctchoice{is greater than $mg$, always.}
        \wrongchoice{may be greater or less than $mg$.}
        \wrongchoice{is less than $mg$ always.}
    \end{choices}
\end{question}
}

\element{60q}{
\begin{question}{60q-Q14}
    When a ball at rest hangs by a single vertical string,
        tension in the string is $mg$.
    If the ball is made to move in a horizontal circle so that the string describes a cone,
        string tension:
    \begin{choices}
        \wrongchoice{is $mg$.}
      \correctchoice{is greater than $mg$, always.}
        \wrongchoice{is less than $mg$, always.}
        \wrongchoice{may be greater or less than $mg$ depending on the speed of the ball.}
    \end{choices}
\end{question}
}

\element{60q}{
\begin{question}{60q-Q15}
    Imagine you're standing on the surface of a shrinking planet.
    If it shrinks to one-tenth its original diameter with no change in mass,
        on the shrunken surface you'd weight:
    \begin{choices}
        \wrongchoice{\num{1/100} as much.}
        \wrongchoice{\num{10} times as much.}
      \correctchoice{\num{100} times as much.}
        \wrongchoice{\num{1000} times as much.}
        \wrongchoice{None of these.}
    \end{choices}
    \end{choices}
\end{question}
}

\element{60q}{
\begin{question}{60q-Q16}
    The fact that the Moon always shows its same face to Earth is
        evidence that the Moon rotates about its axis about once per:
    \begin{choices}
        \wrongchoice{day.}
      \correctchoice{month.}
        \wrongchoice{year.}
        \wrongchoice{None of these, for the moon does not rotate about an axis.}
    \end{choices}
\end{question}
}

\element{60q}{
\begin{question}{60q-Q17}
    The Moon is most responsible for Earth's tides.
    Which pulls more strongly on the Earth and its oceans?
    \begin{choices}
        \wrongchoice{Moon}
      \correctchoice{Sun}
        \wrongchoice{Both about equally}
    \end{choices}
\end{question}
}

\element{60q}{
\begin{question}{60q-Q18}
    A spacecraft on its way from Earth to the Moon is pulled
        equally by Earth and Moon when it is:
    \begin{choices}
        \wrongchoice{closer to the Earth's surface.}
      \correctchoice{closer to the Moon's surface.}
        \wrongchoice{half way from Earth to Moon.}
        \wrongchoice{At no point, since Earth always pulls more strongly.}
    \end{choices}
\end{question}
}

\element{60q}{
\begin{question}{60q-Q19}
    Earth satellites such as the Space Shuttle orbit
        at altitude that are above the Earth's:
    \begin{choices}
      \correctchoice{atmosphere.}
        \wrongchoice{gravitational field.}
        \wrongchoice{both atmosphere and gravitational field.}
    \end{choices}
\end{question}
}

\element{60q}{
\begin{question}{60q-Q20}
    The mass of a classical atom comes mostly from
        its \underline{\hspace{8em}}; and its volume
        from its \underline{\hspace{8em}}.
    \begin{choices}
        \wrongchoice{nucleons; nucleons.}
        \wrongchoice{electrons; electrons.}
        \wrongchoice{electrons; nucleons.}
      \correctchoice{nucleons; nucleons.}
    \end{choices}
\end{question}
}

\element{60q}{
\begin{question}{60q-Q21}
    Consider a block of wood floating on water.
    If you push down on the top of the block until
        it's completely submerged, the bouyant force on it:
    \begin{choices}
      \correctchoice{increases.}
        \wrongchoice{decreases.}
        \wrongchoice{remains the same.}
        \wrongchoice{depends on how far beneath the water surface it is pushed.}
    \end{choices}
\end{question}
}

\element{60q}{
\begin{question}{60q-Q22}
    An inflated balloon with a heavy rock tied to it submerges in water.
    As the balloon sinks deeper and deeper,
        the bouyant force acting on it:
    \begin{choices}
        \wrongchoice{increases.}
      \correctchoice{decreases.}
        \wrongchoice{remains largely unchanged.}
        \wrongchoice{need more information.}
    \end{choices}
\end{question}
}

\element{60q}{
\begin{question}{60q-Q23}
    The principal source of the Earth's internal energy is:
    \begin{choices}
        \wrongchoice{tidal friction.}
        \wrongchoice{gravitational pressure.}
      \correctchoice{radioactivity.}
        \wrongchoice{geothermal heat.}
    \end{choices}
\end{question}
}

\element{60q}{
\begin{question}{60q-Q24}
    The surface of Planet Earth loses energy to outer space due mostly to:
    \begin{choices}
        \wrongchoice{conduction.}
        \wrongchoice{convention.}
      \correctchoice{radiation.}
        \wrongchoice{radioactivity.}
    \end{choices}
\end{question}
}

\element{60q}{
\begin{question}{60q-Q25}
    The ``greenhouse gases'' that contribute to global warming absorb:
    \begin{choices}
        \wrongchoice{more visible radiation than infrared.}
      \correctchoice{more infrared radiation than visible.}
        \wrongchoice{visible and infrared about equally.}
        \wrongchoice{very little radiation of any kind.}
    \end{choices}
\end{question}
}

\element{60q}{
\begin{question}{60q-Q26}
    In a mixture of hydrogen, oxygen, and nitrogen gases at a given temperature,
        the molecules having the greatest average speed are those of:
    \begin{choices}
      \correctchoice{hydrogen.}
        \wrongchoice{oxygen.}
        \wrongchoice{nitrogen.}
        \wrongchoice{but all have the same speed on average.}
    \end{choices}
\end{question}
}

\element{60q}{
\begin{question}{60q-Q27}
    The electrical force of attraction between an
        electron and a proton is greater on the:
    \begin{choices}
        \wrongchoice{proton.}
        \wrongchoice{electron.}
      \correctchoice{neither, both are the same.}
    \end{choices}
\end{question}
}

\element{60q}{
\begin{question}{60q-Q28}
    Immediately after two separated charged particles are released from rest,
        both increase in speed.
    The particles therefore have:
    \begin{choices}
        \wrongchoice{the same sign of charge.}
        \wrongchoice{opposite signs of charge.}
      \correctchoice{either the same or opposite signs of charge.}
        \wrongchoice{Need more information.}
    \end{choices}
\end{question}
}

\element{60q}{
\begin{question}{60q-Q29}
    Compared with the current in the white-hot filament of a common lamp,
        the current in the connecting wire is:
    \begin{choices}
        \wrongchoice{less.}
        \wrongchoice{more.}
      \correctchoice{the same.}
        \wrongchoice{Need more information.}
    \end{choices}
\end{question}
}


\element{60q}{
\begin{question}{60q-Q30}
    As more lamps are connected in a series circuit,
        the current in the power source:
    \begin{choices}
        \wrongchoice{increases.}
      \correctchoice{decreases.}
        \wrongchoice{remains much the same.}
        \wrongchoice{Need more information.}
    \end{choices}
\end{question}
}

\element{60q}{
\begin{question}{60q-Q31}
    As more lamps are connected in parallel in a circuit,  
        the current in the power source:
    \begin{choices}
      \correctchoice{increases.}
        \wrongchoice{decreases.}
        \wrongchoice{remains much the same.}
        \wrongchoice{Need more information.}
    \end{choices}
\end{question}
}

\element{60q}{
\begin{question}{60q-Q32}
    A capacitor loses half its charge every second.
    If after five seconds its charge is $q$,
        what was its initial charge?
    \begin{multicols}{2}
    \begin{choices}
        \wrongchoice{$4q$}
        \wrongchoice{$8q$}
        \wrongchoice{$16q$}
      \correctchoice{$32q$}
        \wrongchoice{None of these}
    \end{choices}
    \end{multicols}
\end{question}
}

\element{60q}{
\begin{question}{60q-Q33}
    The magnetic force on a moving charged particle can change the particle's:
    \begin{choices}
        \wrongchoice{speed.}
      \correctchoice{direction.}
        \wrongchoice{both of speed and direction.}
        \wrongchoice{neither speed nor direction.}
    \end{choices}
\end{question}
}

\element{60q}{
\begin{question}{60q-Q34}
    A step-up transformer in an electrical circuit can step up:
    \begin{choices}
      \correctchoice{voltage.}
        \wrongchoice{energy.}
        \wrongchoice{both voltage and energy.}
        \wrongchoice{neither voltage nor energy.}
    \end{choices}
\end{question}
}

\element{60q}{
\begin{question}{60q-Q35}
    The mutual induction of electric and magnetic fields can produce:
    \begin{choices}
      \correctchoice{light.}
        \wrongchoice{energy.}
        \wrongchoice{both energy and light.}
        \wrongchoice{neither energy nor light.}
    \end{choices}
\end{question}
}

\element{60q}{
\begin{question}{60q-Q36}
    All of the following are electromagnetic waves \emph{except}:
    \begin{choices}
        \wrongchoice{radio waves.}
        \wrongchoice{microwaves.}
        \wrongchoice{light waves.}
        \wrongchoice{x-rays.}
      \correctchoice{none is outside the family; all are electromagnetic waves.}
    \end{choices}
\end{question}
}

\element{60q}{
\begin{question}{60q-Q37}
    You swing to and fro on a playground swing.
    If you stand rather than sit,
        the time for a to-and-fro swing is:
    \begin{choices}
        \wrongchoice{lengthened.}
      \correctchoice{shortened.}
        \wrongchoice{unchanged.}
    \end{choices}
\end{question}
}

\element{60q}{
\begin{question}{60q-Q38}
    Compared with the sound you hear from the siren of a stationary fire engine,
        the sound you hear when it approaches you has an increased:
    \begin{choices}
        \wrongchoice{speed.}
      \correctchoice{frequency.}
        \wrongchoice{Both speed and frequency.}
        \wrongchoice{Neither speed nor frequency.}
    \end{choices}
\end{question}
}

\element{60q}{
\begin{question}{60q-Q39}
    During the time an aircraft produces a sonic boom,
        the aircraft is:
    \begin{choices}
        \wrongchoice{breaking the sound barrier.}
        \wrongghoice{pulling out the subsonic drive.}
      \correctchoice{flying faster than sound.}
        \wrongchoice{Each of these produces a sonic boom.}
    \end{choices}
\end{question}
}

\element{60q}{
\begin{question}{60q-Q40}
    The phenomenon of interference ocurrs for:
    \begin{choices}
        \wrongchoice{sound waves, only.}
        \wrongchoice{light waves, only.}
      \correctchoice{both sound and light waves.}
        \wrongchoice{Neither sound nor light waves.}
    \end{choices}
\end{question}
}

\element{60q}{
\begin{question}{60q-Q41}
    The speed of sound in air depends on:
    \begin{choices}
        \wrongchoice{frequency, only.}
        \wrongchoice{wavelength, only.}
      \correctchoice{air temperature, only.}
        \wrongchoice{frequency, wavelength and air temperature.}
        \wrongchoice{Neither frequency, wavelength nor air temperature.}
    \end{choices}
\end{question}
}

\element{60q}{
\begin{question}{60q-Q42}
    Your friend states that under all conditions, any radio
        wave travels faster than any sound wave.
    You:
    \begin{choices}
      \correctchoice{agree with your friend.}
        \wrongchoice{disagree with your friend.}
    \end{choices}
\end{question}
}

\element{60q}{
\begin{question}{60q-Q43}
    The phenomenon of beats results from sound:
    \begin{choices}
        \wrongchoice{reflections, only.}
        \wrongchoice{refractions, only.}
      \correctchoice{interference, only.}
        \wrongchoice{reflection, refraction and interference.}
        \wrongchoice{Neither Reflection, refractoin nor interference.}
    \end{choices}
\end{question}
}

\element{60q}{
\begin{question}{60q-Q44}
    To view your full-face image in a steamy mirror,
        compared to the height of your face,
        the minimum height of the patch to wipe away is:
    \begin{choices}
        \wrongchoice{one-quarter.}
      \correctchoice{one-half.}
        \wrongchoice{the same.}
        \wrongchoice{dependent on your distance from the mirrow.}
    \end{choices}
\end{question}
}

\element{60q}{
\begin{question}{60q-Q45}
    Light reflecting from a smooth surface undergoes a change in:
    \begin{choices}
        %% NOTE: consider questionmult
        \wrongchoice{frequency, only.}
        \wrongchoice{speed, only.}
        \wrongchoice{wavelength, only.}
        \wrongchoice{Frequency, speed and wavelength.}
      \correctchoice{Neither freuquency, speed nor wavelength.}
    \end{choices}
\end{question}
}

\element{60q}{
\begin{question}{60q-Q46}
    Which of these changes when light refracts in passing
        from one medium to another?
    \begin{choices}
        \wrongchoice{speed.}
        \wrongchoice{wavelength.}
      \correctchoice{both speed and wavelength.}
        \wrongchoice{neither speed nor wavelength.}
    \end{choices}
\end{question}
}

\element{60q}{
\begin{question}{60q-Q47}
    When white light passes through a prism,
        green light is bent more than:
    \begin{choices}
        \wrongchoice{blue light.}
        \wrongchoice{violet light.}
      \correctchoice{red light.}
        %\wrongchoice{two of these choices}
        \wrongchoice{both blue and violet light.}
        \wrongchoice{both red and blue light}
        \wrongchoice{neither blue, violet nor red light.}
    \end{choices}
\end{question}
}

\element{60q}{
\begin{question}{60q-Q48}
    When you look at the red petals of a rose,
        the color light you're seeing is:
    \begin{choices}
      \correctchoice{red.}
        \wrongchoice{green.}
        \wrongchoice{white minus red.}
        \wrongchoice{a mixture of green and yellow.}
        \wrongchoice{cyan.}
    \end{choices}
\end{question}
}

\element{60q}{
\begin{question}{60q-Q49}
    When the color is seen on your TV screen,
        the phosphors being activated on the screen are:
    \begin{choices}
        \wrongchoice{mainly yellow.}
        \wrongchoice{blue and red.}
        \wrongchoice{green and yellow.}
      \correctchoice{red and green.}
    \end{choices}
\end{question}
}

\element{60q}{
\begin{question}{60q-Q50}
    The red glow in the neon tube of an advertising sign is a result of:
    \begin{choices}
        \wrongchoice{flourescence.}
        \wrongchoice{incandescence.}
        \wrongchoice{polarization.}
        \wrongchoice{coherence.}
      \correctchoice{de-excitation.}
    \end{choices}
\end{question}
}

\element{60q}{
\begin{question}{60q-Q51}
    Polarization is a property of:
    \begin{choices}
      \correctchoice{transverse waves.}
        \wrongchoice{longitudinal waves.}
        \wrongchoice{both transverse and longitudinal waves.}
        \wrongchoice{neither transverse nor longitudinal waves.}
    \end{choices}
\end{question}
}

\element{60q}{
\begin{question}{60q-Q52}
    Astrophysicists are able to identify the elements in
        the outer layers of a star by studying its:
    \begin{choices}
        \wrongchoice{doppler effect.}
        \wrongchoice{molecular structure.}
        \wrongchoice{temperature.}
      \correctchoice{spectrum.}
    \end{choices}
\end{question}
}

\element{60q}{
\begin{question}{60q-Q53}
    If the Sun collapsed to become a black hole,
        planet Earth would:
    \begin{choices}
      \correctchoice{continue in its present orbit.}
        \wrongchoice{fly off in a tangent path.}
        \wrongchoice{likely be sucked into the black hole.}
        \wrongchoice{be pulled apart by tidal forces.}
        \wrongchoice{both likely sucked into the black hole and pulled apart by tidal forces.}
    \end{choices}
\end{question}
}

\element{60q}{
\begin{question}{60q-Q54}
    Any atom that emits an alpha particle or beta particle:
    \begin{choices}
      \correctchoice{becomes an atom of a different element, always.}
        \wrongchoice{may become an atom of a different element.}
        \wrongchoice{becomes a different isotope of the same element.}
        \wrongchoice{increases its mass.}
    \end{choices}
\end{question}
}

\element{60q}{
\begin{question}{60q-Q55}
    Suppose the number of neutrons in a reactor that is starting up doubles each minute,
        reaching one billion neutrons in 10 minutes.
    When did the number of neutrons reach half a billion?
    \begin{choices}
        \wrongchoice{1 minute.}
        \wrongchoice{2 minutes.}
        \wrongchoice{5 minutes.}
        \wrongchoice{9 minutes.}
      \correctchoice{neither 1,2,5 nor 9 minutes.}
    \end{choices}
\end{question}
}

\element{60q}{
\begin{question}{60q-Q56}
    When uranium nucleus undergoes fission,
        the energy released is primarily in the form of:
    \begin{choices}
        \wrongchoice{gamma radiation.}
      \correctchoice{kinetic energy of fission fragments.}
        \wrongchoice{kinetic energy of ejected neutrons.}
        \wrongchoice{gamma radiation and kinetic energy of both fission fragments and ejected neutrons.}
    \end{choices}
\end{question}
}

\element{60q}{
\begin{question}{60q-Q57}
    When a fusion reaction converts a pair of hydrogen isotopes to an alpha particle and a neutron,
        most of the energy released is in the form of:
    \begin{choices}
        \wrongchoice{gamma radiation.}
        \wrongchoice{kinetic energy of the alpha particle.}
      \correctchoice{kinetic energy of the neutron.}
        \wrongchoice{gamma radiation, and the kinetic energy of the alpha particle and the neutron are all equal.}
    \end{choices}
\end{question}
}

\element{60q}{
\begin{question}{60q-Q58}
    Because there is an upper limit on the speed of particles,
        there is also an upper limit on:
    \begin{choices}
        \wrongchoice{momentum.}
        \wrongchoice{kinetic energy.}
        \wrongchoice{temperature.}
      %% NOTE: if convert to questionmult
      %\correctchoice{velocity.}
        \wrongchoice{momentum, kinetic energy and temperature.}
      \correctchoice{neither momentum, kinetic energy nor temperature.}
    \end{choices}
\end{question}
}

\element{60q}{
\begin{question}{60q-Q59}
    Relativistic equations for time dilation, length contraction,
        and relativistic momentum and energy hold true at:
    \begin{choices}
        \wrongchoice{only at speeds near that of light.}
        \wrongchoice{only at everyday low speeds.}
      \correctchoice{all speeds.}
        \wrongchoice{only approximately.}
    \end{choices}
\end{question}
}

\element{60q}{
\begin{question}{60q-Q60}
    The equation $E=mc^2$ indicates that energy:
    \begin{choices}
        \wrongchoice{equals mass moving at the speed of light squared.}
        \wrongchoice{equals moving mass.}
        \wrongchoice{is fundamentally different than mass.}
        \wrongchoice{and mass are closely related.}
    \end{choices}
\end{question}
}

\endinput

