
%%--------------------------------------------------
%% Holt: Multiple Choice Questions
%%--------------------------------------------------


%% Chapter 07: Circular Motion
%%--------------------------------------------------


%% Holt Multiple Choice Questions
%%--------------------------------------------------
\element{holt-mc}{
\begin{question}{holt-ch07-Q01}
    An object moves in a circle at a constant speed.
    Which of the following is \emph{not} true of the object?
    \begin{choices}
        \wrongchoice{Its centripetal acceleration points toward the center of the circle}
        \wrongchoice{Its tangential speed is constant}
      \correctchoice{Its velocity is constant}
        \wrongchoice{A centripetal force acts on the object}
    \end{choices}
\end{question}
}

\element{holt-mc}{
\begin{question}{holt-ch07-Q02}
    A car traveling at \SI{15}{\meter\per\second} on a flat surface turns in a circle with a radius of \SI{25}{\meter}.
    %% Begin Question
    What is the centripetal acceleration of the car?
    \begin{multicols}{2}
    \begin{choices}
        \wrongchoice{\SI{2.4e-2}{\meter\per\second\squared}}
        \wrongchoice{\SI{0.60}{\meter\per\second\squared}}
      \correctchoice{\SI{9.0}{\meter\per\second\squared}}
        \wrongchoice{zero}
    \end{choices}
    \end{multicols}
\end{question}
}

\element{holt-mc}{
\begin{question}{holt-ch07-Q03}
    A car traveling at \SI{15}{\meter\per\second} on a flat surface turns in a circle with a radius of \SI{25}{\meter}.
    %% Begin Question
    What is the most direct cause of the car's centripetal acceleration?
    \begin{choices}
        \wrongchoice{The torque on the steering wheel}
        \wrongchoice{The torque on the tires on the road}
      \correctchoice{The force of friction between the tires and the road}
        \wrongchoice{The normal force between the tires and the road}
    \end{choices}
\end{question}
}

\element{holt-mc}{
\begin{question}{holt-ch07-Q04}
    Earth ($m=\SI{5.7e24}{\kilo\gram}$) orbits the sun ($m=\SI{1.99e30}{\kilo\gram}$)
        at a mean distance of \SI{1.50e11}{\meter}.
    What is the gravitational force of the sun on Earth?
    ($G=\SI{6.673e-11}{\newton\per\meter\squared\per\kilo\gram\squared}$)
    %% NOTE: this is solvable without G, both mass and acceleration are known!!
    \begin{multicols}{2}
    \begin{choices}
        \wrongchoice{\SI{5.29e32}{\newton}}
      \correctchoice{\SI{3.52e22}{\newton}}
        \wrongchoice{\SI{5.90e-2}{\newton}}
        \wrongchoice{\SI{1.77e-8}{\newton}}
    \end{choices}
    \end{multicols}
\end{question}
}

\element{holt-mc}{
\begin{question}{holt-ch07-Q05}
    Which of the following is a correct interpretation of the expression
    \begin{equation}
        a_g = g = G\frac{m_E}{r^2}\,?
    \end{equation}
    \begin{choices}[o]
        \wrongchoice{Gravitational field strength changes with an object's distance from Earth.}
        \wrongchoice{Free-fall acceleration changes with an objects distance from Earth.}
        \wrongchoice{Free-fall acceleration is independent of the falling object's mass.}
      \correctchoice{All of the above are correct interpretations}
    \end{choices}
\end{question}
}

\element{holt-mc}{
\begin{question}{holt-ch07-Q06}
    what data do you need to calculate the orbital speed of a satellite?
    \begin{choices}
        \wrongchoice{mass of satellite, mass of planet, radius of orbit}
        \wrongchoice{mass of satellite, radius of planet, area of orbit}
        \wrongchoice{mass of satellite and radius of orbit only}
      \correctchoice{mass of planet and radius of orbit only}
    \end{choices}
\end{question}
}

\element{holt-mc}{
\begin{question}{holt-ch07-Q07}
    Which of the following choices correctly describes the orbital relationship between Earth and the sun?
    \begin{choices}
        \wrongchoice{The sun orbits Earth in a perfect circle}
        \wrongchoice{Earth orbits the sun in a perfect circle}
        \wrongchoice{The sun orbits Earth in an ellipse, with Earth at one focus.}
      \correctchoice{Earth orbits the sun in an ellipse, with the sun at one focus.}
    \end{choices}
\end{question}
}

\newcommand{\holtChSevenQEight}{
\begin{tikzpicture}
    %% NOTE:
\end{tikzpicture}
}

\element{holt-mc}{
\begin{question}{holt-ch07-Q08}
    %% NOTE: Provide description of picture
    \begin{center}
        \holtChSevenQEight
    \end{center}
    Three forces acting on a wheel below have equal magnitudes.
    Which force will produce the greatest torque on the wheel?
    \begin{multicols}{2}
    \begin{choices}
      \correctchoice{$\mathbf{F_1}$}
        \wrongchoice{$\mathbf{F_2}$}
        \wrongchoice{$\mathbf{F_3}$}
        \wrongchoice{Each force will produce the same torque}
    \end{choices}
    \end{multicols}
\end{question}
}

\element{holt-mc}{
\begin{question}{holt-ch07-Q09}
    %% NOTE: Provide description of picture
    \begin{center}
        \holtChSevenQEight
    \end{center}
    If each force is \SI{6.0}{\newton},
        the angle between $\mathbf{F_1}$ and $\mathbf{F_2}$ is \ang{60},
        and the radius of the wheel is \SI{1.0}{\meter},
        what is the resultant torque on the wheel?
    \begin{multicols}{2}
    \begin{choices}
        \wrongchoice{\SI{-18}{\newton\meter}}
        \wrongchoice{\SI{-9.0}{\newton\meter}}
      \correctchoice{\SI{9.0}{\newton\meter}}
        \wrongchoice{\SI{18}{\newton\meter}}
    \end{choices}
    \end{multicols}
\end{question}
}

\element{holt-mc}{
\begin{question}{holt-ch07-Q10}
    A force of \SI{75}{\newton} is applied to a lever.
    This force lifts a load weighing \SI{225}{\newton}.
    What is the mechanical advantage of the lever?
    \begin{multicols}{4}
    \begin{choices}
        \wrongchoice{\num{1/3}}
      \correctchoice{\num{3}}
        \wrongchoice{\num{150}}
        \wrongchoice{\num{300}}
    \end{choices}
    \end{multicols}
\end{question}
}

\element{holt-mc}{
\begin{question}{holt-ch07-Q11}
    A pulley system has an efficiency of \num{87.5} percent.
    How much work must you do to lift a desk weighing
        \SI{1320}{\newton} to a height of \SI{1.50}{\meter}?
    \begin{multicols}{2}
    \begin{choices}
        \wrongchoice{\SI{1510}{\joule}}
        \wrongchoice{\SI{1730}{\joule}}
        \wrongchoice{\SI{1980}{\joule}}
      \correctchoice{\SI{2260}{\joule}}
    \end{choices}
    \end{multicols}
\end{question}
}

\element{holt-mc}{
\begin{question}{holt-ch07-Q12}
    Which of the following statements is correct?
    \begin{choices}
        \wrongchoice{Mass and weight both vary with location.}
        \wrongchoice{Mass varies with location, but weight does not.}
      \correctchoice{Weight varies with location, but mass does not.}
        \wrongchoice{Neither mass nor weight varies with location.}
    \end{choices}
\end{question}
}

\element{holt-mc}{
\begin{question}{holt-ch07-Q13}
    Which astronomer discovered that planets travel in elliptical rather than circular orbits?
    \begin{choices}
      \correctchoice{Johannes Kepler}
        \wrongchoice{Nicolaus Copernicus}
        \wrongchoice{Tycho Brahe}
        \wrongchoice{Claudius Ptolemy}
    \end{choices}
\end{question}
}


\endinput

