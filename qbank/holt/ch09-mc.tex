
%%--------------------------------------------------
%% Holt: Multiple Choice Questions
%%--------------------------------------------------


%% Chapter 09: Thermodynamics
%%--------------------------------------------------


%% Holt Multiple Choice Questions
%%--------------------------------------------------
\element{holt-mc}{
\begin{question}{holt-ch09-Q01}
    What must be true about two given objects for energy to be transferred as heat between them?
    \begin{choices}
        \wrongchoice{The objects must be large}
        \wrongchoice{The objects must be hot}
        \wrongchoice{The objects must contain a large amount of energy}
        \wrongchoice{The objects must have different temperatures}
    \end{choices}
\end{question}
}

\element{holt-mc}{
\begin{question}{holt-ch09-Q02}
    A metal spoon is placed in one of two identical cups of hot coffee.
    Why does the cup with the spoon have a lower temperature after a few minutes?
    \begin{choices}
        \wrongchoice{Energy is removed from the coffee mostly by conduction through the spoon.}
        \wrongchoice{Energy is removed from the coffee mostly by convection through the spoon.}
        \wrongchoice{Energy is removed from the coffee mostly by radiation through the spoon.}
        \wrongchoice{The  metal in the spoon has an extremely large specific heat.}
    \end{choices}
\end{question}
}

\element{holt-mc}{
\begin{question}{holt-ch09-Q03}
    The boiling point of liquid hydrogen is \SI{-252.7}{\degreeCelsius}.
    What is the value of this temperature on the Fahrenheit scale?
    \begin{multicols}{2}
    \begin{choices}
        \wrongchoice{\SI{20.28}{\degree\Fahrenheit}}
        \wrongchoice{\SI{-220.87}{\degree\Fahrenheit}}
        \wrongchoice{\SI{-423.2}{\degree\Fahrenheit}}
        \wrongchoice{\SI{0}{\degree\Fahrenheit}}
    \end{choices}
    \end{multicols}
\end{question}
}

\element{holt-mc}{
\begin{question}{holt-ch09-Q04}
    The boiling point of liquid hydrogen is \SI{-252.7}{\degreeCelsius}.
    What is the value of this temperature in kelvin?
    \begin{multicols}{2}
    \begin{choices}
        \wrongchoice{\SI{20.28}{\kelvin}}
        \wrongchoice{\SI{-220.87}{\kelvin}}
        \wrongchoice{\SI{-423.2}{\kelvin}}
        \wrongchoice{\SI{0}{\kelvin}}
    \end{choices}
    \end{multicols}
\end{question}
}

\element{holt-mc}{
\begin{question}{holt-ch09-Q05}
    A cup of hot chocolate with a temperature of \SI{40}{\degreeCelsius} is placed inside a refrigerator at \SI{5}{\degreeCelsius}.
    An identicla cup of hot chocolate at \SI{90}{\degreeCelsius} is placed on a table in a room at \SI{25}{\degreeCelsius}.
    A third identical cup of hot chocolate at \SI{80}{\degreeCelsius} is placed on an outdoor table, where the surrounding air has a temperature of \SI{0}{\degreeCelsius}.
    For which of the three cups has the most energy been transferred as heat when equilibrium has been reached?
    \begin{choices}
        \wrongchoice{The first cup has the largest energy transfer}
        \wrongchoice{The second cup has the largest energy transfer}
        \wrongchoice{The third cup has the largest energy transfer}
        \wrongchoice{The same amount of energy is transferred as heat for all three cups}
    \end{choices}
\end{question}
}

\element{holt-mc}{
\begin{question}{holt-ch09-Q06}
    What data are requires in order to determine specific heat capacity of an unknown substance by means of calorimetry?
    \begin{choices}
        \wrongchoice{$c_{p,water}$, $T_{water}$, $T_{substance}$, $T_{final}$, $V_{water}$, $V_{substance}$}
        \wrongchoice{$c_{p,substance}$, $T_{water}$, $T_{substance}$, $T_{final}$, $m_{water}$}
        \wrongchoice{$c_{p,substance}$, $T_{substance}$, $m_{water}$, $m_{substance}$}
        \wrongchoice{$c_{p,substance}$, $T_{water}$, $T_{substance}$, $m_{water}$, $m_{substance}$}
    \end{choices}
\end{question}
}

\element{holt-mc}{
\begin{question}{holt-ch09-Q07}
    During a cold spell, Florida orange growers often spray a mist of water over their trees during the night.
    Why is this done?
    \begin{choices}
        \wrongchoice{The large latent heat of vaporization for water keeps the trees from freezing}
        \wrongchoice{The large latent heat of fusion for water prevents it and thus the tree from freezing}
        \wrongchoice{The small latent heat of fusion for water prevents it and thus the tree from freezing}
        \wrongchoice{The small latent heat capacity of water makes the water a good insulator}
    \end{choices}
\end{question}
}

\element{holt-mc}{
\begin{question}{holt-ch09-Q08}
    The graph shows the change in temperature of a \SI{23}{\gram} sample of a substance as energy is added to the substance as heat.
    \begin{center}
    \begin{tikzpicture}
    \end{tikzpicture}
    \end{center}
    What is the specific heat capacity of the liquid?
    \begin{multicols}{2}
    \begin{choices}
        \wrongchoice{\SI{4.4e5}{\joule\per\kilo\gram\per\degreeCelsius}}
        \wrongchoice{\SI{4.0e3}{\joule\per\kilo\gram\per\degreeCelsius}}
        \wrongchoice{\SI{5.0e2}{\joule\per\kilo\gram\per\degreeCelsius}}
        \wrongchoice{\SI{1.1e3}{\joule\per\kilo\gram\per\degreeCelsius}}
    \end{choices}
    \end{multicols}
\end{question}
}

\element{holt-mc}{
\begin{question}{holt-ch09-Q09}
    The graph shows the change in temperature of a \SI{23}{\gram} sample of a substance as energy is added to the substance as heat.
    \begin{center}
    \begin{tikzpicture}
    \end{tikzpicture}
    \end{center}
    What is the latent heat of fusion of the liquid?
    \begin{multicols}{2}
    \begin{choices}
        \wrongchoice{\SI{4.4e5}{\joule\per\kilo\gram}}
        \wrongchoice{\SI{4.0e3}{\joule\per\kilo\gram\per\degreeCelsius}}
        \wrongchoice{\SI{5.0e2}{\joule}}
        \wrongchoice{\SI{1.1e3}{\joule\per\kilo\gram}}
    \end{choices}
    \end{multicols}
\end{question}
}

\element{holt-mc}{
\begin{question}{holt-ch09-Q10}
    The graph shows the change in temperature of a \SI{23}{\gram} sample of a substance as energy is added to the substance as heat.
    \begin{center}
    \begin{tikzpicture}
    \end{tikzpicture}
    \end{center}
    What is the specific heat capacity of the solid?
    \begin{multicols}{2}
    \begin{choices}
        \wrongchoice{\SI{1.85e3}{\joule\per\kilo\gram\per\degreeCelsius}}
        \wrongchoice{\SI{4.0e2}{\joule\per\kilo\gram\per\degreeCelsius}}
        \wrongchoice{\SI{5.0e2}{\joule\per\kilo\gram\per\degreeCelsius}}
        \wrongchoice{\SI{1.1e3}{\joule\per\kilo\gram\per\degreeCelsius}}
    \end{choices}
    \end{multicols}
\end{question}
}


\endinput

