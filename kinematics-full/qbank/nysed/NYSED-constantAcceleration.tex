
%% Constant Accleration Questions
%% NYSED Physics Regents Examination
%%--------------------------------------------------

%% this section contains 59 problems


%% Section June2015
%%--------------------
\element{nysed}{
\begin{question}{June2015-Q40}
    A car, initially traveling at \SI{15}{\meter\per\second} north,
        accelerates to \SI{25}{\meter\per\second} north in \SI{4.0}{\second}.
    The magnitude of the average acceleration is:
    \begin{multicols}{2}
    \begin{choices}
      \correctchoice{\SI{2.5}{\meter\per\second\squared}}
        \wrongchoice{\SI{6.3}{\meter\per\second\squared}}
        \wrongchoice{\SI{10.}{\meter\per\second\squared}}
        \wrongchoice{\SI{20.}{\meter\per\second\squared}}
    \end{choices}
    \end{multicols}
\end{question}
}


%% Section June2014
%%--------------------
\element{nysed}{
\begin{question}{June2014-Q02}
    What is the final speed of an object that starts from rest and accelerates uniformly at \SI{4.0}{\meter\per\second\squared} over a distance of \SI{8.0}{\meter}?
    \begin{multicols}{2}
    \begin{choices}
      \correctchoice{\SI{8.0}{\meter\per\second}}
        \wrongchoice{\SI{16}{\meter\per\second}}
        \wrongchoice{\SI{32}{\meter\per\second}}
        \wrongchoice{\SI{64}{\meter\per\second}}
    \end{choices}
    \end{multicols}
\end{question}
}

\element{nysed}{
\begin{question}{June2014-Q07}
    A truck, initially traveling at a speed of \SI{22}{\meter\per\second},
        increases speed at a constant rate of \SI{2.4}{\meter\per\second\squared} for \SI{3.2}{\second}.
    What is the total distance traveled by the truck during this \SI{3.2}{\second} time interval?
    \begin{multicols}{2}
    \begin{choices}
      \correctchoice{\SI{83}{\meter}}
        \wrongchoice{\SI{70.}{\meter}}
        \wrongchoice{\SI{12}{\meter}}
        \wrongchoice{\SI{58}{\meter}}
    \end{choices}
    \end{multicols}
\end{question}
}


%% Section June2013
%%--------------------
\element{nysed}{
\begin{question}{June2013-Q03}
    A car traveling west in a straight line on a highway decreases its speed from \SI{30.0}{\meter\per\second} to \SI{23.0}{\meter\per\second} in \SI{2.00}{\second}.
    The car's average acceleration during this time interval is:
    \begin{multicols}{2}
    \begin{choices}
      \correctchoice{\SI{3.5}{\meter\per\second\squared} east}
        \wrongchoice{\SI{3.5}{\meter\per\second\squared} west}
        \wrongchoice{\SI{13}{\meter\per\second\squared} east}
        \wrongchoice{\SI{13}{\meter\per\second\squared} west}
    \end{choices}
    \end{multicols}
\end{question}
}

\element{nysed}{
\begin{question}{June2013-Q04}
    In a race, a runner traveled \SI{12}{\meter} in \SI{4.0}{\second} as she accelerated uniformly from rest.
    The magnitude of the acceleration of the runner was:
    \begin{multicols}{2}
    \begin{choices}
        \wrongchoice{\SI{0.25}{\meter\per\second\squared}}
      \correctchoice{\SI{1.5}{\meter\per\second\squared}}
        \wrongchoice{\SI{3.0}{\meter\per\second\squared}}
        \wrongchoice{\SI{48}{\meter\per\second\squared}}
    \end{choices}
    \end{multicols}
\end{question}
}


%% Section June2012
%%--------------------
\element{nysed}{
\begin{question}{June2012-Q06}
    A car, initially traveling east with a speed of \SI{5.0}{\meter\per\second},
        is accelerated uniformly at \SI{2.0}{\meter\per\second\squared} east for \SI{10}{\second} along a straight line.
    During this \SI{10}{\second} interval the car travels a total distance of:
    \begin{multicols}{2}
    \begin{choices}
        \wrongchoice{\SI{50}{\meter}}
        \wrongchoice{\SI{60}{\meter}}
        \wrongchoice{\SI{1.0e2}{\meter}}
      \correctchoice{\SI{1.5e2}{\meter}}
    \end{choices}
    \end{multicols}
\end{question}
}

\element{nysed}{
\begin{question}{June2012-Q08}
    A child riding a bicycle at \SI{15}{\meter\per\second} accelerates at \SI{-3.0}{\meter\per\second\squared} for \SI{4.0}{\second}.
    What is the child's speed at the end of this \SI{4.0}{\second} interval?
    \begin{multicols}{2}
    \begin{choices}
        \wrongchoice{\SI{12}{\meter\per\second}}
        \wrongchoice{\SI{27}{\meter\per\second}}
      \correctchoice{\SI{3.0}{\meter\per\second}}
        \wrongchoice{\SI{7.0}{\meter\per\second}}
    \end{choices}
    \end{multicols}
\end{question}
}


%% Section June2011
%%--------------------
\element{nysed}{
\begin{question}{June2011-Q02}
    If a car accelerates uniformly from rest to \SI{15}{\meter\per\second} over a distance of \SI{100}{\meter},
        the magnitude of a car's acceleration is:
    \begin{multicols}{2}
    \begin{choices}
        \wrongchoice{\SI{0.15}{\meter\per\second\squared}}
      \correctchoice{\SI{1.1}{\meter\per\second\squared}}
        \wrongchoice{\SI{2.3}{\meter\per\second\squared}}
        \wrongchoice{\SI{6.7}{\meter\per\second\squared}}
    \end{choices}
    \end{multicols}
\end{question}
}

\element{nysed}{
\begin{question}{June2011-Q03}
    An object accelerates uniformly from \SI{3.0}{\meter\per\second} east to \SI{8.0}{\meter\per\second} east in \SI{2.0}{\second}.
    What is the magnitude of the acceleration of the object?
    \begin{multicols}{2}
    \begin{choices}
      \correctchoice{\SI{2.5}{\meter\per\second\squared}}
        \wrongchoice{\SI{5.0}{\meter\per\second\squared}}
        \wrongchoice{\SI{5.5}{\meter\per\second\squared}}
        \wrongchoice{\SI{11}{\meter\per\second\squared}}
    \end{choices}
    \end{multicols}
\end{question}
}

\element{nysed}{
\begin{question}{June2011-Q04}
    A rock is dropped from a bridge.
    What happens to the magnitude of the acceleration and the speed of the rock as it falls?
    [Neglect friction.]
    \begin{choices}
        \wrongchoice{Both acceleration and speed increase.}
        \wrongchoice{Both acceleration and speed remain the same.}
        \wrongchoice{Acceleration increases and speed decreases.}
      \correctchoice{Acceleration remains the same and speed increases.}
    \end{choices}
\end{question}
}

\element{nysed}{
\begin{question}{June2011-Q05}
    A soccer ball is kicked on a level field has an initial vertical velocity component of \SI{15.0}{\meter\per\second}.
    Assuming the ball lands at the same height from which it was kicked,
        what is the total time the ball is in the air?
    [Neglect friction.]
    \begin{multicols}{2}
    \begin{choices}
        \wrongchoice{\SI{0.654}{\second}}
        \wrongchoice{\SI{1.53}{\second}}
      \correctchoice{\SI{3.06}{\second}}
        \wrongchoice{\SI{6.12}{\second}}
    \end{choices}
    \end{multicols}
\end{question}
}

\element{nysed}{
\begin{question}{June2011-Q37}
    The graph below shows the relationship between the speed and elapsed time for an object falling freely from rest near the surface of a planet.
    \begin{center}
    \begin{tikzpicture}
        \begin{axis}[
            axis y line=left,
            axis x line=bottom,
            axis line style={->},
            xlabel={Time},
            x unit=\si{\second},
            xtick={0.0,1.0,2.0,3.0,4.0},
            ylabel={Speed},
            y unit=\si{\meter\per\second},
            ytick={0.0,2.0,4.0,6.0,8.0,10.0},
            xmin=0,xmax=4,
            ymin=0,ymax=10,
            grid=major,
            width=0.8\columnwidth,
            height=0.5\columnwidth,
            very thin,
        ]
        \addplot[line width=1pt,domain=0:4]{8*x/3};
        \end{axis}
    \end{tikzpicture}
    \end{center}
    What is the total distance the object falls during
        the first \SI{3.0}{\second}?
    \begin{multicols}{2}
    \begin{choices}
      \correctchoice{\SI{12}{\meter}}
        \wrongchoice{\SI{24}{\meter}}
        \wrongchoice{\SI{44}{\meter}}
        \wrongchoice{\SI{72}{\meter}}
    \end{choices}
    \end{multicols}
\end{question}
}


%% Section June2010
%%--------------------
\element{nysed}{
\begin{question}{June2010-Q03}
    A car traveling on a straight road at \SI{15.0}{\meter\per\second} accelerates uniformly to a speed of \SI{21.0}{\meter\per\second} in \SI{12.0}{\second}.
    The total distance traveled by the car in this \SI{12.0}{\second} time interval is:
    \begin{multicols}{2}
    \begin{choices}
        \wrongchoice{\SI{36.0}{\meter}}
        \wrongchoice{\SI{180}{\meter}}
      \correctchoice{\SI{216}{\meter}}
        \wrongchoice{\SI{252}{\meter}}
    \end{choices}
    \end{multicols}
\end{question}
}


%% Section June2009
%%--------------------
\newcommand{\myJuneZeroNineQthirtySevenTikz}{
    \begin{tikzpicture}
        \begin{axis}[
            axis y line=left,
            axis x line=bottom,
            axis line style={->},
            xlabel={time},
            x unit=\si{\second},
            xtick={0,2,4,6},
            ylabel={velocity},
            y unit=\si{\meter\per\second},
            ytick={0,5,10,15},
            xmin=0,xmax=6.5,
            ymin=0,ymax=16,
            grid=major,
            width=0.8\columnwidth,
            height=0.5\columnwidth,
            very thin,
        ]
        \addplot[line width=1pt,domain=0:4]{2.5*x};
        \addplot[line width=1pt,domain=4:6]{10};
        \end{axis}
    \end{tikzpicture}
}

\element{nysed}{
\begin{question}{June2009-Q37}
    The diagram below represents the motion of a car during a \SI{6.0}{\second} time interval.
    \begin{center}
        \myJuneZeroNineQthirtySevenTikz
    \end{center}
    What is the acceleration of the car at $t=\SI{5.0}{\second}$?
    \begin{multicols}{2}
    \begin{choices}
      \correctchoice{\SI{0.0}{\meter\per\second\squared}}
        \wrongchoice{\SI{2.0}{\meter\per\second\squared}}
        \wrongchoice{\SI{2.5}{\meter\per\second\squared}}
        \wrongchoice{\SI{10.0}{\meter\per\second\squared}}
    \end{choices}
    \end{multicols}
\end{question}
}

\element{nysed}{
\begin{question}{June2009-Q38}
    The diagram below represents the motion of a car during a \SI{6.0}{\second} time interval.
    \begin{center}
        \myJuneZeroNineQthirtySevenTikz
    \end{center}
    What is the total distance traveled by the car during this \SI{6.0}{\second} interval?
    \begin{multicols}{2}
    \begin{choices}
        \wrongchoice{\SI{10.}{\meter}}
        \wrongchoice{\SI{20.}{\meter}}
      \correctchoice{\SI{40.}{\meter}}
        \wrongchoice{\SI{60.}{\meter}}
    \end{choices}
    \end{multicols}
\end{question}
}


%% Section Jan2009
%%--------------------
\element{nysed}{
\begin{question}{Jan2009-Q04}
    As a car driven south in a straight line with \emph{decreasing} speed,
        the acceleration of the car must be:
    \begin{choices}
      \correctchoice{directed northward}
        \wrongchoice{directed southward}
        \wrongchoice{zero}
        \wrongchoice{constant, but not zero}
    \end{choices}
\end{question}
}


%% Section June2008
%%--------------------
\element{nysed}{
\begin{question}{June2008-Q07}
    The speed of an object undergoing constant acceleration increases from \SI{8.0}{\meter\per\second} to \SI{16.0}{\meter\per\second} in \SI{10}{\second}.
    How far does the object travel during the \SI{10}{\second}?
    \begin{multicols}{2}
    \begin{choices}
        \wrongchoice{\SI{3.6e2}{\meter}}
        \wrongchoice{\SI{1.6e2}{\meter}}
      \correctchoice{\SI{1.2e2}{\meter}}
        \wrongchoice{\SI{8.0e1}{\meter}}
    \end{choices}
    \end{multicols}
\end{question}
}

\element{nysed}{
\begin{question}{June2008-Q37}
    The graph below represents the displacement of an object moving in a straight line as a function of time.
    \begin{center}
    \begin{tikzpicture}
        \begin{axis}[
            axis y line=left,
            axis x line=bottom,
            axis line style={->},
            xlabel={Time},
            x unit=\si{\second},
            xtick={0.0,2.0,4.0,6.0,8.0,10,0},
            ylabel={displacement},
            y unit=\si{\meter},
            ytick={0,4,8,12,16},
            xmin=0,xmax=10,
            ymin=0,ymax=16,
            grid=major,
            width=0.8\columnwidth,
            height=0.5\columnwidth,
            very thin,
        ]
        \addplot[line width=1pt,domain=0:4]{2*x};
        \addplot[line width=1pt,domain=4:6]{8};
        \addplot[line width=1pt,domain=6:10]{ 8 - 2*( (x-6) * (x-10) )};
        \end{axis}
    \end{tikzpicture}
    \end{center}
    What was the total distance traveled by the object during the \SI{10}{\second} time interval?
    \begin{multicols}{2}
    \begin{choices}
        \wrongchoice{\SI{0}{\meter}}
        \wrongchoice{\SI{8}{\meter}}
        \wrongchoice{\SI{16}{\meter}}
      \correctchoice{\SI{24}{\meter}}
    \end{choices}
    \end{multicols}
\end{question}
}


%% Section Jan2008
%%--------------------
\element{nysed}{
\begin{question}{Jan2008-Q02}
    A race car starting from rest accelerates uniformly at a rate of \SI{4.9}{\meter\per\second\squared}.
    What is the car's speed after it has traveled \SI{200}{\meter}.
    \begin{multicols}{2}
    \begin{choices}
      \correctchoice{\SI{44.3}{\meter\per\second}}
        \wrongchoice{\SI{62.6}{\meter\per\second}}
        \wrongchoice{\SI{1960}{\meter\per\second}}
        \wrongchoice{\SI{31.3}{\meter\per\second}}
    \end{choices}
    \end{multicols}
\end{question}
}


%% Section June2007
%%--------------------
\element{nysed}{
\begin{question}{June2007-Q02}
    An astronaut standing on a platform on the Moon drops a hammer.
    If the hammer falls \SI{6.0}{\meter} vertically in \SI{2.7}{\second},
        what is the acceleration?
    \begin{multicols}{2}
    \begin{choices}
      \correctchoice{\SI{1.6}{\meter\per\second\squared}}
        \wrongchoice{\SI{2.2}{\meter\per\second\squared}}
        \wrongchoice{\SI{4.4}{\meter\per\second\squared}}
        \wrongchoice{\SI{9.8}{\meter\per\second\squared}}
    \end{choices}
    \end{multicols}
\end{question}
}

\element{nysed}{
\begin{question}{June2007-Q40}
    An observer recorded the following data for the motion of a car undergoing constant acceleration.
    \begin{center}
    \begin{tabu}{cc}
        Time (\si{\second}) & Speed (\si{\meter\per\second}) \\
        \midrule
        3.0 & 4.0 \\
        5.0 & 7.0 \\
        6.0 & 8.5 \\
    \end{tabu}
    \end{center}
    What was the magnitude of the acceleration of the car?
    \begin{multicols}{2}
    \begin{choices}
        \wrongchoice{\SI{1.3}{\meter\per\second\squared}}
        \wrongchoice{\SI{2.0}{\meter\per\second\squared}}
      \correctchoice{\SI{1.5}{\meter\per\second\squared}}
        \wrongchoice{\SI{4.5}{\meter\per\second\squared}}
    \end{choices}
    \end{multicols}
\end{question}
}


%% Section Jan2007
%%--------------------
\element{nysed}{
\begin{question}{Jan2007-Q03}
    A car increases its speed from \SI{9.6}{\meter\per\second} to \SI{11.2}{\meter\per\second} in \SI{4.0}{\second}.
    The average acceleration of the car during this \SI{4.0}{\second} interval is:
    \begin{multicols}{2}
    \begin{choices}
      \correctchoice{\SI{0.40}{\meter\per\second\squared}}
        \wrongchoice{\SI{2.4}{\meter\per\second\squared}}
        \wrongchoice{\SI{2.8}{\meter\per\second\squared}}
        \wrongchoice{\SI{5.2}{\meter\per\second\squared}}
    \end{choices}
    \end{multicols}
\end{question}
}

\element{nysed}{
\begin{question}{Jan2007-Q36}
    A cart travels with a constant nonzero acceleration along a straight line.
    Which graph best represents the relationship between the distance the cart travels and the time of travel?
    \begin{multicols}{2}
    \begin{choices}
        \AMCboxDimensions{down=-2.5em}
        \correctchoice{
            \begin{tikzpicture}
                \begin{axis}[
                    axis y line=left,
                    axis x line=bottom,
                    axis line style={->},
                    xlabel={time},
                    xtick=\empty,
                    ylabel={distance},
                    ytick=\empty,
                    xmin=0,xmax=11,
                    ymin=0,ymax=11,
                    width=\columnwidth,
                    very thin,
                ]
                \addplot[line width=1pt,domain=0:10]{0.1 * x*x};
                \end{axis}
            \end{tikzpicture}
        }
        \wrongchoice{
            \begin{tikzpicture}
                \begin{axis}[
                    axis y line=left,
                    axis x line=bottom,
                    axis line style={->},
                    xlabel={time},
                    xtick=\empty,
                    ylabel={distance},
                    ytick=\empty,
                    xmin=0,xmax=11,
                    ymin=0,ymax=11,
                    width=\columnwidth,
                    very thin,
                ]
                \addplot[line width=1pt,domain=0:10]{10-x};
                \end{axis}
            \end{tikzpicture}
        }
        \wrongchoice{
            \begin{tikzpicture}
                \begin{axis}[
                    axis y line=left,
                    axis x line=bottom,
                    axis line style={->},
                    xlabel={time},
                    xtick=\empty,
                    ylabel={distance},
                    ytick=\empty,
                    xmin=0,xmax=185,
                    ymin=0,ymax=10,
                    width=\columnwidth,
                    very thin,
                ]
                \addplot[line width=1pt,domain=0:180]{8*sin(x)};
                \end{axis}
            \end{tikzpicture}
        }
        \wrongchoice{
            \begin{tikzpicture}
                \begin{axis}[
                    axis y line=left,
                    axis x line=bottom,
                    axis line style={->},
                    xlabel={time},
                    xtick=\empty,
                    ylabel={distance},
                    ytick=\empty,
                    xmin=0,xmax=11,
                    ymin=0,ymax=11,
                    width=\columnwidth,
                    very thin,
                ]
                \addplot[line width=1pt,domain=0:10]{x};
                \end{axis}
            \end{tikzpicture}
        }
    \end{choices}
    \end{multicols}
\end{question}
}


%% Section June2006
%%--------------------
\element{nysed}{
\begin{question}{June2006-Q02}
    A rocket initially at rest on the ground lifts off vertically with a constant acceleration of \SI{2.0e1}{\meter\per\second\squared}.
    How long will it take the rocket to reach an altitude of \SI{9.0e3}{\meter}?
    \begin{multicols}{2}
    \begin{choices}
      \correctchoice{\SI{3.0e1}{\second}}
        \wrongchoice{\SI{4.3e1}{\second}}
        \wrongchoice{\SI{4.5e2}{\second}}
        \wrongchoice{\SI{9.0e2}{\second}}
    \end{choices}
    \end{multicols}
\end{question}
}


%% Section Jan2006
%%--------------------
\element{nysed}{
\begin{question}{Jan2006-Q01}
    The speed of a wagon increases from \SI{2.5}{\meter\per\second} to \SI{9.0}{\meter\per\second} in \SI{3.0}{\second} as it accelerates uniformly down a hill.
    what is the magnitude of the acceleration of the wagon during this \SI{3.0}{\second} interval?
    \begin{multicols}{2}
    \begin{choices}
        \wrongchoice{\SI{0.83}{\meter\per\second\squared}}
      \correctchoice{\SI{2.2}{\meter\per\second\squared}}
        \wrongchoice{\SI{3.0}{\meter\per\second\squared}}
        \wrongchoice{\SI{3.8}{\meter\per\second\squared}}
    \end{choices}
    \end{multicols}
\end{question}
}

\element{nysed}{
\begin{question}{Jan2006-Q38}
    The graph below represents the relationship between speed and time for an object moving along a straight line.
    \begin{center}
    \begin{tikzpicture}
        \begin{axis}[
            axis y line=left,
            axis x line=bottom,
            axis line style={->},
            title={Speed vs. Time},
            xlabel={time},
            xtick={0,1,2,3,4,5},
            x unit=\si{\second},
            ylabel={speed},
            y unit=\si{\meter\per\second},
            ytick={0,5,10,15,20,25},
            xmin=0,xmax=5,
            ymin=0,ymax=25,
            grid=major,
            width=0.8\columnwidth,
            height=0.5\columnwidth,
            very thin,
        ]
        \addplot[line width=1pt,domain=0:5]{5*x};
        \end{axis}
    \end{tikzpicture}
    \end{center}
    What is the total distance traveled by the object during the first \SI{4}{\second}?
    \begin{multicols}{2}
    \begin{choices}
        \wrongchoice{\SI{5}{\meter}}
        \wrongchoice{\SI{20}{\meter}}
      \correctchoice{\SI{40}{\meter}}
        \wrongchoice{\SI{80}{\meter}}
    \end{choices}
    \end{multicols}
\end{question}
}


%% Section Jan2005
%%--------------------
\element{nysed}{
\begin{question}{Jan2005-Q36}
    Which pair of graphs represents the same motion of an object?
    \begin{choices}
        \AMCboxDimensions{down=-2.5em}
        \correctchoice{
            \begin{tikzpicture}
                \begin{groupplot}[
                        axis y line=left,
                        axis x line=bottom,
                        axis line style={->},
                        group style={group size=2 by 1},
                        xtick=\empty,
                        ytick=\empty,
                        width=0.5\columnwidth,
                    ]
                    \nextgroupplot[
                        xlabel={time},
                        ylabel={displacement},
                        xmin=0,xmax=11,
                        ymin=0,ymax=11,
                    ] \addplot[line width=1pt,domain=0:10] {0.1*x*x};
                    \nextgroupplot[
                        xlabel={time},
                        ylabel={velocity},
                        xmin=0,xmax=11,
                        ymin=0,ymax=11,
                    ] \addplot[line width=1pt,domain=0:10] {x};
                \end{groupplot}
            \end{tikzpicture}
        }
        \wrongchoice{
            \begin{tikzpicture}
                \begin{groupplot}[
                        axis y line=left,
                        axis x line=middle,
                        axis line style={->},
                        group style={group size=2 by 1},
                        xtick=\empty,
                        ytick=\empty,
                        width=0.5\columnwidth,
                    ]
                    \nextgroupplot[
                        xlabel={time},
                        ylabel={displacement},
                        x label style={anchor=north east},
                        xmin=0,xmax=11,
                        ymin=-5.5,ymax=5.5,
                    ] \addplot[line width=1pt,domain=0:10] {-5+x};
                    \nextgroupplot[
                        xlabel={time},
                        ylabel={velocity},
                        xmin=0,xmax=11,
                        ymin=-5.5,ymax=5.5,
                    ] \addplot[line width=1pt,domain=0:10] {-3};
                \end{groupplot}
            \end{tikzpicture}
        }
        \wrongchoice{
            \begin{tikzpicture}
                \begin{groupplot}[
                        axis y line=left,
                        axis x line=middle,
                        axis line style={->},
                        group style={group size=2 by 1},
                        xtick=\empty,
                        ytick=\empty,
                        width=0.5\columnwidth,
                    ]
                    \nextgroupplot[
                        xlabel={time},
                        ylabel={displacement},
                        xmin=0,xmax=11,
                        ymin=-5.5,ymax=5.5,
                    ] \addplot[line width=1pt,domain=0:10] {5-x};
                    \nextgroupplot[
                        xlabel={time},
                        ylabel={velocity},
                        x label style={anchor=north east},
                        xmin=0,xmax=11,
                        ymin=-5.5,ymax=5.5,
                    ] \addplot[line width=1pt,domain=0:10] {-5+x};
                \end{groupplot}
            \end{tikzpicture}
        }
        \wrongchoice{
            \begin{tikzpicture}
                \begin{groupplot}[
                        axis y line=left,
                        axis x line=bottom,
                        axis line style={->},
                        group style={group size=2 by 1},
                        xtick=\empty,
                        ytick=\empty,
                        width=0.5\columnwidth,
                    ]
                    \nextgroupplot[
                        xlabel={time},
                        ylabel={displacement},
                        xmin=0,xmax=11,
                        ymin=0,ymax=11,
                    ] \addplot[line width=1pt,domain=0:10] {-0.4 *x * (x-10)};
                    \nextgroupplot[
                        xlabel={time},
                        ylabel={velocity},
                        xmin=0,xmax=11,
                        ymin=0,ymax=11,
                    ] \addplot[line width=1pt,domain=0:10] {8};
                \end{groupplot}
            \end{tikzpicture}
        }
    \end{choices}
\end{question}
}



%% Section Jan2004
%%--------------------
\element{nysed}{
\begin{question}{Jan2004-Q03}
    A skater increases her speed uniformly from \SI{2.0}{\meter\per\second}
        to \SI{7.0}{\meter\per\second} over a distance of \SI{12}{\meter}.
    The magnitude of her acceleration as she travels this \SI{12}{\meter} is:
    \begin{multicols}{2}
    \begin{choices}
      \correctchoice{\SI{1.9}{\meter\per\second\squared}}
        \wrongchoice{\SI{2.2}{\meter\per\second\squared}}
        \wrongchoice{\SI{2.4}{\meter\per\second\squared}}
        \wrongchoice{\SI{3.8}{\meter\per\second\squared}}
    \end{choices}
    \end{multicols}
\end{question}
}


%% Section June2003
%%--------------------
\element{nysed}{
\begin{question}{June2003-Q03}
    A car initially traveling at a speed of \SI{16}{\meter\per\second} accelerates uniformly to a speed of \SI{20}{\meter\per\second} over a distance of \SI{36}{\meter}.
    What is the magnitude of the car's acceleration?
    \begin{multicols}{2}
    \begin{choices}
      \correctchoice{\SI{2.0}{\meter\per\second\squared}}
        \wrongchoice{\SI{0.22}{\meter\per\second\squared}}
        \wrongchoice{\SI{9.0}{\meter\per\second\squared}}
        \wrongchoice{\SI{0.11}{\meter\per\second\squared}}
    \end{choices}
    \end{multicols}
\end{question}
}


%% Section Jan2003
%%--------------------

\element{nysed}{
\begin{question}{Jan2003-Q45}
    Which graph best represents the motion of a block accelerating uniformly down an inclined plane?
    \begin{multicols}{2}
    \begin{choices}
        \AMCboxDimensions{down=-2.5em}
        \correctchoice{
            \begin{tikzpicture}
                \begin{axis}[
                    axis y line=left,
                    axis x line=bottom,
                    axis line style={->},
                    xlabel={time},
                    xtick=\empty,
                    ylabel={distance},
                    ytick=\empty,
                    xmin=0,xmax=11,
                    ymin=0,ymax=11,
                    width=\columnwidth,
                    very thin,
                ]
                \addplot[line width=1pt,domain=0:10]{0.1*x*x};
                \end{axis}
            \end{tikzpicture}
        }
        \wrongchoice{
            \begin{tikzpicture}
                \begin{axis}[
                    axis y line=left,
                    axis x line=bottom,
                    axis line style={->},
                    xlabel={time},
                    xtick=\empty,
                    ylabel={distance},
                    ytick=\empty,
                    xmin=0,xmax=11,
                    ymin=0,ymax=11,
                    width=\columnwidth,
                    very thin,
                ]
                \addplot[line width=1pt,domain=0:10]{x};
                \end{axis}
            \end{tikzpicture}
        }
        \wrongchoice{
            \begin{tikzpicture}
                \begin{axis}[
                    axis y line=left,
                    axis x line=bottom,
                    axis line style={->},
                    xlabel={time},
                    xtick=\empty,
                    ylabel={distance},
                    ytick=\empty,
                    xmin=0,xmax=11,
                    ymin=0,ymax=11,
                    width=\columnwidth,
                    very thin,
                ]
                \addplot[line width=1pt,domain=0:10]{7};
                \end{axis}
            \end{tikzpicture}
        }
        \wrongchoice{
            \begin{tikzpicture}
                \begin{axis}[
                    axis y line=left,
                    axis x line=bottom,
                    axis line style={->},
                    xlabel={time},
                    xtick=\empty,
                    ylabel={distance},
                    ytick=\empty,
                    xmin=0,xmax=11,
                    ymin=0,ymax=11,
                    width=\columnwidth,
                    very thin,
                ]
                \addplot[line width=1pt,domain=0:5]{1+1.2*x};
                \addplot[line width=1pt,domain=5:10]{7};
                \end{axis}
            \end{tikzpicture}
        }
    \end{choices}
    \end{multicols}
\end{question}
}


%% Section Aug2002
%%--------------------
\element{nysed}{
\begin{question}{Aug2002-Q02}
    The speed of a car is increased uniformly from \SI{20}{\meter\per\second} to \SI{30}{\meter\per\second} in \SI{4.0}{\second}.
    The magnitude of the car's average acceleration in this \SI{4.0}{\second} interval is:
    \begin{multicols}{2}
    \begin{choices}
        \wrongchoice{\SI{0.40}{\meter\per\second\squared}}
      \correctchoice{\SI{2.5}{\meter\per\second\squared}}
        \wrongchoice{\SI{10}{\meter\per\second\squared}}
        \wrongchoice{\SI{13}{\meter\per\second\squared}}
    \end{choices}
    \end{multicols}
\end{question}
}

\element{nysed}{
\begin{question}{Aug2002-Q03}
    A roller coaster, traveling with an initial speed of \SI{15}{\meter\per\second},
        decelerates uniformly at \SI{-7.0}{\meter\per\second\squared} to a full stop.
    Approximately how far does the roller coaster travel during its deceleration?
    \begin{multicols}{2}
    \begin{choices}
        \wrongchoice{\SI{1.0}{\meter}}
        \wrongchoice{\SI{2.0}{\meter}}
      \correctchoice{\SI{16}{\meter}}
        \wrongchoice{\SI{32}{\meter}}
    \end{choices}
    \end{multicols}
\end{question}
}

\element{nysed}{
\begin{question}{Aug2002-Q39}
    The graph below shows the velocity of a race car moving along a straight line as a function of time.
    \begin{center}
    \begin{tikzpicture}
        \begin{axis}[
            clip=false,
            axis y line=left,
            axis x line=bottom,
            axis line style={->},
            xlabel={time},
            x unit=\si{\second},
            xtick={0,1,2,3,4},
            ylabel={velocity},
            y unit=\si{\meter\per\second},
            ytick={0,10,20,30,40},
            xmin=0,xmax=4.1,
            ymin=0,ymax=42,
            grid=major,
            width=0.8\columnwidth,
            height=0.5\columnwidth,
            very thin,
        ]
        \addplot[line width=1pt,domain=0:4]{10*x};
        \end{axis}
    \end{tikzpicture}
    \end{center}
    What is the magnitude of the displacement of the car from $t=\SI{2.0}{\second}$ to $t=\SI{4.0}{\second}$?
    \begin{multicols}{2}
    \begin{choices}
        \wrongchoice{\SI{20}{\meter}}
        \wrongchoice{\SI{40}{\meter}}
      \correctchoice{\SI{60}{\meter}}
        \wrongchoice{\SI{80}{\meter}}
    \end{choices}
    \end{multicols}
\end{question}
}


%% Section June2002
%%--------------------
\element{nysed}{
\begin{question}{June2002-Q03}
    An object with an initial speed of \SI{4.0}{\meter\per\second} accelerates uniformly at \SI{2.0}{\meter\per\second\squared} in the direction of its motion for a distance of \SI{5.0}{\meter}.
    What is the final speed of the object?
    \begin{multicols}{2}
    \begin{choices}
      \correctchoice{\SI{6.0}{\meter\per\second}}
        \wrongchoice{\SI{10}{\meter\per\second}}
        \wrongchoice{\SI{14}{\meter\per\second}}
        \wrongchoice{\SI{36}{\meter\per\second}}
    \end{choices}
    \end{multicols}
\end{question}
}

\element{nysed}{
\begin{question}{June2002-Q04}
    After a model rocket reached its maximum height, it then took \SI{5.0}{\second} to return to the launch site.
    What is the approximate maximum height reached by the rocket?
    [Neglect air resistance]
    \begin{multicols}{2}
    \begin{choices}
        \wrongchoice{\SI{49}{\meter}}
        \wrongchoice{\SI{98}{\meter}}
      \correctchoice{\SI{120}{\meter}}
        \wrongchoice{\SI{250}{\meter}}
    \end{choices}
    \end{multicols}
\end{question}
}

\element{nysed}{
\begin{question}{June2002-Q36}
    The displacement-time graph below represents the motion of a cart initially moving in a straight line.
    \begin{center}
    \begin{tikzpicture}
        \begin{axis}[
            clip=false,
            axis y line=left,
            axis x line=bottom,
            axis line style={->},
            xlabel={time},
            xtick=\empty,
            ylabel={displacement},
            ytick=\empty,
            xmin=0,xmax=26,
            ymin=0,ymax=12,
            width=0.8\columnwidth,
            height=0.5\columnwidth,
            very thin,
        ]
        \addplot[line width=1pt,domain=0:5]{0.24*x*x}
            node[black,pos=0,anchor=north] {$A$};
        \addplot[line width=1pt,domain=5:14]{6 + (x-5)/3}
            node[black,pos=0,anchor=south] {$B$};
        \addplot[line width=1pt,domain=14:20]{9}
            node[black,pos=0,anchor=south] {$C$};
        \addplot[line width=1pt,domain=20:24]{9 - 1.25*(x-20)}
            node[black,pos=0,anchor=south west] {$D$}
            node[black,pos=1,anchor=west] {$E$};
        \addplot[only marks,mark=*,mark size=2pt] coordinates
            { (0,0) (5,6) (14,9) (20,9) (24,4) };
        \end{axis}
    \end{tikzpicture}
    \end{center}
    During which interval is the cart moving forward at constant speed?
    \begin{multicols}{2}
    \begin{choices}
        \wrongchoice{$AB$}
      \correctchoice{$BC$}
        \wrongchoice{$CD$}
        \wrongchoice{$DE$}
    \end{choices}
    \end{multicols}
\end{question}
}


%% Section Jan2002
%%--------------------
\element{nysed}{
\begin{question}{Jan2002-Q04}
    A skier starting from rest skis straight down a slope \SI{50}{\meter} long in \SI{5.0}{\second}.
    What is the magnitude of the acceleration of the skier?
    \begin{multicols}{2}
    \begin{choices}
        \wrongchoice{\SI{20}{\meter\per\second\squared}}
        \wrongchoice{\SI{9.8}{\meter\per\second\squared}}
        \wrongchoice{\SI{5.0}{\meter\per\second\squared}}
      \correctchoice{\SI{4.0}{\meter\per\second\squared}}
    \end{choices}
    \end{multicols}
\end{question}
}


%% Section June2001
%%--------------------
\element{nysed}{
\begin{question}{June2001-Q03}
    Which graph best represents the motion of an object whose speed is increasing?
    \begin{multicols}{2}
    \begin{choices}
        \AMCboxDimensions{down=-2.5em}
        \correctchoice{
            \begin{tikzpicture}
                \begin{axis}[
                    axis y line=left,
                    axis x line=bottom,
                    axis line style={->},
                    xlabel={time},
                    xtick=\empty,
                    ylabel={distance},
                    ytick=\empty,
                    xmin=0,xmax=11,
                    ymin=0,ymax=11,
                    width=\columnwidth,
                    very thin,
                ]
                \addplot[line width=1pt,domain=0:10]{0.1*x*x};
                \end{axis}
            \end{tikzpicture}
        }
        \wrongchoice{
            \begin{tikzpicture}
                \begin{axis}[
                    axis y line=left,
                    axis x line=bottom,
                    axis line style={->},
                    xlabel={time},
                    xtick=\empty,
                    ylabel={distance},
                    ytick=\empty,
                    xmin=0,xmax=11,
                    ymin=0,ymax=11,
                    width=\columnwidth,
                    very thin,
                ]
                \addplot[line width=1pt,domain=0:10]{x};
                \end{axis}
            \end{tikzpicture}
        }
        \wrongchoice{
            \begin{tikzpicture}
                \begin{axis}[
                    axis y line=left,
                    axis x line=bottom,
                    axis line style={->},
                    xlabel={time},
                    xtick=\empty,
                    ylabel={distance},
                    ytick=\empty,
                    xmin=0,xmax=11,
                    ymin=0,ymax=11,
                    width=\columnwidth,
                    very thin,
                ]
                \addplot[line width=1pt,domain=0:10]{10-x};
                \end{axis}
            \end{tikzpicture}
        }
        \wrongchoice{
            \begin{tikzpicture}
                \begin{axis}[
                    axis y line=left,
                    axis x line=bottom,
                    axis line style={->},
                    xlabel={time},
                    xtick=\empty,
                    ylabel={distance},
                    ytick=\empty,
                    xmin=0,xmax=11,
                    ymin=0,ymax=11,
                    width=\columnwidth,
                    very thin,
                ]
                \addplot[line width=1pt,domain=0:10]{10/x};
                \end{axis}
            \end{tikzpicture}
        }
    \end{choices}
    \end{multicols}
\end{question}
}

\element{nysed}{
\begin{question}{June2001-Q05}
    A car having an initial velocity of \SI{12}{\meter\per\second} east slows uniformly to \SI{2}{\meter\per\second} east in \SI{4.0}{\second}.
    The acceleration of the car during this \SI{4.0}{\second} interval is:
    \begin{multicols}{2}
    \begin{choices}
      \correctchoice{\SI{2.5}{\meter\per\second\squared} west}
        \wrongchoice{\SI{2.5}{\meter\per\second\squared} east}
        \wrongchoice{\SI{6.0}{\meter\per\second\squared} west}
        \wrongchoice{\SI{6.0}{\meter\per\second\squared} east}
    \end{choices}
    \end{multicols}
\end{question}
}

\element{nysed}{
\begin{question}{June2001-Q14}
    An airplane originally at rest on a runway accelerates uniformly at \SI{6.0}{\meter\per\second\squared} for \SI{12}{\second}.
    During this \SI{12}{\second} interval,
        the airplane travels a distance of approximately:
    \begin{multicols}{2}
    \begin{choices}
        \wrongchoice{\SI{72}{\meter}}
        \wrongchoice{\SI{220}{\meter}}
      \correctchoice{\SI{430}{\meter}}
        \wrongchoice{\SI{860}{\meter}}
    \end{choices}
    \end{multicols}
\end{question}
}


%% Section Jan2001
%%--------------------
\element{nysed}{
\begin{question}{Jan2001-Q02}
    The diagram below represents the relationship between velocity and time for four cars,
        $A$, $B$, $C$, and $D$, in straight-line motion.
    \begin{center}
    \begin{tikzpicture}
        \begin{axis}[
            axis y line=left,
            axis x line=bottom,
            axis line style={->},
            xlabel={time},
            x unit=\si{\second},
            xtick={0,5,10,15,20},
            ylabel={velocity},
            ytick=\empty,
            xmin=0,xmax=21,
            ymin=0,ymax=10,
            width=0.98\columnwidth,
            height=0.618\columnwidth,
            very thin,
        ]
        \addplot[line width=0.7pt,domain=0:20]{8}
            node[pos=0.1,anchor=south east] {$A$};
        \addplot[line width=1pt,domain=0:20]{4 + 4*x/15}
            node[pos=0.1,anchor=south east] {$B$};
        \addplot[line width=1.25pt,domain=0:20]{8*x/15}
            node[pos=0.1,anchor=south east] {$C$};
        \addplot[line width=1.5pt,domain=10:20]{8*(x-10)/5}
            node[pos=0.0,anchor=south east] {$D$};
        \end{axis}
    \end{tikzpicture}
    \end{center}
    Which car has the greatest acceleration during the time interval \SI{10}{\second} to \SI{15}{\second}?
    \begin{multicols}{4}
    \begin{choices}[o]
        \wrongchoice{$A$}
        \wrongchoice{$B$}
        \wrongchoice{$C$}
      \correctchoice{$D$}
    \end{choices}
    \end{multicols}
\end{question}
}

\element{nysed}{
\begin{question}{Jan2001-Q14}
    %% NOTE: Reword
    %The graph below represents the displacment ($x$)
    %    of an object with respect to time ($t$).
    The graph below represents the motion of an object.
    \begin{center}
    \begin{tikzpicture}
        \begin{axis}[
            axis y line=left,
            axis x line=bottom,
            axis line style={->},
            xlabel={time},
            xtick=\empty,
            ylabel={displacement},
            ytick=\empty,
            xmin=0,xmax=10,
            ymin=0,ymax=10,
            width=0.8\columnwidth,
            height=0.5\columnwidth,
            very thin,
        ]
        \addplot[line width=1pt,domain=0:10]{x};
        \end{axis}
    \end{tikzpicture}
    \end{center}
    According the the graph, as time increases,
        the velocity of the object:
    \begin{choices}
        \wrongchoice{decreases}
        \wrongchoice{increases}
      \correctchoice{remains the same}
    \end{choices}
\end{question}
}


%% Section June2000
%%--------------------
\element{nysed}{
\begin{question}{June2000-Q02}
    Which pair of graphs represents the same motion?
    \begin{choices}
        \AMCboxDimensions{down=-2.5em}
        \correctchoice{
            \begin{tikzpicture}
                \begin{groupplot}[
                        axis y line=left,
                        axis x line=bottom,
                        axis line style={->},
                        group style={group size=2 by 1},
                        xtick=\empty,
                        ytick=\empty,
                        width=0.5\columnwidth,
                    ]
                    \nextgroupplot[
                        xlabel={time},
                        ylabel={displacement},
                        xmin=0,xmax=11,
                        ymin=0,ymax=11,
                    ] \addplot[line width=1pt,domain=0:10] {x};
                    \nextgroupplot[
                        xlabel={time},
                        ylabel={velocity},
                        xmin=0,xmax=11,
                        ymin=0,ymax=11,
                    ] \addplot[line width=1pt,domain=0:10] {5};
                \end{groupplot}
            \end{tikzpicture}
        }
        \wrongchoice{
            \begin{tikzpicture}
                \begin{groupplot}[
                        axis y line=left,
                        axis x line=bottom,
                        axis line style={->},
                        group style={group size=2 by 1},
                        xtick=\empty,
                        ytick=\empty,
                        width=0.5\columnwidth,
                    ]
                    \nextgroupplot[
                        xlabel={time},
                        ylabel={displacement},
                        xmin=0,xmax=11,
                        ymin=0,ymax=11,
                    ] \addplot[line width=1pt,domain=0:10] {5};
                    \nextgroupplot[
                        xlabel={time},
                        ylabel={velocity},
                        xmin=0,xmax=11,
                        ymin=0,ymax=11,
                    ] \addplot[line width=1pt,domain=0:10] {x};
                \end{groupplot}
            \end{tikzpicture}
        }
        \wrongchoice{
            \begin{tikzpicture}
                \begin{groupplot}[
                        axis y line=left,
                        axis x line=bottom,
                        axis line style={->},
                        group style={group size=2 by 1},
                        xtick=\empty,
                        ytick=\empty,
                        width=0.5\columnwidth,
                    ]
                    \nextgroupplot[
                        xlabel={time},
                        ylabel={displacement},
                        xmin=0,xmax=11,
                        ymin=0,ymax=11,
                    ] \addplot[line width=1pt,domain=0:10] {10-x};
                    \nextgroupplot[
                        xlabel={time},
                        ylabel={velocity},
                        xmin=0,xmax=11,
                        ymin=0,ymax=11,
                    ] \addplot[line width=1pt,domain=0:10] {x};
                \end{groupplot}
            \end{tikzpicture}
        }
        \wrongchoice{
            \begin{tikzpicture}
                \begin{groupplot}[
                        axis y line=left,
                        axis x line=bottom,
                        axis line style={->},
                        group style={group size=2 by 1},
                        xtick=\empty,
                        ytick=\empty,
                        width=0.5\columnwidth,
                    ]
                    \nextgroupplot[
                        xlabel={time},
                        ylabel={displacement},
                        xmin=0,xmax=11,
                        ymin=0,ymax=11,
                    ] \addplot[line width=1pt,domain=0:10] {x};
                    \nextgroupplot[
                        xlabel={time},
                        ylabel={velocity},
                        xmin=0,xmax=11,
                        ymin=0,ymax=11,
                    ] \addplot[line width=1pt,domain=0:10] {x};
                \end{groupplot}
            \end{tikzpicture}
        }
    \end{choices}
\end{question}
}

\element{nysed}{
\begin{question}{June2000-Q03}
    A runner starts from rest and accelerates uniformly to a speed of \SI{8.0}{\meter\per\second} in \SI{4.0}{\second}.
    The magnitude of the acceleration of the runner is:
    \begin{multicols}{2}
    \begin{choices}
        \wrongchoice{\SI{0.50}{\meter\per\second\squared}}
      \correctchoice{\SI{2.0}{\meter\per\second\squared}}
        \wrongchoice{\SI{9.8}{\meter\per\second\squared}}
        \wrongchoice{\SI{32}{\meter\per\second\squared}}
    \end{choices}
    \end{multicols}
\end{question}
}


%% Section June1999
%%--------------------
\element{nysed}{
\begin{question}{June1999-Q02}
    A truck with an initial speed of \SI{12}{\meter\per\second} accelerates uniformly at \SI{2.0}{\meter\per\second\squared} for \SI{3.0}{\second}.
    What is the total distance traveled by the truck during this \SI{3.0}{\second} interval?
    \begin{multicols}{2}
    \begin{choices}
        \wrongchoice{\SI{9.0}{\meter}}
        \wrongchoice{\SI{25}{\meter}}
        \wrongchoice{\SI{36}{\meter}}
      \correctchoice{\SI{45}{\meter}}
    \end{choices}
    \end{multicols}
\end{question}
}

\element{nysed}{
\begin{question}{June1999-Q05}
    The graph below represents the relationship between the displacement of an object and its time of travel along a straight line.
    \begin{center}
    \begin{tikzpicture}
        \begin{axis}[
            axis line style={->},
            axis y line=left,
            axis x line=bottom,
            label={displacement vs. time},
            xlabel={time},
            x unit=\si{\second},
            xtick={0,1,2,3,4,5,6,7,8},
            ylabel={displacement},
            y unit=\si{\meter},
            ytick={0,2,4,6,8,10},
            xmin=0,xmax=8.1,
            ymin=0,ymax=10.1,
            grid=major,
            width=0.8\columnwidth,
            height=0.5\columnwidth,
            very thin,
        ]
        \addplot[line width=1pt,domain=0:2]{4*x};
        \addplot[line width=1pt,domain=2:4]{8};
        \addplot[line width=1pt,domain=4:8]{16-2*x};
        \end{axis}
    \end{tikzpicture}
    \end{center}
    What is the magnitude of the object's total displacement after \SI{8.0}{\second}?
    \begin{multicols}{2}
    \begin{choices}
      \correctchoice{\SI{0}{\meter}}
        \wrongchoice{\SI{2}{\meter}}
        \wrongchoice{\SI{8}{\meter}}
        \wrongchoice{\SI{16}{\meter}}
    \end{choices}
    \end{multicols}
\end{question}
}

\element{nysed}{
\begin{question}{June1999-Q06}
    The graph below represents the relationship between the displacement of an object and its time of travel along a straight line.
    \begin{center}
    \begin{tikzpicture}
        \begin{axis}[
            axis line style={->},
            label={displacement vs. time},
            xlabel={time},
            x unit=\si{\second},
            xtick={0,1,2,3,4,5,6,7,8},
            ylabel={displacement},
            y unit=\si{\meter},
            ytick={0,2,4,6,8,10},
            xmin=0,xmax=8,
            ymin=0,ymax=10,
            grid=major,
            width=0.8\columnwidth,
            height=0.5\columnwidth,
            very thin,
        ]
        \addplot[line width=1pt,domain=0:2]{4*x};
        \addplot[line width=1pt,domain=2:4]{8};
        \addplot[line width=1pt,domain=4:8]{16-2*x};
        \end{axis}
    \end{tikzpicture}
    \end{center}
    What is the average speed of the object during the first \SI{4.0}{\second}?
    \begin{multicols}{2}
    \begin{choices}
        \wrongchoice{\SI{0}{\meter\per\second}}
      \correctchoice{\SI{2}{\meter\per\second}}
        \wrongchoice{\SI{8}{\meter\per\second}}
        \wrongchoice{\SI{4}{\meter\per\second}}
    \end{choices}
    \end{multicols}
\end{question}
}

%% Section June1998
%%--------------------
\element{nysed}{
\begin{question}{June1998-Q07}
    A car having an initial speed of \SI{16}{\meter\per\second} is uniformly brought to rest in \SI{4.0}{\second}.
    How far does the car travel during this \SI{4.0}{\second} interval?
    \begin{multicols}{2}
    \begin{choices}
      \correctchoice{\SI{32}{\meter}}
        \wrongchoice{\SI{82}{\meter}}
        \wrongchoice{\SI{96}{\meter}}
        \wrongchoice{\SI{4.0}{\meter}}
    \end{choices}
    \end{multicols}
\end{question}
}


%% Section June1997
%%--------------------
\element{nysed}{
\begin{question}{June1997-Q06}
    The displacement-time graph below represents the motion of a cart along a straight line.
    \begin{center}
    \begin{tikzpicture}
        \begin{axis}[
            clip=false,
            axis y line=left,
            axis x line=bottom,
            axis line style={->},
            xlabel={time},
            xtick=\empty,
            ylabel={displacement},
            ytick=\empty,
            xmin=0,xmax=26,
            ymin=0,ymax=12,
            width=0.8\columnwidth,
            height=0.5\columnwidth,
            very thin,
        ]
        \addplot[line width=1pt,domain=0:5]{0.24*x*x}
            node[black,pos=0,anchor=north west] {$A$};
        \addplot[line width=1pt,domain=5:14]{6 + (x-5)/3}
            node[black,pos=0,anchor=south] {$B$};
        \addplot[line width=1pt,domain=14:20]{9}
            node[black,pos=0,anchor=south] {$C$};
        \addplot[line width=1pt,domain=20:24]{9 - 1.25*(x-20)}
            node[black,pos=0,anchor=south west] {$D$}
            node[black,pos=1,anchor=west] {$E$};
        \addplot[only marks,mark=*,mark size=2pt] coordinates
            { (0,0) (5,6) (14,9) (20,9) (24,4) };
        \end{axis}
    \end{tikzpicture}
    \end{center}
    During which interval was the cart accelerating?
    \begin{multicols}{2}
    \begin{choices}
      \correctchoice{$AB$}
        \wrongchoice{$BC$}
        \wrongchoice{$CD$}
        \wrongchoice{$DE$}
    \end{choices}
    \end{multicols}
\end{question}
}

\element{nysed}{
\begin{question}{June1997-Q09}
    A \SI{1000}{\kilo\gram} car traveling with a velocity of \SI[retain-explicit-plus]{+20}{\meter\per\second} decelerates at \SI{-5.0}{\meter\per\second\squared} until it comes to rest.
    What is the total distance the car travels as it decelerates to rest?
    \begin{multicols}{2}
    \begin{choices}
        \wrongchoice{\SI{10}{\meter}}
        \wrongchoice{\SI{20}{\meter}}
      \correctchoice{\SI{40}{\meter}}
        \wrongchoice{\SI{80}{\meter}}
    \end{choices}
    \end{multicols}
\end{question}
}

\element{nysed}{
\begin{question}{June1997-Q54}
    A bicyclist accelerates from rest to a speed of \SI{5.0}{\meter\per\second} in \SI{10}{\second}.
    During the same \SI{10}{\second},
        a car accelerates from a speed of \SI{22}{\meter\per\second} to a speed of \SI{27}{\meter\per\second}.
    Compared to the acceleration of the bicycle,
        the acceleration of the car is:
    %% NOTE: bike=0.5 m/s/s, car=0.5 m/s/s
    \begin{multicols}{3}
    \begin{choices}
        \wrongchoice{less}
        \wrongchoice{greater}
      \correctchoice{the same}
    \end{choices}
    \end{multicols}
\end{question}
}


%% Section June1996
%%--------------------
\element{nysed}{
\begin{question}{June1996-Q03}
    The graph below represents the relationship between speed and time for a car moving in a straight line.
    \begin{center}
    \begin{tikzpicture}
        \begin{axis}[
            axis y line=left,
            axis x line=bottom,
            axis line style={->},
            xlabel={Time},
            x unit=\si{\second},
            xtick={0.0,1.0,2.0,3.0},
            ylabel={Speed},
            y unit=\si{\meter\per\second},
            ytick={0,10,20,30},
            xmin=0,xmax=3,
            ymin=0,ymax=30,
            grid=major,
            width=0.8\columnwidth,
            height=0.5\columnwidth,
            very thin,
        ]
        \addplot[line width=1pt,domain=0:4]{10*x};
        \end{axis}
    \end{tikzpicture}
    \end{center}
    The magnitude of the car's acceleration is:
    \begin{multicols}{2}
    \begin{choices}
      \correctchoice{\SI{1.0}{\meter\per\second\squared}}
        \wrongchoice{\SI{0.10}{\meter\per\second\squared}}
        \wrongchoice{\SI{10}{\meter\per\second\squared}}
        \wrongchoice{\SI{0.0}{\meter\per\second\squared}}
    \end{choices}
    \end{multicols}
\end{question}
}

\element{nysed}{
\begin{question}{June1996-Q04}
    Oil drips at \SI{0.4}{\second} intervals from a car that has an oil leak.
    Which pattern best represents the spacing of oil drops as the car accelerates uniformly from rest?
    \begin{choices}
        \AMCboxDimensions{down=-0.5em}
        \wrongchoice{
            \begin{tikzpicture}
                \draw[dashed,white!80!black] (-0.2,-1em) rectangle (6.2,1em);
                \foreach \x in {0,15,...,60} \fill (\x mm,0) circle (2pt);
            \end{tikzpicture}
        }
        \correctchoice{
            \begin{tikzpicture}
                \draw[dashed,white!80!black] (-0.2,-1em) rectangle (6.2,1em);
                \foreach \x in {0,1,...,4} \fill ({0.375*\x*\x},0) circle (2pt);
            \end{tikzpicture}
        }
        \wrongchoice{
            \begin{tikzpicture}
                \draw[dashed,white!80!black] (-0.2,-1em) rectangle (6.2,1em);
                \foreach \x in {0,10,20} \fill (\x mm,0) circle (2pt);
                \foreach \x in {40,60} \fill (\x mm,0) circle (2pt);
            \end{tikzpicture}
        }
        \wrongchoice{
            \begin{tikzpicture}
                \draw[dashed,white!80!black] (-0.2,-1em) rectangle (6.2,1em);
                \foreach \x in {0,1,...,5} \fill ({6*rnd},0) circle (2pt);
            \end{tikzpicture}
        }
    \end{choices}
\end{question}
}

\element{nysed}{
\begin{question}{June1996-Q05}
    In an experiment that measures how fast a student reacts,
        a meter stick dropped from rest falls \SI{0.20}{\meter} before the student catches it.
    The reaction time of the student is approximately:
    \begin{multicols}{2}
    \begin{choices}
        \wrongchoice{\SI{0.10}{\second}}
      \correctchoice{\SI{0.20}{\second}}
        \wrongchoice{\SI{0.30}{\second}}
        \wrongchoice{\SI{0.40}{\second}}
    \end{choices}
    \end{multicols}
\end{question}
}

\element{nysed}{
\begin{question}{June1996-Q06}
    A race car traveling \SI{10}{\meter\per\second} accelerates at the rate of \SI{1.5}{\meter\per\second\squared} while traveling a distance of \SI{600}{\meter}.
    The final speed of the race car is approximately:
    \begin{multicols}{2}
    \begin{choices}
        \wrongchoice{\SI{1900}{\meter\per\second}}
        \wrongchoice{\SI{910}{\meter\per\second}}
        \wrongchoice{\SI{150}{\meter\per\second}}
      \correctchoice{\SI{44}{\meter\per\second}}
    \end{choices}
    \end{multicols}
\end{question}
}


%% Section June1994
%%--------------------



%% Section June1990
%%--------------------
\element{nysed}{
\begin{question}{June1990-Q01}
    A cart starting from rest travels a distance of \SI{3.6}{\meter} in \SI{1.8}{\second}.
    The average speed of the cart is:
    \begin{multicols}{2}
    \begin{choices}
        \wrongchoice{\SI{0.20}{\meter\per\second}}
      \correctchoice{\SI{2.0}{\meter\per\second}}
        \wrongchoice{\SI{0.50}{\meter\per\second}}
        \wrongchoice{\SI{5.0}{\meter\per\second}}
    \end{choices}
    \end{multicols}
\end{question}
}

\element{nysed}{
\begin{question}{June1990-Q02}
    An object has a constant acceleration of \SI{2.0}{\meter\per\second\squared}.
    The time required for the object to accelerate from \SI{8.0}{\meter\per\second} to \SI{28}{\meter\per\second} is:
    \begin{multicols}{2}
    \begin{choices}
        \wrongchoice{\SI{20}{\second}}
        \wrongchoice{\SI{16}{\second}}
      \correctchoice{\SI{10}{\second}}
        \wrongchoice{\SI{4.0}{\second}}
    \end{choices}
    \end{multicols}
\end{question}
}

\element{nysed}{
\begin{question}{June1990-Q04}
    A car moving at a speed of \SI{8.0}{\meter\per\second} enters a highway and accelerates at \SI{3.0}{\meter\per\second\squared}.
    How fast will the car be moving after it has accelerated for \SI{56}{\meter}?
    \begin{multicols}{2}
    \begin{choices}
        \wrongchoice{\SI{24}{\meter\per\second}}
      \correctchoice{\SI{20}{\meter\per\second}}
        \wrongchoice{\SI{18}{\meter\per\second}}
        \wrongchoice{\SI{4.0}{\meter\per\second}}
    \end{choices}
    \end{multicols}
\end{question}
}

\element{nysed}{
\begin{question}{June1990-Q06}
    The graph at the right represents the relationship between distance and time for an object in motion.
    \begin{center}
    \begin{tikzpicture}
        \begin{axis}[
            axis y line=left,
            axis x line=bottom,
            axis line style={->},
            xlabel={time},
            xtick=\empty,
            ylabel={distance},
            ytick=\empty,
            xmin=0,xmax=11,
            ymin=0,ymax=11,
            width=0.8\columnwidth,
            height=0.5\columnwidth,
            clip=false,
            very thin,
        ]
        \draw[very thick] (axis cs:0,0) -- (axis cs:2,0) -- (axis cs:4,4) -- (axis cs:7,4) to[out=0,in=220] (axis cs:10,8);
        %% labels
        \node[anchor=north west] at (axis cs:0,0) {$A$};
        \node[anchor=north] at (axis cs:2,0) {$B$};
        \node[anchor=south] at (axis cs:4,4) {$C$};
        \node[anchor=south] at (axis cs:7,4) {$D$};
        \node[anchor=south] at (axis cs:10,8) {$E$};
        \end{axis}
    \end{tikzpicture}
    \end{center}
    During which interval is the speed of the object changing?
    \begin{multicols}{2}
    \begin{choices}
        \wrongchoice{$AB$}
        \wrongchoice{$BC$}
        \wrongchoice{$CD$}
      \correctchoice{$DE$}
    \end{choices}
    \end{multicols}
\end{question}
}


%% Section June1989
%%--------------------
\element{nysed}{
\begin{question}{June1989-Q06}
    If an object's velocity changes from \SI{25}{\meter\per\second} to \SI{15}{\meter\per\second} in \SI{2.0}{\second},
        the magnitude of the object's acceleration Is:
    \begin{multicols}{2}
    \begin{choices}
      \correctchoice{\SI{5.0}{\meter\per\second\squared}}
        \wrongchoice{\SI{7.5}{\meter\per\second\squared}}
        \wrongchoice{\SI{13}{\meter\per\second\squared}}
        \wrongchoice{\SI{20}{\meter\per\second\squared}}
    \end{choices}
    \end{multicols}
\end{question}
}

\element{nysed}{
\begin{question}{June1989-Q07}
    An object initially traveling in a straight line with a speed of \SI{5.0}{\meter\per\second} is accelerated at \SI{2.0}{\meter\per\second\squared} for \SI{4.0}{\second}.
    The total distance traveled by the object in the \SI{4.0}{\second} is:
    \begin{multicols}{2}
    \begin{choices}
      \correctchoice{\SI{36}{\meter}}
        \wrongchoice{\SI{24}{\meter}}
        \wrongchoice{\SI{16}{\meter}}
        \wrongchoice{\SI{4.0}{\meter}}
    \end{choices}
    \end{multicols}
\end{question}
}


%% Section June1986
%%--------------------
\element{nysed}{
\begin{question}{June1986-Q01}
    The graph below represents the motion of a body that is moving with:
    \begin{center}
    \begin{tikzpicture}
        \begin{axis}[
            axis y line=left,
            axis x line=bottom,
            axis line style={->},
            xlabel={time},
            xtick=\empty,
            ylabel={distance},
            ytick=\empty,
            xmin=0,xmax=11,
            ymin=0,ymax=11,
            width=0.8\columnwidth,
            height=0.5\columnwidth,
            very thin,
        ]
        \addplot[line width=1pt,domain=0:10]{x};
        \end{axis}
    \end{tikzpicture}
    \end{center}
    \begin{choices}
        \wrongchoice{increasing acceleration}
        \wrongchoice{decreasing acceleration}
        \wrongchoice{increasing speed}
      \correctchoice{constant speed}
    \end{choices}
\end{question}
}

\element{nysed}{
\begin{question}{June1986-Q03}
    An object initially at rest accelerates at \SI{5}{\meter\per\second\squared} until it attains a speed of \SI{30}{\meter\per\second}.
    What distance does the object move while accelerating?
    \begin{multicols}{2}
    \begin{choices}
        \wrongchoice{\SI{30}{\meter}}
      \correctchoice{\SI{90}{\meter}}
        \wrongchoice{\SI{3}{\meter}}
        \wrongchoice{\SI{600}{\meter}}
    \end{choices}
    \end{multicols}
\end{question}
}

\newcommand{\nysedJuneNineteenEightySixQSixtyOne}{
\begin{tikzpicture}
    \begin{axis}[
        axis y line=left,
        axis x line=bottom,
        axis line style={->},
        xlabel={time},
        x unit=\si{\second},
        xtick={0,1,2,3,4,5,6},
        ylabel={displacement},
        y unit=\si{\meter},
        ytick={0,1,2,3,4},
        xmin=0,xmax=6.33,
        ymin=0,ymax=4,
        width=0.8\columnwidth,
        height=0.5\columnwidth,
        very thin,
    ]
    \addplot[line width=1pt,mark=\empty] plot coordinates {(0,0) (2,3) (3,3) (4,2) (6,0)};
    \draw[dashed] (axis cs:0,2) -- (axis cs:4,2);
    \draw[dashed] (axis cs:0,3) -- (axis cs:2,3);
    \draw[dashed] (axis cs:2,0) -- (axis cs:2,3);
    \draw[dashed] (axis cs:3,0) -- (axis cs:3,3);
    \draw[dashed] (axis cs:4,0) -- (axis cs:4,2);
    \end{axis}
\end{tikzpicture}
}

\element{nysed}{
\begin{question}{June1986-Q61}
    The graph below represents the displacement of an object as a function of time.
    \begin{center}
        \nysedJuneNineteenEightySixQSixtyOne
    \end{center}
    How far is the object from the starting point at the end of \SI{3}{\second}?
    \begin{multicols}{2}
    \begin{choices}
        \wrongchoice{\SI{0}{\meter}}
        \wrongchoice{\SI{2.0}{\meter}}
      \correctchoice{\SI{3.0}{\meter}}
        \wrongchoice{\SI{9.0}{\meter}}
    \end{choices}
    \end{multicols}
\end{question}
}

\element{nysed}{
\begin{question}{June1986-Q62}
    The graph below represents the displacement of an object as a function of time.
    \begin{center}
        \nysedJuneNineteenEightySixQSixtyOne
    \end{center}
    What is the velocity of the object at $t=\SI{1}{\second}$?
    \begin{multicols}{2}
    \begin{choices}
        \wrongchoice{\SI{1.0}{\meter\per\second}}
        \wrongchoice{\SI{2.0}{\meter\per\second}}
        \wrongchoice{\SI{3.0}{\meter\per\second}}
      \correctchoice{\SI{1.5}{\meter\per\second}}
    \end{choices}
    \end{multicols}
\end{question}
}

\element{nysed}{
\begin{question}{June1986-Q63}
    The graph below represents the displacement of an object as a function of time.
    \begin{center}
        \nysedJuneNineteenEightySixQSixtyOne
    \end{center}
    During which time interval is the object at rest?
    \begin{multicols}{2}
    \begin{choices}
        \wrongchoice{\SIrange{0}{2}{\second}}
      \correctchoice{\SIrange{2}{3}{\second}}
        \wrongchoice{\SIrange{3}{4}{\second}}
        \wrongchoice{\SIrange{4}{6}{\second}}
    \end{choices}
    \end{multicols}
\end{question}
}

\element{nysed}{
\begin{question}{June1986-Q64}
    The graph below represents the displacement of an object as a function of time.
    \begin{center}
        \nysedJuneNineteenEightySixQSixtyOne
    \end{center}
    What is the average velocity of the object from $t=\SI{0}{\second}$ to $t=\SI{3}{\second}$?
    \begin{multicols}{2}
    \begin{choices}
      \correctchoice{\SI{1.0}{\meter\per\second}}
        \wrongchoice{\SI{2.0}{\meter\per\second}}
        \wrongchoice{\SI{3.0}{\meter\per\second}}
        \wrongchoice{\SI{0}{\meter\per\second}}
    \end{choices}
    \end{multicols}
\end{question}
}

\element{nysed}{
\begin{question}{June1986-Q65}
    The graph below represents the displacement of an object as a function of time.
    \begin{center}
        \nysedJuneNineteenEightySixQSixtyOne
    \end{center}
    During which time interval is the object accelerating?
    \begin{multicols}{2}
    \begin{choices}
        \wrongchoice{\SIrange{0}{2}{\second}}
        \wrongchoice{\SIrange{2}{3}{\second}}
      \correctchoice{\SIrange{3}{4}{\second}}
        \wrongchoice{\SIrange{4}{6}{\second}}
    \end{choices}
    \end{multicols}
\end{question}
}


\endinput

